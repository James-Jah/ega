\section{Projective bundles; ample sheaves}
\label{section:II.4}


\subsection{Definition of projective bundles}
\label{subsection:II.4.1}

\begin{definition}[4.1.1]
\label{II.4.1.1}
Let $Y$ be a prescheme, $\sh{E}$ a quasi-coherent $\sh{O}_Y$-module, and $\bb{S}_{\sh{O}_Y}(\sh{E})$ the symmetric $\sh{O}_Y$-algebra of $\sh{E}$ \sref{II.1.7.4}, which is quasi-coherent \sref{II.1.7.7}.
We define the \emph{projective bundle on $Y$ defined by $\sh{E}$}, denoted $\bb{P}(\sh{E})$, to be the $Y$-scheme $P=\Proj(\bb{S}_{\sh{O}_Y}(\sh{E}))$.
The $\sh{O}_P$-module $\sh{O}_P(1)$ is called the \emph{fundamental sheaf on $P$}.
\end{definition}

When $Y$ is affine of ring $A$, then we have $\sh{E}=\widetilde{E}$ for some $A$-module $E$, and we then write $\bb{P}(E)$ instead of $\bb{P}(\widetilde{E})$.

When we take $\sh{E}=\sh{O}_Y^n$, we write $\bb{P}_Y^{n-1}$ instead of $\bb{P}(\sh{E})$;
if, further, $Y$ is affine of ring $A$, then we also write $\bb{P}_A^{n-1}$ instead of $\bb{P}_Y^{n-1}$.
Since $\bb{S}_{\sh{O}_Y}(\sh{O}_Y)$ is canonically identified with $\sh{O}_Y[T]$ \sref{II.1.7.4}, $\bb{P}_Y^0$ is canonically identified with $Y$ \sref{II.3.1.7};
Example~\sref{II.2.4.3} is then exactly $\bb{P}_K^1$.

\begin{env}[4.1.2]
\label{II.4.1.2}
Let $\sh{E}$ and $\sh{F}$ be quasi-coherent $\sh{O}_Y$-modules;
let $u:\sh{E}\to\sh{F}$ be an $\sh{O}_Y$-homomorphism;
there is a canonically corresponding homomorphism $\bb{S}(u):\bb{S}_{\sh{O}_Y}(\sh{E})\to\bb{S}_{\sh{O}_Y}(\sh{F})$ of graded $\sh{O}_Y$-algebras \sref{II.1.7.4}.
If $u$ is \emph{surjective}, then so too is $\bb{S}(u)$, and thus \sref{II.3.6.2}[(i)] $\Proj(\bb{S}(u))$ is a \emph{closed immersion} $\bb{P}(\sh{F})\to\bb{P}(\sh{E})$, which we denote by $\bb{P}(u)$.
We can thus say that $\bb{P}(\sh{E})$ is a \emph{contravariant functor} in $\sh{E}$, with the condition that we only consider \emph{surjective} morphisms of quasi-coherent $\sh{O}_Y$-modules.

Still supposing that $u$ is surjective, and letting $P=\bb{P}(\sh{E})$, $Q=\bb{P}(\sh{F})$, and $j=\bb{P}(u)$, we have, up to isomorphism, that
\[
\label{II.4.1.2.1}
  j^*(\sh{O}_P(n)) = \sh{O}_Q(n)
  \qquad\mbox{for all $n\in\bb{Z}$}
  \tag{4.1.2.1}
\]
by \sref{II.3.6.3}.
\end{env}

\begin{env}[4.1.3]
\label{II.4.1.3}
Now let $\psi:Y'\to Y$ be a morphism, and let $\sh{E}'=\psi^*(\sh{E})$;
then $\bb{S}_{\sh{O}_{Y'}}(\sh{E}') = \psi^*(\bb{S}_{\sh{O}_Y}(\sh{E}))$ \sref{II.1.7.5};
thus \sref{II.3.5.3}
\[
\label{II.4.1.3.1}
  \bb{P}(\psi^*(\sh{E})) = \bb{P}(\sh{E})\times_Y Y'
  \tag{4.1.3.1}
\]
up to canonical isomorphism;
furthermore, we clearly have that
\[
  \psi^*((\bb{S}_{\sh{O}_Y}(\sh{E}))(n)) = (\bb{S}_{\sh{O}_{Y'}}(\sh{E}'))(n)
\]
for all $n\in\bb{Z}$, whence, letting $P=\bb{P}(\sh{E})$ and $P'=\bb{P}(\sh{E}')$, we have \sref{II.3.5.4}, up to isomorphism, that
\[
\label{II.4.1.3.2}
  \sh{O}_{P'}(n) = \sh{O}_p(n)\otimes_Y\sh{O}_{Y'}
  \qquad\mbox{for all $n\in\bb{Z}$.}
  \tag{4.1.3.2}
\]
\end{env}

\oldpage[II]{72}
\begin{proposition}[4.1.4]
\label{II.4.1.4}
Let $\sh{L}$ be an invertible $\sh{O}_Y$-module.
For every quasi-coherent $\sh{O}_Y$-module $\sh{E}$, there exists a canonical $Y$-isomorphism $i_\sh{L}:\bb{P}(\sh{E})\xrightarrow{\sim}\bb{P}(\sh{E}\otimes\sh{L})$;
furthermore, if we let $P=\bb{P}(\sh{E})$ and $Q=\bb{P}(\sh{E}\otimes\sh{L})$, then $i_\sh{L}^*(\sh{O}_Q(n))$ is canonically isomorphic to $\sh{O}_P(n)\otimes_Y\sh{L}^{\otimes n}$ for all $n\in\bb{Z}$.
\end{proposition}

\begin{proof}
Note first of all that, if $A$ is a ring, $E$ an $A$-module, and $L$ a \emph{free monogenous} $A$-module, then we can canonically define a homomorphism of $A$-modules
\[
  \bb{S}_n(E\otimes L) \to \bb{S}_n(E)\otimes L^{\otimes n}
\]
by sending $(x_1\otimes y_1)\ldots(x_n\otimes y_n)$ to the element
\[
  (x_1x_2\ldots x_n)\otimes(y_1\otimes y_2\otimes\ldots\otimes y_n)
  \qquad\mbox{($x_i\in E$, $y_i\in L$, for $i\leq i\leq n$);}
\]
we can immediately see (by restricting to the case where $L=A$) that this homomorphism is in fact an isomorphism.
We thus obtain a canonical isomorphism of graded $A$-algebras $\bb{S}_A(E\otimes L)\xrightarrow{\sim}\bigoplus_{n\geq0}\bb{S}_n(E)\otimes L^{\otimes n}$.
By returning to the conditions of \sref{II.4.1.4}, the above remarks allow us to define a canonical isomorphism of graded $\sh{O}_Y$-algebras
\[
\label{II.4.1.4.1}
  \bb{S}_{\sh{O}_Y}(\sh{E}\otimes_{\sh{O}_Y}\sh{L}) \xrightarrow{\sim} \bigoplus_{n\geq0}\bb{S}_n(\sh{E})\otimes_{\sh{O}_Y}\sh{L}^{\otimes n}
  \tag{4.1.4.1}
\]
by defining this isomorphism as an isomorphism of presheaves, and taking into account \sref{II.1.7.4}, \sref[I]{I.1.3.9}, and \sref[I]{I.1.3.12}.
The proposition then follows from \sref{II.3.1.8}[(iii)] and \sref{II.3.2.10}.
\end{proof}

\begin{env}[4.1.5]
\label{II.4.1.5}
With the hypotheses of \sref{II.4.1.1}, let $P=\bb{P}(\sh{E})$, and denote by $p$ the structure morphism $P\to Y$.
Since, by definition, $\sh{E}=(\bb{S}_{\sh{O}_Y}(\sh{E}))_1$, we have a canonical homomorphism $\alpha_1:\sh{E}\to p_*(\sh{O}_P(1))$ \sref{II.3.3.2.2}, and thus \sref[0]{0.4.4.3} also a canonical homomorphism
\[
\label{II.4.1.5.1}
  \alpha_1^\sharp: p^*(\sh{E}) \to \sh{O}_P(1).
  \tag{4.1.5.1}
\]
\end{env}

\begin{proposition}[4.1.6]
\label{II.4.1.6}
The canonical homomorphism \sref{II.4.1.5.1} is surjective.
\end{proposition}

\begin{proof}
We have seen, in \sref{II.3.3.2}, that $\alpha_1^\sharp$ corresponds functorially to the canonical homomorphism $\sh{E}\otimes_{\sh{O}_Y}\bb{S}_{\sh{O}_Y}(\sh{E}) \to (\bb{S}_{\sh{O}_Y}(\sh{E}))(1)$;
since, by definition, $\sh{E}$ generates $\bb{S}_{\sh{O}_Y}(\sh{E})$, this homomorphism is surjective, whence the conclusion, by \sref{II.3.2.4}
\end{proof}


\subsection{Morphisms from a prescheme to a projective bundle}
\label{subsection:II.4.2}

\begin{env}[4.2.1]
\label{II.4.2.1}
Keeping the notation of \sref{II.4.1.5}, let $X$ be a $Y$-prescheme, $q:X\to Y$ the structure morphism, and let $r:X\to P$ be a $Y$-\emph{morphism} such that the following diagram commutes:
\[
  \xymatrix{
    P \ar[d]_p & X \ar[l]_r \ar[dl]^q
  \\Y
  }
\]
\oldpage[II]{73}

Since the functor $r^*$ is right exact \sref[0]{0.4.3.1}, we obtain, from the surjective homomorphism in \sref{II.4.1.5.1}, a surjective homomorphism
\[
  r^*(\alpha_1^\sharp): r^*(p^*(\sh{E})) \to r^*(\sh{O}_P(1)).
\]

But $r^*(p^*(\sh{E}))=q^*(\sh{E})$, and $r^*(\sh{O}_P(1))$ is locally isomorphic to $r^*(\sh{O}_P)=\sh{O}_X$, or, in other words, the latter is an \emph{invertible} sheaf $\sh{L}_r$ on $\sh{O}_X$, and so we have defined, given $r$, a canonical surjective $\sh{O}_X$-homomorphism
\[
\label{II.4.2.1.1}
  \varphi_r:q^*(\sh{E}) \to \sh{L}_r.
  \tag{4.2.1.1}
\]

When $Y=\Spec(A)$ is affine, and $\mathscr{E}=\widetilde{E}$, we can further clarify this homomorphism in the following way:
given $f\in E$, it follows from \sref{II.2.6.3} that
\[
\label{II.4.2.1.2}
  r^{-1}(D_+(f)) = X_{\varphi_r^\flat(f)}.
  \tag{4.2.1.2}
\]

Now let $V$ be an affine open subset of $X$ that is contained inside $r^{-1}(D_+(f))$, and let $B$ be its ring, which is an $A$-algebra;
let $S=\bb{S}_A(E)$;
the restriction of $r$ to $V$ corresponds to an $A$-homomorphism $\omega:\bb{S}_f\to B$, and we have that $q^*(\sh{E})|V = (E\otimes_A B)\supertilde$ and $\sh{L}_r|V = \widetilde{L_r}$, whence $L_r = (S(1))_{(f)}\otimes_{S_{(f)}}B_{[\omega]}$ \sref[I]{I.1.6.5}.
The restriction of $\varphi_r$ to $q^*(\sh{E})|V$ corresponds to the $B$-homomorphism $u:E\otimes_A B\to L_r$, which sends $x\otimes1$ to $(x/1)\otimes f = (f/1)\otimes\omega(x/f)$.
The canonical extension of $\varphi_r$ to a homomorphism of $\sh{O}_X$-algebras
\[
  \psi_r: q^*(\bb{S}(\sh{E})) = \bb{S}(q^*(\sh{E})) \to \bb{S}(\sh{L}_r) = \bigoplus_{n\geq0}\sh{L}_r^{\otimes n}
\]
is thus such that the restriction of $\psi_r$ to $q^*(\bb{S}_n(\sh{E}))|V$ corresponds to the homomorphism $\bb{S}_n(\sh{E}\otimes_A B) = \bb{S}_n(E)\otimes_A B \to L_r^{\otimes n}$ that sends $s\otimes1$ to $(f/1)^{\otimes n}\otimes\omega(s/f^n)$.
\end{env}

\begin{env}[4.2.2]
\label{II.4.2.2}
Conversely, suppose that we are given a morphism $q:X\to Y$, an invertible $\sh{O}_X$-module $\sh{L}$, and a quasi-coherent $\sh{O}_Y$-module $\sh{E}$;
to each homomorphism $\varphi:q^*(\sh{E})\to\sh{L}$ there canonically corresponding homomorphism of quasi-coherent $\sh{O}_X$-algebras
\[
  \psi: \bb{S}(q^*(\sh{E})) = q^*(\bb{S}(\sh{E})) \to \bigoplus_{n\geq0}\sh{L}^{\otimes n}
\]
and thus \sref{II.3.7.1} a $Y$-morphism $r_{\sh{L},\psi}:G(\psi)\to\Proj(\bb{S}(\sh{E}))=\bb{P}(\sh{E})$, which we denote by $r_{\sh{L},\varphi}$.
If $\varphi$ is \emph{surjective}, then so too is $\psi$, and thus \sref{II.3.7.4} $r_{\sh{L},\varphi}$ is \emph{everywhere defined}.
Furthermore, with the notation of \sref{II.4.2.1} and \sref{II.4.2.2}:
\end{env}

\begin{proposition}[4.2.3]
\label{II.4.2.3}
Given a morphism $q:X\to Y$ and a quasi-coherent $\sh{O}_Y$-module $\sh{E}$, maps $r\to(\sh{L}_r,\varphi_r)$ and $(\sh{L},\varphi)\to r_{\sh{L},\varphi}$ give a bijective correspondence between the set of $Y$-morphisms $r:X\to\bb{P}(\sh{E})$ and the set of equivalence classes of pairs $(\sh{L},\varphi)$ of an invertible $\sh{O}_X$-module $\sh{L}$ and a surjective homomorphism $\varphi:q^*(\sh{E})\to\sh{L}$, where such pairs $(\sh{L},\varphi)$ and $(\sh{L}',\varphi')$ are defined to be equivalent if there exists an $\sh{O}_X$-isomorphism $\tau:\sh{L}\xrightarrow{\sim}\sh{L}'$ such that $\varphi'=\tau\circ\varphi$.
\end{proposition}

\begin{proof}
Start first with a $Y$-morphism $r:X\to\bb{P}(\sh{E})$, and construct $\sh{L}_r$ and $\varphi_r$ \sref{II.4.2.1}, and let $r'=r_{\sh{L}_r,\varphi_r}$;
it follows immediately from \sref{II.4.2.1} and \sref{II.3.7.2} that the morphisms $r$ and $r'$ are identical (by taking the generator of $\sh{L}_r$ to be the element $(f/1)\otimes1$ to define the homomorphisms $v_n$ of \sref{II.3.7.2}).
Conversely, take a pair $(\sh{L},\varphi)$ and construct
\oldpage[II]{74}
$r''=r_{\sh{L},\varphi}$, and then $\sh{L}_{r''}$ and $\varphi_{r''}$;
we will show that there exists a canonical isomorphism $\tau:\sh{L}_{r''}\xrightarrow{\sim}\sh{L}$ such that $\varphi=\tau\circ\varphi_{r''}$;
to define it, we can restrict to the case where $Y=\Spec(A)$ and $X=\Spec(B)$ are affine, and (with the notation of \sref{II.4.2.1} and \sref{II.3.7.2}) associate to each element $(x/1)\otimes1$ of $L_{r''}$ (where $x\in E$) the element $v_1(x)c$ of $L$.
We immediately see that $\tau$ does not depend on the chosen generator $c$ of $L$;
since $v_1$ is surjective by hypothesis, to prove that $\tau$ is an isomorphism it suffices to to show that, if $x/1=0$ in $(S(1))_{(f)}$, then $v_1(x)/1=0$ in $B_g$;
but the first equality implies that $f^nx=0$ in $\bb{S}_{n+1}(E)$ for some $n$, and this implies that $v_{n+1}(f^nx) = g^nv_1(x) = 0$ in $B$, whence the conclusion.
Finally, it is immediate that, for any two equivalent pairs $(\sh{L},\varphi)$ and $(\sh{L}',\varphi')$, we have $r_{\sh{L},\varphi}=r_{\sh{L}',\varphi'}$.
\end{proof}

In particular, for $X=Y$:
\begin{theorem}[4.2.4]
\label{II.4.2.4}
The set of $Y$-sections of $\bb{P}(\sh{E})$ is in canonical bijective correspondence with the set of quasi-coherent sub-$\sh{O}_Y$-modules $\sh{F}$ of $\sh{E}$ such that $\sh{E}/\sh{F}$ is invertible.
\end{theorem}

We note that this property corresponds to the classical definition of ``projective space'' as the set of hyperplanes of a vector space (the classical case corresponding to $Y=\Spec(K)$, where $K$ is a field, and $\sh{E}=\widetilde{E}$, where $E$ is a finite-dimensional $K$-vector space; the $\sh{F}$ having the property described in \sref{II.4.2.4} then correspond to the hyperplanes of $E$, and we already know that the $Y$-sections of $\bb{P}(\sh{E})$ are then the \emph{$K$-rational points of $\bb{P}(\sh{E})$} \sref[I]{I.3.4.5}).

\begin{remark}[4.2.5]
\label{II.4.2.5}
Since there is a canonical bijective correspondence between $Y$-morphisms from $X$ to $P$ and their graph morphisms, $X$-sections of $P\times_Y X$ \sref[I]{I.3.3.14}, we see that, conversely, \sref{II.4.2.3} can be deduced from \sref{II.4.2.4}.
Denote by $\Hyp_Y(X,\sh{E})$ the set of quasi-coherent sub-$\sh{O}_X$-modules $\sh{F}$ of $\sh{E}\otimes_Y\sh{O}_X=q^*(\sh{E})$ such that $q^*(\sh{E})/\sh{F}$ is an invertible $\sh{O}_X$-module.
If $g:X'\to X$ is a $Y$-morphism, then it follows from the fact that $g^*$ is right exact that $g^*q^*(\sh{E})/\sh{F})=g^*q^*(\sh{E}))/g^*(\sh{F})$, and so the latter sheaf is invertible, and thus $\Hyp_Y(X,\sh{E})$ is a \emph{contravariant functor} into the category of $Y$-preschemes.
We can thus interpret the theorem \sref{II.4.2.4} as defining a \emph{canonical isomorphism} of functors $\Hom_Y(X,\bb{P}(\sh{E}))$ and $\Hyp_Y(X,\sh{E})$, where both functors are contravariant in the variable $X$ and map into the category of $Y$-preschemes.
This also gives a characterisation of the projective bundle $P=\bb{P}(\sh{E})$ by the following \emph{universal property}, which is much closer to the geometric intuition than the constructions from §§2--3:
for every morphism $q:X\to Y$ and every invertible $\sh{O}_X$-module $\sh{L}$ that is a quotient of $\sh{E}\otimes_{\sh{O}_Y}\sh{O}_X$, there exists a unique $Y$-morphism $r:X\to P$ such that $\sh{L}=r^*(\sh{O}_P(1))$.

We will see later that we can similarly define, amongst other things, ``Grassmannian'' schemes.
\end{remark}

\begin{corollary}[4.2.6]
\label{II.4.2.6}
Suppose that every invertible $\sh{O}_Y$-module is trivial \sref[I]{I.2.4.8}.
Let $V$ be the group $\Hom_{\sh{O}_Y}(\sh{E},\sh{O}_Y)$, considered as a module over the ring $A=\Gamma(Y,\sh{O}_Y)$, and let $V^*$ be the subset of $V$ consisting of surjective homomorphisms.
Then the set of $Y$-sections of $\bb{P}(\sh{E})$ is canonically identified with $V^*/A^*$, where $A^*$ is the group of units of $A$.
\end{corollary}

\oldpage[II]{75}
In particular:
\begin{enumerate}
  \item The corollary \sref{II.4.2.6} applies whenever $Y$ is a \emph{local scheme} \sref[I]{I.2.4.8}.
    Let $Y$ be an arbitrary prescheme, $y$ a point of $Y$, and $Y'=\Spec(\kres(y))$;
    then the fibre $p^{-1}(y)$ of $\bb{P}(\sh{E})$ can, by \sref{II.4.1.3.1}, be identified with $\bb{P}(\sh{E}^y)$, where $\sh{E}^y = \sh{E}_y\otimes_{\sh{O}_y}\kres(y) = \sh{E}_y/\mathfrak{m}_y\sh{E}_y$ is considered as a vector space over $\kres(y)$.
    More generally, if $K$ is an extension of $\kres(y)$, then $p^{-1}(y)\otimes_{\kres(y)}K$ can be identified with $\bb{P}(\sh{E}^y\otimes_{\kres(y)}K)$.
    The corollary \sref{II.4.2.6} then shows that the set of \emph{geometric points of $\bb{P}(\sh{E})$ with values in the extension $K$ of $\kres(y)$} \sref[I]{I.3.4.5}, which we can also call the \emph{rational geometric fibre over $K$ of $\bb{P}(\sh{E})$ over the point $y$}, can be identified with the \emph{projective space} associated to the \emph{dual} of the $K$-vector space $\sh{E}^y\otimes_{\kres(y)}K$.
  \item Suppose that $Y$ is affine of ring $A$, and, further, that every invertible $\sh{O}_Y$-module is trivial;
    further, take $\sh{E}=\sh{O}_Y^n$;
    then, in \sref{II.4.2.6}, $V$ can be identified with $A^n$ \sref[I]{I.1.3.8}, and $V^*$ with the sets of systems $(f_i)_{1\leq i\leq n}$ of elements of $A$ that generate the ideal $A$;
    any two such systems define the same $Y$-section of $\bb{P}_Y^{n-1}=\bb{P}_A^{n-1}$, or, in other words, \emph{the same point of $\bb{P}_A^{n-1}$ with values in $A$}, if and only if one of them can be obtained from the other by multiplication by an invertible element of $A$.
\end{enumerate}

These properties justify the terminology ``projective bundle'' for $\bb{P}(\sh{E})$.
We note that the definitions that we will similarly obtain for ``projective space'' is in fact \emph{dual} to the classical definition;
this is imposed upon us by the necessity of being able to define $\bb{P}(\sh{E})$ for \emph{arbitrary} quasi-coherent $\sh{O}_Y$-modules $\sh{E}$, and not just locally free ones.

\begin{remark}[4.2.7]
\label{II.4.2.7}
We will see, in Chapter~V, that, if $Y$ is connected and locally Noetherian, and if $\sh{E}$ is locally free, then, letting $P=\bb{P}(\sh{E})$, every invertible $\sh{O}_P$-module is isomorphic to an $\sh{O}_P$-module of the form $\sh{L}'\otimes_{\sh{O}_Y}\sh{O}_P(m)$, with $\sh{L}'$ some invertible $\sh{O}_Y$-module, well defined up to isomorphism, and $m$ some well defined integer.
In other words, $\HH^1(P,\sh{O}_P^*)$ is isomorphic to $\bb{Z}\times\HH^1(Y,\sh{O}_Y^*)$ \sref[0]{0.5.4.7}.
We will also see (\sref[III]{III.2.1.14}, taking \sref[0]{0.5.4.10} into account) that $p_*(\sh{L}^{\otimes m})=0$ if $m<0$, and $p_*(\sh{L}^{\otimes m})$ is isomorphic to $\sh{L}'\otimes_{\sh{O}_Y}(\bb{S}_{\sh{O}_Y}(\sh{E}))_m$ if $m\geq0$.
If $\sh{F}$ is a quasi-coherent $\sh{O}_Y$-module, then every $Y$-morphism $\bb{P}(\sh{E})\to\bb{P}(\sh{F})$ is determined by the data of an invertible $\sh{O}_Y$-module, an integer $m\geq0$, and an $\sh{O}_Y$-homomorphism $\psi:\sh{F}\to\sh{L}'\otimes_{\sh{O}_Y}(\bb{S}_{\sh{O}_Y}(\sh{E}))_m$ such that the corresponding homomorphism $\psi^\sharp$ of $\sh{O}_{\bb{P}(\sh{F})}$-modules is surjective.
We will also see that, if the $Y$-morphism in question is an isomorphism, then $m=1$ and $\sh{F}$ is isomorphic to $\sh{E}\otimes_{\sh{O}_Y}\sh{L}'$ (the converse of \sref{II.4.1.4}).
This will allow us to determine the sheaf of germs of automorphisms of $\bb{P}(\sh{E})$ as the quotient of the sheaf of groups $\shAut(\sh{E})$ (which is locally isomorphic to $\GL(n,\sh{O}_Y)$ is $\sh{E}$ is of rank $n$) by $\sh{O}_Y^*$.
\end{remark}

\begin{env}[4.2.8]
\label{II.4.2.8}
Keeping the notation of \sref{II.4.2.1}, let $u:X'\to X$ be a morphism;
if the $Y$-morphism $r:X\to P$ corresponds to the homomorphism $\varphi:q^*(\sh{E})\to\sh{L}$, then the $Y$-morphism $r\circ u$ corresponds to $u^*(\varphi):u^*(q^*(\sh{E}))\to u^*(\sh{L})$, as follows immediately from the definitions.
\end{env}

\begin{env}[4.2.9]
\label{II.4.2.9}
Let $\sh{E}$ and $\sh{F}$ be quasi-coherent $\sh{O}_Y$-modules, $v:\sh{E}\to\sh{F}$ a surjective homomorphism, and $j=\bb{P}(v)$ the corresponding closed immersion $\bb{P}(\sh{F})\to\bb{P}(\sh{E})$ \sref{II.4.1.2}.
If the $Y$-morphism $r:X\to\bb{P}(\sh{F})$ corresponds to the homomorphism $\varphi:q^*(\sh{F})\to\sh{L}$, then the
\oldpage[II]{76}
$Y$-morphism $j\circ r$ corresponds to $q^*(\sh{E})\xrightarrow{q^*(v)}q^*(\sh{F})\xrightarrow{\varphi}\sh{L}$;
this again follows from the definition given in \sref{II.4.2.1}.
\end{env}

\begin{env}[4.2.10]
\label{II.4.2.10}
Let $\psi:Y'\to Y$ be a morphism, and let $\sh{E}'=\psi^*(\sh{E})$.
If the $Y$-morphism $r:X\to P$ corresponds to the homomorphism $\varphi:q^*(\sh{E})\to\sh{L}$, then the $Y'$-morphism
\[
  r_{(Y')}: X_{(Y')} \to P' = \bb{P}(\sh{E}')
\]
corresponds to $\varphi_{(Y')}:q_{(Y')}^*(\sh{E}') = q^*(\sh{E})\otimes_Y\sh{O}_{Y'} \to \sh{L}\otimes_Y\sh{O}_{Y'}$.
Indeed, by \sref{II.4.1.3.1}, we have the commutative diagram
\[
  \xymatrix{
    Y' \ar[d]
    & P'=P_{(Y')} \ar[l]_{p_{(Y')}} \ar[d]^{u}
    & X_{(Y')} \ar[l]_{r_{(Y')}} \ar[d]^{v}
  \\Y
    & P \ar[l]_{p}
    & X \ar[l]_{r}
  }
\]

From \sref{II.4.1.3.1}, we have
\[
  (r_{(Y')})^*(\sh{O}_{P'}(1)) = (r_{(Y')})^*(u^*(\sh{O}_P(1))) = v^*(r^*(\sh{O}_P(1))) = v^*(\sh{L}) = \sh{L}\otimes_Y\sh{O}_{Y'};
\]
we also know that $u^*(\alpha_1^\sharp)$ is exactly the canonical homomorphism $\alpha_1^\sharp:(p_{(Y')})^*(\sh{E}')\to\sh{O}_{P'}(1)$;
we can see this by explicitly calculating the canonical homomorphisms $\alpha_1^\sharp$ to $P$ and $P'$ as in \sref{II.4.1.6}.
Whence our claim.
\end{env}


\subsection{The Segre morphism}
\label{subsection:II.4.3}

\begin{env}[4.3.1]
\label{II.4.3.1}
Let $Y$ be a prescheme, and $\sh{E}$ and $\sh{F}$ quasi-coherent $\sh{O}_Y$-modules;
let $P_1=\bb{P}(\sh{E})$ and $P_2=\bb{P}(\sh{F})$, and denote the structure morphisms by $p_1:P_1\to Y$ and $p_2:P_2\to Y$.
Let $Q=P_1\times_Y P_2$, and let $q_1:Q\to P_1$ and $q_2:Q\to P_2$ be the canonical projections;
then the $\sh{O}_Q$-module $\sh{L}=\sh{O}_{P_1}(1)\otimes_Y \sh{O}_{P_2}(1) = q_1^*(\sh{O}_{P_1}(1))\otimes_{\sh{O}_Q}q_2^*(\sh{O}_{P_2}(1))$ is invertible, since it is the tensor product of of two invertible $\sh{O}_Q$-modules \sref[0]{0.5.4.4}.
Also, if $r=p_1\circ q_1=p_2\circ q_2$ is the structure morphism $Q\to Y$, then $r^*(\sh{E}\otimes_{\sh{O}_Y}\sh{F}) = q_1^*(p_1^*(\sh{E}))\otimes_{\sh{O}_Q}q_2^*(p_2^*(\sh{F}))$ \sref[0]{0.4.3.3};
the canonical surjective homomorphisms \sref{II.4.1.5.1} $p_1^*(\sh{E})\to\sh{O}_{P_1}(1)$ and $p_2^*(\sh{F})\to\sh{O}_{P_2}(1)$ thus give, by taking the tensor product, a canonical homomorphism
\[
\label{II.4.3.1.1}
  s: r^*(\sh{E}\otimes_{\sh{O}_Y}\sh{F}) \to \sh{L}
  \tag{4.3.1.1}
\]
which is evidently surjective;
from this we obtain \sref{II.4.2.2} a canonical morphism, called the \emph{Segre morphism}:
\[
\label{II.4.3.1.2}
  \varsigma: \bb{P}(\sh{E})\times_Y\bb{P}(\sh{F}) \to \bb{P}(\sh{E}\otimes_{\sh{O}_Y}\sh{F}).
  \tag{4.3.1.2}
\]

We can study the morphism $\varsigma$ more explicitly in the case where $Y=\Spec(A)$ is affine, and $\sh{E}=\widetilde{E}$ and $\sh{F}=\widetilde{F}$, where $E$ and $F$ are $A$-modules, whence $\sh{E}\otimes_{\sh{O}_Y}\sh{F}=(E\otimes_A F)^\sim$ \sref[I]{I.1.3.12};
let $R=\bb{S}_A(E)$, $S=\bb{S}_A(F)$, and $T=\bb{S}_A(E\otimes_A F)$;
let $f\in E$ and $g\in F$, and consider the affine open
\[
  D_+(f) \times_Y D_+(g) = \Spec(B)
\]
\oldpage[II]{77}
of $Q$, where $B=R_{(f)}\otimes_A S_{(g)}$;
the restriction of $\sh{L}$ to this affine open is $\widetilde{L}$, where
\[
  L = (R(1))_{(f)} \otimes_A (S(1))_{(g)}
\]
and the element $c=(f/1)\otimes(g/1)$ is a generator of $L$ considered as a free $B$-module \sref{II.2.5.7}.
The homomorphism \sref{II.4.3.1.1} corresponds to the homomorphism
\[
  (x\otimes y)\otimes b \mapsto b((x/1)\otimes(y/1))
\]
from $(E\otimes_A F)\otimes_A B$ to $L$.
With the notation of \sref{II.3.7.2}, we thus have that $v_1(x\otimes y)=(x/f)\otimes(y/g)$;
the restriction of $\varsigma$ to $D_+(f)\times_Y D_+(g)$ is a morphism from this affine scheme to $D_+(f\otimes g)$, corresponding to the ring homomorphism $\omega:T_{(f\otimes g)}\to R_{(f)}\otimes_A S_{(g)}$ defined by
\[
\label{II.4.3.1.3}
  \omega((x\otimes y)/(f\otimes g)) = (x/f)\otimes(y/g)
  \tag{4.3.1.3}
\]
for $x\in E$ and $y\in F$.
\end{env}

\begin{env}[4.3.2]
\label{II.4.3.2}
It follows from \sref{II.4.2.3} that we have a canonical isomorphism
\[
\label{II.4.3.2.1}
  \tau: \varsigma^*(\sh{O}_P(1)) \xrightarrow{\sim} \sh{O}_{P_1}(1)\otimes_Y\sh{O}_{P_2}(1)
  \tag{4.3.2.1}
\]
where we let $P=\bb{P}(\sh{E}\otimes_{\sh{O}_Y}\sh{F})$.
We will show that, for $x\in\Gamma(Y,\sh{E})$ and $y\in\Gamma(Y,\sh{F})$, we have
\[
\label{II.4.3.2.2}
  \tau(\alpha_1(x\otimes y)) = \alpha_1(x)\otimes\alpha_1(y).
  \tag{4.3.2.2}
\]

Indeed, we can restrict to the case where $Y$ is affine, and we then have, with the notation of \sref{II.4.3.1} and \sref{II.2.6.2}, that $\alpha_1^{f\otimes g}(x\otimes y)=(x\otimes y)/1$ in $(T(1))_{(f\otimes g)}$, that $\alpha_1^f(x)=x/1$ in $(R(1))_{(f)}$, and that $\alpha_1^g(y)=y/1$ in $(S(1))_{(g)}$.
The definition of $\tau$ given in \sref{II.4.2.3} and the calculation of $v_1$ done in \sref{II.4.3.1} then immediately prove the claim \sref{II.4.3.2.2}.
From this we obtain the equation
\[
\label{II.4.3.2.3}
  \varsigma^{-1}(P_{x\otimes y}) = (P_1)_x\times_Y(P_2)_y
  \tag{4.3.2.3}
\]
with the notation of \sref{II.3.1.4}.
Indeed, taking \sref{II.3.3.3} into account, the equation \sref{II.4.3.2.2} (by restricting to the affine case, with the help of \sref[I]{I.3.2.7} and \sref[I]{I.3.2.3}) leaves us only to prove the following lemma:
\begin{lemma}[4.3.2.4]
\label{II.4.3.2.4}
Let $B$ and $B'$ be $A$-algebras, and let $Y=\Spec(A)$, $Z=\Spec(B)$, and $Z'=\Spec(B')$;
then $D(t\otimes t')=D(t)\times_Y D(t')$ for any $t\in B$, $t'\in B'$.
\end{lemma}
\begin{proof}
Indeed, if $p:Z\times_Y Z'\to Z$ and $p':Z\times_Y Z'\to Z'$ are the canonical projections, then it follows from \sref[I]{I.1.2.2.2} that $p^{-1}(D(t))=D(t\otimes1)$ and $p'^{-1}(D(t'))=D(1\otimes t')$;
the conclusion follows from \sref[I]{I.3.2.7} and \sref[I]{I.1.1.9.1}, since $(t\otimes1)(1\otimes t')=t\otimes t'$.
\end{proof}
\end{env}

\begin{proposition}[4.3.3]
\label{II.4.3.3}
The Segre morphism is a closed immersion.
\end{proposition}

\begin{proof}
Since the question is local on $Y$, we can restrict to the case where $Y$ is affine.
With the notation of \sref{II.4.3.1} and \sref{II.4.3.1}, the $D_+(f\otimes g)$ then form a basis for the topology of $P$, since the $f\otimes g$ generate $T$ when $f$ runs over $E$ and $g$ runs over $F$.
By \sref{II.4.3.2.3}, we also know that $\varsigma^{-1}(D_+(f\otimes g))=D_+(f)\times_Y D_+(g)$.
It thus suffices \sref[I]{I.4.2.4} to prove that the restriction of $\varsigma$ to $D_+(f)\times_Y D_+(g)$ is a closed immersion into $D_+(f\otimes g)$.
But, with the same notation, the equation \sref{II.4.3.1.3} shows that $\omega$ is \emph{surjective}, which completes the proof.
\end{proof}

\begin{env}[4.3.4]
\label{II.4.3.4}
The Segre morphism is \emph{functorial} in $\sh{E}$ and $\sh{F}$, if we consider only
\oldpage[II]{78}
\emph{surjective} homomorphisms of quasi-coherent $\sh{O}_Y$-modules.
Indeed, we must then show that, if $\sh{E}\to\sh{E}'$ is a surjective $\sh{O}_Y$-homomorphism, then the diagram
\[
  \xymatrix{
    \bb{P}(\sh{E}')\times\bb{P}(\sh{F}) \ar[r]^{j\times1} \ar[d]_{\varsigma}
    & \bb{P}(\sh{E})\times\bb{P}(\sh{F}) \ar[d]^{\varsigma}
  \\\bb{P}(\sh{E}'\otimes\sh{F}) \ar[r]
    &\bb{P}(\sh{E}\otimes\sh{F})
  }
\]
commutes, where $j$ denotes the canonical closed immersion $\bb{P}(\sh{E}')\to\bb{P}(\sh{E})$.
Let $P'_1=\bb{P}(\sh{E}')$ and keep the notation from \sref{II.4.3.1};
then $j\times1$ is a closed immersion \sref[I]{I.4.3.1} and, up to isomorphism,
\[
  (j\times1)^*(\sh{O}_{P_1}(1)\otimes\sh{O}_{P_2}(1))
  = j^*(\sh{O}_{P_1}(1))\otimes\sh{O}_{P_2}(1)
  = \sh{O}_{P'_1}(1)\otimes\sh{O}_{P_2}(1)
\]
by \sref{II.4.1.2.1} and \sref[I]{I.9.1.5};
our claim then follows from \sref{II.4.2.8} and \sref{II.4.2.9}.
\end{env}

\begin{env}[4.3.5]
\label{II.4.3.5}
With the notation of \sref{II.4.3.1}, let $\psi:Y'\to Y$ be a morphism, and let $\sh{E}'=\psi^*(\sh{E})$ and $\sh{F}'=\psi^*(\sh{F})$;
then the Segre morphism $\bb{P}(\sh{E}')\times\bb{P}(\sh{F}')\to\bb{P}(\sh{E}'\otimes\sh{F}')$ can be identified with $\varsigma_{(Y')}$.
Indeed, keeping the notation of \sref{II.4.3.1}, let $P'_1=\bb{P}(\sh{E}')$ and $P_2=\bb{P}(\sh{F}')$;
we know \sref{II.4.1.3.1} that $P'_i$ can be identified with $(P_i)_{(Y')}$ ($i=1,2$), and so the structure morphism $P'_1\times P'_2\to Y'$ can be identified with $r_{(Y')}$.
Also $\sh{E}'\otimes\sh{F}'$ can be identified with $\psi^*(\sh{E}\otimes\sh{F})$, and so $\bb{P}(\sh{E}'\otimes\sh{F}')$ can be identified with $(\bb{P}(\sh{E}\otimes\sh{F}))_{(Y')}$ \sref{II.4.1.3.1}.
Finally, $\sh{O}_{P'_1}(1)\otimes_Y\sh{O}_{P'_2}(1)=\sh{L}'$ can be identified with $\sh{L}\otimes_Y\sh{O}_{Y'}$, by \sref{II.4.1.3.1} and \sref[I]{I.9.1.11}.
The canonical homomorphism $(r_{(Y)})^*(\sh{E}'\otimes\sh{F}')\to\sh{L}'$ can then be identified with $s_{(Y')}$, and our claim follows from \sref{II.4.2.10}.
\end{env}

\begin{remark}[4.3.6]
\label{II.4.3.6}
The prescheme given by the \emph{sum} of $\bb{P}(\sh{E})$ and $\bb{P}(\sh{F})$ is even canonically isomorphic to a \emph{closed subprescheme of $\bb{P}(\sh{E}\oplus\sh{F})$}.
Indeed, the surjective homomorphisms $\sh{E}\oplus\sh{F}\to\sh{E}$ and $\sh{E}\oplus\sh{F}\to\sh{F}$ correspond to closed immersions $\bb{P}(\sh{E})\to\bb{P}(\sh{E}\oplus\sh{F})$ and $\bb{P}(\sh{F})\to\bb{P}(\sh{E}\oplus\sh{F})$;
everything then reduces to showing that the underlying spaces of the closed subpreschemes of $\bb{P}(\sh{E}\oplus\sh{F})$ obtained in this way have empty intersection.
Since the question is local on $Y$, we can adopt the notation of \sref{II.4.3.1};
but $\bb{S}_n(E)$ and $\bb{S}_n(F)$ can be identified with submodules of $\bb{S}_n(E\oplus F)$ with intersection consisting only of $0$;
if $\mathfrak{p}$ is a graded prime ideal of $\bb{S}(E)$ such that $\mathfrak{p}\cap\bb{S}_n(E)\neq\bb{S}_n(E)$ for any $n\geq0$, then there exists a corresponding graded prime ideal of $\bb{S}(E\oplus F)$ whose intersection with $\bb{S}_n(E)$ is $\mathfrak{p}\cap\bb{S}_n(E)$, but who also \emph{contains} $\bb{S}_+(F)$, as we immediately see;
thus no point in $\Proj(\bb{S}(E))$ can have the same image in $\Proj(\bb{S}(E\oplus F))$ as any point in $\Proj(\bb{S}(F))$.
\end{remark}


\subsection{Immersions into projective bundles; very ample sheaves}
\label{subsection:II.4.4}

\begin{proposition}[4.4.1]
\label{II.4.4.1}
Let $Y$ be a quasi-compact scheme, or a prescheme whose underlying space is Noetherian, $q:X\to Y$ a morphism \emph{of finite type}, and $\sh{L}$ an invertible $\sh{O}_X$-module.
\begin{enumerate}
  \item[\rm{(i)}] Let $\sh{S}$ be a positively-graded quasi-coherent $\sh{O}_Y$-algebra, and $\psi:q^*(\sh{S})\to\bigoplus_{n\geq0}\sh{L}^{\otimes n}$ a homomorphism of graded algebras.
    For $r_{\sh{L},\psi}$ to be everywhere defined and an immersion, it is necessary and
\oldpage[II]{79}
    sufficient for there to exist an integer $n\geq0$ and a quasi-coherent sub-$\sh{O}_Y$-module \emph{of finite type} $\sh{E}$ of $\sh{S}_n$ such that the homomorphism $\psi'=\psi_n\circ q^*(j):q^*(\sh{E})\to\sh{L}^{\otimes n}=\sh{L}'$ (where $j$ is the injection $\sh{E}\to\sh{S}_n$) is surjective and such that the morphism $r_{\sh{L}',\varphi'}:X\to\bb{P}(\sh{E})$ is an immersion.
  \item[\rm{(ii)}] Let $\sh{F}$ be a quasi-coherent $\sh{O}_Y$-module, and $\varphi:q^*(\sh{F})\to\sh{L}$ a surjective homomorphism.
    For the morphism $r_{\sh{L},\varphi}$ to be an immersion $X\to\bb{P}(\sh{F})$, it is necessary and sufficient for there to exist a quasi-coherent sub-$\sh{O}_Y$-module \emph{of finite type} $\sh{E}$ of $\sh{F}$ such that the homomorphism $\varphi'=\varphi\circ q(j):q^*(\sh{E})\to\sh{L}$ (where $j$ is the canonical injection $\sh{E}\to\sh{F}$) is surjective and such that the morphism $r_{\sh{L},\varphi'}:X\to\bb{P}(\sh{E})$ is an immersion.
\end{enumerate}
\end{proposition}

\begin{proof}
\medskip\noindent
\begin{enumerate}
  \item[\rm{(i)}] The fact that $r_{\sh{L},\varphi}$ is everywhere defined and is an immersion is equivalent, by \sref{II.3.8.5}, to the existence of some $n\geq0$ and $\sh{E}$ such that, if $\sh{S}'$ is the subalgebra of $\sh{S}$ generated by $\sh{E}$, the homomorphism $q^*(\sh{E})\to\sh{L}^{\otimes n}$ is surjective and the morphism $r_{\sh{L},\psi'}:X\to\Proj(\sh{S}')$ is everywhere defined and is an immersion.
    We already have a canonical surjective homomorphism $\bb{S}(\sh{E})\to\sh{S}'$ to which there exists a corresponding closed immersion $\Proj(\sh{S}')\to\bb{P}(\sh{E})$ \sref{II.3.6.2};
    whence the conclusion.
  \item[\rm{(ii)}] Since $\sh{F}$ is the inductive limit of its quasi-coherent submodules of finite type $\sh{E}_\lambda$ \sref[I]{I.9.4.9}, $\bb{S}(\sh{F})$ is the inductive limit of the $\bb{S}(\sh{E}_\lambda)$;
    the conclusion then follows from \sref{II.3.8.4}, by observing that we can take all the $n_i$ in the proof of \sref{II.3.8.4} to be equal to $1$:
    indeed, supposing that $Y$ is affine, if $r=r_{\sh{L},\varphi}$ is an immersion, then $r(X)$ is a quasi-compact subspace of $\bb{P}(\sh{F})$ that we can cover by finitely many open subsets of $\bb{P}(\sh{F})$ of the form $D_+(f)$, with $f\in F$, such that $D_+(f)\cap r(X)$ is closed.
\end{enumerate}
\end{proof}

\begin{definition}[4.4.2]
\label{II.4.4.2}
Let $Y$ be a prescheme, and $q:X\to Y$ a morphism.
We say that an invertible $\sh{O}_X$-module $\sh{L}$ is \emph{very ample for $q$}, or \emph{relative to $q$} (or \emph{very ample for} (or \emph{relative to}) \emph{$Y$}, or simply \emph{very ample}, if $q$ is clear from the context) if there exists a quasi-coherent $\sh{O}_Y$-module $\sh{E}$ and a $Y$-immersion $i$ from $X$ to $P=\bb{P}(\sh{E})$ such that $\sh{L}$ is isomorphic to $i^*(\sh{O}_P(1))$.
\end{definition}

It is equivalent \sref{II.4.2.3} to say that there exists a quasi-coherent $\sh{O}_Y$-module $\sh{E}$ and a \emph{surjective} homomorphism $\varphi:q^*(\sh{E})\to\sh{L}$ such that $r_{\sh{L},\varphi}:X\to\bb{P}(\sh{E})$ is an \emph{immersion}.

We note that the existence of a very ample (for $Y$) $\sh{O}_X$-module implies that $q$ is \emph{separated} (\sref{II.3.1.3} and \sref[I]{I.5.5.1}[(i) and (ii)]).

\begin{corollary}[4.4.3]
\label{II.4.4.3}
Suppose that there exists a graded quasi-coherent $\sh{O}_Y$-algebra $\sh{S}$, generated by $\sh{S}_1$, and a $Y$-immersion $i:X\to P=\Proj(\sh{S})$ such that $\sh{L}$ is isomorphic to $i^*(\sh{O}_P(1))$;
then $\sh{L}$ is very ample relative to $q$.
\end{corollary}

\begin{proof}
If $\sh{F}=\sh{S}_1$, then the canonical homomorphism $\bb{S}(\sh{F})\to\sh{S}$ is surjective, and so, by compositing with the corresponding closed immersion $\Proj(\sh{S})\to\bb{P}(\sh{F})$ \sref{II.3.6.2} and the immersion $i$, we obtain an immersion $j:X\to\bb{P}(\sh{F})=P'$ such that $\sh{L}$ is isomorphic to $j^*(\sh{O}_{P'}(1))$ \sref{II.3.6.3}.
\end{proof}

\begin{proposition}[4.4.4]
\label{II.4.4.4}
Let $q:X\to Y$ be a quasi-compact morphism, and $\sh{L}$ an invertible $\sh{O}_X$-module.
Then the following properties are equivalent:
\begin{enumerate}
  \item[\rm{(a)}] $\sh{L}$ is very ample relative to $q$.
  \item[\rm{(b)}] $q_*(\sh{L})$ is quasi-coherent, the canonical homomorphism $\sigma:q^*(q_*(\sh{L}))\to\sh{L}$ is surjective, and the morphism $r_{\sh{L},\sigma}:X\to\bb{P}(q_*(\sh{L}))$ is an immersion.
\end{enumerate}
\end{proposition}

\begin{proof}
Since $q$ is quasi-compact, we know that $q_*(\sh{L})$ is quasi-coherent if $q$ is separated \sref[I]{I.9.2.2}.

\oldpage[II]{80}
We know \sref{II.3.4.7} that the existence of a surjective homomorphism $\varphi:q^*(\sh{E})\to\sh{L}$ (with $\sh{E}$ a quasi-coherent $\sh{O}_Y$-module) implies that $\sigma$ is surjective;
furthermore, given the factorisation $q^*(\sh{E})\to q^*(q_*(\sh{L}))\xrightarrow{\sigma}\sh{L}$ of $\varphi$, there is a canonically corresponding factorisation
\[
  q^*(\bb{S}(\sh{E})) \to q^*(\bb{S}(q_*(\sh{L}))) \to \bigoplus_{n\geq0}\sh{L}^{\otimes n}
\]
and so \sref{II.3.8.3} the hypothesis that $r_{\sh{L},\varphi}$ is an immersion implies that so too is $j=r_{\sh{L},\sigma}$;
furthermore \sref{II.4.2.4}, $\sh{L}$ is isomorphic to $j^*(\sh{O}_{P'}(1))$, where $P'=\bb{P}(q_*(\sh{L}))$.
We thus see that (a) and (b) are equivalent.
\end{proof}

\begin{corollary}[4.4.5]
\label{II.4.4.5}
Suppose that $q$ is quasi-compact.
For $\sh{L}$ to be very ample relative to $Y$, it is necessary and sufficient for there to exist an open cover $(U_\alpha)$ of $Y$ such that $\sh{L}|q^{-1}(U_\alpha)$ is very ample relative to $U_\alpha$ for every $\alpha$.
\end{corollary}

\begin{proof}
Indeed, condition (b) of \sref{II.4.4.4} is local on $Y$.
\end{proof}

\begin{proposition}[4.4.6]
\label{II.4.4.6}
Let $Y$ be a quasi-compact scheme, or a prescheme whose underlying space is Noetherian, $q:X\to Y$ a morphism \emph{of finite type}, and $\sh{L}$ an invertible $\sh{O}_X$-module.
Then conditions (a) and (b) of \sref{II.4.4.4} are equivalent to the following:
\begin{enumerate}
  \item[\rm{(a')}] There exists a quasi-coherent $\sh{O}_Y$-module $\sh{E}$ \emph{of finite type} and a surjective homomorphism $\varphi:q^*(\sh{E})\to\sh{L}$ such that $r_{\sh{L},\varphi}$ is an immersion.
  \item[\rm{(b')}] There exists a coherent sub-$\sh{O}_Y$-module $\sh{E}$ of $q_*(\sh{L})$ \emph{of finite type} that has the properties stated in condition~(a').
\end{enumerate}
\end{proposition}

\begin{proof}
It is clear that (a') or (b') imply (a);
also (a) implies (a'), by \sref{II.4.4.1}, and similarly (b) implies (b').
\end{proof}

\begin{corollary}[4.4.7]
\label{II.4.4.7}
Suppose that $Y$ is a quasi-compact scheme, or a Noetherian prescheme.
If $\sh{L}$ is very ample for $q$, then there exists a graded quasi-coherent $\sh{O}_Y$-algebra $\sh{S}$ such that $\sh{S}_1$ is of finite type and generates $\sh{S}$, and also a \emph{dominant open} $Y$-immersion $i:X\to P=\Proj(\sh{S})$ such that $\sh{L}$ is isomorphic to $i^*(\sh{O}_P(1))$.
\end{corollary}

\begin{proof}
Indeed, condition~(b) of \sref{II.4.4.6} is satisfied;
the structure morphism $p:\bb{P}(\sh{E})=P'\to Y$ is then separated and of finite type \sref{II.3.1.3}, and so $P'$ is a quasi-compact scheme (resp. a Noetherian prescheme) if $Y$ is a quasi-compact scheme (resp. a Noetherian prescheme).
Let $Z$ be the closure \sref[I]{I.9.5.11} of the subprescheme $X'$ of $P'$ associated to the immersion $j=r_{\sh{L},\varphi}$ from $X$ into $P'$;
it is clear that $j$ factors as a dominant open immersion $i:X\to Z$ followed by the canonical injection $Z\to P'$.
But $Z$ can be identified with a prescheme $\Proj(\sh{S})$, where $\sh{S}$ is a graded $\sh{O}_Y$-algebra equal to the quotient of $\bb{S}(\sh{E})$ by a graded quasi-coherent sheaf of ideals \sref{II.3.6.2}, and it is clear that $\sh{S}_1$ is of finite type and generates $\sh{S}$;
furthermore, $\sh{O}_Z(1)$ is the inverse image of $\sh{O}_{P'}(1)$ by the canonical injection \sref{II.3.6.3}, and so $\sh{L}=i^*(\sh{O}_Z(1))$.
\end{proof}

\begin{proposition}[4.4.8]
\label{II.4.4.8}
Let $q:X\to Y$ be a morphism, $\sh{L}$ a very ample (relative to $q$) $\sh{O}_X$-module, and $\sh{L}'$ an invertible $\sh{O}_X$-module, such that there exists a quasi-coherent $\sh{O}_Y$-module $\sh{E}'$ and a surjective homomorphism $q^*(\sh{E}')\to\sh{L}'$.
Then $\sh{L}\otimes_{\sh{O}_X}\sh{L}'$ is very ample relative to $q$.
\end{proposition}

\begin{proof}
The hypothesis implies the existence of a $Y$-morphism $r':X\to P'=\bb{P}(\sh{E}')$ such that $\sh{L}'=r'^*(\sh{O}_{P'}(1))$ \sref{II.4.2.1}.
There is, by hypothesis, a quasi-coherent $\sh{O}_Y$-module $\sh{E}$ and a
\oldpage[II]{81}
$Y$-immersion $r:X\to P=\bb{P}(\sh{E})$ such that $\sh{L}=r^*(\sh{O}_P(1))$.
Let $Q=\bb{P}(\sh{E}\otimes\sh{E}')$, and consider the Segre morphism $\varsigma:P\times_Y P'\to Q$ \sref{II.4.3.1}.
Since $r$ is an immersion, so too is $(r,r')_Y:X\to P\times_Y P'$ \sref[I]{I.5.3.14};
but since $\varsigma$ is an immersion \sref{II.4.3.3}, so too is $r'':X\xrightarrow{(r,r')}P\times_Y P'\xrightarrow{\varsigma}Q$.
But also \sref{II.4.3.2.1} $\varsigma(\sh{O}_Q(1))$ is isomorphic to $\sh{O}_P(1)\otimes_Y\sh{O}_{P'}(1)$, and so \sref[I]{I.9.1.4} $r''^*(\sh{O}_Q(1))$ is isomorphic to $\sh{L}\otimes\sh{L}'$, which proves the proposition.
\end{proof}

\begin{corollary}[4.4.9]
\label{II.4.4.9}
Let $q:X\to Y$ be a morphism.
\begin{enumerate}
  \item Let $\sh{L}$ be an invertible $\sh{O}_X$-module, and $\sh{K}$ an invertible $\sh{O}_Y$-module.
    For $\sh{L}$ to be very ample relative to $q$, it is necessary and sufficient for $\sh{L}\otimes q^*(\sh{K})$ to be so.
  \item If $\sh{L}$ and $\sh{L}'$ are very ample (relative to $q$) $\sh{O}_X$-modules, then so too is $\sh{L}\otimes\sh{L}'$;
    in particular, $\sh{L}^{\otimes n}$ is very ample relative to $q$ for all $n>0$.
\end{enumerate}
\end{corollary}

\begin{proof}
Claim~(ii) is an immediate consequence of \sref{II.4.4.8}, as well as the necessity of condition~(i);
conversely, if $\sh{L}\otimes q^*(\sh{K})$ is very ample, then so too is $(\sh{L}\otimes q^*(\sh{K}))\otimes q^*(\sh{K}^{-1})$, by the above, and the latter $\sh{O}_X$-module is isomorphic to $\sh{L}$ (\sref[0]{0.5.4.3} and \sref[0]{0.5.4.5}).
\end{proof}

\begin{proposition}[4.4.10]
\label{II.4.4.10}
\medskip\noindent
\begin{enumerate}
  \item[\rm{(i)}] For every prescheme $Y$, every invertible $\sh{O}_Y$-module $\sh{L}$ is very ample relative to the identity morphism $1_Y$.
  \item[\rm{(i \emph{bis})}] Let $f:X\to Y$ be a morphism, and $j:X'\to X$ an immersion.
    If $\sh{L}$ is a very ample (relative to $f$) $\sh{O}_X$-module, then $j^*(\sh{L})$ is very ample relative to $f\circ j$.
  \item[\rm{(ii)}] Let $Z$ be a quasi-compact prescheme, $f:X\to Y$ a morphism of finite type, $g:Y\to Z$ a quasi-compact morphism, $\sh{L}$ a very ample (relative to $f$) $\sh{O}_X$-module, and $\sh{K}$ a very ample (relative to $g$) $\sh{O}_Y$-module.
    Then there exists some integer $n_0>0$ such that $\sh{L}\otimes f^*(\sh{K}^{\otimes n})$ is very ample relative to $g\circ f$ for all $n\geq n_0$.
  \item[\rm{(iii)}] Let $f:X\to Y$ and $g:Y'\to Y$ be morphisms, and let $X'=X_{(Y')}$.
    If $\sh{L}$ is a very ample (relative to $f$) $\sh{O}_X$-module, then $\sh{L}'=\sh{L}\otimes_Y\sh{O}_{Y'}$ is a very ample (relative to $f_{(Y')}$) $\sh{O}_X$-module.
  \item[\rm{(iv)}] Let $f_i:X_i\to Y_i$ ($i=1,2$) be $S$-morphism.
    If $\sh{L}_i$ is a very ample (relative to $f_i$) $\sh{O}_{X_i}$-module ($i=1,2$), then $\sh{L}_1\otimes_S\sh{L}_2$ is very ample relative to $f_1\times_S f_2$.
  \item[\rm{(v)}] Let $f:X\to Y$ and $g:Y\to Z$ be morphisms.
    If an $\sh{O}_X$-module $\sh{L}$ is very ample relative to $g\circ f$, then it is also very ample relative to $f$.
  \item[\rm{(vi)}] Let $f:X\to Y$ be a morphism, and $j$ the canonical injection $X_\red\to X$.
    If an $\sh{O}_X$-module $\sh{L}$ is very ample relative to $f$, then $j^*(\sh{L})$ is very ample relative to $f_\red$.
\end{enumerate}
\end{proposition}

\begin{proof}
Property~(i~\emph{bis}) follows immediately from the definition \sref{II.4.4.2}, and it is immediate that (vi) follows formally from (i~\emph{bis}) and (v), by an argument copied from the proof of \sref[I]{I.5.5.12}, which we leave to the reader.
To prove (v), we consider, as in \sref[I]{I.5.5.12}, the factorisation $X\xrightarrow{\Gamma_f}X\times_Z Y\xrightarrow{p_2}Y$, where $p_2=(g\circ f)\times1_Y$.
It follows from the hypothesis and from (i) and (iv) that $\sh{L}\otimes_{\sh{O}_Z}\sh{O}_Y$ is very ample for $p_2$;
but also $\sh{L}=\Gamma_f^*(\sh{L}\otimes_{\sh{O}_Z}\sh{O}_Y)$ \sref[I]{I.9.1.4}, and $\Gamma_f$ is an immersion \sref[I]{I.5.3.11};
we can thus apply (i~\emph{bis}).

\oldpage[II]{82}
To prove (i), we apply the definition \sref{II.4.4.2} with $\sh{E}=\sh{L}$, and note that then $\bb{P}(\sh{E})$ can be identified with $Y$ \sref{II.4.1.4}.

Now we prove (iii).
There exists a quasi-coherent $\sh{O}_Y$-module $\sh{E}$ and a $Y$-immersion $i:X\to\bb{P}(\sh{E})=P$ such that $\sh{L}=i^*(\sh{O}_P(1))$;
if we let $\sh{E}'=g^*(\sh{E})$, then $\sh{E}'$ is a quasi-coherent $\sh{O}_{Y'}$-module, and we have that $P'=\bb{P}(\sh{E}')=P_{(Y')}$ \sref{II.4.1.3.1}, that $i_{(Y')}$ is an immersion from $X_{(Y')}$ into $P'$ \sref[I]{I.4.3.2}, and that $\sh{L}'$ is isomorphic to $(i_{(Y')})^*(\sh{O}_{P'}(1))$ \sref{II.4.2.10}.

To prove (iv), note that there is, by hypothesis, a $Y_i$-immersion $r_i:X_i\to P_i=\bb{P}(\sh{E}_i)$, where $\sh{E}_i$ is a quasi-coherent $\sh{O}_{Y_i}$-module, and $\sh{L}_i=r_i^*(\sh{O}_{P_i}(1))$ ($i=1,2$);
$r_1\times_S r_2$ is an $S$-immersion of $X_1\times_S X_2$ into $P_1\times_S P_2$ \sref[I]{I.4.3.1}, and the inverse image of $\sh{O}_{P_1}(1)\otimes_S\sh{O}_{P_2}(1)$ under this immersion is $\sh{L}_1\otimes_S\sh{L}_2$.
Now let $T=Y_1\times_S Y_2$, and let $p_1$ and $p_2$ be the projections from $T$ to $Y_1$ and $Y_2$, respectively.
If we let $P'_i=\bb{P}(p_i^*(\sh{E}_i))$ ($i=1,2$), then $P'_i=P_i\times_{Y_i}T$, by \sref{II.4.1.3.1}, and so
\[
  P'_1\times_T P'_2
  = (P_1\times_{Y_1}T)\times_T(P_2\times_{Y_2}T)
  = P_1\times_{Y_1}(T\times_{Y_2}P_2)
  = P_1\times_{Y_1}(Y_1\times_S P_2)
  = P_1\times_S P_2
\]
up to canonical isomorphism.
Similarly, $\sh{O}_{P'_i}(1)=\sh{O}_{P_i}(1)\otimes_{Y_i}\sh{O}_T$ \sref{II.4.1.3.2}, and an analogous calculation (based in particular on \sref[I]{I.9.1.9.1} and \sref[I]{I.9.1.2}) shows that, in the above identification, $\sh{O}_{P'_1}(1)\otimes_T\sh{O}_{P'_2}(1)$ can be identified with $\sh{O}_{P_1}\otimes_S\sh{O}_{P_2}(1)$.
We can thus consider $r_1\times_S r_2$ as a $T$-immersion from $X_1\times_S X_2$ into $P'_1\times_T P'_2$, with the inverse image of $\sh{O}_{P'_1}(1)\otimes_T\sh{O}_{P'_2}(1)$ under this immersion being $\sh{L}_1\otimes_S\sh{L}_2$.
We then finish the argument as in \sref{II.4.4.8} by using the Segre morphism.

It remains only to prove (ii).
We can first of all restrict to the case where $Z$ is an affine scheme, since, in general, there exists a finite cover $(U_i)$ of $Z$ by affine opens;
if the proposition were proven for $\sh{K}|g^{-1}(U_i)$, $\sh{L}|f^{-1}(g^{-1}(U))$, and an integer $n_i$, then it would suffice to take $n_0$ to be the largest of the $n_i$ to prove the proposition for $\sh{K}$ and $\sh{L}$ \sref{II.4.4.5}.
The hypothesis implies that $f$ and $g$ are separated morphisms, and so $X$ and $Y$ are quasi-compact \emph{schemes}.

There is an immersion $r:X\to P=\bb{P}(\sh{E})$, where $\sh{E}$ is a quasi-coherent $\sh{O}_Y$-module \emph{of finite type}, and $\sh{L}=r^*(\sh{O}_P(1))$, by \sref{II.4.4.6}.
We will see that there exists a very ample (relative to the composed morphism $P\xrightarrow{h}Y\xrightarrow{g}Z$) $\sh{O}_P$-module $\sh{M}$ such that $\sh{O}_P(1)$ is isomorphic to $\sh{M}\otimes_Y\sh{K}^{\otimes(-m)}$ for some integer $m$.
For $n\geq m+1$, $\sh{O}_P(1)\otimes_Y\sh{K}^{\otimes n}$ will then be very ample for $Z$, by hypothesis and by (iv) applied to the morphisms $h:P\to Y$ and $1_Y$;
since $r$ is an immersion and $\sh{L}\otimes f^*(\sh{K}^{\otimes n}) = r^*(\sh{O}_P(1)\otimes_Y\sh{K}^{\otimes n})$, the conclusion will then follow from (i~\emph{bis}).
To prove our claim concerning $\sh{O}_P(1)$, we will use the following lemma:

  \begin{lemma}[4.4.10.1]
  \label{II.4.4.10.1}
  Let $Z$ be a quasi-compact scheme, or a prescheme whose underlying space is Noetherian, and let $g:Y\to Z$ be a quasi-compact morphism, $\sh{K}$ a very ample (with respect to $g$) invertible $\sh{O}_Y$-module, and $\sh{E}$ a quasi-coherent $\sh{O}_Y$-module of finite type.
  Then there exists an integer $m_0$ such that, for all $m\geq m_0$, $\sh{E}$ is isomorphic to a quotient of an $\sh{O}_Y$-module of the form $g^*(\sh{F})\otimes\sh{K}^{\otimes(-m)}$, where $\sh{F}$ is a quasi-coherent $\sh{O}_Z$-module of finite type (depending on $m$).
  \end{lemma}

This lemma will be proven in \sref{II.4.5.10.1};
the reader can verify that \sref{II.4.4.10} is not used anywhere in \sref{subsection:II.4.5}.

\oldpage[II]{83}
Assuming this lemma, there exists a closed immersion $j_1$ from $P$ to
\[
  P_1 = \bb{P}(g^*(\sh{F})\otimes\sh{K}^{\otimes(-m)})
\]
such that $\sh{O}_P(1)$ is isomorphic to $j_1^*(\sh{O}_{P_1}(1))$ \sref{II.4.1.2}.
Now, there exists an isomorphism from $P_1$ to $P_2=\bb{P}(g^*(\sh{F}))$, sending $\sh{O}_{P_2}(1)\otimes_Y\sh{K}^{\otimes(-m)}$ to $\sh{O}_{P_1}(1)$ \sref{II.4.1.4};
we thus have a closed immersion $j_2:P\to P_2$ such that $\sh{O}_P(1)$ is isomorphic to $j_2^*(\sh{O}_{P_2}(1))\otimes_Y\sh{K}^{\otimes(-m)}$.
Finally, $P_2$ can be identified with $P_3\times_Z Y$, where $P_3=\bb{P}(\sh{F})$, and $\sh{O}_{P_2}(1)$ with $\sh{O}_{P_3}(1)\otimes_Z\sh{O}_Y$ \sref{II.4.1.3}.
By definition, $\sh{O}_{P_3}(1)$ is very ample for $Z$;
since so too is $\sh{K}$, we conclude, from (iv), that $\sh{O}_{P_2}(1)\otimes_Y\sh{K}$ is very ample for $Z$;
so too is $\sh{M}=j_2^*(\sh{O}_{P_2}(1)\otimes_Y\sh{K})$ by (i~\emph{bis}), and $\sh{O}_P(1)$ is isomorphic to $\sh{M}\otimes_Y\sh{K}^{\otimes(-m-1)}$, which finishes the proof.
\end{proof}

\begin{proposition}[4.4.11]
\label{II.4.4.11}
Let $f:X\to Y$ and $f':X'\to Y$ be morphisms, $X''$ the sum prescheme $X\sqcup X'$, and $f''$ the morphism $X''\to Y$ that agrees with $f$ (resp. $f'$) on $X$ (resp. $X'$).
Let $\sh{L}$ (resp. $\sh{L}'$) be an invertible $\sh{O}_X$-module (resp. invertible $\sh{O}_{X'}$-module), and let $\sh{L}''$ be the invertible $\sh{O}_{X''}$-module that agrees with $\sh{L}$ (resp. $\sh{L}'$) on $X$ (resp. $X'$).
For $\sh{L}''$ to be very ample relative to $f''$, it is necessary and sufficient for $\sh{L}$ to be very ample relative to $f$ and for $\sh{L}'$ to be very ample relative to $f'$.
\end{proposition}

\begin{proof}
We can immediately restrict to the case where $Y$ is affine.
If $\sh{L}''$ is very ample then so too are $\sh{L}$ and $\sh{L}'$, by \sref{II.4.4.10}[(i~\emph{bis})].
Conversely, if $\sh{L}$ and $\sh{L}'$ are very ample, then it follows immediately from the definition \sref{II.4.4.2} and from \sref{II.4.3.6} that $\sh{L}''$ is very ample.
\end{proof}


\subsection{Ample sheaves}
\label{subsection:II.4.5}

\begin{env}[4.5.1]
\label{II.4.5.1}
Given a prescheme $X$ and an invertible $\sh{O}_X$-module $\sh{L}$, we define, for every $\sh{O}_X$-module $\sh{F}$ (when there will be no confusion possible over $\sh{L}$) $\sh{F}(n)=\sh{F}\otimes_{\sh{O}_X}\sh{L}^{\otimes n}$ ($n\in\bb{Z}$);
we also define $S=\bigoplus_{n\geq0}\Gamma(X,\sh{L}^{\otimes n})$ (a graded subring of the ring $\Gamma_\bullet(\sh{L})$ defined in \sref[0]{0.5.4.6}).
If we consider $X$ as a $\bb{Z}$-prescheme, and we denote by $p$ the structure morphism $X\to\Spec(\bb{Z})$, then there is a bijective correspondence between homomorphisms $p^*(\widetilde{S})\to\bigoplus_{n\geq0}\sh{L}^{\otimes n}$ of graded $\sh{O}_X$-algebras and endomorphisms of the graded ring $S$ \sref[I]{I.2.2.5};
the homomorphism $\varepsilon:p^*(\widetilde{S})\to\bigoplus_{n\geq0}\sh{L}^{\otimes n}$ that then corresponds to the \emph{identity} automorphism of $S$ is said to be \emph{canonical}.
There is a corresponding \sref{II.3.7.1} morphism $G(\varepsilon)\to\Proj(S)$ that is also said to be \emph{canonical}.
\end{env}

\begin{theorem}[4.5.2]
\label{II.4.5.2}
Let $X$ be a quasi-compact scheme or a prescheme whose underlying space is Noetherian, $\sh{L}$ an invertible $\sh{O}_X$-module, and $S$ the graded ring $\bigoplus_{n\geq0}\Gamma(X,\sh{L}^{\otimes n})$.
Then the following conditions are equivalent:
\begin{enumerate}
  \item[\rm{(a)}] When $f$ runs over the set of homogeneous elements of $S_+$, the $X_f$ form a base of the topology of $X$.
  \item[\rm{(a')}] When $f$ runs over the set of homogeneous elements of $S_+$, the $X_f$ that are affine form a cover of $X$.
  \item[\rm{(b)}] The canonical morphism $G(\varepsilon)\to\Proj(S)$ \sref{II.4.5.1} is everywhere defined and is a dominant open immersion.
\oldpage[II]{84}
  \item[\rm{(b')}] The canonical morphism $G(\varepsilon)\to\Proj(S)$ is everywhere defined and is a homeomorphism from the underlying space of $X$ to a subspace of $\Proj(S)$.
  \item[\rm{(c)}] For every quasi-coherent $\sh{O}_X$-module $\sh{F}$, if we denote by $\sh{F}_n$ the sub-$\sh{O}_X$-module of $\sh{F}(n)$ generated by the sections of $\sh{F}(n)$ over $X$, then $\sh{F}$ is the sum of the sub-$\sh{O}_X$-modules $\sh{F}_n(-n)$ over the integers $n>0$.
  \item[\rm{(c')}] Property~\rm{(c)} holds for every quasi-coherent sheaf of ideals of $\sh{O}_X$.
\end{enumerate}

Furthermore, if $(f_\alpha)$ is a family of homogeneous elements of $S_+$ such that the $X_{f_\alpha}$ are affine, then the restriction to $\bigcup_\alpha X_{f_\alpha}$ of the canonical morphism $X\to\Proj(S)$ is an isomorphism from $\bigcup_\alpha X_{f_\alpha}$ to $\bigcup_\alpha(\Proj(S))_{f_\alpha}$.
\end{theorem}

\begin{proof}
  It is clear that (b) implies (b'), and (b') implies (a) by \sref{II.3.7.3.1} (taking into account the fact that $\varepsilon^\flat$ is the identity).
  Condition~(a) implies (a'), since every $x\in X$ has an affine neighbourhood $U$ such that $\sh{L}|U$ is isomorphic to $\sh{O}_X|U$;
  if $f\in\Gamma(X,\sh{L}^{\otimes n})$ is such that $x\in X_f\subset U$, then $X_f$ is also the set of $x'\in U$ such that $(f|U)(x')\neq0$, and it is thus an affine open subset \sref[I]{I.1.3.6}.
  To prove that (a') implies (b), it suffices to prove the last claim of the theorem, and to further prove that, if $X=\bigcup_\alpha X_{f_\alpha}$, then condition~(iv) of \sref{II.3.8.2} is satisfied.
  This latter point follows immediately from \sref[I]{I.9.3.1}[(i)].
  As for the last claim of \sref{II.4.5.2}, since $X_{f_\alpha}$ is the inverse image of $(\Proj(S))_{f_\alpha}$ under $G(\varepsilon)\to\Proj(S)$, it suffices to apply \sref[I]{I.9.3.2}.
  Thus (a), (a'), (b), and (b') are all equivalent.

  To show that (a') implies (c), note that, if $X_f$ is affine (with $f\in S_k$), then $\sh{F}|X_f$ is generated by its sections over $X_f$ \sref[I]{I.1.3.9};
  on the other hand \sref[I]{I.9.3.1}[(ii)], such a section $s$ is of the form $(t|X_f)\otimes(f|X_f)^{-m}$, where $t\in\Gamma(X,\sh{F}(km))$;
  by definition, $t$ is also a section of $\sh{F}_{km}$, so $s$ is indeed a section of $\sh{F}_{km}(-km)$ over $X_f$, which proves (c).
  It is clear that (c) implies (c'), so it remains only to show that (c') implies (a).
  But let $U$ be an open neighbourhood of $x\in X$, and let $\sh{I}$ be a quasi-coherent sheaf of ideals of $\sh{O}_X$ defining a closed subprescheme of $X$ that has $X\setmin U$ as its underlying space \sref[I]{I.5.2.1}.
  Hypothesis~(c') implies that there exists an integer $n>0$ and a section $f$ of $\sh{I}(n)$ over $X$ such that $f(x)\neq0$.
  But we clearly have $f\in S_n$, and $x\in X_f\subset U$, which proves (a).
\end{proof}

When $X$ is a prescheme whose underlying space is Noetherian, the equivalent conditions of \sref{II.4.5.2} imply that $X$ is a \emph{scheme}, since it is isomorphic to a subprescheme of the scheme $S=\Proj(A)$, by \sref{II.4.5.2}[(b)].

\begin{definition}[4.5.3]
\label{II.4.5.3}
We say that an invertible $\sh{O}_X$-module $\sh{L}$ is \emph{ample} if $X$ is a quasi-compact scheme and if the equivalent conditions of \sref{II.4.5.2} are satisfied.
\end{definition}

It evidently follows from criterion~(a) of \sref{II.4.5.2} that, if $\sh{L}$ is an ample $\sh{O}_X$-module, then, for every open subset $U$ of $X$, $\sh{L}|U$ is an ample $(\sh{O}_X|U)$-module.

It follows from the proof of \sref{II.4.5.2} that the \emph{affine} $X_f$ form a base of the topology of $X$.
Furthermore:

\begin{corollary}[4.5.4]
\label{II.4.5.4}
Let $\sh{L}$ be an ample $\sh{O}_X$-module.
For every finite subspace $Z$ of $X$ and every neighbourhood $U$ of $Z$, there exists an integer $n$ and some $f\in\Gamma(X,\sh{L}^{\otimes n})$ such that $X_f$ is an affine neighbourhood of $Z$ contained in $U$.
\end{corollary}

\begin{proof}
\oldpage[II]{85}
By \sref{II.4.5.2}[(b)], it suffices to prove that, for every finite subset $Z'$ of $\Proj(S)$ and every open neighbourhood $U$ of $Z'$, there exists a homogeneous element $f\in S_+$ such that $Z\subset(\Proj(S))_f\subset U$ \sref{II.2.4.1}.
But, by definition, the closed set $Y$, complement of $U$ in $\Proj(S)$, is of the form $V_+(\mathfrak{I})$, where $\mathfrak{I}$ is a graded ideal of $S$ that does not contain $S_+$ \sref{II.2.3.2};
also, the points of $Z'$ are, by definition, graded ideals $\mathfrak{p}_i$ of $S_+$ that do not contain $\sh{I}$ \sref{II.2.3.1}.
There thus exists an element $f\in\mathfrak{I}$ that does not belong to any of the $\mathfrak{p}_i$ (Bourbaki, \emph{Alg. comm.}, chap.~II, \S1, no.~1, prop.~2), and, since the $\mathfrak{p}_i$ are graded, the argument made \emph{loc. cit.} shows that we can even take $f$ to be homogeneous;
this element then satisfies the claim.
\end{proof}

\begin{proposition}[4.5.5]
\label{II.4.5.5}
Suppose that $X$ is a quasi-compact scheme or a prescheme whose underlying space is Noetherian.
Then conditions~(a) to (c') of \sref{II.4.5.2} are equivalent to the following:
\begin{enumerate}
  \item[\rm{(d)}] For every quasi-coherent $\sh{O}_X$-module $\sh{F}$ of finite type, there exists an integer $n_0$ such that, for all $n\geq n_0$, $\sh{F}(n)$ is generated by its sections over $X$.
  \item[\rm{(d')}] For every quasi-coherent $\sh{O}_X$-module $\sh{F}$ of finite type, there exist integers $n>0$ and $k>0$ such that $\sh{F}$ is isomorphic to a quotient of the $\sh{O}_X$-module $\sh{L}^{\otimes(-n)}\otimes\sh{O}_X^k$.
  \item[\rm{(d'')}] Property~(d') holds for every quasi-coherent sheaf of ideals of $\sh{O}_X$ of finite type.
\end{enumerate}
\end{proposition}

\begin{proof}
Since $X$ is quasi-compact, if a quasi-coherent $\sh{O}_X$-module $\sh{F}$ of finite type is such that $\sh{F}(n)$ (which is of finite type) is generated by its sections over $X$, then $\sh{F}(n)$ is generated by a \emph{finite} number of these sections \sref[0]{0.5.2.3}, and so (d) implies (d'), and it is clear that (d') implies (d'').
Since every quasi-coherent $\sh{O}_X$-module $\sh{G}$ is the inductive limits of its sub-$\sh{O}_X$-modules of finite type \sref[I]{I.9.4.9}, to satisfy condition~(c') of \sref{II.4.5.2}, it suffices to do so for a quasi-coherent sheaf of ideals of $\sh{O}_X$ that is of finite type, and (d'') thus implies (c').
It remains only to show that, if $\sh{L}$ is ample, then property~(d) is satisfied.
Consider a finite cover of $X$ by $X_{f_i}$ ($f_i\in S_{n_i}$), that we can assume to be affine;
by replacing the $f_i$ with suitable powers (which does not alter the $X_{f_i}$), we can assume that all the $n_i$ are equal to one single integer $m$.
The sheaf $\sh{F}|X_{f_i}$, being of finite type, by hypothesis, is generated by a finite number of its sections $h_{ij}$ over $X_{f_i}$ \sref[I]{I.1.3.13};
so there exists an integer $k_0$ such that the section $h_{ij}\otimes f_i^{\otimes k_0}$ extends to a section of $\sh{F}(k_0m)$ over $X$ for every pair $(i,j)$ \sref[I]{I.9.3.1}.
\emph{A fortiori}, the $h_{ij}\otimes f_i^{\otimes k_0}$ extend to sections of $\sh{F}(km)$ over $X$ for every $k\geq k_0$, and, for these values of $k$, $\sh{F}(km)$ is thus generated by its sections over $X$.
For every $p$ such that $0<p<m$, $\sh{F}(p)$ is also of finite type, and so there exists an integer $k_p$ such that $\sh{F}(p)(km)=\sh{F}(p+km)$ is generated by its sections over $X$ for all $k\geq k_p$.
Taking $n_0$ to be the largest of the $k_pm$, we thus conclude that $\sh{F}(n)$ is generated by its sections over $X$ for all $n\geq n_0$, since such an $n$ is of the form $n=km+p$, with $k\geq k_p$ and $0\leq p<m$.
\end{proof}

\begin{proposition}[4.5.6]
\label{II.4.5.6}
  Let $X$ be a quasi-compact scheme, and $\sh{L}$ an invertible $\sh{O}_X$-module.
  \begin{enumerate}
    \item[\rm{(i)}] Let $n>0$ be an integer. For $\sh{L}$ to be ample, it is necessary and sufficient for $\sh{L}^{\otimes n}$ to be ample.
    \item[\rm{(ii)}] Let $\sh{L}'$ be an invertible $\sh{O}_X$-module such that, for all $x\in X$, there exists an integer $n>0$
\oldpage[II]{86}
      and a section $s'$ of $\sh{L}'^{\otimes n}$ over $X$ such that $s'(x)\neq0$.
      Then, if $\sh{L}$ is ample, so too is $\sh{L}\otimes\sh{L}'$.
  \end{enumerate}
\end{proposition}

\begin{proof}
  Property~(i) is an evident consequence of criterion~(a) of \sref{II.4.5.2}, since $X_{f^{\otimes n}}=X_f$.
  On the other hand, if $\sh{L}$ is ample, then, for every $x\in X$ and every neighbourhood $U$ of $x$, there exists some $m>0$ and $f\in\Gamma(X,\sh{L}^{\otimes m})$ such that $x\in X_f\subset U$ \sref{II.4.5.2}[(a)];
  if $f'\in\Gamma(X,\sh{L}'^{\otimes n})$ is such that $f'(x)\neq0$, then $s(x)\neq0$ for $s=f^{\otimes n}\otimes f'^{\otimes m}\in\Gamma(X,(\sh{L}\otimes\sh{L}')^{\otimes mn})$, and so $x\in X_s\subset X_f\subset U$, which proves that $\sh{L}\otimes\sh{L}'$ is ample \sref{II.4.5.2}[(a)].
\end{proof}

\begin{corollary}[4.5.7]
\label{II.4.5.7}
The tensor product of two ample $\sh{O}_X$-modules is ample.
\end{corollary}

\begin{corollary}[4.5.8]
\label{II.4.5.8}
Let $\sh{L}$ be an ample $\sh{O}_X$-module, and $\sh{L}'$ an invertible $\sh{O}_X$-module;
then there exists an integer $n_0>0$ such that $\sh{L}^{\otimes n}\otimes\sh{L}'$ is ample and generated by its sections over $X$ for $n\geq n_0$.
\end{corollary}

\begin{proof}
It follows from \sref{II.4.5.5} that there exists an integer $m_0$ such that $\sh{L}^{\otimes m}\otimes\sh{L}'$ is generated by its sections over $X$ for all $m\geq m_0$;
by \sref{II.4.5.6}, we can then take $n_0=m_0+1$.
\end{proof}

\begin{remark}[4.5.9]
\label{II.4.5.9}
Let $P=\HH^1(X,\sh{O}_X^\times)$ be the group of classes of invertible $\sh{O}_X$-modules \sref[0]{0.5.4.7}, and let $P^+$ be the subset of $P$ consisting of classes of ample sheaves.
Suppose that $P^+$ is \emph{non-empty}.
Then it follows from \sref{II.4.5.7} and \sref{II.4.5.8} that
\[
  P^+ + P^+ \subset P^+
  \quad\text{and}\quad
  P^+ - P^+ = P
\]
or, in other words, $P^+\cup\{0\}$ is the set of \emph{positive} elements in $P$ for a \emph{preorder} structure on $P$ that is compatible with its group structure, and is even \emph{archimedian}, by \sref{II.4.5.8}.
This is why we sometimes say ``positive sheaf'' instead of ample sheaf, and ``negative sheaf'' for the inverse of an ample sheaf (but we will not use this terminology).
\end{remark}

\begin{proposition}[4.5.10]
\label{II.4.5.10}
Let $Y$ be an affine scheme, $q:X\to Y$ a quasi-compact separated morphism, and $\sh{L}$ an invertible $\sh{O}_X$-module.
\begin{enumerate}
  \item[\rm{(i)}] If $\sh{L}$ is very ample for $q$, then $\sh{L}$ is ample.
  \item[\rm{(ii)}] Suppose further that the morphism $q$ is \emph{of finite type}.
    Then, for $\sh{L}$ to be ample, it is necessary and sufficient for it to posses one of the following properties:
    \begin{enumerate}
      \item[\rm{(e)}] There exists $n_0>0$ such that, for every integer $n\geq n_0$, $\sh{L}^{\otimes n}$ is very ample for $q$.
      \item[\rm{(e')}] There exists $n>0$ such that $\sh{L}^{\otimes n}$ is very ample for $q$.
    \end{enumerate}
\end{enumerate}
\end{proposition}

\begin{proof}
The first claim follows from the definition \sref{II.4.4.2} of a very ample $\sh{O}_X$-module: if $A$ is the ring of $Y$, then there exists an $A$-module $E$ and a surjective homomorphism
\[
  \psi: q^*((\bb{S}(E))^{\supertilde}) \to \bigoplus_{n\geq0}\sh{L}^{\otimes n}
\]
such that $i=r_{\sh{L},\psi}$ is an everywhere-defined immersion $X\to P=\bb{P}(\widetilde{E})$ and such that $\sh{L}=i^*(\sh{O}_P(1))$;
since the $D_+(f)$ for $f$ homogeneous in $(\bb{S}(E))_+$ form a base for the topology of $P$, and since $i^{-1}(D_+(f))=X_{\psi^\flat(f)}$, by \sref{II.3.7.3.1}, we see that condition~(a) of \sref{II.4.5.2} is satisfied, and so $\sh{L}$ is ample.

Now to prove that, if $q$ is of finite type and $\sh{L}$ is ample, then condition~(e) is satisfied.
Firstly, it follows from criterion~(b) of \sref{II.4.5.2} and from \sref{II.4.4.1}[(i)] that there exists
\oldpage[II]{87}
an integer $k_0$ such that $\sh{L}^{\otimes k_0}$ is very ample relative to $q$.
Also, by \sref{II.4.5.5}, there exists an integer $m_0$ such that, for all $m\geq m_0$, $\sh{L}^{\otimes m}$ is generated by its sections over $X$.
Let $n_0=k_0+m_0$;
if $n\geq n_0$, then we can write $n=k_0+m$ with $m\geq m_0$, whence $\sh{L}^{\otimes n}=\sh{L}^{\otimes k_0}\otimes\sh{L}^{\otimes m}$.
Since $\sh{L}^{\otimes m}$ is generated by its sections over $X$, it follows from \sref{II.4.4.8} and \sref{II.3.4.7} that $\sh{L}^{\otimes n}$ is very ample relative to $q$.
Finally, it is clear that (e) implies (e'), and (e') implies that $\sh{L}$ is ample by (i) and by \sref{II.4.5.6}[(i)]

  \begin{env}[4.5.10.1]
  \label{II.4.5.10.1}
  \emph{[Proof of Lemma~\sref{II.4.4.10.1}].}  
  Let
  \end{env}
\end{proof}


% \subsection{Relatively ample sheaves}
% \label{subsection:II.4.6}
