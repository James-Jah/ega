\section{Projective bundles; ample sheaves}
\label{section:II.4}


\subsection{Definition of projective bundles}
\label{subsection:II.4.1}

\begin{definition}[4.1.1]
\label{II.4.1.1}
Let $Y$ be a prescheme, $\sh{E}$ a quasi-coherent $\sh{O}_Y$-module, and $\bb{S}_{\sh{O}_Y}(\sh{E})$ the symmetric $\sh{O}_Y$-algebra of $\sh{E}$ \sref{II.1.7.4}, which is quasi-coherent \sref{II.1.7.7}.
We define the \emph{projective bundle on $Y$ defined by $\sh{E}$}, denoted $\bb{P}(\sh{E})$, to be the $Y$-scheme $P=\Proj(\bb{S}_{\sh{O}_Y}(\sh{E}))$.
The $\sh{O}_P$-module $\sh{O}_P(1)$ is called the \emph{fundamental sheaf on $P$}.
\end{definition}

When $Y$ is affine of ring $A$, then we have $\sh{E}=\widetilde{E}$ for some $A$-module $E$, and we then write $\bb{P}(E)$ instead of $\bb{P}(\widetilde{E})$.

When we take $\sh{E}=\sh{O}_Y^n$, we write $\bb{P}_Y^{n-1}$ instead of $\bb{P}(\sh{E})$;
if, further, $Y$ is affine of ring $A$, then we also write $\bb{P}_A^{n-1}$ instead of $\bb{P}_Y^{n-1}$.
Since $\bb{S}_{\sh{O}_Y}(\sh{O}_Y)$ is canonically identified with $\sh{O}_Y[T]$ \sref{II.1.7.4}, $\bb{P}_Y^0$ is canonically identified with $Y$ \sref{II.3.1.7};
Example~\sref{II.2.4.3} is then exactly $\bb{P}_K^1$.

\begin{env}[4.1.2]
\label{II.4.1.2}
Let $\sh{E}$ and $\sh{F}$ be quasi-coherent $\sh{O}_Y$-modules;
let $u:\sh{E}\to\sh{F}$ be an $\sh{O}_Y$-homomorphism;
there is a canonically corresponding homomorphism $\bb{S}(u):\bb{S}_{\sh{O}_Y}(\sh{E})\to\bb{S}_{\sh{O}_Y}(\sh{F})$ of graded $\sh{O}_Y$-algebras \sref{II.1.7.4}.
If $u$ is \emph{surjective}, then so too is $\bb{S}(u)$, and thus \sref{II.3.6.2}[(i)] $\Proj(\bb{S}(u))$ is a \emph{closed immersion} $\bb{P}(\sh{F})\to\bb{P}(\sh{E})$, which we denote by $\bb{P}(u)$.
We can thus say that $\bb{P}(\sh{E})$ is a \emph{contravariant functor} in $\sh{E}$, with the condition that we only consider \emph{surjective} morphisms of quasi-coherent $\sh{O}_Y$-modules.

Still supposing that $u$ is surjective, and letting $P=\bb{P}(\sh{E})$, $Q=\bb{P}(\sh{F})$, and $j=\bb{P}(u)$, we have, up to isomorphism, that
\[
\label{II.4.1.2.1}
  j^*(\sh{O}_P(n)) = \sh{O}_Q(n)
  \qquad\mbox{for all $n\in\bb{Z}$}
  \tag{4.1.2.1}
\]
by \sref{II.3.6.3}.
\end{env}

\begin{env}[4.1.3]
\label{II.4.1.3}
Now let $\psi:Y'\to Y$ be a morphism, and let $\sh{E}'=\psi^*(\sh{E})$;
then $\bb{S}_{\sh{O}_{Y'}}(\sh{E}') = \psi^*(\bb{S}_{\sh{O}_Y}(\sh{E}))$ \sref{II.1.7.5};
thus \sref{II.3.5.3}
\[
\label{II.4.1.3.1}
  \bb{P}(\psi^*(\sh{E})) = \bb{P}(\sh{E})\times_Y Y'
  \tag{4.1.3.1}
\]
up to canonical isomorphism;
furthermore, we clearly have that
\[
  \psi^*((\bb{S}_{\sh{O}_Y}(\sh{E}))(n)) = (\bb{S}_{\sh{O}_{Y'}}(\sh{E}'))(n)
\]
for all $n\in\bb{Z}$, whence, letting $P=\bb{P}(\sh{E})$ and $P'=\bb{P}(\sh{E}')$, we have \sref{II.3.5.4}, up to isomorphism, that
\[
\label{II.4.1.3.2}
  \sh{O}_{P'}(n) = \sh{O}_p(n)\otimes_Y\sh{O}_{Y'}
  \qquad\mbox{for all $n\in\bb{Z}$.}
  \tag{4.1.3.2}
\]
\end{env}

\oldpage[II]{72}
\begin{proposition}[4.1.4]
\label{II.4.1.4}
Let $\sh{L}$ be an invertible $\sh{O}_Y$-module.
For every quasi-coherent $\sh{O}_Y$-module $\sh{E}$, there exists a canonical $Y$-isomorphism $i_\sh{L}:\bb{P}(\sh{E})\xrightarrow{\sim}\bb{P}(\sh{E}\otimes\sh{L})$;
furthermore, if we let $P=\bb{P}(\sh{E})$ and $Q=\bb{P}(\sh{E}\otimes\sh{L})$, then $i_\sh{L}^*(\sh{O}_Q(n))$ is canonically isomorphic to $\sh{O}_P(n)\otimes_Y\sh{L}^{\otimes n}$ for all $n\in\bb{Z}$.
\end{proposition}

\begin{proof}
Note first of all that, if $A$ is a ring, $E$ an $A$-module, and $L$ a \emph{free monogenous} $A$-module, then we can canonically define a homomorphism of $A$-modules
\[
  \bb{S}_n(E\otimes L) \to \bb{S}_n(E)\otimes L^{\otimes n}
\]
by sending $(x_1\otimes y_1)\ldots(x_n\otimes y_n)$ to the element
\[
  (x_1x_2\ldots x_n)\otimes(y_1\otimes y_2\otimes\ldots\otimes y_n)
  \qquad\mbox{($x_i\in E$, $y_i\in L$, for $i\leq i\leq n$);}
\]
we can immediately see (by restricting to the case where $L=A$) that this homomorphism is in fact an isomorphism.
We thus obtain a canonical isomorphism of graded $A$-algebras $\bb{S}_A(E\otimes L)\xrightarrow{\sim}\bigoplus_{n\geq0}\bb{S}_n(E)\otimes L^{\otimes n}$.
By returning to the conditions of \sref{II.4.1.4}, the above remarks allow us to define a canonical isomorphism of graded $\sh{O}_Y$-algebras
\[
\label{II.4.1.4.1}
  \bb{S}_{\sh{O}_Y}(\sh{E}\otimes_{\sh{O}_Y}\sh{L}) \xrightarrow{\sim} \bigoplus_{n\geq0}\bb{S}_n(\sh{E})\otimes_{\sh{O}_Y}\sh{L}^{\otimes n}
  \tag{4.1.4.1}
\]
by defining this isomorphism as an isomorphism of presheaves, and taking into account \sref{II.1.7.4}, \sref[I]{I.1.3.9}, and \sref[I]{I.1.3.12}.
The proposition then follows from \sref{II.3.1.8}[(iii)] and \sref{II.3.2.10}.
\end{proof}

\begin{env}[4.1.5]
\label{II.4.1.5}
With the hypotheses of \sref{II.4.1.1}, let $P=\bb{P}(\sh{E})$, and denote by $p$ the structure morphism $P\to Y$.
Since, by definition, $\sh{E}=(\bb{S}_{\sh{O}_Y}(\sh{E}))_1$, we have a canonical homomorphism $\alpha_1:\sh{E}\to p_*(\sh{O}_P(1))$ \sref{II.3.3.2.2}, and thus \sref[0]{0.4.4.3} also a canonical homomorphism
\[
\label{II.4.1.5.1}
  \alpha_1^\sharp: p^*(\sh{E}) \to \sh{O}_P(1).
  \tag{4.1.5.1}
\]
\end{env}

\begin{proposition}[4.1.6]
\label{II.4.1.6}
The canonical homomorphism \sref{II.4.1.5.1} is surjective.
\end{proposition}

\begin{proof}
We have seen, in \sref{II.3.3.2}, that $\alpha_1^\sharp$ corresponds functorially to the canonical homomorphism $\sh{E}\otimes_{\sh{O}_Y}\bb{S}_{\sh{O}_Y}(\sh{E}) \to (\bb{S}_{\sh{O}_Y}(\sh{E}))(1)$;
since, by definition, $\sh{E}$ generates $\bb{S}_{\sh{O}_Y}(\sh{E})$, this homomorphism is surjective, whence the conclusion, by \sref{II.3.2.4}
\end{proof}


\subsection{Morphisms from a prescheme to a projective bundle}
\label{subsection:II.4.2}

\begin{env}[4.2.1]
\label{II.4.2.1}
Keeping the notation of \sref{II.4.1.5}, let $X$ be a $Y$-prescheme, $q:X\to Y$ the structure morphism, and let $r:X\to P$ be a $Y$-\emph{morphism} such that the following diagram commutes:
\[
  \xymatrix{
    P \ar[d]_p & X \ar[l]_r \ar[dl]^q
  \\Y
  }
\]
\oldpage[II]{73}

Since the functor $r^*$ is right exact \sref[0]{0.4.3.1}, we obtain, from the surjective homomorphism in \sref{II.4.1.5.1}, a surjective homomorphism
\[
  r^*(\alpha_1^\sharp): r^*(p^*(\sh{E})) \to r^*(\sh{O}_P(1)).
\]

But $r^*(p^*(\sh{E}))=q^*(\sh{E})$, and $r^*(\sh{O}_P(1))$ is locally isomorphic to $r^*(\sh{O}_P)=\sh{O}_X$, or, in other words, the latter is an \emph{invertible} sheaf $\sh{L}_r$ on $\sh{O}_X$, and so we have defined, given $r$, a canonical surjective $\sh{O}_X$-homomorphism
\[
\label{II.4.2.1.1}
  \varphi_r:q^*(\sh{E}) \to \sh{L}_r.
  \tag{4.2.1.1}
\]

When $Y=\Spec(A)$ is affine, and $\mathscr{E}=\widetilde{E}$, we can further clarify this homomorphism in the following way:
given $f\in E$, it follows from \sref{II.2.6.3} that
\[
\label{II.4.2.1.2}
  r^{-1}(D_+(f)) = X_{\varphi_r^\flat(f)}.
  \tag{4.2.1.2}
\]

Now let $V$ be an affine open subset of $X$ that is contained inside $r^{-1}(D_+(f))$, and let $B$ be its ring, which is an $A$-algebra;
let $S=\bb{S}_A(E)$;
the restriction of $r$ to $V$ corresponds to an $A$-homomorphism $\omega:\bb{S}_f\to B$, and we have that $q^*(\sh{E})|V = (E\otimes_A B)\supertilde$ and $\sh{L}_r|V = \widetilde{L_r}$, whence $L_r = (S(1))_{(f)}\otimes_{S_{(f)}}B_{[\omega]}$ \sref[I]{I.1.6.5}.
The restriction of $\varphi_r$ to $q^*(\sh{E})|V$ corresponds to the $B$-homomorphism $u:E\otimes_A B\to L_r$, which sends $x\otimes1$ to $(x/1)\otimes f = (f/1)\otimes\omega(x/f)$.
The canonical extension of $\varphi_r$ to a homomorphism of $\sh{O}_X$-algebras
\[
  \psi_r: q^*(\bb{S}(\sh{E})) = \bb{S}(q^*(\sh{E})) \to \bb{S}(\sh{L}_r) = \bigoplus_{n\geq0}\sh{L}_r^{\otimes n}
\]
is thus such that the restriction of $\psi_r$ to $q^*(\bb{S}_n(\sh{E}))|V$ corresponds to the homomorphism $\bb{S}_n(\sh{E}\otimes_A B) = \bb{S}_n(E)\otimes_A B \to L_r^{\otimes n}$ that sends $s\otimes1$ to $(f/1)^{\otimes n}\otimes\omega(s/f^n)$.
\end{env}

\begin{env}[4.2.2]
\label{II.4.2.2}
Conversely, suppose that we are given a morphism $q:X\to Y$, an invertible $\sh{O}_X$-module $\sh{L}$, and a quasi-coherent $\sh{O}_Y$-module $\sh{E}$;
to each homomorphism $\varphi:q^*(\sh{E})\to\sh{L}$ there canonically corresponding homomorphism of quasi-coherent $\sh{O}_X$-algebras
\[
  \psi: \bb{S}(q^*(\sh{E})) = q^*(\bb{S}(\sh{E})) \to \bigoplus_{n\geq0}\sh{L}^{\otimes n}
\]
and thus \sref{II.3.7.1} a $Y$-morphism $r_{\sh{L},\psi}:G(\psi)\to\Proj(\bb{S}(\sh{E}))=\bb{P}(\sh{E})$, which we denote by $r_{\sh{L},\varphi}$.
If $\varphi$ is \emph{surjective}, then so too is $\psi$, and thus \sref{II.3.7.4} $r_{\sh{L},\varphi}$ is \emph{everywhere defined}.
Furthermore, with the notation of \sref{II.4.2.1} and \sref{II.4.2.2}:
\end{env}

\begin{proposition}[4.2.3]
\label{II.4.2.3}
Given a morphism $q:X\to Y$ and a quasi-coherent $\sh{O}_Y$-module $\sh{E}$, maps $r\to(\sh{L}_r,\varphi_r)$ and $(\sh{L},\varphi)\to r_{\sh{L},\varphi}$ give a bijective correspondence between the set of $Y$-morphisms $r:X\to\bb{P}(\sh{E})$ and the set of equivalence classes of pairs $(\sh{L},\varphi)$ of an invertible $\sh{O}_X$-module $\sh{L}$ and a surjective homomorphism $\varphi:q^*(\sh{E})\to\sh{L}$, where such pairs $(\sh{L},\varphi)$ and $(\sh{L}',\varphi')$ are defined to be equivalent if there exists an $\sh{O}_X$-isomorphism $\tau:\sh{L}\xrightarrow{\sim}\sh{L}'$ such that $\varphi'=\tau\circ\varphi$.
\end{proposition}

\begin{proof}
Start first with a $Y$-morphism $r:X\to\bb{P}(\sh{E})$, and construct $\sh{L}_r$ and $\varphi_r$ \sref{II.4.2.1}, and let $r'=r_{\sh{L}_r,\varphi_r}$;
it follows immediately from \sref{II.4.2.1} and \sref{II.3.7.2} that the morphisms $r$ and $r'$ are identical (by taking the generator of $\sh{L}_r$ to be the element $(f/1)\otimes1$ to define the homomorphisms $v_n$ of \sref{II.3.7.2}).
Conversely, take a pair $(\sh{L},\varphi)$ and construct
\oldpage[II]{74}
$r''=r_{\sh{L},\varphi}$, and then $\sh{L}_{r''}$ and $\varphi_{r''}$;
we will show that there exists a canonical isomorphism $\tau:\sh{L}_{r''}\xrightarrow{\sim}\sh{L}$ such that $\varphi=\tau\circ\varphi_{r''}$;
to define it, we can restrict to the case where $Y=\Spec(A)$ and $X=\Spec(B)$ are affine, and (with the notation of \sref{II.4.2.1} and \sref{II.3.7.2}) associate to each element $(x/1)\otimes1$ of $L_{r''}$ (where $x\in E$) the element $v_1(x)c$ of $L$.
We immediately see that $\tau$ does not depend on the chosen generator $c$ of $L$;
since $v_1$ is surjective by hypothesis, to prove that $\tau$ is an isomorphism it suffices to to show that, if $x/1=0$ in $(S(1))_{(f)}$, then $v_1(x)/1=0$ in $B_g$;
but the first equality implies that $f^nx=0$ in $\bb{S}_{n+1}(E)$ for some $n$, and this implies that $v_{n+1}(f^nx) = g^nv_1(x) = 0$ in $B$, whence the conclusion.
Finally, it is immediate that, for any two equivalent pairs $(\sh{L},\varphi)$ and $(\sh{L}',\varphi')$, we have $r_{\sh{L},\varphi}=r_{\sh{L}',\varphi'}$.
\end{proof}

In particular, for $X=Y$:
\begin{theorem}[4.2.4]
\label{II.4.2.4}
The set of $Y$-sections of $\bb{P}(\sh{E})$ is in canonical bijective correspondence with the set of quasi-coherent sub-$\sh{O}_Y$-modules $\sh{F}$ of $\sh{E}$ such that $\sh{E}/\sh{F}$ is invertible.
\end{theorem}

We note that this property corresponds to the classical definition of ``projective space'' as the set of hyperplanes of a vector space (the classical case corresponding to $Y=\Spec(K)$, where $K$ is a field, and $\sh{E}=\widetilde{E}$, where $E$ is a finite-dimensional $K$-vector space; the $\sh{F}$ having the property described in \sref{II.4.2.4} then correspond to the hyperplanes of $E$, and we already know that the $Y$-sections of $\bb{P}(\sh{E})$ are then the \emph{$K$-rational points of $\bb{P}(\sh{E})$} \sref[I]{I.3.4.5}).

\begin{remark}[4.2.5]
\label{II.4.2.5}
Since there is a canonical bijective correspondence between $Y$-morphisms from $X$ to $P$ and their graph morphisms, $X$-sections of $P\times_Y X$ \sref[I]{I.3.3.14}, we see that, conversely, \sref{II.4.2.3} can be deduced from \sref{II.4.2.4}.
Denote by $\Hyp_Y(X,\sh{E})$ the set of quasi-coherent sub-$\sh{O}_X$-modules $\sh{F}$ of $\sh{E}\otimes_Y\sh{O}_X=q^*(\sh{E})$ such that $q^*(\sh{E})/\sh{F}$ is an invertible $\sh{O}_X$-module.
If $g:X'\to X$ is a $Y$-morphism, then it follows from the fact that $g^*$ is right exact that $g^*q^*(\sh{E})/\sh{F})=g^*q^*(\sh{E}))/g^*(\sh{F})$, and so the latter sheaf is invertible, and thus $\Hyp_Y(X,\sh{E})$ is a \emph{contravariant functor} into the category of $Y$-preschemes.
We can thus interpret the theorem \sref{II.4.2.4} as defining a \emph{canonical isomorphism} of functors $\Hom_Y(X,\bb{P}(\sh{E}))$ and $\Hyp_Y(X,\sh{E})$, where both functors are contravariant in the variable $X$ and map into the category of $Y$-preschemes.
This also gives a characterisation of the projective bundle $P=\bb{P}(\sh{E})$ by the following \emph{universal property}, which is much closer to the geometric intuition than the constructions from §§2--3:
for every morphism $q:X\to Y$ and every invertible $\sh{O}_X$-module $\sh{L}$ that is a quotient of $\sh{E}\otimes_{\sh{O}_Y}\sh{O}_X$, there exists a unique $Y$-morphism $r:X\to P$ such that $\sh{L}=r^*(\sh{O}_P(1))$.

We will see later that we can similarly define, amongst other things, ``Grassmannian'' schemes.
\end{remark}

\begin{corollary}[4.2.6]
\label{II.4.2.6}
Suppose that every invertible $\sh{O}_Y$-module is trivial \sref[I]{I.2.4.8}.
Let $V$ be the group $\Hom_{\sh{O}_Y}(\sh{E},\sh{O}_Y)$, considered as a module over the ring $A=\Gamma(Y,\sh{O}_Y)$, and let $V^*$ be the subset of $V$ consisting of surjective homomorphisms.
Then the set of $Y$-sections of $\bb{P}(\sh{E})$ is canonically identified with $V^*/A^*$, where $A^*$ is the group of units of $A$.
\end{corollary}

\oldpage[II]{75}
In particular:
\begin{enumerate}
  \item The corollary \sref{II.4.2.6} applies whenever $Y$ is a \emph{local scheme} \sref[I]{I.2.4.8}.
    Let $Y$ be an arbitrary prescheme, $y$ a point of $Y$, and $Y'=\Spec(\kres(y))$;
    then the fibre $p^{-1}(y)$ of $\bb{P}(\sh{E})$ can, by \sref{II.4.1.3.1}, be identified with $\bb{P}(\sh{E}^y)$, where $\sh{E}^y = \sh{E}_y\otimes_{\sh{O}_y}\kres(y) = \sh{E}_y/\mathfrak{m}_y\sh{E}_y$ is considered as a vector space over $\kres(y)$.
    More generally, if $K$ is an extension of $\kres(y)$, then $p^{-1}(y)\otimes_{\kres(y)}K$ can be identified with $\bb{P}(\sh{E}^y\otimes_{\kres(y)}K)$.
    The corollary \sref{II.4.2.6} then shows that the set of \emph{geometric points of $\bb{P}(\sh{E})$ with values in the extension $K$ of $\kres(y)$} \sref[I]{I.3.4.5}, which we can also call the \emph{rational geometric fibre over $K$ of $\bb{P}(\sh{E})$ over the point $y$}, can be identified with the \emph{projective space} associated to the \emph{dual} of the $K$-vector space $\sh{E}^y\otimes_{\kres(y)}K$.
  \item Suppose that $Y$ is affine of ring $A$, and, further, that every invertible $\sh{O}_Y$-module is trivial;
    further, take $\sh{E}=\sh{O}_Y^n$;
    then, in \sref{II.4.2.6}, $V$ can be identified with $A^n$ \sref[I]{I.1.3.8}, and $V^*$ with the sets of systems $(f_i)_{1\leq i\leq n}$ of elements of $A$ that generate the ideal $A$;
    any two such systems define the same $Y$-section of $\bb{P}_Y^{n-1}=\bb{P}_A^{n-1}$, or, in other words, \emph{the same point of $\bb{P}_A^{n-1}$ with values in $A$}, if and only if one of them can be obtained from the other by multiplication by an invertible element of $A$.
\end{enumerate}

These properties justify the terminology ``projective bundle'' for $\bb{P}(\sh{E})$.
We note that the definitions that we will similarly obtain for ``projective space'' is in fact \emph{dual} to the classical definition;
this is imposed upon us by the necessity of being able to define $\bb{P}(\sh{E})$ for \emph{arbitrary} quasi-coherent $\sh{O}_Y$-modules $\sh{E}$, and not just locally free ones.

\begin{remark}[4.2.7]
\label{II.4.2.7}
We will see, in Chapter~V, that, if $Y$ is connected and locally Noetherian, and if $\sh{E}$ is locally free, then, letting $P=\bb{P}(\sh{E})$, every invertible $\sh{O}_P$-module is isomorphic to an $\sh{O}_P$-module of the form $\sh{L}'\otimes_{\sh{O}_Y}\sh{O}_P(m)$, with $\sh{L}'$ some invertible $\sh{O}_Y$-module, well defined up to isomorphism, and $m$ some well defined integer.
In other words, $\HH^1(P,\sh{O}_P^*)$ is isomorphic to $\bb{Z}\times\HH^1(Y,\sh{O}_Y^*)$ \sref[0]{0.5.4.7}.
We will also see (\sref[III]{III.2.1.14}, taking \sref[0]{0.5.4.10} into account) that $p_*(\sh{L}^{\otimes m})=0$ if $m<0$, and $p_*(\sh{L}^{\otimes m})$ is isomorphic to $\sh{L}'\otimes_{\sh{O}_Y}(\bb{S}_{\sh{O}_Y}(\sh{E}))_m$ if $m\geq0$.
If $\sh{F}$ is a quasi-coherent $\sh{O}_Y$-module, then every $Y$-morphism $\bb{P}(\sh{E})\to\bb{P}(\sh{F})$ is determined by the data of an invertible $\sh{O}_Y$-module, an integer $m\geq0$, and an $\sh{O}_Y$-homomorphism $\psi:\sh{F}\to\sh{L}'\otimes_{\sh{O}_Y}(\bb{S}_{\sh{O}_Y}(\sh{E}))_m$ such that the corresponding homomorphism $\psi^\sharp$ of $\sh{O}_{\bb{P}(\sh{F})}$-modules is surjective.
We will also see that, if the $Y$-morphism in question is an isomorphism, then $m=1$ and $\sh{F}$ is isomorphic to $\sh{E}\otimes_{\sh{O}_Y}\sh{L}'$ (the converse of \sref{II.4.1.4}).
This will allow us to determine the sheaf of germs of automorphisms of $\bb{P}(\sh{E})$ as the quotient of the sheaf of groups $\shAut(\sh{E})$ (which is locally isomorphic to $\GL(n,\sh{O}_Y)$ is $\sh{E}$ is of rank $n$) by $\sh{O}_Y^*$.
\end{remark}

\begin{env}[4.2.8]
\label{II.4.2.8}
Keeping the notation of \sref{II.4.2.1}, let $u:X'\to X$ be a morphism;
if the $Y$-morphism $r:X\to P$ corresponds to the homomorphism $\varphi:q^*(\sh{E})\to\sh{L}$, then the $Y$-morphism $r\circ u$ corresponds to $u^*(\varphi):u^*(q^*(\sh{E}))\to u^*(\sh{L})$, as follows immediately from the definitions.
\end{env}

\begin{env}[4.2.9]
\label{II.4.2.9}
Let $\sh{E}$ and $\sh{F}$ be quasi-coherent $\sh{O}_Y$-modules, $v:\sh{E}\to\sh{F}$ a surjective homomorphism, and $j=\bb{P}(v)$ the corresponding closed immersion $\bb{P}(\sh{F})\to\bb{P}(\sh{E})$ \sref{II.4.1.2}.
If the $Y$-morphism $r:X\to\bb{P}(\sh{F})$ corresponds to the homomorphism $\varphi:q^*(\sh{F})\to\sh{L}$, then the
\oldpage[II]{76}
$Y$-morphism $j\circ r$ corresponds to $q^*(\sh{E})\xrightarrow{q^*(v)}q^*(\sh{F})\xrightarrow{\varphi}\sh{L}$;
this again follows from the definition given in \sref{II.4.2.1}.
\end{env}

\begin{env}[4.2.10]
\label{II.4.2.10}
Let $\psi:Y'\to Y$ be a morphism, and let $\sh{E}'=\psi^*(\sh{E})$.
If the $Y$-morphism $r:X\to P$ corresponds to the homomorphism $\varphi:q^*(\sh{E})\to\sh{L}$, then the $Y'$-morphism
\[
  r_{(Y')}: X_{(Y')} \to P' = \bb{P}(\sh{E}')
\]
corresponds to $\varphi_{(Y')}:q_{(Y')}^*(\sh{E}') = q^*(\sh{E})\otimes_Y\sh{O}_{Y'} \to \sh{L}\otimes_Y\sh{O}_{Y'}$.
Indeed, by \sref{II.4.1.3.1}, we have the commutative diagram
\[
  \xymatrix{
    Y' \ar[d]
    & P'=P_{(Y')} \ar[l]_{p_{(Y')}} \ar[d]^{u}
    & X_{(Y')} \ar[l]_{r_{(Y')}} \ar[d]^{v}
  \\Y
    & P \ar[l]_{p}
    & X \ar[l]_{r}
  }
\]

From \sref{II.4.1.3.1}, we have
\[
  (r_{(Y')})^*(\sh{O}_{P'}(1)) = (r_{(Y')})^*(u^*(\sh{O}_P(1))) = v^*(r^*(\sh{O}_P(1))) = v^*(\sh{L}) = \sh{L}\otimes_Y\sh{O}_{Y'};
\]
we also know that $u^*(\alpha_1^\sharp)$ is exactly the canonical homomorphism $\alpha_1^\sharp:(p_{(Y')})^*(\sh{E}')\to\sh{O}_{P'}(1)$;
we can see this by explicitly calculating the canonical homomorphisms $\alpha_1^\sharp$ to $P$ and $P'$ as in \sref{II.4.1.6}.
Whence our claim.
\end{env}


\subsection{The Segre morphism}
\label{subsection:II.4.3}

\begin{env}[4.3.1]
\label{II.4.3.1}
Let $Y$ be a prescheme, and $\sh{E}$ and $\sh{F}$ quasi-coherent $\sh{O}_Y$-modules;
let $P_1=\bb{P}(\sh{E})$ and $P_2=\bb{P}(\sh{F})$, and denote the structure morphisms by $p_1:P_1\to Y$ and $p_2:P_2\to Y$.
Let $Q=P_1\times_Y P_2$, and let $q_1:Q\to P_1$ and $q_2:Q\to P_2$ be the canonical projections;
then the $\sh{O}_Q$-module $\sh{L}=\sh{O}_{P_1}(1)\otimes_Y \sh{O}_{P_2}(1) = q_1^*(\sh{O}_{P_1}(1))\otimes_{\sh{O}_Q}q_2^*(\sh{O}_{P_2}(1))$ is invertible, since it is the tensor product of of two invertible $\sh{O}_Q$-modules \sref[0]{0.5.4.4}.
Also, if $r=p_1\circ q_1=p_2\circ q_2$ is the structure morphism $Q\to Y$, then $r^*(\sh{E}\otimes_{\sh{O}_Y}\sh{F}) = q_1^*(p_1^*(\sh{E}))\otimes_{\sh{O}_Q}q_2^*(p_2^*(\sh{F}))$ \sref[0]{0.4.3.3};
the canonical surjective homomorphisms \sref{II.4.1.5.1} $p_1^*(\sh{E})\to\sh{O}_{P_1}(1)$ and $p_2^*(\sh{F})\to\sh{O}_{P_2}(1)$ thus give, by taking the tensor product, a canonical homomorphism
\[
\label{II.4.3.1.1}
  s: r^*(\sh{E}\otimes_{\sh{O}_Y}\sh{F}) \to \sh{L}
  \tag{4.3.1.1}
\]
which is evidently surjective;
from this we obtain \sref{II.4.2.2} a canonical morphism, called the \emph{Segre morphism}:
\[
\label{II.4.3.1.2}
  \varsigma: \bb{P}(\sh{E})\times_Y\bb{P}(\sh{F}) \to \bb{P}(\sh{E}\otimes_{\sh{O}_Y}\sh{F}).
  \tag{4.3.1.2}
\]

We can study the morphism $\varsigma$ more explicitly in the case where $Y=\Spec(A)$ is affine, and $\sh{E}=\widetilde{E}$ and $\sh{F}=\widetilde{F}$, where $E$ and $F$ are $A$-modules, whence $\sh{E}\otimes_{\sh{O}_Y}\sh{F}=(E\otimes_A F)^\sim$ \sref[I]{I.1.3.12};
\end{env}


% \subsection{Immersions in projective bundles; very ample sheaves}
% \label{subsection:II.4.4}


% \subsection{Ample sheaves}
% \label{subsection:II.4.5}


% \subsection{Relatively ample sheaves}
% \label{subsection:II.4.6}
