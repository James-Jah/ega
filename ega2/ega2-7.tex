\section{Valuative criteria}
\label{section:II.7}

In this section, we give valuative criteria for separation and properness for a given morphism, that is, criteria which introduce a variable auxiliary scheme of the form $\Spec(A)$, where $A$ is a valuation ring.
Under certain suitable ``Noetherian'' hypotheses, we can refine our criteria and restrict to the case where $A$ is a \emph{discrete} valuation ring.
This will be the only case that we need to concern ourselves with in all that follows, and we introduce arbitrary valuation rings, in the general case, only to discuss the links with the classical study of such objects.


\subsection{Reminder on valuation rings}
\label{subsection:II.7.1}

\begin{env}[7.1.1]
\label{II.7.1.1}
Amongst the many diverse equivalent properties that characterise valuation rings, we will use the following: a ring $A$ is said to be a \emph{valuation ring} if it is an integral ring which is not a field, and $A$ is \emph{maximal} in the set of local rings strictly contained in the field of fractions $K$ of $A$ under the domination relation \sref[I]{I.8.1.1}.
Recall that a valuation ring is \emph{integrally closed}.
If $A$ is a valuation ring, then so too is $A_\mathfrak{p}$ for any prime ideal $\mathfrak{p}\neq0$ of $A$.
\end{env}

\begin{env}[7.1.2]
\label{II.7.1.2}
Let $K$ be a field, and $A$ a local subring of $K$ that is not a field;
\oldpage[II]{139}
then there exists a valuation ring that both dominates $A$ and has $K$ as its field of fractions (\cite[p.~1-07, lemma~2]{I-1}).

Now let $B$ be a valuation ring, $k$ its residue field, $K$ its field of fractions, and $L$ an extension of $k$.
Then there exists a \emph{complete} valuation ring $C$ that dominates $B$ and whose residue field is $L$.
Indeed, $L$ is the algebraic extension of a pure transcendental extension $L'=k(T_\mu)_{\mu\in M}$;
we know that we can extend the valuation of $B$ corresponding to $B$ to a valuation of $K'=K(T_\mu)_{\mu\in M}$ in such a way that $L'$ is the residue field of this valuation (\cite[p.~98]{II-24});
replacing $B$ by the completion of the ring of this extended valuation, we see that that we can restrict to the case where $B$ is complete and $L$ is an algebraic closure of $k$.
If $\overline{K}$ is an algebraic closure of $K$, we can then extend the valuation that defines $B$ to $\overline{K}$, and the corresponding residue field is an algebraic closure of $k$, as we can see by lifting to $\overline{K}$ the coefficients of a unitary polynomial of $k[T]$.
We are thus finally led to the case where $L=k$, and it then suffices to take $C$ to be the completion of $B$ in order to satisfy our claim.
\end{env}

\begin{env}[7.1.3]
\label{II.7.1.3}
Let $K$ be a field, and $A$ a subring of $K$;
the integral closure $A'$ of $A$ in $K$ is the intersection of the valuation rings that contain $A$ and have $K$ as their field of fractions (\cite[p.~51, th.~2]{I-11}).
Proposition~\sref{II.7.1.2} can then be expressed geometrically in an equivalent form:
\end{env}

\begin{proposition}[7.1.4]
\label{II.7.1.4}
Let $Y$ be a prescheme, $p:X\to Y$ a morphism, $x$ a point of $X$, $y=p(x)$, and $y'\neq y$ a specialisation \sref[0]{0.2.1.2} of $y$.
Then there exists a local scheme $Y'$ which is the spectrum of some valuation ring, and a separated morphism $f:Y'\to Y$ such that, denoting the unique closed point of $Y'$ by $a$ and the generic point of $Y'$ by $b$, we have $f(a)=y'$ and $f(b)=y$.
We can furthermore suppose that one of the two additional following properties are satisfied:
\begin{enumerate}
    \item[\rm{(i)}] $Y'$ is the spectrum of a complete valuation ring whose residue field is algebraically closed, and there exists a $\kres(y)$-homomorphism $\kres(x)\to\kres(b)$.
    \item[\rm{(ii)}] There exists a $\kres(y)$-isomorphism $\kres(x)\xrightarrow{\sim}\kres(b)$.
\end{enumerate}
\end{proposition}

\begin{proof}
\label{proof-II.7.1.4}
Let $Y_1$ be the reduced closed subprescheme of $Y$ that has $\overline\{y\}$ as its underlying space \sref[I]{I.5.2.1}, and let $X_1$ be the closed subprescheme given by the inverse image $p^{-1}(Y_1)$;
since $y'\in\overline{\{y\}}$ by hypothesis, and since $\kres(x)$ is the same in $X$ and in $X_1$, we can assume that $Y$ is \emph{integral}, with generic point $y$;
$\sh{O}_{y'}$ is then an integral local ring that is not a field, and whose field of fractions is $\sh{O}_y=\kres(y)$, and $\kres(x)$ is then an extension of $\kres(y)$.
To satisfy the conditions $f(a)=y'$ and $f(b)=y$ as well as the additional condition (i) (resp. (ii)), we take $Y'=\Spec(A')$, where $A'$ is a valuation ring that dominates $\sh{O}_{y'}$ (resp. a valuation ring that dominates $\sh{O}_{y'}$ and whose field of fractions is $\kres(x)$);
the existence such an of $A'$ is guaranteed by \sref{II.7.1.2}.
\end{proof}

\begin{env}[7.1.5]
\label{II.7.1.5}
Recall that a local ring $A$ is said to be \emph{of dimension 1} if there exists a prime ideal distinct from the maximal ideal $\mathfrak{m}$, and if every prime ideal of $A$ distinct from $\mathfrak{m}$ is a \emph{minimal} prime ideal;
when $A$ is \emph{integral}, it is equivalent to ask that $\mathfrak{m}$ and $(0)$ be the only prime ideals, with $\mathfrak{m}\neq(0)$;
in other words, $Y=\Spec(A)$ consists of two
\oldpage[II]{140}
points $a$ and $b$: $a$ is the unique \emph{closed} point, we have $\mathfrak{j}_a=\mathfrak{m}$, and $\kres(a)=k$ is the \emph{residue field} $k=A/\mathfrak{m}$;
$b$ is the \emph{generic point} of $Y$, $\mathfrak{j}_b=(0)$, with the set $\{b\}$ being the unique open subset of $Y$ distinct from both $\emp$ and $Y$ (an open subset which is thus \emph{everywhere dense}), and $\kres(b)=K$ is the \emph{field of fractions} of $A$.
\end{env}

\begin{env}[7.1.6]
\label{II.7.1.6}
For a local ring $A$, Noetherian and of dimension 1, we know (\cite[pp.~2-08 and 17-01]{I-1}) that the following conditions are equivalent:
\begin{enumerate}
    \item[\rm{(a)}] $A$ is normal;
    \item[\rm{(b)}] $A$ is regular;
    \item[\rm{(c)}] $A$ is a valuation ring;
\end{enumerate}
furthermore, $A$ is then a \emph{discrete valuation ring}.
Propositions~\sref{II.7.1.2} and \sref{II.7.1.3} then have the following analogues for discrete valuation rings:
\end{env}

\begin{proposition}[7.1.7]
\label{II.7.1.7}
Let $A$ be an integral local Noetherian ring that is not a field, $K$ its field of fractions, and $L$ an extension of finite type of $K$;
then there exists a discrete valuation ring that dominates $A$ and has $L$ as its field of fractions.
\end{proposition}

\begin{proof}
\label{proof-II.7.1.7}
Suppose first of all that $L=K$.
Let $\mathfrak{m}$ be the maximal ideal of $A$, $(x_1,\ldots,x_n)$ a system of non-null generators of $\mathfrak{m}$, and $B$ the subring $A[x_2/x_1,\ldots,x_n/x_1]$ of $K$, which is Noetherian.
It is immediate that the ideal $\mathfrak{m}B$ of $B$ is identical to the principal ideal $x_1B$;
if $\mathfrak{p}$ is a minimal prime ideal of $x_1B$, then $\mathfrak{p}$ is of rank 1 (\cite[t.~I, p.~277]{I-13});
in other words, $B_\mathfrak{p}$ is a local Noetherian ring \emph{of dimension 1};
it is clear that $\mathfrak{p}B_\mathfrak{p}\cap A$ is an ideal of $A$ that contains $\mathfrak{m}$ and that does not contain $1$, and is thus equal to $\mathfrak{m}$, and so $B_\mathfrak{p}$ \emph{dominates} $A$ \sref[I]{I.8.1.1}.
It follows from the Krull-Akizuki Theorem (\cite[p.~293]{II-25}) that the integral closure $C$ of $B_\mathfrak{p}$ is a Noetherian ring (even though $C$ is not necessarily a $B_\mathfrak{p}$-module of finite type);
if $\mathfrak{n}$ is a maximal ideal of $C$, then $C_\mathfrak{n}$ is a normal local Noetherian ring of dimension 1 (\cite[p.~295]{II-25}), and thus a discrete valuation ring that dominates $B_\mathfrak{p}$ and \emph{a fortiori} $A$.

Now, if $L$ is an extension of finite type of $K$, we can, by the above, restrict to the case where $A$ is already a discrete valuation ring.
Let $w$ be a valuation of $K$ associated to $A$;
there exists a discrete valuation $w'$ of $L$ that \emph{extends} $w$: we can restrict, by induction on the number of generators of $L$, to the case where $L=K(\alpha)$, and then the proposition is classical (\cite[p.~106]{II-24}).
\end{proof}

\begin{corollary}[7.1.8]
\label{II.7.1.8}
Let $A$ be a Noetherian integral ring, $K$ its field of fractions, and $L$ an extension of finite type of $K$.
Then the integral closure of $A$ in $L$ is the intersection of the discrete valuation rings that have $L$ as their field of fractions and that contain $A$.
\end{corollary}

\begin{proof}
\label{proof-II.7.1.8}
Indeed, such a discrete valuation ring, being normal, contains \emph{a fortiori} every element of $L$ that is integral over $A$.
It thus suffices to prove that, if $x\in L$ is not integral over $A$, then there exists a discrete valuation ring $C$ that has $L$ as its field of fractions, contains $A$, and does not contain $x$.
The hypothesis on $x$ implies that $x\not\in B=A[1/x]$, or, in other words, that $1/x$ is not invertible in the Noetherian ring $B$.
There is thus a prime ideal $\mathfrak{p}$ of $B$ that contains $1/x$.
The integral local ring $B_\mathfrak{p}$ is Noetherian and contained in $L$, which is an extension  of finite type of the field of fractions of $B_\mathfrak{p}$ (with the latter containing $K$).
By \sref{II.7.1.7}, there thus exists a discrete valuation ring $C$ that dominates $B_\mathfrak{p}$ and has $L$ as its field of fractions;
since $1/x\in\mathfrak{p}B_\mathfrak{p}$ belongs to the maximal ideal of $C$, we have that $x\not\in C$, which concludes the proof.
\end{proof}

The geometric form of \sref{II.7.1.7} is the following:

\oldpage[II]{141}
\begin{proposition}[7.1.9]
\label{II.7.1.9}
Let $Y$ be a locally Noetherian prescheme, $p:X\to Y$ a morphism locally of finite type, $x$ a point of $X$, $y=p(x)$, and $y'\neq y$ a specialisation of $y$.
Then there exists a local scheme $Y'$, spectrum of a discrete valuation ring, a separated morphism $f:Y'\to Y$, and a rational $Y$-map $g$ from $Y'$ to $X$, such that, denoting the closed point of $Y'$ by $a$, and the generic point of $Y'$ by $b$, we have $f(a)=y'$, $f(b)=y$, $g(b)=x$, and such that, in the commutative diagram
\[
\label{II.7.1.9.1}
    \xymatrix{
        & \kres(x) \ar[dl]_{\gamma}
    \\  \kres(b)
        & \kres(y) \ar[u]_{\pi} \ar[l]^{\varphi}
    }
\tag{7.1.9.1}
\]
(where $\pi$, $\varphi$, and $\gamma$ are the homomorphisms corresponding to $p$, $f$, and $g$, respectively) the morphism $\gamma$ is a bijection.
\end{proposition}

\begin{proof}
\label{proof-II.7.1.9}
As in \sref{II.7.1.4}, we can restrict to the case where $Y$ is integral with generic point $y$ (taking \sref[I]{I.6.4.3}[iv] into account), and, since the question is local on $X$ and $Y$, we can assume that $p$ is of finite type;
we are then in the situation of \sref{II.7.1.4}, with the additional property that $\kres(x)$ is an extension \emph{of finite type} of $\kres(y)$ \sref[I]{I.6.4.11} and that $\sh{O}_{y'}$ is Noetherian;
this lets us apply \sref{II.7.1.7} and take $Y'=\Spec(A')$, where $A'$ is a discrete valuation ring that dominates $\sh{O}_{y'}$ and whose field of fractions is $\kres(x)$.
We have thus defined a commutative diagram \sref{II.7.1.9.1} where $\gamma$ is a bijection, with $\pi$ and $\varphi$ corresponding to the morphisms $p$ and $f$.
Furthermore, since $X$ and $Y$ are locally Noetherian \sref[I]{I.6.6.2} and since $Y'$ is integral, there exists exactly one rational $Y$-map $g$ from $Y'$ to $X$ to which corresponds the isomorphism $\gamma$ \sref[I]{I.7.1.15}, which finishes the proof.
\end{proof}

\subsection{Valuative criterion for separatedness}
\label{subsection:II.7.2}

\begin{proposition}[7.2.1]
\label{II.7.2.1}
Let $X$ and $Y$ be preschemes, and $f:X\to Y$ a quasi-compact morphism.
The following two conditions are equivalent:
\begin{enumerate}
    \item[\rm{(a)}] The morphism $f$ is closed.
    \item[\rm{(b)}] For all $x\in X$, and every specialisation $y'\neq y$ of $y=f(x)$, there exists a specialisation $x'$ of $x$ such that $f(x')=x$.
\end{enumerate}
\end{proposition}

\begin{proof}
\label{proof-II.7.2.1}
Condition~(b) implies that $f(\overline{\{x\}})=\overline{\{y\}}$, and is thus a consequence of (a).
To show that (b) implies (a), consider a closed subset $X'$ of the underlying space $X$;
let $Y'=\overline{f(X')}$, and show that $Y'=f(X')$ as follows.
Consider the closed reduced subpreschemes of $X$ and $Y$ whose underlying spaces are $X'$ and $Y'$ (respectively) \sref[I]{I.5.2.1};
there then exists a morphism $f':X'\to Y'$ such that the diagram
\[
    \xymatrix{
        X' \ar[r]^{f'} \ar[d]
        & Y' \ar[d]
    \\  X \ar[r]_{f}
        & Y
    }
\]
commutes \sref[I]{I.5.2.2}, and, since $f$ is quasi-compact, so too is $f'$.
We are thus led to proving that, if $f$ is a quasi-compact and \emph{dominant} morphism, then
\oldpage[II]{142}
condition~(b) implies that $f(X)=Y$.
But let $y'$ be a point of $Y$, and let $y$ be the generic point of an irreducible component of $Y$ that contains $y'$;
by (b), it suffices to show that $f^{-1}(y)$ is not empty.
But we know that this property is a consequence of the fact that $f$ is quasi-compact and dominant \sref[I]{I.6.6.5}.
\end{proof}

\begin{corollary}[7.2.2]
\label{II.7.2.2}
Let $f:X\to Y$ be a quasi-compact immersion.
For the underlying space $X$ to be closed in $Y$, it is necessary and sufficient for it to contain every specialisation (in $Y$) of all of its points.
\end{corollary}

\begin{proposition}[7.2.3]
\label{II.7.2.3}
Let $Y$ be a prescheme (resp. a locally Noetherian prescheme), and $f:X\to Y$ a morphism (resp. a morphism locally of finite type).
The following conditions are equivalent:
\begin{enumerate}
    \item[\rm{(a)}] $f$ is separated.
    \item[\rm{(b)}] The diagonal morphism $X\to X\times_Y X$ is quasi-compact, and, for every $Y$-prescheme of the form $Y'=\Spec(A)$, with $A$ some valuation ring (resp. some discrete valuation ring), any two $Y$-morphisms from $Y'$ to $X$ that agree on the generic point of $Y'$ are equal.
    \item[\rm{(c)}] The diagonal morphism $X\to X\times_Y X$ is quasi-compact, and, for every $Y$-prescheme of the form $Y'=\Spec(A)$, with $A$ some valuation ring (resp. some discrete valuation ring), any two $Y'$-sections of $X'=X_{(Y')}$ that agree on the generic point of $Y'$ are equal.
\end{enumerate}
\end{proposition}

\begin{proof}
\label{proof-II.7.2.3}
The equivalence of (b) and (c) follows from the bijective correspondence between $Y$-morphisms from $Y'$ to $X$ and $Y'$-sections of $X'$ \sref[I]{I.3.3.14}.
If $X$ is separated over $Y$, condition~(b) is satisfied, by \sref[I]{I.7.2.2.1}, since $Y'$ is integral.
It remains to show that (b) implies that the diagonal morphism $\Delta:X\to X\times_Y X$ is closed, and it is equivalent to show that it satisfies the criteria of \sref{II.7.2.2}.
But let $z$ be a point of the diagonal $\Delta(X)$, and $z'\neq z$ a specialisation of $z$ in $X\times_Y X$.
There then exists \sref{II.7.1.4} a valuation ring $A$ and a morphism $f$ from $Y'=\Spec(A)$ to $X\times_Y X$ such that $f$ sends the closed point $a$ of $Y'$ to $z'$, and the generic point $b$ of $Y'$ to $z$;
this morphism makes $Y'$ an $(X\times_Y X)$-prescheme, and \emph{a fortiori} a $Y$-prescheme.
If we compose the two projections of $X\times_Y X$ with $f$, then we obtain two $Y$-morphisms, $g_1$ and $g_2$, from $Y'$ to $X$, which, by hypothesis, agree on the point $b$;
they are thus equal to one single morphism $g$, which implies \sref[I]{I.5.3.1} that $f$ factors as $f=\Delta\circ g$, and thus $z'\in\Delta(X)$.
If we suppose that $Y$ is locally Noetherian and $f$ is locally of finite type, then $X\times_Y X$ is locally Noetherian \sref[I]{I.6.6.7};
we can thus follow the same argument as before by supposing that $A$ is a discrete valuation ring, by \sref{II.7.1.9}.
\end{proof}

\begin{remark}[7.2.4]
\label{II.7.2.4}
\begin{enumerate}
    \item[\rm{(i)}] The hypothesis that the morphism $\Delta$ is quasi-compact is always satisfied whenever $Y$ is locally Noetherian and $f$ is locally of finite type, because $X\times_Y X$ is then locally Noetherian \sref[I]{I.6.6.4}[i].
        In the general case, this also implies that, for every cover $(U_\alpha)$ of $X$ by affine opens, the sets $U_\alpha\cap U_\beta$ are \emph{quasi-compact}.
    \item[\rm{(ii)}] For $f$ to be separated, it is \emph{sufficient} for condition~(b) or (c) to be satisfied for some valuation ring $A$ that is \emph{complete} and whose residue field is \emph{algebraically closed};
        this follows from the proofs of \sref{II.7.2.3} and \sref{II.7.2.4}.
\end{enumerate}
\end{remark}


\oldpage[II]{143}

\subsection{Valuative criterion for properness}
\label{subsection:II.7.3}

\begin{proposition}[7.3.1]
\label{II.7.3.1}
Let $A$ be a valuation ring, $Y=\Spec(A)$, $b$ the generic point of $Y$, $X$ an integral \emph{scheme}, and $f:X\to Y$ a \emph{closed} morphism such that $f^{-1}(b)$ consists of a single point $x$ and such that the corresponding homomorphism $\kres(b)\to\kres(x)$ is bijective.
Then $f$ is an isomorphism.
\end{proposition}

\begin{proof}
\label{proof-II.7.3.1}
Since $f$ if closed and dominant, we have that $f(X)=Y$;
it suffices \sref[I]{I.4.2.2} to prove that, for all $y'\neq b$ in $Y$, there exists \emph{exactly one} point $x'$ such that $f(x')=y'$, and that the corresponding homomorphism $\sh{O}_{y'}\to\sh{O}_{x'}$ is bijective, since then $f$ will be a homeomorphism.
But if $f(x')=y'$ then $\sh{O}_{x'}$ is a local ring contained in $K=\kres(x)=\kres(b)$ and dominating $\sh{O}_{y'}$;
the latter is the local ring $A_{y'}$, and is thus a valuation ring \sref{II.7.1.1} that has $K$ as its field of fractions.
Also, $\sh{O}_{x'}\neq K$, since $x'$ is not the generic point of $X$ \sref[0]{0.2.1.3};
we thus conclude that $\sh{O}_{x'}=\sh{O}_{y'}$.
Since $X$ is an integral scheme, the fact that $\sh{O}_{x'}=\sh{O}_{x''}$ implies that $x'=x''$ \sref[I]{I.8.2.2}, which finishes the proof.
\end{proof}

\begin{env}[7.3.2]
\label{II.7.3.2}
Let $A$ be a valuation ring, $K$ its field of fractions, $Y=\Spec(A)$, and $b$ the generic point of $Y$, such that $\sh{O}_b=\kres(b)$ is equal to $K$;
let $f:X\to Y$ be a morphism.
We know \sref[I]{I.7.1.4} that the \emph{rational $Y$-sections} of $X$ are in bijective correspondence with the \emph{germs} of $Y$-sections (defined in a neighbourhood of $b$) at the point $b$, whence we have a canonical map
\[
\label{II.7.3.2.1}
    \Gamma_\mathrm{rat}(X/Y) \to \Gamma(f^{-1}(b)/\Spec(K))
\tag{7.3.2.1}
\]
with the elements of $\Gamma(f^{-1}(b)/\Spec(K))$ being identified, by definition \sref[I]{I.3.4.5}, with the \emph{points of $f^{-1}(b)=X\otimes_A K$ that are rational over $K$}.
When $f$ is \emph{separated}, it follows from \sref[I]{I.5.4.7} that the map \sref{II.7.3.2.1} is \emph{injective}, since $Y$ is an integral scheme.

Composing \sref{II.7.3.2.1} with the canonical map $\Gamma(X/Y)\to\Gamma_\mathrm{rat}(X/Y)$ \sref[I]{I.7.1.2}, we obtain a canonical map
\[
\label{II.7.3.2.2}
    \Gamma(X/Y) \to \Gamma(f^{-1}(b)/\Spec(K)).
\tag{7.3.2.2}
\]
When $f$ is \emph{separated}, this map is again \emph{injective} \sref[I]{I.5.4.7}.
\end{env}

\begin{proposition}[7.3.3]
\label{II.7.3.3}
Let $A$ be a valuation ring with field of fractions $K$, $Y=\Spec(A)$, $b$ the generic point of $Y$, and $f:X\to Y$ a \emph{separated} and \emph{closed} morphism.
Then the canonical map \sref{II.7.3.2.2} is bijective (which is equivalent to saying that it is \emph{surjective}, and implies that the rational $Y$-sections of $X$ are \emph{everywhere defined}).
\end{proposition}

\begin{proof}
\label{proof-II.7.3.3}
So let $x$ be a point of $f^{-1}(b)$ that is \emph{rational} over $K$.
Since $f$ is separated, so too is the morphism $f^{-1}(b)\to\Spec(K)$ corresponding to $f$ \sref[I]{I.5.5.1}[iv], and, since every section of $f^{-1}(b)$ is a closed immersion \sref[I]{I.5.4.6}, $\{x\}$ is closed \emph{in $f^{-1}(b)$}.
Consider the reduced closed subprescheme $X'$ of $X$ that has the closure $\overline{\{x\}}$ of $\{x\}$ \emph{in $X$} as its underlying space.
It is clear that the restriction of $f$ to $X'$ satisfies the hypotheses of \sref{II.7.3.1}, and is thus an \emph{isomorphism} from $X'$ to $Y$, whose inverse isomorphism is the desired $Y$-section of $X$.
\end{proof}

\begin{env}[7.3.4]
\label{II.7.3.4}
To state the two following results, we use a terminology that will be justified and discussed in chapter~IV: if $F$ is a subset
\oldpage[II]{144}
of a prescheme $Y$, we define the \emph{codimension} of $F$ in $Y$, denoted $\codim_Y F$, to be the lower bound of the integers $\dim(\sh{O}_z)$ over all $z$ in $F$.
\end{env}

\begin{corollary}[7.3.5]
\label{II.7.3.5}
Let $Y$ be a locally Noetherian reduced prescheme, and $N$ the set of points $y\in Y$ where $Y$ is not regular \sref[0]{0.4.1.4};
suppose that $\codim_Y N\geq2$.
Let $f:X\to Y$ be a morphism of finite type, both \emph{separated} and \emph{closed}, and let $g$ be a rational $Y$-section of $X$;
if $Y'$ is the set of points of $Y$ where $g$ is not defined (a set which is \emph{closed} \sref[I]{I.7.2.1}), then $\codim_Y Y'\geq2$.
\end{corollary}

\begin{proof}
\label{proof-II.7.3.5}
It suffices to prove that $g$ is defined at every point $z\in Y$ such that $\dim\sh{O}_z\leq1$.
If $\dim\sh{O}_z=0$, then $z$ is the generic point of an irreducible component of $Y$ \sref[I]{I.1.1.14}, and so belongs to every everywhere-dense open subset of $Y$, and, in particular, to the domain of definition of $g$.
So suppose that $\dim\sh{O}_z=1$; by hypothesis, $\sh{O}_z$ is then a regular Noetherian local ring, and thus \sref{I.7.1.6} a \emph{discrete valuation ring}.
Let $Z=\Spec(\sh{O}_z)$;
since $U=Y\setmin Y'$ is everywhere dense, $U\cap Z$ is nonempty \sref[I]{I.2.4.2};
let $g'$ be the rational map from $Z$ to $X$ induced by $g$ \sref[I]{I.7.2.8};
it suffices to show that $g'$ is a \emph{morphism} \sref[I]{I.7.2.9}.
But $g'$ can be thought of as a rational $Z$-section of the $Z$-prescheme $f^{-1}(Z)=X\times_Y Z$;
it is clear that the morphism $f^{-1}(Z)\to Z$ corresponding to $f$ is closed, and it follows from \sref[I]{I.5.5.1}[i] that it is separated;
we thus conclude from \sref{II.7.3.3} that $g'$ is everywhere defined;
since $Z$ is reduced, and $X$ is separated over $Y$, $g'$ is a morphism \sref[I]{I.7.2.2}.
\end{proof}

\begin{corollary}[7.3.6]
\label{II.7.3.6}
Let $S$ be a locally Noetherian prescheme, and $X$ and $Y$ both $S$-preschemes;
suppose that $Y$ is reduced, and further that the set $N$ of points $y\in Y$ where $Y$ is not regular is such that $\codim_Y N\geq2$;
suppose finally that the structure morphism $X\to S$ is proper.
Let $f$ be a rational $S$-map from $Y$ to $X$, and let $Y'$ be the points of $Y$ where $f$ is not defined;
then $\codim_Y Y'\geq2$.
\end{corollary}

\begin{proof}
\label{proof-II.7.3.6}
We know \sref[I]{I.7.1.2} that we can identify the rational $S$-maps from $Y$ to $X$ with the rational $Y$-sections of $X\times_S Y$;
since the structure morphism $X\times_S Y\to Y$ is closed \sref{II.5.4.1}, we can apply \sref{II.7.3.5}, whence the corollary.
\end{proof}

\begin{remark}[7.3.7]
\label{II.7.3.7}
The hypotheses on $Y$ in \sref{II.7.3.5} and \sref{II.7.3.6} will be satisfied in particular when $Y$ is \emph{normal} \sref[0]{0.4.1.4}, by \sref{II.7.1.6}.
\end{remark}

We can characterise the universally closed morphisms (resp. proper morphisms) by a converse of \sref{II.7.3.3}:
\begin{theorem}[7.3.8]
\label{II.7.3.8}
Let $Y$ be a prescheme (resp. a locally Noetherian prescheme), and $f:X\to Y$ a quasi-compact separated morphism (resp. a morphism of finite type).
The following conditions are equivalent:
\begin{enumerate}
    \item[\rm{(a)}] $f$ is universally closed (resp. proper).
    \item[\rm{(b)}] For every $Y$-scheme of the form $Y'=\Spec(A)$, where $A$ is a valuation ring (resp. a discrete valuation ring) with field of fractions $K$, the canonical map
        \[
            \Hom_Y(Y',X) \to \Hom_Y(\Spec(K),X)
        \]
        corresponding to the canonical injection $A\to K$ is surjective (resp. bijective).
\oldpage[II]{145}
    \item[\rm{(c)}] For every $Y$-scheme of the form $Y'=\Spec(A)$, where $A$ is a valuation ring (resp. a discrete valuation ring), the canonical map \sref{II.7.3.2.2} relative to the $Y'$-prescheme $X_{(Y')}$ is surjective (resp. bijective).
\end{enumerate}
\end{theorem}

\begin{proof}
\label{proof-II.7.3.8}
The equivalence of (b) and (c) follows immediately from \sref[I]{I.3.3.14};
(a) implies (b), since (a) implies, in either case, that $f_(Y')$ is separated \sref[I]{I.5.5.1}[iv] and closed, and it suffices to apply \sref{II.7.3.3}.
It remains to prove that (b) implies (a).
We first consider the case where $Y$ is arbitrary, and $f$ is separated and quasi-compact.
If condition (b) is satisfied by $f$, then it is also satisfied by $f_{(Y'')}:X_{(Y'')}\to Y''$, where $Y''$ is an arbitrary $Y$-prescheme, thanks to the equivalence between (b) and (c), and the fact that $X_{(Y'')}\times_{Y''}Y' = X\times_Y Y'$ for every morphism $Y'\to Y''$ \sref[I]{I.3.3.9.1};
since, further, $f_{(Y'')}$ is separated and quasi-compact whenever $f$ is (\sref[I]{I.5.5.1}[iv] and \sref{II.6.6.4}[iii]), we are led to proving that (b) implies that $f$ is \emph{closed}.
For this, it suffices to verify condition (b) of \sref{II.7.2.1}.
So let $x\in X$, and $y'$ be a specialisation of $y=f(x)$, distinct from $y$;
by \sref{II.7.1.4}, there exists a scheme $Y'$, the spectrum of some valuation ring, and a separated morphism $g:Y'\to Y$ such that, letting $a$ denote the closed point and $b$ the generic point of $Y'$, we have that $g(a)=y'$, $g(b)=y$, and that there exists a $\kres(y)$-homomorphism $\kres(x)\to\kres(b)$.
The latter corresponds canonically to a $Y$-morphism $\Spec(\kres(b))\to X$ \sref[I]{I.2.4.6}, and it thus follows from (b) that there exists a $Y$-morphism $h:Y'\to X$ to which the previous morphism corresponds.
We then have that $h(b)=x$;
if we set $h(a)=x'$, then $x'$ is a specialisation of $x$, and we have that $f(x')=f(h(a))=g(a)=y'$.

If now $Y$ is locally Noetherian and $f$ of finite type, then hypothesis (b) implies, first of all, that $f$ is \emph{separated}, by \sref{II.7.2.3}, with the diagonal morphism $X\to X\times_Y X$ being quasi-compact \sref{II.7.2.4}.
Further, to show that $f$ is proper, it suffices to show that $f_{(Y'')}:X_{(Y'')}\to Y''$ is \emph{closed} for every $Y$-prescheme $Y''$ \emph{of finite type}, taking \sref{II.5.6.3} into account.
Since $Y''$ is then locally Noetherian, we can follow the same reasoning as in the first case by taking $Y'$ to be the spectrum of a discrete valuation ring, and applying \sref{II.7.1.9} instead of \sref{II.7.1.4}.
\end{proof}

\begin{remarks}
\label{II.7.3.9}
\begin{enumerate}
    \item[\rm{(i)}] Whenever $Y$ is an arbitrary prescheme and $f$ a separated morphism, for $f$ to be universally closed, it \emph{suffices} that condition (b) or (c) be satisfied for the \emph{complete} valuation rings $A$ whose residue field is \emph{algebraically closed};
        this follows from the above proof and from \sref{II.7.1.4}.
    \item[\rm{(ii)}] From criterion (c) of \sref{II.7.3.8} we obtain a new proof of the fact that a projective morphism $X\to Y$ is closed \sref{II.5.5.3}, and it is closer to the classical approach.
        We can indeed assume that $Y$ is affine, and thus that $X$ can be identified with a closed subprescheme of a projective bundle $\bb{P}_Y^n$ \sref{II.5.3.3};
        to prove that $X\to Y$ is closed, it suffices to verify that the structure morphism $\bb{P}_Y^n\to Y$ is closed, and criteria (c) of \sref{II.7.3.8}, combined with \sref{II.4.1.3.1}, tells us that we can reduce to proving the following fact:
        \emph{if $Y$ is the spectrum of a valuation ring $A$, with field of fractions $K$, then every point of $\bb{P}_Y^n$ with values in $K$ comes from (by restriction to the generic point of $Y$) a point of $\bb{P}_Y^n$ with values in $A$.}
        But every invertible $\sh{O}_Y$-module is trivial \sref[I]{I.2.4.8};
        so it follows from \sref{II.4.2.6} that a point of $\bb{P}_Y^n$ with values in $K$ can be identified with a class of elements $(\zeta c_0,\zeta c_1,\ldots,\zeta c_n)$ of $K$, where $\zeta\neq0$ and the $c$ are elements of $K$ that are not all zero.
        However, by multiplying the $c_i$ by an element of $A$ of
\oldpage[II]{146}
        suitable valuation, we can suppose that the $c_i$ all belong to $A$, and that at least one of them is invertible.
        But then \sref{II.4.2.6} the system $(c_0,\ldots,c_n)$ also defines a point of $\bb{P}_Y^n$ with values in $A$, which proves our claim.
    \item[\rm{(iii)}] Criteria \sref{II.7.2.3} and \sref{II.7.3.8} are particularly simple when we consider the data of a $Y$-prescheme $X$ as being equivalent to the data of the functor
        \[
            X(Y') = \Hom_Y(Y',X)
        \]
        for $Y$-preschemes $Y'$;
        these criteria allow us, for example, to prove that, under certain conditions, the ``Picard schemes'' are proper.
\end{enumerate}
\end{remarks}

\begin{corollary}
\label{II.7.3.10}
Let $Y$ be an integral scheme (resp. a locally Noetherian integral scheme), $X$ an integral scheme, and $f:X\to Y$ a dominant morphism.
\begin{enumerate}
    \item[\rm{(i)}] If $f$ is quasi-compact and universally closed, then every valuation ring whose field of fractions is the field $R(X)$ of rational functions on $X$, and which is dominated by a local ring $Y$, also dominates by a local ring of $X$.
    \item[\rm{(ii)}] Conversely, suppose that $f$ is of finite type, and that the property described in (i) is verified by every valuation ring (resp. every discrete valuation ring) that has $R(X)$ as its field of fractions.
        Then $f$ is proper.
\end{enumerate}
\end{corollary}

\begin{proof}
\label{proof-II.7.3.10}
Note first of all that the hypotheses imply, in any case, that $f$ is separated \sref[I]{I.5.5.9}.
\begin{enumerate}
    \item[\rm{(i)}] Let $K=R(Y)$, $L=R(X)$, $y$ a point of $Y$, and $A$ a valuation ring that dominates $\sh{O}_y$ and has $L$ as its field of fractions;
        the injection $\sh{O}_y\to A$ then defines a morphism $h$ from $Y'=\Spec(A)$ to $Y$ \sref[I]{I.2.4.4} such that $h(a)=y$, where we write $a$ to denote the closed point of $Y'$;
        furthermore, if $\eta$ is the generic point of $Y$, which is also the generic point of $\Spec(\sh{O}_y)$, then we have $h(b)=\eta$, writing $b$ to denote the generic point of $Y'$ (since $K\subset L$ by hypothesis).
        If $\xi$ is the generic point of $X$, then $\kres(\xi)=\kres(b)=L$ by hypothesis, whence we have a $Y$-morphism $g:\Spec(L)\to X$ such that $g(b)=\xi$;
        by \sref{II.7.3.8}, $g$ comes from a $Y$-morphism $g':Y'\to X$.
        If $x=g'(a)$, it is clear that $A$ dominates $\sh{O}_x$.
    \item[\rm{(ii)}] The question being local on $Y$, we can always suppose that $Y$ is affine (resp. affine and Noetherian).
        Since $f$ is of finite type, we can apply, in either case, Chow's lemma \sref{II.5.6.1}.
        There is thus a projective morphism $p:P\to Y$, an immersion morphism $j:X'\to P$, and a projective morphism $g:X'\to X$ that is both surjective and birational, with $X$ integral, such that the diagram
        \[
            \xymatrix{
                P \ar[d]_p
                & X' \ar[l]_j \ar[d]^g
            \\  Y
                & X \ar[l]^f
            }
        \]
        commutes.
        It suffices to prove that $j$ is a \emph{closed} immersion, since then $f\circ g=p\circ j$ will be a projective morphism, and thus closed, and, since $g$ is surjective, $f$ will also be proper \sref{II.5.4.3}.
        Let $Z$ be the reduced closed subprescheme of $P$ that has $\overline{j(X')}$ as its underlying space \sref[I]{I.5.2.1};
        since $X'$ is integral, $j$ factors as $i\circ h$, where $i:Z\to P$ is the canonical injection, $h:X'\to Z$ a dominant open immersion \sref[I]{I.5.2.3}, and $Z$ is integral;
\oldpage[II]{147}
        furthermore, $Z$ is projective over $Y$, and we see that we can restrict to the case where $P$ is \emph{integral} and $j$ is \emph{dominant} and \emph{birational}, and everything then reduces to showing that $j$ is \emph{surjective}.
        But let $z\in P$;
        then $\sh{O}_z$ is an integral (resp. integral and Noetherian) local ring whose field of fractions is
        \[
            L = R(P) = R(X') = R(X).
        \]
        We can restrict to the case where $z$ is not the generic point of $P$.
        There then exists (\sref{II.7.1.2} and \sref{II.7.1.7}) a valuation ring (resp. a discrete valuation ring) $A$ which dominates $\sh{O}_z$ and has $L$ as its field of fractions.
        \emph{A fortiori}, $A$ dominates $\sh{O}_y$, where $y=p(z)$, and, by hypothesis, there thus exists some $x\in X$ such that $A$ dominates $\sh{O}_x$.
        Since $g$ is proper, the first part of the proof shows that $A$ also dominates $\sh{O}_{x'}$ for some $x'\in X'$;
        it then follows that $\sh{O}_z$ and $\sh{O}_{j(x')}=\sh{O}_{x'}$ are allied \sref[I]{I.8.1.4}, and, since $P$ is a scheme, this implies that $z=j(x')$ \sref[I]{I.8.2.2} and finishes the proof.
\end{enumerate}
\end{proof}

\begin{corollary}
\label{II.7.3.11}
Let $X$ and $Y$ be integral schemes, and $f:X\to Y$ a dominant, quasi-compact, and universally closed morphism.
Suppose further that $Y$ is affine of (integral) ring $B$.
Then $\Gamma(X,\sh{O}_X)$ is canonically isomorphic to a subring of the integral closure of $B$ in $R(X)$.
\end{corollary}

\begin{proof}
\label{proof-II.7.3.11}
Indeed \sref[I]{I.8.2.1.1}, $\Gamma(X,\sh{O}_X)$ can be identified with the intersection of the $\sh{O}_x$ over $x\in X$;
by \sref{II.7.3.10}, \sref{II.7.1.2}, and \sref{II.7.1.3}, $\Gamma(X,\sh{O}_X)$ is then contained in the intersection of the valuation rings that contain $B$ and that have $R(X)$ as their field of fractions;
the conclusion then follows from \sref{II.7.1.3}.
\end{proof}

\begin{remarks}
\label{II.7.3.12}
Under the hypotheses of \sref{II.7.3.11}, and when we suppose that $R(X)$ is an extension of finite type of $R(Y)$, we can, in many cases, conclude that $\Gamma(X,\sh{O}_X)$ is a \emph{module of finite type} over the ring $B=\Gamma(Y,\sh{O}_X)$.
For example, this will be the case whenever $B$ is an \emph{algebra of finite type over a field}, since we then know that the integral closure of $B$ in an extension of finite type of its field of fractions is a $B$-module of finite type (\cite[t.~I, p.~267, th.~9]{I-13});
the conclusion then follows from \sref{II.7.3.11} and the fact that $B$ is Noetherian.

In particular, \emph{a proper affine scheme $X$ over a field $K$ is finite}.
Indeed, by \sref{II.1.6.4}, \sref{II.5.4.6}, and \sref[I]{I.6.4.4}[(c)], we can restrict to the case where $X$ is \emph{reduced}.
Furthermore, it suffices to prove that each of the closed subpreschemes of $X$ that have an irreducible component of $X$ as their underlying space (of which there are finitely many) is finite over $K$, which means (taking \sref{II.5.4.5} into account) that we are finally reduced to the case where $X$ is \emph{integral}.
But then the result follows from the above remarks.

In chapter~III, we will again prove this above proposition by other methods, and as a consequence of more general results, by showing that, if $f:X\to Y$ is proper and $Y$ is locally Noetherian, $f_*(\sh{F})$ is coherent for any coherent $\sh{O}_X$-module $\sh{F}$ \sref[III]{III.4.4.2}.

Finally, note that criterion~\sref{II.7.3.10} is taken as the \emph{definition} of proper morphisms in classical algebraic geometry.
We only mention this here as a remark, since criterion~\sref{II.7.3.8} seems more manageable in all the applications with which we are familiar.
\end{remarks}


\oldpage[II]{148}

\subsection{Algebraic curves and function fields of dimension 1}
\label{subsection:II.7.4}

The aim of this section is to show how to formulate the classical notion of algebraic curves (as introduced, by example, in the book of C.~Chevalley \cite{II-23}) in the language of schemes.
All throughout this section, \emph{we write $k$ to mean a field, all the schemes in question are $k$-schemes of finite type, and all the morphisms are $k$-morphisms}.

\begin{proposition}
\label{II.7.4.1}
Let $X$ be a prescheme of finite type over $k$ (and thus Noetherian);
let $x_i$ ($1\leq i\leq n$) be the generic points of the irreducible components $X_i$ of $X$, and let $K_i=\kres(x_i)$ ($1\leq i\leq n$).
Then the following conditions are equivalent:
\begin{enumerate}
    \item[\rm{(a)}] Each of the $K_i$ is an extension of $k$ with transcendence degree equal to $1$.
    \item[\rm{(b)}] For every closed point $x$ of $X$, the local ring $\sh{O}_x$ is of dimension $1$ \sref{II.7.1.5}.
    \item[\rm{(c)}] The closed irreducible subsets of $X$ that are distinct from the $X_i$ are exactly the closed points of $X$.
\end{enumerate}
\end{proposition}

\begin{proof}
\label{proof-II.7.4.1}
Since $X$ is quasi-compact, every closed irreducible subset $F$ of $X$ contains a closed point \sref[0]{0.2.1.3}.
By \sref[I]{I.2.4.2}, there is a bijective correspondence between the prime ideals of $\sh{O}_x$ and the closed irreducible subsets of $X$ that contain $x$ \sref[I]{I.1.1.14};
the equivalence between (b) and (c) follows immediately from this.
Now, if $\fk{p}_\alpha$ ($1\leq\alpha\leq r$) are the minimal prime ideals of the local Noetherian ring $\sh{O}_x$, then the local rings $\sh{O}_x/\fk{p}_\alpha$ are integral, and have the $K_i$ such that $x\in X_i$ as their fields of fractions.
Furthermore, we know (\cite[p.~4-06, th.~2]{I-1}) that the dimension of a local integral $k$-algebra of finite type is equal to the transcendence degree over $k$ of its field of fractions.
Finally, the dimension of $\sh{O}_x$ is bounded above by the dimensions of the $\sh{O}_x/\fk{p}_\alpha$;
but condition~(a) implies that these dimensions are equal to $1$, and so (a) implies (b);
conversely, if $\sh{O}_x$ is of dimension $1$, then none of the $\fk{p}_\alpha$ can be equal to the maximal ideal of $\sh{O}_x$, otherwise $\sh{O}_x$ would be of dimension $0$;
thus each of the $\sh{O}_x/\fk{p}_\alpha$ are of dimension $1$, which shows that (b) implies (a).
\end{proof}

We note that, under the conditions of \sref{II.7.4.1}, the set $X$ is either \emph{empty} or \emph{infinite}, as an immediate result of \sref[I]{I.6.4.4}.

\begin{definition}
\label{II.7.4.2}
We define an \emph{algebraic curve over $k$} to be a non-empty algebraic \emph{scheme} over $k$ that satisfies the conditions of \sref{II.7.4.1}.
\end{definition}

In the language of dimensions, which will be introduced in Chapter~IV, this can be expressed by saying that an algebraic curve over $k$ is a non-empty algebraic $k$-scheme \emph{whose irreducible components are all of dimension $1$}.

We note that, if $X$ is an algebraic curve over $k$, then the closed reduced subpreschemes $X_i$ ($1\leq i\leq n$) of $X$ that have the irreducible components of $X$ as their underlying space are also algebraic curves over $k$.

\begin{corollary}
\label{II.7.4.3}
Let $X$ be an irreducible algebraic curve.
The only non-closed point of $X$ is its generic point.
The closed subsets of $X$ that are distinct from $X$ are the finite sets of closed points;
these are also the only subsets of $X$ that are not everywhere dense.
\end{corollary}

\begin{proof}
\label{proof-II.7.4.3}
If a point $x\in X$ is not closed, then its closure in $X$ is an irreducible closed subset of $X$, and thus necessarily the whole of $X$, by \sref{II.7.4.1}, and thus $x$ is the generic point of $X$.
A closed subset $F$ of $X$ that is distinct from $X$ cannot contain
\oldpage[II]{149}
the generic point of $X$, and so all its points are closed (in $X$, and \emph{a fortiori} in $F$);
by considering the closed reduced subpreschemes of $X$ that have $F$ as their underlying space \sref[I]{I.5.2.1}, it thus follows from \sref[I]{I.6.2.2} that $F$ is finite and discrete.
The closure in $X$ of any infinite subset of $X$ is thus necessarily equal to $X$ itself.
\end{proof}

If $X$ is an arbitrary algebraic curve, by applying \sref{II.7.4.3} to the irreducible components of $X$, we see that the only non-closed points of $X$ are the generic points of these components.

\begin{corollary}
\label{II.7.4.4}
Let $X$ and $Y$ be irreducible algebraic curves over $k$, and $f:X\to Y$ a $k$-morphism.
For $f$ to be dominant, it is necessary and sufficient for $f^{-1}(y)$ to be finite for all $y\in Y$.
\end{corollary}

\begin{proof}
\label{proof-II.7.4.4}

\end{proof}