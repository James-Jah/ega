\cite{I-1}.

\section{Affine morphisms}
\label{section:2.1}

\subsection{$S$-preschemes and $\mathcal{O}_S$-algebras}
\label{subsection:2.1.1}

\begin{env}[1.1.1]
\label{2.1.1.1}
Let $S$ be a prescheme, $X$ an $S$-prescheme, and $f:X\to S$ its structure morphism.
We know \sref[0]{0.4.2.4} that the direct image $f_*(\sh{O}_X)$ is an $\sh{O}_S$-algebra, which we
\oldpage[II]{6}
denote $\sh{A}(X)$ when there is little chance of confusion; if $U$ is an open subset of $S$, then we have
\[
  \sh{A}(f^{-1}(U))=\sh{A}(X)|U.
\]
Similarly, for every $\sh{O}_X$-module $\sh{F}$ (resp. every $\sh{O}_X$-algebra $\sh{B}$), we write $\sh{A}(\sh{F})$ (resp. $\sh{A}(\sh{B})$) for the direct image $f_*(\sh{F})$ (resp. $f_*(\sh{B})$) which is an $\sh{A}(X)$-module (resp. an $\sh{A}(X)$-algebra) and not only an $\sh{O}_S$-module (resp. an $\sh{O}_S$-algebra).
\end{env}

\begin{env}[1.1.2]
\label{2.1.1.2}
Let $Y$ be a second $S$-prescheme, $g:Y\to S$ its structure morphism, and $h:X\to Y$ an $S$-morphism; we then have the commutative diagram
\[
\label{eq:2.1.1.2.1}
  \xymatrix{
    X\ar[rr]^h\ar[rd]_f & &
    Y\ar[ld]^g\\
    & S.
  }
  \tag{1.1.2.1}
\]

We have by definition $h=(\psi,\theta)$, where $\theta:\sh{O}_Y\to h_*(\sh{O}_X)=\psi_*(\sh{O}_X)$ is a homomorphism of sheaves of rings; we induce \sref[0]{0.4.2.2} a homomorphism of $\sh{O}_S$-algebras $g_*(\theta):g_*(\sh{O}_Y)\to g_*(h_*(\sh{O}_X))=f_*(\sh{O}_X)$, in other words, a homomorphism of $\sh{O}_S$-algebras $\sh{A}(Y)\to\sh{A}(X)$, which we denote by $\sh{A}(h)$.
If $h':Y\to Z$ is a second $S$-morphism, then it is immediate that $\sh{A}(h'\circ h)=\sh{A}(h)\circ\sh{A}(h')$.
We have thus defined a \emph{contravariant functor $\sh{A}(X)$} from the category of $S$-preschemes to the category of $\sh{O}_S$-algebras.

Now let $\sh{F}$ be an $\sh{O}_X$-module, $\sh{G}$ an $\sh{O}_Y$-module, and $u:\sh{G}\to\sh{F}$ an $h$-morphism, that is \sref[0]{0.4.4.1} a homomorphism of $\sh{O}_Y$-modules $\sh{G}\to h_*(\sh{F})$.
Then $g_*(u):g_*(\sh{G})\to g_*(h_*(\sh{F}))=f_*(\sh{F})$ is a homomorphism $\sh{A}(\sh{G})\to\sh{A}(\sh{F})$ of $\sh{O}_S$-modules, which we denote by $\sh{A}(u)$; in addition, the pair $(\sh{A}(h),\sh{A}(u))$ form a \emph{di-homomorphism} from the $\sh{A}(Y)$-module $\sh{A}(\sh{G})$ to the $\sh{A}(X)$-module $\sh{A}(\sh{F})$.
\end{env}

\begin{env}[1.1.3]
\label{2.1.1.3}
If we fix the prescheme $S$, then we can consider the pairs $(X,\sh{F})$, where $X$ is an $S$-prescheme and $\sh{F}$ is an $\sh{O}_X$-module, as forming a \emph{category}, by defining a \emph{morphism} $(X,\sh{F})\to(Y,\sh{G})$ as a pair $(h,u)$, where $h:X\to Y$ is an $S$-morphism and $u:\sh{G}\to\sh{F}$ is an $h$-morphism.
We can theen say that $(\sh{A}(X),\sh{A}(\sh{F}))$ is a \emph{contravariant functor} with values in the category whose objects are pairs consisting of an $\sh{O}_S$-algebra and a module over that algebra, and the morphisms are the di-homomorphisms.
\end{env}

\subsection{Affine preschemes over a prescheme}
\label{subsection:2.1.2}

\begin{definition}[1.2.1]
\label{2.1.2.1}
Let $X$ be an $S$-prescheme, $f:X\to S$ its structure morphism.
We say that $X$ is \emph{affine over $S$} if there exists a cover $(S_\alpha)$ of $S$ by affine open sets such that for all $\alpha$, the induced prescheme on $X$ by the open set $f^{-1}(S_\alpha)$ is affine.
\end{definition}

\begin{example}[1.2.2]
\label{2.1.2.2}
Every closed subprescheme of $S$ is an affine $S$-prescheme over $S$ (\sref[I]{1.4.2.3} and \sref[I]{1.4.2.4}).
\end{example}

\begin{remark}[1.2.3]
\label{2.1.2.3}
An affine prescheme $X$ over $S$ is not necessarily an affine scheme, as the example $X=S$ shows \sref{2.1.2.2}.
On the other hand, if an affine scheme $X$ is an $S$-prescheme, then $X$ is not necessarily affine over
\oldpage[II]{7}
$S$ (see Example~\sref{2.1.3.3}).
However, remember that if $S$ is a \emph{scheme}, then every $S$-prescheme which is an affine scheme is affine over $S$ \sref[I]{1.5.5.10}.
\end{remark}

\begin{proposition}[1.2.4]
\label{2.1.2.4}
Every $S$-prescheme which is affine over $S$ is separated over $S$ (in other words, it is an $S$-scheme).
\end{proposition}

\begin{proof}
\label{proof-2.1.2.4}
This follows immediately from \sref[I]{1.5.5.5} and \sref[I]{1.5.5.8}.
\end{proof}

\begin{proposition}[1.2.5]
\label{2.1.2.5}
Let $X$ be an $S$-scheme affine over $S$, $f:X\to S$ its structure morphism.
For every open $U\subset S$, $f^{-1}(U)$ is affine over $U$.
\end{proposition}

















