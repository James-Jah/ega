\section{Quasi-affine morphisms; quasi-projective morphisms; proper morphisms; projective morphisms}
\label{section-quasi-affine-projective-proper-morphisms}

\begin{defn}[5.1.1]
\label{2.5.1.1}
We define a quasi-affine scheme to be a scheme isomorphic to some subscheme induced on some quasi-compact open subset of an affine scheme.
We say that a morphism $f:X\to Y$ is quasi-affine, or that $X$ (considered as a $Y$-prescheme via $f$) is a quasi-affine $Y$-scheme, if there exists a cover $(U_\alpha)$ of $Y$ by affine open subsets such that the $f^{-1}(U_\alpha)$ are quasi-affine schemes.
\end{defn}

It is clear that a quasi-affine morphism is \emph{separated} (\sref[I]{1.5.5.5} and \sref[I]{1.5.5.8}) and \emph{quasi-compact} \sref[I]{1.6.6.1};
every affine morphisms is evidently quasi-affine.

Recall that, for any prescheme $X$, setting $A=\Gamma(X,\OO_X)$, the identity homomorphism $A\to A=\Gamma(X,\OO_X)$ defines a morphism $X\to\Spec(A)$, said to be \emph{canonical} \sref[I]{1.2.2.4};
this is nothing but the canonical morphism defined in \sref{2.4.5.1} for the specific case where $\sh{L}=\OO_X$, if we remember that $\Proj(A[T])$ is canonically identified with $\Spec(A)$ \sref{2.3.1.7}.

\begin{prop}[5.1.2]
\label{2.5.1.2}
Let $X$ be a quasi-compact scheme or a prescheme whose underlying space is Noetherian, and $A$ the ring $\Gamma(X,\OO_X)$.
The following conditions are equivalent.
\begin{enumerate}[label=\emph{(\alph*)}]
    \item $X$ is a quasi-affine scheme.
    \item The canonical morphism $u:X\to\Spec(A)$ is an open immersion.
    \item[\emph{(b')}] The canonical morphism $u:X\to\Spec(A)$ is a homeomorphism from $X$ to some subspace of the underlying space of $\Spec(A)$.
    \item The $\OO_X$-module $\OO_X$ is very ample relative to $u$ \sref{2.4.4.2}.
    \item[\emph{(c')}] The $\OO_X$-module $\OO_X$ is ample \sref{2.4.5.1}.
    \item When $f$ ranges over $A$, the $X_f$ form a base for the topology of $X$.
    \item[\emph{(d')}] When $f$ ranges over $A$, the $X_f$ that are affine form a cover of $X$.
\oldpage[II]{95}
    \item Every quasi-coherent $\OO_X$-module is generated by its sections over $X$.
    \item[\emph{(e')}] Every quasi-coherent sheaf of ideals of $\OO_X$ of finite type is generated by its sections over $X$.
\end{enumerate}
\end{prop}

\begin{proof}
\label{proof-2.5.1.2}
It is clear that \emph{(b)} implies \emph{(a)}, and \emph{(a)} implies \emph{(c)} by \sref{2.4.4.4}[b] applied to the identity morphism (taking into account the remark preceding this proposition);
Furthermore, \emph{(c)} implies \emph{(c')} \sref{2.4.5.10}[i], and \emph{(c')}, \emph{(b)}, and \emph{(b')} are all equivalent by \sref{2.4.5.2}[b] and \sref{2.4.5.2}[b'].
Finally, \emph{(c')} is the same as each of \emph{(d)}, \emph{(d')}, \emph{(e)}, and \emph{(e')} by \sref{2.4.5.2}[a], \sref{2.4.5.2}[a'], \sref{2.4.5.2}[c], and \sref{2.4.5.5}[d''].
\end{proof}

We further observe that, with the previous notation, the $X_f$ that are affine form a \emph{base} for the topology of $X$, and that the canonical morphism $u$ is \emph{dominant} \sref{2.4.5.2}.

\begin{cor}[5.1.3]
\label{1.5.1.3}
Let $X$ be a quasi-compact prescheme.
If there exists a morphism $v:X\to Y$ from $X$ to some affine scheme $Y$ (which would be a homeomorphism from $X$ to some open subspace of $Y$), then $X$ is quasi-affine.
\end{cor}

\begin{proof}
\label{proof-2.5.1.3}
There exists a family $(g_\alpha)$ of sections of $\OO_Y$ over $Y$ such that the $D(g_\alpha)$ form a base for the topology of $v(X)$;
if $v=(\psi,\theta)$ and we set $f_\alpha=\theta(g_\alpha)$, then we have $X_{f_\alpha}=\psi^{-1}(D(g_\alpha))$ \sref[I]{1.2.2.4.1}, so the $X_{f_\alpha}$ form a base for the topology of $X$, and the criterion \sref{2.5.1.2}[d] is satisfied.
\end{proof}

\begin{cor}[5.1.4]
\label{2.5.1.4}
If $X$ is a quasi-affine scheme, then \emph{every} invertible $\OO_X$-module is very ample (relative to the canonical morphism), and \emph{a fortiori} ample.
\end{cor}

\begin{proof}
\label{proof-2.5.1.4}
Such a module $\sh{L}$ is generated by its sections over $X$ \sref{2.5.1.2}[e], so $\sh{L}\otimes\OO_X=\sh{L}$ is very ample \sref{2.4.4.8}.
\end{proof}

\begin{cor}[5.1.5]
\label{2.5.1.5}
Let $X$ be a quasi-compact prescheme.
If there exists an invertible $\OO_X$-module $\sh{L}$ such that $\sh{L}$ and $\sh{L}^{-1}$ are ample, then $X$ is a quasi-affine scheme.
\end{cor}

\begin{proof}
\label{proof-2.5.1.5}
Indeed, $\OO_X=\sh{L}\otimes\sh{L}^{-1}$ is then ample \sref{2.4.5.7}.
\end{proof}

\begin{prop}[5.1.6]
\label{2.5.1.6}
Let $f:X\to Y$ be a quasi-compact morphism.
Then the following conditions are equivalent.
\begin{enumerate}[label=\emph{(\alph*)}]
    \item The morphism $f$ is quasi-affine.
    \item The $\OO_Y$-algebra $f_*(\OO_X)=\sh{A}(X)$ is quasi-coherent, and the canonical morphism $X\to\Spec(\sh{A}(X))$ corresponding to the identity morphism $\sh{A}(X)\to\sh{A}(X)$ \sref{2.1.2.7} is an open immersion.
    \item[\emph{(b')}] The $\OO_Y$-algebra $\sh{A}(X)$ is quasi-coherent, and the canonical morphism $X\to\Spec(\sh{A}(X))$ is a homeomorphism from $X$ to some subspace of $\Spec(\sh{A}(X))$.
    \item The $\OO_X$-module $\OO_X$ is very ample for $f$.
    \item[\emph{(c')}] The $\OO_X$-module $\OO_X$ is ample for $f$.
    \item The morphism $f$ is separated, and, for every quasi-coherent $\OO_X$-module $\sh{F}$, the canonical homomorphism $\sigma:f^*(f_*(\sh{F}))\to\sh{F}$ \sref[0]{0.4.4.3} is surjective.
\end{enumerate}

Further, whenever $f$ is quasi-affine, every invertible $\OO_X$-module $\sh{L}$ is very ample relative to $f$.
\end{prop}

\begin{proof}
\label{proof-2.5.1.6}
The equivalence between \emph{(a)} and \emph{(c')} follows from the local (on $Y$) character of the $f$-\unsure{amplitude} \sref{2.4.6.4}, Definition~\sref{2.5.1.1}, and the criterion \sref{2.5.1.2}[c'].
The other properties are local on $Y$
\oldpage[II]{96}
and thus follow immediately from \sref{2.5.1.2} and \sref{2.5.1.4}, taking into account the fact that $f_*(\sh{F})$ is quasi-coherent whenever $f$ is separated \sref[I]{1.9.2.2}[a].
\end{proof}
