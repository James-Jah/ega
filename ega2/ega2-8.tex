\section{Blowup schemes; based cones; projective closure}
\label{section:2.8}


\subsection{Blowup preschemes}
\label{subsection:2.8.1}

\begin{env}[8.1.1]
\label{2.8.1.1}
Let $Y$ be a prescheme, and, for every integer $n\geq 0$, let $\sh{I}_n$ be a quasi-coherent sheaf of ideals of $\sh{O}_Y$; suppose that the following conditions are satisfied:
\[
\label{eq:2.8.1.1.1}
  \sh{I}_0=\sh{O}_Y,\ \sh{I}_n\subset\sh{I}_m\text{ for }m\leq n,
\tag{8.1.1.1}
\]
\[
\label{eq:2.8.1.1.2}
  \sh{I}_m\sh{I}_n\subset\sh{I}_{m+n}\text{ for any }m,n.
\tag{8.1.1.2}
\]

\oldpage[II]{153}
We note that these hypotheses imply
\[
\label{eq:2.8.1.1.3}
  \sh{I}_1^n\subset\sh{I}_n.
\tag{8.1.1.3}
\]

Set
\[
\label{eq:2.8.1.1.4}
  \sh{S}=\bigoplus_{n\geq 0}\sh{I}_n.
\tag{8.1.1.4}
\]

It follows from \eref{eq:2.8.1.1.1} and \eref{eq:2.8.1.1.2} that $\sh{S}$ is a quasi-coherent graded $\sh{O}_Y$-algebra, and thus defines a $Y$-scheme $X=\Proj(\sh{S})$.
If $\sh{J}$ is an \emph{invertible} sheaf of ideals of $\sh{O}_Y$, then $\sh{I}_n\otimes_{\sh{O}_Y}\sh{J}^{\otimes n}$ is canonically identified with $\sh{I}_n\sh{J}^n$.
If we then replace the $\sh{I}_n$ by the $\sh{I}_n\sh{J}^n$, and, in doing so, replace $\sh{S}$ by a quasi-coherent $\sh{O}_Y$-algebra $\sh{S}_{(\sh{J})}$, then $X_{(\sh{J})}=\Proj(\sh{S}_{(\sh{J})})$ is canonically isomorphic to $X$ \sref{2.3.1.8}.
\end{env}

\begin{env}[8.1.2]
\label{2.8.1.2}
Suppose that $Y$ is \emph{locally integral}, so that the sheaf $\sh{R}(Y)$ of rational functions is a quasi-coherent $\sh{O}_Y$-algebra \sref[1]{1.7.3.7}.
We say that a sub-$\sh{O}_Y$-module $\sh{I}$ of $\sh{R}(Y)$ is a \emph{fractional ideal} of $\sh{R}(Y)$ if it is of \emph{finite type} \sref[0]{0.5.2.1}.
Suppose we have, for all $n\geq0$, a quasi-coherent fractional ideal $\sh{I}_n$ of $\sh{R}(Y)$, such that $\sh{I}_0 = \sh{O}_Y$, and such that condition \eref{eq:2.8.1.1.2} (but not necessarily the second condition \eref{eq:2.8.1.1.1}) is satisfied;
we can then again define a graded quasi-coherent $\sh{O}_Y$-algebra by Equation \eref{eq:2.8.1.1.4}, and the corresponding $Y$-scheme $X = \Proj(\sh{S})$;
we will again have a canonical isomorphism from $X$ to $X_{{\sh{J}}}$ for every \emph{invertible} fractional ideal $\sh{J}$ of $\sh{R}(Y)$.
\end{env}

\begin{definition}[8.1.3]
\label{2.8.1.3}
Let $Y$ be a prescheme (resp. a locally integral prescheme), and $\sh{I}$ a quasi-coherent ideal of $\sh{O}_Y$ (resp. a quasi-coherent fractional ideal of $\sh{R}(Y)$).
We say that the $Y$-scheme $X = \Proj(\bigoplus_{n\geq0}\sh{I}^n)$ is obtained by blowing up the ideal $\sh{I}$, or is the blow-up prescheme of $Y$ relative to $\sh{I}$.
When $\sh{I}$ is a quasi-coherent ideal of $\sh{O}_Y$, and $Y'$ is the closed subprescheme of $Y$ defined by $\sh{I}$, we also say that $X$ is the $Y$-scheme obtained by blowing up $Y'$.
\end{definition}

By definition, $\sh{S} = \bigoplus_{n\geq0}\sh{I}^n$ is then generated by $\sh{S}_1 = \sh{I}$;
if $\sh{I}$ is an $\sh{O}_Y$-module of \emph{finite type}, then $X$ is \emph{projective} over $Y$ \sref{2.5.5.2}.
Without any hypotheses on $\sh{I}$, the $\sh{O}_X$-module $\sh{O}_X(1)$ is \emph{invertible} \sref{2.3.2.5} and \emph{very ample}, by \sref{2.4.4.3} applied to the structure morphism $X\to Y$.

We note that, if $j:X\to Y$ is the structure morphism, then the restriction of $f$ to $f^{-1}(Y\setminus Y')$ is an \emph{isomorphism} to $Y\setminus Y'$ whenever $\sh{I}$ is an \emph{ideal of $\sh{O}_Y$} and $Y'$ is the closed subprescheme that it defines: indeed, the question being local on $Y$, it suffices to assume that $\sh{I} = \sh{O}_Y$, and our claim then follows from \sref{2.3.1.7}.

If we replace $\sh{I}$ by $\sh{I}^d$ ($d>0$), then the blow-up $Y$-scheme $X$ is replaced by a canonically isomorphic $Y$-scheme $X'$ \sref{2.8.1.1};
similarly, for every \emph{invertible} ideal (resp. \emph{invertible} fractional ideal) $\sh{J}$, the blow-up prescheme $X_{(\sh{J})}$ relative to the ideal $\sh{I}\sh{J}$ is canonically isomorphic to $X$ \sref{2.8.1.1}.

In particular, whenever $\sh{I}$ is an \emph{invertible} ideal (resp. \emph{invertible} fractional ideal), the $Y$-scheme obtained by blowing up $\sh{I}$ is \emph{isomorphic to $Y$} \sref{2.3.1.7}.

\begin{proposition}[8.1.3]
\label{2.8.1.4}
Let $Y$ be an integral prescheme.
\begin{enumerate}
    \item[\rm{(i)}] For every sequence $(\sh{I}_n)$ of quasi-coherent fractional ideals of $\sh{R}(Y)$ that satisfies \eref{eq:2.8.1.1.2}
\oldpage[II]{154}
        and such that $\sh{I}_0 = \sh{O}_Y$, the $Y$-scheme $X=\Proj(\bigoplus_{n\geq0}\sh{I}^n)$ is integral, and the structure morphism $f:X\to Y$ is dominant.
    \item[\rm{(ii)}] Let $\sh{I}$ be a quasi-coherent fractional ideal of $\sh{R}(Y)$, and let $X$ be the $Y$-scheme given by the blow up of $Y$ relative to $\sh{I}$.
        If $\sh{I} \neq 0$, then the structure morphism $f:X\to Y$ is then birational and surjective.
\end{enumerate}
\end{proposition}

\begin{proof}
\label{proof-2.8.1.4}
\begin{enumerate}
    \item[\rm{(i)}] This follows from the fact that $\sh{S} = \bigoplus_{n\geq0}\sh{I}_n$ is an \emph{integral} $\sh{O}_Y$-algebra (\sref{2.3.1.12} and \sref{2.3.1.14}), since, for all $y\in Y$, $\sh{O}_y$ is an integral ring \sref[I]{1.5.1.4}.
    \item[\rm{(ii)}] By (i), $X$ is integral;
        if, furthermore, $x$ and $y$ are the generic points of $X$ and $Y$ (respectively), then we have $f(x) = y$, and it remains to show that $\kres(x)$ is of rank 1 over $\kres(y)$.
        But $x$ is also the generic point of the fibre $f^{-1}(y)$;
        if $\psi$ is the canonical morphism $Z\to Y$, where $Z=\Spec(\kres(y))$, then the prescheme $f^{-1}(y)$ can be identified with $\Proj(\sh{S}')$, where $\sh{S}' = \psi^*(\sh{S})$ \sref{2.3.5.3}.
        But it is clear that $\sh{S}' = \bigoplus_{n\geq0}(\sh{I}_y)^n$, and, since $\sh{I}$ is a quasi-coherent fractional ideal of $\sh{R}(Y)$ that is not zero, $\sh{I}_y \neq 0$ \sref[I]{1.7.3.6}, whence $\sh{I}_y = \kres(y)$;
        then $\Proj(\sh{S}')$ can be identified with $\Spec(\kres(y))$ \sref{2.3.1.7}, whence the conclusion.
\end{enumerate}
\end{proof}

We show a \emph{converse} of \sref{2.8.1.4} in \sref[III]{3.2.3.8}.

\begin{env}[8.1.5]
\label{2.8.1.5}
We return to the setting and notation of \sref{2.8.1.1}.
By definition, the injection homomorphisms $\sh{I}_{n+1}\to\sh{I}_n$ \eref{eq:2.8.1.1.1} define, for every $k\in\bb{Z}$, an injective homomorphism of degree zero of graded $\sh{S}$-modules
\[
\label{eq:2.8.1.5.1}
  u_k: \sh{S}_+(k+1) \to \sh{S}(k);
\tag{8.1.5.1}
\]
since $\sh{S}_+(k+1)$ and $\sh{S}(k+1)$ are canonically \textbf{(TN)}-isomorphic, they give a canonical correspondence between $u_k$ and an injective homomorphism of $\sh{O}_X$-modules \sref{2.3.4.2}:
\[
\label{eq:2.8.1.5.2}
  \widetilde{u}_k: \sh{O}_X(k+1) \to \sh{O}_X(k).
\tag{8.1.5.2}
\]

Recall as well \sref{2.3.2.6} that we have defined canonical homomorphisms
\[
\label{eq:2.8.1.5.3}
  \lambda: \sh{O}_X(h) \otimes_{\sh{O}_X} \sh{O}_X(k) \to \sh{O}_X(h+k)
\tag{8.1.5.3}
\]
and, since the diagram
\[
  \xymatrix{
    \sh{S}(h) \otimes_{\sh{S}} \sh{S}(k) \otimes_{\sh{S}} \sh{S}(l)
      \ar[r]
      \ar[d]
  & \sh{S}(h+k) \otimes_{\sh{S}} \sh{S}(l)
      \ar[d]
  \\\sh{S}(h) \otimes_{\sh{S}} \sh{S}(k+l)
      \ar[r]
  & \sh{S}(h+k+l)
  }
\]
commutes, it follows from the functoriality of the $\lambda$ \sref{2.3.2.6} that the homomorphisms \eref{eq:2.8.1.5.3} define the structure of a \emph{graded quasi-coherent $\sh{O}_X$-algebra} on
\[
\label{eq:2.8.1.5.4}
  \sh{S}_X = \bigoplus_{n\in\bb{Z}}\sh{O}_X(n).
\tag{8.1.5.4}
\]
Furthermore, the diagram
\[
  \xymatrix{
    \sh{S}(h) \otimes_{\sh{S}} \sh{S}(k+1)
      \ar[r]
      \ar[d]_{1\otimes u_k}
  & \sh{S}(h+k+1)
      \ar[d]^{u_{k+h}}
  \\\sh{S}(h) \otimes_{\sh{S}} \sh{S}(k)
      \ar[r]
  & \sh{S}(h+k)
  }
\]
commutes; the functoriality of the $\lambda$ then implies that we have a commutative diagram
\[
  \xymatrix{
    \sh{O}_X(h) \otimes_{\sh{O}_X} \sh{O}_X(k+1)
      \ar[r]^{\lambda}
      \ar[d]_{1\otimes \widetilde{u}_k}
  & \sh{O}_X(h+k+1)
      \ar[d]^{\widetilde{u}_{k+h}}
  \\\sh{O}_X(h) \otimes_{\sh{O}_X} \sh{O}_X(k)
      \ar[r]^{\lambda}
  & \sh{O}_X(h+k)
  }
\]
where the horizontal arrows are the canonical homomorphisms.
We can thus say that the $\widetilde{u}_k$ define an \emph{injective homomorphism} (of degree zero) \emph{of graded $\sh{S}_X$-modules}
\[
\label{eq:2.8.1.5.6}
  \widetilde{u}: \sh{S}_X(1) \to \sh{S}_X.
\tag{8.1.5.6}
\]
\end{env}

\begin{env}[8.1.6]
\label{2.8.1.6}
Keeping the notation from \sref{2.8.1.5}, we now note that, for $n\geq0$, the composite homomorphism $\widetilde{v}_n = \widetilde{u}_{n-1} \circ \widetilde{u}_{n-2} \circ \ldots \circ \widetilde{u}_0$ is an \emph{injective} homomorphism $\sh{O}_X(n) \to \sh{O}_X$;
we denote by $\sh{I}_{n,X}$ its image, which is thus a quasi-coherent ideal of $\sh{O}_X$, \emph{isomorphic} to $\sh{O}_X(n)$.
Furthermore, the diagram
\[
  \xymatrix{
    \sh{O}_X(m) \otimes_{\sh{O}_X} \sh{O}_X(n)
      \ar[r]^{\lambda}
      \ar[d]_{\widetilde{v}_m \otimes \widetilde{v}_n}
  & \sh{O}_X(m+n)
      \ar[d]^{\widetilde{v}_{m+n}}
  \\\sh{O}_X
      \ar[r]_{\id}
  & \sh{O}_X
  }
\]
commutes for $m\geq0$, $n\geq0$.
We thus deduce the following inclusions:
\[
\label{eq:2.8.1.6.1}
  \sh{I}_{0,X} = \sh{O}_X, \quad \sh{I}_{n,X} \subset \sh{I}_{m,X}
  \qquad\mbox{for $0\leq m\leq n$;}
\tag{8.1.6.1}
\]
\[
\label{eq:2.8.1.6.2}
  \sh{I}_{m,X}\sh{I}_{n,X} \subset \sh{I}_{m+n,X}
  \qquad\qquad\mbox{for $m\geq0$, $n\geq0$.}
\tag{8.1.6.2}
\]
\end{env}

\oldpage[II]{156}

\begin{proposition}[8.1.7]
\label{2.8.1.7}
Let $Y$ be a prescheme, $\sh{I}$ a quasi-coherent ideal of $\sh{O}_Y$, and $X = \Proj(\bigoplus_{n\geq0}\sh{I}^n)$ the $Y$-scheme given by blowing up $\sh{I}$.
We then have, for all $n>0$, a canonical isomorphism
\[
\label{eq:2.8.1.7.1}
  \sh{O}_X(n) \xrightarrow{\sim} \sh{I}^n\sh{O}_X = \sh{I}_{n,X}
\tag{8.1.7.1}
\]
(cf. \sref[0]{0.4.3.5}), and thus that $\sh{I}^n\sh{O}_X$ is a very-ample invertible $\sh{O}_X$-module if $n>0$.
\end{proposition}

\begin{proof}
\label{proof-2.8.1.7}
The last claim is immediate, since $\sh{O}_X(1)$ is invertible \sref{2.3.2.5} and very ample for $Y$ by definition (\sref{2.4.4.3} and \sref{2.4.4.9}).
Also by definition, the image of $v_n$ is exactly $\sh{I}^n\sh{S}$, and \eref{eq:2.8.1.7.1} then follows from the exactness of the functor $\widetilde{\sh{M}}$ \sref{2.3.2.4} and from Equation \eref{eq:2.3.2.4.1}.
\end{proof}

\begin{corollary}[8.1.8]
\label{2.8.1.8}
Under the hypotheses of \sref{2.8.1.7}, if $f:X\to Y$ is the structure morphism, and $Y'$ the closed subprescheme of $Y$ defined by $\sh{I}$, then the closed subprescheme $X' = f^{-1}(Y')$ of $X$ is defined by $\sh{I}\sh{O}_X$ (which is canonically isomorphic to $\sh{O}_X(1)$), from which we obtain a canonical short exact sequence
\[
\label{eq:2.8.1.8.1}
  0 \to \sh{O}_X(1) \to \sh{O}_X \to \sh{O}_{X'} \to 0.
\tag{8.1.8.1}
\]
\end{corollary}

\begin{proof}
\label{proof-2.8.1.8}
This follows from \eref{2.8.1.7.1} and from \sref[I]{1.4.4.5}.
\end{proof}

\begin{env}[8.1.9]
\label{2.8.1.9}
Under the hypotheses of \sref{2.8.1.7}, we can be more precise about the structure of the $\sh{I}_{n,X}$.
Note that the homomorphism
\[
  \widetilde{u}_{-1}: \sh{O}_X \to \sh{O}_X(-1)
\]
canonically corresponds to a section $s$ of $\sh{O}_X(-1)$ over $X$, which we call the \emph{canonical section} (relative to $\sh{I}$) \sref[0]{0.5.1.1}.
In Diagram \eref{eq:2.8.1.5.5}, the horizontal arrows are isomorphisms \sref{2.3.2.7}; by replacing $h$ with $k$, and $k$ with $-1$ in this diagram, we obtain that $\widetilde{u}_k = 1_k \otimes \widetilde{u}_{-1}$ (where $1_k$ denotes the identity on $\sh{O}_X(h)$), or, equivalently, that the homomorphism $\widetilde{u}_k$ is given exactly by \emph{tensoring with the canonical section $s$} (for all $k\in\bb{Z}$).
The homomorphism $\widetilde{u}$ \eref{eq:2.8.1.5.6} can then be understood in the same way.

Thus, for all $n\geq0$, the homomorphism $\widetilde{v}_n: \sh{O}_X(n)\to\sh{O}_X$ is given exactly by tensoring with $s^{\otimes n}$;
we thus deduce:
\end{env}

\begin{corollary}[8.1.10]
\label{2.8.1.10}
With the notation of \sref{2.8.1.8}, the underlying space of $X'$ is the set of $x\in X$ such that $s(x)=0$, where $s$ denotes the canonical section of $\sh{O}_X(-1)$.
\end{corollary}

\begin{proof}
\label{proof-2.8.1.10}
Indeed, if $c_x$ is a generator of the fibre $(\sh{O}_X(1))_x$ at a point $x$, then $s_x\otimes c_x$ is canonically identified with a generator of the fibre of $\sh{I}_{1,X}$ at the point $x$, and is thus invertible if and only if $s_x\not\in\fk{m}_x(\sh{O}_x(-1))_x$, or, equivalently, if and only if $s(x)\neq0$.
\end{proof}

\begin{proposition}
\label{2.8.1.11}
Let $Y$ be an integral prescheme, $\sh{I}$ a quasi-coherent fractional ideal of $\sh{R}(Y)$, and $X$ the $Y$-scheme given by blowing up $\sh{I}$.
Then $\sh{I}\sh{O}_X$ is an invertible $\sh{O}_X$-module that is very ample for $Y$.
\end{proposition}

\begin{proof}
\label{proof-2.8.1.11}
The question being local on $Y$ \sref{2.4.4.5}, we can reduce to the case where $Y=\Spec(A)$, with $A$ some integral ring of ring of fractions $K$, and $\sh{I}=\widetilde{\fk{I}}$, with $\fk{I}$ some fractional ideal of $K$;
there then exists an element $a\neq0$ of $A$ such that $a\fk{I}\subset A$.
Let $S = \bigoplus_{n\geq0}\fk{I}^n$;
the map $x\mapsto ax$ is an $A$-isomorphism from $\fk{I}^{n+1} = (S(1))_n$ to $a\fk{I}^{n+1} = a\fk{I}S_n \subset \fk{I}^n = S_n$,
\oldpage[II]{157}
and thus defines a (TN)-isomorphism of degree zero of graded $S$-modules $S_+(1)\to a\fk{I}S$.
On the other hand, $x\mapsto a^{-1}x$ is an isomorphism of degree zero of graded $S$-modules $a\fk{I}S \xrightarrow{\sim} \fk{I}S$.
We thus obtain, by composition \sref{2.3.2.4}, an isomorphism of $\sh{O}_X$-modules $\sh{O}_X(1) \xrightarrow{\sim} \sh{I}\sh{O}_X$, and, since $S$ is generated by $S_1=\fk{I}$, $\sh{O}_X(1)$ is invertible \sref{2.3.2.5} and very ample (\sref{2.4.4.3} and \sref{2.4.4.9}), whence our claim.
\end{proof}


\subsection{Preliminary results on the localisation of graded rings}
\label{subsection:2.8.2}

\begin{env}[8.2.1]
\label{2.8.2.1}
Let $S$ be a graded ring, but not assumed (for the moment) to be only in positive degree.
We define
\[
\label{eq:2.8.2.1.1}
  S^\geq = \bigoplus_{n\geq0} S_n,
  \qquad
  S^\leq = \bigoplus_{n\leq0} S_n
\tag{8.2.1.1}
\]
which are both graded subrings of $S$, in only positive and negative degrees (respectively).
If $f$ is a homogeneous elements of degree $d$ (positive or negative) of $S$, then the ring of fraction $S_f=S'$ is again endowed with the structure of a graded ring, by taking $S'_n$ ($n\in\bb{Z}$) to be the set of the $x/f^k$ for $x\in S_{n+kd}$ ($k\geq0$);
we define $S_{(f)}=S'_0$, and will write $S_f^\geq$ and $S_f^\leq$ for $S^{'\geq}$ and $S^{'\leq}$ (respectively).
If $d>0$, then
\[
\label{eq:2.8.2.1.2}
  (S^\geq)_f = S_f
\tag{8.2.1.2}
\]
since, if $x\in S_{n+kd}$ with $n+kd<0$, then we can write $x/f^k = xf^h/f^{h+k}$, and we also have that $n+(h+k)d>0$ for $h$ sufficiently large and $>0$.
We thus conclude, by definition, that
\[
\label{eq:2.8.2.1.3}
  (S^\geq)_{(f)} = (S_f^\geq)_0 = S_{(f)}.
\tag{8.2.1.3}
\]

If $M$ is a graded $S$-module, then we similarly define
\[
\label{eq:2.8.2.1.4}
  M^\geq = \bigoplus_{n\geq0} M_n,
  \qquad
  M^\leq = \bigoplus_{n\leq0} M_n
\tag{8.2.1.4}
\]
which are (respectively) a graded $S^\geq$-module and a graded $S^\leq$-module, and their intersection is the $S_0$ module $M_0$.
If $f\in S_d$, then we define $M_f$ to be the graded $S_f$-module whose elements of degree $n$ are the $z/f^k$ for $z\in M_{n+kd}$ ($k\geq0$);
we denote by $M_{(f)}$ the set of elements of degree zero of $M_f$, and this is an $S_{(f)}$-module, and we will write $M_f^\geq$ and $M_f^\leq$ to mean $(M_f)^\geq$ and $(M_f)^\leq$ (respectively).
If $d>0$, then we see, as above, that
\[
\label{eq:2.8.2.1.5}
  (M^\geq)_f = M_f
\tag{8.2.1.5}
\]
and
\[
\label{eq:2.8.2.1.6}
  (M^\geq)_{(f)} = (M_f^\geq)_0 = M_{(f)}.
\tag{8.2.1.6}
\]
\end{env}

\begin{env}[8.2.2]
\label{2.8.2.2}
Let $\bb{z}$ be an indeterminate, we we will call the \emph{homogenisation variable}.
If $S$ is a graded ring (in positive or negative degrees), then the polynomial algebra\footnote{This should not be confused with the use of the notation $\widehat{S}$ to denote the completed separation of a ring.}
\[
\label{eq:2.8.2.2.1}
  \widehat{S} = S[\bb{z}]
\tag{8.2.2.1}
\]
\oldpage[II]{158}
is a graded $S$-algebra, where we define the degree of $f\bb{z}^n$ ($n\geq0$), with $f$ homogeneous, as
\[
\label{eq:2.8.2.2.2}
  \deg(f\bb{z}^n) = n+\deg f.
\tag{8.2.2.2}
\]
\end{env}

\begin{lemma}[8.2.3]
\label{2.8.2.3}
\begin{enumerate}
  \item[\rm{(i)}] There are canonical isomorphisms of (non-graded) rings
    \[
    \label{eq:2.8.2.3.1}
      \widehat{S}_{(\bb{z})}
      \xrightarrow{\sim}
      \widehat{S}/(\bb{z}-1)\widehat{S}
      \xrightarrow{\sim}
      S.
    \tag{8.2.3.1}
    \]
  \item[\rm{(ii)}] There is a canonical isomorphism of (non-graded) rings
    \[
    \label{eq:2.8.2.3.2}
      \widehat{S}_{(f)} \xrightarrow{\sim} S_f^\leq
    \tag{8.2.3.2}
    \]
    for all $f\in S_d$ with $d>0$.
\end{enumerate}
\end{lemma}

\begin{proof}
\label{proof-2.8.2.3}
The first of the isomorphisms in \eref{eq:2.8.2.3.1} was defined in \sref{2.2.2.5}, and the second is trivial;
the isomorphism $\widehat{S}_{(\bb{z})} \xrightarrow{\sim} S$ thus defined thus gives a correspondence between $x\bb{z}^n/\bb{z}^{n+k}$ (where $\deg(x) = k$ for $k\geq -n$) and the element $x$.
The homomorphism \eref{eq:2.8.2.3.2} gives a correspondence between $x\bb{z}^n/f^k$ (where $\deg(x) = kd-n$) and the element $x/f^k$ of degree $-n$ in $S_f^\leq$, and it is again clear that this does indeed give an isomorphism.
\end{proof}

\begin{env}[8.2.4]
\label{2.8.2.4}
Let $M$ be a graded $S$-module.
It is clear that the $S$-module
\[
\label{eq:2.8.2.4.1}
  \widehat{M} = M \otimes_S \widehat{S} = M \otimes_S S[\bb{z}]
\tag{8.2.4.1}
\]
is the direct sum of the $S$-modules $M\otimes S\bb{z}^n$, and thus of the abelian groups $M_k\otimes S\bb{z}^n$ ($k\in\bb{Z}$, $n\geq0$);
we define on $\widehat{M}$ the structure of a graded $\widehat{S}$-module by setting
\[
\label{eq:2.8.2.4.2}
  \deg(x\otimes\bb{z}^n) = n+\deg x
\tag{8.2.4.2}
\]
for all homogeneous $x$ in $M$.
We leave it to the reader to prove the analogue of \sref{2.8.2.3}:
\end{env}

\begin{lemma}[8.2.5]
\label{2.8.2.5}
\begin{enumerate}
  \item[\rm{(i)}] There is a canonical di-isomorphism of (non-graded) modules
    \[
    \label{eq:2.8.2.5.1}
      \widehat{M}_{(\bb{z})} \xrightarrow{\sim} M.
    \tag{8.2.5.1}
    \]
  \item[\rm{(ii)}] For all $f\in S_d$ ($d>0$), there is a di-isomorphism of (non-graded) modules
    \[
    \label{eq:2.8.2.5.2}
      \widehat{M}_{(f)} \xrightarrow{\sim} M_f^\leq.
    \tag{8.2.5.2}
    \]
\end{enumerate}
\end{lemma}

\begin{env}[8.2.6]
\label{2.8.2.6}
Let $S$ be a \emph{positively}-graded ring, and consider the decreasing sequence of graded ideals of $S$
\[
\label{eq:2.8.2.6.1}
  S_{[n]} = \bigoplus_{m\geq n} S_m
  \qquad\mbox{($n\geq0$)}
\tag{8.2.6.1}
\]
(so, in particular, we have $S_{[0]}=S$ and $S_{[1]}=S_+$).
Since it is evident that $S_{[m]}S_{[n]} \subset S_{[m+n]}$, we can define a \emph{graded ring} $S^\natural$ by setting
\[
\label{eq:2.8.2.6.2}
  S^\natural = \bigoplus_{n\geq0} S_n^\natural
  \quad
  \text{with}
  \quad
  S_n^\natural = S_{[n]}.
\tag{8.2.6.2}
\]
$S_0^\natural$ is then the ring $S$ considered as a \emph{non-graded} ring, and $S^\natural$ is thus an $S_0^\natural$-algebra.
For every homogeneous element $f\in S_d$ ($d>0$), we denote by $f^\natural$ the element $f$ considered as belonging to $S_{[d]} = S_d^\natural$.
With this notation:
\end{env}

\oldpage[II]{159}
\begin{lemma}[8.2.7]
\label{2.8.2.7}
Let $S$ be a positively-graded ring, and $f$ a homogeneous element of $S_d$ ($d>0$).
There are canonical ring isomorphisms
\[
\label{eq:2.8.2.7.1}
  S_f \xrightarrow{\sim} \bigoplus_{n\in\bb{Z}} S(n)_{(f)}
\tag{8.2.7.1}
\]
\[
\label{eq:2.8.2.7.2}
  (S_f^\geq)_{f/1} \xrightarrow{\sim} S_f
\tag{8.2.7.2}
\]
\[
\label{eq:2.8.2.7.3}
  S_{(f^\natural)}^\natural \xrightarrow{\sim} S_f^\geq
\tag{8.2.7.3}
\]
where the first two are isomorphisms of graded rings.
\end{lemma}

\begin{proof}
\label{proof-2.8.2.7}
It is immediate, by definition, that we have $(S_f)_n = (S(n)_f)_0$, whence the isomorphism in \eref{eq:2.8.2.7.1}, which is exactly the identity.
Next, since $f/1$ is invertible in $S_f$, there is a canonical isomorphism $S_f \xrightarrow{\sim} (S_f^\geq)_{f/1} = (S_f)_{f/1}$, by \eref{eq:2.8.2.1.2} applied to $S_f$;
the inverse isomorphism is, by definition, the isomorphism in \eref{eq:2.8.2.7.2}.
Finally, if $x = \sum_{m\geq n}y_m$ is an element of $S_{[n]}$ with $n=kd$, then the element $x/(f^\natural)^k$ corresponds to the element $\sum_m y_m/f^k$ of $S_f^\geq$, and we can quickly verify that this defines an isomorphism \eref{eq:2.8.2.7.3}.
\end{proof}

\begin{env}[8.2.8]
\label{2.8.2.8}
If $M$ is a graded $S$-module, then we similarly define, for all $n\in\bb{Z}$,
\[
\label{eq:2.8.2.8.1}
  M_{[n]} = \bigoplus_{m\geq n} M_m
\tag{8.2.8.1}
\]
and, since $S_{[m]}M_{[n]} \subset M_{[m+n]}$ ($m\geq0$), we can define a graded $S^\natural$-module $M^\natural$ by setting
\[
\label{eq:2.8.2.8.2}
  M^\natural = \bigoplus_{n\in\bb{Z}}
  \quad
  \text{with}
  \quad
  M_n^\natural = M_{[n]}.
\tag{8.2.8.2}
\]
\end{env}

We leave to the reader the proof of:
\begin{lemma}[8.2.9]
\label{2.8.2.9}
With the notation of \sref{2.8.2.7} and \sref{2.8.2.8}, there are canonical di-isomorphisms of modules
\[
\label{eq:2.8.2.9.1}
  M_f \xrightarrow{\sim} \bigoplus_{n\in\bb{Z}} M(n)_{(f)}
\tag{8.2.9.1}
\]
\[
\label{eq:2.8.2.9.2}
  (M_f^\geq)_{f/1} \xrightarrow{\sim} M_f
\tag{8.2.9.2}
\]
\[
\label{eq:2.8.2.9.3}
  M_{(f^\natural)}^\natural \xrightarrow{\sim} M_f^\geq
\tag{8.2.9.3}
\]
where the first two are di-isomorphisms of graded modules.
\end{lemma}

\begin{lemma}[8.2.10]
\label{2.8.2.10}
Let $S$ be a positively-graded ring.
\begin{enumerate}
  \item[\rm{(i)}] For $S^\natural$ to be an $S_0^\natural$-algebra of finite type (resp. a Noetherian $S_0^\natural$-algebra), it is necessary and sufficient for $S$ to be an $S_0^\natural$-algebra of finite type (resp. a Noetherian $S_0^\natural$-algebra).
  \item[\rm{(ii)}] For $S_{n+1}^\natural = S_1^\natural S_n^\natural$ ($n\geq n_0$), it is necessary and sufficient for $S_{n+1}=S_1S_n$ ($n\geq n_0$).
  \item[\rm{(iii)}] For $S_n^\natural = S_1^\natural$ ($n\geq n_0$), it is necessary and sufficient for $S_n=S_1^n$ ($n\geq n_0$).
  \item[\rm{(iv)}] If $(f_\alpha)$ is a set of homogeneous elements of $S_+$ such that $S_+$ is the radical in $S_+$ of the ideal of $S_+$ generated by the $f_\alpha$, then $S_+^\natural$ is the radical in $S_+^\natural$ of the ideal of $S_+^\natural$ generated by the $f_\alpha^\natural$.
\end{enumerate}
\end{lemma}

\begin{proof}
\label{proof-2.8.2.10}
\begin{enumerate}
  \item[\rm{(i)}] If $S^\natural$ is an $S_0^\natural$-algebra of finite type, then $S_+=S_1^\natural$ is a module of finite type over $S=S_0^\natural$, by \sref{2.2.1.6}[i], and so $S$ is an $S_0$-algebra of finite type \sref{2.2.1.4};
    if $S^\natural$ is a Noetherian ring, then so too is $S_0^\natural=S$ \sref{2.2.1.5}.
    Conversely, if $S$ is an $S_0$-algebra
\oldpage[II]{160}
    of finite type, then we know \sref{2.2.1.6}[ii] that there exist $h>0$ and $m_0>0$ such that $S_{n+h}=S_hS_n$ for $n\geq m_0$;
    we can clearly assume that $m_0\geq h$.
    Furthermore, the $S_m$ are $S_0$-modules of finite type \sref{2.2.1.6}[i].
    So, if $n\geq m_0+h$, then $S_n^\natural = S_hS_{n-h}^\natural = S_h^\natural S_{n-h}^\natural$;
    and if $m<m_0+h$ then, letting $E = S_{m_0}+\ldots+S_{m_0+h-1}$, we have that
    \[
      S_m^\natural = S_m + \ldots + S_{m_0+h-1} + S_hE + S_h^2E + \ldots.
    \]
    For $1\leq m\leq m_0$, let $G_m$ be the union of the finite systems of generators of the $S_0$-modules $S_i$ for $m\leq i\leq m_0+h-1$, thought of as a subset of $S_{[m]}$.
    For $m_0+1\leq m\leq m_0+h-1$, let $G_m$ be the union of the finite system of generators of the $S_0$-modules $S_i$ for $m\leq i\leq m_0+h-1$ and of $S_hE$, thought of as a subset of $S_{[m]}$.
    It is clear that $S_m^\natural=S_0^\natural G_m$ for $1\leq m\leq m_0+h-1$, and thus the union $G$ of the $G_m$ for $1\leq m\leq m_0+h-1$ is a system of generators of the $S_0^\natural$-algebra $S^\natural$.
    We thus conclude that, if $S=S_0^\natural$ is a Noetherian ring, then so too is $S^\natural$.
  \item[\rm{(ii)}] It is clear that, if $S_{n+1}=S_1S_n$ for $n\geq n_0$, then $S_{n+1}^\natural=S_1S_n^\natural$, and \emph{a fortiori} $S_{n+1}^\natural=S_1^\natural S_n^\natural$ for $n\geq n_0$.
    Conversely, this last equality can be written as
    \[
      S_{n+1} + S_{n+2} + \ldots
      =
      (S_1 + S_2 + \ldots)(S_n + S_{n+1} + \ldots)
    \]
    and comparing terms of degree $n+1$ (in $S$) on both sides gives that $S_{n+1}=S_1S_n$.
  \item[\rm{(iii)}] If $S_n=S_1^n$ for $n\geq n_0$, then $S_n^\natural=S_1^n+S_1^{n+1}+\ldots$;
    since $S_1^\natural$ contains $S_1+S_1^2+\ldots$, we have that $S_n^\natural\subset S_1^{\natural n}$, and thus $S_n^\natural=S_1^{\natural n}$ for $n\geq n_0$.
    Conversely, the only terms of $S_1^{\natural n}=(S_1+S_2+\ldots)^n$ that are of degree $n$ in $S$ are those of $S_1^n$;
    the equality $S_n^\natural=S_1^{\natural n}$ thus implies that $S_n=S_1^n$.
  \item[\rm{(iv)}] It suffices to show that, if an element $g\in S_{k+h}$ is considered as an element of $S_k^\natural$ ($k>0$, $h\geq0$), then there exists an integer $n>0$ such that $g^n$ is a linear combination (in $S_{kn}^\natural$) of the $f_\alpha^\natural$ with coefficients in $S^\natural$.
    By hypothesis, there exists an integer $m_0$ such that, for $m\geq m_0$, we have, \emph{in $S$}, that $g^m = \sum_\alpha c_{\alpha m}f_\alpha$, where the indices $\alpha$ here are \emph{independent of $m$};
    furthermore, we can clearly assume that the $c_{\alpha m}$ are homogeneous, with
    \[
      \deg(c_{\alpha m}) = m(k+h)-\deg f_\alpha
    \]
    in $S$.
    So take $m_0$ sufficiently large enough to ensure that $km_0>\deg f_\alpha$ for all the $f_\alpha$ that appear in $g^{m_0}$;
    for all $\alpha$, let $c'_{\alpha m}$ be the element $c_{\alpha m}$ considered as having degree $km-\deg f_\alpha$ \emph{in $S^\natural$};
    we then have, in $S^\natural$, that $g^m = \sum_\alpha c'_{\alpha m}f_\alpha^\natural$, which finishes the proof.
\end{enumerate}
\end{proof}

\begin{env}[8.2.11]
\label{2.8.2.11}
Consider the graded $S_0$-algebra
\[
\label{eq:2.8.2.11.1}
  S^\natural \otimes_S S_0
  =
  S^\natural/S_+S^\natural
  =
  \bigoplus_{n\geq0} S_{[n]}/S_+S_{[n]}.
\tag{8.2.11.1}
\]

Since $S_n$ is a quotient $S_0$-module of $S_{[n]}/S_+S_{[n]}$, there is a canonical homomorphism of graded $S_0$-algebras
\[
\label{eq:2.8.2.11.2}
  S^\natural \otimes_S S_0 \to S
\tag{8.2.11.2}
\]
which is clearly \emph{surjective}, and thus corresponds \sref{2.2.9.2} to a canonical \emph{closed immersion}
\[
\label{eq:2.8.2.11.3}
  \Proj(S) \to \Proj(S^\natural \otimes_S S_0).
\tag{8.2.11.3}
\]
\end{env}

\oldpage[II]{161}
\begin{proposition}[8.2.12]
\label{2.8.2.12}
The canonical morphism \eref{eq:2.8.2.11.3} is bijective.
For the homomorphism \eref{eq:2.8.2.11.2} to be (TN)-bijective, it is necessary and sufficient for there to exist some $n_0$ such that $S_{n+1}=S_1S_n$ for $n\geq n_0$.
If this latter condition is satisfied, then \eref{eq:2.8.2.11.3} is an isomorphism;
the converse is true whenever $S$ is Noetherian.
\end{proposition}

\begin{proof}
\label{proof:2.8.2.12}
To prove the first claim, it suffices \sref{2.2.8.3} to show that the kernel $\fk{I}$ of the homomorphism \eref{eq:2.8.2.11.2} consists of \emph{nilpotent} elements.
But if $f\in S_{[n]}$ is an element whose class modulo $S_+S_{[n]}$ belongs to this kernel, then this implies that $f\in S_{[n+1]}$;
then $f^{n+1}$, considered as an element of $S_{[n(n+1)]}$, is also an element of $S_+S_{[n(n+1)]}$, since it can be written as $f\cdot f^n$;
so the class of $f^{n+1}$ modulo $S_+S_{[n(n+1)]}$ is zero, which proves our claim.
Since the hypothesis that $S_{n+1}=S_1S_n$ for $n\geq n_0$ is equivalent to $S_{n+1}^\natural=S_1^\natural S_n^\natural$ for $n\geq n_0$ \sref{2.8.2.10}[ii], this hypothesis is equivalent, by definition, to the fact that \eref{eq:2.8.2.11.2} is (TN)-injective, and thus (TN)-bijective, and so \eref{eq:2.8.2.11.3} is an isomorphism, by \sref{2.2.9.1}.
Conversely, if \eref{eq:2.8.2.11.3} is an isomorphism, then the sheaf $\widetilde{\fk{I}}$ on $\Proj(S^\natural\otimes_S S_0)$ is zero \sref{2.2.9.2}[i];
since $S^\natural\otimes_S S_0$ is Noetherian, as a quotient of $S^\natural$ \sref{2.8.2.10}[i], we conclude from \sref{2.2.7.3} that $\fk{I}$ satisfies condition (TN), and so $S_{n+1}^\natural=S_1^\natural S_n^\natural$ for $n\geq n_0$, and this finishes the proof, by \sref{2.8.2.10}[ii].
\end{proof}

\begin{env}[8.2.13]
\label{2.8.2.13}
Consider now the canonical injections $(S_+)^n\to S_{[n]}$, which define an injective homomorphism of degree zero of graded rings
\[
\label{eq:2.8.2.13.1}
  \bigoplus_{n\geq0} (S_+)^n \to S^\natural.
\tag{8.2.13.1}
\]
\end{env}

\begin{proposition}[8.2.14]
\label{2.8.2.14}
For the homomorphism \eref{eq:2.8.2.13.1} to be a (TN)-isomorphism, it is necessary and sufficient for there to exist some $n_0$ such that $S_n=S_1^n$ for all $n\geq n_0$.
Whenever this is the case, the morphisms corresponding to \eref{eq:2.8.2.13.1} is everywhere defined and also an isomorphism
\[
  \Proj(S^\natural) \xrightarrow{\sim} \Proj(\bigoplus_{n\geq0}(S_+)^n);
\]
the converse is true whenever $S$ is Noetherian.
\end{proposition}

\begin{proof}
\label{proof:2.8.2.14}
The first two claims are evident, given \sref{2.8.2.10}[iii] and \sref{2.2.9.1}.
The third will follow from \sref{2.8.2.10}[i and iii] and the following lemma:
\begin{lemma}[8.2.14.1]
\label{2.8.2.14.1}
Let $T$ be a positively-graded ring that is also a $T_0$-algebra of finite type.
If the morphism corresponding to the injective homomorphism $\bigoplus_{n\geq0}T_1^n\to T$ is everywhere defined and also an isomorphism $\Proj(T)\to\Proj(\bigoplus_{n\geq0}T_1^n)$, then there exists some $n_0$ such that $T_n=T_1^n$ for $n\geq n_0$.
\end{lemma}

Let $g_i$ ($1\leq i\leq r$) be generators of the $T_0$-module $T_1$.
The hypothesis implies first of all that the $D_+(g_i)$ cover $\Proj(T)$ \sref{2.2.8.1}.
Let $(h_j)_{1\leq j\leq s}$ be a system of homogeneous elements of $T_+$, with $\deg(h_j)=n_j$, that form, with the $g_i$, a system of generators of the ideal $T_+$, or, equivalently \sref{2.2.1.3}, a system of generators of $T$ as a $T_0$-algebra;
if we set $T'=\bigoplus_{n\geq0}T_1^n$, then the element $h_j/g_i^{n_j}$ of the ring $T_{(g_i)}$ must, by hypothesis, belong to the subring $T'_{(g_i)}$, and so there exists some integer $k$ such that $T_1^k h_j\subset T_1^{k+n_j}$ for all $j$.
We thus conclude, by induction on $r$, that $T_1^k h_j^r \subset T'$ for all $r\geq1$, and, by definition of the $h_j$, we thus have that $T_1^k T\subset T'$.
Also, there exists, for all $j$, an integer $m_j$ such that $h_j^{m_j}$ belongs to the ideal of $T$ generated by the $g_i$ \sref{2.2.3.14}, so $h_j^{m_j}\in T_1 T$, and
\oldpage[II]{162}
$h_j^{m_j k}\in T_1^k T\subset T'$.
There is thus an integer $m_0\geq k$ such that $h_j^m\in T_1^{mn}$ for $m\geq m_0$.
So, if $q$ is the largest of the integers $n_j$, then $n_0=qsm_0+k$ is the required number.
Indeed, an element of $S_n$, for $n\geq n_0$, is the sum of monomials belonging to $T_1^\alpha u$, where $u$ is a product of powers of the $h_j$;
if $\alpha\geq k$, then it follows from the above that $T_1^\alpha u\subset T_1^n$;
in the other case, one of the exponents of the $h_j$ is $\geq m_0$, so $u\in T_1^\beta v$, where $\beta\geq k$ and $v$ is again a product of powers of the $h_j$;
we can then reduce to the previous case, and so we conclude that $T_1^\alpha u\subset T_1^n$ in all cases.
\end{proof}

\begin{remark}[8.2.15]
\label{2.8.2.15}
The condition $S_n=S_1^n$ for $n\geq n_0$ clearly implies that $S_{n+1}=S_1S_n$ for $n\geq n_0$, but the converse is not necessarily true, even if we assume that $S$ is Noetherian.
For example, let $K$ be a field, $A=K[\bb{x}]$, and $B=K[\bb{y}]/\bb{y}^2K[\bb{y}]$, where $\bb{x}$ and $\bb{y}$ are indeterminates, with $\bb{x}$ taken to have degree 1 and $\bb{y}$ to have degree 2, and let $S=A\otimes_K B$, so that $S$ is a graded algebra over $K$ that has a basis given by the elements $1$, $\bb{x}^n$ ($n\geq1$), and $\bb{x}^n\bb{y}$ ($n\geq0$).
It is immediate that $S_{n+1}=S_1S_n$ for $n\geq2$, but $S_1^n=K\bb{x}^n$ while $S_n=K\bb{x}^n+K\bb{x}^n\bb{y}$ for $n\geq2$.
\end{remark}


\subsection{Based cones}
\label{subsection:2.8.3}

\begin{env}[8.3.1]
\label{2.8.3.1}
Let $Y$ be a prescheme;
in all of this section, we will consider only \emph{$Y$-preschemes} and \emph{$Y$-morphisms}.
Let $\sh{S}$ be a quasi-coherent \emph{positively}-graded $\sh{O}_Y$-algebra;
\emph{we further assume that $\sh{S}_0=\sh{O}_Y$}.
Following the notation introduced in \sref{2.8.2.2}, we let
\[
\label{eq:2.8.3.1.1}
  \widehat{\sh{S}} = \sh{S}[\bb{z}] = \sh{S} \otimes_{\sh{O}_Y} \sh{O}_Y[\bb{z}]
\tag{8.3.1.1}
\]
which we consider as a positively-graded $\sh{O}_Y$-algebra by defining the degrees as in \eref{eq:2.8.2.2.2}, so that, for every affine open subset $U$ of $Y$, we have
\[
  \Gamma(U,\widehat{\sh{S}}) = (\Gamma(U,\sh{S}))[\bb{z}].
\]
In what follows, we write
\[
\label{eq:2.8.3.1.2}
  X = \Proj(\sh{S}),
  \quad
  C = \Spec(\sh{S}),
  \quad
  \widehat{C} = \Proj(\widehat{\sh{S}})
\tag{8.3.1.2}
\]
(where, in the definition of $C$, we consider $\sh{S}$ as a non-graded $\sh{O}_Y$-algebra), and we say that $C$ (resp. $\widehat{C}$) is the \emph{affine cone} (resp. \emph{projective cone}) defined by $\sh{S}$;
we will sometimes say ``cone'' instead of ``affine cone''.
By an abuse of language, we also say that $C$ (resp. $\widehat{C}$) is the \emph{affine cone based at $X$} (resp. the \emph{projective cone based at $X$})\footnote{\emph{[Trans.] A more literal translation of the French (\emph{cône projetant affine/projectif}) would be the \emph{projecting affine/projective cone}, but it seems that this terminology already exists to mean something else.}}, with the implicit understanding that the prescheme $X$ is given in the form $\Proj(\sh{S})$;
finally, we say that $\widehat{C}$ is the \emph{projective closure} of $C$ (with the data of $\sh{S}$ being implicit in the structure of $C$).
\end{env}

\begin{proposition}[8.3.2]
\label{2.8.3.2}
There exist canonical $Y$-morphisms
\[
\label{eq:2.8.3.2.1}
  Y \xrightarrow{\varepsilon} C \xrightarrow{i} \widehat{C}
\tag{8.3.2.1}
\]
\[
\label{eq:2.8.3.2.2}
  X \xrightarrow{j} \widehat{C}
\tag{8.3.2.2}
\]
such that $\varepsilon$ and $j$ are closed immersions, and $i$ is an affine morphism, which is a dominant open immersion, for which
\[
\label{eq:2.8.3.2.3}
  i(C) = \widehat{C}\setminus j(X);
\tag{8.3.2.3}
\]
furthermore, $\widehat{C}$ is the smallest closed subprescheme of $\widehat{C}$ containing $i(C)$.
\end{proposition}

\begin{proof}
\label{proof-2.8.3.2}
To define $i$, consider the open subset of $\widehat{C}$ given by
\[
\label{eq:2.8.3.2.4}
  \widehat{C}_{\bb{z}} = \Spec(\widehat{\sh{S}}/(\bb{z}-1)\widehat{\sh{S}})
\tag{8.3.2.4}
\]
\sref{2.3.1.4}, where $\bb{z}$ is canonically identified with a section of $\sh{S}$ over $Y$.
The isomorphism $i: C\xrightarrow{\sim}\widehat{C}_{\bb{z}}$ then corresponds to the canonical isomorphism \eref{eq:2.8.2.3.1}
\[
  \widehat{\sh{S}}/(\bb{z}-1)\widehat{\sh{S}} \xrightarrow{\sim} \sh{S}.
\]

The morphism $\varepsilon$ corresponds to the augmentation homomorphism $\sh{S}\to\sh{S}_0=\sh{O}_Y$, which has kernel $\sh{S}_+$ \sref{2.1.2.7}, and, since the latter is surjective, $\varepsilon$ is a closed immersion \sref{2.1.4.10}.
Finally, $j$ corresponds \sref{2.3.5.1} to the surjective homomorphism of degree zero $\widehat{\sh{S}}\to\sh{S}$, which restricts to the identity on $\sh{S}$ and is zero on $\bb{z}\widehat{\sh{S}}$, which is its kernel;
$j$ is everywhere defined, and is a closed immersion, by \sref{2.3.6.2}.

To prove the other claims of \sref{2.8.3.2}, we can clearly restrict to the case where $Y=\Spec(A)$ is affine, and $\sh{S}=\widetilde{S}$, with $S$ a graded $A$-algebra, whence $\widehat{\sh{S}}=(\widehat{S})^\sim$;
the homogeneous elements $f$ of $S_+$ can then be identified with sections of $\widehat{\sh{S}}$ over $Y$, and the open subset of $\widehat{C}$, denoted $D_+(f)$ in \sref{2.2.3.3}, can then be written as $\widehat{C}_f$ \sref{2.3.1.4};
similarly, the open subset of $C$ denoted $D(f)$ in \sref[I]{1.1.1.1} can be written as $C_f$ \sref[0]{0.5.5.2}.
With this in mind, it follows from \sref{2.2.3.14} and from the definition of $\widehat{S}$ that, in this case, the open subsets $\widehat{C}_{\bb{z}}=i(C)$ and $\widehat{C}_f$ (with $f$ homogeneous in $S_+$) form a \emph{cover} of $\widehat{C}$.
Furthermore, with this notation,
\[
\label{eq:2.8.3.2.5}
  i^{-1}(\widehat{C}_f) = C_f;
\tag{8.3.2.5}
\]
indeed, $\widehat{C}_f\cap i(C) = \widehat{C}_f\cap\widehat{C}_{\bb{z}} = \widehat{C}_{f\bb{z}} = \Spec(\widehat{S}_{(f\bb{z})})$.
But, if $d=\deg(f)$, then $\widehat{S}_{(f\bb{z})}$ is canonically isomorphic to $(\widehat{S}_{(\bb{z})})_{f/\bb{z}^d}$ \sref{2.2.2.2}, and it follows from the definition of the isomorphism in \eref{eq:2.8.2.3.1} that the image of $(\widehat{S}_{(\bb{z})})_{f/\bb{z}^d}$ under the corresponding isomorphism of rings of fractions is exactly $S_f$.
Since $C_f=\Spec(S_f)$, this proves \eref{2.8.3.2.5} and shows, at the same time, that the morphism $i$ is affine;
furthermore, the restriction of $i$ to $C_f$, thought of as a morphism to $\widehat{C}_f$, corresponds \sref[I]{1.1.7.3} to the canonical homomorphism $\widehat{S}_{(f)}\to\widehat{S}_{(f\bb{z})}$, and, by the above and \eref{eq:2.8.2.3.2}, we can claim the following result:
\begin{env}[8.3.2.6]
\label{2.8.3.2.6}
If $Y=\Spec(A)$ is affine, and $\sh{S}=\widetilde{S}$, then, for every homogeneous $f$ in $S_+$, $\widehat{C}_f$ is canonically identified with $\Spec(S_f^\leq)$, and the morphism $C_f\to\widehat{C}_f$ given by restricting $i$ then corresponds to the canonical injection $S_f^\leq\to S_f$.
\end{env}

Now note that (for $Y$ affine) the complement of $\widehat{C}_{\bb{z}}$ in $\widehat{C}=\Proj(\widehat{S})$
\oldpage[II]{164}
is, by definition, the set of graded prime ideals of $\widehat{S}$ containing $\bb{z}$, which is exactly $j(X)$, by definition of $j$, which proves \eref{eq:2.8.3.2.3}.

Finally, to prove the last claim of \sref{2.8.3.2}, we can assume that $Y$ is affine.
With the above notation, note that, in the ring $\widehat{S}$, $\bb{z}$ is not a zero divisor;
since $i(C)=\widehat{C}$, it suffices to prove the following lemma:
\begin{lemma}[8.3.2.7]
\label{2.8.3.2.7}
Let $T$ be a positively-graded ring, $Z=\Proj(T)$, and $g$ a homogeneous element of $T$ of degree $d>0$.
If $g$ is not a zero divisor in $T$, then $Z$ is the smallest closed subprescheme of $Z$ that contains $Z_g=D_+(g)$.
\end{lemma}

By \sref[I]{1.4.1.9}, the question is local on $Z$;
for every homogeneous element $h\in T_e$ ($e>0$), it thus suffices to prove that $Z_h$ is the smallest closed subprescheme of $Z_h$ that contains $Z_{gh}$;
it follows from the definitions and from \sref[I]{1.4.3.2} that this condition is equivalent to asking for the canonical homomorphism $T_{(h)}\to T_{(gh)}$ to be \emph{injective}.
But this homomorphism can be identified with the canonical homomorphism $T_{(h)}\to(T_{(h)})_{g^e/h^d}$ \sref{2.2.2.3}.
But since $g^e$ is not a zero divisor in $T$, $g^e/h^d$ is not a zero divisor in $T_h$ (nor \emph{a fortiori} in $T_{(h)}$), since the fact that $(g^e/h^d)(t/h^m)=0$ (for $t\in T$ and $m>0$) implies the existence of some $n>0$ such that $h^ng^et=0$, whence $h^nt=0$, and thus $t/h^m=0$ in $T_h$.
This thus finishes the proof \sref[0]{0.1.2.2}.
\end{proof}

\begin{env}[8.3.3]
\label{2.8.3.3}
We will often identify the affine cone $C$ with the subprescheme induced by the projective cone $\widehat{C}$ on the open subset $i(C)$ by means of the open immersion $i$.
The closed subprescheme of $C$ associated to the closed immersion $\varepsilon$ is called the \unsure{\emph{vertex prescheme}} of $C$;
we also say that $\varepsilon$, which is a $Y$-section of $C$, is the \unsure{\emph{vertex section}}, or the \emph{null section},  or $C$;
we can identify $Y$ with the \unsure{vertex prescheme} of $C$ by means of $\varepsilon$.
Also, $i\circ\varepsilon$ is a $Y$-section of $\widehat{C}$, and thus also a closed immersion \sref[I]{1.5.4.6}, corresponding to the canonical surjective homomorphism of degree zero $\widehat{\sh{S}} = \sh{S}[\bb{z}] \to \sh{O}_Y[\bb{z}]$ \sref{2.3.1.7}, whose kernel is $\sh{S}_+[\bb{z}] = \sh{S}_+\widehat{\sh{S}}$;
the subprescheme of $\widehat{C}$ associated to this closed immersion is also called the \unsure{\emph{vertex prescheme}} of $\widehat{C}$, and $i\circ\varepsilon$ the \unsure{\emph{vertex section}} of $\widehat{C}$;
it can be identified with $Y$ by means of $i\circ\varepsilon$.
Finally, the closed subprescheme of $\widehat{C}$ associated to $j$ is called the \unsure{\emph{points of infinity}} of $\widehat{C}$, and can be identified with $X$ by means of $j$.
\end{env}

\begin{env}[8.3.4]
\label{2.8.3.4}
The subpreschemes of $C$ (resp. $\widehat{C}$) induced on the \emph{open} subsets
\[
\label{eq:2.8.3.4.1}
  E = C \setminus \varepsilon(Y),
  \qquad
  \widehat{E} = \widehat{C} \setminus i(\varepsilon(Y))
\tag{8.3.4.1}
\]
are called (by an abuse of language) the \emph{affine pointed cone} and the \emph{projective pointed cone} (respectively) defined by \emph{\sh{S}};
we note that, despite this nomenclature, \emph{$E$ is not necessarily affine} over $Y$, nor $\widehat{E}$ projective over $Y$ \sref{2.8.4.3}.
When we identify $C$ with $i(C)$, we thus have the underlying spaces
\[
\label{eq:2.8.3.4.2}
  C \cup \widehat{E} = \widehat{C},
  \qquad
  C \cap \widehat{E} = E
\tag{8.3.4.2}
\]
so that $\widehat{C}$ can be considered as being obtained by \emph{gluing} the open subpreschemes $C$ and $\widehat{E}$;
furthermore, by \eref{eq:2.8.3.2.3},
\[
\label{eq:2.8.3.4.3}
  E = \widehat{E} \setminus j(X).
\tag{8.3.4.3}
\]

If $Y=\Spec(A)$ is affine, then, with the notation of \sref{2.8.3.2},
\[
\label{eq:2.8.3.4.4}
  E = \bigcup C_f,
  \qquad
  \widehat{E} = \bigcup \widehat{C}_f,
  \qquad
  C_f = C \cap \widehat{C}_f
\tag{8.3.4.4}
\]
where $f$ runs over the set of homogeneous elements of $S_+$ (or only a subset $M$ of this set, with $M$ generating an ideal of $S_+$ whose radical in $S_+$ is $S_+$ itself, or, equivalently, such that the $X_f$ for $f\in M$ cover $X$ \sref{2.2.3.14}).
The gluing of $C$ and $\widehat{C}_f$ along $C_f$ is thus determined by the injection morphisms $C_f\to C$ and $C_f\to\widehat{C}_f$, which, as we have seen \sref{2.8.3.2.6}, correspond (respectively) to the canonical homomorphisms $S\to S_f$ and $S_f^\leq\to S_f$.
\end{env}

\begin{proposition}[8.3.5]
\label{2.8.3.5}
With the notation of \sref{2.8.3.1} and \sref{2.8.3.4}, the morphism associated \sref{2.3.5.1} to the canonical injection $\varphi:\sh{S} \to \widehat{\sh{S}} = \sh{S}[\bb{z}]$ is a surjective affine morphism (called the canonical retraction)
\[
\label{eq:2.8.3.5.1}
  p:\widehat{E} \to X
\tag{8.3.5.1}
\]
such that
\[
\label{eq:2.8.3.5.2}
  p \circ j = 1_X.
\tag{8.3.5.2}
\]
\end{proposition}

\begin{proof}
\label{proof-2.8.3.5}
To prove the proposition, we can restrict to the case where $Y$ is affine.
Taking into account the expression in \eref{eq:2.8.3.4.4} for $\widehat{E}$, the fact that the domain of definition $G(\varphi)$ of $p$ is equal to $\widehat{E}$ will follow from the first of the following claims:
\begin{env}[8.3.5.3]
\label{2.8.3.5.3}
If $Y=\Spec(A)$ is affine, and $\sh{S}=\widetilde{S}$, then, for all homogeneous $f\in S_+$,
\[
\label{eq:2.8.3.5.4}
  p^{-1}(X_f) = \widehat{C}_f
\tag{8.3.5.4}
\]
and the restriction of $p$ to $\widehat{C}_f=\Spec(S_f^\leq)$, thought of as a morphism from $\widehat{C}_f$ to $X_f$, corresponds to the canonical injection $S_{(f)}\to S_f^\leq$.
If, further, $f\in S_1$, then $\widehat{C}_f$ is isomorphic to $X_f\otimes_{\bb{Z}}\bb{Z}[T]$ (where $T$ is an indeterminate).
\end{env}

Indeed, Equation \eref{eq:2.8.3.5.4} is exactly a particular case of \eref{2.2.8.1.1}, and the second claim is exactly the definition of $\Proj(\varphi)$ whenever $Y$ is affine \sref{2.2.8.1}.
Then Equation \eref{eq:2.8.3.5.2} and the fact that $p$ is surjective show that the composition $\sh{S}\to\widehat{\sh{S}}\to\sh{S}$ of the canonical homomorphisms is the identity on $\sh{S}$.
Finally, the last claim of \sref{2.8.3.5.3} follows from the fact that $S_f^\leq$ is isomorphic to $S_{(f)}[T]$ whenever $f\in S_1$ \sref{2.2.2.1}.
\end{proof}

\begin{corollary}[8.3.6]
\label{2.8.3.6}
The restriction
\[
\label{eq:2.8.3.6.1}
  \pi: E \to X
\tag{8.3.6.1}
\]
of $p$ to $E$ is a surjective affine morphism.
If $Y$ is affine and $f$ homogeneous in $S_+$, then
\[
\label{eq:2.8.3.6.2}
  \pi^{-1}(X_f) = C_f
\tag{8.3.6.2}
\]
and the restriction of $\pi$ to $C_f$ corresponds to the canonical injection $S_{(f)}\to S_f$.
If, further, $f\in S_1$, then $C_f$ is isomorphic to $X_f\otimes_{\bb{Z}}\bb{Z}[T,T^{-1}]$ (where $T$ is an indeterminate).
\end{corollary}

\begin{proof}
\label{proof-2.8.3.6}
Equation \eref{eq:2.8.3.6.2} follows immediately from \sref{2.8.3.5.3} and \eref{eq:2.8.3.2.5}, and shows the surjectivity of $\pi$;
we have already seen that the immersion $i$, restricted to $C_f$, corresponds
\oldpage[II]{166}
to the injection $S_f^\leq\to S_f$ \sref{2.8.3.2}.
Finally, the last claim is a consequence of the fact that, for $f\in S_1$, $S_f$ is isomorphic to $S_{(f)}[T,T^{-1}]$ \sref{2.2.2.1}.
\end{proof}


% \subsection{Projective closure of a vector bundle}
% \label{subsection:2.8.4}


% \subsection{Functorial behaviour}
% \label{subsection:2.8.5}


% \subsection{A canonical isomorphism for pointed cones}
% \label{subsection:2.8.6}


% \subsection{Blowing up \unsure{projective} cones}
% \label{subsection:2.8.7}


% \subsection{Ample sheaves and contractions}
% \label{subsection:2.8.8}


% \subsection{Grauert's ampleness criterion: statement}
% \label{subsection:2.8.9}


% \subsection{Grauert's ampleness criterion: proof}
% \label{subsection:2.8.10}


% \subsection{Uniqueness of contractions}
% \label{subsection:2.8.11}


% \subsection{Quasi-coherent sheaves on \unsure{projective} cones}
% \label{subsection:2.8.12}

% \begin{env}[8.12.1]
% \label{2.8.12.1}
% Let us take the hypotheses and notation of \sref{2.8.3.1}.
% Let $\sh{M}$ be a \emph{quasi-coherent graded $\sh{S}$-module}; to avoid any confusion, we denote by $\widetilde{\sh{M}}$ the quasi-coherent $\sh{O}_C$-module
% \oldpage[II]{192}
% associated to $\sh{M}$ \sref{2.1.4.3} when $\sh{M}$ is considered as a \emph{nongraded} $\sh{S}$-module, and by $\shProj_0(\sh{M})$ the quasi-coherent $\sh{O}_X$-module associated to $\sh{M}$, $\sh{M}$ being considered this time as a graded $\sh{S}$-module (in other words, the $\sh{O}_X$-module denoted by $\widetilde{\sh{M}}$ in \sref{2.3.2.2}).
% In addition, we set
% \[
% \label{eq:2.8.12.1.1}
%   \sh{M}_X=\shProj_0(\sh{M})=\bigoplus_{n\in\bb{Z}}\shProj_0(\sh{M}(n));
%   \tag{8.12.1.1}
% \]
% the quasi-coherent graded $\sh{O}_X$-algebra $\sh{S}_X$ being defined by \eref{eq:2.8.6.1.1}, $\shProj(\sh{M})$ is equipped with a structure of a \emph{(quasi-coherent) graded $\sh{S}_X$-module}, by means of the canonical homomorphisms \eref{eq:2.3.2.6.1}
% \[
% \label{eq:2.8.12.1.2}
%   \sh{O}_X(m)\otimes_{\sh{O}_X}\shProj_0(\sh{M}(n))\to\shProj_0(\sh{S}(m)\otimes_\sh{S}\sh{M}(n))\to\shProj_0(\sh{M}(m+n)),
%   \tag{8.12.1.2}
% \]
% the verification of the axioms of sheaves of modules being done using the commutative diagram \eref{eq:2.2.5.11.4}.

% If $Y=\Spec(A)$ is affine, $\sh{S}=\widetilde{S}$, and $\sh{M}=\widetilde{M}$, where $S$ is a graded $A$-algebra and $M$ is a graded $S$-module, then, for every homogeneous element $f\in S_+$, we have
% \[
% \label{eq:2.8.12.1.3}
%   \Gamma(X_f,\shProj(\widetilde{M}))=M_f
%   \tag{8.12.1.3}
% \]
% by the definitions and \eref{eq:2.8.2.9.1}.

% Now consider the quasi-coherent graded $\widehat{\sh{S}}$-module
% \[
% \label{eq:2.8.12.1.4}
%   \widehat{\sh{M}}=\sh{M}\otimes_\sh{S}\widehat{\sh{S}}
%   \tag{8.12.1.4}
% \]
% ($\widehat{\sh{S}}$ defined by \eref{eq:2.8.3.1.1}); we deduce a quasi-coherent graded $\sh{O}_{\widehat{C}}$-module $\shProj_0(\widehat{\sh{M}})$, which we will also denote by
% \[
% \label{eq:2.8.12.1.5}
%   \sh{M}^\square=\shProj_0(\widehat{\sh{M}}).
%   \tag{8.12.1.5}
% \]

% It is clear \sref{2.3.2.4} that $\sh{M}^\square$ is an additive functor which is \emph{exact} in $\sh{M}$, commuting with direct sums and with inductive limits.
% \end{env}
