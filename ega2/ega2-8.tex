\section{Blowup schemes; projective cones; projective closure}
\label{section:2.8}

\subsection{Blowup preschemes}
\label{subsection:2.8.1}

\begin{env}[8.1.1]
\label{2.8.1.1}
Let $Y$ be a prescheme, and, for every integer $n\geq 0$, let $\sh{I}_n$ be a quasi-coherent sheaf of ideals of $\sh{O}_Y$; suppose that the following conditions are satisfied:
\[
\label{eq:2.8.1.1.1}
  \sh{I}_0=\sh{O}_Y,\ \sh{I}_n\subset\sh{I}_m\text{ for }m\leq n,
  \tag{8.1.1.1}
\]
\[
\label{eq:2.8.1.1.2}
  \sh{I}_m\sh{I}_n\subset\sh{I}_{m+n}\text{ for any }m,n.
  \tag{8.1.1.2}
\]

\oldpage[II]{153}
We note that these hypotheses imply
\[
\label{eq:2.8.1.1.3}
  \sh{I}_1^n\subset\sh{I}_n.
  \tag{8.1.1.3}
\]

Set
\[
\label{eq:2.8.1.1.4}
  \sh{S}=\bigoplus_{n\geq 0}\sh{I}_n.
  \tag{8.1.1.4}
\]

It follows from \eref{eq:2.8.1.1.1} and \eref{eq:2.8.1.1.2} that $\sh{S}$ is a quasi-coherent graded $\sh{O}_Y$-algebra, and thus defines a $Y$-scheme $X=\Proj(\sh{S})$.
If $\sh{J}$ is an \emph{invertible} sheaf of ideals of $\sh{O}_Y$, then $\sh{I}_n\otimes_{\sh{O}_Y}\sh{J}^{\otimes n}$ is canonically identified with $\sh{I}_n\sh{J}^n$.
If we then replace the $\sh{I}_n$ by the $\sh{I}_n\sh{J}^n$, and, in doing so, replace $\sh{S}$ by a quasi-coherent $\sh{O}_Y$-algebra $\sh{S}_{(\sh{J})}$, then $X_{(\sh{J})}=\Proj(\sh{S}_{(\sh{J})})$ is canonically isomorphic to $X$ \sref{2.3.1.8}.
\end{env}

\begin{env}[8.1.2]
\label{2.8.1.2}
Suppose that $Y$ is \emph{locally integral}, so that the sheaf $\sh{R}(Y)$ of rational functions is a quasi-coherent $\sh{O}_Y$-algebra \sref[1]{1.7.3.7}.
We say that a sub-$\sh{O}_Y$-module $\sh{I}$ of $\sh{R}(Y)$ is a \emph{fractional ideal} of $\sh{R}(Y)$ if it is of \emph{finite type} \sref[0]{0.5.2.1}.
Suppose we have, for all $n\geq0$, a quasi-coherent fractional ideal $\sh{I}_n$ of $\sh{R}(Y)$, such that $\sh{I}_0 = \sh{O}_Y$, and such that condition \eref{2.8.1.1.2} (but not necessarily the second condition \eref{2.8.1.1.1}) is satisfied;
we can then again define a graded quasi-coherent $\sh{O}_Y$-algebra by Equation \eref{2.8.1.1.4}, and the corresponding $Y$-scheme $X = \Proj(\sh{S})$;
we will again have a canonical isomorphism from $X$ to $X_{{\sh{J}}}$ for every \emph{invertible} fractional ideal $\sh{J}$ of $\sh{R}(Y)$.
\end{env}

\begin{definition}[8.1.3]
\label{2.8.1.3}
Let $Y$ be a prescheme (resp. a locally integral prescheme), and $\sh{I}$ a quasi-coherent ideal of $\sh{O}_Y$ (resp. a quasi-coherent fractional ideal of $\sh{R}(Y)$).
We say that the $Y$-scheme $X = \Proj(\bigoplus_{n\geq0}\sh{I}^n)$ is obtained by blowing up the ideal $\sh{I}$, or is the blow-up prescheme of $Y$ relative to $\sh{I}$.
When $\sh{I}$ is a quasi-coherent ideal of $\sh{O}_Y$, and $Y'$ is the closed sub-prescheme of $Y$ defined by $\sh{I}$, we also say that $X$ is the $Y$-scheme obtained by blowing up $Y'$.
\end{definition}

By definition, $\sh{S} = \bigoplus_{n\geq0}\sh{I}^n$ is then generated by $\sh{S}_1 = \sh{I}$;
if $\sh{I}$ is an $\sh{O}_Y$-module of \emph{finite type}, then $X$ is \emph{projective} over $Y$ \sref{2.5.5.2}.
Without any hypotheses on $\sh{I}$, the $\sh{O}_X$-module $\sh{O}_X(1)$ is \emph{invertible} \sref{2.3.2.5} and \emph{very ample}, by \sref{2.4.4.3} applied to the structure morphism $X\to Y$.

We note that, if $j:X\to Y$ is the structure morphism, then the restriction of $f$ to $f^{-1}(Y\setminus Y')$ is an \emph{isomorphism} to $Y\setminus Y'$ whenever $\sh{I}$ is an \emph{ideal of $\sh{O}_Y$} and $Y'$ is the closed sub-prescheme that it defines: indeed, the question being local on $Y$, it suffices to assume that $\sh{I} = \sh{O}_Y$, and our claim then follows from \sref{2.3.1.7}.

If we replace $\sh{I}$ by $\sh{I}^d$ ($d>0$), then the blow-up $Y$-scheme $X$ is replaced by a canonically isomorphic $Y$-scheme $X'$ \sref{2.8.1.1};
similarly, for every \emph{invertible} ideal (resp. \emph{invertible} fractional ideal) $\sh{J}$, the blow-up prescheme $X_{(\sh{J})}$ relative to the ideal $\sh{I}\sh{J}$ is canonically isomorphic to $X$ \sref{2.8.1.1}.

In particular, whenever $\sh{I}$ is an \emph{invertible} ideal (resp. \emph{invertible} fractional ideal), the $Y$-scheme obtained by blowing up $\sh{I}$ is \emph{isomorphic to $Y$} \sref{2.3.1.7}.

\begin{proposition}[8.1.3]
\label{2.8.1.4}
Let $Y$ be an integral prescheme.
\begin{enumerate}
    \item[\rm{(i)}] For every sequence $(\sh{I}_n)$ of quasi-coherent fractional ideals of $\sh{R}(Y)$ that satisfies \eref{2.8.1.1.2}
\oldpage[II]{154}
        and such that $\sh{I}_0 = \sh{O}_Y$, the $Y$-scheme $X=\Proj(\bigoplus_{n\geq0}\sh{I}^n)$ is integral, and the structure morphism $f:X\to Y$ is dominant.
    \item[\rm{(ii)}] Let $\sh{I}$ be a quasi-coherent fractional ideal of $\sh{R}(Y)$, and let $X$ be the $Y$-scheme given by the blow up of $Y$ relative to $\sh{I}$.
        If $\sh{I} \neq 0$, then the structure morphism $f:X\to Y$ is then birational and surjective.
\end{enumerate}
\end{proposition}

\begin{proof}
\label{proof-2.8.1.4}
\begin{enumerate}
    \item[\rm{(i)}] This follows from the fact that $\sh{S} = \bigoplus_{n\geq0}\sh{I}_n$ is an \emph{integral} $\sh{O}_Y$-algebra (\sref{2.3.1.12} and \sref{2.3.1.14}), since, for all $y\in Y$, $\sh{O}_y$ is an integral ring \sref[I]{1.5.1.4}.
    \item[\rm{(ii)}] By (i), $X$ is integral;
        if, furthermore, $x$ and $y$ are the generic points of $X$ and $Y$ (respectively), then we have $f(x) = y$, and it remains to show that $\kres(x)$ is of rank 1 over $\kres(y)$.
        But $x$ is also the generic point of the fibre $f^{-1}(y)$;
        if $\psi$ is the canonical morphism $Z\to Y$, where $Z=\Spec(\kres(y))$, then the prescheme $f^{-1}(y)$ can be identified with $\Proj(\sh{S}')$, where $\sh{S}' = \psi^*(\sh{S})$ \sref{2.3.5.3}.
        But it is clear that $\sh{S}' = \bigoplus_{n\geq0}(\sh{I}_y)^n$, and, since $\sh{I}$ is a quasi-coherent fractional ideal of $\sh{R}(Y)$ that is not zero, $\sh{I}_y \neq 0$ \sref[I]{1.7.3.6}, whence $\sh{I}_y = \kres(y)$;
        then $\Proj(\sh{S}')$ can be identified with $\Spec(\kres(y))$ \sref{2.3.1.7}, whence the conclusion.
\end{enumerate}
\end{proof}

We show the \emph{converse} of \sref{2.8.1.4} in \sref[III]{3.2.3.8}.

% \subsection{Quasi-coherent sheaves on projective cones}
% \label{subsection:2.8.12}

% \begin{env}[8.12.1]
% \label{2.8.12.1}
% Let us take the hypotheses and notation of \sref{2.8.3.1}.
% Let $\sh{M}$ be a \emph{quasi-coherent graded $\sh{S}$-module}; to avoid any confusion, we denote by $\widetilde{\sh{M}}$ the quasi-coherent $\sh{O}_C$-module
% \oldpage[II]{192}
% associated to $\sh{M}$ \sref{2.1.4.3} when $\sh{M}$ is considered as a \emph{nongraded} $\sh{S}$-module, and by $\shProj_0(\sh{M})$ the quasi-coherent $\sh{O}_X$-module associated to $\sh{M}$, $\sh{M}$ being considered this time as a graded $\sh{S}$-module (in other words, the $\sh{O}_X$-module denoted by $\widetilde{\sh{M}}$ in \sref{2.3.2.2}).
% In addition, we set
% \[
% \label{eq:2.8.12.1.1}
%   \sh{M}_X=\shProj_0(\sh{M})=\bigoplus_{n\in\bb{Z}}\shProj_0(\sh{M}(n));
%   \tag{8.12.1.1}
% \]
% the quasi-coherent graded $\sh{O}_X$-algebra $\sh{S}_X$ being defined by \eref{eq:2.8.6.1.1}, $\shProj(\sh{M})$ is equipped with a structure of a \emph{(quasi-coherent) graded $\sh{S}_X$-module}, by means of the canonical homomorphisms \eref{eq:2.3.2.6.1}
% \[
% \label{eq:2.8.12.1.2}
%   \sh{O}_X(m)\otimes_{\sh{O}_X}\shProj_0(\sh{M}(n))\to\shProj_0(\sh{S}(m)\otimes_\sh{S}\sh{M}(n))\to\shProj_0(\sh{M}(m+n)),
%   \tag{8.12.1.2}
% \]
% the verification of the axioms of sheaves of modules being done using the commutative diagram \eref{eq:2.2.5.11.4}.

% If $Y=\Spec(A)$ is affine, $\sh{S}=\widetilde{S}$, and $\sh{M}=\widetilde{M}$, where $S$ is a graded $A$-algebra and $M$ is a graded $S$-module, then, for every homogeneous element $f\in S_+$, we have
% \[
% \label{eq:2.8.12.1.3}
%   \Gamma(X_f,\shProj(\widetilde{M}))=M_f
%   \tag{8.12.1.3}
% \]
% by the definitions and \eref{eq:2.8.2.9.1}.

% Now consider the quasi-coherent graded $\widehat{\sh{S}}$-module
% \[
% \label{eq:2.8.12.1.4}
%   \widehat{\sh{M}}=\sh{M}\otimes_\sh{S}\widehat{\sh{S}}
%   \tag{8.12.1.4}
% \]
% ($\widehat{\sh{S}}$ defined by \eref{eq:2.8.3.1.1}); we deduce a quasi-coherent graded $\sh{O}_{\widehat{C}}$-module $\shProj_0(\widehat{\sh{M}})$, which we will also denote by
% \[
% \label{eq:2.8.12.1.5}
%   \sh{M}^\square=\shProj_0(\widehat{\sh{M}}).
%   \tag{8.12.1.5}
% \]

% It is clear \sref{2.3.2.4} that $\sh{M}^\square$ is an additive functor which is \emph{exact} in $\sh{M}$, commuting with direct sums and with inductive limits.
% \end{env}
