\section{Blowup schemes; based cones; projective closure}
\label{section:II.8}

\subsection{Blowup preschemes}
\label{subsection:II.8.1}

\begin{env}[8.1.1]
\label{II.8.1.1}
Let $Y$ be a prescheme, and, for every integer $n\geq 0$, let $\sh{I}_n$ be a quasi-coherent sheaf of ideals of $\sh{O}_Y$; suppose that the following conditions are satisfied:
\[
\label{II.8.1.1.1}
  \sh{I}_0=\sh{O}_Y,\ \sh{I}_n\subset\sh{I}_m\text{ for }m\leq n,
\tag{8.1.1.1}
\]
\[
\label{II.8.1.1.2}
  \sh{I}_m\sh{I}_n\subset\sh{I}_{m+n}\text{ for any }m,n.
\tag{8.1.1.2}
\]

\oldpage[II]{153}
We note that these hypotheses imply
\[
\label{II.8.1.1.3}
  \sh{I}_1^n\subset\sh{I}_n.
\tag{8.1.1.3}
\]

Set
\[
\label{II.8.1.1.4}
  \sh{S}=\bigoplus_{n\geq 0}\sh{I}_n.
\tag{8.1.1.4}
\]

It follows from \sref{II.8.1.1.1} and \sref{II.8.1.1.2} that $\sh{S}$ is a quasi-coherent graded $\sh{O}_Y$-algebra, and thus defines a $Y$-scheme $X=\Proj(\sh{S})$.
If $\sh{J}$ is an \emph{invertible} sheaf of ideals of $\sh{O}_Y$, then $\sh{I}_n\otimes_{\sh{O}_Y}\sh{J}^{\otimes n}$ is canonically identified with $\sh{I}_n\sh{J}^n$.
If we then replace the $\sh{I}_n$ by the $\sh{I}_n\sh{J}^n$, and, in doing so, replace $\sh{S}$ by a quasi-coherent $\sh{O}_Y$-algebra $\sh{S}_{(\sh{J})}$, then $X_{(\sh{J})}=\Proj(\sh{S}_{(\sh{J})})$ is canonically isomorphic to $X$ \sref{II.3.1.8}.
\end{env}

\begin{env}[8.1.2]
\label{II.8.1.2}
Suppose that $Y$ is \emph{locally integral}, so that the sheaf $\sh{R}(Y)$ of rational functions is a quasi-coherent $\sh{O}_Y$-algebra \sref[1]{I.7.3.7}.
We say that a $\sh{O}_Y$-submodule $\sh{I}$ of $\sh{R}(Y)$ is a \emph{fractional ideal} of $\sh{R}(Y)$ if it is of \emph{finite type} \sref[0]{0.5.2.1}.
Suppose we have, for all $n\geq0$, a quasi-coherent fractional ideal $\sh{I}_n$ of $\sh{R}(Y)$, such that $\sh{I}_0 = \sh{O}_Y$, and such that condition \sref{II.8.1.1.2} (but not necessarily the second condition \sref{II.8.1.1.1}) is satisfied;
we can then again define a quasi-coherent graded $\sh{O}_Y$-algebra by Equation~\sref{II.8.1.1.4}, and the corresponding $Y$-scheme $X = \Proj(\sh{S})$;
we will again have a canonical isomorphism from $X$ to $X_{{\sh{J}}}$ for every \emph{invertible} fractional ideal $\sh{J}$ of $\sh{R}(Y)$.
\end{env}

\begin{definition}[8.1.3]
\label{II.8.1.3}
Let $Y$ be a prescheme (resp. a locally integral prescheme), and $\sh{I}$ a quasi-coherent ideal of $\sh{O}_Y$ (resp. a quasi-coherent fractional ideal of $\sh{R}(Y)$).
We say that the $Y$-scheme $X = \Proj(\bigoplus_{n\geq0}\sh{I}^n)$ is obtained by blowing up the ideal $\sh{I}$, or is the blow-up prescheme of $Y$ relative to $\sh{I}$.
When $\sh{I}$ is a quasi-coherent ideal of $\sh{O}_Y$, and $Y'$ is the closed subprescheme of $Y$ defined by $\sh{I}$, we also say that $X$ is the $Y$-scheme obtained by blowing up $Y'$.
\end{definition}

By definition, $\sh{S} = \bigoplus_{n\geq0}\sh{I}^n$ is then generated by $\sh{S}_1 = \sh{I}$;
if $\sh{I}$ is an $\sh{O}_Y$-module of \emph{finite type}, then $X$ is \emph{projective} over $Y$ \sref{II.5.5.2}.
Without any hypotheses on $\sh{I}$, the $\sh{O}_X$-module $\sh{O}_X(1)$ is \emph{invertible} \sref{II.3.2.5} and \emph{very ample}, by \sref{II.4.4.3} applied to the structure morphism $X\to Y$.

We note that, if $j:X\to Y$ is the structure morphism, then the restriction of $f$ to $f^{-1}(Y\setmin Y')$ is an \emph{isomorphism} to $Y\setmin Y'$ whenever $\sh{I}$ is an \emph{ideal of $\sh{O}_Y$} and $Y'$ is the closed subprescheme that it defines: indeed, since the questions is local on $Y$, it suffices to assume that $\sh{I} = \sh{O}_Y$, and our claim then follows from \sref{II.3.1.7}.

If we replace $\sh{I}$ by $\sh{I}^d$ ($d>0$), then the blow-up $Y$-scheme $X$ is replaced by a canonically isomorphic $Y$-scheme $X'$ \sref{II.8.1.1};
similarly, for every \emph{invertible} ideal (resp. \emph{invertible} fractional ideal) $\sh{J}$, the blow-up prescheme $X_{(\sh{J})}$ relative to the ideal $\sh{I}\sh{J}$ is canonically isomorphic to $X$ \sref{II.8.1.1}.

In particular, whenever $\sh{I}$ is an \emph{invertible} ideal (resp. \emph{invertible} fractional ideal), the $Y$-scheme obtained by blowing up $\sh{I}$ is \emph{isomorphic to $Y$} \sref{II.3.1.7}.

\begin{proposition}[8.1.3]
\label{II.8.1.4}
Let $Y$ be an integral prescheme.
\begin{enumerate}
    \item[\rm{(i)}] For every sequence $(\sh{I}_n)$ of quasi-coherent fractional ideals of $\sh{R}(Y)$ that satisfies \sref{II.8.1.1.2}
\oldpage[II]{154}
        and such that $\sh{I}_0 = \sh{O}_Y$, the $Y$-scheme $X=\Proj(\bigoplus_{n\geq0}\sh{I}^n)$ is integral, and the structure morphism $f:X\to Y$ is dominant.
    \item[\rm{(ii)}] Let $\sh{I}$ be a quasi-coherent fractional ideal of $\sh{R}(Y)$, and let $X$ be the $Y$-scheme given by the blow up of $Y$ relative to $\sh{I}$.
        If $\sh{I} \neq 0$, then the structure morphism $f:X\to Y$ is then birational and surjective.
\end{enumerate}
\end{proposition}

\begin{proof}
\medskip\noindent
\begin{enumerate}
    \item[\rm{(i)}] This follows from the fact that $\sh{S} = \bigoplus_{n\geq0}\sh{I}_n$ is an \emph{integral} $\sh{O}_Y$-algebra (\sref{II.3.1.12} and \sref{II.3.1.14}), since, for all $y\in Y$, $\sh{O}_y$ is an integral ring \sref[I]{I.5.1.4}.
    \item[\rm{(ii)}] By (i), $X$ is integral;
        if, furthermore, $x$ and $y$ are the generic points of $X$ and $Y$ (respectively), then we have $f(x) = y$, and it remains to show that $\kres(x)$ is of rank 1 over $\kres(y)$.
        But $x$ is also the generic point of the fibre $f^{-1}(y)$;
        if $\psi$ is the canonical morphism $Z\to Y$, where $Z=\Spec(\kres(y))$, then the prescheme $f^{-1}(y)$ can be identified with $\Proj(\sh{S}')$, where $\sh{S}' = \psi^*(\sh{S})$ \sref{II.3.5.3}.
        But it is clear that $\sh{S}' = \bigoplus_{n\geq0}(\sh{I}_y)^n$, and, since $\sh{I}$ is a quasi-coherent fractional ideal of $\sh{R}(Y)$ that is not zero, $\sh{I}_y \neq 0$ \sref[I]{I.7.3.6}, whence $\sh{I}_y = \kres(y)$;
        then $\Proj(\sh{S}')$ can be identified with $\Spec(\kres(y))$ \sref{II.3.1.7}, whence the conclusion.
\end{enumerate}
\end{proof}

We show a \emph{converse} of \sref{II.8.1.4} in \sref[III]{III.2.3.8}.

\begin{env}[8.1.5]
\label{II.8.1.5}
We return to the setting and notation of \sref{II.8.1.1}.
By definition, the injection homomorphisms $\sh{I}_{n+1}\to\sh{I}_n$ \sref{II.8.1.1.1} define, for every $k\in\bb{Z}$, an injective homomorphism of degree zero of graded $\sh{S}$-modules
\[
\label{II.8.1.5.1}
  u_k: \sh{S}_+(k+1) \to \sh{S}(k);
\tag{8.1.5.1}
\]
since $\sh{S}_+(k+1)$ and $\sh{S}(k+1)$ are canonically \textbf{(TN)}-isomorphic, they give a canonical correspondence between $u_k$ and an injective homomorphism of $\sh{O}_X$-modules \sref{II.3.4.2}:
\[
\label{II.8.1.5.2}
  \widetilde{u}_k: \sh{O}_X(k+1) \to \sh{O}_X(k).
\tag{8.1.5.2}
\]

Recall as well \sref{II.3.2.6} that we have defined canonical homomorphisms
\[
\label{II.8.1.5.3}
  \lambda: \sh{O}_X(h) \otimes_{\sh{O}_X} \sh{O}_X(k) \to \sh{O}_X(h+k)
\tag{8.1.5.3}
\]
and, since the diagram
\[
  \xymatrix{
    \sh{S}(h) \otimes_{\sh{S}} \sh{S}(k) \otimes_{\sh{S}} \sh{S}(l)
      \ar[r]
      \ar[d]
  & \sh{S}(h+k) \otimes_{\sh{S}} \sh{S}(l)
      \ar[d]
  \\\sh{S}(h) \otimes_{\sh{S}} \sh{S}(k+l)
      \ar[r]
  & \sh{S}(h+k+l)
  }
\]
commutes, it follows from the functoriality of the $\lambda$ \sref{II.3.2.6} that the homomorphisms \sref{II.8.1.5.3} define the structure of a \emph{quasi-coherent graded $\sh{O}_X$-algebra} on
\[
\label{II.8.1.5.4}
  \sh{S}_X = \bigoplus_{n\in\bb{Z}}\sh{O}_X(n).
\tag{8.1.5.4}
\]
Furthermore, the diagram
\[
  \xymatrix{
    \sh{S}(h) \otimes_{\sh{S}} \sh{S}(k+1)
      \ar[r]
      \ar[d]_{1\otimes u_k}
  & \sh{S}(h+k+1)
      \ar[d]^{u_{k+h}}
  \\\sh{S}(h) \otimes_{\sh{S}} \sh{S}(k)
      \ar[r]
  & \sh{S}(h+k)
  }
\]
commutes; the functoriality of the $\lambda$ then implies that we have a commutative diagram
\[
  \xymatrix{
    \sh{O}_X(h) \otimes_{\sh{O}_X} \sh{O}_X(k+1)
      \ar[r]^{\lambda}
      \ar[d]_{1\otimes \widetilde{u}_k}
  & \sh{O}_X(h+k+1)
      \ar[d]^{\widetilde{u}_{k+h}}
  \\\sh{O}_X(h) \otimes_{\sh{O}_X} \sh{O}_X(k)
      \ar[r]^{\lambda}
  & \sh{O}_X(h+k)
  }
\]
where the horizontal arrows are the canonical homomorphisms.
We can thus say that the $\widetilde{u}_k$ define an \emph{injective homomorphism} (of degree zero) \emph{of graded $\sh{S}_X$-modules}
\[
\label{II.8.1.5.6}
  \widetilde{u}: \sh{S}_X(1) \to \sh{S}_X.
\tag{8.1.5.6}
\]
\end{env}

\begin{env}[8.1.6]
\label{II.8.1.6}
Keeping the notation from \sref{II.8.1.5}, we now note that, for $n\geq0$, the composite homomorphism $\widetilde{v}_n = \widetilde{u}_{n-1} \circ \widetilde{u}_{n-2} \circ \ldots \circ \widetilde{u}_0$ is an \emph{injective} homomorphism $\sh{O}_X(n) \to \sh{O}_X$;
we denote by $\sh{I}_{n,X}$ its image, which is thus a quasi-coherent ideal of $\sh{O}_X$, \emph{isomorphic} to $\sh{O}_X(n)$.
Furthermore, the diagram
\[
  \xymatrix{
    \sh{O}_X(m) \otimes_{\sh{O}_X} \sh{O}_X(n)
      \ar[r]^{\lambda}
      \ar[d]_{\widetilde{v}_m \otimes \widetilde{v}_n}
  & \sh{O}_X(m+n)
      \ar[d]^{\widetilde{v}_{m+n}}
  \\\sh{O}_X
      \ar[r]_{\id}
  & \sh{O}_X
  }
\]
commutes for $m\geq0$, $n\geq0$.
We thus deduce the following inclusions:
\[
\label{II.8.1.6.1}
  \sh{I}_{0,X} = \sh{O}_X, \quad \sh{I}_{n,X} \subset \sh{I}_{m,X}
  \qquad\mbox{for $0\leq m\leq n$;}
\tag{8.1.6.1}
\]
\[
\label{II.8.1.6.2}
  \sh{I}_{m,X}\sh{I}_{n,X} \subset \sh{I}_{m+n,X}
  \qquad\qquad\mbox{for $m\geq0$, $n\geq0$.}
\tag{8.1.6.2}
\]
\end{env}

\oldpage[II]{156}

\begin{proposition}[8.1.7]
\label{II.8.1.7}
Let $Y$ be a prescheme, $\sh{I}$ a quasi-coherent ideal of $\sh{O}_Y$, and $X = \Proj(\bigoplus_{n\geq0}\sh{I}^n)$ the $Y$-scheme given by blowing up $\sh{I}$.
We then have, for all $n>0$, a canonical isomorphism
\[
\label{II.8.1.7.1}
  \sh{O}_X(n) \xrightarrow{\sim} \sh{I}^n\sh{O}_X = \sh{I}_{n,X}
\tag{8.1.7.1}
\]
(cf. \sref[0]{0.4.3.5}), and thus that $\sh{I}^n\sh{O}_X$ is a very-ample invertible $\sh{O}_X$-module if $n>0$.
\end{proposition}

\begin{proof}
The last claim is immediate, since $\sh{O}_X(1)$ is invertible \sref{II.3.2.5} and very ample for $Y$ by definition (\sref{II.4.4.3} and \sref{II.4.4.9}).
Also by definition, the image of $v_n$ is exactly $\sh{I}^n\sh{S}$, and \sref{II.8.1.7.1} then follows from the exactness of the functor $\widetilde{\sh{M}}$ \sref{II.3.2.4} and from Equation~\sref{II.3.2.4.1}.
\end{proof}

\begin{corollary}[8.1.8]
\label{II.8.1.8}
Under the hypotheses of \sref{II.8.1.7}, if $f:X\to Y$ is the structure morphism, and $Y'$ the closed subprescheme of $Y$ defined by $\sh{I}$, then the closed subprescheme $X' = f^{-1}(Y')$ of $X$ is defined by $\sh{I}\sh{O}_X$ (which is canonically isomorphic to $\sh{O}_X(1)$), from which we obtain a canonical short exact sequence
\[
\label{II.8.1.8.1}
  0 \to \sh{O}_X(1) \to \sh{O}_X \to \sh{O}_{X'} \to 0.
\tag{8.1.8.1}
\]
\end{corollary}

\begin{proof}
This follows from \sref{II.8.1.7.1} and from \sref[I]{I.4.4.5}.
\end{proof}

\begin{env}[8.1.9]
\label{II.8.1.9}
Under the hypotheses of \sref{II.8.1.7}, we can be more precise about the structure of the $\sh{I}_{n,X}$.
Note that the homomorphism
\[
  \widetilde{u}_{-1}: \sh{O}_X \to \sh{O}_X(-1)
\]
canonically corresponds to a section $s$ of $\sh{O}_X(-1)$ over $X$, which we call the \emph{canonical section} (relative to $\sh{I}$) \sref[0]{0.5.1.1}.
In the diagram in \sref{II.8.1.5.5}, the horizontal arrows are isomorphisms \sref{II.3.2.7}; by replacing $h$ with $k$, and $k$ with $-1$ in this diagram, we obtain that $\widetilde{u}_k = 1_k \otimes \widetilde{u}_{-1}$ (where $1_k$ denotes the identity on $\sh{O}_X(h)$), or, equivalently, that the homomorphism $\widetilde{u}_k$ is given exactly by \emph{tensoring with the canonical section $s$} (for all $k\in\bb{Z}$).
The homomorphism $\widetilde{u}$ \sref{II.8.1.5.6} can then be understood in the same way.

Thus, for all $n\geq0$, the homomorphism $\widetilde{v}_n: \sh{O}_X(n)\to\sh{O}_X$ is given exactly by tensoring with $s^{\otimes n}$;
we thus deduce:
\end{env}

\begin{corollary}[8.1.10]
\label{II.8.1.10}
With the notation of \sref{II.8.1.8}, the underlying space of $X'$ is the set of $x\in X$ such that $s(x)=0$, where $s$ denotes the canonical section of $\sh{O}_X(-1)$.
\end{corollary}

\begin{proof}
Indeed, if $c_x$ is a generator of the fibre $(\sh{O}_X(1))_x$ at a point $x$, then $s_x\otimes c_x$ is canonically identified with a generator of the fibre of $\sh{I}_{1,X}$ at the point $x$, and is thus invertible if and only if $s_x\not\in\mathfrak{m}_x(\sh{O}_x(-1))_x$, or, equivalently, if and only if $s(x)\neq0$.
\end{proof}

\begin{proposition}
\label{II.8.1.11}
Let $Y$ be an integral prescheme, $\sh{I}$ a quasi-coherent fractional ideal of $\sh{R}(Y)$, and $X$ the $Y$-scheme given by blowing up $\sh{I}$.
Then $\sh{I}\sh{O}_X$ is an invertible $\sh{O}_X$-module that is very ample for $Y$.
\end{proposition}

\begin{proof}
Since the questions is local on $Y$ \sref{II.4.4.5}, we can reduce to the case where $Y=\Spec(A)$, with $A$ some integral ring of ring of fractions $K$, and $\sh{I}=\widetilde{\mathfrak{I}}$, with $\mathfrak{I}$ some fractional ideal of $K$;
there then exists an element $a\neq0$ of $A$ such that $a\mathfrak{I}\subset A$.
Let $S = \bigoplus_{n\geq0}\mathfrak{I}^n$;
the map $x\mapsto ax$ is an $A$-isomorphism from $\mathfrak{I}^{n+1} = (S(1))_n$ to $a\mathfrak{I}^{n+1} = a\mathfrak{I}S_n \subset \mathfrak{I}^n = S_n$,
\oldpage[II]{157}
and thus defines a (TN)-isomorphism of degree zero of graded $S$-modules $S_+(1)\to a\mathfrak{I}S$.
On the other hand, $x\mapsto a^{-1}x$ is an isomorphism of degree zero of graded $S$-modules $a\mathfrak{I}S \xrightarrow{\sim} \mathfrak{I}S$.
We thus obtain, by composition \sref{II.3.2.4}, an isomorphism of $\sh{O}_X$-modules $\sh{O}_X(1) \xrightarrow{\sim} \sh{I}\sh{O}_X$, and, since $S$ is generated by $S_1=\mathfrak{I}$, $\sh{O}_X(1)$ is invertible \sref{II.3.2.5} and very ample (\sref{II.4.4.3} and \sref{II.4.4.9}), whence our claim.
\end{proof}


\subsection{Preliminary results on the localisation of graded rings}
\label{subsection:II.8.2}

\begin{env}[8.2.1]
\label{II.8.2.1}
Let $S$ be a graded ring, but not assumed (for the moment) to be only in positive degree.
We define
\[
\label{II.8.2.1.1}
  S^\geq = \bigoplus_{n\geq0} S_n,
  \qquad
  S^\leq = \bigoplus_{n\leq0} S_n
\tag{8.2.1.1}
\]
which are both graded subrings of $S$, in only positive and negative degrees (respectively).
If $f$ is a homogeneous elements of degree $d$ (positive or negative) of $S$, then the ring of fraction $S_f=S'$ is again endowed with the structure of a graded ring, by taking $S'_n$ ($n\in\bb{Z}$) to be the set of the $x/f^k$ for $x\in S_{n+kd}$ ($k\geq0$);
we define $S_{(f)}=S'_0$, and will write $S_f^\geq$ and $S_f^\leq$ for $S^{'\geq}$ and $S^{'\leq}$ (respectively).
If $d>0$, then
\[
\label{II.8.2.1.2}
  (S^\geq)_f = S_f
\tag{8.2.1.2}
\]
since, if $x\in S_{n+kd}$ with $n+kd<0$, then we can write $x/f^k = xf^h/f^{h+k}$, and we also have that $n+(h+k)d>0$ for $h$ sufficiently large and $>0$.
We thus conclude, by definition, that
\[
\label{II.8.2.1.3}
  (S^\geq)_{(f)} = (S_f^\geq)_0 = S_{(f)}.
\tag{8.2.1.3}
\]

If $M$ is a graded $S$-module, then we similarly define
\[
\label{II.8.2.1.4}
  M^\geq = \bigoplus_{n\geq0} M_n,
  \qquad
  M^\leq = \bigoplus_{n\leq0} M_n
\tag{8.2.1.4}
\]
which are (respectively) a graded $S^\geq$-module and a graded $S^\leq$-module, and their intersection is the $S_0$ module $M_0$.
If $f\in S_d$, then we define $M_f$ to be the graded $S_f$-module whose elements of degree $n$ are the $z/f^k$ for $z\in M_{n+kd}$ ($k\geq0$);
we denote by $M_{(f)}$ the set of elements of degree zero of $M_f$, and this is an $S_{(f)}$-module, and we will write $M_f^\geq$ and $M_f^\leq$ to mean $(M_f)^\geq$ and $(M_f)^\leq$ (respectively).
If $d>0$, then we see, as above, that
\[
\label{II.8.2.1.5}
  (M^\geq)_f = M_f
\tag{8.2.1.5}
\]
and
\[
\label{II.8.2.1.6}
  (M^\geq)_{(f)} = (M_f^\geq)_0 = M_{(f)}.
\tag{8.2.1.6}
\]
\end{env}

\begin{env}[8.2.2]
\label{II.8.2.2}
Let $\bb{z}$ be an indeterminate, we we will call the \emph{homogenisation variable}.
If $S$ is a graded ring (in positive or negative degrees), then the polynomial algebra\footnote{This should not be confused with the use of the notation $\widehat{S}$ to denote the completed separation of a ring.}
\[
\label{II.8.2.2.1}
  \widehat{S} = S[\bb{z}]
\tag{8.2.2.1}
\]
\oldpage[II]{158}
is a graded $S$-algebra, where we define the degree of $f\bb{z}^n$ ($n\geq0$), with $f$ homogeneous, as
\[
\label{II.8.2.2.2}
  \deg(f\bb{z}^n) = n+\deg f.
\tag{8.2.2.2}
\]
\end{env}

\begin{lemma}[8.2.3]
\label{II.8.2.3}
\begin{enumerate}
  \item[\rm{(i)}] There are canonical isomorphisms of (non-graded) rings
    \[
    \label{II.8.2.3.1}
      \widehat{S}_{(\bb{z})}
      \xrightarrow{\sim}
      \widehat{S}/(\bb{z}-1)\widehat{S}
      \xrightarrow{\sim}
      S.
    \tag{8.2.3.1}
    \]
  \item[\rm{(ii)}] There is a canonical isomorphism of (non-graded) rings
    \[
    \label{II.8.2.3.2}
      \widehat{S}_{(f)} \xrightarrow{\sim} S_f^\leq
    \tag{8.2.3.2}
    \]
    for all $f\in S_d$ with $d>0$.
\end{enumerate}
\end{lemma}

\begin{proof}
The first of the isomorphisms in \sref{II.8.2.3.1} was defined in \sref{II.2.2.5}, and the second is trivial;
the isomorphism $\widehat{S}_{(\bb{z})} \xrightarrow{\sim} S$ thus defined thus gives a correspondence between $x\bb{z}^n/\bb{z}^{n+k}$ (where $\deg(x) = k$ for $k\geq -n$) and the element $x$.
The homomorphism \sref{II.8.2.3.2} gives a correspondence between $x\bb{z}^n/f^k$ (where $\deg(x) = kd-n$) and the element $x/f^k$ of degree $-n$ in $S_f^\leq$, and it is again clear that this does indeed give an isomorphism.
\end{proof}

\begin{env}[8.2.4]
\label{II.8.2.4}
Let $M$ be a graded $S$-module.
It is clear that the $S$-module
\[
\label{II.8.2.4.1}
  \widehat{M} = M \otimes_S \widehat{S} = M \otimes_S S[\bb{z}]
\tag{8.2.4.1}
\]
is the direct sum of the $S$-modules $M\otimes S\bb{z}^n$, and thus of the abelian groups $M_k\otimes S\bb{z}^n$ ($k\in\bb{Z}$, $n\geq0$);
we define on $\widehat{M}$ the structure of a graded $\widehat{S}$-module by setting
\[
\label{II.8.2.4.2}
  \deg(x\otimes\bb{z}^n) = n+\deg x
\tag{8.2.4.2}
\]
for all homogeneous $x$ in $M$.
We leave it to the reader to prove the analogue of \sref{II.8.2.3}:
\end{env}

\begin{lemma}[8.2.5]
\label{II.8.2.5}
\begin{enumerate}
  \item[\rm{(i)}] There is a canonical di-isomorphism of (non-graded) modules
    \[
    \label{II.8.2.5.1}
      \widehat{M}_{(\bb{z})} \xrightarrow{\sim} M.
    \tag{8.2.5.1}
    \]
  \item[\rm{(ii)}] For all $f\in S_d$ ($d>0$), there is a di-isomorphism of (non-graded) modules
    \[
    \label{II.8.2.5.2}
      \widehat{M}_{(f)} \xrightarrow{\sim} M_f^\leq.
    \tag{8.2.5.2}
    \]
\end{enumerate}
\end{lemma}

\begin{env}[8.2.6]
\label{II.8.2.6}
Let $S$ be a \emph{positively}-graded ring, and consider the decreasing sequence of graded ideals of $S$
\[
\label{II.8.2.6.1}
  S_{[n]} = \bigoplus_{m\geq n} S_m
  \qquad\mbox{($n\geq0$)}
\tag{8.2.6.1}
\]
(so, in particular, we have $S_{[0]}=S$ and $S_{[1]}=S_+$).
Since it is evident that $S_{[m]}S_{[n]} \subset S_{[m+n]}$, we can define a \emph{graded ring} $S^\natural$ by setting
\[
\label{II.8.2.6.2}
  S^\natural = \bigoplus_{n\geq0} S_n^\natural
  \quad
  \text{with}
  \quad
  S_n^\natural = S_{[n]}.
\tag{8.2.6.2}
\]
$S_0^\natural$ is then the ring $S$ considered as a \emph{non-graded} ring, and $S^\natural$ is thus an $S_0^\natural$-algebra.
For every homogeneous element $f\in S_d$ ($d>0$), we denote by $f^\natural$ the element $f$ considered as belonging to $S_{[d]} = S_d^\natural$.
With this notation:
\end{env}

\oldpage[II]{159}
\begin{lemma}[8.2.7]
\label{II.8.2.7}
Let $S$ be a positively-graded ring, and $f$ a homogeneous element of $S_d$ ($d>0$).
There are canonical ring isomorphisms
\[
\label{II.8.2.7.1}
  S_f \xrightarrow{\sim} \bigoplus_{n\in\bb{Z}} S(n)_{(f)}
\tag{8.2.7.1}
\]
\[
\label{II.8.2.7.2}
  (S_f^\geq)_{f/1} \xrightarrow{\sim} S_f
\tag{8.2.7.2}
\]
\[
\label{II.8.2.7.3}
  S_{(f^\natural)}^\natural \xrightarrow{\sim} S_f^\geq
\tag{8.2.7.3}
\]
where the first two are isomorphisms of graded rings.
\end{lemma}

\begin{proof}
It is immediate, by definition, that we have $(S_f)_n = (S(n)_f)_0$, whence the isomorphism in \sref{II.8.2.7.1}, which is exactly the identity.
Next, since $f/1$ is invertible in $S_f$, there is a canonical isomorphism $S_f \xrightarrow{\sim} (S_f^\geq)_{f/1} = (S_f)_{f/1}$, by \sref{II.8.2.1.2} applied to $S_f$;
the inverse isomorphism is, by definition, the isomorphism in \sref{II.8.2.7.2}.
Finally, if $x = \sum_{m\geq n}y_m$ is an element of $S_{[n]}$ with $n=kd$, then the element $x/(f^\natural)^k$ corresponds to the element $\sum_m y_m/f^k$ of $S_f^\geq$, and we can quickly verify that this defines an isomorphism \sref{II.8.2.7.3}.
\end{proof}

\begin{env}[8.2.8]
\label{II.8.2.8}
If $M$ is a graded $S$-module, then we similarly define, for all $n\in\bb{Z}$,
\[
\label{II.8.2.8.1}
  M_{[n]} = \bigoplus_{m\geq n} M_m
\tag{8.2.8.1}
\]
and, since $S_{[m]}M_{[n]} \subset M_{[m+n]}$ ($m\geq0$), we can define a graded $S^\natural$-module $M^\natural$ by setting
\[
\label{II.8.2.8.2}
  M^\natural = \bigoplus_{n\in\bb{Z}}
  \quad
  \text{with}
  \quad
  M_n^\natural = M_{[n]}.
\tag{8.2.8.2}
\]
\end{env}

We leave to the reader the proof of:
\begin{lemma}[8.2.9]
\label{II.8.2.9}
With the notation of \sref{II.8.2.7} and \sref{II.8.2.8}, there are canonical di-isomorphisms of modules
\[
\label{II.8.2.9.1}
  M_f \xrightarrow{\sim} \bigoplus_{n\in\bb{Z}} M(n)_{(f)}
\tag{8.2.9.1}
\]
\[
\label{II.8.2.9.2}
  (M_f^\geq)_{f/1} \xrightarrow{\sim} M_f
\tag{8.2.9.2}
\]
\[
\label{II.8.2.9.3}
  M_{(f^\natural)}^\natural \xrightarrow{\sim} M_f^\geq
\tag{8.2.9.3}
\]
where the first two are di-isomorphisms of graded modules.
\end{lemma}

\begin{lemma}[8.2.10]
\label{II.8.2.10}
Let $S$ be a positively-graded ring.
\begin{enumerate}
  \item[\rm{(i)}] For $S^\natural$ to be an $S_0^\natural$-algebra of finite type (resp. a Noetherian $S_0^\natural$-algebra), it is necessary and sufficient for $S$ to be an $S_0^\natural$-algebra of finite type (resp. a Noetherian $S_0^\natural$-algebra).
  \item[\rm{(ii)}] For $S_{n+1}^\natural = S_1^\natural S_n^\natural$ ($n\geq n_0$), it is necessary and sufficient for $S_{n+1}=S_1S_n$ ($n\geq n_0$).
  \item[\rm{(iii)}] For $S_n^\natural = S_1^\natural$ ($n\geq n_0$), it is necessary and sufficient for $S_n=S_1^n$ ($n\geq n_0$).
  \item[\rm{(iv)}] If $(f_\alpha)$ is a set of homogeneous elements of $S_+$ such that $S_+$ is the radical in $S_+$ of the ideal of $S_+$ generated by the $f_\alpha$, then $S_+^\natural$ is the radical in $S_+^\natural$ of the ideal of $S_+^\natural$ generated by the $f_\alpha^\natural$.
\end{enumerate}
\end{lemma}

\begin{proof}
\begin{enumerate}
  \item[\rm{(i)}] If $S^\natural$ is an $S_0^\natural$-algebra of finite type, then $S_+=S_1^\natural$ is a module of finite type over $S=S_0^\natural$, by \sref{II.2.1.6}[i], and so $S$ is an $S_0$-algebra of finite type \sref{II.2.1.4};
    if $S^\natural$ is a Noetherian ring, then so too is $S_0^\natural=S$ \sref{II.2.1.5}.
    Conversely, if $S$ is an $S_0$-algebra
\oldpage[II]{160}
    of finite type, then we know \sref{II.2.1.6}[ii] that there exist $h>0$ and $m_0>0$ such that $S_{n+h}=S_hS_n$ for $n\geq m_0$;
    we can clearly assume that $m_0\geq h$.
    Furthermore, the $S_m$ are $S_0$-modules of finite type \sref{II.2.1.6}[i].
    So, if $n\geq m_0+h$, then $S_n^\natural = S_hS_{n-h}^\natural = S_h^\natural S_{n-h}^\natural$;
    and if $m<m_0+h$ then, letting $E = S_{m_0}+\ldots+S_{m_0+h-1}$, we have that
    \[
      S_m^\natural = S_m + \ldots + S_{m_0+h-1} + S_hE + S_h^2E + \ldots.
    \]
    For $1\leq m\leq m_0$, let $G_m$ be the union of the finite systems of generators of the $S_0$-modules $S_i$ for $m\leq i\leq m_0+h-1$, thought of as a subset of $S_{[m]}$.
    For $m_0+1\leq m\leq m_0+h-1$, let $G_m$ be the union of the finite system of generators of the $S_0$-modules $S_i$ for $m\leq i\leq m_0+h-1$ and of $S_hE$, thought of as a subset of $S_{[m]}$.
    It is clear that $S_m^\natural=S_0^\natural G_m$ for $1\leq m\leq m_0+h-1$, and thus the union $G$ of the $G_m$ for $1\leq m\leq m_0+h-1$ is a system of generators of the $S_0^\natural$-algebra $S^\natural$.
    We thus conclude that, if $S=S_0^\natural$ is a Noetherian ring, then so too is $S^\natural$.
  \item[\rm{(ii)}] It is clear that, if $S_{n+1}=S_1S_n$ for $n\geq n_0$, then $S_{n+1}^\natural=S_1S_n^\natural$, and \emph{a fortiori} $S_{n+1}^\natural=S_1^\natural S_n^\natural$ for $n\geq n_0$.
    Conversely, this last equality can be written as
    \[
      S_{n+1} + S_{n+2} + \ldots
      =
      (S_1 + S_2 + \ldots)(S_n + S_{n+1} + \ldots)
    \]
    and comparing terms of degree $n+1$ (in $S$) on both sides gives that $S_{n+1}=S_1S_n$.
  \item[\rm{(iii)}] If $S_n=S_1^n$ for $n\geq n_0$, then $S_n^\natural=S_1^n+S_1^{n+1}+\ldots$;
    since $S_1^\natural$ contains $S_1+S_1^2+\ldots$, we have that $S_n^\natural\subset S_1^{\natural n}$, and thus $S_n^\natural=S_1^{\natural n}$ for $n\geq n_0$.
    Conversely, the only terms of $S_1^{\natural n}=(S_1+S_2+\ldots)^n$ that are of degree $n$ in $S$ are those of $S_1^n$;
    the equality $S_n^\natural=S_1^{\natural n}$ thus implies that $S_n=S_1^n$.
  \item[\rm{(iv)}] It suffices to show that, if an element $g\in S_{k+h}$ is considered as an element of $S_k^\natural$ ($k>0$, $h\geq0$), then there exists an integer $n>0$ such that $g^n$ is a linear combination (in $S_{kn}^\natural$) of the $f_\alpha^\natural$ with coefficients in $S^\natural$.
    By hypothesis, there exists an integer $m_0$ such that, for $m\geq m_0$, we have, \emph{in $S$}, that $g^m = \sum_\alpha c_{\alpha m}f_\alpha$, where the indices $\alpha$ here are \emph{independent of $m$};
    furthermore, we can clearly assume that the $c_{\alpha m}$ are homogeneous, with
    \[
      \deg(c_{\alpha m}) = m(k+h)-\deg f_\alpha
    \]
    in $S$.
    So take $m_0$ sufficiently large enough to ensure that $km_0>\deg f_\alpha$ for all the $f_\alpha$ that appear in $g^{m_0}$;
    for all $\alpha$, let $c'_{\alpha m}$ be the element $c_{\alpha m}$ considered as having degree $km-\deg f_\alpha$ \emph{in $S^\natural$};
    we then have, in $S^\natural$, that $g^m = \sum_\alpha c'_{\alpha m}f_\alpha^\natural$, which finishes the proof.
\end{enumerate}
\end{proof}

\begin{env}[8.2.11]
\label{II.8.2.11}
Consider the graded $S_0$-algebra
\[
\label{II.8.2.11.1}
  S^\natural \otimes_S S_0
  =
  S^\natural/S_+S^\natural
  =
  \bigoplus_{n\geq0} S_{[n]}/S_+S_{[n]}.
\tag{8.2.11.1}
\]

Since $S_n$ is a quotient $S_0$-module of $S_{[n]}/S_+S_{[n]}$, there is a canonical homomorphism of graded $S_0$-algebras
\[
\label{II.8.2.11.2}
  S^\natural \otimes_S S_0 \to S
\tag{8.2.11.2}
\]
which is clearly \emph{surjective}, and thus corresponds \sref{II.2.9.2} to a canonical \emph{closed immersion}
\[
\label{II.8.2.11.3}
  \Proj(S) \to \Proj(S^\natural \otimes_S S_0).
\tag{8.2.11.3}
\]
\end{env}

\oldpage[II]{161}
\begin{proposition}[8.2.12]
\label{II.8.2.12}
The canonical morphism \sref{II.8.2.11.3} is bijective.
For the homomorphism \sref{II.8.2.11.2} to be (TN)-bijective, it is necessary and sufficient for there to exist some $n_0$ such that $S_{n+1}=S_1S_n$ for $n\geq n_0$.
If this latter condition is satisfied, then \sref{II.8.2.11.3} is an isomorphism;
the converse is true whenever $S$ is Noetherian.
\end{proposition}

\begin{proof}
To prove the first claim, it suffices \sref{II.2.8.3} to show that the kernel $\mathfrak{I}$ of the homomorphism \sref{II.8.2.11.2} consists of \emph{nilpotent} elements.
But if $f\in S_{[n]}$ is an element whose class modulo $S_+S_{[n]}$ belongs to this kernel, then this implies that $f\in S_{[n+1]}$;
then $f^{n+1}$, considered as an element of $S_{[n(n+1)]}$, is also an element of $S_+S_{[n(n+1)]}$, since it can be written as $f\cdot f^n$;
so the class of $f^{n+1}$ modulo $S_+S_{[n(n+1)]}$ is zero, which proves our claim.
Since the hypothesis that $S_{n+1}=S_1S_n$ for $n\geq n_0$ is equivalent to $S_{n+1}^\natural=S_1^\natural S_n^\natural$ for $n\geq n_0$ \sref{II.8.2.10}[ii], this hypothesis is equivalent, by definition, to the fact that \sref{II.8.2.11.2} is (TN)-injective, and thus (TN)-bijective, and so \sref{II.8.2.11.3} is an isomorphism, by \sref{II.2.9.1}.
Conversely, if \sref{II.8.2.11.3} is an isomorphism, then the sheaf $\widetilde{\mathfrak{I}}$ on $\Proj(S^\natural\otimes_S S_0)$ is zero \sref{II.2.9.2}[i];
since $S^\natural\otimes_S S_0$ is Noetherian, as a quotient of $S^\natural$ \sref{II.8.2.10}[i], we conclude from \sref{II.2.7.3} that $\mathfrak{I}$ satisfies condition (TN), and so $S_{n+1}^\natural=S_1^\natural S_n^\natural$ for $n\geq n_0$, and this finishes the proof, by \sref{II.8.2.10}[ii].
\end{proof}

\begin{env}[8.2.13]
\label{II.8.2.13}
Consider now the canonical injections $(S_+)^n\to S_{[n]}$, which define an injective homomorphism of degree zero of graded rings
\[
\label{II.8.2.13.1}
  \bigoplus_{n\geq0} (S_+)^n \to S^\natural.
\tag{8.2.13.1}
\]
\end{env}

\begin{proposition}[8.2.14]
\label{II.8.2.14}
For the homomorphism \sref{II.8.2.13.1} to be a (TN)-isomorphism, it is necessary and sufficient for there to exist some $n_0$ such that $S_n=S_1^n$ for all $n\geq n_0$.
Whenever this is the case, the morphisms corresponding to \sref{II.8.2.13.1} is everywhere defined and also an isomorphism
\[
  \Proj(S^\natural) \xrightarrow{\sim} \Proj(\bigoplus_{n\geq0}(S_+)^n);
\]
the converse is true whenever $S$ is Noetherian.
\end{proposition}

\begin{proof}
The first two claims are evident, given \sref{II.8.2.10}[iii] and \sref{II.2.9.1}.
The third will follow from \sref{II.8.2.10}[i and iii] and the following lemma:
\begin{lemma}[8.2.14.1]
\label{II.8.2.14.1}
Let $T$ be a positively-graded ring that is also a $T_0$-algebra of finite type.
If the morphism corresponding to the injective homomorphism $\bigoplus_{n\geq0}T_1^n\to T$ is everywhere defined and also an isomorphism $\Proj(T)\to\Proj(\bigoplus_{n\geq0}T_1^n)$, then there exists some $n_0$ such that $T_n=T_1^n$ for $n\geq n_0$.
\end{lemma}

Let $g_i$ ($1\leq i\leq r$) be generators of the $T_0$-module $T_1$.
The hypothesis implies first of all that the $D_+(g_i)$ cover $\Proj(T)$ \sref{II.2.8.1}.
Let $(h_j)_{1\leq j\leq s}$ be a system of homogeneous elements of $T_+$, with $\deg(h_j)=n_j$, that form, with the $g_i$, a system of generators of the ideal $T_+$, or, equivalently \sref{II.2.1.3}, a system of generators of $T$ as a $T_0$-algebra;
if we set $T'=\bigoplus_{n\geq0}T_1^n$, then the element $h_j/g_i^{n_j}$ of the ring $T_{(g_i)}$ must, by hypothesis, belong to the subring $T'_{(g_i)}$, and so there exists some integer $k$ such that $T_1^k h_j\subset T_1^{k+n_j}$ for all $j$.
We thus conclude, by induction on $r$, that $T_1^k h_j^r \subset T'$ for all $r\geq1$, and, by definition of the $h_j$, we thus have that $T_1^k T\subset T'$.
Also, there exists, for all $j$, an integer $m_j$ such that $h_j^{m_j}$ belongs to the ideal of $T$ generated by the $g_i$ \sref{II.2.3.14}, so $h_j^{m_j}\in T_1 T$, and
\oldpage[II]{162}
$h_j^{m_j k}\in T_1^k T\subset T'$.
There is thus an integer $m_0\geq k$ such that $h_j^m\in T_1^{mn}$ for $m\geq m_0$.
So, if $q$ is the largest of the integers $n_j$, then $n_0=qsm_0+k$ is the required number.
Indeed, an element of $S_n$, for $n\geq n_0$, is the sum of monomials belonging to $T_1^\alpha u$, where $u$ is a product of powers of the $h_j$;
if $\alpha\geq k$, then it follows from the above that $T_1^\alpha u\subset T_1^n$;
in the other case, one of the exponents of the $h_j$ is $\geq m_0$, so $u\in T_1^\beta v$, where $\beta\geq k$ and $v$ is again a product of powers of the $h_j$;
we can then reduce to the previous case, and so we conclude that $T_1^\alpha u\subset T_1^n$ in all cases.
\end{proof}

\begin{remark}[8.2.15]
\label{II.8.2.15}
The condition $S_n=S_1^n$ for $n\geq n_0$ clearly implies that $S_{n+1}=S_1S_n$ for $n\geq n_0$, but the converse is not necessarily true, even if we assume that $S$ is Noetherian.
For example, let $K$ be a field, $A=K[\bb{x}]$, and $B=K[\bb{y}]/\bb{y}^2K[\bb{y}]$, where $\bb{x}$ and $\bb{y}$ are indeterminates, with $\bb{x}$ taken to have degree 1 and $\bb{y}$ to have degree 2, and let $S=A\otimes_K B$, so that $S$ is a graded algebra over $K$ that has a basis given by the elements $1$, $\bb{x}^n$ ($n\geq1$), and $\bb{x}^n\bb{y}$ ($n\geq0$).
It is immediate that $S_{n+1}=S_1S_n$ for $n\geq2$, but $S_1^n=K\bb{x}^n$ while $S_n=K\bb{x}^n+K\bb{x}^n\bb{y}$ for $n\geq2$.
\end{remark}


\subsection{Based cones}
\label{subsection:II.8.3}

\begin{env}[8.3.1]
\label{II.8.3.1}
Let $Y$ be a prescheme;
in all of this section, we will consider only \emph{$Y$-preschemes} and \emph{$Y$-morphisms}.
Let $\sh{S}$ be a quasi-coherent \emph{positively}-graded $\sh{O}_Y$-algebra;
\emph{we further assume that $\sh{S}_0=\sh{O}_Y$}.
Following the notation introduced in \sref{II.8.2.2}, we let
\[
\label{II.8.3.1.1}
  \widehat{\sh{S}} = \sh{S}[\bb{z}] = \sh{S} \otimes_{\sh{O}_Y} \sh{O}_Y[\bb{z}]
\tag{8.3.1.1}
\]
which we consider as a positively-graded $\sh{O}_Y$-algebra by defining the degrees as in \sref{II.8.2.2.2}, so that, for every affine open subset $U$ of $Y$, we have
\[
  \Gamma(U,\widehat{\sh{S}}) = (\Gamma(U,\sh{S}))[\bb{z}].
\]
In what follows, we write
\[
\label{II.8.3.1.2}
  X = \Proj(\sh{S}),
  \quad
  C = \Spec(\sh{S}),
  \quad
  \widehat{C} = \Proj(\widehat{\sh{S}})
\tag{8.3.1.2}
\]
(where, in the definition of $C$, we consider $\sh{S}$ as a non-graded $\sh{O}_Y$-algebra), and we say that $C$ (resp. $\widehat{C}$) is the \emph{affine cone} (resp. \emph{projective cone}) defined by $\sh{S}$;
we will sometimes say ``cone'' instead of ``affine cone''.
By an abuse of language, we also say that $C$ (resp. $\widehat{C}$) is the \emph{affine cone \unsure{based at $X$}} (resp. the \emph{projective cone \unsure{based at $X$}})\footnote{\emph{[Trans.] A more literal translation of the French (\emph{c\^one projetant (affine/projectif)}) would be the \emph{projecting (affine/projective) cone}, but it seems that this terminology already exists to mean something else.}}, with the implicit understanding that the prescheme $X$ is given in the form $\Proj(\sh{S})$;
finally, we say that $\widehat{C}$ is the \emph{projective closure} of $C$ (with the data of $\sh{S}$ being implicit in the structure of $C$).
\end{env}

\begin{proposition}[8.3.2]
\label{II.8.3.2}
There exist canonical $Y$-morphisms
\[
\label{II.8.3.2.1}
  Y \xrightarrow{\varepsilon} C \xrightarrow{i} \widehat{C}
\tag{8.3.2.1}
\]
\[
\label{II.8.3.2.2}
  X \xrightarrow{j} \widehat{C}
\tag{8.3.2.2}
\]
such that $\varepsilon$ and $j$ are closed immersions, and $i$ is an affine morphism, which is a dominant open immersion, for which
\[
\label{II.8.3.2.3}
  i(C) = \widehat{C}\setminus j(X);
\tag{8.3.2.3}
\]
furthermore, $\widehat{C}$ is the smallest closed subprescheme of $\widehat{C}$ containing $i(C)$.
\end{proposition}

\begin{proof}
To define $i$, consider the open subset of $\widehat{C}$ given by
\[
\label{II.8.3.2.4}
  \widehat{C}_{\bb{z}} = \Spec(\widehat{\sh{S}}/(\bb{z}-1)\widehat{\sh{S}})
\tag{8.3.2.4}
\]
\sref{II.3.1.4}, where $\bb{z}$ is canonically identified with a section of $\sh{S}$ over $Y$.
The isomorphism $i: C\xrightarrow{\sim}\widehat{C}_{\bb{z}}$ then corresponds to the canonical isomorphism \sref{II.8.2.3.1}
\[
  \widehat{\sh{S}}/(\bb{z}-1)\widehat{\sh{S}} \xrightarrow{\sim} \sh{S}.
\]

The morphism $\varepsilon$ corresponds to the augmentation homomorphism $\sh{S}\to\sh{S}_0=\sh{O}_Y$, which has kernel $\sh{S}_+$ \sref{II.1.2.7}, and, since the latter is surjective, $\varepsilon$ is a closed immersion \sref{II.1.4.10}.
Finally, $j$ corresponds \sref{II.3.5.1} to the surjective homomorphism of degree zero $\widehat{\sh{S}}\to\sh{S}$, which restricts to the identity on $\sh{S}$ and is zero on $\bb{z}\widehat{\sh{S}}$, which is its kernel;
$j$ is everywhere defined, and is a closed immersion, by \sref{II.3.6.2}.

To prove the other claims of \sref{II.8.3.2}, we can clearly restrict to the case where $Y=\Spec(A)$ is affine, and $\sh{S}=\widetilde{S}$, with $S$ a graded $A$-algebra, whence $\widehat{\sh{S}}=(\widehat{S})\supertilde$;
the homogeneous elements $f$ of $S_+$ can then be identified with sections of $\widehat{\sh{S}}$ over $Y$, and the open subset of $\widehat{C}$, denoted $D_+(f)$ in \sref{II.2.3.3}, can then be written as $\widehat{C}_f$ \sref{II.3.1.4};
similarly, the open subset of $C$ denoted $D(f)$ in \sref[I]{I.1.1.1} can be written as $C_f$ \sref[0]{0.5.5.2}.
With this in mind, it follows from \sref{II.2.3.14} and from the definition of $\widehat{S}$ that, in this case, the open subsets $\widehat{C}_{\bb{z}}=i(C)$ and $\widehat{C}_f$ (with $f$ homogeneous in $S_+$) form a \emph{cover} of $\widehat{C}$.
Furthermore, with this notation,
\[
\label{II.8.3.2.5}
  i^{-1}(\widehat{C}_f) = C_f;
\tag{8.3.2.5}
\]
indeed, $\widehat{C}_f\cap i(C) = \widehat{C}_f\cap\widehat{C}_{\bb{z}} = \widehat{C}_{f\bb{z}} = \Spec(\widehat{S}_{(f\bb{z})})$.
But, if $d=\deg(f)$, then $\widehat{S}_{(f\bb{z})}$ is canonically isomorphic to $(\widehat{S}_{(\bb{z})})_{f/\bb{z}^d}$ \sref{II.2.2.2}, and it follows from the definition of the isomorphism in \sref{II.8.2.3.1} that the image of $(\widehat{S}_{(\bb{z})})_{f/\bb{z}^d}$ under the corresponding isomorphism of rings of fractions is exactly $S_f$.
Since $C_f=\Spec(S_f)$, this proves \sref{II.8.3.2.5} and shows, at the same time, that the morphism $i$ is affine;
furthermore, the restriction of $i$ to $C_f$, thought of as a morphism to $\widehat{C}_f$, corresponds \sref[I]{I.1.7.3} to the canonical homomorphism $\widehat{S}_{(f)}\to\widehat{S}_{(f\bb{z})}$, and, by the above and \sref{II.8.2.3.2}, we can claim the following result:
\begin{env}[8.3.2.6]
\label{II.8.3.2.6}
If $Y=\Spec(A)$ is affine, and $\sh{S}=\widetilde{S}$, then, for every homogeneous $f$ in $S_+$, $\widehat{C}_f$ is canonically identified with $\Spec(S_f^\leq)$, and the morphism $C_f\to\widehat{C}_f$ given by restricting $i$ then corresponds to the canonical injection $S_f^\leq\to S_f$.
\end{env}

Now note that (for $Y$ affine) the complement of $\widehat{C}_{\bb{z}}$ in $\widehat{C}=\Proj(\widehat{S})$
\oldpage[II]{164}
is, by definition, the set of graded prime ideals of $\widehat{S}$ containing $\bb{z}$, which is exactly $j(X)$, by definition of $j$, which proves \sref{II.8.3.2.3}.

Finally, to prove the last claim of \sref{II.8.3.2}, we can assume that $Y$ is affine.
With the above notation, note that, in the ring $\widehat{S}$, $\bb{z}$ is not a zero divisor;
since $i(C)=\widehat{C}$, it suffices to prove the following lemma:
\begin{lemma}[8.3.2.7]
\label{II.8.3.2.7}
Let $T$ be a positively-graded ring, $Z=\Proj(T)$, and $g$ a homogeneous element of $T$ of degree $d>0$.
If $g$ is not a zero divisor in $T$, then $Z$ is the smallest closed subprescheme of $Z$ that contains $Z_g=D_+(g)$.
\end{lemma}

By \sref[I]{I.4.1.9}, the question is local on $Z$;
for every homogeneous element $h\in T_e$ ($e>0$), it thus suffices to prove that $Z_h$ is the smallest closed subprescheme of $Z_h$ that contains $Z_{gh}$;
it follows from the definitions and from \sref[I]{I.4.3.2} that this condition is equivalent to asking for the canonical homomorphism $T_{(h)}\to T_{(gh)}$ to be \emph{injective}.
But this homomorphism can be identified with the canonical homomorphism $T_{(h)}\to(T_{(h)})_{g^e/h^d}$ \sref{II.2.2.3}.
But since $g^e$ is not a zero divisor in $T$, $g^e/h^d$ is not a zero divisor in $T_h$ (nor \emph{a fortiori} in $T_{(h)}$), since the fact that $(g^e/h^d)(t/h^m)=0$ (for $t\in T$ and $m>0$) implies the existence of some $n>0$ such that $h^ng^et=0$, whence $h^nt=0$, and thus $t/h^m=0$ in $T_h$.
This thus finishes the proof \sref[0]{0.1.2.2}.
\end{proof}

\begin{env}[8.3.3]
\label{II.8.3.3}
We will often identify the affine cone $C$ with the subprescheme induced by the projective cone $\widehat{C}$ on the open subset $i(C)$ by means of the open immersion $i$.
The closed subprescheme of $C$ associated to the closed immersion $\varepsilon$ is called the \unsure{\emph{vertex prescheme}} of $C$;
we also say that $\varepsilon$, which is a $Y$-section of $C$, is the \unsure{\emph{vertex section}}, or the \emph{null section},  or $C$;
we can identify $Y$ with the \unsure{vertex prescheme} of $C$ by means of $\varepsilon$.
Also, $i\circ\varepsilon$ is a $Y$-section of $\widehat{C}$, and thus also a closed immersion \sref[I]{I.5.4.6}, corresponding to the canonical surjective homomorphism of degree zero $\widehat{\sh{S}} = \sh{S}[\bb{z}] \to \sh{O}_Y[\bb{z}]$ \sref{II.3.1.7}, whose kernel is $\sh{S}_+[\bb{z}] = \sh{S}_+\widehat{\sh{S}}$;
the subprescheme of $\widehat{C}$ associated to this closed immersion is also called the \unsure{\emph{vertex prescheme}} of $\widehat{C}$, and $i\circ\varepsilon$ the \unsure{\emph{vertex section}} of $\widehat{C}$;
it can be identified with $Y$ by means of $i\circ\varepsilon$.
Finally, the closed subprescheme of $\widehat{C}$ associated to $j$ is called the \emph{part at infinity} of $\widehat{C}$, and can be identified with $X$ by means of $j$.
\end{env}

\begin{env}[8.3.4]
\label{II.8.3.4}
The subpreschemes of $C$ (resp. $\widehat{C}$) induced on the \emph{open} subsets
\[
\label{II.8.3.4.1}
  E = C \setmin \varepsilon(Y),
  \qquad
  \widehat{E} = \widehat{C} \setmin i(\varepsilon(Y))
\tag{8.3.4.1}
\]
are called (by an abuse of language) the \emph{pointed affine cone} and the \emph{pointed projective cone} (respectively) defined by $\sh{S}$;
we note that, despite this nomenclature, \emph{$E$ is not necessarily affine} over $Y$, nor $\widehat{E}$ projective over $Y$ \sref{II.8.4.3}.
When we identify $C$ with $i(C)$, we thus have the underlying spaces
\[
\label{II.8.3.4.2}
  C \cup \widehat{E} = \widehat{C},
  \qquad
  C \cap \widehat{E} = E
\tag{8.3.4.2}
\]
so that $\widehat{C}$ can be considered as being obtained by \emph{gluing} the open subpreschemes $C$ and $\widehat{E}$;
furthermore, by \sref{II.8.3.2.3},
\[
\label{II.8.3.4.3}
  E = \widehat{E} \setmin j(X).
\tag{8.3.4.3}
\]

If $Y=\Spec(A)$ is affine, then, with the notation of \sref{II.8.3.2},
\[
\label{II.8.3.4.4}
  E = \bigcup C_f,
  \qquad
  \widehat{E} = \bigcup \widehat{C}_f,
  \qquad
  C_f = C \cap \widehat{C}_f
\tag{8.3.4.4}
\]
where $f$ runs over the set of homogeneous elements of $S_+$ (or only a subset $M$ of this set, with $M$ generating an ideal of $S_+$ whose radical in $S_+$ is $S_+$ itself, or, equivalently, such that the $X_f$ for $f\in M$ cover $X$ \sref{II.2.3.14}).
The gluing of $C$ and $\widehat{C}_f$ along $C_f$ is thus determined by the injection morphisms $C_f\to C$ and $C_f\to\widehat{C}_f$, which, as we have seen \sref{II.8.3.2.6}, correspond (respectively) to the canonical homomorphisms $S\to S_f$ and $S_f^\leq\to S_f$.
\end{env}

\begin{proposition}[8.3.5]
\label{II.8.3.5}
With the notation of \sref{II.8.3.1} and \sref{II.8.3.4}, the morphism associated \sref{II.3.5.1} to the canonical injection $\varphi:\sh{S} \to \widehat{\sh{S}} = \sh{S}[\bb{z}]$ is a surjective affine morphism (called the canonical retraction)
\[
\label{II.8.3.5.1}
  p:\widehat{E} \to X
\tag{8.3.5.1}
\]
such that
\[
\label{II.8.3.5.2}
  p \circ j = 1_X.
\tag{8.3.5.2}
\]
\end{proposition}

\begin{proof}
To prove the proposition, we can restrict to the case where $Y$ is affine.
Taking into account the expression in \sref{II.8.3.4.4} for $\widehat{E}$, the fact that the domain of definition $G(\varphi)$ of $p$ is equal to $\widehat{E}$ will follow from the first of the following claims:
\begin{env}[8.3.5.3]
\label{II.8.3.5.3}
If $Y=\Spec(A)$ is affine, and $\sh{S}=\widetilde{S}$, then, for all homogeneous $f\in S_+$,
\[
\label{II.8.3.5.4}
  p^{-1}(X_f) = \widehat{C}_f
\tag{8.3.5.4}
\]
and the restriction of $p$ to $\widehat{C}_f=\Spec(S_f^\leq)$, thought of as a morphism from $\widehat{C}_f$ to $X_f$, corresponds to the canonical injection $S_{(f)}\to S_f^\leq$.
If, further, $f\in S_1$, then $\widehat{C}_f$ is isomorphic to $X_f\otimes_{\bb{Z}}\bb{Z}[T]$ (where $T$ is an indeterminate).
\end{env}

Indeed, Equation~\sref{II.8.3.5.4} is exactly a particular case of \sref{II.2.8.1.1}, and the second claim is exactly the definition of $\Proj(\varphi)$ whenever $Y$ is affine \sref{II.2.8.1}.
Then Equation~\sref{II.8.3.5.2} and the fact that $p$ is surjective show that the composition $\sh{S}\to\widehat{\sh{S}}\to\sh{S}$ of the canonical homomorphisms is the identity on $\sh{S}$.
Finally, the last claim of \sref{II.8.3.5.3} follows from the fact that $S_f^\leq$ is isomorphic to $S_{(f)}[T]$ whenever $f\in S_1$ \sref{II.2.2.1}.
\end{proof}

\begin{corollary}[8.3.6]
\label{II.8.3.6}
The restriction
\[
\label{II.8.3.6.1}
  \pi: E \to X
\tag{8.3.6.1}
\]
of $p$ to $E$ is a surjective affine morphism.
If $Y$ is affine and $f$ homogeneous in $S_+$, then
\[
\label{II.8.3.6.2}
  \pi^{-1}(X_f) = C_f
\tag{8.3.6.2}
\]
and the restriction of $\pi$ to $C_f$ corresponds to the canonical injection $S_{(f)}\to S_f$.
If, further, $f\in S_1$, then $C_f$ is isomorphic to $X_f\otimes_{\bb{Z}}\bb{Z}[T,T^{-1}]$ (where $T$ is an indeterminate).
\end{corollary}

\begin{proof}
Equation~\sref{II.8.3.6.2} follows immediately from \sref{II.8.3.5.3} and \sref{II.8.3.2.5}, and shows the surjectivity of $\pi$;
we have already seen that the immersion $i$, restricted to $C_f$, corresponds
\oldpage[II]{166}
to the injection $S_f^\leq\to S_f$ \sref{II.8.3.2}.
Finally, the last claim is a consequence of the fact that, for $f\in S_1$, $S_f$ is isomorphic to $S_{(f)}[T,T^{-1}]$ \sref{II.2.2.1}.
\end{proof}

\begin{remark}[8.3.7]
\label{II.8.3.7}
Whenever $Y$ is affine, the elements of the underlying space of $E$ are the (not-necessarily-graded) prime ideals $\mathfrak{p}$ of $S$ not containing $S_+$, by definition of the immersion $\varepsilon$ \sref{II.8.3.2}.
For such an ideal $\mathfrak{p}$, the $\mathfrak{p}\cap S_n$ clearly satisfy the conditions of \sref{II.2.1.9}, and so there exists exactly one \emph{graded} prime ideal $\mathfrak{q}$ of $S$ such that $\mathfrak{q}\cap S_n=\mathfrak{p}\cap S_n$ for all $n$;
the map $\pi:E\to X$ of underlying spaces can then be understood via the equation
\[
\label{II.8.3.7.1}
  \pi(\mathfrak{p}) = \mathfrak{q}.
\tag{8.3.7.1}
\]

Indeed, to prove this equation, it suffices to consider some homogeneous $f$ in $S_+$ such that $\mathfrak{p}\in D(f)$, and to note that $\mathfrak{q}_{(f)}$ is the inverse image of $\mathfrak{p}_f$ under the injection $S_{(f)}\to S_f$.
\end{remark}

\begin{corollary}[8.3.8]
\label{II.8.3.8}
If $\sh{S}$ is generated by $\sh{S}_1$, then the morphisms $p$ and $\pi$ are of finite type;
for all $x\in X$, the fibre $p^{-1}(x)$ is isomorphic to $\Spec(\kres(x)[T])$, and the fibre $\pi^{-1}$ isomorphic to $\Spec(\kres(x)[T,T^{-1}])$
\end{corollary}

\begin{proof}
This follows immediately from \sref{II.8.3.5} and \sref{II.8.3.6} by noting that, whenever $Y$ is affine and $S$ is generated by $S_1$, the $X_f$, for $f\in S_1$, form a cover of $X$ \sref{II.2.3.14}.
\end{proof}

\begin{remark}[8.3.9]
\label{II.8.3.9}
The pointed affine cone corresponding to the graded $\sh{O}_Y$-algebra $\sh{O}_Y[T]$ (where $T$ is an indeterminate) can be identified with $G_m=\Spec(\sh{O}_Y[T,T^{-1}])$, since it is exactly $C_T$, as we have seen in \sref{II.8.3.2} (see \sref{II.8.4.4} for a more general result).
This prescheme is canonical endowed with the structure of a ``\emph{$Y$-scheme in commutative groups}''.
This idea will be explained in detail later on, but, for now, can be quickly summarised as follows.
A $Y$-scheme in groups is a $Y$-scheme $G$ endowed with two $Y$-morphisms, $p:G\times_Y G\to G$ and $s:G\to G$, that satisfy conditions formally analogous to the axioms of the composition law and the symmetry law of a group: the diagram
\[
  \xymatrix{
    G \times G \times G
      \ar[r]^{p \times 1}
      \ar[d]_{1 \times p}
  & G \times G
      \ar[d]^{p}
  \\G \times G
      \ar[r]_{p}
  & G
  }
\]
should commute (``associativity''), and there should be a condition which corresponds to the fact that, for groups, the maps
\[
  (x,y)
  \mapsto
  (x,x^{-1},y)
  \mapsto
  (x,x^{-1}y)
  \mapsto
  x(x^{-1}y)
\]
and
\[
  (x,y)
  \mapsto
  (x,x^{-1},y)
  \mapsto
  (x,yx^{-1})
  \mapsto
  (yx^{-1})x
\]
should both reduce to $(x,y)\mapsto y$;
the sequence of morphisms corresponding, for example, to the first composite map is
\[
  G \times G
  \xrightarrow{(1,s) \times 1}
  G \times G \times G
  \xrightarrow{1 \times p}
  G\times G
  \xrightarrow{p}
  G
\]
and the reader should write down the second sequence.

\oldpage[II]{167}
It is immediate \sref[I]{I.3.4.3} that the data of a structure of a $Y$-scheme in groups on a $Y$-scheme $G$ is equivalent to the data, for \emph{every} $Y$-prescheme $Z$, of a \emph{group} structure on the set $\Hom_Y(Z,G)$, where these structures should be such that, for every $Y$-morphism $Z\to Z'$, the corresponding map $\Hom_Y(Z',G)\to\Hom_Y(Z,G)$ is a group homomorphism.
In the particular case of $G_m$ that we consider here, $\Hom_Y(Z,G)$ can be identified with the set of $Z$-sections of $Z\times_Y G_m$ \sref[I]{I.3.3.14}, and thus with the set of $Z$-sections of $\Spec(\sh{O}_Z[T,T^{-1}])$;
finally, the same reasoning as in \sref[I]{I.3.3.15} shows that this set is canonically identified with the set of \emph{invertible} elements of the ring $\Gamma(Z,\sh{O}_Z)$, and the group structure on this set is the structure coming from the multiplication in the ring $\Gamma(Z,\sh{O}_Z)$.
The reader can verify that the morphisms $p$ and $s$ from above are obtained in the following way: they correspond, by \sref{II.1.2.7} and \sref{II.1.4.6}, to the homomorphisms of $\sh{O}_Y$-algebras
\begin{align*}
  \pi: &\sh{O}_Y[T,T^{-1}] \to \sh{O}_Y[T,T^{-1},T',T^{'-1}]
\\\sigma: &\sh{O}_Y[T,T^{-1}] \to \sh{O}_Y[T,T^{-1}]
\end{align*}
and are entirely defined by the data of $\pi(T)=TT'$ and $\sigma(T)=T^{-1}$.

With this in mind, $G_m$ can be considered as a ``\emph{universal domain of operators}'' for every \emph{affine cone} $C=\Spec(\sh{S})$, where $\sh{S}$ is a quasi-coherent positively-graded $\sh{O}_Y$-algebra.
This means that we can canonically define a $Y$-morphism $G_m\times_Y C\to C$ which has the formal properties of an external law of a set endowed with a group of operators;
or, again, as above for schemes in groups, we can give, for every $Y$-prescheme $Z$, an external law on $\Hom_Y(Z,C)$, having the group $\Hom_Y(Z,G_m)$ as its set of operators, with the usual axioms of sets endowed with a group of operators, and a compatibility condition with respect to the $Y$-morphisms $Z\to Z'$.
In the current case, the morphism $G_m\times_Y C\to C$ is defined by the data of a homomorphism of $\sh{O}_Y$-algebras $\sh{S} \to \sh{S}\otimes_{\sh{O}_Y}\sh{O}_Y[T,T^{-1}] = \sh{S}[T,T^{-1}]$, which associates, to each section $s_n\in\Gamma(U,\sh{S}_n)$ (where $U$ is an open subset of $Y$), the section $s_n T^n\in\Gamma(U,\sh{S}\otimes_{\sh{O}_Y}\sh{O}_Y[T,T^{-1}])$.

Conversely, suppose that we are given a quasi-coherent, \emph{a priori non-graded}, $\sh{O}_Y$-algebra, and, on $C=\Spec(\sh{S})$, a structure of a ``\emph{$Y$-scheme in sets endowed with a group of operators}'' that has the $Y$-scheme in groups $G_m$ as its domain of operators;
then we canonically obtain a \emph{grading} of $\sh{O}_Y$-algebras on $\sh{S}$.
Indeed, the data of a $Y$-morphism $G_m\times_Y C\to C$ is equivalent to that of a homomorphism of $\sh{O}_Y$-algebras $\psi:\sh{S}\to\sh{S}[T,T^{-1}]$, which can be written as $\psi=\sum_{n\in\bb{Z}}\psi_n T^n$, where the $\psi_n:\sh{S}\to\sh{S}$ are homomorphisms of $\sh{O}_Y$-modules (with $\psi_n(s)=0$ except for finitely many $n$ for every section $s\in\Gamma(U,\sh{S})$, for any open subset $U$ of $Y$).
We can then prove that the axioms of sets endowed with a group of operators imply that the $\psi_n(\sh{S})=\sh{S}_n$ define a grading (in positive or negative degree) of $\sh{O}_Y$-algebras on $\sh{S}$, with the $\psi_n$ being the corresponding projectors.
We also have the notation of a structure of an ``\emph{affine cone}'' on every affine $Y$-scheme, defined in a ``geometric'' way without any reference to any prior grading.
\oldpage[II]{168}
We will not further develop this point of view here, and we leave the work of precisely formulating the definitions and results corresponding to the information given above to the reader.
\end{remark}


\subsection{Projective closure of a vector bundle}
\label{subsection:II.8.4}

\begin{env}[8.4.1]
\label{II.8.4.1}
Let $Y$ be a prescheme, and $\sh{E}$ a quasi-coherent $\sh{O}_Y$-module.
If we take $\sh{S}$ to be the graded $\sh{O}_Y$-algebra $\bb{S}_{\sh{O}_Y}(\sh{E})$, then Definition~\eref{eq:2.8.3.1.1} shows that $\widehat{\sh{S}}$ can be identified with $\bb{S}_{\sh{O}_Y}(\sh{E}\oplus\sh{O}_Y)$.
With the affine cone $\Spec(\sh{S})$ defined by $\sh{S}$ being, by definition, $\bb{V}(\sh{E})$, and $\Proj(\sh{S})$ being, by definition, $\bb{P}(\sh{E})$, we see that:
\end{env}
\begin{proposition}[8.4.2]
\label{II.8.4.2}
The projective closure of a vector bundle $\bb{V}(\sh{E})$ on $Y$ is canonically isomorphic to $\bb{P}(\sh{E}\oplus\sh{O}_Y)$, and the part at infinity of the latter is canonically isomorphic to $\bb{P}(\sh{E})$.
\end{proposition}

\begin{remark}[8.4.3]
\label{II.8.4.3}
Take, for example, $\sh{E}=\sh{O}_Y^r$ with $r\geq2$;
then the pointed cones $E$ and $\widehat{E}$ defined by $\sh{S}$ are nether affine nor projective on $Y$ if $Y\neq\emp$.
The second claim is immediate, because $\widehat{C}=\bb{P}(\sh{O}_Y^{r+1})$ is projective on $Y$, and the underlying spaces of $E$ and $\widehat{E}$ are non-closed open subsets of $\widehat{C}$, and so the canonical immersions $E\to\widehat{C}$ and $\widehat{E}\to\widehat{C}$ are not projective \sref{II.5.5.3}, and we conclude by appealing to \sref{II.5.5.5}[v].
Now, supposing, for example, that $Y=\Spec(A)$ is affine, and $r=2$, then $C=\Spec(A[T_1,T_2])$, and $E$ is then the prescheme induced by $C$ on the open subset $D(T_1)\cup D(T_2)$;
but we have already seen that the latter is not affine \sref[I]{I.5.5.11};
\emph{a fortiori} $\widehat{E}$ cannot be affine, since $E$ is the open subset where the section $\bb{z}$ over $\widehat{E}$ does not vanish \sref{II.8.3.2}.

However:
\end{remark}
\begin{proposition}[8.4.4]
\label{II.8.4.4}
If $\sh{L}$ is an invertible $\sh{O}_Y$-module, then there are canonical isomorphisms for both the pointed cones $E$ and $\widehat{E}$ corresponding to $C=\bb{V}(\sh{L})$:
\[
\label{eq:2.8.4.4.1}
  \Spec\left(
    \bigoplus_{n\in\bb{Z}}\sh{L}^{\otimes n}
  \right)
  \xrightarrow{\sim}
  E
\tag{8.4.4.1}
\]
\[
\label{eq:2.8.4.4.2}
  \bb{V}(\sh{L}^{-1})
  \xrightarrow{\sim}
  \widehat{E}.
\tag{8.4.4.2}
\]

Furthermore, there exists a canonical isomorphism from the projective closure of $\bb{V}(\sh{L})$ to the projective closure of $\bb{V}(\sh{L}^{-1})$ that sends the null section (resp. the part at infinity) of the former to the part at infinity (resp. the null section) of the second.
\end{proposition}

\begin{proof}
We have here that $\sh{S}=\bigoplus_{n\geq0}\sh{L}^{\otimes n}$;
the canonical injection
\[
  \sh{S} \to \bigoplus_{n\in\bb{Z}} \sh{L}^{\otimes n}
\]
defines a canonical dominant morphism
\[
\label{eq:2.8.4.4.3}
  \Spec\left(
    \bigoplus_{n\in\bb{Z}} \sh{L}^{\otimes n}
  \right)
  \to
  \bb{V}(\sh{L})
  =
  \Spec\left(
    \bigoplus_{n\geq0} \sh{L}^{\otimes n}
  \right)
\tag{8.4.4.3}
\]
and it suffices to prove that this morphism is an isomorphism from the scheme $\Spec(\bigoplus_{n\in\bb{Z}} \sh{L}^{\otimes n})$ to $E$.
Since the questions is local on $Y$, we can assume that $Y=\Spec(A)$ is affine
\oldpage[II]{169}
and that $\sh{L}=\sh{O}_Y$, and so $\sh{S}=(A[T])\supertilde$ and $\bigoplus_{n\in\bb{Z}}\sh{L}^{\times n} = (A[T,T^{-1}])\supertilde$.
But $A[T,T^{-1}]$ is the ring of fractions $A[T]_T$ of $A[T]$, and thus \eref{eq:2.8.4.4.3} identifies \unsure{$\bigoplus_{n\in\bb{Z}}\sh{L}^{\otimes n}$} with the prescheme induced by $C=\bb{V}(\sh{L})$ on the open subset $D(T)$;
the complement $V(T)$ of this open subset in $C$ is the underlying space of the closed subprescheme of $C$ defined by the ideal $TA[T]$, which is exactly the null section of $C$, and so $E=D(T)$.

The isomorphism in \eref{eq:2.8.4.4.2} will be a consequence of the last claim, since $\bb{V}(\sh{L}^{-1})$ is the complement of the part at infinity of its projective closure, and $\widehat{E}$ is the complement of the null section of the projective closure $C=\bb{V}(\sh{L})$.
But these projective closures are $\bb{P}(\sh{L}^{-1}\oplus\sh{O}_Y)$ and $\bb{P}(\sh{L}\oplus\sh{O}_Y)$ (respectively);
but we can write $\sh{L}\oplus\sh{O}_Y = \sh{L}\otimes(\sh{L}^{-1}\oplus\sh{O}_Y)$.
The existence of the desired canonical isomorphism then follows from \sref{II.4.1.4}, and everything reduces to showing that this isomorphism swaps the null sections and the parts at infinity.
For this, we can reduce to the case where $Y=\Spec(A)$ is affine, $L=Ac$, and $L^{-1}=Ac'$, with the canonical isomorphism $L\otimes L^{-1}\to A$ sending $c\otimes c'$ to the element $1$ of $A$.
Then $\bb{S}(L\oplus A)$ is the tensor product of $A[\bb{z}]$ with $\bigoplus_{n\geq0}Ac^{\otimes n}$, and $\bb{S}(L^{-1}\oplus A)$ is the tensor product of $A[\bb{z}]$ with $\bigoplus_{n\geq0}Ac^{'\otimes n}$, and the isomorphism defined in \sref{II.4.1.4} sends $\bb{z}^h\otimes c^{'\otimes(n-h)}$ to the element $\bb{z}^{n-h}\otimes c^{\otimes h}$.
But, in $\bb{P}(\sh{L}^{-1}\oplus\sh{O}_Y)$, the part at infinity is the set of points where the section $\bb{z}$ vanishes, and the null section is the section of points where the section $c'$ vanishes;
since we have analogous definitions for $\bb{P}(\sh{L}\oplus\sh{O}_Y)$, the conclusion follows immediately from the above explanation.
\end{proof}


\subsection{Functorial behaviour}
\label{subsection:II.8.5}

\begin{env}[8.5.1]
\label{II.8.5.1}
Let $Y$ and $Y'$ be prescheme, $q:Y'\to Y$ a morphism, and $\sh{S}$ (resp. $\sh{S}'$) a \emph{positively}-graded quasi-coherent $\sh{O}_Y$-algebra (resp. \emph{positively}-graded quasi-coherent $\sh{O}_{Y'}$-algebra).
Consider a $q$-morphism of graded algebras
\[
\label{eq:2.8.5.1.1}
  \varphi: \sh{S} \to \sh{S}'.
\tag{8.5.1.1}
\]

We know \sref{II.1.5.6} that this corresponds, canonically, to a morphism
\[
  \Phi = \Spec(\varphi): \Spec(\sh{S}') \to \Spec(\sh{S})
\]
such that the diagram
\[
\label{eq:2.8.5.1.2}
  \xymatrix{
    C'
      \ar[r]^{\Phi}
      \ar[d]
  & C
      \ar[d]
  \\Y'
      \ar[r]_{q}
  & Y
  }
\tag{8.5.1.2}
\]
commutes, where we write $C=\Spec(\sh{S})$ and $C'=\Spec(\sh{S}')$.
\emph{Suppose, further, that $\sh{S}_0=\sh{O}_Y$ and $\sh{S}'_0=\sh{O}_{Y'}$};
let $\varepsilon:Y\to C$ and $\varepsilon:Y'\to C'$ be the canonical immersions \sref{II.8.3.2};
we then have a commutative diagram
\[
\label{eq:2.8.5.1.3}
  \xymatrix{
    Y'
      \ar[r]^{q}
      \ar[d]_{\varepsilon'}
  & Y
      \ar[d]^{\varepsilon}
  \\C'
      \ar[r]_{\Phi}
  & C
  }
\tag{8.5.1.3}
\]
\oldpage[II]{170}
which corresponds to the diagram
\[
  \xymatrix{
    \sh{S}
      \ar[r]^{\varphi}
      \ar[d]
  & \sh{S}'
      \ar[d]
  \\\sh{O}_Y
      \ar[r]
  & \sh{O}_{Y'}
  }
\]
where the vertical arrows are the augmentation homomorphisms, and so the commutativity follows from the hypothesis that $\varphi$ is assumed to be a homomorphism of \emph{graded} algebras.
\end{env}

\begin{proposition}[8.5.2]
\label{II.8.5.2}
If $E$ (resp. $E'$) is the pointed affine cone defined by $\sh{S}$ (resp. $\sh{S}'$), then $\Phi^{-1}(E)\subset E'$;
if, further, $\Proj(\varphi):G(\varphi)\to\Proj(\sh{S})$ is everywhere defined (or, equivalently, if $G(\varphi)=\Proj(\sh{S}')$), then $\Phi^{-1}(E)=E'$, and conversely.
\end{proposition}

\begin{proof}
The first claim follows from the commutativity of \eref{eq:2.8.5.1.3}.
To prove the second, we can restrict to the case where $Y=\Spec(A)$ and $Y'=\Spec(A')$ are affine, and $\sh{S}=\widetilde{S}$ and $\sh{S}'=\widetilde{S'}$.
For every homogeneous $f$ in $S_+$, writing $f'=\varphi(f)$, we have that $\Phi^{-1}(C_f)=C'_{f'}$ \sref[I]{I.2.2.4.1};
saying that $G(\varphi)=\Proj(S')$ implies that the radical (in $S'_+$) of the ideal generated by the $f'=\varphi(f)$ is $S'_+$ itself (\sref{II.2.8.1} and \sref{II.2.3.14}), and this is equivalent to saying that the $C'_{f'}$ cover $E'$ \eref{II.8.3.4.4}.
\end{proof}

\begin{env}[8.5.3]
\label{II.8.5.3}
The $q$-morphism $\varphi$ canonically extends to a $q$-morphism of graded algebras
\[
\label{II.8.5.3.1}
  \widehat{\varphi}: \widehat{\sh{S}} \to \widehat{\sh{S}'}
\tag{8.5.3.1}
\]
by letting $\widehat{\varphi}(\bb{z})=\bb{z}$.
This induces a morphism
\[
  \widehat{\Phi} = \Proj(\widehat{\varphi}) : G(\widehat{\varphi}) \to \widehat{C} = \Proj(\widehat{\sh{S}})
\]
such that the diagram
\[
  \xymatrix{
    G(\widehat{\varphi})
      \ar[r]^{\widehat{\Phi}}
      \ar[d]
  & \widehat{C}
      \ar[d]
  \\Y'
      \ar[r]_{q}
  & Y
  }
\]
commutes \sref{II.3.5.6}.
It follows immediately from the definitions that, if we write $i:C\to\widehat{C}$ and $i':C'\to\widehat{C'}$ to mean the canonical open immersions \sref{II.8.3.2}, then $i'(C')\subset G(\widehat{\varphi})$, and the diagram
\[
\label{eq:2.8.5.3.2}
  \xymatrix{
    C'
      \ar[r]^{\Phi}
      \ar[d]_{i}
  & C
      \ar[d]^{i'}
  \\G(\widehat{\varphi})
      \ar[r]_{\widehat{\Phi}}
  & \widehat{C}
  }
\tag{8.5.3.2}
\]
commutes.
Finally, if we let $X=\Proj(\sh{S})$ and $X'=\Proj(\sh{S}')$, and if $j:X\to\widehat{C}$ and $j':X'\to\widehat{C'}$ are the canonical closed immersions \sref{II.8.3.2}, then it follows from the definition of these immersions that $j'(G(\varphi))\subset G(\widehat{\varphi})$, and that the diagram
\oldpage[II]{171}
\[
\label{II.8.5.3.3}
  \xymatrix{
    G(\varphi)
      \ar[r]^{\Proj(\varphi)}
      \ar[d]_{j'}
  & X
      \ar[d]^{j}
  \\G(\widehat{\varphi})
      \ar[r]_{\widehat{\Phi}}
  & \widehat{C}
  }
\tag{8.5.3.3}
\]
commutes.
\end{env}

\begin{proposition}[8.5.4]
\label{II.8.5.4}
If $\widehat{E}$ (resp. $\widehat{E'}$) is the pointed projective cone defined by $\sh{S}$ (resp. by $\sh{S}'$), then $\widehat{\Phi}^{-1}(\widehat{E}) \subset \widehat{E'}$;
furthermore, if $p:\widehat{E}\to X$ and $p':\widehat{E'}\to X'$ are the canonical retractions, then $p'(\widehat{\Phi}^{-1}(\widehat{E})) \subset G(\widehat{\varphi})$, and the diagram
\[
\label{II.8.5.4.1}
  \xymatrix{
    \widehat{\Phi}^{-1}(\widehat{E})
      \ar[r]^{\widehat{\Phi}}
      \ar[d]_{p'}
  & \widehat{E}
      \ar[d]^{p}
  \\G(\varphi)
      \ar[r]_{\Proj(\varphi)}
  & X
  }
\tag{8.5.4.1}
\]
commutes.
If $\Proj(\varphi)$ is everywhere defined, then so too is $\widehat{\Phi}$, and we have that $\widehat{\Phi}^{-1}(\widehat{E}) = \widehat{E'}$
\end{proposition}

\begin{proof}
The first claim follows from the commutativity of Diagrams \sref{II.8.5.1.3} and \sref{II.8.5.3.2}, and the two following claims from the definition of the canonical retractions \sref{II.8.3.5} and the definition of $\widehat{\varphi}$.
To see that $\widehat{\Phi}$ is everywhere defined whenever $\Proj(\varphi)$ is, we can restrict to the case where $Y=\Spec(A)$ and $Y'=\Spec(A')$ are affine, and where $\sh{S}=\widetilde{S}$ and $\sh{S}'=\widetilde{S'}$;
the hypothesis is that, when $f$ runs over the set of homogeneous elements of $S_+$, the radical in $S'_+$ of the ideal generated in $S'_+$ by the $\varphi(f)$ is $S'_+$ itself;
we thus immediately conclude that the radical in $(S'[\bb{z}])_+$ of the ideal generated by $\bb{z}$ and the $\varphi(f)$ is $(S'[\bb{z}])_+$ itself, whence our claim;
this also shows that $\widehat{E'}$ is the union of the $\widehat{C'}_{\varphi(f)}$, and hence equal to $\widehat{\Phi}^{-1}(\widehat{E})$.
\end{proof}

\begin{corollary}[8.5.5]
\label{II.8.5.5}
Whenever $\Proj(\varphi)$ is everywhere defined, the inverse image under $\widehat{\Phi}$ of the underlying space of the part at infinity (resp. of the vertex prescheme) of $\widehat{C'}$ is the underlying space of the part at infinity (resp. of the vertex prescheme) of $\widehat{C}$.
\end{corollary}

\begin{proof}
This follows immediately from \sref{II.8.5.4} and \sref{II.8.5.2}, taking into account the equalities \sref{II.8.3.4.1} and \sref{II.8.3.4.2}.
\end{proof}


\subsection{A canonical isomorphism for pointed cones}
\label{subsection:II.8.6}

\begin{env}[8.6.1]
\label{II.8.6.1}
Let $Y$ be a prescheme, $\sh{S}$ a quasi-coherent positively-graded $\sh{O}_Y$-algebra \emph{such that $\sh{S}_0=\sh{O}_Y$}, and let $X$ be the $Y$-scheme $\Proj(\sh{S})$.
We are going to apply the results of \sref{subsection:II.8.5} to the case where $Y'=X$, and $q:X\to Y$ is the structure morphism;
let
\[
\label{II.8.6.1.1}
  \sh{S}_X=\bigoplus_{n\in\bb{Z}}\sh{O}_X(n)
  \tag{8.6.1.1}
\]
\oldpage[II]{172}
which is a quasi-coherent graded $\sh{O}_X$-algebra, with multiplication defined by means of the canonical homomorphisms \sref{II.3.2.6.1}
\[
  \sh{O}_X(m)\otimes_{\sh{O}_X}\sh{O}_X(n)\to\sh{O}_X(m+n)
\]
whose associativity is ensured by the commutative diagram in \sref{II.2.5.11.4}.
Let $\sh{S}'$ be the quasi-coherent positively-graded $\sh{O}_X$-subalgebra $\sh{S}_X^\geq=\bigoplus_{n\geq0}\sh{O}_X(n)$ of $\sh{S}_X$.

Finally, consider the canonical $q$-morphism
\[
\label{II.8.6.1.2}
  \alpha: \sh{S} \to \sh{S}_X^\geq
\tag{8.6.1.2}
\]
defined in \sref{II.3.3.2.3} as a homomorphism $\sh{S}\to q_*(\sh{S}_X)$, but which clearly sends $\sh{S}$ to $q_*(\sh{S}_X^\geq)$.
Write
\[
\label{II.8.6.1.3}
  C_X = \Spec(\sh{S}_X^\geq),
  \quad
  \widehat{C}_X = \Proj(\sh{S}_X^\geq[\bb{z}]),
  \quad
  X' = \Proj(\sh{S}_X^\geq)
\tag{8.6.1.3}
\]
and denote by $E_X$ and $\widehat{E}_X$ the corresponding pointed affine and pointed projective cones (respectively);
denote the canonical morphisms defined in \sref{subsection:II.8.3} by $\varepsilon_X:X\to C_X$, $i_X:C\to\widehat{C}_X$, $j_X:X'\to\widehat{C}_X$, $p_X:\widehat{E}_X\to X'$, and $\pi_X:E_X\to X'$.
\end{env}

\begin{proposition}[8.6.2]
\label{II.8.6.2}
The structure morphism $u:X'\to X$ is an \emph{isomorphism}, and the morphism $\Proj(\alpha)$ is everywhere defined and identical to $u$.
The morphism $\Proj(\widehat{\alpha}):\widehat{C}_X\to\widehat{C}$ is everywhere defined, and its restrictions to $\widehat{E}_X$ and $E_X$ are \emph{isomorphisms} to $\widehat{E}$ and $E$ (respectively).
Finally, if we identify $X'$ with $X$ via $u$, then the morphisms $p_X$ and $\pi_X$ are identified with the structure morphisms of the $X$-preschemes $\widehat{E}_X$ and $E_X$.
\end{proposition}

\begin{proof}
We can clearly restrict to the case where $Y=\Spec(A)$ is affine, and $\sh{S}=\widetilde{S}$;
then $X$ is the union of affine open subsets $X_f$, where $f$ runs over the set of homogeneous elements of $S_+$, with the ring of each $X_f$ being $S_{(f)}$.
It follows from \sref{II.8.2.7.1} that
\[
\label{II.8.6.2.1}
  \Gamma(X_f, \sh{S}_X^\geq) = S_f^\geq.
\tag{8.6.2.1}
\]

So $u^{-1}(X_f)=\Proj(S_f^\geq)$.
But if $f\in S_d$ ($d>0$), then $\Proj(S_f^\geq)$ is canonically isomorphic to $\Proj((S_f^\geq)^{(d)})$ \sref{II.2.4.7}, and we also know that $(S_f^\geq)^{(d)}=(S^{(d)})_f^\geq$ can be identified with $S_{(f)}[T]$ \sref{II.2.2.1} by the map $T\mapsto f/1$;
we thus conclude \sref{II.3.1.7} that the structure morphism $u^{-1}(X_f)\to X_f$ is an isomorphism, whence the first claim.
To prove the second, note that the restriction $u^{-1}(X_f)\cap G(\alpha)\to X=\Proj(S)$ of $\Proj(\alpha)$ corresponds to the canonical map $x\mapsto x/1$ from $S$ to $S_f^\geq$ \sref{II.2.6.2};
we thus deduce, first of all, that $G(\alpha)=X'$, and then, taking into account the fact that $u^{-1}(X_f)=(u^{-1}(X_f))_{f/1}$, that it follows from \sref{II.2.8.1.1} that the image of $u^{-1}(X_f)$ under $\Proj(\alpha)$ is contained in $X_f$, and the restriction of $\Proj(\alpha)$ to $u^{-1}(X_f)$, thought of as a morphism to $X_f=\Spec(S_{(f)})$, is indeed identical to that of $u$.
Finally, applying \sref{II.8.3.5.4} to $p_X$ instead of $p$, we see that $p_X^{-1}(u^{-1}(X_f)) = \Spec((S_f^\geq)_{f/1}^\leq)$, and this open subset is, by \sref{II.8.5.4.1}, the inverse image under $\Proj(\widehat{\alpha})$ of $p^{-1}(X_f)=\Spec(S_f^\leq)$ \sref{II.8.3.5.3}.
Taking \sref{II.8.2.3.2} into account, the restriction of $\Proj(\widehat{\alpha})$ to $p_X^{-1}(u^{-1}(X_f))$ corresponds to the isomorphism inverse to \sref{II.8.2.7.2}, restricted to $S_f^\leq$, whence the third claim;
the last claim is evident by definition.

\oldpage[II]{173}
We note also that it follows from the commutative diagram in \sref{II.8.5.3.2} that \emph{the restriction to $C_X$ of $\Proj(\widehat{\alpha})$ is exactly the morphism $\Spec(\alpha)$}.
\end{proof}

\begin{corollary}[8.6.3]
\label{II.8.6.3}
Considered as $X$-schemes, $\widehat{E}_X$ is canonically isomorphic to $\Spec(\sh{S}_X^\leq)$, and $E_X$ to $\Spec(\sh{S}_X)$.
\end{corollary}

\begin{proof}
Since we know that the morphisms $p_X$ and $\pi_X$ are affine (\sref{II.8.3.5} and \sref{II.8.3.6}), it suffices (given \sref{II.1.3.1}) to prove the corollary in the case where $Y=\Spec(A)$ is affine and $\sh{S}=\widetilde{S}$.
The first claim follows from the existence of the canonical isomorphisms \sref{II.8.2.7.2} $(S_f^\geq)_{f/1}^\leq \xrightarrow{\sim} S_f^\leq$ and from the fact that these isomorphisms are compatible with the map sending $f$ to $fg$ (where $f$ and $g$ are homogeneous in $S_+$).
Similarly, applying \sref{II.8.3.6.2} to $\pi_X$ instead of $\pi$, we see that $\pi_X^{-1}(u^{-1}(X_f)) = \Spec((S_f^\geq)_{f/1})$ for $f$ homogeneous in $S_+$, and the second claim then follows from the existence of the canonical isomorphisms \sref{II.8.2.7.2} $(S_f^\geq)_{f/1} \xrightarrow{\sim} S_f$.

We can then say that $\widehat{C}_X$, thought of as an $X$-scheme, is given by \emph{gluing} the affine $X$-schemes $C_X=\Spec(\sh{S}_X^\geq)$ and $\widehat{E}_X=\Spec(\sh{S}_X^\leq)$ over $X$, where the intersection of the two affine $X$-schemes is the open subset $E_X=\Spec(\sh{S}_X)$.
\end{proof}

\begin{corollary}[8.6.4]
\label{II.8.6.4}
Assume that $\sh{O}_X(1)$ is an invertible $\sh{O}_X$-module, and that $\sh{S}_X$ is isomorphic to $\bigoplus_{n\in\bb{Z}}(\sh{O}_X(1))^{\otimes n}$ (which will be the case, in particular, whenever $\sh{S}$ is generated by $\sh{S}_1$ (\sref{II.3.2.5} and \sref{II.3.2.7})).
Then the pointed projective cone $\widehat{E}$ can be identified with the rank-1 vector bundle $\bb{V}(\sh{O}_X(-1))$ on $X$, and the pointed affine cone $E$ with the subprescheme of this vector bundle induced on the complement of the null section.
With this identification, the canonical retraction $\widehat{E}\to X$ is identified with the structure morphism of the $X$-scheme $\bb{V}(\sh{O}_X(-1))$.
Finally, there exists a canonical $Y$-morphism $\bb{V}(\sh{O}_X(1))\to C$, whose restriction to the complement of the null section of $\bb{V}(\sh{O}_X(1))$ is an isomorphism from this complement to the pointed affine cone $E$.
\end{corollary}

\begin{proof}
If we write $\sh{L}=\sh{O}_X(1)$, then $\sh{S}_X^\geq$ is identical to $\bb{S}_{\sh{O}_X}(\sh{L})$, and so $\widehat{E}_X$ is canonically identified with $\bb{V}(\sh{L}^{-1})$, by \sref{II.8.6.3}, and $C_X$ with $\bb{V}(\sh{L})$.
The morphism $\bb{V}(\sh{L})\to C$ is the restriction of $\Proj(\widehat{\alpha})$, and the claims of the corollary are then particular cases of \sref{II.8.6.2}.
\end{proof}

We note that the inverse image under the morphism $\bb{V}(\sh{O}_X(1))\to C$ of the underlying space of the vertex prescheme of $C$ is the underlying space of the null section of $\bb{V}(\sh{O}_X(1))$ \sref{II.8.5.5};
but, in general, the corresponding subpreschemes of $C$ and of $\bb{V}(\sh{O}_X(1))$ are not isomorphic.
This problem will be studied below.


\subsection{Blowing up based cones}
\label{subsection:II.8.7}

\begin{env}[8.7.1]
\label{II.8.7.1}
Under the conditions of \sref{II.8.6.1}, we have, writing $r=\Proj(\widehat{\alpha})$, a commutative diagram
\[
\label{II.8.7.1.1}
  \xymatrix{
    X
      \ar[r]^{i_X\circ\varepsilon_X}
      \ar[d]_{q}
  & \widehat{C}_X
      \ar[d]^{r}
  \\Y
      \ar[r]^{i\circ\varepsilon}
  & \widehat{C}
  }
\tag{8.7.1.1}
\]
\oldpage[II]{174}
by \sref{II.8.5.1.3} and \sref{II.8.5.3.2};
furthermore, the restriction of $r$ to the complement $\widehat{C}_X\setmin i_X(\varepsilon_X(X))$ of the null section is an \emph{isomorphism} to the complement $\widehat{C}\setmin i(\varepsilon(Y))$ of the null section, by \sref{II.8.6.2}.
If we suppose, to simplify things, that $Y$ is affine, that $\sh{S}$ is of finite type and generated by $\sh{S}_1$, and that $X$ is projective over $Y$ and $\widehat{C}_X$ projective over $X$ \sref{II.5.5.1}, then $\widehat{C}_X$ is projective over $Y$ \sref{II.5.5.5}[ii], and \emph{a fortiori} over $\widehat{C}$ \sref{II.5.5.5}[v].
We then have a projective $Y$-morphism $r:\widehat{C}_X\to\widehat{C}$ (whose restriction to $C_X$ is a projective $Y$-morphism $C_X\to C$) that \unsure{\emph{contracts $X$ to $Y$}} and that induces an \emph{isomorphism} when we restrict to the \emph{complements of $X$ and $Y$}.
We thus have a connection between $C_X$ and $C$, analogous to that which exists between a blow-up prescheme and the original prescheme \sref{II.8.1.3}.
We will effectively show that $C_X$ can be identified with the homogeneous spectrum of a graded $\sh{O}_C$-algebra.
\end{env}

\begin{env}[8.7.2]
\label{II.8.7.2}
Keeping the notation of \sref{II.8.6.1}, consider, for all $n\geq0$, the quasi-coherent ideal
\[
\label{II.8.7.2.1}
  \sh{S}_{[n]} = \bigoplus_{m\geq n}\sh{S}_m
\tag{8.7.2.1}
\]
of the graded $\sh{O}_Y$-algebra $\sh{S}$.
It is clear that
\[
\label{II.8.7.2.2}
  \sh{S}_{[0]} = \sh{S},
  \qquad
  \sh{S}_{[n]} \subset \sh{S}_{[m]}
  \qquad\qquad
  \mbox{for $m\leq n$}
\tag{8.7.2.2}
\]
\[
\label{II.8.7.2.3}
  \sh{S}_{n} \sh{S}_{[m]} \subset \sh{S}_{[m+n]}.
\tag{8.7.2.3}
\]

Consider the $\sh{O}_C$-module associated to $\sh{S}_{[n]}$, which is a quasi-coherent ideal of $\sh{O}_C = \widetilde{\sh{S}}$ \sref{II.1.4.4}
\[
\label{II.8.7.2.4}
  \sh{I}_n = (\sh{S}_{[n]})\supertilde.
\tag{8.7.2.4}
\]

We thus deduce, from \sref{II.8.7.2.2} and \sref{II.8.7.2.3}, using \sref{II.1.4.4} and \sref{II.1.4.8.1}, the analogous formulas
\[
\label{II.8.7.2.5}
  \sh{I}_{[0]} = \sh{O}_C,
  \qquad
  \sh{I}_{[n]} \subset \sh{I}_{[m]}
  \qquad
  \text{for }m\leq n
\tag{8.7.2.5}
\]
\[
\label{II.8.7.2.6}
  \sh{I}_{n} \sh{I}_{[m]} \subset \sh{I}_{[m+n]}.
\tag{8.7.2.6}
\]

We are thus in the setting of \sref{II.8.1.1}, which leads us to introduce the quasi-coherent graded $\sh{O}_C$-algebra
\[
\label{II.8.7.2.7}
  \sh{S}^\natural
  =
  \bigoplus_{n\geq0}\sh{I}_n
  =
  \left(
    \bigoplus_{n\geq0} \sh{S}_{[n]}
  \right)\supertilde.
\tag{8.7.2.7}
\]
\end{env}

\begin{proposition}[8.7.3]
\label{II.8.7.3}
There is a canonical $C$-isomorphism
\[
\label{II.8.7.3.1}
  h: C_X \xrightarrow{\sim} \Proj(\sh{S}^\natural).
\tag{8.7.3.1}
\]
\end{proposition}

\begin{proof}
Suppose first of all that $Y=\Spec(A)$ is affine, so that $\sh{S}=\widetilde{S}$, with $S$ a positively-graded $A$-algebra, and $C=\Spec(S)$.
Definition~\sref{II.8.2.7.4} then shows, with the notation of \sref{II.8.2.6}, that $\sh{S}^\natural=(S^\natural)\supertilde$.
To define \sref{II.8.7.3.1}, consider a homogeneous element $f\in S_d$ ($d>0$) and the corresponding element $f^\natural\in S^\natural$ \sref{II.8.2.6};
the $S$-isomorphism in \sref{II.8.2.7.3} then defines a $C$-isomorphism
\[
\label{II.8.7.3.2}
  \Spec(S_f^\geq) \xrightarrow{\sim} \Spec(S_{(f^\natural)}^\natural).
\tag{8.7.3.2}
\]

\oldpage[II]{175}
But with the notation of \sref{II.8.6.2}, if $v:C_X\to X$ is the structure morphism, then it follows from \sref{II.8.6.2.1} that $v^{-1}(X_f)=\Spec(S_f^\geq)$.
We also have that $\Spec(S_{(f^\natural)}^\natural)=D_+(f^\natural)$, which means that \sref{II.8.7.3.2} defines an isomorphism $v^{-1}(X_f)\to D_+(f^\natural)$.
Furthermore, if $g\in S_e$ ($e>0$), then the diagram
\[
  \xymatrix{
    v^{-1}(X_{fg})
      \ar[r]^{\sim}
      \ar[d]
  & D_+(f^\natural g^\natural)
      \ar[d]
  \\v^{-1}(X_f)
      \ar[r]^{\sim}
  & D_+(f^\natural)
  }
\]
commutes, by definition of the isomorphism in \sref{II.8.2.7.3}.
Finally, by definition, $S_+$ is generated by the homogeneous $f$, and so it follows from \sref{II.8.2.10}[iv] and from \sref{II.2.3.14} that the $D_+(f^\natural)$ form a cover of $\Proj(S^\natural)$, and that the $v^{-1}(X_f)$ form a cover of $C_X$, since the $X_f$ form a cover of $X$;
in this case, we have thus defined the isomorphism \sref{II.8.7.3.1}.

To prove \sref{II.8.7.3} in the general case, it suffices to show that, if $U$ and $U'$ are affine open subsets of $Y$, given by rings $A$ and $A'$ (respectively), and such that $U'\subset U$, then, setting $\sh{S}|U=\widetilde{S}$ and $\sh{S}|U'=\widetilde{S'}$, the diagram
\[
\label{II.8.7.3.3}
  \xymatrix{
    C_{U'}
      \ar[r]
      \ar[d]
  & \Proj(S^{'\natural})
      \ar[d]
  \\C_U
      \ar[r]
  & \Proj(S^\natural)
  }
\tag{8.7.3.3}
\]
commutes.
But $S$ is canonically identified with $S\otimes_A A'$, and so $S^{'\natural}$ is canonically identified with
\[
  S^\natural \otimes_S S' = S^\natural \otimes_A A';
\]
thus $\Proj(S^{'\natural}) = \Proj(S^\natural)\times_U U'$ \sref{II.2.8.10};
similarly, if $X=\Proj(S)$ and $X'=\Proj(S')$, then $X'=X\times_U U'$ and $\sh{S}_{X'}=\sh{S}_X\otimes_{\sh{O}_U}U'$ \sref{II.3.5.4}, or, equivalently, $\sh{S}_{X'}=j^*(\sh{S}_X)$, where $j$ is the projection $X'\to X$.
We then \sref{II.1.5.2} have that $C_{U'} = C_U\times_X X' = C_U\times_U U'$, and the commutativity of \sref{II.8.7.3.3} is then immediate.
\end{proof}

\begin{remark}[8.7.4]
\label{II.8.7.4}
\begin{enumerate}
  \item[\rm{(i)}] The end of the proof of \sref{II.8.7.3} can be immediately generalised in the following way.
    Let $g:Y'\to Y$ be a morphism, $\sh{S}'=g^*(\sh{S})$, and $X'=\Proj(\sh{S}')$;
    then we have a commutative diagram
    \[
    \label{II.8.7.4.1}
      \xymatrix{
        C_{X'}
          \ar[r]
          \ar[d]
      & \Proj(\sh{S}^{'\natural})
          \ar[d]
      \\C_X
          \ar[r]
      & \Proj(\sh{S}^\natural)
      }
    \tag{8.7.4.1}
    \]

    Now let $\varphi:\sh{S}''\to\sh{S}$ be a homomorphism of graded $\sh{O}_Y$-algebras such that, if we write $X''=\Proj(\sh{S}'')$, then $u=\Proj(\varphi):X\to X''$ is everywhere defined;
    we also have
\oldpage[II]{176}
    a $Y$-morphism $v:C\to C''$ (with $C''=\Spec(\sh{S}'')$) such that $\sh{A}(v)=\varphi$, and, since $\varphi$ is a homomorphism of graded algebras, $\varphi$ induces a $v$-morphism of graded algebras $\psi:\sh{S}^{\prime\prime\natural}\to\sh{S}^\natural$ \sref{II.1.4.1}.
    Furthermore, it follows from \sref{II.8.2.10}[iv] and from the hypothesis on $\varphi$ that $\Proj(\psi)$ is everywhere defined.
    Finally, taking \sref{II.3.5.6.1} into account, there is a canonical $u$-morphism $\sh{S}_{X''}\to\sh{S}_X$, whence \sref{II.1.5.6} a morphism $w:C_{X''}\to C_X$.
    With this in mind, the diagram
    \[
    \label{II.8.7.4.2}
      \xymatrix{
        C_{X''}
          \ar[r]^-{\sim}
          \ar[d]_{w}
      & \Proj(\sh{S}^{\prime\prime\natural})
          \ar[d]^{\Proj(\psi)}
      \\C_X
          \ar[r]^-{\sim}
      & \Proj(\sh{S}^\natural)
      }
    \tag{8.7.4.2}
    \]
    is commutative, as we can immediately verify by restricting to the case where $Y$ is affine.
  \item[\rm{(ii)}] Note that, by \sref{II.8.7.2.5} and \sref{II.8.7.2.6}, we have $\sh{I}_1^m\subset\sh{I}_m\subset\sh{I}_1$ for all $m>0$.
    But, by definition, $\sh{I}_1=(\sh{S}_+)\supertilde$, and so $\sh{I}_1$ defines the closed subprescheme $\varepsilon(Y)$ in $C$ (\sref{II.1.4.10} and \sref{II.8.3.2});
    we thus conclude that, for all $m>0$, \emph{the support of $\sh{O}_C/\sh{I}_m$ is contained in the underlying space of the vertex prescheme $\varepsilon(Y)$};
    on the inverse image of the pointed affine cone $E$, the structure morphism $\Proj(\sh{S}^\natural)\to C$ thus restricts to an \emph{isomorphism} (by \sref{II.8.7.3} and \sref{II.8.7.1}).
    Furthermore, by canonically identifying $C$ with an open subset of $\widehat{C}$ \sref{II.8.3.3}, we can clearly extend the ideals $\sh{I}_m$ of $\sh{O}_C$ to ideals $\sh{J}_m$ of $\sh{O}_{\widehat{C}}$, by asking for it to agree with $\sh{O}_{\widehat{C}}$ on the open subset $\widehat{E}$ of $\widehat{C}$.
    If we define $\sh{T}=\bigoplus_{n\geq 0}\sh{J}_m$, which is a quasi-coherent graded $\sh{O}_{\widehat{C}}$-algebra, we can extend the isomorphism \sref{II.8.7.3.1} to a $\widehat{C}$-isomorphism
    \[
    \label{II.8.7.4.3}
      \widehat{C}_X \xrightarrow{\sim} \Proj(\sh{T}).
    \tag{8.7.4.3}
    \]

    Indeed, over $\widehat{E}$, it follows from the above that $\Proj(\sh{T})$ is canonically identified with $\widehat{E}$, and we thus define the isomorphism \sref{II.8.7.4.3} over $\widehat{E}$ by asking for it to agree with the canonical isomorphism $\widehat{E}_X\to\widehat{E}$ \sref{II.8.6.2};
    it is clear that this isomorphism and \sref{II.8.7.3.1} then agree over $\widehat{E}$.
\end{enumerate}
\end{remark}

\begin{corollary}[8.7.5]
\label{II.8.7.5}
Suppose that there exists some $n_0>0$ such that
\[
\label{II.8.7.5.1}
  \sh{S}_{n+1} = \sh{S}_1\sh{S}_n
  \qquad
  \text{for }n\geq n_0.
\tag{8.7.5.1}
\]

Then the \unsure{vertex subprescheme} of $C_X$ (isomorphic to $X$) is the inverse image under the canonical morphism $r:C_X\to C$ of the vertex subprescheme of $C$ (isomorphic to $Y$).
Conversely, if this property is true, and if we further assume that $Y$ is Noetherian and that $\sh{S}$ is of finite type, then there exists some $n_0>0$ such that \sref{II.8.7.5.1} holds true.
\end{corollary}

\begin{proof}
The first claim being local on $Y$, we can assume that $Y=\Spec(A)$ is affine, so that $\sh{S}=\widetilde{S}$, with $S$ a positively-graded $A$-algebra.
The claim then follows from \sref{II.8.2.12}, since $\Proj(S^\natural\otimes_S S_0) = C_X\times_C\varepsilon(Y)$ (by the identification in \sref{II.8.7.3.1}), or, in other words, since this prescheme is the inverse image of $\varepsilon(Y)$ in $C_X$ \sref[I]{I.4.4.1}.
The converse also follows from \sref{II.8.2.12} whenever $Y$ is Noetherian affine and $S$ is of finite type.
\oldpage[II]{177}
If $Y$ is Noetherian (but not necessarily affine) and $\sh{S}$ is of finite type, then there exists a finite cover of $Y$ by Noetherian affine open subsets $U_i$, and we then deduce from the above that, for all $i$, there exists an integer $n_i$ such that $\sh{S}_{n+1}|U_i = (\sh{S}_1|U_i)(\sh{S}_n|U_i)$ for $n\geq n_i$;
the largest of the $n_i$ then ensures that \sref{II.8.7.5.1} holds true.
\end{proof}

\begin{env}[8.7.6]
\label{II.8.7.6}
Now consider the $C$-prescheme $Z$ given by \emph{blowing up} the \emph{vertex subprescheme $\varepsilon(Y)$} in the affine cone $C$;
by Definition~\sref{II.8.1.3}, it is exactly the prescheme $\Proj(\bigoplus_{n\geq0}\sh{S}_+^n)$;
the canonical injection
\[
\label{II.8.7.6.1}
  \iota: \bigoplus_{n\geq0} \sh{S}_+^n \to \sh{S}^\natural
\tag{8.7.6.1}
\]
defines (by the identification in \sref{II.8.7.3}) a canonical dominant $C$-morphism
\[
\label{II.8.7.6.2}
  G(\iota)\to Z
  \tag{8.7.6.2}
\]
where $G(\iota)$ is an open subset of $C_X$ \sref{II.3.5.1};
note that it could be the case that $G(\iota)\neq C_X$, as shown by the example where $Y=\Spec(K)$, with $K$ a field, and $\sh{S}=\widetilde{S}$, with $S=K[\bb{y}]$, where $\bb{y}$ is an indeterminate \emph{of degree 2};
if $R_n$ denotes the set $(S_+)^n$, thought of as a subset of $S_{[n]}=S_n^\natural$, then $S_+^\natural$ is not the radical in $S_+^\natural$ of the ideal generated by the union of the $R_n$ (cf. \sref{II.2.3.14}).
\end{env}

\begin{corollary}[8.7.7]
\label{II.8.7.7}
Assume that there exists some $n_0>0$ such that
\[
\label{II.8.7.7.1}
  \sh{S}_n=\sh{S}_1^n
  \quad
  \text{for }n\geq n_0.
  \tag{8.7.7.1}
\]

Then the canonical morphism \sref{II.8.7.6.2} is everywhere defined, and is an isomorphism $C_X\xrightarrow{\sim}Z$.
Conversely, if this property is true, and if we further assume that $Y$ is Noetherian and that $\sh{S}$ is of finite type, then there exists some $n_0$ such that \sref{II.8.7.7.1} holds true.
\end{corollary}

\begin{proof}
\label{II.8.7.7}
The first claim is local on $Y$, and thus follows from \sref{II.8.2.14};
the converse follows similarly, arguing as in \sref{II.8.7.5}.
\end{proof}

\begin{remark}[8.7.8]
\label{II.8.7.8}
Since condition \sref{II.8.7.7.1} implies \sref{II.8.7.5.1}, we see that, whenever it holds true, not only can $C_X$ be identified with the prescheme given by blowing up the vertex (identified with $Y$) of the affine cone $C$, but also the vertex (identified with $X$) of $C_X$ can be identified with the closed subprescheme given by the inverse image of the vertex $Y$ of $C$.
Furthermore, hypothesis~\sref{II.8.7.7.1} implies that, on $X=\Proj(\sh{S})$, the $\sh{O}_X$-modules $\sh{O}_X(n)$ are invertible (\sref{II.3.2.5} and \sref{II.3.2.9}), and that $\sh{O}_X(n)=\sh{L}^{\otimes n}$ with $\sh{L}=\sh{O}_X(1)$ (\sref{II.3.2.7} and \sref{II.3.2.9});
by Definition~\sref{II.8.6.1.1}, $C_X$ is thus the \emph{vector bundle} $\bb{V}(\sh{L})$ on $X$, and its vertex is the \emph{null section} of this vector bundle.
\end{remark}

\subsection{Ample sheaves and contractions}
\label{subsection:II.8.8}

\begin{env}[8.8.1]
\label{II.8.8.1}
Let $Y$ be a prescheme, $f:X\to Y$ a \emph{separated} and \emph{quasi-compact} morphism, and $\sh{L}$ an invertible $\sh{O}_X$-module that is \emph{ample relative to $f$}.
Consider the positively-graded $\sh{O}_Y$-algebra
\[
\label{II.8.8.1.1}
  \sh{S} = \sh{O}_Y \oplus \bigoplus_{n\geq1}f_*(\sh{L}^{\otimes n})
\tag{8.8.1.1}
\]
\oldpage[II]{178}
which is quasi-coherent \sref[I]{I.9.2.2}[a].
There is a canonical homomorphisms of graded $\sh{O}_X$-algebras
\[
\label{II.8.8.1.2}
  \tau: f^*(\sh{S}) \to \bigoplus_{n\geq0}\sh{L}^{\otimes n}
\tag{8.8.1.2}
\]
which, in degrees $\geq1$, agrees with the canonical homomorphism $\sigma:f^*(f_*(\sh{L}^{\otimes n})) \to \sh{L}^{\otimes n}$ \sref[0]{0.4.4.3}, and is the identity in degree 0.
The hypothesis that $\sh{L}$ is $f$-ample then implies (\sref{II.4.6.3} and \sref{II.3.6.1}) that the corresponding $Y$-morphism
\[
\label{II.8.8.1.3}
  r = r_{\sh{L},\tau} : X \to P = \Proj(\sh{S})
\tag{8.8.1.3}
\]
is everywhere defined and is a \emph{dominant open immersion}, and that
\[
\label{II.8.8.1.4}
  r^*(\sh{O}_P(n)) = \sh{L}^{\otimes n}
  \qquad\qquad
  \mbox{for all $n\in\bb{Z}$.}
\tag{8.8.1.4}
\]
\end{env}

\begin{proposition}[8.8.2]
\label{II.8.8.2}
Let $C=\Spec(\sh{S})$ be the affine cone defined by $\sh{S}$;
if $\sh{L}$ is $f$-ample, then there exists a canonical $Y$-morphism
\[
\label{II.8.8.2.1}
  g : V = \bb{V}(\sh{L}) \to C
\tag{8.8.2.1}
\]
such that the diagram
\[
\label{II.8.8.2.2}
  \xymatrix{
    X
      \ar[r]^{j}
      \ar[d]_{f}
  & \bb{V}(\sh{L})
      \ar[r]^{\pi}
      \ar[d]^{g}
  & X
      \ar[d]^{f}
  \\Y
      \ar[r]^{\varepsilon}
  & C
      \ar[r]^{\psi}
  & Y
  }
  \tag{8.8.2.2}
\]
commutes, where $\psi$ and $\pi$ are the structure morphisms, and $j$ and $\varepsilon$ the canonical immersions sending $X$ and $Y$ (respectively) to the null section of $\bb{V}(\sh{L})$ and the vertex prescheme of $C$ (respectively).
Furthermore, the restriction of $g$ to $\bb{V}(\sh{L})\setmin j(X)$ is an open immersion
\[
\label{II.8.8.2.3}
  \bb{V}(\sh{L})\setmin j(X)\to E=C\setmin\varepsilon(Y)
  \tag{8.8.2.3}
\]
into the pointed affine cone $E$ corresponding to $\sh{S}$.
\end{proposition}

\begin{proof}
With the notation of \sref{II.8.8.1}, let $\sh{S}_P^\geq = \bigoplus_{n\geq0}\sh{O}_P(n)$ and $C_p=\Spec(\sh{S}_P^\geq)$.
We know \sref{II.8.6.2} that there is a canonical morphism $h=\Spec(\alpha):C_p\to C$ such that the diagram
\[
\label{II.8.8.2.4}
  \xymatrix{
    C_P
      \ar[r]
      \ar[d]_{h}
  & P
      \ar[d]^{p}
  \\C
      \ar[r]^{\psi}
  & Y
  }
\tag{8.8.2.4}
\]
commutes; furthermore, if $\varepsilon_P:P\to C_P$ is the canonical immersion, then the diagram
\[
\label{II.8.8.2.5}
  \xymatrix{
    P
      \ar[r]^{p}
      \ar[d]_{\varepsilon_P}
  & C_P
      \ar[d]^{h}
  \\Y
      \ar[r]^{\varepsilon}
  & C
  }
\tag{8.8.2.5}
\]
commutes \sref{II.8.7.1.1}, and, finally, the restriction of $H$ to the pointed affine cone $E_P$ is an \emph{isomorphism} $E_P\xrightarrow{\sim}E$ \sref{II.8.6.2}.
It follows from \sref{II.8.8.1.4} that
\[
  r^*(\sh{S}_P^\geq)=\bb{S}_{\sh{O}_X}(\sh{L})
\]
\oldpage[II]{179}
and so we have a canonical $P$-morphism $q:\bb{V}(\sh{L})\to C_P$, with the commutative diagram
\[
\label{II.8.8.2.6}
  \xymatrix{
    \bb{V}(\sh{L})
      \ar[r]^{\pi}
      \ar[d]_{q}
  & X
      \ar[d]^{r}
  \\C_P
      \ar[r]
  & P
  }
\tag{8.8.2.6}
\]
identifying $\bb{V}(\sh{L})$ with the product $C_P\times_P X$ \sref{II.1.5.2};
since $r$ is an open immersion, so too is $q$ \sref[I]{I.4.3.2}.
Furthermore, the restriction of $q$ to $\bb{V}(\sh{L})\setmin j(X)$ sends this prescheme to $E_p$, by \sref{II.8.5.2}, and the diagram
\[
\label{II.8.8.2.7}
  \xymatrix{
    X
      \ar[r]^{j}
      \ar[d]_{r}
  & \bb{V}(\sh{L})
      \ar[d]^{q}
  \\P
      \ar[r]^{\varepsilon_P}
  & C_P
  }
\tag{8.8.2.7}
\]
is commutative (since it is a particular case of \sref{II.8.5.1.3}).
The claims of \sref{II.8.8.2} immediately follow from these facts, by taking $g$ to be the composite morphism $h\circ q$.
\end{proof}

\begin{remark}[8.8.3]
\label{II.8.8.3}
Assume further that $Y$ is a \emph{Noetherian} prescheme, and that $f$ is a \emph{proper} morphism.
Since $r$ is then \emph{proper}\sref{II.5.4.4}, and thus closed, and since it is also a dominant open immersion, $r$ is necessarily an \emph{isomorphism} $X\xrightarrow{\sim}P$.
Furthermore, we will see, in Chapter~III \sref[III]{III.2.3.5.1}, that $\sh{S}$ is then necessarily an $\sh{O}_Y$-algebra \emph{of finite type}.
It then follows that $\sh{S}^\natural$ is an $\sh{S}_0^\natural$-algebra \emph{of finite type} (\sref{II.8.2.10}[i] and \sref{II.8.7.2.7});
since $C_P$ is $C$-isomorphic to $\Proj(\sh{S}^\natural)$ \sref{II.8.7.3}, we see that the morphism $h:C_P\to C$ is \emph{projective};
since the morphism $r$ is an isomorphism, so too is $q:\bb{V}(\sh{L})\to C_P$, and we thus conclude that the morphism $g:\bb{V}(\sh{L})\to C$ is \emph{projective}.
Furthermore, since the restriction of $h$ to $E_P$ is an isomorphism to $E$, and since $q$ is an isomorphism, the restriction \sref{II.8.8.2.3} of $g$ is an isomorphism $\bb{V}(\sh{L})\setmin j(X)\xrightarrow{\sim}E$.

If we further assume that $L$ is \emph{very ample} for $f$, then, as we will also see in Chapter~III \sref[III]{III.2.3.5.1}, there exists some integer $n_0>0$ such that $\sh{S}_n=\sh{S}_1^n$ for $n\geq n_0$.
We then conclude, by \sref{II.8.7.7}, that $\bb{V}(\sh{L})$ can be identified with the prescheme $Z$ given by \emph{blowing up the vertex prescheme \emph{(identified with $Y$)} in the affine cone $C$}, and that the \emph{null section} of $\bb{V}(\sh{L})$ (identified with $Y$) is the \emph{inverse image} of the vertex subprescheme $Y$ of $C$.

Some of the above results can in fact be proven even without the Noetherian hypothesis:
\end{remark}
\begin{corollary}[8.8.4]
\label{II.8.8.4}
Let $Y$ be a prescheme (resp. a quasi-compact scheme), $f:X\to Y$ a proper morphism, and $\sh{L}$ an invertible $\sh{O}_X$-module that is ample relative to $f$.
Then the morphism in \sref{II.8.8.2.1} is proper (resp. projective), and its restriction \sref{II.8.8.2.3} is an isomorphism.
\end{corollary}

\begin{proof}
To prove that $g$ is proper, we can restrict to the case where $Y$ is affine, and it then suffices to consider the case where $Y$ is a quasi-compact scheme.
The same arguments as in \sref{II.8.8.3} first of all show that $r$ is an \emph{isomorphism} $X\xrightarrow{\sim}P$;
then $q$ is also an isomorphism, and, since the restriction of $h$ to $E_P$ is an isomorphism $E_P\xrightarrow{\sim}E$, we have already seen that \sref{II.8.8.2.3} is an isomorphism.
It remains only to prove that $g$ is \emph{projective}.

Since $f$ is of finite type, by hypothesis, we can apply \sref{II.3.8.5} to the homomorphism
\oldpage[II]{180}
$\tau$ from \sref{II.8.8.1.2}:
there is an integer $d>0$ and a quasi-coherent $\sh{O}_Y$-submodule $\sh{E}$ of finite type of $\sh{S}_d$ such that, if $\sh{S}'$ is the $\sh{O}_Y$-subalgebra of $\sh{S}$ generated by $\sh{E}$, and $\tau'=\tau\circ q^*(\varphi)$ (where $\varphi$ is the canonical injection $\sh{S}'\to\sh{S}$), then $r'=r_{\sh{L},\tau'}$ is an immersion
\[
  X\to P'=\Proj(\sh{S}').
\]
Furthermore, since $\varphi$ is injective, $r'$ is also a \emph{dominant immersion} \sref{II.3.7.6};
the same argument as for $r$ then shows that $r'$ is a \emph{surjective closed immersion};
since $r'$ factors as $X\xrightarrow{r}\Proj(\sh{S})\xrightarrow{\Phi}\Proj(\sh{S}')$, where $\Phi=\Proj(\varphi)$, we thus conclude that $\Phi$ is also a \emph{surjective closed immersion}.
But this implies that $\Phi$ is an \emph{isomorphism};
we can restrict to the case where $Y=\Spec(A)$ is affine, and $\sh{S}=\widetilde{S}$ and $\sh{S}'=\widetilde{S'}$, with $S$ a graded $A$-algebra and $S'$ a graded subalgebra of $S$.
For every homogeneous element $t\in S'$, we have that $S'_{(t)}$ is a subring of $S_{(t)}$;
if we return to the definition of $\Proj(\varphi)$ \sref{II.2.8.1}, we see that it suffices to prove that, if $B'$ is a subring of a ring $B$, and if the morphism $\Spec(B)\to\Spec(B')$ corresponding to the canonical injection $B'\to B$ is a closed immersion, then this morphism is necessarily an \emph{isomorphism};
but this follows from \sref[I]{I.4.2.3}.
Furthermore, $\Phi^*(\sh{O}_{P'}(n))=\sh{O}_P(n)$ (\sref{II.3.5.2}[ii] and \sref{II.3.5.4}), and so $r^{'*}(\sh{O}_{P'}(n))$ is isomorphic to $\sh{L}^{\otimes n}$ \sref{II.4.6.3}.
Let $\sh{S}''=\sh{S}^{'(d)}$, so that \sref{II.3.1.8}[i] $X$ is canonically identified with $P''=\Proj(\sh{S}'')$, and $\sh{L}''=\sh{L}^{\otimes d}$ with $\sh{O}_{P''}(1)$ \sref{II.3.2.9}[ii].

Now, if $C''=\Spec(\sh{S}'')$, then $\sh{S}_{P''}^\geq=\bigoplus_{n\geq 0}\sh{O}_{P''}(n)$ can be identified with $\bigoplus_{n\geq 0}\sh{L}^{\prime\prime\otimes n}$, and thus $C_{P''}=\Spec(\sh{S}_{P''}^\geq)$ with $\bb{V}(\sh{L}'')$;
we also know \sref{II.8.7.3} that $C_{P''}$ is $C''$-isomorphic to $\Proj(\sh{S}^{\prime\prime\natural})$;
by the definition of $\sh{S}''$, we know that $\sh{S}^{\prime\prime\natural}$ is generated by $\sh{S}_1^{\prime\prime\natural}$, and that $\sh{S}_1^{\prime\prime\natural}$ is of finite type over $\sh{S}_0^{\prime\prime\natural}=\sh{S}''$ (\sref{II.8.2.10}[i and iii]), and so $\Proj(\sh{S}^{\prime\prime\natural})$ is \emph{projective} over $C''$ \sref{II.5.5.1}.
Consider the diagram
\[
\label{II.8.8.4.1}
  \xymatrix{
    \bb{V}(\sh{L})
      \ar[r]^-{g}
      \ar[d]_{u}
  & \Spec(\sh{S}) = C
      \ar[d]^{v}
  \\\bb{V}(\sh{L}'')
      \ar[r]^-{g''}
  & \Spec(\sh{S}'') = C''
  }
  \tag{8.8.4.1}
\]
where $g$ and $g''$ correspond, by \sref{II.1.5.6}, to the canonical $j$-morphisms
\[
  \sh{S}\to\bigoplus_{n\geq0}\sh{L}^{\otimes n}
  \quad\mbox{and}\quad
  \sh{S}''\to\bigoplus_{n\geq0}\sh{L}^{\prime\prime\otimes n}
\]
\sref{II.3.3.2.3} (see \sref{II.8.8.5} below), and $v$ and $u$ to the inclusion morphisms $\sh{S}''\to\sh{S}$ and $\bigoplus_{n\geq0}\sh{L}^{\otimes nd}\to\bigoplus_{n\geq0}\sh{L}^{\otimes n}$ (respectively);
it is immediate \sref{II.3.3.2} that this diagram is commutative.
We have just seen that $g''$ is a projective morphism;
we also know that $u$ is a \emph{finite} morphism.
Since the question is local on $X$, we can assume that $X$ is affine of ring $A$, and that $\sh{L}=\sh{O}_X$;
everything then reduces to noting that the ring $A[T]$ is a module of finite type over its subring $A[T^d]$ (with $T$ an indeterminate).
Since $Y$ is a quasi-compact scheme, and since $C''$ is affine over $Y$, we know that $C''$ is also a quasi-compact scheme,
\oldpage[II]{181}
and so $g''\circ u$ is a projective morphism \sref{II.5.5.5}[ii];
by commutativity of \sref{II.8.8.4.1}, $v\circ g$ is also projective, and, since $v$ is affine, thus separated, we finally conclude that $g$ is projective \sref{II.5.5.5}[v].
\end{proof}

\begin{env}[8.8.5]
Consider again the situation in \sref{II.8.8.1}.
We will see that the morphism $g:\bb{V}(\sh{L})\to C$ can be also be defined in a way that works for any invertible (but not necessarily ample) $\sh{O}_X$-module $\sh{L}$.
For this, consider the $f$-morphism
\[
\label{II.8.8.5.1}
  \tau^\flat:\sh{S}\to\bigoplus_{n\geq0}\sh{L}^{\otimes n}
  \tag{8.8.5.1}
\]
corresponding to the morphism $\tau$ from \sref{II.8.8.1.2}.
This induces \sref{II.1.5.6} a morphism $g':V\to C$ such that, if $\pi:V\to X$ and $\psi:C\to Y$ are the structure morphisms, the diagrams
\[
\label{II.8.8.5.2}
  \xymatrix{
    X
      \ar[d]_{f}
  & V
      \ar[l]_{\pi}
      \ar[d]^{g'}
  \\Y
  & C
      \ar[l]_{\psi}
  }
  \qquad
  \xymatrix{
    X
      \ar[r]^{j}
      \ar[d]_{f}
  & V
      \ar[d]^{g'}
  \\Y
      \ar[r]^{\varepsilon}
  & C
  }
\tag{8.8.5.2}
\]
commute (\sref{II.8.5.1.2} and \sref{II.8.5.1.3}).
We will show that (if we assume that $\sh{L}$ is $f$-ample) \emph{the morphisms $g$ and $g'$ are identical}.

Since the questions is local on $Y$, we can assume that $Y=\Spec(A)$ is affine, and (by \sref{II.8.8.1.3}) identify $X$ with an open subset of $P=\Proj(S)$, where $S=A\oplus\bigoplus_{n\geq0}\Gamma(X,\sh{L}^{\otimes n})$;
we then deduce, by \sref{II.8.8.1.4}, that $\Gamma(X,\sh{O}_P(n))=\Gamma(X,\sh{L}^{\otimes n})$ for all $n\in\bb{Z}$.
Taking into account the definition of $h=\Spec(\alpha)$, where $\alpha$ is the canonical $p$-morphism $\widetilde{S}\to\sh{S}_P^\geq$ \sref{II.8.6.1.2}, we have to show that the restriction to $X$ of $\alpha^\sharp:p^*(\widetilde{S})\to\sh{S}_P^\geq$ is identical to $\tau$.
Taking \sref[0]{0.4.4.3} into account, it suffices to show that, if we compose the canonical homomorphism $\alpha_n:S_n\to\Gamma(P,\sh{O}_P(n))$ with the restriction homomorphism $\Gamma(P,\sh{O}_P(n))\to\Gamma(X,\sh{O}_P(n))=\Gamma(X,\sh{L}^{\otimes n})$, then we obtain the identity, for all $n>0$;
but this follows immediately from the definition of the algebra $S$ and of $\alpha_n$ \sref{II.2.6.2}.
\end{env}

\begin{proposition}[8.8.6]
\label{II.8.8.6}
Assume (with the notation of \sref{II.8.8.5}) that, if we write $f=(f_0,\lambda)$, then the homomorphism $\lambda:\sh{O}_Y\to j_*(\sh{O}_X)$ is bijective;
then:
\begin{enumerate}
  \item[\rm{(i)}] if we write $g=(g_0,\mu)$, then $\mu:\sh{O}_C\to g_*(\sh{O}_V)$ is an isomorphism; and
  \item[\rm{(ii)}] if $X$ is integral (resp. locally integral and normal), then $C$ is integral (resp. normal).
\end{enumerate}
\end{proposition}

\begin{proof}
Indeed, the $f$-morphism $\tau^\flat$ is then an \emph{isomorphism}
\[
  \tau^\flat:\sh{S}=\psi_*(\sh{O}_C)\to f_*(\pi_*(\sh{O}_V))=\psi_*(g_*(\sh{O}_V))
\]
and the $Y$-morphism $g$ can be considered as that for which the homomorphism $\sh{A}(g)$ \sref{II.1.1.2} is equal to $\tau^\flat$.
To see that $\mu$ is an isomorphism of $\sh{O}_C$-modules, it suffices \sref{II.1.4.2} to see that $\sh{A}(\mu):\psi_*(\sh{O}_C)\to\psi_*(g_*(\sh{O}_V))$ is an isomorphism.
But, by Definition~\sref{II.1.1.2}, we have that $\sh{A}(\mu)=\sh{A}(g)$, whence the conclusion of (i).

To prove (ii), we can restrict to the case where $Y$ is affine, and so $\sh{S}=\widetilde{S}$, with
\oldpage[II]{182}
$S=\bigoplus_{n\geq0}\Gamma(X,\sh{L}^{\otimes n})$;
the hypothesis that $X$ is integral implies that the ring $S$ is integral \sref[I]{I.7.4.4}, and thus so too is $C$ \sref[I]{I.5.1.4}.
To show that $C$ is normal, we will use the following lemma:

\begin{lemma}[8.8.6.1]
\label{II.8.8.6.1}
Let $Z$ be a normal integral prescheme.
Then the ring $\Gamma(Z,\sh{O}_Z)$ is integral and integrally closed.
\end{lemma}

\begin{proof}
It follows from \sref[I]{I.8.2.1.1} that $\Gamma(Z,\sh{O}_Z)$ is the intersection, in the field of rational functions $R(Z)$, of the integrally closed rings $\sh{O}_z$ over all $z\in Z$.
\end{proof}

With this in mind, we first show that $V$ is \emph{locally integral} and \emph{normal};
for this, we can restrict to the case where $X=\Spec(A)$ is affine, with ring $A$ integral and integrally closed \sref{II.6.3.8}, and where $\sh{L}=\sh{O}_X$.
Since then $V=\Spec(A[T])$, and $A[T]$ is integral and integrally closed \cite[p.~99]{II-24}, this proves our claim.
For every affine open subset $U$ of $C$, $g^{-1}(U)$ is quasi-compact, since the morphism $g$ is quasi-compact;
since $V$ is locally integral, the connected components of $g^{-1}(U)$ are open integral preschemes in $g^{-1}(U)$, and thus finite in number, and, since $V$ is normal, these preschemes are also normal \sref{II.6.3.8}.
Then $\Gamma(U,\sh{O}_C)$, which is equal to $\Gamma(g^{-1}(U),\sh{O}_V)$, by (i), is the \unsure{direct sum} of finitely-many integral and integrally closed rings \sref{II.8.8.6.1}, which proves that $C$ is normal \sref{II.6.3.4}.
\end{proof}

\subsection{Grauert's ampleness criterion: statement}
\label{subsection:II.8.9}

We intend to show that the properties proven in \sref{II.8.8.2} \emph{characterise} $f$-ample $\sh{O}_X$-modules, and, more precisely, to prove the following criterion:
\begin{theorem}[8.9.1]
\label{II.8.9.1}
\emph{(Grauert's criterion).}
Let $Y$ be a prescheme, $p:X\to Y$ a separated and quasi-compact morphism, and $\sh{L}$ an invertible $\sh{O}_X$-module.
For $\sh{L}$ to be ample relative to $p$, it is necessary and sufficient for there to exist a $Y$-prescheme $C$, a $Y$-section $\varepsilon:Y\to C$ of $C$, and a $Y$-morphism $q:\bb{V}(\sh{L})\to C$, satisfying the following properties:
\begin{enumerate}
  \item[\rm{(i)}] the diagram
    \[
      \label{II.8.9.1.1}
        \xymatrix{
          X
            \ar[r]^{j}
            \ar[d]_{p}
        & \bb{V}(\sh{L})
            \ar[d]^{q}
        \\Y
            \ar[r]^{\varepsilon}
        & C
        }
      \tag{8.9.1.1}
    \]
    commutes, where $j$ is the null section of the vector bundle $\bb{V}(\sh{L})$; and
  \item[\rm{(ii)}] the restriction of $q$ to $\bb{V}(\sh{L})\setmin j(X)$ is a quasi-compact open immersion
    \[
      \bb{V}(\sh{L})\setmin j(X)\to X
    \]
    whose image does not intersect $\varepsilon(Y)$.
\end{enumerate}
\end{theorem}

Note that, if $C$ is \emph{separated} over $Y$, we can, in condition~(ii), remove the hypothesis that the open immersion is quasi-compact;
to see that this property (of quasi-compactness) is in fact a consequence of the other conditions, we can restrict to the case where $Y$ is affine, and the claim then follows from \sref[I]{I.5.5.1}{i} and \sref[I]{I.5.5.10}.
We can also remove
\oldpage[II]{183}
the same hypothesis if we assume that $X$ is Noetherian, since then $V$ is also Noetherian, and the claim follows from \sref[I]{I.6.3.5}.

\begin{corollary}[8.9.2]
\label{II.8.9.2}
If the morphism $p:X\to Y$ is proper, then we can, in the statement of Theorem~\sref{II.8.9.1}, assume that $q$ is proper, and replace ``open immersion'' by ``isomorphism''.
\end{corollary}

In a more suggestive manner, we can say (whenever $p:X\to Y$ is proper) that \emph{$\sh{L}$ is ample relative to $p$ if and only if we can ``\emph{contract}'' the null section of the vector bundle $\bb{V}(\sh{L})$ to the base prescheme $Y$.}
An important particular case is that where $Y$ is the spectrum of a field, and where the operation of ``contraction'' consists of contract the null section $\bb{V}(\sh{L})$ \emph{to a single point}.

\begin{env}[8.9.3]
\label{II.8.9.3}
The necessity of the conditions in Theorem~\sref{II.8.9.1} and Corollary~\sref{II.8.9.2} follow immediately from \sref{II.8.8.2} and \sref{II.8.8.4}.

To show that the conditions of \sref{II.8.9.1} suffices, consider a slightly more general situation.
For this, let (with the notation of \sref{II.8.8.2})
\[
  \sh{S}'=\bigoplus_{n\geq 0}\sh{L}^{\otimes n}
\]
and
\[
  V=\bb{V}(\sh{L})=\Spec(\sh{S}').
\]

The closed subprescheme $j(X)$, null section of $\bb{V}(\sh{L})$, is defined by the quasi-coherent sheaf of ideals $\sh{J}=(\sh{S}'_+)\supertilde$ of $\sh{O}_V$ \sref{II.1.4.10}.
This $\sh{O}_V$-module is \emph{invertible}, since this property is local on $X$, and this reduces to remarking that the ideal $TA[T]$ in a ring of polynomials $A[T]$ is a free cyclic $A[T]$-module.
Furthermore, it is immediate (again, because the question is local on $X$) that
\[
  \sh{L}=j^*(\sh{J})
\]
and
\[
  j_*(\sh{L})=\sh{J}/\sh{J}^2.
\]

Now, if
\[
  \pi:\bb{V}(\sh{L})\to X
\]
is the structure morphism, then $\pi_*(\sh{J})=\sh{S}'_+$ and $\pi_*(\sh{J}/\sh{J}^2)=\sh{L}$;
there are thus canonical homomorphisms $\sh{L}\to\pi_*(\sh{J})\to\sh{L}$, the first being the canonical injection $\sh{L}\to\sh{S}'_+$, and the second the canonical projection from $\sh{S}'_+$ to $\sh{S}'_1=\sh{L}$, and their composition being the identity.
We can also canonically embed $\pi_*(\sh{J}) = \sh{S}'_+ = \bigoplus_{n\geq1}\sh{L}^{\otimes n}$ into the \emph{product} $\prod_{n\geq1}\sh{L}^{\otimes n} = \varprojlim_n \pi_*(\sh{J}/\sh{J}^{n+1})$ (since $\pi_*(\sh{J}/\sh{J}^{n+1}) = \sh{L}\oplus\sh{L}^{\otimes 2}\oplus\ldots\oplus\sh{L}^{\otimes n}$), and we thus have canonical homomorphisms
\[
\label{II.8.9.3.1}
  \sh{L}\to\varprojlim\pi_*(\sh{J}/\sh{J}^{n+1})\to\sh{L}
  \tag{8.9.3.1}
\]
whose composition is the identity.

With this in mind, the generalisation of \sref{II.8.9.1} that we are going to prove is the following:
\end{env}

\begin{proposition}[8.9.4]
\label{II.8.9.4}
Let $Y$ be a prescheme, $V$ a $Y$-prescheme, and $X$ a closed subprescheme of $V$ defined by an ideal $\sh{J}$ of $\sh{O}_V$, which is an \emph{invertible} $\sh{O}_V$-module;
if $j:X\to V$ is
\oldpage[II]{184}
the canonical injection, then let $\sh{L} = j^*(\sh{J}) = \sh{J}\otimes_{\sh{O}_V}\sh{O}_X$, so that $j_*(\sh{L})=\sh{J}/\sh{J}^2$.
Assume that the structure morphism $p:X\to Y$ is separated and quasi-compact, and that the following conditions are satisfied:
\begin{enumerate}
  \item[\rm{(i)}] there exists a $Y$-morphism $\pi:V\to X$ of finite type such that $\pi\circ j=1_X$, and so $\pi_*(\sh{J}/\sh{J}^2)=\sh{L}$;
  \item[\rm{(ii)}] there exists a homomorphism of $\sh{O}_X$-modules $\varphi:\sh{L}\to\varprojlim\pi_*(\sh{J}/\sh{J}^{n+1})$ such that the composition
    \[
      \sh{L}
      \xrightarrow{\varphi}
      \varprojlim \pi_*(\sh{J}/\sh{J}^{n+1})
      \xrightarrow{\alpha}
      \pi_*(\sh{J}/\sh{J}^2) = \sh{L}
    \]
    (where $\alpha$ is the canonical homomorphism) is the identity;
  \item[\rm{(iii)}] there exists a $Y$-prescheme $C$, a $Y$-section $\varepsilon$ of $C$, and a $Y$-morphism $q:V\to C$ such that the diagram
    \[
    \label{II.8.9.4.1}
      \xymatrix{
        X
          \ar[r]^{j}
          \ar[d]_{p}
      & V
          \ar[d]^{q}
      \\Y
          \ar[r]_{\varepsilon}
      & C
      }
    \tag{8.9.4.1}
    \]
    commutes; and
  \item[\rm{(iv)}] the restriction of $q$ to $W=V\setmin j(X)$ is a quasi-compact open immersion into $C$, whose image does not intersect $\varepsilon(Y)$.
\end{enumerate}
Then $\sh{L}$ is ample relative to $p$.
\end{proposition}


\subsection{Grauert's ampleness criterion: proof}
\label{subsection:II.8.10}

\begin{lemma}[8.10.1]
\label{II.8.10.1}
Let $\pi:V\to X$ be a morphism, $j:X\to V$ an $X$-section of $V$ that is also a closed immersion, and $\sh{J}$ a quasi-coherent ideal of $\sh{O}_V$ that defines the closed subprescheme of $V$ associated to $j$.
Then the following all hold true.
\begin{enumerate}
  \item[\rm{(i)}] For all $n\geq0$, $\pi_*(\sh{O}_V/\sh{J}^{n+1})$ and $\pi_*(\sh{J}/\sh{J}^{n+1})$ are quasi-coherent $\sh{O}_X$-modules, and $\pi_*(\sh{O}_V/\sh{J})=\sh{O}_X$ and $\pi_*(\sh{J}/\sh{J}^2)=j^*(\sh{J})$.
  \item[\rm{(ii)}] If $X=\{\xi\}=\Spec(k)$, where $k$ is a field, then $\varprojlim\pi_*(\sh{O}_V/\sh{J}^{n+1})$ is isomorphic to the separated completion of the local ring $\sh{O}_{j(\xi)}$ for the $\mathfrak{m}_{j(\xi)}$-preadic topology.
  \item[\rm{(iii)}] Assume that $\sh{J}$ is an invertible $\sh{O}_V$-module (which implies that
    \[
      \sh{L} = j^*(\sh{J}) = \pi_*(\sh{J}/\sh{J}^2)
    \]
    is an invertible $\sh{O}_X$-module), and that there exists a homomorphism $\varphi:\sh{L}\to\varprojlim\pi_*(\sh{J}/\sh{J}^{n+1})$ such that the composition $\sh{L} \xrightarrow{\varphi} \varprojlim\pi_*(\sh{J}/\sh{J}^{n+1}) \xrightarrow{\alpha} \pi_*(\sh{J}/\sh{J}^2)$ (where $\alpha$ is the canonical homomorphism) is the identity.
    If we write $\sh{S}=\bigoplus_{n\geq0}\sh{L}^{\otimes n}$, then $\varphi$ canonically induces an isomorphism of $\sh{O}_X$-algebras from the completion $\widehat{\sh{S}}$ of $\sh{S}$ relative to its canonical filtration (the completion being isomorphic to the product $\prod_{n\geq0}\sh{L}^{\otimes n}$) to $\varprojlim\pi_*(\sh{O}_V/\sh{J}^{n+1})$.
\end{enumerate}
\end{lemma}

\begin{proof}
Note first of all that the support of the $\sh{O}_V$-module $\sh{O}_V/\sh{J}^{n+1}$ is $j(X)$, and the support of $\sh{J}/\sh{J}^{n+1}$ is contained in $j(X)$.
In the case of (ii), $j(X)$ is a closed point $j(\xi)$ of $V$,
\oldpage[II]{185}
and, by definition, $\pi_*(\sh{O}_V/\sh{J}^{n+1})$ is the fibre of $\sh{O}_V/\sh{J}^{n+1}$ at the point $j(\xi)$, or, equivalently, setting $C=\sh{O}_{j(\xi)}$, and denoting by $\mathfrak{m}$ the maximal ideal of $C$, the $C$-module $C/\mathfrak{m}^{n+1}$;
claim (ii) is then evident.

To prove (i), note that the question is local on $X$;
we can thus restrict to the case where $X$ is affine.
Let $U$ be an affine open subset of $V$;
then $j(X)\cap U$ is an affine open subset of $j(X)$, so $U_0=\pi(j(X)\cap U)$, which is isomorphic to it, is an affine open subset of $X$;
for every affine open subset $W_0\subset U_0$ in $X$, $W=\pi^{-1}(W_0)\cap U$ is an affine open subset of $V$, since $X$ is a scheme \sref[I]{I.5.5.10};
in particular, $U'=U\cap\pi^{-1}(U_0)$ is an affine open subset of $V$, and clearly $\pi(U')=U_0$ and $j(U_0)=j(X)\cap U$.
Then, by definition, $\Gamma(W_0,\pi_*(\sh{O}_V/\sh{J}^{n+1})) = \Gamma(\pi^{-1}(W_0),\sh{O}_V/\sh{J}^{n+1})$;
but since every point of $\pi^{-1}(W_0)$ not belonging to $j(W_0)$ has an open neighbourhood in $\pi^{-1}(W_0)$ not intersecting $j(X)$, and in which $\sh{O}_V/\sh{J}^{n+1}$ is thus zero, it is clear that the sections of $\sh{O}_V/\sh{J}^{n+1}$ over $\pi^{-1}(W_0)$ and over $W$ are in bijective correspondence.
In other words, if $\pi'$ is the restriction of $\pi$ to $U'$, then the $(\sh{O}_X|U_0)$-modules $\pi_*(\sh{O}_V/\sh{J}^{n+1})|U_0$ and $\pi'_*((\sh{O}_V/\sh{J}^{n+1})|U')$ are identical.
Since $U'$ and $U_0$ are affine, and since the $U_0$ cover $X$, we thus conclude \sref[I]{I.1.6.3} that $\pi_*(\sh{O}_V/\sh{J}^{n+1})$ is quasi-coherent, and the proof is identical for $\pi_*(\sh{J}/\sh{J}^{n+1})$.

Finally, to prove (iii), note that $\sh{S}$ is exactly $\bb{S}_{\sh{O}_X}(\sh{L})$;
so $\varphi$ canonically induces a homomorphism of $\sh{O}_X$-algebras $\psi:\sh{S}\to\varprojlim\pi_*(\sh{O}_V/\sh{J}^{n+1})$ \sref{II.1.7.4};
furthermore, this homomorphism sends $\sh{L}^{\otimes n}$ to $\varprojlim_m\pi_*(\sh{J}^n/\sh{J}^{n+1})$, and is thus continuous for the topologies considered, and indeed then extends to a homomorphism $\widehat{\psi}:\widehat{\sh{S}}\to\varprojlim\pi_*(\sh{O}_V/\sh{J}^{n+1})$.
To see that this is indeed an isomorphism, we can, as in the proof of (i), restrict to the case where $X=\Spec(A)$ and $V=\Spec(B)$ are affine, with $\sh{J}=\widetilde{\mathfrak{J}}$, where $\mathfrak{J}$ is an ideal of $B$;
there is an injection $A\to B$ corresponding to $\pi$ that identifies $A$ with a subring of $B$ that is \emph{complementary} to $B$, and $\sh{L}$ (resp. $\pi_*(\sh{O}_V/\sh{J}^{n+1})$) is the quasi-coherent $\sh{O}_X$-module associated to the $A$-module $L=\mathfrak{J}/\mathfrak{J}^2$ (resp. $B/\mathfrak{J}^{n+1}$).
Since $\sh{J}$ is an \emph{invertible} $\sh{O}_V$-module, we can further assume that $\mathfrak{J}=Bt$, where $t$ is not a zero divisor in $B$.
From the fact that $B=A\oplus Bt$, we deduce that, for all $n>0$,
\[
  B = A \oplus At \oplus At^2 \oplus \ldots \oplus At^n \oplus Bt^{n+1}
\]
and so there exists a canonical $A$-isomorphism from the ring of formal series $A[[T]]$ to $C=\varprojlim B/\mathfrak{J}^{n+1}$ that sends $T$ to $t$.
We also have that $L=A\bar{t}$, where $\bar{t}$ is the class of $t$ modulo $Bt^2$, and the homomorphism $\varphi$ sends, by hypothesis, $\bar{t}$ to an element $t'\in C$ that is congruent to $t$ modulo $Ct^2$.
We thus deduce, by induction on $n$, that
\[
  A \oplus At' \oplus \ldots \oplus At^{'n} \oplus Ct^{n+1}
  =
  A \oplus At \oplus \ldots \oplus At^n \oplus Ct^{n+1}
\]
which proves that the homomorphism $\widehat{\psi}$ does indeed correspond to an isomorphism from $\prod_{n\geq0}L^{\otimes n}$ to $C$.
\end{proof}

\begin{lemma}[8.10.2]
\label{II.8.10.2}
Under the hypotheses of Lemma~\sref{II.8.10.1}, let $g:X'\to X$ be a morphism,
\oldpage[II]{186}
write $V'=V\times_X X'$, and let $\pi':V'\to X'$ and $g:V'\to V$ be the canonical projections, so that we have the commutative diagram
\[
  \xymatrix{
    V
      \ar[d]_{\pi}
  & V'
      \ar[l]_{g'}
      \ar[d]^{\pi'}
  \\X
  & X'
      \ar[l]^{g}
  }
\]

Then $j'=j\times1_{X'}$ is an $X'$-section of $V'$ that is also a closed immersion, and $\sh{J}'=g^{'*}(\sh{J})\sh{O}_{V'}$ is the quasi-coherent ideal of $\sh{O}_{V'}$ that defines the closed subprescheme of $V'$ associated to $j'$.
Furthermore, $\pi'_*(\sh{O}_{V'}/\sh{J}{'n+1}) = g^*(\pi_*(\sh{O}_V/\sh{J}^{n+1}))$.
Finally, $\sh{J}'$ is an $\sh{O}_{V'}$-module that is canonically isomorphic to $g^{'*}(\sh{J})$, and is, in particular, invertible if $\sh{J}$ is an invertible $\sh{O}_V$-module.
\end{lemma}

\begin{proof}
The fact that $j'$ is a closed immersion follows from \sref[I]{I.4.3.1}, and it is an $X'$-section of $V'$ by functoriality of extension of the base prescheme.
Furthermore, if $Z$ (resp. $Z'$) is the closed subprescheme of $V$ (resp. $V'$) associated to $j$ (resp. $j'$), then $Z'=g^{'-1}(Z)$ \sref[I]{I.4.3.1}, and the second claim then follows from \sref[I]{I.4.4.5}.
To prove the other claims, we see, as in \sref{II.8.10.1}, that we can restrict to the case where $X$, $V$, and $X'$ (and thus also $V'$) are affine;
we keep the notation from the proof of \sref{II.8.10.1}, and let $X'=\Spec(A')$.
Then $V'=\Spec(B')$, where $B'=B\otimes_A A'$, and $\sh{J}'=\widetilde{\mathfrak{J}''}$, where $\mathfrak{J}'=\Im(\mathfrak{J}\otimes_A A')$.
Then $B'/\mathfrak{J}^{'n+1}=(B/\mathfrak{J}^{n+1})\otimes_A A'$;
furthermore, since $\mathfrak{J}$ is a direct factor (as an $A$-module) of $B$, $\mathfrak{J}\otimes_A A'$ is a direct factor (as an $A'$-module) of $B'$, and is thus canonically identified with $\mathfrak{J}'$.
\end{proof}

\begin{corollary}[8.10.3]
\label{II.8.10.3}
Assume that the hypotheses of Lemma~\sref{II.8.10.1} are satisfied, and assume further that $\pi$ is of finite type, and that $\sh{J}$ is an invertible $\sh{O}_V$-module.
Then, for all $x\in X$, the local ring at the point $j(x)$ of the fibre $\pi^{-1}(x)$ is a regular (thus integral) ring of dimension 1, whose completion is isomorphic to the formal series ring $\kres(x)[[T]]$ (where $T$ is an indeterminate);
furthermore, there exists exactly one irreducible component of $\pi^{-1}(x)$ that contains $j(x)$.
\end{corollary}

\begin{proof}
Since $\pi^{-1}(x)=V\times_X\Spec(\kres(x))$, we are led, by \sref{II.8.10.2}, to the case where $X$ is the spectrum of a field $K$.
Since $\pi$ is of finite type \sref[I]{I.6.4.3}[iv], $\sh{O}_{j(x)}$ is a Noetherian local ring, and thus separated for the $\mathfrak{m}_{j(x)}$-preadic topology \sref[0]{0.7.3.5};
it follows from \sref{II.8.10.1}[ii and iii] that the completion of this ring is isomorphic to $K[[T]]$, and so $\sh{O}_{j(x)}$ is regular and of dimension 1 (\cite[p.~17-01, th.~1]{I-1});
finally, since $\sh{O}_{j(x)}$ is integral, $j(x)$ belongs to exactly one of the (finitely many) irreducible components of $V$ \sref[I]{I.5.1.4}.
\end{proof}

\begin{corollary}[8.10.4]
\label{II.8.10.4}
Suppose that the hypotheses of Lemma~\sref{II.8.10.1} are satisfied, and assume further that $\sh{J}$ is an invertible $\sh{O}_V$-module.
Let $W=V\setmin j(X)$;
for every quasi-coherent ideal $\sh{K}$ of $\sh{O}_X$, let $\sh{K}_V=\pi^*(\sh{K})\sh{O}_V$ and $\sh{K}_W=\sh{K}_V|W$.
Then $\sh{K}_V$ is the largest quasi-coherent ideal of $\sh{O}_V$ whose restriction to $W$ is $\sh{K}_W$.
\end{corollary}

\begin{proof}
Indeed, we see as in \sref{II.8.10.1} that the question is local on $X$ and $V$;
we can thus reuse the notation from the proof of \sref{II.8.10.1}, with $\mathfrak{J}=Bt$, where $t$ is not a zero divisor in $B$.
Furthermore, we have $W=\Spec(B_t)$ and $\sh{K}=\widetilde{\mathfrak{K}}$, where $\mathfrak{K}$ is an ideal of $A$;
whence $\pi^*(\sh{K})\sh{O}_V=(\mathfrak{K}.B)\supertilde$ \sref[I]{I.1.6.9}, $\sh{K}_W=(\mathfrak{K}.B_t)\supertilde$, and the largest ideal
\oldpage[II]{187}
of $B$ whose canonical image in $B_t$ is $\mathfrak{K}.B_t$ is the inverse image of $\mathfrak{K}.B_t$, that is, the set of $s\in B$ such that, for some integer $n>0$, we have $t^ns\in\mathfrak{K}.B$.
We have to show that this last relation implies that $s\in\mathfrak{K}.B$, or again that the canonical image of $t$ is not a zero divisor in $B/\mathfrak{K}B=(A/\mathfrak{K})\otimes_AB$, which follows from \sref{II.8.10.2} applied to $X'=\Spec(A/\mathfrak{K})$.
\end{proof}

\begin{corollary}[8.10.5]
\label{II.8.10.5}
Suppose that the hypotheses of \sref{II.8.10.3} are satisfied;
let $W=V\setmin j(X)$, $x$ be a point of $X$, $\sh{K}$ a quasi-coherent ideal of $\sh{O}_X$, and $z$ the generic point of the irreducible component of $\pi^{-1}(x)$ that contains $j(x)$ \sref{II.8.10.3}.
\begin{enumerate}
    \item[\rm{(i)}] Let $g$ be a section of $\sh{O}_V$ over $V$ such that $g|W$ is a section of $\sh{K}_W$ over $W$ (using the notation from \sref{II.8.10.4}).
        Then $g$ is a section of $\sh{K}_V$;
        if further $g(z)\neq0$, and if, for every integer $m>0$, we denote by $g_m^x$ the germ at the point $x$ of the canonical image $g_m$ of $g$ in $\Gamma(X,\pi_*(\sh{O}_V/\sh{J}^{m+1}))$, then there exists an integer $m>0$ such that the image of $g_m^x$ in
        \[
            (\pi_*(\sh{O}_V/\sh{J}^{m+1}))_x \otimes_{\sh{O}_x} \kres(x)
        \]
        is $\neq0$.
    \item[\rm{(ii)}] Suppose further that the conditions of \sref{II.8.10.1}[iii] are fulfilled.
        Then, if there exists a section $g$ of $\sh{K}_V$ over $V$ such that $g(z)\neq0$, then there exists an integer $n\geq0$ and a section $f$ of $\sh{K}.\sh{L}^{\otimes n}=\sh{K}\otimes\sh{L}^{\otimes n}\subset\sh{L}^{\otimes n}$ such that $f(x)\neq0$.
        If $g$ is a section of $\sh{J}$, we can take $n>0$.
\end{enumerate}
\end{corollary}

\begin{proof}
\begin{enumerate}
    \item[\rm{(i)}] Since the ideal of $\sh{O}_W$ generated by $g|W$ is contained in $\sh{K}_W$ by hypothesis, the ideal of $\sh{O}_V$ generated by $g$ is contained in $\sh{K}_V$ by \sref{II.8.10.4}, or, in other words, $g$ is a section of $\sh{K}_V$.
        To prove the second claim of (i), we can again assume that $X$ and $V$ are affine, and reuse the notation from \sref{II.8.10.1};
        the fibre $\pi^{-1}(x)$ is then affine of ring $B'=B\otimes_A\kres(x)$, and there exists in $B'$ an element $t'$ which is not a zero divisor and is such that $B'=\kres(x)\oplus B't'$.
        Since $j(x)$ is a specialisation of $z$ and since $g(z)\neq0$, we necessarily have that $g_{(j)x}\neq0$.
        But $\sh{O}_{j(x)}$ is a separated local ring \sref{II.8.10.3}, and thus embeds into its completion, and the image of $g$ in this completion is thus not null.
        But this completion is isomorphic to $\varprojlim_n(B'/B't^{'n+1})$ \sref{II.8.10.3};
        if $g'=g\otimes1\in B'$, there then exists an integer $m$ such that $g'\not\in B't^{'m+1}$, or, again, the image $g'_m$ of $g'$ in $B'/B't^{'m+1}$ is not null.
        But since $g'_m$ is exactly the image of $g_m^x$, our claim is proved.
    \item[\rm{(ii)}] By \sref{II.8.10.1}[iii], $\pi_*(\sh{O}_V/\sh{J}^{m+1})$ is isomorphic to the direct sum of the $\sh{L}^{\otimes k}$ for $0\leq k\leq m$;
        we denote by $f_k$ the section of $\sh{L}^{\otimes k}$ over $X$ that is the component of the element of $\bigoplus_{k=0}^m\Gamma(X,\sh{L}^{\otimes k})$ which corresponds to $g_m$ by this isomorphism.
        Choosing $m$ as in (i), there is thus an index $k$ such that $f_k(x)\neq0$, by (i).
        To see that $f_k$ is a section of $\sh{K}\sh{L}^{\otimes k}$, it suffices to consider, as above, the case where $X$ and $V$ are affine, and this follows immediately from the fact that $g\in\mathfrak{K}.B$ (with the notation from \sref{II.8.10.4}).
        The final claim follows from the fact that the hypothesis $g\in\Gamma(V,\sh{J})$ implies that $f_0=0$.
\end{enumerate}
\end{proof}

\begin{env}[8.10.6]
\label{II.8.10.6}
\emph{Proof of \sref{II.8.9.4}.}
The question is local on $Y$ \sref{II.4.6.4};
since $\varepsilon$ is a $Y$-section, we can thus replace $C$ by an affine open neighbourhood $U$ of a point of $\varepsilon(Y)$ such that $\varepsilon(Y)\cap U$ is closed in $U$.
In other words, we can assume that $C$ is affine, and that $Y$ is a closed subprescheme of $C$ (and thus also affine) defined by a quasi-coherent sheaf
\oldpage[II]{188}
$\sh{I}$ of ideals of $\sh{O}_C$.
Since $p$ is separated and quasi-compact, $X$ is thus a quasi-compact scheme, and we are reduced to proving that $\sh{L}$ is \emph{ample} \sref{II.4.6.4}.
By criterion~\sref{II.4.5.2}[a)], we must thus prove the following:
for every quasi-coherent ideal $\sh{K}$ of $\sh{O}_X$ and every point $x\in X$ not belonging to the support of $\sh{O}_X/\sh{K}$, there exists an integer $n>0$ and a section $f$ of $\sh{K}\otimes\sh{L}^{\otimes n}$ over $X$ such that $f(x)\neq0$.

For this, set
\begin{align*}
  \sh{K}_V &= \pi^*(\sh{K})\sh{O}_V\\
  \sh{K}_W &= \sh{K}_V|W
\end{align*}
where $W=V\setmin j(X)$;
since the restriction of $q$ to $W$ is a quasi-compact immersion to $C$, it follows from \sref[I]{I.9.4.2} that $\sh{K}_W$ is the restriction to $W$ of a quasi-coherent ideal $\sh{K}'_V$ of $\sh{O}_V$ of the form
\[
    \sh{K}'_V = q^*(\sh{K}_C)\sh{O}_V
\]
where $\sh{K}_C$ is a quasi-coherent ideal of $\sh{O}_C$.
Furthermore, since, by hypotheses, $q^{-1}(Y)\subset j(X)$, and since $Y$ is defined by the ideal $\sh{I}$, the restriction to $W$ of $q^*(\sh{I})\sh{O}_V$ is identical to that of $\sh{O}_V$, and so $\sh{K}_W$ is also the restriction to $W$ of $q^*(\sh{I}\sh{K}_C)\sh{O}_V$, and we can thus suppose that $\sh{K}_C\subset\sh{I}$, whence
\[
\label{II.8.10.6.1}
    \sh{K}'_V \subset q^*(\sh{I})\sh{O}_V \subset \sh{J}
\tag{8.10.6.1}
\]
taking into account \sref[I]{I.4.4.6} and the commutativity of \sref{II.8.9.4.1}.
Furthermore, we deduce from \sref{II.8.10.4} that
\[
\label{II.8.10.6.2}
    \sh{K}'_V \subset \sh{K}_V.
\tag{8.10.6.2}
\]
With this in mind, it follows from \sref{II.8.10.3} that $j(x)$ belongs to exactly one irreducible component of $\pi^{-1}(x)$;
let $z$ be the generic point of this component, and let $z'=q(z)$.
By \sref{II.8.10.5}, the proof will be finished (taking \sref{II.8.10.6.1} and \sref{II.8.10.6.2} into account) if we show the existence of a section $g$ of $\sh{K}'_V$ over $V$ such that $g(z)\neq0$.
But, by hypothesis, $\sh{K}$ has a restriction equal to that of $\sh{O}_X$ in an open neighbourhood of $x$;
also, it follows from \sref{II.8.10.3} that $z\neq j(x)$, and so $z\in W$, and thus $(\sh{K}_W)_z=\sh{O}_{V,z}$, whence, by definition, $(\sh{K}_C)_{z'}=\sh{O}_{C,z}$.
Since $C$ is affine, there is thus a section $g'$ of $\sh{K}_C$ over $C$ such that $g'(z')\neq0$, and by taking $g$ to be the section of $\sh{K}'_V$ corresponding canonically to $g'$, we indeed have $g(z)\neq0$, which finishes the proof.
\end{env}

\begin{remark}[8.10.7]
\label{II.8.10.7}
We ignore the question of whether or not condition (ii) in \sref{II.8.9.4} is superfluous or not.
In any case, the conclusion does not hold if we do not assume the existence of a $Y$-morphism $\pi:V\to X$ such that $\pi\circ j=1_X$;
we briefly point out how we can indeed construct a counterexample, whose details will not be developed until later on.
We take $Y=\Spec(k)$, where $k$ is a field, and $C=\Spec(A)$, where $A=k[T_1,T_2]$, and the $Y$-section $\varepsilon$ corresponding to the augmentation homomorphism $A\to k$.
We denote by $C'$ the scheme induced by $C$ by blowing up the closed point $a=\varepsilon(Y)$ of $C$;
if $D$ is the inverse image of $a$ in $C'$, we consider in $D$ a closed point $b$,
\oldpage[II]{189}
and we denote by $V$ the scheme induced by $C'$ by blowing up $b$;
$X$ is the closed subprescheme of $V$ given by the inverse image of $a$ by the structure morphism $q:V\to C$.
We now show that $X$ is the union of two irreducible components, $X_1$ and $X_2$, where $X_1$ is the inverse image of $b$ in $V$.
It is immediate that the ideal $\sh{J}$ of $\sh{O}_V$ that defines $X$ is again invertible, we we can show that $j^*(\sh{J})=\sh{L}$ (where $j$ is the canonical injection $X\to V$) is not ample, by considering the ``degree'' of the inverse image of $\sh{L}$ in $X_1$, which would be $>0$ if $\sh{L}$ were ample, but we can show (by an elementary intersection calculation) that it is in fact equal to $0$.
\end{remark}


\subsection{Uniqueness of contractions}
\label{subsection:II.8.11}

\begin{lemma}[8.11.1]
\label{II.8.11.1}
Let $U$ and $V$ be preschemes, and $h=(h_0,\lambda):U\to V$ a surjective morphism.
Suppose that
\begin{enumerate}
    \item $\lambda:\sh{O}_V\to h_*(\sh{O}_U) = (h_0)_*(\sh{O}_U)$ is an isomorphism;
    \item the underlying space of $V$ can be identified with the quotient of the underlying space of $U$ by the relation $h_0(x)=h_0(y)$ (\emph{a condition which always holds whenever the morphism $h$ is \emph{open} or \emph{closed}, or, \emph{a fortiori} when $h$ is \emph{proper}.})
\end{enumerate}
Then, for every prescheme $W$, the map
\[
\label{II.8.11.1.1}
    \Hom(V,W) \to \Hom(U,W)
\tag{8.11.1.1}
\]
that, to each morphism $v=(v_0,\nu)$ from $V$ to $W$, associates the morphism $u=v\circ h=(u_0,\mu)$, is a bijection from $\Hom(V,W)$ to the set of $u$ such that $u_0$ is constant on every fibre $h_0^{-1}(x)$.
\end{lemma}

\begin{proof}
It is clear that, if $u=v\circ h$, so that $u_0=v_0\circ h_0$, then $u_0$ is constant on every set $h_0^{-1}(x)$.
Conversely, if $u$ has this property, we will show that there exists exactly one $v\in\Hom(V,W)$ such that $u=v\circ h$.
The existence and uniqueness of the continuous map $v_0:V\to W$ such that $u_0=v_0\circ h_0$ follows from the hypotheses, since $h_0$ can be identified with the canonical map from $U$ to $U/R$.
We can also, replacing $V$ by some isomorphic prescheme if necessary, suppose that $\lambda$ is the identity;
by hypothesis, $\mu$ is then a homomorphism $\mu:\sh{O}_W\to(u_0)_*(\sh{O}_U) = (v_0)_*((h_0)_*(\sh{O}_U))$ such that the corresponding homomorphism $\mu^\sharp:u_0^*(\sh{O}_W)\to\sh{O}_U$ is \emph{local} on every fibre.
Since $(v_0)_*((h_0)_*(\sh{O}_U))=(v_0)_*(\sh{O}_V)$, we necessarily have that $\nu=u$, and everything then reduces to showing that the corresponding homomorphism $\nu^\sharp:v_0^*(\sh{O}_W)\to\sh{O}_V$ is local on every fibre.
But every $y\in V$ is of the form $h_0(x)$ for some $x\in U$;
let $z=v_0(y)=u_0(x)$.
Then \sref[0]{0.3.5.5} the homomorphism $\mu_x^\sharp$ factors as
\[
    \mu_x^\sharp: \sh{O}_z \xrightarrow{\nu_y^\sharp} \sh{O}_y \xrightarrow{\lambda_x^\sharp} \sh{O}_x.
\]
By hypothesis, $\lambda_x^\sharp$ and $\mu_x^\sharp$ are local homomorphisms;
thus $\lambda_x^\sharp$ sends every invertible element of $\sh{O}_y$ to an invertible element of $\sh{O}_x$;
if $\nu_y^\sharp$ sent a non-invertible element of $\sh{O}_z$ to an invertible element of $\sh{O}_y$, then $\mu_x^\sharp$ would send this element of $\sh{O}_z$ to an invertible element of $\sh{O}_x$, contradicting the hypothesis, whence the lemma.
\end{proof}

\begin{corollary}[8.11.2]
\label{II.8.11.2}
Let $U$ be an integral prescheme, and $V$ a normal prescheme;
then every morphism $h:U\to V$ that is universally closed, birational, and radicial, is also an isomorphism.
\end{corollary}

\begin{proof}
If $h=(h_0,\lambda)$, then it follows from the hypotheses that $h_0$ is injective and closed, and that $h_0(U)$
\oldpage[II]{190}
is dense in $V$, and so $h_0$ is a \emph{homeomorphism} from $U$ to $V$.
To prove the corollary, it will suffice to show that $\lambda:\sh{O}_V\to(h_0)_*(\sh{O}_U)$ is an isomorphism: we can then apply \sref{II.8.11.1}, which proves that the map \sref{II.8.11.1.1} is bijective (the fibres $h_0^{-1}(x)$ each consisting of a single point);
thus $h$ will be an isomorphism.
The question clearly being local on $V$, we can suppose that $V=\Spec(A)$ is affine, of an integral and integrally closed ring \sref{II.8.8.6.1};
$h$ then corresponds \sref[I]{I.2.2.4} to a homomorphism $\varphi:A\to\Gamma(U,\sh{O}_U)$, and everything reduces to showing that $\varphi$ is an isomorphism.
But, if $K$ is the field of fractions of $A$, then $\Gamma(U,\sh{O}_U)$ has, by hypothesis, $K$ as its field of fractions, and $A$ is a subring of $\Gamma(U,\sh{O}_U)$, with $\varphi$ being the canonical injection \sref[I]{I.8.2.7}.
Since the morphism $h$ satisfies the hypotheses of \sref{II.7.3.11}, $\Gamma(U,\sh{O}_U)$ is a subring of the integral closure of $A$ in $K$, and is thus identical to $A$ by hypothesis.
\end{proof}

\begin{remark}[8.11.3]
\label{II.8.11.3}
We will see in chapter~III \sref[III]{III.4.4.11} that, whenever $V$ is a \emph{locally Noetherian} prescheme, every morphism $h:U\to V$ that is proper and quasi-finite (in particular, every morphism satisfying the hypotheses of \sref{II.8.11.2}) is necessarily \emph{finite}.
The conclusion of \sref{II.8.11.2} then follows in this case from \sref{II.6.1.15}.
\end{remark}

\begin{env}[8.11.4]
\label{II.8.11.4}
We will now see that, in Grauert's criterion, we can often prove that the prescheme $C$ and the ``contraction'' $q$ are determined in an \emph{essentially unique} manner.
\end{env}

\begin{lemma}[8.11.5]
\label{II.8.11.5}
Let $Y$ be a prescheme, $p:X\to Y$ a proper morphism, $\sh{L}$ a $p$-ample invertible $\sh{O}_X$-module, $C$ a $Y$-prescheme, $\varepsilon:Y\to C$ a $Y$-section, and $q:V=\bb{V}(\sh{L})\to C$ a $Y$-morphism, all such that the diagram in \sref{II.8.9.1.1} commutes.
Suppose further that, if $p=(p_0,\theta)$, then $\theta:\sh{O}_Y\to p_*(\sh{O}_X)$ is an isomorphism.
Let $\sh{S}'=\bigoplus_{n\geq0}p_*(\sh{L}^{\otimes n})$ and $C'=\Spec(\sh{S}')$, and let $q':\bb{V}(\sh{L})\to C'$ be the canonical $Y$-morphism \sref{II.8.8.5}.
Then there exists exactly one $Y$-morphism $u:C'\to C$ such that $q=u\circ q'$.
\end{lemma}

\begin{proof}
The hypothesis on $\theta$ implies, in particular, that $p$ is surjective;
since, by \sref{II.8.8.4}, the restriction of $q'$ to $\bb{V}(\sh{L})\setmin j(X)$ is an \emph{isomorphism} to $C'\setmin\varepsilon'(Y)$ (where $\varepsilon$ is the vertex section of $C'$), it follows from \sref{II.8.8.4} that $q'$ is \emph{proper} and \emph{surjective};
furthermore, by \sref{II.8.8.6}, if we let $q'=(q'_0,\tau)$, then $\tau:\sh{O}_{C'}\to q'_*(\sh{O}_V)$ is an isomorphism.
We are thus in a situation where we can apply \sref{II.8.11.1}, and we will have proven the lemma if we show that $q$ is constant on every fibre $q^{'-1}(z')$, where $z'\in C'$.
But this condition is trivially satisfied for $z'\not\in\varepsilon'(Y)$.
If $z'\in\varepsilon'(Y)$, then there exists exactly one $y\in Y$ such that $z'=\varepsilon'(y)$, and, by commutativity of \sref{II.8.8.5.2} and the fact that $q'$ sends $\bb{V}(\sh{L})\setmin j(X)$ to $C'\setmin\varepsilon'(Y)$, $q^{'-1}(z')=j(p^{-1}(y))$;
the commutativity of the diagram in \sref{II.8.9.1.1} then proves our claim.
\end{proof}

\begin{corollary}[8.11.6]
\label{II.8.11.6}
Under the hypotheses of \sref{II.8.11.5}, suppose further that $q$ is proper, and that the restriction of $q$ to $\bb{V}(\sh{L})\setmin j(X)$ is an isomorphism to $C\setmin\varepsilon(Y)$.
Then the morphism $u$ is universally closed, surjective, and radicial, and its restriction to $C'\setmin\varepsilon'(Y)$ is an isomorphism to $C\setmin\varepsilon(Y)$.
\end{corollary}

\begin{proof}
Since $q'$ is an isomorphism from $\bb{V}(\sh{L})\setmin j(X)$ to $C'\setmin\varepsilon'(Y)$ \sref{II.8.8.4}, the last claim follows immediately from the fact that $q=u\circ q'$.
Furthermore, the commutativity of the diagrams
\oldpage[II]{191}
in \sref{II.8.8.5.2} and \sref{II.8.9.1.1} shows that the restriction of $u$ to the closed subprescheme $\varepsilon'(Y)$ of $C'$ is an isomorphism to the closed subprescheme $\varepsilon(Y)$ of $C$, from which we immediately deduce that, for all $z'\in\varepsilon'(Y)$, if $z=u(z')$, then $u$ defines an isomorphism from $\kres(z)$ to $\kres(z')$.
These remarks prove that $u$ is bijective and radicial;
furthermore, if $\psi:C\to Y$ and $\psi':C'\to Y$ are the structure morphisms, then $\psi'=\psi\circ u$, and, since $\psi'$ is separated \sref{II.1.2.4}, so too is $u$ \sref[I]{I.5.5.1}[v].
We have already seen, in the proof of \sref{II.8.11.5}, that $q'$ is surjective;
since $q=u\circ q'$ is proper, we finally conclude, from \sref{II.5.4.3} and \sref{II.5.4.9}, that $u$ is universally closed.
\end{proof}

\begin{proposition}[8.11.7]
\label{II.8.11.7}
Let $Y$ be a prescheme, $X$ an \emph{integral} prescheme, $p:X\to Y$ a proper morphism, $\sh{L}$ a $p$-ample invertible $\sh{O}_X$-module, $C$ a \emph{normal} $Y$-prescheme, $\varepsilon:Y\to C$ a $Y$-section, and $q:V=\bb{V}(\sh{L})\to C$ a $Y$ morphism, all such that the diagram in \sref{II.8.9.1.1} commutes.
Suppose further that, if $p=(p_0,\theta)$, then $\theta:\sh{O}_Y\to p_*(\sh{O}_X)$ is an isomorphism.
Let $\sh{S}'=\bigoplus_{n\geq0}p_*(\sh{L}^{\otimes n})$ and $C'=\Spec(\sh{S}')$, and let $q':\bb{V}(\sh{L})\to C'$ be the canonical $Y$-morphism \sref{II.8.8.5}.
Then the unique $Y$-morphism $u:C'\to C$ such that $q=u\circ q'$ is an \emph{isomorphism}.
\end{proposition}

\begin{proof}
It follows from \sref{II.8.8.6} that $C'$ is integral;
since $u$ is a homeomorphism of the underlying subspaces $C'\to C$ ($u$ being bijective and closed, by \sref{II.8.11.6}), $C$ is irreducible, thus integral, and, since the restriction of $u$ to a non-empty open subset of $C'$ is an isomorphism to an open subset of $C$, $u$ is birational.
Since $C$ is assumed to be normal, it suffices to apply \sref{II.8.11.2} to obtain the conclusion.
\end{proof}

\begin{remark}[8.11.8]
\label{II.8.11.8}
\begin{itemize}
    \item[\rm{(i)}] The hypothesis that $C$ is normal implies that $X$ is also normal.
        Indeed, $C'=\Spec(\sh{S}')$ is then normal, being isomorphic to $C$, and integral, by \sref{II.8.8.6};
        we thus conclude that $\Proj(\sh{S}')$ is \emph{normal}.
        Indeed, the question is local on $Y$;
        if $Y$ is affine, with $\sh{S}'=\widetilde{S'}$, then the ring $S'=\Gamma(C',\sh{S}')$ is integral and integrally closed \sref{II.8.8.6.1}, and so, for every homogeneous element $f\in S'_+$, the graded ring $S'_f$ is integral and integrally closed \cite[t.~I, p.~257 and 261]{I-13}, and thus so too is the ring $S'_{(f)}$ of its degree-zero terms, because the intersection of $S'_f$ with the field of fractions of $S'_{(f)}$ is equal to $S'_{(f)}$;
        this proves our claim \sref{II.6.3.4}.
        Finally, since $X$ is isomorphic to an open subprescheme of $\Proj(\sh{S}')$ \sref{II.8.8.1}, $X$ is indeed normal.
        We can thus express \sref{II.8.11.7} in the following form:
        \emph{If $X$ is integral and normal, and $p=(p_0,\theta):X\to Y$ is a proper morphism such that $\theta:\sh{O}_Y\to p_*(\sh{O}_X)$ is an isomorphism, then, for every $p$-ample $\sh{O}_X$-module $\sh{L}$, there exists exactly one way of contracting the null section of $V=\bb{V}(\sh{L})$ to obtain a normal $Y$-scheme $C$ and a proper $Y$-morphism $q:V\to C$.}
    \item[\rm{(ii)}] When $p$ is proper, the hypothesis $p_*(\sh{O}_X)=\sh{O}_Y$ can be considered as an auxiliary hypothesis, not really restricting the generality of the result.
        Indeed, if it is not satisfied, then it suffices to replace $Y$ with the $Y$-scheme $Y'=\Spec(p_*(\sh{O}_X))$, and to consider $X$ as a $Y'$-scheme.
        We will return to this general method in chapter~III, \textsection~4.
\end{itemize}
\end{remark}


\subsection{Quasi-coherent sheaves on based cones}
\label{subsection:II.8.12}

\begin{env}[8.12.1]
\label{II.8.12.1}
Let us use the hypotheses and notation of \sref{II.8.3.1}.
Let $\sh{M}$ be a \emph{quasi-coherent graded $\sh{S}$-module}; to avoid any confusion, we denote by $\widetilde{\sh{M}}$ the quasi-coherent $\sh{O}_C$-module
\oldpage[II]{192}
associated to $\sh{M}$ \sref{II.1.4.3} when $\sh{M}$ is considered as a \emph{non-graded} $\sh{S}$-module, and by $\shProj_0(\sh{M})$ the quasi-coherent $\sh{O}_X$-module associated to $\sh{M}$, $\sh{M}$ being considered this time as a graded $\sh{S}$-module (in other words, the $\sh{O}_X$-module denoted by $\widetilde{\sh{M}}$ in \sref{II.3.2.2}).
In addition, we set
\[
\label{II.8.12.1.1}
  \sh{M}_X=\shProj_0(\sh{M})=\bigoplus_{n\in\bb{Z}}\shProj_0(\sh{M}(n));
\tag{8.12.1.1}
\]
the quasi-coherent graded $\sh{O}_X$-algebra $\sh{S}_X$ being defined by \sref{II.8.6.1.1}, $\shProj(\sh{M})$ is equipped with a structure of a \emph{(quasi-coherent) graded $\sh{S}_X$-module}, by means of the canonical homomorphisms \sref{II.3.2.6.1}
\[
\label{II.8.12.1.2}
  \sh{O}_X(m)\otimes_{\sh{O}_X}\shProj_0(\sh{M}(n))\to\shProj_0(\sh{S}(m)\otimes_\sh{S}\sh{M}(n))\to\shProj_0(\sh{M}(m+n)),
\tag{8.12.1.2}
\]
the verification of the axioms of sheaves of modules being done using the commutative diagram in \sref{II.2.5.11.4}.

If $Y=\Spec(A)$ is affine, $\sh{S}=\widetilde{S}$, and $\sh{M}=\widetilde{M}$, where $S$ is a graded $A$-algebra and $M$ is a graded $S$-module, then, for every homogeneous element $f\in S_+$, we have
\[
\label{II.8.12.1.3}
  \Gamma(X_f,\shProj(\widetilde{M}))=M_f
\tag{8.12.1.3}
\]
by the definitions and \sref{II.8.2.9.1}.

Now consider the quasi-coherent graded $\widehat{\sh{S}}$-module
\[
\label{II.8.12.1.4}
  \widehat{\sh{M}}=\sh{M}\otimes_\sh{S}\widehat{\sh{S}}
\tag{8.12.1.4}
\]
($\widehat{\sh{S}}$ being defined by \sref{II.8.3.1.1}); this induces a quasi-coherent graded $\sh{O}_{\widehat{C}}$-module $\shProj_0(\widehat{\sh{M}})$, which we will also denote by
\[
\label{II.8.12.1.5}
  \sh{M}^\square=\shProj_0(\widehat{\sh{M}}).
\tag{8.12.1.5}
\]

It is clear \sref{II.3.2.4} that $\sh{M}^\square$ is an additive functor which is \emph{exact} in $\sh{M}$, commuting with direct sums and with inductive limits.
\end{env}

\begin{proposition}[8.12.2]
\label{II.8.12.2}
With the notation of \sref{II.8.3.2}, we have canonical functorial isomorphisms
\[
\label{II.8.12.2.1}
    i^*(\sh{M}^\square)\xrightarrow{\sim}\widetilde{\sh{M}},
    \quad
    j^*(\sh{M}^\square)\xrightarrow{\sim}\shProj_0(\sh{M}).
\tag{8.12.2.1}
\]
Indeed, $i^*(\sh{M}^\square)$ is canonically identified with $(\widehat{\sh{M}}/(\bb{z}-1)\widehat{\sh{M}})\supertilde$ on $\Spec(\widehat{\sh{S}}/(\bb{z}-1)\widehat{\sh{S}})$ by \sref{II.3.2.3};
the first of the canonical isomorphisms \sref{II.8.12.2.1} is then immediately induced \sref{II.1.4.1} by the canonical isomorphism $\widehat{\sh{M}}/(\bb{z}-1)\widehat{\sh{M}}\xrightarrow{\sim}\sh{M}$.
The canonical immersion $j:X\to C$ corresponds to the canonical homomorphism $\widehat{\sh{S}}\to\sh{S}$ with kernel $\bb{z}\widehat{\sh{S}}$ \sref{II.8.3.2};
the second homomorphism \sref{II.8.12.2.1} is the particular case of the canonical homomorphism \sref{II.3.5.2}[ii], since here we have $\widehat{\sh{M}}\otimes_{\widehat{\sh{S}}}\sh{S}=\sh{M}$;
to verify that this is an isomorphism, we can restrict to the case where $Y=\Spec(A)$ is affine, $\sh{S}=\widetilde{S}$, and $\sh{M}=\widetilde{M}$;
by appealing to \sref{II.2.8.8}, the proof that, for all homogeneous $f$ in $S_+$, the preceding homomorphism, restricted to $X_f$, restricts to an isomorphism, is then immediate.
\end{proposition}

\oldpage[II]{193}

By an abuse of language, we again say, thanks to the existence of the first isomorphism \sref{II.8.12.2.1}, that $\sh{M}^\square$ is the \emph{projective closure} of the $\sh{O}_X$-module $\widetilde{\sh{M}}$ (it being implicit that the data of the $\sh{O}_C$-module $\widetilde{\sh{M}}$ includes the grading of the $\sh{S}$-module $\sh{M}$).

\begin{env}[8.12.3]
\label{II.8.12.3}
With the notation of \sref{II.8.3.5}, we have a canonical functorial homomorphism
\[
\label{II.8.12.3.1}
    p^*(\shProj(\sh{M}))\to\sh{M}^\square|\widehat{E}.
\tag{8.12.3.1}
\]

Indeed, this is a particular case of the homomorphism $\nu^\sharp$ defined more generally in \sref{II.3.5.6}.
If $Y=\Spec(A)$ is affine, $\sh{S}=\widetilde{S}$, and $\sh{M}=\widetilde{M}$, then, by appealing to \sref{II.2.8.8}, the restriction of \sref{II.8.12.3.1} to $p^{-1}(X_f)=\widehat{C}_f$ (for some homogeneous $f$ in $S_+$) corresponds to the canonical homomorphism
\[
\label{II.8.12.3.2}
    M_{(f)}\otimes_{S_{(f)}}S_f^\leq \to M_f^\leq
\tag{8.12.3.2}
\]
taking into account \sref{II.8.2.3.2} and \sref{II.8.2.5.2}.
\end{env}

\begin{env}[8.12.4]
\label{II.8.12.4}
Let us place ourselves in the settings of \sref{II.8.5.1}, and assume its hypotheses and keep its notation.
It follows from \sref{II.1.5.6} that, for every quasi-coherent graded $\sh{S}$-module $\sh{S}$, we have, on one hand, a canonical isomorphism
\[
\label{II.8.12.4.1}
    \Phi^*(\widetilde{\sh{M}}) \xrightarrow{\sim} (q^*(\sh{M})\otimes_{q^*(\sh{S})}\sh{S}')\supertilde
\tag{8.12.4.1}
\]
of $\sh{O}_{C'}$-modules;
on the other hand, \sref{II.3.5.6} implies the existence of a canonical $\Proj(\varphi)$-morphism
\[
\label{II.8.12.4.2}
    \shProj_0\sh{M} \to (\shProj_0(q^*(\sh{M}))\otimes_{q^*(\sh{S})}\sh{S}')|G(\varphi)
\tag{8.12.4.2}
\]
and also of a canonical $\widehat{\Phi}$-morphism
\[
\label{II.8.12.4.3}
    \shProj_0\widehat{\sh{M}} \to (\shProj_0(q^*(\widehat{\sh{M}}))\otimes_{q^*(\widehat{\sh{S}})}\widehat{\sh{S}}')|G(\widehat{\varphi}).
\tag{8.12.4.3}
\]
\end{env}

\begin{env}[8.12.5]
\label{II.8.12.5}
Consider now the setting of \sref{II.8.6.1}, with the same notation;
we thus take $Y'=X$, the morphism $q:X\to Y$ being the structure morphism, and $\varphi$ the canonical $q$-morphism \sref{II.8.6.1.2}.
We then have a canonical isomorphism
\[
\label{II.8.12.5.1}
    q^*(\sh{M})\otimes_{q^*(\sh{S})}\sh{S}_X^\geq \xrightarrow{\sim} \sh{M}_X^\geq
\tag{8.12.5.1}
\]
by setting $\sh{M}_X^\geq=\bigoplus_{n\geq0}\shProj_0(\sh{M}(n))$.
We can indeed restrict to the case where $Y=\Spec(A)$ is affine, $\sh{S}=\widetilde{S}$, and $\sh{M}=\widetilde{M}$, and define the isomorphism \sref{II.8.12.5.1} on each of the affine open subsets $X_f$ (where $f$ is homogeneous in $S_+$), by verifying the compatibility with taking a homogeneous multiple of $f$.
But the restriction to $X_f$ of the left-hand side of \sref{II.8.12.5.1} is $\widetilde{M}'=((M\otimes_A S_{(f)})\otimes_{S\otimes_A S_{(f)}}S_f^\geq)\supertilde$ by \sref{II.8.6.2.1};
since we have a canonical isomorphism from $M\otimes_A S_{(f)}$ to $M\otimes_S(S\otimes_A S_{(f)})$, we have an induced isomorphism from $\widetilde{M}'$ to $(M\otimes_S S_f^\geq)\supertilde$, and the latter is canonically isomorphic, by \sref{II.8.2.9.1}, to the restriction to $X_f$ of the right-hand side of \sref{II.8.12.5.1}, and satisfies the required compatibility conditions.
\oldpage[II]{194}

Replacing $\sh{M}$ by $\widehat{\sh{M}}$, $\sh{S}$ by $\widehat{\sh{S}}$, and $\sh{S}_X$ by $(\sh{S}_X^\geq)^\wedge$ in the previous argument, we similarly have a canonical isomorphism
\[
\label{II.8.12.5.2}
    q^*(\widehat{\sh{M}})\otimes_{q^*(\widehat{\sh{S}})}(\sh{S}_X^\geq)^\wedge \xrightarrow{\sim} (\sh{M}_X^\geq)^\wedge.
\tag{8.12.5.2}
\]

If we recall \sref{II.8.6.2} that the structure morphism $u:\Proj(\sh{S}_X^\geq)\to X$ is an isomorphism, then we deduce, first of all, from the above, that we have a canonical $u$-isomorphism
\[
\label{II.8.12.5.3}
    \shProj_0\sh{M} \xrightarrow{\sim} \shProj_0(\sh{M}_X^\geq)
\tag{8.12.5.3}
\]
as a particular case of \sref{II.8.12.4.2}.
We note that, with the notation from the proof of \sref{II.8.6.2}, this reduces to seeing that the canonical homomorphism $M_{(f)}\otimes_{S_{(f)}}(S_f^\geq)^{(d)}\to(M_f^\geq)^{(d)}$ is an isomorphism whenever $f\in S_d$, which is immediate.

Secondly, the isomorphism \sref{II.8.12.5.2} gives us, by this time applying \sref{II.8.12.4.3} to the canonical morphism $r=\Proj(\widehat{\alpha}):\widehat{C}_X\to\widehat{C}$, a canonical $r$-morphism
\[
\label{II.8.12.5.4}
    \sh{M}^\square \to (\sh{M}_X^\geq)^\square.
\tag{8.12.5.4}
\]

Recall now \sref{II.8.6.2} that the restrictions of $r$ to the pointed cones $\widehat{E}_X$ and $E_X$ are \emph{isomorphisms} to $\widehat{E}$ and $E$ (respectively).
Furthermore:
\end{env}

\begin{proposition}[8.12.6]
\label{II.8.12.6}
The restrictions to $\widehat{E}_X$ and $E_X$ of the canonical $r$-morphism \sref{II.8.12.5.4} are isomorphisms
\[
\label{II.8.12.6.1}
    \sh{M}^\square|\widehat{E} \xrightarrow{\sim} (\sh{M}_X^\geq)^\square|\widehat{E}_X
\tag{8.12.6.1}
\]
\[
\label{II.8.12.6.2}
    \sh{M}^\sim|\widehat{E} \xrightarrow{\sim} (\sh{M}_X^\geq)\supertilde|\widehat{E}_X.
\tag{8.12.6.2}
\]
\end{proposition}

\begin{proof}
We restrict to the case where $Y$ is affine, as in the proof of \sref{II.8.6.2} (whose notation we adopt);
by reducing to definitions \sref{II.2.8.8}, we have to show that the canonical homomorphism
\[
    \widehat{M}_{(f)}\otimes_{\widehat{S}_{(f)}}(S_f^\geq)_{(f/1)}^\wedge \to (M\otimes_S S_f^\geq)_{(f/1)}^\wedge
\]
is an isomorphism;
but, by \sref{II.8.2.3.2} and \sref{II.8.2.5.2}, the left-hand side is canonically identified with $M_f^\leq\otimes_{S_f^\leq}(S_f^\geq)_{f/1}^\leq$, and thus with $M_f^\leq$, by \sref{II.8.2.7.2}, and the right-hand side with $(M_f^\geq)_{f/1}^\leq$, and thus also with $M_f^\leq$, by \sref{II.8.2.9.2}, whence the conclusion concerning \sref{II.8.12.6.1};
\sref{II.8.12.6.2} then follows from \sref{II.8.12.6.1} and \sref{II.8.12.2.1}.
\end{proof}

\begin{corollary}[8.12.7]
\label{II.8.12.7}
With the identifications of \sref{II.8.6.3}, the restriction of $(\sh{M}_X^\geq)^\square$ to $\widehat{E}_X$ can be identified with $(\sh{M}_X^\leq)\supertilde$, and the restriction of $(\sh{M}_X^\geq)^\square$ to $E_x$ with $\widetilde{\sh{M}}_X$.
\end{corollary}

\begin{proof}
We can restrict to the affine case, and this follows from the identification of $(M_f^\geq)_{f/1}^\leq$ with $M_f^\leq$, and of $(M_f^\geq)_{f/1}$ with $M_f$ \sref{II.8.2.9.2}.
\end{proof}

\begin{proposition}[8.12.8]
\label{II.8.12.8}
Under the hypotheses of \sref{II.8.6.4}, the canonical homomorphism \sref{II.8.12.3.1} is an isomorphism.
\end{proposition}

\begin{proof}
Taking into account the fact that $\Proj(\sh{S}_X^\geq)\to X$ is an isomorphism \sref{II.8.6.2}, and the
\oldpage[II]{195}
isomorphisms \sref{II.8.12.5.4} and \sref{II.8.12.6.1}, we are led to proving the corresponding proposition for the canonical homomorphism $p_X^*(\shProj_0(\sh{M}_X^\geq))\to(\sh{M}_X^\geq)^\square|E_X$, or, in other words, we can restrict to the case where $\sh{S}_1$ is an invertible $\sh{O}_Y$-module, and where $\sh{S}$ is generated by $\sh{S}_1$.
With the notation of \sref{II.8.12.3}, we then have, for some $f\in S_1$, that $S_f^\leq=S_{(f)}[1/f]$, and the canonical homomorphism $M_{(f)}\otimes_{S_{(f)}}S_f^\leq\to M_f^\leq$ is an isomorphism, by the definition of $M_f^\leq$.
\end{proof}

\begin{env}[8.12.9]
\label{II.8.12.9}
Consider now the quasi-coherent $\sh{S}$-modules
\[
    \sh{M}_{[n]}=\bigoplus_{m\geq n}\sh{M}_m
\]
and (with the notation of \sref{II.8.7.2}) the graded quasi-coherent $\sh{S}^\natural$-module
\[
\label{II.8.12.9.1}
    \sh{M}^\natural=\left(\bigoplus_{n\geq0}\sh{M}_{[n]}\right)\supertilde.
\tag{8.12.9.1}
\]

We have seen \sref{II.8.7.3} that there exists a canonical $C$-isomorphism $h:C_X\xrightarrow{\sim}\Proj(\sh{S}^\natural)$.
Furthermore:
\end{env}

\begin{proposition}[8.12.10]
\label{II.8.12.10}
There exists a canonical $h$-isomorphism
\[
\label{II.8.12.10.1}
    \shProj_0(\sh{M}^\natural) \xrightarrow{\sim} \widetilde{\sh{M}}_X.
\tag{8.12.10.1}
\]
\end{proposition}

\begin{proof}
We argue as in \sref{II.8.7.3}, this time using the existence of the di-isomorphism \sref{II.8.2.9.3} instead of \sref{II.8.2.7.3}.
We leave the details to the reader.
\end{proof}


\subsection{Projective closures of subsheaves and closed subschemes}
\label{subsection:II.8.13}

\begin{env}[8.13.1]
\label{II.8.13.1}
With hypotheses and notation as in \sref{II.8.12.1}, consider a \emph{not-necessarily graded} quasi-coherent sub-$\sh{S}$-module $\sh{N}$ of $\sh{M}$.
We can then consider the quasi-coherent $\sh{O}_C$-module $\widetilde{\sh{N}}$ associated to $\sh{N}$, which is a sub-$\sh{O}_C$-module of $\widetilde{\sh{M}}$.
We have seen elsewhere \sref{II.8.12.2.1} that $\widetilde{\sh{M}}$ can be identified with the restriction of $\sh{M}^\square$ to $C$.
Since the canonical injection $i:C\to\widehat{C}$ is an affine morphism \sref{II.8.3.2}, and \emph{a fortiori} quasi-compact, the \emph{canonical extension} $(\widetilde{\sh{N}})^-$, the largest sub-$\sh{O}_{\widehat{C}}$-module contained in $\sh{M}^\square$ and inducing $\widetilde{\sh{N}}$ on $C$, is a \emph{quasi-coherent} $\sh{O}_{\widehat{C}}$-module \sref[I]{I.9.4.2}.
We will give a more explicit description by using a graded $\widehat{\sh{S}}$-module.
\end{env}

\begin{env}[8.13.2]
\label{II.8.13.2}
For this, consider, for every integer $n\geq0$, the homomorphism $\bigoplus_{i\leq n}\sh{M}_i\to\sh{M}$ which, for every open $U$ of $Y$, sends the family
\[
    (s_i) \in \bigoplus_{i\leq n}\Gamma(U,\sh{M}_i)
\]
to the section $\sum_i s_i\in\Gamma(U,\sh{M})$.
Denote by $\sh{N}'_n$ the inverse image of $\sh{N}$ by this homomorphism, which is a quasi-coherent sub-$\sh{S}$-module of $\bigoplus_{i\leq n}\sh{M}_i$.
Now consider the homomorphism $\bigoplus_{i\leq n}\sh{M}_i\to\widehat{\sh{M}}=\sh{M}[\bb{z}]$ which sends $(s_i)$ to the section $\sum_{i\leq n}s_i\bb{z}^{n-i}\in\Gamma(U,\widehat{\sh{M}}_n)$, and let $\sh{N}_n$ be the image of $\sh{N}'_n$ under this homomorphism;
we immediately have that $\overline{\sh{N}}=\bigoplus_{n\geq0}\sh{N}_n$ is a (quasi-coherent) sub-$\widehat{\sh{S}}$-module of $\widehat{\sh{M}}$;
we say that $\overline{\sh{N}}$ is induced from $\sh{N}$ by \emph{homogenisation}, via the ``homogenising variable'' $\bb{z}$.
We note
\oldpage[II]{196}
that, if $\sh{N}$ is already a \emph{graded} sub-$\sh{S}$-module of $\sh{M}$, then $\overline{\sh{N}}$ can be identified with the direct sum of the components $\widehat{\sh{N}}_n$ of degree $n\geq0$ in $\widehat{\sh{N}}=\sh{N}[\bb{z}]$.
\end{env}

\begin{proposition}[8.13.3]
\label{II.8.13.3}
The $\sh{O}_{\widehat{C}}$-module $\shProj_0(\overline{\sh{N}})$ is the canonical extension $(\widetilde{\sh{N}})^-$ of $\widetilde{\sh{N}}$ to $\widehat{C}$.
\end{proposition}

\begin{proof}
The question is local on $Y$ and $\widehat{C}$ by the definition of the canonical extension \sref[I]{I.9.4.1}.
We can thus already suppose that $Y=\Spec(A)$ is affine, with $\sh{S}=\widetilde{S}$, $\sh{M}=\widetilde{M}$, and $\sh{N}=\widetilde{N}$, where $N$ is a non-necessarily-graded sub-$S$-module of $M$.
Furthermore \sref{II.8.3.2.6}, $\widehat{C}$ is a union of affine opens $\widehat{C}_z=C$ and $\widehat{C}_f=\Spec(S_f^\leq)$ (with $f$ homogeneous in $S_+$).
It thus suffices to show that: (1) the restriction of $\shProj_0(\overline{\sh{N}})$ to $C$ is $\widetilde{\sh{N}}$; (2) the restriction of $\shProj_0(\overline{\sh{N}})$ to each $\widehat{C}_f$ is the canonical extension of the restriction of $\sh{N}$ to $C\cap\widehat{C}_f=\Spec(S_f)$ \sref{II.8.3.2.6}.
For the first point, note that $\shProj_0(\overline{\sh{N}})|C$ can be identified with $(\overline{N}_{(\bb{z})})\supertilde$ \sref{II.8.3.2.4};
but $\overline{N}_{(\bb{z})}$ is canonically identified \sref{II.2.2.5} with the image of $\overline{N}$ in $\widehat{M}/(\bb{z}-1)\widehat{M}$, and by the canonical isomorphism of the latter with $M$ \sref{II.8.2.5}, this image can be identified with $N$, by the definition of $\overline{N}$ given in \sref{II.8.13.2}.

To prove the second point, note that the injection $i:C\cap\widehat{C}_f\to\widehat{C}$ corresponds to the canonical injection $S_f^\leq\to S_f$ \sref{II.8.3.2.6};
we also have that $\Gamma(\widehat{C}_f,\sh{M}^\square)=M_f^\leq$, that $\Gamma(\widehat{C}_f,i_*(\widetilde{\sh{N}}))=N$, and, by \sref{II.8.12.2.1}, that $\Gamma(\widehat{C}_f,i_*(i^*(\sh{M}^\square)))=M_f$.
Taking \sref[I]{I.9.4.2} into account, we are thus led to showing that $\overline{N}_{(f)}\subset\widehat{M}_{(f)}=M_f^\leq$ is canonically identified with the inverse image of $N_f$ under the canonical injection $M_f^\leq\to M_f$.
Indeed, let $d=\deg(f)>0$, and suppose that an element $(\sum_{k\leq md}x_k)/f^m$ of $M_f$ (with $x_k\in M_k$) is of the form $y/f^m$ with $y\in N$.
By multiplying $y$ and the $x_k$ by one single suitable $f^h$, we can already assume that $\sum_{k\leq md}x_k=y$.
But in the identification of \sref{II.8.2.5.2}, $(\sum_{k\leq md}x_k)/f^m$ corresponds to $\sum_{k\leq md}x_k\bb{z}^{md-k}/f^m$, and this is indeed an element of $\overline{N}_{(f)}$, since $\sum_{k\leq md}x_k\in N$;
the converse is evident.
\end{proof}

\begin{remark}[8.13.4]
\label{II.8.13.4}
\begin{enumerate}
    \item[\rm{(i)}] The most important case of application of \sref{II.8.13.3} is that where $\sh{M}=\sh{S}$, with $\widetilde{\sh{N}}$ then being an \emph{arbitrary} quasi-coherent sheaf of ideals $\sh{J}$ of $\sh{O}_C$ \sref{II.1.4.3}, corresponding bijectively to a \emph{closed subprescheme} $Z$ of $C$.
        Then the canonical extension $\overline{\sh{J}}$ of $\sh{J}$ is the quasi-coherent sheaf of ideals of $\sh{O}_{\widehat{C}}$ that defines the \emph{closure} $\overline{Z}$ of $Z$ in $\widehat{C}$ \sref[I]{I.9.5.10};
        Proposition~\sref{II.8.13.3} gives a canonical way of defining $\overline{Z}$ by using a graded ideal in $\widehat{\sh{S}}=\sh{S}[\bb{z}]$.
    \item[\rm{(ii)}] Suppose, to simplify things, that $Y$ is affine, and adopt the notation from the proof of \sref{II.8.13.3}.
        For every non-zero $x\in N$, let $d(x)$ be the largest degree of the homogeneous components $x_i$ of $x$ in $M$;
        by definition, $\overline{N}$ is the submodule of $\widehat{M}$ consisting of $0$ and elements of the form $h(x,k)=\bb{z}^k\sum_{i\leq d(x)}x_i\bb{z}^{d(x)-i}$ (for integral $k\geq0$);
        it is thus generated, as a module over $\widehat{S}=S[\bb{z}]$, by the elements of the form
        \[
            h(x,0) = \sum_{i\leq d(x)}x_i\bb{z}^{d(x)-i}.
        \]
        \oldpage[II]{197}
        We say that $h(x,0)$ is induced from $x$ by \emph{homogenisation} via the ``homogenising variable'' $\bb{z}$.
        But since $h(x,0)$ does not depend additively on $x$ (nor \emph{a fortiori} $S$-linearly), \emph{we will refrain from believing} (even when $M=S$) that the $h(x,0)$ form a \emph{system of generators} of the graded $\widehat{S}$-module $\overline{N}$ when we let $x$ run over a \emph{system of generators} of the $S$-module $N$.
        This is, however, the case (considered only in elementary algebraic geometry) when $N$ is a \emph{free cyclic} $S$-module, since, if $t$ is a basis of $N$, then $h(t,0)$ generates the $\widehat{S}$-module $\overline{N}$.
\end{enumerate}
\end{remark}


\subsection{Supplement on sheaves associated to graded $\mathcal{S}$-modules}
\label{subsection:II.8.14}

\begin{env}[8.14.1]
\label{II.8.14.1}
Let $Y$ be a prescheme, $\sh{S}$ a \emph{positively-graded} quasi-coherent $\sh{O}_Y$-algebra, $X=\Proj(\sh{S})$, and $q:X\to Y$ the structure morphism (which is separated, by \sref{II.3.1.3}).
Using the notation of \sref{II.8.12.1}, we have defined a functor $\sh{M}_X=\shProj(\sh{M})$ in $\sh{M}$, from the category of graded quasi-coherent $\sh{S}$-modules to the category of graded quasi-coherent $\sh{S}_X$-modules;
it is further clear \sref{II.3.2.4} that this is an \emph{additive} and \emph{exact} functor, commuting with inductive limits.

Note, furthermore, that it follows immediately from the definition \sref{II.8.12.1.1} that we have
\[
\label{II.8.14.1.1}
    \shProj(\sh{M}(n)) = (\shProj(\sh{M}))(n)
    \quad\mbox{for all $n\in\bb{Z}$.}
\tag{8.14.1.1}
\]
\end{env}

\begin{env}[8.14.2]
\label{II.8.14.2}
We will first extend the canonical homomorphisms $\lambda$ and $\mu$, defined in \sref{II.3.2.6}, to $\sh{S}_X$-modules of the form $\shProj(\sh{M})$.
For this, note that, for any $m\in\bb{Z}$ and $n\in\bb{Z}$, we have, by \sref{II.2.1.2.1}, a canonical homomorphism of $\sh{O}_X$-modules
\[
\label{II.8.14.2.1}
    \lambda_{mn}:
    \shProj_0((\shHom_{\sh{S}}(\sh{M},\sh{N}))(n-m))
    \to
    \shHom_{\sh{O}_X}(\shProj_0(\sh{M}(m)),\shProj_0(\sh{N}(n)))
\tag{8.14.2.1}
\]
for any graded quasi-coherent $\sh{S}$-modules $\sh{M}$ and $\sh{N}$.
This induces a homomorphism
\[
    \mu_k:
    \shProj_0((\shHom_{\sh{S}}(\sh{M},\sh{N}))(k))
    \to
    (\shHom_{\sh{S}_X}(\shProj(\sh{M}),\shProj(\sh{N})))_k
\]
given by sending every $u\in\Gamma(U,\shProj_0((\shHom_{\sh{S}}(\sh{M},\sh{N}))(k)))$ to the homomorphism $\mu_k(u)$, of degree $k$, of graded $\bb{Z}$-modules $\Gamma(U,\shProj(\sh{M}))\to\Gamma(U,\shProj(\sh{N}))$ (where $U$ is open in $X$) which, in each $\Gamma(U,\shProj_0(\sh{M}(m)))$, agrees with $\mu_{m,m+k}(u)$;
furthermore, by returning to the definition of the $\mu_{mn}$ \sref{II.2.5.12.1}, we immediately see that $\mu_k(u)$ is in fact a homomorphism of degree $k$ of graded $\Gamma(U,\sh{S}_X)$-modules, and, furthermore, that the $\mu_k$ define a homomorphism of \emph{graded $\sh{S}_X$-modules}
\[
\label{II.8.14.2.3}
    \shProj(\shHom_{\sh{S}}(\sh{M},\sh{N}))
    \to
    \shHom_{\sh{S}_X}(\shProj(\sh{M}),\shProj(\sh{N})).
\tag{8.14.2.3}
\]

Similarly, taking the associativity diagram \sref{II.2.5.11.4} into account, the homomorphisms \sref{II.8.14.2.1} give a homomorphism of \emph{graded $\sh{S}_X$-modules}
\[
\label{II.8.14.2.4}
    \lambda:
    \shProj(\sh{M})\otimes_{\sh{S}_X}\shProj(\sh{N})
    \to
    \shProj(\sh{M}\otimes_{\sh{S}}\sh{N}).
\tag{8.14.2.4}
\]
\end{env}

\oldpage[II]{198}

\begin{proposition}[8.14.3]
\label{II.8.14.3}
The homomorphism \sref{II.8.14.2.4} is bijective;
so too is \sref{II.8.14.2.3} whenever the graded $\sh{S}$-module $\sh{M}$ admits a finite presentation \sref{II.3.1.1}.
\end{proposition}

\begin{proof}
The question is clearly local on $X$ and $Y$;
we can thus suppose that $Y=\Spec(A)$ is affine, with $\sh{S}=\widetilde{S}$, $\sh{M}=\widetilde{M}$, and $\sh{N}=\widetilde{N}$, where $S$ is a positively-graded $A$-algebra, and $M$ and $N$ are graded $S$-modules.
If $f$ is a homogeneous element of $S_+$, then the homomorphisms \sref{II.8.14.2.1} and \sref{II.8.14.2.2}, restricted to the affine open $D_+(f)$, correspond to the canonical homomorphisms \sref{II.2.5.11.1} and \sref{II.2.5.12.1}:
\begin{align*}
    M(m)_{(f)}\otimes_{S_{(f)}}N(n)_{(f)}
    &\to
    (M\otimes_S N)(m+n)_{(f)}
\\  (\Hom_S(M,N))(n-m)_{(f)}
    &\to
    \Hom_{S_{(f)}}(M(m)_{(f)},N(n)_{(f)}).
\end{align*}

If we refer to the definitions of these homomorphisms, we thus see (taking \sref{II.8.2.9.1} into account) that the restriction of \sref{II.8.14.2.4} to $D_+(f)$ corresponds to the canonical homomorphism
\[
    M_f\otimes_{S_f}N_f \to (M\otimes_S N)_f
\]
defined in \sref[0]{0.1.3.4}, and we know that this latter homomorphism is an isomorphism.
Similarly, the restriction of \sref{II.8.14.2.3} to $D_+(f)$ corresponds to the canonical homomorphism \sref[0]{0.1.3.5}
\[
    (\Hom_S(M,N))_f \to \Hom_{S_f}(M_f,N_f)
\]
taking into account the fact that, since $M$ is of finite type, the module $\Hom_S(M,N)$, the direct sum of the subgroups consisting of \emph{homogeneous} homomorphisms of $S$-modules \sref{II.2.1.2}, agrees with the set of \emph{all} homomorphisms $M\to N$ of $S$-modules.
The hypothesis that $M$ admits a finite presentation then implies \sref[0]{0.1.3.5} that the canonical homomorphism in question is indeed an isomorphism.
\end{proof}

\begin{proposition}[8.14.4]
\label{II.8.14.4}
If $U$ is a quasi-compact open of $X$, then there exists an integer $d$ such that, for every integer $n$ that is a multiple of $d$, $\sh{O}_X(n)|U$ is invertible, with its inverse being $\sh{O}_X(-n)|U$.
\end{proposition}

\begin{proof}
Since $q(U)$ is quasi-compact, it is covered by a finite number of affine opens $V_i$, and so every $x\in U$ is contained in some affine open of the form $D_+(f)$, where $f$ is a homogeneous element of degree $>0$ of one of the rings $\Gamma(V_i,\sh{S})$.
Since $U$ is quasi-compact, we can cover it by a finite number of such opens $D_+(f_j)$;
let $d$ be a common multiple of the degrees of the $f_j$.
This $d$ satisfies the desired property, by \sref{II.2.5.17}.
\end{proof}

\begin{env}[8.14.5]
\label{II.8.14.5}
With the hypotheses and notation of \sref{II.8.14.1}, we defined, in \sref{II.3.3.2}, canonical homomorphisms of $\sh{O}_Y$-modules
\[
\label{II.8.14.5.1}
    \alpha_n: \sh{M}_n \to q_*(\shProj_0(\sh{M}(n)))
    \qquad (n\in\bb{Z}).
\tag{8.14.5.1}
\]

Generalising the notation of \sref{II.3.3.1}, we set, for every \emph{graded $\sh{S}_X$-module $\sh{F}$},
\[
\label{II.8.14.5.2}
    \boldsymbol{\Gamma}_*(\sh{F}) = \bigoplus_{n\in\bb{Z}}q_*(\sh{F}_n).
\tag{8.14.5.2}
\]

In particular, $\boldsymbol{\Gamma}(\sh{S}_X)=\bigoplus_{n\in\bb{Z}}q_*(\sh{O}_X(n))$ is the graded $\sh{O}_Y$-algebra denoted by $\boldsymbol{\Gamma}_*(\sh{O}_X)$ in \sref{II.3.3.1.2};
it is clear that $\boldsymbol{\Gamma}(\sh{F})$ is a \emph{graded $\boldsymbol{\Gamma}_*(\sh{S}_X)$-algebra} \sref[0]{0.4.2.2}.
When
\oldpage[II]{199}
we take $\sh{M}=\sh{S}$ in the homomorphisms \sref{II.8.14.5.1}, we obtain the homomorphism of graded $\sh{O}_Y$-algebras
\[
\label{II.8.14.5.3}
    \alpha:\sh{S}\to\boldsymbol{\Gamma}(\sh{S}_X)
\tag{8.14.5.3}
\]
previously defined in \sref{II.3.3.2}, and which makes $\boldsymbol{\Gamma}_*(\sh{F})$ a \emph{graded $\sh{S}$-module};
the homomorphisms \sref{II.8.14.5.1} then define a homomorphism (of degree $0$) of \emph{graded $\sh{S}$-modules}
\[
\label{II.8.14.5.4}
    \alpha:\sh{M}\to\boldsymbol{\Gamma}_*(\shProj(\sh{M})).
\tag{8.14.5.4}
\]
\end{env}

\begin{env}[8.14.6]
\label{II.8.14.6}
In general, for a graded quasi-coherent $\sh{S}_X$-module $\sh{F}$, it is not certain that the graded $\sh{S}$-module $\boldsymbol{\Gamma}_*(\sh{F})$ will necessarily be quasi-coherent.
Consider an open $X'$ of $X$ such that the restriction $q':X'\to Y$ of $q$ to $X'$ is a \emph{quasi-compact} morphism.
Since $q'$ is further separated, $q'_*(\sh{F}')$ is then a quasi-coherent $\sh{O}_Y$-module for every quasi-coherent $\sh{O}_{X'}$ module $\sh{F}'$ \sref[I]{I.9.2.2}[b].
We set
\[
\label{II.8.14.6.1}
    \sh{S}_{X'} = \sh{S}_X|X' = \bigoplus_{n\in\bb{Z}}\sh{O}_X(n)|X'
\tag{8.14.6.1}
\]
and, for every graded $\sh{S}_{X'}$-module $\sh{F}'$,
\[
\label{II.8.14.6.2}
    \boldsymbol{\Gamma}'_*(\sh{F}') = \bigoplus_{n\in\bb{Z}}q'_*(\sh{F}'_n).
\tag{8.14.6.2}
\]

The previous remark then shows that, if $\sh{F}'$ is a quasi-coherent $\sh{S}_{X'}$-module, then $\boldsymbol{\Gamma}'_*(\sh{F}')$ is a graded \emph{quasi-coherent} $\sh{S}$-module \sref[I]{I.9.6.1}.

We note also that the canonical injection $j:X'\to X$ is \emph{quasi-compact}, because $q'=q\circ j$ is quasi-compact and $q$ is separated \sref[I]{I.6.6.4}[v].
Then $\sh{F}=j_*(\sh{F}')$ is a graded quasi-coherent $\sh{S}_X$-module for every graded quasi-coherent $\sh{S}_{X'}$-module $\sh{F}$', and it follows from the previous definitions that
\[
\label{II.8.14.6.3}
    \boldsymbol{\Gamma}'_*(\sh{F}') = \boldsymbol{\Gamma}_*(\sh{F}).
\tag{8.14.6.3}
\]

With the same hypotheses on $X'$, for every graded quasi-coherent $\sh{S}$-module $\sh{M}$, we set
\[
\label{II.8.14.6.4}
    \shProj'(\sh{M}) = \shProj(\sh{M})|X'
\tag{8.14.6.4}
\]
which is a graded quasi-coherent $\sh{S}_{X'}$-module.
The canonical homomorphism
\[
    \shProj(\sh{M}) \to j_*(\shProj'(\sh{M}))
\]
\sref[0]{0.4.4.3} thus gives a canonical homomorphism $\boldsymbol{\Gamma}_*(\shProj(\sh{M}))\to\boldsymbol{\Gamma}'_*(\shProj'(\sh{M}))$ of graded $\sh{S}$-modules, and, by composition with \sref{II.8.14.5.4}, we obtain a functorial canonical homomorphism (of degree $0$) of graded quasi-coherent $\sh{S}$-modules
\[
\label{II.8.14.6.5}
    \alpha': \sh{M} \to \boldsymbol{\Gamma}'_*(\shProj'(\sh{M})).
\tag{8.14.6.5}
\]
\end{env}

\begin{env}[8.14.7]
\label{II.8.14.7}
Keeping the hypotheses on $X'$ from \sref{II.8.14.6}, let $\sh{F}'$ be a \emph{graded quasi-coherent $\sh{S}_{X'}$-module} such that $\shProj'(\boldsymbol{\Gamma}'_*(\sh{F}'))$ is also a graded \emph{quasi-coherent} $\sh{S}_{X'}$-module.
\oldpage[II]{200}
We will define a functorial canonical homomorphism (of degree $0$) of graded $\sh{S}_{X'}$-modules
\[
\label{II.8.14.7.1}
    \beta': \shProj'(\boldsymbol{\Gamma}'_*(\sh{F}')) \to \sh{F}'.
\tag{8.14.7.1}
\]

Suppose first of all that $Y=\Spec(A)$ is affine, and that $\sh{S}=\widetilde{S}$, where $S$ is a positively-graded $A$-algebra;
then $\boldsymbol{\Gamma}'_*(\sh{F}')=\widetilde{M}$, where $M=\bigoplus{n\in\bb{Z}}\Gamma(X',\sh{F}'_n)$ is a graded $S$-module.
Let $f\in S_d$ be such that $D_+(f)\subset X'$;
by definition \sref{II.2.6.2}, $\alpha_d(f)$ restricted to $D_+(f)$ is the section of $\sh{O}_X(d)$ over $D_+(f)$ corresponding to the element $f/1$ of $(S(d))_{(f)}$, and is thus invertible;
thus so too is $\alpha_d(f^n)$ for every $n>0$.
From this, we immediately conclude that we have defined an $S_f$-homomorphism (of degree $0$) of graded modules $\beta_f:M_f\to\Gamma(D_+(f),\sh{F}')$ by sending each element $z/f^n\in M_f$ (where $z\in M$) to the section $(z|D_+(f))(\alpha_d(f^n)|D_+(f))^{-1}$ of $\sh{F}'$ over $D_+(f)$.
Furthermore, we have a commutative diagram corresponding to \sref{II.2.6.4.1}, whence the definition of $\beta'$ in this case.
To pass to the general case, we must consider an $A$-algebra $A'$, the graded $A'$-algebra $S'=S\otimes_A A'$, and use the commutative diagram analogous to \sref{II.2.8.13.2};
we leave the details to the reader.
\end{env}

\begin{proposition}[8.14.8]
\label{II.8.14.8}
If $X'$ is an open of $X=\Proj(\sh{S})$ such that $q':X'\to Y$ is quasi-compact, then the homomorphism $\beta'$ defined in \sref{II.8.14.7} is bijective.
\end{proposition}

\begin{proof}
We can clearly restrict to the case where $Y$ is affine, and everything then reduces to proving (with the notation of \sref{II.8.14.7}) that the homomorphism $\beta_f:M_f\to\Gamma(D_+(f),\sh{F}')$ is an isomorphism.
But replacing $f$ by one of its powers changes neither $D_+(f)$ nor $\beta_f$;
since $X'$ is \emph{quasi-compact} by hypothesis, we can always assume, by \sref{II.8.14.4}, that the sheaf $\sh{O}_X(d)$ is \emph{invertible}.
Since $X'$ is a scheme (because $q'$ is separated), the proposition is then exactly \sref[I]{I.9.3.1}.
\end{proof}

\begin{corollary}[8.14.9]
\label{II.8.14.9}
Under the hypotheses of \sref{II.8.14.8}, every graded quasi-coherent $\sh{S}_{X'}$-module is isomorphic to a graded $\sh{S}_{X'}$-module of the form $\shProj'(\sh{M})$, where $\sh{M}$ is a graded quasi-coherent $\sh{S}$-module.
Further, if $\sh{F}'$ is of finite type, and if we assume that $Y$ is a quasi-compact scheme, or a prescheme whose underlying space is Noetherian, then we can assume that $\sh{M}$ is of finite type.
\end{corollary}

\begin{proof}
The proof starting from \sref{II.8.14.8} follows exactly the same route as the proof of \sref{II.3.4.5} starting from \sref{II.3.4.4}, and we leave the details to the reader.
\end{proof}

\begin{proposition}[8.14.10]
\label{II.8.14.10}
Under the hypotheses of \sref{II.8.14.7}, let $\sh{M}$ be a graded quasi-coherent $\sh{S}$-module, and $\sh{F}'$ a graded quasi-coherent $\sh{S}_{X'}$-module;
the composite homomorphisms
\[
\label{II.8.14.10.1}
    \shProj'(\sh{M})
    \xrightarrow{\shProj'(\alpha')}
    \shProj'(\boldsymbol{\Gamma}'_*(\shProj'(\sh{M})))
    \xrightarrow{\beta'}
    \shProj'(\sh{M})
\tag{8.14.10.1}
\]
\[
\label{II.8.14.10.2}
    \boldsymbol{\Gamma}'_*(\sh{F}')
    \xrightarrow{\alpha'}
    \boldsymbol{\Gamma}'_*(\shProj'(\boldsymbol{\Gamma}'_*(\sh{F}')))
    \xrightarrow{\boldsymbol{\Gamma}'_*(\beta')}
    \boldsymbol{\Gamma}'_*(\sh{F}')
\tag{8.14.10.2}
\]
are the identity isomorphisms.
\end{proposition}

\begin{proof}
The question is local on $Y$, and the proof follows as in \sref{II.2.6.5};
we leave the details to the reader.
\end{proof}

\begin{remark}[8.14.11]
\label{II.8.14.11}
In chapter~III \sref[III]{III.2.3.1}, we will see that, when $Y$ is \emph{locally Noetherian}, and $\sh{S}$ is a graded quasi-coherent $\sh{O}_Y$-algebra \emph{of finite type} (in which case
\oldpage[II]{201}
we can take $X'=X$), then the homomorphism $\alpha$ \sref{II.8.14.5.4} is \textbf{(TN)}-\emph{bijective} for every graded quasi-coherent $\sh{S}$-module $\sh{M}$ satisfying condition~\textbf{(TF)}.
\end{remark}

\begin{remark}[8.14.12]
\label{II.8.14.12}
The situation described in \sref{II.8.14.4} is a particular case of the following.
Let $X$ be a ringed space, and $\sh{S}$ a (positively- and negatively-) graded $\sh{O}_X$-algebra;
suppose that there exists an integer $d>0$ such that $\sh{S}_d$ and $\sh{S}_{-d}$ are \emph{invertible}, with the canonical homomorphism
\[
\label{II.8.14.12.1}
    \sh{S}_d\otimes_{\sh{O}_X}\sh{S}_{-d} \to \sh{O}_X
\tag{8.14.12.1}
\]
being an \emph{isomorphism} (such that $\sh{S}_{-d}$ is identified with $\sh{S}_d^{-1}$).
We then say that the graded $\sh{O}_X$-algebra $\sh{S}$ is \emph{periodic}, \emph{of period $d$}.
This nomenclature stems from the following property:
\emph{under the preceding hypotheses, for every graded $\sh{S}$-module $\sh{F}$, the canonical homomorphism}
\[
\label{II.8.14.12.2}
    \sh{S}_d\otimes\sh{F}_n \to \sh{F}_{n+d}
\tag{8.14.12.1}
\]
\emph{is an isomorphism for all $n\in\bb{Z}$.}
Indeed, the question is local on $X$, and we can assume that $\sh{S}_d$ has an \emph{invertible} section $s$ over $X$, with its inverse $s'$ being a section of $\sh{S}_{-d}$.
The homomorphism $\sh{F}_{n+d}\to\sh{S}_d\otimes\sh{F}_n$, which sends each section $z\in\Gamma(U,\sh{F}_{n+d})$ to the section $(s|U)\otimes(s'|U)z$ of $\sh{S}_d\otimes\sh{F}_n$ over $U$, is then the inverse of \sref{II.8.14.12.2}, whence our claim.
This induces, for all $k\in\bb{Z}$, a canonical isomorphism
\[
    (\sh{S}_d)^{\otimes k}\otimes\sh{F}_n \xrightarrow{\sim} \sh{F}_{n+kd}.
\]
Then \emph{the data of a graded $\sh{S}$-module $\sh{F}$ is equivalent to the data of $\sh{S}_0$-modules $\sh{F}_i$ ($0\leq i\leq d-1$) and canonical homomorphisms}
\[
    \sh{S}_i\otimes\sh{F}_j \to \sh{F}_{i+j}
    \qquad
    \mbox{for $0\leq i,j\leq d-1$}
\]
(setting $\sh{F}_{i+j}=\sh{S}_d\otimes_{\sh{S}_0}\sh{F}_{i+j-d}$ whenever $i+j\geq d$).
Of course, for theses homomorphisms to give a well-defined $\sh{S}$-module structure on the direct sum of the $(\sh{S}_d)^{\otimes k}\otimes\sh{F}_i$ ($k\in\bb{Z}$, $0\leq i\leq d-1$), they should satisfy some associativity conditions that we will not explain.

In the case where $d=1$ (which is the one considered in \sref{subsection:II.3.3}), we can thus say that the category of graded $\sh{S}$-modules (resp. quasi-coherent $\sh{S}$-modules if $X$ is a prescheme and $\sh{S}$ is quasi-coherent) is \emph{equivalent} to the category of arbitrary $\sh{S}_0$-modules (resp. quasi-coherent $\sh{S}_0$-modules);
it is in this way that we can think of the results of this paragraph as generalising those of §3.
Furthermore, we see that, under suitable finiteness conditions, the results of this paragraph (along with \sref{II.8.14.11}) reduces, in some sense, the study of graded quasi-coherent algebras on a prescheme, and graded modules ``modulo \textbf{(TN)}'' on such algebras, to the study of the particular case where the algebras in question are \emph{periodic} (and where condition~\textbf{(TN)} for $\sh{M}$ \sref{II.3.4.2} thus implies that $\sh{M}=0$).
\end{remark}

\begin{remark}[8.14.13]
\label{II.8.14.13}
Under the hypotheses of \sref{II.8.14.1}, let $d$ be an integer $>0$;
we have defined a canonical $Y$-isomorphism $h$ from $X$ to $X^{(d)}=\Proj(\sh{S}^{(d)})$ \sref{II.3.1.8}.
For every
\oldpage[II]{202}
graded quasi-coherent $\sh{S}$-module $\sh{M}$ and every integer $k$ such that $0\leq k\leq d-1$, we also have (with the notation of \sref{II.3.1.1}) a canonical $h$-isomorphism
\[
\label{II.8.14.13.1}
    (\shProj(\sh{M}))^{(d,k)} \xleftarrow{\sim} \shProj(\sh{M}^{(d,k)}).
\tag{8.14.13.1}
\]

Suppose, first of all, that $Y=\Spec(A)$ is affine, $\sh{S}=\widetilde{S}$, and $\sh{M}=\widetilde{M}$, where $S$ is a positively-graded $A$-algebra, and $M$ a graded $S$-module.
We know, for every $f\in S_e$ ($e>0$), that $h$ sends $D_+(f)$ to $D_+(f^d)$, and corresponds to the canonical isomorphism $S_{(f^d)}\to S_{(f)}$ \sref{II.2.2.2}.
The restriction of \sref{II.8.14.13.1} to $D_+(f^d)$ then corresponds to the canonical di-isomorphism $M_{f^d}\to M_f$ restricted to the elements of $M_{f^d}$ whose degree is congruent to $k$ (modulo $d$).
We leave to the reader the task of showing that these isomorphisms are compatible with passing from $f$ to some homogeneous multiple $fg$, and then that there is an analogous compatibility with passing from $S$ to a graded $A'$-algebra $S'=S\otimes_A A'$, where $A'$ is some $A$-algebra.
In particular, this gives us an $h$-isomorphism
\[
\label{II.8.14.13.2}
    (\sh{S}^{(d)})_{X^{(d)}} \xrightarrow{\sim} (\sh{S}_X)^{(d)}
\tag{8.14.13.2}
\]
that respects the multiplicative structures of both the source and the target, and that, thanks to \sref{II.8.14.13.1}, becomes an $h$-di-isomorphism from a graded $(\sh{S}^{(d)})_{X^{(d)}}$-module to a graded $(\sh{S}_X)^{(d)}$-module.
Similarly, we have an $h$-isomorphism
\[
\label{II.8.14.13.3}
    \shProj_0(\sh{S}^{(d,k)}(n)) \xrightarrow{\sim} \sh{O}_X(nd+k),
\tag{8.14.13.3}
\]
which completes the result of \sref{II.3.2.9}[ii].

The isomorphism in \sref{II.8.14.13.1} immediately induces an isomorphism of graded $\sh{S}^{(d)}$-modules
\[
\label{II.8.14.13.4}
    \boldsymbol{\Gamma}_*^{(d)}(\shProj(\sh{M}^{(d,k)})) \xrightarrow{\sim} \boldsymbol{\Gamma}_*((\shProj(\sh{M}))^{(d,k)})
\tag{8.14.13.4}
\]
where $\boldsymbol{\Gamma}_*^{(d)}$ corresponds to the structure morphism $q^{(d)}:X^{(d)}\to Y$;
it can be immediately verified that the canonical homomorphism $\alpha$ \sref{II.8.14.5.4}, and the analogous homomorphism $\alpha^{(d)}$ for $X^{(d)}$, make the following diagram commute:
\[
\label{II.8.14.13.5}
    \xymatrix{
        &\sh{M}^{(d,k)} \ar[dl]_{\alpha^{(d)}} \ar[dr]^{\alpha}&
    \\  \boldsymbol{\Gamma}_*^{(d)}(\shProj(\sh{M}^{(d,k)})) \ar[rr]^{\sim}
        && \boldsymbol{\Gamma}_*((\shProj(\sh{M}))^{(d,k)})
    }
\tag{8.14.13.5}
\]
where we proceed by supposing that $Y$ is affine and then calculating the restrictions of the images under $\alpha^{(d)}$ and $\alpha$ of some single element of $M^{(d,k)}$ to the open subsets $D_+(f^d)$ and $D_+(f)$ (using the same notation as above).
\end{remark}

\begin{proposition}[8.14.14]
\label{II.8.14.14}
Let $Y$ be a quasi-compact prescheme, $\sh{S}$ a graded quasi-coherent $\sh{O}_Y$-algebra of finite type, and $\sh{M}$ a graded quasi-coherent $\sh{S}$-module satisfying condition~\textbf{(TF)};
let $X=\Proj(\sh{S})$.
Then $\sh{S}_X$ is a periodic graded $\sh{O}_X$-algebra \sref{II.8.14.12}, and there exists
\oldpage[II]{203}
some period $d$ of $\sh{S}_X$ such that the $(\shProj(\sh{M}))^{(d,k)}$ ($0\leq k\leq d-1$) are $(\sh{S}_X)^{(d)}$-modules of finite type.
\end{proposition}

\begin{proof}
Indeed, \sref{II.3.1.10} proves that there exists some $d$ such that $\sh{S}^{(d)}$ is generated by $\sh{S}_d=(\sh{S}^{(d)})_1$, with the latter being an $\sh{S}_0$-module of finite type.
To prove the first claim, we can thus, by \sref{II.8.14.13.2}, restrict to the case where $d=1$, and the proposition then follows from \sref{II.3.2.7}.
Furthermore, taking \sref{II.8.14.13.1} into account, the second claim is a consequence of \sref{II.2.1.6}[iii] and \sref{II.3.4.3}.
\end{proof}
