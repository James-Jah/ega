\section{Blowup schemes; projective cones; projective closure}
\label{section:2.8}


\subsection{Blowup preschemes}
\label{subsection:2.8.1}

\begin{env}[8.1.1]
\label{2.8.1.1}
Let $Y$ be a prescheme, and, for every integer $n\geq 0$, let $\sh{I}_n$ be a quasi-coherent sheaf of ideals of $\sh{O}_Y$; suppose that the following conditions are satisfied:
\[
\label{eq:2.8.1.1.1}
  \sh{I}_0=\sh{O}_Y,\ \sh{I}_n\subset\sh{I}_m\text{ for }m\leq n,
\tag{8.1.1.1}
\]
\[
\label{eq:2.8.1.1.2}
  \sh{I}_m\sh{I}_n\subset\sh{I}_{m+n}\text{ for any }m,n.
\tag{8.1.1.2}
\]

\oldpage[II]{153}
We note that these hypotheses imply
\[
\label{eq:2.8.1.1.3}
  \sh{I}_1^n\subset\sh{I}_n.
\tag{8.1.1.3}
\]

Set
\[
\label{eq:2.8.1.1.4}
  \sh{S}=\bigoplus_{n\geq 0}\sh{I}_n.
\tag{8.1.1.4}
\]

It follows from \eref{eq:2.8.1.1.1} and \eref{eq:2.8.1.1.2} that $\sh{S}$ is a quasi-coherent graded $\sh{O}_Y$-algebra, and thus defines a $Y$-scheme $X=\Proj(\sh{S})$.
If $\sh{J}$ is an \emph{invertible} sheaf of ideals of $\sh{O}_Y$, then $\sh{I}_n\otimes_{\sh{O}_Y}\sh{J}^{\otimes n}$ is canonically identified with $\sh{I}_n\sh{J}^n$.
If we then replace the $\sh{I}_n$ by the $\sh{I}_n\sh{J}^n$, and, in doing so, replace $\sh{S}$ by a quasi-coherent $\sh{O}_Y$-algebra $\sh{S}_{(\sh{J})}$, then $X_{(\sh{J})}=\Proj(\sh{S}_{(\sh{J})})$ is canonically isomorphic to $X$ \sref{2.3.1.8}.
\end{env}

\begin{env}[8.1.2]
\label{2.8.1.2}
Suppose that $Y$ is \emph{locally integral}, so that the sheaf $\sh{R}(Y)$ of rational functions is a quasi-coherent $\sh{O}_Y$-algebra \sref[1]{1.7.3.7}.
We say that a sub-$\sh{O}_Y$-module $\sh{I}$ of $\sh{R}(Y)$ is a \emph{fractional ideal} of $\sh{R}(Y)$ if it is of \emph{finite type} \sref[0]{0.5.2.1}.
Suppose we have, for all $n\geq0$, a quasi-coherent fractional ideal $\sh{I}_n$ of $\sh{R}(Y)$, such that $\sh{I}_0 = \sh{O}_Y$, and such that condition \eref{eq:2.8.1.1.2} (but not necessarily the second condition \eref{eq:2.8.1.1.1}) is satisfied;
we can then again define a graded quasi-coherent $\sh{O}_Y$-algebra by Equation \eref{eq:2.8.1.1.4}, and the corresponding $Y$-scheme $X = \Proj(\sh{S})$;
we will again have a canonical isomorphism from $X$ to $X_{{\sh{J}}}$ for every \emph{invertible} fractional ideal $\sh{J}$ of $\sh{R}(Y)$.
\end{env}

\begin{definition}[8.1.3]
\label{2.8.1.3}
Let $Y$ be a prescheme (resp. a locally integral prescheme), and $\sh{I}$ a quasi-coherent ideal of $\sh{O}_Y$ (resp. a quasi-coherent fractional ideal of $\sh{R}(Y)$).
We say that the $Y$-scheme $X = \Proj(\bigoplus_{n\geq0}\sh{I}^n)$ is obtained by blowing up the ideal $\sh{I}$, or is the blow-up prescheme of $Y$ relative to $\sh{I}$.
When $\sh{I}$ is a quasi-coherent ideal of $\sh{O}_Y$, and $Y'$ is the closed sub-prescheme of $Y$ defined by $\sh{I}$, we also say that $X$ is the $Y$-scheme obtained by blowing up $Y'$.
\end{definition}

By definition, $\sh{S} = \bigoplus_{n\geq0}\sh{I}^n$ is then generated by $\sh{S}_1 = \sh{I}$;
if $\sh{I}$ is an $\sh{O}_Y$-module of \emph{finite type}, then $X$ is \emph{projective} over $Y$ \sref{2.5.5.2}.
Without any hypotheses on $\sh{I}$, the $\sh{O}_X$-module $\sh{O}_X(1)$ is \emph{invertible} \sref{2.3.2.5} and \emph{very ample}, by \sref{2.4.4.3} applied to the structure morphism $X\to Y$.

We note that, if $j:X\to Y$ is the structure morphism, then the restriction of $f$ to $f^{-1}(Y\setminus Y')$ is an \emph{isomorphism} to $Y\setminus Y'$ whenever $\sh{I}$ is an \emph{ideal of $\sh{O}_Y$} and $Y'$ is the closed sub-prescheme that it defines: indeed, the question being local on $Y$, it suffices to assume that $\sh{I} = \sh{O}_Y$, and our claim then follows from \sref{2.3.1.7}.

If we replace $\sh{I}$ by $\sh{I}^d$ ($d>0$), then the blow-up $Y$-scheme $X$ is replaced by a canonically isomorphic $Y$-scheme $X'$ \sref{2.8.1.1};
similarly, for every \emph{invertible} ideal (resp. \emph{invertible} fractional ideal) $\sh{J}$, the blow-up prescheme $X_{(\sh{J})}$ relative to the ideal $\sh{I}\sh{J}$ is canonically isomorphic to $X$ \sref{2.8.1.1}.

In particular, whenever $\sh{I}$ is an \emph{invertible} ideal (resp. \emph{invertible} fractional ideal), the $Y$-scheme obtained by blowing up $\sh{I}$ is \emph{isomorphic to $Y$} \sref{2.3.1.7}.

\begin{proposition}[8.1.3]
\label{2.8.1.4}
Let $Y$ be an integral prescheme.
\begin{enumerate}
    \item[\rm{(i)}] For every sequence $(\sh{I}_n)$ of quasi-coherent fractional ideals of $\sh{R}(Y)$ that satisfies \eref{eq:2.8.1.1.2}
\oldpage[II]{154}
        and such that $\sh{I}_0 = \sh{O}_Y$, the $Y$-scheme $X=\Proj(\bigoplus_{n\geq0}\sh{I}^n)$ is integral, and the structure morphism $f:X\to Y$ is dominant.
    \item[\rm{(ii)}] Let $\sh{I}$ be a quasi-coherent fractional ideal of $\sh{R}(Y)$, and let $X$ be the $Y$-scheme given by the blow up of $Y$ relative to $\sh{I}$.
        If $\sh{I} \neq 0$, then the structure morphism $f:X\to Y$ is then birational and surjective.
\end{enumerate}
\end{proposition}

\begin{proof}
\label{proof-2.8.1.4}
\begin{enumerate}
    \item[\rm{(i)}] This follows from the fact that $\sh{S} = \bigoplus_{n\geq0}\sh{I}_n$ is an \emph{integral} $\sh{O}_Y$-algebra (\sref{2.3.1.12} and \sref{2.3.1.14}), since, for all $y\in Y$, $\sh{O}_y$ is an integral ring \sref[I]{1.5.1.4}.
    \item[\rm{(ii)}] By (i), $X$ is integral;
        if, furthermore, $x$ and $y$ are the generic points of $X$ and $Y$ (respectively), then we have $f(x) = y$, and it remains to show that $\kres(x)$ is of rank 1 over $\kres(y)$.
        But $x$ is also the generic point of the fibre $f^{-1}(y)$;
        if $\psi$ is the canonical morphism $Z\to Y$, where $Z=\Spec(\kres(y))$, then the prescheme $f^{-1}(y)$ can be identified with $\Proj(\sh{S}')$, where $\sh{S}' = \psi^*(\sh{S})$ \sref{2.3.5.3}.
        But it is clear that $\sh{S}' = \bigoplus_{n\geq0}(\sh{I}_y)^n$, and, since $\sh{I}$ is a quasi-coherent fractional ideal of $\sh{R}(Y)$ that is not zero, $\sh{I}_y \neq 0$ \sref[I]{1.7.3.6}, whence $\sh{I}_y = \kres(y)$;
        then $\Proj(\sh{S}')$ can be identified with $\Spec(\kres(y))$ \sref{2.3.1.7}, whence the conclusion.
\end{enumerate}
\end{proof}

We show a \emph{converse} of \sref{2.8.1.4} in \sref[III]{3.2.3.8}.

\begin{env}[8.1.5]
\label{2.8.1.5}
We return to the setting and notation of \sref{2.8.1.1}.
By definition, the injection homomorphisms $\sh{I}_{n+1}\to\sh{I}_n$ \eref{eq:2.8.1.1.1} define, for every $k\in\bb{Z}$, an injective homomorphism of degree zero of graded $\sh{S}$-modules
\[
\label{eq:2.8.1.5.1}
  u_k: \sh{S}_+(k+1) \to \sh{S}(k);
\tag{8.1.5.1}
\]
since $\sh{S}_+(k+1)$ and $\sh{S}(k+1)$ are canonically \textbf{(TN)}-isomorphic, they give a canonical correspondence between $u_k$ and an injective homomorphism of $\sh{O}_X$-modules \sref{2.3.4.2}:
\[
\label{eq:2.8.1.5.2}
  \widetilde{u}_k: \sh{O}_X(k+1) \to \sh{O}_X(k).
\tag{8.1.5.2}
\]

Recall as well \sref{2.3.2.6} that we have defined canonical homomorphisms
\[
\label{eq:2.8.1.5.3}
  \lambda: \sh{O}_X(h) \otimes_{\sh{O}_X} \sh{O}_X(k) \to \sh{O}_X(h+k)
\tag{8.1.5.3}
\]
and, since the diagram
\[
  \xymatrix{
    \sh{S}(h) \otimes_{\sh{S}} \sh{S}(k) \otimes_{\sh{S}} \sh{S}(l)
      \ar[r]
      \ar[d]
  & \sh{S}(h+k) \otimes_{\sh{S}} \sh{S}(l)
      \ar[d]
  \\\sh{S}(h) \otimes_{\sh{S}} \sh{S}(k+l)
      \ar[r]
  & \sh{S}(h+k+l)
  }
\]
commutes, it follows from the functoriality of the $\lambda$ \sref{2.3.2.6} that the homomorphisms \eref{eq:2.8.1.5.3} define the structure of a \emph{graded quasi-coherent $\sh{O}_X$-algebra} on
\[
\label{eq:2.8.1.5.4}
  \sh{S}_X = \bigoplus_{n\in\bb{Z}}\sh{O}_X(n).
\tag{8.1.5.4}
\]
Furthermore, the diagram
\[
  \xymatrix{
    \sh{S}(h) \otimes_{\sh{S}} \sh{S}(k+1)
      \ar[r]
      \ar[d]_{1\otimes u_k}
  & \sh{S}(h+k+1)
      \ar[d]^{u_{k+h}}
  \\\sh{S}(h) \otimes_{\sh{S}} \sh{S}(k)
      \ar[r]
  & \sh{S}(h+k)
  }
\]
commutes; the functoriality of the $\lambda$ then implies that we have a commutative diagram
\[
  \xymatrix{
    \sh{O}_X(h) \otimes_{\sh{O}_X} \sh{O}_X(k+1)
      \ar[r]^{\lambda}
      \ar[d]_{1\otimes \widetilde{u}_k}
  & \sh{O}_X(h+k+1)
      \ar[d]^{\widetilde{u}_{k+h}}
  \\\sh{O}_X(h) \otimes_{\sh{O}_X} \sh{O}_X(k)
      \ar[r]^{\lambda}
  & \sh{O}_X(h+k)
  }
\]
where the horizontal arrows are the canonical homomorphisms.
We can thus say that the $\widetilde{u}_k$ define an \emph{injective homomorphism} (of degree zero) \emph{of graded $\sh{S}_X$-modules}
\[
\label{eq:2.8.1.5.6}
  \widetilde{u}: \sh{S}_X(1) \to \sh{S}_X.
\tag{8.1.5.6}
\]
\end{env}

\begin{env}[8.1.6]
\label{2.8.1.6}
Keeping the notation from \sref{2.8.1.5}, we now note that, for $n\geq0$, the composite homomorphism $\widetilde{v}_n = \widetilde{u}_{n-1} \circ \widetilde{u}_{n-2} \circ \ldots \circ \widetilde{u}_0$ is an \emph{injective} homomorphism $\sh{O}_X(n) \to \sh{O}_X$;
we denote by $\sh{I}_{n,X}$ its image, which is thus a quasi-coherent ideal of $\sh{O}_X$, \emph{isomorphic} to $\sh{O}_X(n)$.
Furthermore, the diagram
\[
  \xymatrix{
    \sh{O}_X(m) \otimes_{\sh{O}_X} \sh{O}_X(n)
      \ar[r]^{\lambda}
      \ar[d]_{\widetilde{v}_m \otimes \widetilde{v}_n}
  & \sh{O}_X(m+n)
      \ar[d]^{\widetilde{v}_{m+n}}
  \\\sh{O}_X
      \ar[r]_{\id}
  & \sh{O}_X
  }
\]
commutes for $m\geq0$, $n\geq0$.
We thus deduce the following inclusions:
\[
\label{eq:2.8.1.6.1}
  \sh{I}_{0,X} = \sh{O}_X, \quad \sh{I}_{n,X} \subset \sh{I}_{m,X}
  \qquad\mbox{for $0\leq m\leq n$;}
\tag{8.1.6.1}
\]
\[
\label{eq:2.8.1.6.2}
  \sh{I}_{m,X}\sh{I}_{n,X} \subset \sh{I}_{m+n,X}
  \qquad\qquad\mbox{for $m\geq0$, $n\geq0$.}
\tag{8.1.6.2}
\]
\end{env}

\oldpage[II]{156}

\begin{proposition}[8.1.7]
\label{2.8.1.7}
Let $Y$ be a prescheme, $\sh{I}$ a quasi-coherent ideal of $\sh{O}_Y$, and $X = \Proj(\bigoplus_{n\geq0}\sh{I}^n)$ the $Y$-scheme given by blowing up $\sh{I}$.
We then have, for all $n>0$, a canonical isomorphism
\[
\label{eq:2.8.1.7.1}
  \sh{O}_X(n) \xrightarrow{\sim} \sh{I}^n\sh{O}_X = \sh{I}_{n,X}
\tag{8.1.7.1}
\]
(cf. \sref[0]{0.4.3.5}), and thus that $\sh{I}^n\sh{O}_X$ is a very-ample invertible $\sh{O}_X$-module if $n>0$.
\end{proposition}

\begin{proof}
\label{proof-2.8.1.7}
The last claim is immediate, since $\sh{O}_X(1)$ is invertible \sref{2.3.2.5} and very ample for $Y$ by definition (\sref{2.4.4.3} and \sref{2.4.4.9}).
Also by definition, the image of $v_n$ is exactly $\sh{I}^n\sh{S}$, and \eref{eq:2.8.1.7.1} then follows from the exactness of the functor $\widetilde{\sh{M}}$ \sref{2.3.2.4} and from Equation \eref{eq:2.3.2.4.1}.
\end{proof}

\begin{corollary}[8.1.8]
\label{2.8.1.8}
Under the hypotheses of \sref{2.8.1.7}, if $f:X\to Y$ is the structure morphism, and $Y'$ the closed sub-prescheme of $Y$ defined by $\sh{I}$, then the closed sub-prescheme $X' = f^{-1}(Y')$ of $X$ is defined by $\sh{I}\sh{O}_X$ (which is canonically isomorphic to $\sh{O}_X(1)$), from which we obtain a canonical short exact sequence
\[
\label{eq:2.8.1.8.1}
  0 \to \sh{O}_X(1) \to \sh{O}_X \to \sh{O}_{X'} \to 0.
\tag{8.1.8.1}
\]
\end{corollary}

\begin{proof}
\label{proof-2.8.1.8}
This follows from \eref{2.8.1.7.1} and from \sref[I]{1.4.4.5}.
\end{proof}

\begin{env}[8.1.9]
\label{2.8.1.9}
Under the hypotheses of \sref{2.8.1.7}, we can be more precise about the structure of the $\sh{I}_{n,X}$.
Note that the homomorphism
\[
  \widetilde{u}_{-1}: \sh{O}_X \to \sh{O}_X(-1)
\]
canonically corresponds to a section $s$ of $\sh{O}_X(-1)$ over $X$, which we call the \emph{canonical section} (relative to $\sh{I}$) \sref[0]{0.5.1.1}.
In Diagram \eref{eq:2.8.1.5.5}, the horizontal arrows are isomorphisms \sref{2.3.2.7}; by replacing $h$ with $k$, and $k$ with $-1$ in this diagram, we obtain that $\widetilde{u}_k = 1_k \otimes \widetilde{u}_{-1}$ (where $1_k$ denotes the identity on $\sh{O}_X(h)$), or, equivalently, that the homomorphism $\widetilde{u}_k$ is given exactly by \emph{tensoring with the canonical section $s$} (for all $k\in\bb{Z}$).
The homomorphism $\widetilde{u}$ \eref{eq:2.8.1.5.6} can then be understood in the same way.

Thus, for all $n\geq0$, the homomorphism $\widetilde{v}_n: \sh{O}_X(n)\to\sh{O}_X$ is given exactly by tensoring with $s^{\otimes n}$;
we thus deduce:
\end{env}

\begin{corollary}[8.1.10]
\label{2.8.1.10}
With the notation of \sref{2.8.1.8}, the underlying space of $X'$ is the set of $x\in X$ such that $s(x)=0$, where $s$ denotes the canonical section of $\sh{O}_X(-1)$.
\end{corollary}

\begin{proof}
\label{proof-2.8.1.10}
Indeed, if $c_x$ is a generator of the fibre $(\sh{O}_X(1))_x$ at a point $x$, then $s_x\otimes c_x$ is canonically identified with a generator of the fibre of $\sh{I}_{1,X}$ at the point $x$, and is thus invertible if and only if $s_x\not\in\fk{m}_x(\sh{O}_x(-1))_x$, or, equivalently, if and only if $s(x)\neq0$.
\end{proof}

\begin{proposition}
\label{2.8.1.11}
Let $Y$ be an integral prescheme, $\sh{I}$ a quasi-coherent fractional ideal of $\sh{R}(Y)$, and $X$ the $Y$-scheme given by blowing up $\sh{I}$.
Then $\sh{I}\sh{O}_X$ is an invertible $\sh{O}_X$-module that is very ample for $Y$.
\end{proposition}

\begin{proof}
\label{proof-2.8.1.11}
The question being local on $Y$ \sref{2.4.4.5}, we can reduce to the case where $Y=\Spec(A)$, with $A$ some integral ring of ring of fractions $K$, and $\sh{I}=\widetilde{\fk{I}}$, with $\fk{I}$ some fractional ideal of $K$;
there then exists an element $a\neq0$ of $A$ such that $a\fk{I}\subset A$.
Let $S = \bigoplus_{n\geq0}\fk{I}^n$;
the map $x\mapsto ax$ is an $A$-isomorphism from $\fk{I}^{n+1} = (S(1))_n$ to $a\fk{I}^{n+1} = a\fk{I}S_n \subset \fk{I}^n = S_n$,
\oldpage[II]{157}
and thus defines a (TN)-isomorphism of degree zero of graded $S$-modules $S_+(1)\to a\fk{I}S$.
On the other hand, $x\mapsto a^{-1}x$ is an isomorphism of degree zero of graded $S$-modules $a\fk{I}S \xrightarrow{\sim} \fk{I}S$.
We thus obtain, by composition \sref{2.3.2.4}, an isomorphism of $\sh{O}_X$-modules $\sh{O}_X(1) \xrightarrow{\sim} \sh{I}\sh{O}_X$, and, since $S$ is generated by $S_1=\fk{I}$, $\sh{O}_X(1)$ is invertible \sref{2.3.2.5} and very ample (\sref{2.4.4.3} and \sref{2.4.4.9}), whence our claim.
\end{proof}


\subsection{Preliminary results on the localisation of graded rings}
\label{subsection:2.8.2}

\begin{env}[8.2.1]
\label{2.8.2.1}
Let $S$ be a graded ring, but not assumed (for the moment) to be only in positive degree.
We define
\[
\label{eq:2.8.2.1.1}
  S^\geq = \bigoplus_{n\geq0} S_n,
  \qquad
  S^\leq = \bigoplus_{n\leq0} S_n
\tag{8.2.1.1}
\]
\end{env}

% \subsection{Quasi-coherent sheaves on projective cones}
% \label{subsection:2.8.12}

% \begin{env}[8.12.1]
% \label{2.8.12.1}
% Let us take the hypotheses and notation of \sref{2.8.3.1}.
% Let $\sh{M}$ be a \emph{quasi-coherent graded $\sh{S}$-module}; to avoid any confusion, we denote by $\widetilde{\sh{M}}$ the quasi-coherent $\sh{O}_C$-module
% \oldpage[II]{192}
% associated to $\sh{M}$ \sref{2.1.4.3} when $\sh{M}$ is considered as a \emph{nongraded} $\sh{S}$-module, and by $\shProj_0(\sh{M})$ the quasi-coherent $\sh{O}_X$-module associated to $\sh{M}$, $\sh{M}$ being considered this time as a graded $\sh{S}$-module (in other words, the $\sh{O}_X$-module denoted by $\widetilde{\sh{M}}$ in \sref{2.3.2.2}).
% In addition, we set
% \[
% \label{eq:2.8.12.1.1}
%   \sh{M}_X=\shProj_0(\sh{M})=\bigoplus_{n\in\bb{Z}}\shProj_0(\sh{M}(n));
%   \tag{8.12.1.1}
% \]
% the quasi-coherent graded $\sh{O}_X$-algebra $\sh{S}_X$ being defined by \eref{eq:2.8.6.1.1}, $\shProj(\sh{M})$ is equipped with a structure of a \emph{(quasi-coherent) graded $\sh{S}_X$-module}, by means of the canonical homomorphisms \eref{eq:2.3.2.6.1}
% \[
% \label{eq:2.8.12.1.2}
%   \sh{O}_X(m)\otimes_{\sh{O}_X}\shProj_0(\sh{M}(n))\to\shProj_0(\sh{S}(m)\otimes_\sh{S}\sh{M}(n))\to\shProj_0(\sh{M}(m+n)),
%   \tag{8.12.1.2}
% \]
% the verification of the axioms of sheaves of modules being done using the commutative diagram \eref{eq:2.2.5.11.4}.

% If $Y=\Spec(A)$ is affine, $\sh{S}=\widetilde{S}$, and $\sh{M}=\widetilde{M}$, where $S$ is a graded $A$-algebra and $M$ is a graded $S$-module, then, for every homogeneous element $f\in S_+$, we have
% \[
% \label{eq:2.8.12.1.3}
%   \Gamma(X_f,\shProj(\widetilde{M}))=M_f
%   \tag{8.12.1.3}
% \]
% by the definitions and \eref{eq:2.8.2.9.1}.

% Now consider the quasi-coherent graded $\widehat{\sh{S}}$-module
% \[
% \label{eq:2.8.12.1.4}
%   \widehat{\sh{M}}=\sh{M}\otimes_\sh{S}\widehat{\sh{S}}
%   \tag{8.12.1.4}
% \]
% ($\widehat{\sh{S}}$ defined by \eref{eq:2.8.3.1.1}); we deduce a quasi-coherent graded $\sh{O}_{\widehat{C}}$-module $\shProj_0(\widehat{\sh{M}})$, which we will also denote by
% \[
% \label{eq:2.8.12.1.5}
%   \sh{M}^\square=\shProj_0(\widehat{\sh{M}}).
%   \tag{8.12.1.5}
% \]

% It is clear \sref{2.3.2.4} that $\sh{M}^\square$ is an additive functor which is \emph{exact} in $\sh{M}$, commuting with direct sums and with inductive limits.
% \end{env}
