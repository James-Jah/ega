\section{Subpreschemes and immersion morphisms}
\label{section:I.4}

\subsection{Subpreschemes}
\label{subsection:I.4.1}

\begin{env}[4.1.1]
\label{I.4.1.1}
As the notion of a quasi-coherent sheaf \sref[0]{0.5.1.3} is local, a quasi-coherent $\sh{O}_X$-module $\sh{F}$ over a prescheme $X$ can be defined by the following condition: for each affine open $V$ of $X$, $\sh{F}|V$ is isomorphic to the sheaf associated to a $\Gamma(V,\sh{O}_X)$-module \sref{I.1.4.1}.
It is clear that, over a prescheme $X$, the structure sheaf $\sh{O}_X$ is quasi-coherent, and that the kernels, cokernels, and images of homomorphisms of quasi-coherent $\sh{O}_X$-modules, as well as inductive limits and direct sums of quasi-coherent $\sh{O}_X$-modules, are also quasi-coherent (Theorem~\sref{I.1.3.7} and Corollary~\sref{I.1.3.9}).
\end{env}

\begin{proposition}[4.1.2]
\label{I.4.1.2}
Let $X$ be a prescheme, and $\sh{I}$ a quasi-coherent sheaf of ideals of $\sh{O}_X$.
Then the support $Y$ of the sheaf $\sh{O}_X/\sh{I}$ is closed, and if we denote by $\sh{O}_Y$ the restriction of $\sh{O}_X/\sh{I}$ to $Y$, then $(Y,\sh{O}_Y)$ is a prescheme.
\end{proposition}

\begin{proof}
\label{proof-I.4.1.2}
\oldpage[I]{120}
It evidently suffices \sref{I.2.1.3} to consider the case where $X$ is an affine scheme, and to show that, in this case, $Y$ is closed in $X$, and is an \emph{affine scheme}.
Indeed, if $X=\Spec(A)$, then we have $\sh{O}_X=\widetilde{A}$ and $\sh{I}=\widetilde{\mathfrak{I}}$, where $\mathfrak{I}$ is an ideal of $A$ \sref{I.1.4.1};
$Y$ is then equal to the closed subset $V(\mathfrak{I})$ of $X$ and can be identified with the prime spectrum of the ring $B=A/\mathfrak{I}$ \sref{I.1.1.11};
in addition, if $\vphi$ is the canonical homomorphism $A\to B=A/\mathfrak{I}$, then the direct image ${}^a\vphi_*(\widetilde{B})$ is canonically identified with the sheaf $\widetilde{A}/\widetilde{\mathfrak{I}}=\sh{O}_X/\sh{I}$ (Proposition~\sref{I.1.6.3} and Corollary~\sref{I.1.3.9}), which finishes the proof.
\end{proof}

We say that $(Y,\sh{O}_Y)$ is the \emph{subprescheme} of $(X,\sh{O}_X)$ \emph{defined by the sheaf of ideals $\sh{I}$};
this is a particular case of the more general notion of a \emph{subprescheme}:

\begin{definition}[4.1.3]
\label{I.4.1.3}
We say that a ringed space $(Y,\sh{O}_Y)$ is a subprescheme of a prescheme $(X,\sh{O}_X)$ if:
\begin{enumerate}
  \item[1st.] $Y$ is a locally closed subspace of $X$;
  \item[2nd.] if $U$ denotes the largest open subset of $X$ containing $Y$ such that $Y$ is closed in $U$ (\emph{equivalently}, the complement in $X$ of the boundary of $Y$ with respect to $\overline{Y}$), then $(Y,\sh{O}_Y)$ is a subprescheme of $(U,\sh{O}_X|U)$ defined by a quasi-coherent sheaf of ideals of $\sh{O}_X|U$.
\end{enumerate}
We say that the subprescheme $(Y,\sh{O}_Y)$ of $(X,\sh{O}_X)$ is closed if $Y$ is closed in $X$ (in which case $U=X$).
\end{definition}

It follows immediately from this definition and Proposition~\sref{I.4.1.2} that the closed subpreschemes of $X$ are in canonical \emph{bijective correspondence} with the quasi-coherent sheaves of ideals $\sh{J}$ of $\sh{O}_X$, since if two such sheaves $\sh{J}$ and $\sh{J}'$ have the same (closed) support $Y$, and if the restrictions of $\sh{O}_X/\sh{J}$ and $\sh{O}_X/\sh{J}'$ to $Y$ are identical, then we have $\sh{J}'=\sh{J}$.

\begin{env}[4.1.4]
\label{I.4.1.4}
Let $(Y,\sh{O}_Y)$ be a subprescheme of $X$, $U$ the largest open subset of $X$ such that $Y$ is closed (and thus contained) in $U$, and $V$ an open subset of $X$ contained in $U$;
then $V\cap Y$ is closed in $V$.
In addition, if $Y$ is defined by the quasi-coherent sheaf of ideals $\sh{J}$ of $\sh{O}_X|U$, then $\sh{J}|V$ is a quasi-coherent sheaf of ideals of $\sh{O}_X|V$, and it is immediate that the prescheme induced by $Y$ on $Y\cap V$ is the closed subprescheme of $V$ defined by the sheaf of ideals $\sh{J}|V$.
Conversely:
\end{env}

\begin{proposition}[4.1.5]
\label{I.4.1.5}
Let $(Y,\sh{O}_Y)$ be a ringed space such that $Y$ is a subspace of $X$, and there exists a cover $(V_\alpha)$ of $Y$ by open subsets of $X$ such that, for each $\alpha$, $Y\cap V_\alpha$ is closed in $V_\alpha$, and the ringed space $(Y\cap V_\alpha,\sh{O}_Y|(Y\cap V_\alpha))$ is a closed subprescheme of the prescheme induced on $V_\alpha$ by $X$.
Then $(Y,\sh{O}_Y)$ is a subprescheme of $X$.
\end{proposition}

\begin{proof}
\label{proof-I.4.1.5}
The hypotheses imply that $Y$ is locally closed in $X$ and that the largest open $U$ in which $Y$ is closed (and thus contained) contains all the $V_\alpha$;
we can thus reduce to the case where $U=X$ and $Y$ is closed in $X$.
We then define a quasi-coherent sheaf of ideals $\sh{J}$ of $\sh{O}_X$ by taking $\sh{J}|V_\alpha$ to be the sheaf of ideals of $\sh{O}_X|V_\alpha$ which define the closed subprescheme $(Y\cap V_\alpha,\sh{O}_Y|(Y\cap V_\alpha))$, and, for each open subset $W$ of $X$ not intersecting $Y$, $\sh{J}|W=\sh{O}_X|W$.
We see immediately, by Definition~\sref{I.4.1.3} and \sref{I.4.1.4}, that there exists a unique sheaf of ideals $\sh{J}$ satisfying these conditions, and that it defines the closed subprescheme $(Y,\sh{O}_Y)$.
\end{proof}

In particular, the \emph{induced} (by $X$) prescheme on an \emph{open subset} of $X$ is a \emph{subprescheme} of $X$.

\begin{proposition}[4.1.6]
\label{I.4.1.6}
A subprescheme (resp. a closed subprescheme) of a subprescheme
\oldpage[I]{121}
(resp. closed subprescheme) of $X$ is canonically identified with a subprescheme (resp. closed subprescheme) of $X$.
\end{proposition}

\begin{proof}
\label{proof-I.4.1.6}
Since a locally closed subset of a locally closed subspace of $X$ is a locally closed subspace of $X$, it is clear \sref{I.4.1.5} that the question is local and that we can thus suppose that $X$ is affine;
the proposition then follows from the canonical identification of $A/\mathfrak{J}'$ with $(A/\mathfrak{J})/(\mathfrak{J}'/\mathfrak{J})$, where $A$ is some ring, and $\mathfrak{J}$ and $\mathfrak{J}'$ are ideals of $A$ such that $\mathfrak{J}\subset\mathfrak{J}'$.
\end{proof}

In what follows, we will always make use of the above identification.

\begin{env}[4.1.7]
\label{I.4.1.7}
Let $Y$ be a subprescheme of a prescheme $X$, and denote by $\psi$ the canonical injection $Y\to X$ of the \emph{underlying subspaces};
we know that the inverse image $\psi^*(\sh{O}_X)$ is the restriction $\sh{O}_X|Y$ \sref[0]{0.3.7.1}.
If, for each $y\in Y$, we denote by $\omega_y$ the canonical homomorphism $(\sh{O}_X)_y\to(\sh{O}_Y)_y$, then these homomorphisms are the restrictions to stalks of a \emph{surjective} homomorphism $\omega$ of sheaves of rings $\sh{O}_X|Y\to\sh{O}_Y$: indeed, it suffices to check locally on $Y$, that is to say, we can suppose that $X$ is affine and that the subprescheme $Y$ is closed; if in this case $\sh{I}$ is the sheaf of ideals in $\sh{O}_X$ which defines $Y$, then the $\omega_y$ are none other than the restriction to stalks of the homomorphism $\sh{O}_X|Y\to(\sh{O}_X/\sh{I})|Y$.
We have thus defined a \emph{monomorphism of ringed spaces} \sref[0]{0.4.1.1} $j=(\psi,\omega^\flat)$ which is evidently a morphism $Y\to X$ of preschemes \sref{I.2.2.1}, and we call this the \emph{canonical injection morphism}.

If $f:X\to Z$ is a morphism, we then say that the composite morphism $Y\xrightarrow{j}X\xrightarrow{f}Z$ is the \emph{restriction} of $f$ to the subprescheme $Y$.
\end{env}

\begin{env}[4.1.8]
\label{I.4.1.8}
Conforming to the general definitions (T, I, 1.1), we will say that a morphism of preschemes $f:Z\to X$ is \emph{majoré}\footnote{\emph{[Trans.] There doesn't seem to be an English equivalent of this, except for `bounded above', which doesn't make much sense in this context. We would normally just say that `$f$ factors through $j$', but to avoid having to entirely restructure the often-lengthy sentences in the original, we sometimes (but as little as we can) use `majoré'.}} by the injection morphism $j:Y\to X$ of a subprescheme $Y$ of $X$ if $f$ factors as $Z\xrightarrow{g}Y\xrightarrow{j}X$, where $g$ is a morphism of preschemes;
further, $g$ is necessarily \emph{unique} since $j$ is a monomorphism.
\end{env}

\begin{proposition}[4.1.9]
\label{I.4.1.9}
For a morphism $f:Z\to X$ to be factor through an injection morphism $j:Y\to X$, it is necessary and sufficient that $f(Z)\subset Y$ and, for all $z\in Z$, letting $y=f(z)$, that the homomorphism $(\sh{O}_X)_y\to\sh{O}_z$ corresponding to $f$ factor as $(\sh{O}_Z)_y\to(\sh{O}_Y)_y\to\sh{O}_z$ (\emph{or equivalently,} for the kernel of $(\sh{O}_X)_y\to\sh{O}_z$ to contain the kernel of $(\sh{O}_X)_y\to(\sh{O}_Y)_y$).
\end{proposition}

\begin{proof}
\label{proof-I.4.1.9}
The conditions are evidently necessary.
To see that they are sufficient, we can reduce to the case where $Y$ is a \emph{closed} subprescheme of $X$, by replacing $X$ by an open $U$ such that $Y$ is closed in $U$ \sref{I.4.1.3} if necessary;
$Y$ is then defined by a quasi-coherent sheaf of ideals $\sh{I}$ of $\sh{O}_X$.
Let $f=(\psi,\theta)$, and let $\sh{J}$ be the sheaf of ideals of $\psi^*(\sh{O}_X)$, kernel of $\theta^\sharp:\psi^*(\sh{O}_X)\to\sh{O}_Z$;
considering the properties of the functor $\psi^*$ \sref[0]{0.3.7.2}, the hypotheses imply that, for each $z\in Z$, we have $(\psi^*(\sh{I}))_z\subset\sh{J}_z$, and, as a result, that $\psi^*(\sh{I})\subset\sh{J}$.
Thus $\theta^\sharp$ factors as
\[
  \psi^*(\sh{O}_X)\to\psi^*(\sh{O}_X)/\psi^*(\sh{I})=\psi^*(\sh{O}_X/\sh{I})\xrightarrow{\omega}\sh{O}_Z,
\]
the first arrow being the canonical homomorphism.
Let $\psi'$ be the continuous map $Z\to Y$ coinciding with $\psi$;
it is clear that we have ${\psi'}^*(\sh{O}_Y)=\psi^*(\sh{O}_X/\sh{J})$;
on the other hand, $\omega$ is evidently a local homomorphism, so $g=(\psi',\omega^\flat)$ is a morphism
\oldpage[I]{122}
of preschemes $Z\to Y$ \sref{I.2.2.1}, which, according to the above, is such that $f=j\circ g$, hence the proposition.
\end{proof}

\begin{corollary}[4.1.10]
\label{I.4.1.10}
For an injection morphism $Z\to X$ to be factor through the injection morphism $Y\to X$, it is necessary and sufficient for $Z$ to be a subprescheme of $Y$.
\end{corollary}

We then write $Z\leq Y$, and this condition is evidently an \emph{ordering} on the set of subpreschemes of $X$.

\subsection{Immersion morphisms}
\label{subsection:I.4.2}

\begin{definition}[4.2.1]
\label{I.4.2.1}
We say that a morphism $f:Y\to X$ is an immersion (resp. a closed immersion, an open immersion) if it factors as $Y\xrightarrow{g}Z\xrightarrow{j}X$, where $g$ is an isomorphism, $Z$ is a subprescheme of $X$ (resp. a closed subprescheme, a subprescheme induced by an open set), and $j$ is the injection morphism.
\end{definition}

The subprescheme $Z$ and the isomorphism $g$ are then determined in a \emph{unique} way, since if $Z'$ is a second subprescheme of $X$, $j'$ the injection $Z'\to X$, and $g'$ an isomorphism $Y\to Z'$ such that $j\circ g=j'\circ g'$, then we have $j'=j\circ g\circ{g'}^{-1}$, hence $Z'\leq Z$ \sref{I.4.1.10}, and we can similarly show that $Z\leq Z'$, hence $Z'=Z$, and, since $j$ is a monomorphism of preschemes, $g'=g$.

We say that $f=j\circ g$ is the \emph{canonical factorization} of the immersion $f$, and the subprescheme $Z$ and the isomorphism $g$ are those \emph{associated to $f$}.

It is clear that an immersion is a \emph{monomorphism} of preschemes \sref{I.4.1.7} and \emph{a fortiori} a \emph{radicial} morphism \sref{I.3.5.4}.

\begin{proposition}[4.2.2]
\label{I.4.2.2}
\medskip\noindent
\begin{enumerate}
  \item[{\rm(a)}] For a morphism $f=(\psi,\theta):Y\to X$ to be an open immersion, it is necessary and sufficient for $\psi$ to be a homeomorphism from $Y$ to some open subset of $X$, and, for all $y\in Y$, that the homomorphism $\theta_y^\sharp:\sh{O}_{\psi(y)}\to\sh{O}_y$ be bijective.
  \item[{\rm(b)}] For a morphism $f=(\psi,\theta):Y\to X$ to be an immersion (resp. a closed immersion), it is necessary and sufficient for $\psi$ to be a homeomorphism from $Y$ to some locally closed (resp. closed) subset of $X$, and, for all $y\in Y$, that the homomorphism $\theta_y^\sharp:\sh{O}_{\psi(y)}\to\sh{O}_y$ be surjective.
\end{enumerate}
\end{proposition}

\begin{proof}
\label{proof-I.4.2.2}
\medskip\noindent
\begin{enumerate}
  \item[(a)] The conditions are evidently necessary.
    Conversely, if they are satisfied, then it is clear that $\theta^\sharp$ is an isomorphism from $\sh{O}_Y$ to $\psi^*(\sh{O}_X)$, and $\psi^*(\sh{O}_X)$ is the sheaf induced by ``transport of structure'' via $\psi^{-1}$ from $\sh{O}_X|\psi(Y)$;
    hence the conclusion.
  \item[(b)] The conditions are evidently necessary---we prove that they are sufficient.
    Consider first the particular case where we suppose that $X$ is an affine scheme, and that $Z=\psi(Y)$ is \emph{closed} in $X$.
    We then know \sref[0]{0.3.4.6} that $\psi_*(\sh{O}_Y)$ has support equal to $Z$, and that, denoting its restriction to $Z$ by $\sh{O}_Z'$, the ringed space $(Z,\sh{O}_Z')$ is induced from $(Y,\sh{O}_Y)$ by transport of structure via the homeomorphism $\psi$ considered as a map from $Y$ to $Z$.
    Let us now show that $f_*(\sh{O}_Y)=\psi_*(\sh{O}_Y)$ is a \emph{quasi-coherent} $\sh{O}_X$-module.
    Indeed, for all $x\not\in Z$, $\psi_*(\sh{O}_Y)$ restricted to a suitable neighborhood of $x$ is zero.
    On the contrary, if $z\in Z$, then we have $x=\psi(y)$ for some well-defined $y\in Y$;
    let $V$ be an affine open neighborhood of $y$ in $Y$;
    $\psi(V)$ is then open in $Z$, and so equal to the intersection of $Z$ with an open subset $U$ of $X$, and the restriction of $U$ to $\psi_*(\sh{O}_Y)$ is identical to the restriction of $U$ to the direct image
\oldpage[I]{123}
    $(\psi_V)_*(\sh{O}_Y|V)$, where $\psi_V$ is the restriction of $\psi$ to $V$.
    The restriction of the morphism $(\psi,\theta)$ to $(V,\sh{O}_Y|V)$ is a morphism from this aforementioned prescheme to $(X,\sh{O}_X)$, and, as a result, is of the form $({}^a\vphi,\widetilde{\vphi})$, where $\vphi$ is the homomorphism from the ring $A=\Gamma(X,\sh{O}_X)$ to the ring $\Gamma(V,\sh{O}_Y)$ \sref{I.1.7.3};
    we conclude that $(\psi_V)_*(\sh{O}_Y|V)$ is a quasi-coherent $\sh{O}_X$-module \sref{I.1.6.3}, which proves our assertion, due to the local nature of quasi-coherent sheaves.
    In addition, the hypothesis that $\psi$ is a homeomorphism implies \sref[0]{0.3.4.5} that, for all $y\in Y$, $\psi_y$ is an isomorphism $(\psi_*(\sh{O}_Y))_{\psi(y)}\to\sh{O}_y$;
    since the diagram
    \[
      \xymatrix{
        \sh{O}_{\psi(y)}\ar[r]^{\theta_{\psi(y)}}\ar[d]_{\psi_y\circ\rho_{\psi(y)}} &
        (\psi_*(\sh{O}_Y))_{\psi(y)}\ar[d]^{\psi_y}\\
        (\psi^*(\sh{O}_X))_y\ar[r]^{\theta_y^\sharp} &
        \sh{O}_y
      }
    \]
    is commutative, and the vertical arrows are the isomorphisms \sref[0]{0.3.7.2}, the hypothesis that $\theta_y^\sharp$ is surjective implies that $\theta_{\psi(y)}$ is surjective as well.
    Since the support of $\psi_*(\sh{O}_Y)$ is $Z=\psi(Y)$, $\theta$ is a \emph{surjective} homomorphism from $\sh{O}_X=\widetilde{A}$ to the quasi-coherent $\sh{O}_X$-module $f_*(\sh{O}_Y)$.
    As a result, there exists a unique isomorphism $\omega$ from a sheaf quotient $\widetilde{A}/\widetilde{\mathfrak{J}}$ (with $\mathfrak{J}$ being an ideal of $A$) to $f_*(\sh{O}_Y)$ which, when composed with the canonical homomorphism $\widetilde{A}\to\widetilde{A}/\widetilde{\mathfrak{J}}$, gives $\theta$ \sref{I.1.3.8};
    if $\sh{O}_Z$ denotes the restriction of $\widetilde{A}/\widetilde{\mathfrak{J}}$ to $Z$, then $(Z,\sh{O}_Z)$ is a subprescheme of $(X,\sh{O}_X)$, and $f$ factors through the canonical injection of this subprescheme into $X$ and the isomorphism $(\psi_0,\omega_0)$, where $\psi_0$ is $\psi$ considered as a map from $Y$ to $Z$, and $\omega_0$ the restriction of $\omega$ to $\sh{O}_Z$.

    We now pass to the general case.
    Let $U$ be an affine open subset of $X$ such that $U\cap\psi(Y)$ is closed in $U$ and nonempty.
    By restricting $f$ to the prescheme induced by $Y$ on the open subset $\psi^{-1}(U)$, and by considering it as a morphism from this prescheme to the prescheme induced by $X$ on $U$, we reduce to the first case;
    the restriction of $f$ to $\psi^{-1}(U)$ is thus a closed immersion $\psi^{-1}(U)\to U$, canonically factoring as $j_U\circ g_U$, where $g_U$ is an isomorphism from the prescheme $\psi^{-1}(U)$ to a subprescheme $Z_U$ of $U$, and $j_U$ is the canonical injection $Z_U\to U$.
    Let $V$ be a second affine open subset of $X$ such that $V\subset U$;
    since the restriction $Z_V'$ of $Z_U$ to $V$ is a subprescheme of the prescheme $V$, the restriction of $f$ to $\psi^{-1}(V)$ factors as $j_V'\circ g_V'$, where $j_V'$ is the canonical injection $Z_V'\to V$ and $g_V'$ is an isomorphism from $\psi^{-1}(V)$ to $Z_V'$.
    By the uniqueness of the canonical factorization of an immersion \sref{I.4.2.1}, we necessarily have that $Z_V'=Z_V$ and $g_V'=g_V$.
    We conclude \sref{I.4.1.5} that there is a subprescheme $Z$ of $X$ whose underlying space is $\psi(Y)$, and whose restriction to each $U\cap\psi(Y)$ is $Z_U$;
    the $g_U$ are then the restrictions to $\psi^{-1}(U)$ of an isomorphism $g:Y\to Z$ such that $f=j\circ g$, where $j$ is the canonical injection $Z\to X$.
\end{enumerate}
\end{proof}

\begin{corollary}[4.2.3]
\label{I.4.2.3}
Let $X$ be an affine scheme.
For a morphism $f=(\psi,\theta):Y\to X$ to be a closed immersion, it is necessary and sufficient for $Y$ to be an affine scheme, and the homomorphism $\Gamma(\psi):\Gamma(\sh{O}_X)\to\Gamma(\sh{O}_Y)$ to be surjective.
\end{corollary}

\begin{corollary}[4.2.4]
\label{I.4.2.4}
\medskip\noindent
\begin{enumerate}
  \item[{\rm(a)}] Let $f$ be a morphism $Y\to X$, and $(V_\lambda)$ a cover of $f(Y)$ by open subsets of $X$.
    For $f$ to be an immersion (resp. an open immersion), it is necessary and sufficient
\oldpage[I]{124}
    for its restriction to each of the induced preschemes $f^{-1}(V_\lambda)$ to be an immersion (resp. an open immersion) into $V_\lambda$.
  \item[{\rm(b)}] Let $f$ be a morphism $Y\to X$, and $(V_\lambda)$ an open cover of $X$.
    For $f$ to be a closed immersion, it is necessary and sufficient for its restriction to each of the induced preschemes $f^{-1}(V_\lambda)$ to be a closed immersion into $V_\lambda$.
\end{enumerate}
\end{corollary}

\begin{proof}
\label{proof-I.4.2.4}
Let $f=(\psi,\theta)$;
in the case (a), $\theta_y^\sharp$ is surjective (resp. bijective) for all $y\in Y$, and in the case (b), $\theta_y^\sharp$ is surjective for all $y\in Y$;
it thus suffices to check, in case (a), that $\psi$ is a homeomorphism from $Y$ to a locally closed (resp. open) subset of $X$, and, in case (b), that $\psi$ is a homeomorphism from $Y$ to a closed subset of $X$.
Now $\psi$ is evidently injective, and sends each neighborhood of $y$ in $Y$ to a neighborhood of $\psi(y)$ is $\psi(Y)$ for all $y\in Y$, by virtue of the hypothesis;
in case (a), $\psi(Y)\cap V_\lambda$ is locally closed (resp. open) in $V_\lambda$, so $\psi(Y)$ is locally closed (resp. open) in the union of the $V_\lambda$, and \emph{a fortiori} in $X$;
in case (b), $\psi(Y)\cap V_\lambda$ is closed in $V_\lambda$, so $\psi(Y)$ is closed in $X$ since $X=\bigcup_\lambda V_\lambda$.
\end{proof}

\begin{proposition}[4.2.5]
\label{I.4.2.5}
The composition of any two immersions (resp. of two open immersions, of two closed immersions) is an immersion (resp. an open immersion, a closed immersion).
\end{proposition}

\begin{proof}
\label{proof-I.4.2.5}
This follows easily from \sref{I.4.1.6}.
\end{proof}

\subsection{Products of immersions}
\label{subsection:I.4.3}

\begin{proposition}[4.3.1]
\label{I.4.3.1}
Let $\alpha:X'\to X$, $\beta:Y'\to Y$ be two $S$-morphisms; if $\alpha$ and $\beta$ are immersions (resp. open immersions, closed immersions), then $\alpha\times_S\beta$ is an immersion (resp. an open immersion, a closed immersion).
In addition, if $\alpha$ (resp.~$\beta$) identifies $X'$ (resp.~$Y'$) with a subprescheme $X''$ (resp.~$Y''$) of $X$ (resp.~$Y$), then $\alpha\times_S\beta$ identifies the underlying space of $X'\times_S Y'$ with the subspace $p^{-1}(X'')\cap q^{-1}(Y'')$ of the underlying space of $X\times_S Y$, where $p$ and $q$ denote the projections from $X\times_S Y$ to $X$ and $Y$ respectively.
\end{proposition}

\begin{proof}
\label{proof-I.4.3.1}
According to Definition~\sref{I.4.2.1}, we can restrict to the case where $X'$ and $Y'$ are subpreschemes, and $\alpha$ and $\beta$ the injection morphisms.
The proposition has already been proven for the subpreschemes induced by open sets \sref{I.3.2.7};
since every subprescheme is a closed subprescheme of a prescheme induced by an open set \sref{I.4.1.3}, we can reduce to the case where $X'$ and $Y'$ are \emph{closed} subpreschemes.

Let us first show that we can assume $S$ to be \emph{affine}.
Let $(S_\lambda)$ be a cover of $S$ by affine open sets;
if $\vphi$ and $\psi$ are the structure morphisms of $X$ and $Y$, then let $X_\lambda=\vphi^{-1}(S_\lambda)$ and $Y_\lambda=\psi^{-1}(S_\lambda)$.
The restriction $X_\lambda'$ (resp. $Y_\lambda'$) of $X'$ (resp. $Y'$) to $X_\lambda\cap X'$ (resp. $Y_\lambda\cap Y'$) is a closed subprescheme of $X_\lambda$ (resp. $Y_\lambda$), the preschemes $X_\lambda$, $Y_\lambda$, $X_\lambda'$, and $Y_\lambda'$ can be considered as $S_\lambda$-preschemes, and the products $X_\lambda\times_S Y_\lambda$ and $X_\lambda\times_{S_\lambda}Y_\lambda$ (resp. $X_\lambda'\times_S Y_\lambda'$ and $X_\lambda'\times_{S_\lambda}Y_\lambda'$) are identical \sref{I.3.2.5}.
If the proposition is true when $S$ is affine, then the restriction of $\alpha\times_S\beta$ to each of the $X_\lambda'\times_S Y_\lambda'$ is thus an immersion \sref{I.3.2.7}.
Since the product $X_\lambda'\times_S Y_\mu'$ (resp. $X_\lambda\times_S Y_\mu$) can be identified with $(X_\lambda'\cap X_\mu')\times_S(Y_\lambda'\cap Y_\mu')$ (resp. $(X_\lambda\cap X_\mu)\times_S(Y_\lambda\cap Y_\mu)$) \sref{I.3.2.6.4}, the restriction of $\alpha\times_S\beta$
\oldpage[I]{125}
to each of the $X_\lambda'\times_S Y_\mu'$ is also an immersion;
the same is true for $\alpha\times_S\beta$ by \sref{I.4.2.4}.

Next, we show that we can assume $X$ and $Y$ to be \emph{affine}.
Indeed, let $(U_i)$ (resp. $(V_j)$) be a cover of $X$ (resp. $Y$) by affine open sets, and let $X_i'$ (resp. $Y_j'$) be the restriction of $X'$ (resp. $Y'$) to $X'\cap U_i$ (resp. $Y'\cap V_j$), which is a closed subprescheme of $U_i$ (resp. $V_j$);
$U_i\times_S V_j$ can be identified with the restriction of $X\times_S Y$ to $p^{-1}(U_i)\cap q^{-1}(V_j)$ \sref{I.3.2.7};
similarly, if $p'$ and $q'$ are the projections from $X'\times_S Y'$, then $X_i'\times_S Y_j'$ can be identified with the restriction of $X'\times_S Y'$ to ${p'}^{-1}(X_i')\cap{q'}^{-1}(Y_j')$.
Set $\gamma=\alpha\times_S\beta$;
we have, by definition, $p\circ\gamma=\alpha\circ p'$ and $q\circ\gamma=\beta\circ q'$;
since $X_i'=\alpha^{-1}(U_i)$ and $Y_j'=\beta^{-1}(V_j)$, we also have ${p'}^{-1}(X_i')=\gamma^{-1}(p^{-1}(U_i))$ and ${q'}^{-1}(Y_j')=\gamma^{-1}(q^{-1}(V_j))$, hence
\[
  {p'}^{-1}(X_i')\cap{q'}^{-1}(Y_j')=\gamma^{-1}(p^{-1}(U_i)\cap q^{-1}(V_j))=\gamma^{-1}(U_i\times_S V_j),
\]
and we then conclude as in the previous part of the proof.

So suppose $X$, $Y$, and $S$ are affine, and let $B$, $C$, and $A$ be their respective rings.
Then $B$ and $C$ are $A$-algebras, and $X'$ and $Y'$ are affine schemes whose rings are quotient algebras $B'$ and $C'$ of $B$ and $C$ respectively.
In addition, we have $\alpha=({}^a\rho,\widetilde{\rho})$ and $\beta=({}^a\sigma,\widetilde{\sigma})$, where $\rho$ and $\sigma$ are (respectively) the canonical homomorphisms $B\to B'$ and $C\to C'$ \sref{I.1.7.3}.
With that in mind, we know that $X\times_S Y$ (resp. $X'\times_S Y'$) is an affine scheme with ring $B\otimes_A C$ (resp. $B'\otimes_A C'$), and $\alpha\times_S\beta=({}^a\tau,\widetilde{\tau})$, where $\tau$ is the homomorphism $\rho\otimes\sigma$ from $B\otimes_A C$ to $B'\otimes_A C'$ (Proposition~\sref{I.3.2.2} and Corollary~\sref{I.3.2.3});
since this homomorphism is surjective, $\alpha\times_S\beta$ is an immersion.
In addition, if $\mathfrak{b}$ (resp. $\mathfrak{c}$) is the kernel of $\rho$ (resp. $\sigma$), then the kernel of $\tau$ is $u(\mathfrak{b})+v(\mathfrak{c})$, where $u$ (resp. $v$) is the homomorphism $b\mapsto b\otimes 1$ (resp. $c\mapsto 1\otimes c$).
Since $p=({}^a u,\widetilde{u})$ and $q=({}^a v,\widetilde{v})$, this kernel corresponds, in the prime spectrum of $B\otimes_A C$, to the closed set $p^{-1}(X')\cap q^{-1}(Y')$ (\sref{I.1.2.2.1} and Proposition~\sref{I.1.1.2}[iii]), which finishes the proof.
\end{proof}

\begin{corollary}[4.3.2]
\label{I.4.3.2}
If $f:X\to Y$ is an immersion (resp. an open immersion, a closed immersion) and an $S$-morphism, then $f_{(S')}$ is an immersion (resp. an open immersion, a closed immersion) for every extension $S'\to S$ of the base prescheme.
\end{corollary}

\subsection{Inverse images of a subprescheme}
\label{subsection:I.4.4}

\begin{proposition}[4.4.1]
\label{I.4.4.1}
Let $f:X\to Y$ be a morphism, $Y'$ a subprescheme (resp. a closed subprescheme, a prescheme induced by an open set) of $Y$, and $j:Y'\to Y$ the injection morphism.
Then the projection $p:X\times_Y Y'\to X$ is an immersion (resp. a closed immersion, an open immersion);
the underlying space of the subprescheme of $X$ associated to $p$ is $f^{-1}(Y')$;
in addition, if $j'$ is the injection morphism of this subprescheme into $X$, then for a morphism $h:Z\to X$ to be such that $f\circ h:Z\to Y$ factors through $j$, it is necessary and sufficient for $h$ to factor through $j'$.
\end{proposition}

\begin{proof}
\label{proof-I.4.4.1}
Since $p=1_X\times_Y j$ \sref{I.3.3.4}, the first claim follows from Proposition~\sref{I.4.3.1};
the second is a particular case of Corollary~\sref{I.3.5.10} (after swapping the roles of $X$ and $Y'$).
Finally, if we have $f\circ h=j\circ h'$, where $h'$ is a morphism $Z\to Y'$, then it follows from the definition of the product that we have $h=p\circ u$, where $u$ is a morphism $Z\to X\times_Y Y'$, whence the last claim.
\end{proof}

We say that the subprescheme of $X$ thus defined is the \emph{inverse image} of the subprescheme $Y'$ of $Y$ under the morphism $f$, terminology which is consistent with that introduced
\oldpage[I]{126}
more generally in \sref{I.3.3.6}.
When we speak of $f^{-1}(Y')$ as a subprescheme of $X$, this will always be the subprescheme we mean.

When the preschemes $f^{-1}(Y')$ and $X$ are identical, $j'$ is the identity and each morphism $h:Z\to X$ thus factors through $j'$, so \emph{the morphism $f:X\to Y$ factors as $X\xrightarrow{g}Y'\xrightarrow{j}Y$}.

When $y$ is a \emph{closed} point of $Y$ and $Y'=\Spec(\kres(y))$ is the smallest closed subprescheme of $Y$ having $\{y\}$ as its underlying space \sref{I.4.1.9}, the closed subprescheme $f^{-1}(Y')$ is canonically isomorphic to the \emph{fibre} $f^{-1}(y)$ defined in \sref{I.3.6.2}, and we will use this identification in all that follows.

\begin{corollary}[4.4.2]
\label{I.4.4.2}
Let $f:X\to Y$ and $g:Y\to Z$ be morphisms, and $h=g\circ f$ their composition.
For each subprescheme $Z'$ of $Z$, the subpreschemes $f^{-1}(g^{-1}(Z'))$ and $h^{-1}(Z')$ of $X$ are identical.
\end{corollary}

\begin{proof}
\label{proof-I.4.4.2}
This follows from the existence of the canonical isomorphism $X\times_Y(Y\times_Z Z')\isoto X\times_Z Z'$ \sref{I.3.3.9.1}.
\end{proof}

\begin{corollary}[4.4.3]
\label{I.4.4.3}
Let $X'$ and $X''$ be subpreschemes of $X$, and $j':X'\to X$, and $j'':X''\to X$ their injection morphisms;
then ${j'}^{-1}(X'')$ and ${j''}^{-1}(X')$ are both equal to the greatest lower bound $\inf(X',X'')$ of $X'$ and $X''$ for the ordering $\leq$ on subpreschemes, and this is canonically isomorphic to $X'\times_X X''$.
\end{corollary}

\begin{proof}
\label{proof-I.4.4.3}
This follows immediately from Proposition~\sref{I.4.4.1} and Corollary~\sref{I.4.1.10}.
\end{proof}

\begin{corollary}[4.4.4]
\label{I.4.4.4}
Let $f:X\to Y$ be a morphism, and $Y'$ and $Y''$ subpreschemes of $Y$;
then we have $f^{-1}(\inf(Y',Y''))=\inf(f^{-1}(Y'),f^{-1}(Y''))$.
\end{corollary}

\begin{proof}
\label{proof-I.4.4.4}
This follows from the existence of the canonical isomorphism between $(X\times_Y Y')\times_X(X\times_Y Y'')$ and $X\times_Y(Y'\times_Y Y'')$ \sref{I.3.3.9.1}.
\end{proof}

\begin{proposition}[4.4.5]
\label{I.4.4.5}
Let $f:X\to Y$ be a morphism, and $Y'$ a closed subprescheme of $Y$ defined by a quasi-coherent sheaf of ideals $\sh{K}$ of $\sh{O}_Y$ \sref{I.4.1.3};
the closed subprescheme $f^{-1}(Y')$ of $X$ is then defined by the quasi-coherent sheaf of ideals $f^*(\sh{K})\sh{O}_X$ of $\sh{O}_X$.
\end{proposition}

\begin{proof}
\label{proof-I.4.4.5}
The statement is evidently local on $X$ and $Y$;
it thus suffices to note that if \erratum[II]{$A$ is a $B$-algebra} and $\mathfrak{K}$ an ideal of $B$, then we have $A\otimes_B(B/\mathfrak{K})=A/\mathfrak{K}A$, and to then apply \sref{I.1.6.9}.
\end{proof}

\begin{corollary}[4.4.6]
\label{I.4.4.6}
Let $X'$ be a closed subprescheme of $X$ defined by a quasi-coherent sheaf of ideals $\sh{J}$ of $\sh{O}_X$, and $i$ the injection $X'\to X$;
for the restriction $f\circ i$ of $f$ to $X'$ to factor through the injection $j:Y'\to Y$ \emph{(in other words, for it to factor as $j\circ g$, with $g$ a morphism $X'\to Y'$)}, it is necessary and sufficient that $f^*(\sh{K})\subset\sh{J}$.
\end{corollary}

\begin{proof}
\label{proof-I.4.4.6}
It suffices to apply Proposition~\sref{I.4.4.1} to $i$, taking Proposition~\sref{I.4.4.5} into account.
\end{proof}

\subsection{Local immersions and local isomorphisms}
\label{subsection:I.4.5}

\begin{definition}[4.5.1]
\label{I.4.5.1}
Let $f:X\to Y$ be a morphism of preschemes.
We say that $f$ is a local immersion at a point $x\in X$ if there exists an open neighborhood $U$ of $x$ in $X$ and an open neighborhood $V$ of $f(x)$ in $Y$ such that the restriction of $f$ to the induced prescheme $U$ is a closed immersion of $U$ into the induced prescheme $V$.
We say that $f$ is a local immersion if $f$ is a local immersion at each point of $X$.
\end{definition}

\begin{definition}[4.5.2]
\label{I.4.5.2}
We say that a morphism $f:X\to Y$ is a local isomorphism at
\oldpage[I]{127}
a point $x\in X$ if there exists an open neighborhood $U$ of $x$ in $X$ such that the restriction of $f$ to the induced prescheme $U$ is an open immersion of $U$ into $Y$.
We say that $f$ is a local isomorphism if $f$ is a local isomorphism at each point of $X$.
\end{definition}

\begin{env}[4.5.3]
\label{I.4.5.3}
An immersion (resp. a closed immersion) $f:X\to Y$ can be characterized as a local immersion such that $f$ is a homeomorphism from the underlying space of $X$ to a subset (resp. a closed subset) of $Y$.
An open immersion $f$ can be characterized as an \emph{injective} local isomorphism.
\end{env}

\begin{proposition}[4.5.4]
\label{I.4.5.4}
Let $X$ be an irreducible prescheme, and $f:X\to Y$ a dominant injective morphism.
If $f$ is a local immersion, then $f$ is an immersion, and $f(X)$ is open in $Y$.
\end{proposition}

\begin{proof}
\label{proof-I.4.5.4}
Let $x\in X$, and let $U$ be an open neighborhood of $x$, and $V$ an open neighborhood of $f(x)$ in $Y$ such that the restriction of $f$ to $U$ is a closed immersion into $V$;
since $U$ is dense in $X$, $f(U)$ is dense in $Y$ by hypothesis, so $f(U)=V$, and $f$ is a homeomorphism from $U$ to $V$;
the hypothesis that $f$ is injective implies that $f^{-1}(V)=U$, hence the proposition.
\end{proof}

\begin{proposition}[4.5.5]
\label{I.4.5.5}
\medskip\noindent
\begin{enumerate}
  \item[{\rm(i)}] The composition of any two local immersions (resp. of two local isomorphisms) is a local immersion (resp. a local isomorphism).
  \item[{\rm(ii)}] Let $f:X\to X'$ and $g:Y\to Y'$ be two $S$-morphisms.
    If $f$ and $g$ are local immersions (resp. local isomorphisms), then so too is $f\times_S g$.
  \item[{\rm(iii)}] If an $S$-morphism $f$ is a local immersion (resp. a local isomorphism), then so too is $f_{(S')}$ for every extension $S'\to S$ of the base prescheme.
\end{enumerate}
\end{proposition}

\begin{proof}
\label{proof-I.4.5.5}
According to \sref{I.3.5.1}, it suffices to prove (i) and (ii).

(i) follows immediately from the transitivity of closed (resp. open) immersions \sref{I.4.2.5} and from the fact that if $f$ is a homeomorphism from $X$ to a closed subset of $Y$, then for every open $U\subset X$, $f(U)$ is open in $f(X)$, so there exists an open subset $V$ of $Y$ such that $f(U)=V\cap f(X)$, and, as a result, $f(U)$ is closed in $V$.

To prove (ii), let $p$ and $q$ be the projections from $X\times_X Y$, and $p'$ and $q'$ the projections from $X'\times_S Y'$.
There exists, by hypothesis, open neighborhoods $U$, $U'$, $V$, and $V'$ of $x=p(z)$, $x'=p'(z')$, $y=q(z)$, and $y'=q'(z')$ (respectively), such that the restrictions of $f$ and $g$ to $U$ and $V$ (respectively) are closed (resp. open) immersions into $U'$ and $V'$ (respectively).
Since the underlying spaces of $U\times_S V$ and $U'\times_S V'$ can be identified with the open neighborhoods $p^{-1}(U)\cap q^{-1}(V)$ and ${p'}^{-1}(U')\cap{q'}^{-1}(V')$ of $z$ and $z'$ (respectively) \sref{I.3.2.7}, the proposition follows from Proposition~\sref{I.4.3.1}.
\end{proof}

