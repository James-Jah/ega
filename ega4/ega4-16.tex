\section{Differential invariants. Differentially smooth morphisms}
\label{section:IV.16}

\oldpage[IV-4]{5}
In this paragraph we will present, in global form, some notions of differential calculus particularly useful in algebraic geometry.
We will ignore many classic developments in differential geometry (connections, infinitesimal transformations associated to vector fields, jets, etc.), although these notions are translated in a particularly natural way for schemes.
We will similarly ignore phenomena exclusive to characteristic $p>0$ (some of which are seen, in the affine case, in \sref[0]{0.21}.
For certain complements to the differential formalism for preschemes the reader may consult Expos\'es~II and VII of \cite{IV-42} as well as subsequent chapters of this treatise. 

\subsection{Normal invariants of an immersion}
\label{IV.16.1}

\begin{env}[16.1.1]
\label{IV.16.1.1}

Let $(X, \sh{O}_X), (Y, \sh{O}_Y)$ be two ringed spaces and $f = (\psi, \theta): Y \to X$ a morphism of ringed spaces \sref[0]{0.4.1.1} such that the homomorphism
\[
  \theta^\#: \psi^*(\sh{O}_X) \to \sh{O}_Y
\]
is surjective, so that $\sh{O}_Y$ is identified with a sheaf of quotient rings $\psi^*(\sh{O}_X)/\sh{I}_f$. 
We can then endow $\psi^*(\sh{O}_X)$ with the $\sh{I}_f$-preadic filtration.
\end{env}

\begin{definition}[16.1.2]
\label{IV.16.1.2}
The $\sh{O}_Y$-augmented sheaf of rings $\psi^*(\sh{O}_X)/\sh{I}_f^{n+1}$ is called the $n$'th \emph{normal invariant} of $f$;
the ringed space $(Y, \psi^*(\sh{O}_X)/\sh{I}_f^{n+1})$ is called the $n$'th \emph{infinitesimal neighborhood} of $Y$ along $f$ and is denoted by $Y^{(n)}_f$ or simply $Y^{(n)}$.
The sheaf of graded rings associated to the sheaf of filtered rings $\psi^*(\sh{O}_X)$
\[
  \label{IV.16.1.2.1}
  \shGr_\bullet(f) = \bigoplus_{n \geq 0}(\sh{I}_f^{n}/\sh{I}_f^{n+1} )
  \tag{16.1.2.1}
\]
is called the sheaf of graded rings \emph{associated to} $f$. The sheaf $\shGr_1(f) = \sh{I}_f/\sh{I}_f^{2}$ is called the \emph{conormal sheaf} of $f$ (that will be denoted by $\sh{N}_{Y/X}$ when there is no risk of confusion). 
\end{definition}

It is clear that the $\sh{O}_{Y^{(n)}} = \psi^*(\sh{O}_X)/\sh{I}_f^{n+1}$ (that we also denote $\sh{O}_{Y_f^{(n)}})$ form a
\oldpage[IV-4]{6}
projective system of sheaves of rings on $Y$, the transition homomorphism $\phi_{nm}:\sh{O}_{Y^{(m)}} \to \sh{O}_{Y^{(n)}}$ for $n \leq m$ identifies $\sh{O}_{Y^{(n)}}$ with the quotient of $\sh{O}_{Y^{(m)}}$ by the power $(\sh{I}_f/\sh{I}_f^{n+1} )^m$ of the \emph{agumentation ideal} of $\sh{O}_{Y^{(n)}}$, kernel of $\phi_{0n}: \sh{O}_{Y^{(n)}} \to \sh{O}_{Y}$.
The $Y^{(n)}$ therefore form a inductive system of ringed spaces, all having underlying space $Y$, and we have canonical morphisms of ringed spaces $h_n: Y^{(n)} \to X$ equal to $(\psi, \theta_n)$, where $\theta^\#_n$ is the canonical morphism $\psi^*(\sh{O}_X) \to \psi^*(\sh{O}_X)/\sh{I}_f^{n+1}$.
It is clear that the sheaf $\shGr_\bullet(f)$ is a sheaf of graded algebras over the sheaf of rings $\sh{O}_Y = \shGr_0(f)$ and the $\shGr_k(f)$ of $\sh{O}_Y$-modules.

As with every sheaf of filtered rings, we have a \emph{canonical surjective homomorphism} of graded $\sh{O}_Y$-algebras
\[
  \label{IV.16.1.2.2}
  \bb{S}_{\sh{O}_Y}^\bullet(\shGr_1(f)) \to \shGr_\bullet(f)
  \tag{16.1.2.2}
\]
which coincide in degrees $0$ and $1$ with the identities.

\begin{examples}[16.1.3]
\label{IV.16.1.3}
\begin{enumerate}
  \item[(i)] Suppose that $X$ is a locally ringed space, $Y$ is reduced to a single point $y$ (endowed with a ring $\sh{O}_y$) and that, if $x = \psi(y)$, $\theta^\#:\sh{O}_x \to \sh{O}_y$ is a \emph{surjective} homomorphism of rings having as kernel the maximal ideal $\mathfrak{m}_x$ of $\sh{O}_x$.
  So the $\sh{O}_{Y^{(n)}}$ are identified with the rings $\sh{O}_x/\mathfrak{m}_x^{n+1}$ and $\shGr_\bullet(f)$ with the graded ring associated with the local ring $\sh{O}_x$ endowed with the $\mathfrak{m}_x$-preadic filtration.
  \item[(ii)] Suppose that $Y$ is a closed subset of an open subspace $U$ of $X$ and that the $\sh{O}_Y$ is induced on $Y$ by a quotient sheaf $\sh{O}_U/\sh{I}$, where $\sh{I}$ is an ideal of $\sh{O}_U$ such that $\sh{I}_x = \sh{O}_x$ for every $x \not\in Y$;
  if $X$ is a locally ringed space we also suppose that $\sh{I}_x \neq \sh{O}_x$ for $y \in Y$ so that $(Y, \sh{O}_Y)$ is a locally ringed space.
  
  Let $\psi_0: Y \to U$ be the canonical injection and denote by $\theta_0: \sh{O}_U \to (\psi_0)_*(\sh{O}_Y)$ the homomorphism such that $\theta_0^\#$ is the canonical homomorphism $\psi^*_0(\sh{O}_U) = \sh{O}_U|Y \to (\sh{O}_U/\sh{I})|Y$, so that $j_0=(\psi_0, \theta_0):Y \to U$ is a morphism of ringed spaces (and of locally ringed spaces if $X$ is a locally ringed space);
  if $i:U \to X$ is the canonical injection (morphism of ringed spaces), $j = i\circ j_0$ is the morphism $(\psi, \theta)$ of $Y$ to $X$ where $\psi: Y \to X$ is the canonical injection and $\theta:\sh{O}_X \to \psi_*(\sh{O}_Y)$ is the homomorphism such that $\theta^\# = \theta_0^\#$.
  Since $\theta^\#$ is surjective we can apply the previous definitions;
  $\sh{O}_{Y^{(n)}}$ is equal to $\psi^*_0(\sh{O}_U/\sh{I}^{n+1})$, and we have $(\psi_0)_*(\sh{O}_{Y^{(n)}} ) = \sh{O}_U/\sh{I}^{n+1}$, and $\shGr_n(j) = \shGr_n(j_0) = \psi^*_0(\sh{I}^n/\sh{I}^{n+1}) = j^*_0(\sh{I}^n/\sh{I}^{n+1})$.
\end{enumerate}
\end{examples}

\begin{env}[16.1.4]
\label{IV.16.1.4}
The example \sref{IV.16.1.3}[ii] shows that in general the $\sh{O}_{Y^{(n)}}$ are \emph{not canonically endowed with a structure of $\sh{O}_Y$-module}, or \emph{a fortiori} with a structure of $\sh{O}_Y$-algebra.
The data of such structure is equivalent to the data of a homomorphism of sheaves of rings $\lambda_n:\sh{O}_Y \to \sh{O}_{Y^{(n)}}$, right inverse to the augmentation morphism $\phi_{0n}$;
it is also equivalent to the data of a morphism of ringed spaces $(I_Y, \lambda_n): Y^{(n)} \to Y$ right inverse to the canonical morphism $(I_Y, \phi_{0n}): Y \to Y^{(n)}$.
\end{env}

\begin{proposition}[16.1.5]
\label{IV.16.1.5}
Let $f = (\psi, \theta): Y \to X$ be an immersion of preschemes. We have:
\begin{enumerate}
  \item[(i)] $\shGr_\bullet(f)$ is a quasi-coherent graded $\sh{O}_Y$-algebra.
\oldpage[IV-4]{7}
  \item[(ii)] The $Y^{(n)}$ are preschemes, canonically isomorphic to subpreschemes of $X$.
  \item[(iii)] Every homomorphism of sheaves of rings $\lambda_n: \sh{O}_Y \to \sh{O}_{Y^{(n)}}$, right inverse to the augmentation homomorphism $\phi_{0n}$, makes the $\sh{O}_{Y^{(n)}}$ and $\sh{O}_{Y^{(k)}}$ for $k\leq n$ quasi-coherent $\sh{O}_Y$-algebras;
  the structure of $\sh{O}_Y$-module deducted from the preceding structures on the $\shGr_k(f)$ for $k \leq n$ coincide with the ones defined in \sref{IV.16.1.2}.
\end{enumerate}
\end{proposition}

\begin{proof}
(i) Since the question is local on $X$ and $Y$, we can reduce to the case where $Y$ is a closed subpreschemes of $X$ defined by an quasi-coherent ideal $\sh{I}$ of $\sh{O}_X$;
since $\sh{O}_Y$ is the restriction to $Y$ of $\sh{O}_X/\sh{I}$ the assertion (i) is evident, and $Y^{(n)}$ is the closed subprescheme of $X$ defined by the quasi-coherent ideal $\sh{I}^{n+1}$ of $\sh{O}_X$.
Finally, to prove (iii) we notice that the data of $\lambda_n$ makes the ideal $\sh{I}/\sh{I}^n$ of the augmentation $\phi_{0n}$ and their quotients $\sh{I}/\sh{I}^{k+1} (1\leq k \leq n)$ $\sh{O}_Y$-modules, and it suffices to prove by induction on $k$ that the $\sh{I}/\sh{I}^{k+1}$ are quasi-coherent $\sh{O}_Y$-modules and the structure of quotient $\sh{O}_Y$-module induced on $\sh{I}^k/\sh{I}^{k+1}$ is the same as defined on \sref{IV.16.1.2}.
The second assertion is immediate, $\sh{I}^k/\sh{I}^{k+1}$ being killed by $\sh{I}/\sh{I}^{n+1}$;
the first result, by induction on $k$, is trivial for $k=1$ and for $\sh{I}/\sh{I}^{k+1}$ being an extension of $\sh{I}/\sh{I}^{k}$ by $\sh{I}^k/\sh{I}^{k+1}$ \hyperref[section:III.1.4.17]{(\textbf{III}, 1.4.17)}.
\end{proof}

\begin{corollary}[16.1.6]
\label{IV.16.1.6}
Under the general hypothesis of \sref{IV.16.1.5}, if the immersion $f$ is locally of finite presentation then the $\shGr_n(f)$ are quasi-coherent $\sh{O}_Y$-modules of finite type.
\end{corollary}

\begin{proof}
Indeed, with the notation from the proof of \sref{IV.16.1.5}, $\sh{I}$ is an ideal of finite type of $\sh{O}_X$ \sref{IV.1.4.7}, therefore the $\sh{I}^n/\sh{I}^{n+1}$ are $\sh{O}_Y$-modules of finite type, hence the conclusion.
\end{proof}

\begin{corollary}[16.1.7]
\label{IV.16.1.7}
Under the general hypotheses of \sref{IV.16.1.5}, let $g:X \to Y$ be a morphism of preschemes, left inverse to $f$.
Therefore, for every $n$, the composite morphism $(I, \lambda_n): Y^{(n)}\xrightarrow{h_n} X \xrightarrow{g} Y$ defines a homomorphism of sheaves of rings $\lambda_n: \sh{O}_Y \to \sh{O}_{Y^{(n)}}$ right inverse to the augmentation $\phi_{0n}$, making $\sh{O}_{Y^{(n)}}$ a quasi-coherent $\sh{O}_Y$-algebra;
via these homomorphisms, the transition homomorphism $\phi_{nm}:\sh{O}_{Y^{(m)}} \to \sh{O}_{Y^{(n)}}$ ($n\leq m$) are homomorphisms of $\sh{O}_Y$-algebras. 
Also, if $g$ is locally of finite type, then the $\sh{O}_{Y^{(n)}}$ are quasi-coherent $\sh{O}_Y$-modules of finite type.
\end{corollary}

\begin{proof}
The first assertion is an immediate result from the definitions and \sref{IV.16.1.5}.
On the other hand, if $g$ is locally of finite type, then $f$ is locally of finite presentation \sref{IV.1.4.3}[(v)];
the $\shGr_n(f)$ being then quasi-coherent $\sh{O}_Y$-modules of finite type by \sref{IV.16.1.6}, the same goes for the $\sh{O}_Y$-modules $\sh{I}/\sh{I}^{n+1}$, being extensions of a finite number of the $\shGr_k(f)$ \sref[III]{III.1.4.17}.
\end{proof}

\begin{proposition}[16.1.8]
\label{IV.16.1.8}
Let $X$ be a locally Noetherian prescheme, $j:Y \to X$ an immersion;
Then the $Y^{(n)}$ are locally Noetherian preschemes, the $\shGr_n(j)$ are coherent $\sh{O}_Y$-modules and the $\shGr_\bullet(j)$ is a coherent sheaf of rings over the space $Y$.
\end{proposition}

\begin{proof}
Everything is local on $X$ and $Y$, so we reduce to the case where $X$ is affine and $j$ is a closed immersion and therefore all the assertions are evident except for the last, which results from the fact that if $A$ is a Noetherian ring and $\mathfrak{I}$ is an ideal of $A$, $\gr_\mathfrak{I}^\bullet(A)$ is a Noetherian ring, taking into account the exactness of the functor $\psi^*$ and \sref[0]{0.5.3.7}.
\end{proof}

\begin{proposition}[16.1.9]
\label{IV.16.1.9}
\oldpage[IV-4]{8}
Let $X$ be a prescheme, $j: Y \to X$ an immersion locally of finite presentation, $y$ a point of $Y$. The following conditions are equivalent:
\begin{enumerate}
  \item[(a)] There exists an open neighborhood $U$ of y in $Y$ such that $j|U$ is a homeomorphism of $U$ onto an open set of $X$.
  \item[(b)] There is an integer $n>0$ such that the canonical homomorphism
  \[
    (\phi_{n-1,n})_y: \sh{O}_{Y^{(n)},y} \to \sh{O}_{Y^{(n-1)},y}
  \]
  is bijective.
  \item[(c)] There is an integer $n>0$ such that $(\shGr_n(j))_y = 0$.
  
  In addition, if the integer $n$ satisfies \emph{(b)} or \emph{(c)}, then there is a neighborhood $V$ of $y$ in $Y$ such that $\shGr_m(j)|V = 0$ for $m \geq n$ and that $\phi_{nm}|V: \sh{O}_{Y^{(m)}}|V \to \sh{O}_{Y^{(n)}}|V$ is bijective for $m \geq n$. 
\end{enumerate}
\end{proposition}

\begin{proof}
The question being local on $Y$, we can restrict ourselves to the case where $j$ is a closed immersion, $Y$ being defined by a quasi-coherent ideal \emph{of finite type} $\mathfrak{I}$ of $\sh{O}_X$.
The equivalence of (b) and (c), for a given $n$, is immediate;
also, since $\sh{I}^n/\sh{I}^{n+1}$ is an $\sh{O}_X$-module of finite type, there is an open neighborhood $U$ of $y$ in $X$ such that $\sh{I}^n|U = \sh{I}^{n+1}|U$ \sref[0]{0.5.2.2}, so we also have $\sh{I}^n|U = \sh{I}^m|U$ for $m \geq n$ proving the last assertions.
To prove that (a) implies (b), we can restrict ourselves to the cases where the underlying space of $Y$ is equal to the underlying space of $X$ and where $\sh{I}$ is generated by a finite number of sections over $X$:
since $\sh{I}$ is contained in the nilradical $\sh{N}$ of $\sh{O}_X$ \sref[I]{I.5.1.2}, it is now nilpotent which proves b).
Finally, to prove that (b) implies (a), we can restrict ourselves to the case where $\sh{I}^n = \sh{I}^m$; 
therefore, for every $y \in Y$, since $\sh{I}_y \subset \mathfrak{m}_y$, maximal ideal of $\sh{O}_{X,y}$, we must have $\sh{I}^n_y = 0$ because of Nakayama's lemma, since $\sh{I}_y$ is an ideal of finite type.
The set of $x \in X$ such that $\sh{I}^n_x = 0$ is an open $U$ of $X$ contained in $Y$ \sref[0]{0.5.2.2};
since on the other hand $\sh{I}_x \neq 0$ for $x \notin Y$, we must have $U = Y$.
\end{proof}

\begin{corollary}[16.1.10]
\label{IV.16.1.10}
For a restriction of the immersion $j$ to an open neighborhood of $y$ in $Y$ to be an open immersion (in other words, for $j$ to be a \emph{local isomorphism} on the point $y$), it is necessary and sufficient that $(\shGr_1(j))_y = (\sh{N}_{Y/X})_y = 0$.
\end{corollary}

\begin{proof}
The condition is clearly necessary, and the previous reasoning applied to $n=1$ proves that it is sufficient.
\end{proof}

\begin{remark}[16.1.11]
\label{IV.16.1.11}
\begin{enumerate}
  \item[(i)] Under the conditions of the definition \sref{IV.16.1.1}, the projective limit of the projective system $(\sh{O}_{Y^{(n)}}, \phi_{nm})$ of sheaves of rings over $Y$ is called the \emph{normal invariant of infinite order} of $f$, and sometimes denoted by $\sh{O}_{Y^{(\infty)}}$.
  When $X$ is a locally noetherian prescheme, $j:Y \to X$ a closed immersion, $Y$ then is a closed subprescheme of $X$ defined by a coherent ideal $\sh{I}$ and $\sh{O}_{Y^{(\infty)}}$ is exactly the \emph{formal completion} of $\sh{O}_X$ along $Y$ \sref[I]{I.10.8.4}, and $Y^{(\infty)} = (Y, \sh{O}_{Y^{(\infty)}})$ is the formal prescheme that is the \emph{completion} of $X$ along $Y$ \sref[I]{I.10.8.5}.
  In all cases, we could say that $Y^{(\infty)}$ is the \emph{formal neighborhood} of $Y$ in $X$ (via the morphism $f$).
  In the particular case we have just considered, it is the formal prescheme that is the inductive limit of the infinitesimal neighborhoods of order $n$.
  \item[(ii)] Note that for a morphism of preschemes $f=(\psi, \theta): Y \to X$, it can happen that the homomorphism $\theta^\#:\psi^*(\sh{O}_X) \to \sh{O}_Y$ is surjective without $f$ being a local 
\oldpage[IV-4]{9}
  immersion and without $f$ being injective.
  We have an example by taking $Y$ to be a sum of preschemes $Y_\lambda$ all isomorphic to $\Spec(\sh{O}_x)$, where $x \in X$, ad taking $f$ to be the morphism equal to the canonical morphism in each of the $Y_\lambda$.
\end{enumerate}
\end{remark}

\subsection{Functorial properties of the normal invariants of an immersion}
\label{IV.16.2}

\begin{env}[16.2.1]
\label{IV.16.2.1}
Let $f = (\psi, \theta): Y \to X$ and $f' = (\psi', \theta'): Y' \to X'$ by two morphisms of ringed spaces such that $\theta^\#$ and $\theta'^\#$ are surjective;
consider a commutative diagram of morphisms of ringed spaces
\[
  \label{IV.16.2.1.1}
  \xymatrix{
    Y \ar[r]^f & X \\
    Y'\ar[r]_{f'} \ar[u]^u & X'\ar[u]_v  
  }
  \tag{16.2.1.1}
\]

Let $u = (\rho, \lambda), v = (\sigma, \mu)$. 
We have $\rho^*(\psi^*(\sh{O}_X)) = \psi'^*(\sigma^*(\sh{O}_X))$ and as a result a commutative diagram of homomorphisms of sheaves of rings over $Y'$
\[
  \xymatrix{
    \rho^*(\psi^*(\sh{O}_X)) = \psi'^*(\sigma^*(\sh{O}_X)) \ar[r]^-{\psi'^*(\mu^\#)}\ar[d]_{\rho^*(\theta^\#)} & \psi'^*(\sh{O}_{X'}) \ar[d]^{\theta'^\#} \\
    \rho^*(\sh{O}_Y)\ar[r]_-{\lambda^\#}  & \sh{O}_{Y'}  
  }
\]
from which we conclude, if $\sh{I}$ and $\sh{I'}$ are the kernels of $\theta^\#$ and $\theta'^\#$, that we have $\psi'^*(\mu^\#)(\rho^*(\sh{I})) \subset \sh{I'}$, having in mind the exactness of the functor $\rho^*$.
We deduce that, for every integer $n$, $\psi'^*(\mu^\#)(\rho^*(\sh{I}^n)) \subset \sh{I'}^n$, which shows that $\psi'^*(\mu^\#)$ defines, passing to the quotients, a homomorphism of sheaves of rings
\[
  \label{IV.16.2.1.2}
  \nu_n: \rho^*(\psi^*(\sh{O}_X)/\sh{I}^{n+1}) \to \psi'^*(\sh{O}_{X'})/\sh{I'}^{n+1}
  \tag{16.2.1.2}
\]
and therefore a morphism of ringed spaces $w_n = (\rho, \nu_n): Y'^{(n)} \to Y^{(n)}$ (which, for $n = 0$, is none other than $u$).
It results immediately from this definition that the diagrams
\[
  \xymatrix@R=1pc{
    Y^{(n)} \ar[r]^-{h_{mn}} & Y^{(m)} \ar[r]^-{h_m} & X \\
    & & & (n \leq m) \\
    Y'^{(n)} \ar[r]_-{h'_{mn}} \ar[uu]^-{w_n} & Y'^{(m)} \ar[r]_-{h'_m} \ar[uu]^-{w_m} & X' \ar[uu]_-v \\
  }
\]
(where the horizontal arrows are the canonical morphisms \sref{IV.16.1.2}) are commutative.

By passage to the quotients via the morphisms \sref{IV.16.2.1.2}, and taking into
\oldpage[IV-4]{10}
account the exactness of the functor $\rho^*$, we obtain a di-homomorphism of graded algebras (relative to the morphism $\lambda^\#: \rho^*(\sh{O}_Y) \to \sh{O}_{Y'}$)
\[
  \label{IV.16.2.1.3}
  \gr(u): \rho^*(\shGr_\bullet(f)) \to \shGr_\bullet(f')
  \tag{16.2.1.3}
\]
(or, if you like, a $\rho$-morphism \sref[0]{0.3.5.1} $\shGr_\bullet(f) \to \shGr_\bullet(f')$), and in particular a di-homomorphism of conormal sheafs
\[
  \gr_1(u): \rho^*(\shGr_1(f)) \to \shGr_1(f').
\]

It is also immediate that these homomorpisms give rise to a commutative diagram
\[
  \label{IV.16.2.1.4}
  \xymatrix{
    \rho^*(\bb{S}_{\sh{O}_Y}^\bullet(\shGr_1(f)) ) \ar[r] \ar[d]_-{\bb{S}(\gr_1(u))} & \rho^*(\shGr_\bullet(f)) \ar[d]^-{\gr(u)}\\
    \bb{S}_{\sh{O}_Y}^\bullet(\shGr_1(f')) \ar[r] & \shGr_\bullet(f')
  }
  \tag{16.2.1.4}
\]
where the horizontal arrow are the canonical morphisms \sref{IV.16.1.2.2}.

Finally, if we have a commutative diagram of morphisms of ringed spaces
\[
  \xymatrix{
    Y \ar[r]^{f} & X \\
    Y' \ar[r]_{f'} \ar[u]^u & X' \ar[u]_v\\
    Y'' \ar[r]_{f''} \ar[u]^{u'} & X'' \ar[u]_{v'}\\
  }
\]
where $f'' = (\psi'', \theta'')$ is such that $\theta''^\#$ is surjective, and if $w_n'$ and $w_n''$ are defined from $u'$, $v'$ for one and $u'' = u \circ u'$, $v'' = v \circ v'$ for the other, we have $w_n'' = w_n \circ w_n'$, which follows immediately from the definitions and from \sref[0]{0.3.5.5};
we have also $\gr(u'') = \gr(u') \circ \rho'^*(\gr(u))$ if $u' = (\rho', \lambda')$.
Therefore we can say that $Y^{(n)}$ and $\shGr_\bullet(f)$ \emph{depend functorially} on $f$. 
\end{env}

\begin{proposition}[16.2.2]
\label{IV.16.2.2}
With the notation and hypotheses of \sref{IV.16.2.1}, suppose also that $f$, $f'$, $u$ and $v$ are morphisms of preschemes. We have:
\begin{enumerate}
  \item[(i)] The morphisms $w_n:Y'^{(n)} \to Y^{(n)}$ are morphisms of preschemes.
  \item[(ii)] If $Y' = Y \times_X X'$, $u$ and $f'$ the canonical projections, and if $f$ is an immersion or if $v$ is flat, we have $Y'^{(n)} = Y^{(n)} \times_X X'$.
  \item[(iii)] If $Y' = Y \times_X X'$ and if $v$ is flat (resp. if $f$ is an immersion), the homomorphism 
  \[
    \Gr(u) = \gr(u)\otimes I : \shGr_\bullet(f)\otimes_{\sh{O}_Y}\sh{O}_{Y'} \to \shGr_\bullet(f')
  \]
  is bijective (resp. surjective).
\end{enumerate}
\end{proposition}

\begin{proof}
\begin{enumerate}
  \item[(i)] The hypothesis grant us immediately that, for every $y' \in Y'$, $\rho_{y'}^*(\theta_{\psi'(y')}^\#)$ is a \emph{local} homomorphism \sref[I]{I.1.6.2}, so $w_n$ is a morphism of preschemes \sref[I]{I.2.2.1}.
  \oldpage[IV-4]{11}
  \item[(ii) and (iii)] If $f$ is an immersion, we can restrict ourselves to the case where $f$ is a closed immersion, $Y$ being defined by a quasi-coherent ideal $\sh{I}$ of $\sh{O}_X$ and $Y^{(n)}$ by the ideal $\sh{I}^{n+1}$;
  the assertions results from \sref[I]{I.4.4.5}.

  Second, suppose that $v$ is flat;
  we can restrict ourselves to the case where $X = \Spec(A)$, $Y = \Spec(B)$, $X' = \Spec(A')$ are affines, $A'$ being a flat $A$-module;
  so $Y' = \Spec(B')$ where $B' = B \otimes_A A'$;
  in addition, if $\mathfrak{I}$ is the kernel of the homomorphism $A \to B$, the kernel $\mathfrak{I'}$ of $A' \to B'$ is identified with $\mathfrak{I}\otimes_A A'$ by flatness, and $\sh{I}'^n/\sh{I'}^{n+1}$ is equal to
  \begin{align*}
    \psi'^*(\sigma^*((\mathfrak{I}^n/\mathfrak{I}^{n+1})^\sim) \otimes_{\sigma^*(\sh{O}_X)} \sh{O}_{X'}) =& \\
    \psi'^*(\sigma^*((\mathfrak{I}^n/\mathfrak{I}^{n+1} ))^\sim) \otimes_{\psi'^*(\sigma^*(\sh{O}_X))} &\psi'^*(\sh{O}_{X'}) = \rho^*(\sh{I}^n/\sh{I}^{n+1})\otimes_{\rho^*(\psi^*(\sh{O}_X))} \psi'^*(\sh{O}_{X'}) 
  \end{align*}
  and in particular for $n = 0$, we have
  \[
    \sh{O}_{Y'} = \rho^*(\sh{O}_Y) \otimes_{\rho^*(\psi^*(\sh{O}_X))} \psi'^*(\sh{O}_{X'})
  \]
  from which we have canonical isomorphism of $\sh{I}'^n/\sh{I'}^{n+1}$ with
  \[
    \rho^*(\sh{I}^n/\sh{I}^{n+1})\otimes_{\rho^*(\sh{O}_Y)} \sh{O}_{Y'} = (\sh{I}^n/\sh{I}^{n+1}) \otimes_{\sh{O}_Y} \sh{O}_{Y'}
  \]
  which proves (iii).
  Let now $C_n = \Gamma(Y, \sh{O}_{Y^{(n)}}), C'_n = \Gamma(Y', \sh{O}_{Y'^{(n)}})$.
  As $Y^{(n)}$ and $Y'^{(n)}$ are affine schemes \sref{IV.16.1.5}, the kernel $\mathfrak{K}_n$ (resp. $\mathfrak{K}'_n$) of the homomorphism $C_n \to C_{n-1}$ (resp. $C'_n \to C'_{n-1}$) is $\Gamma(Y, \sh{I}^n/\sh{I}^{n+1})$ (resp. $\Gamma(Y, \sh{I}'^n/\sh{I'}^{n+1})$);
  therefore we can deduce from the preceding results that $\mathfrak{K}'_n = \mathfrak{K}_n \otimes_A A'$.
  Now, we have a commutative diagram
  \[
    \xymatrix{
      0 \ar[r] & \mathfrak{K}_n \ar[d]^-r\ar[r]  \otimes_A A' & C_n \otimes_A A' \ar[d]^-{s_n}\ar[r]  & C_{n-1} \otimes_A A' \ar[d]^-{s_{n-1}}\ar[r]  & 0 \\
      0 \ar[r] & \mathfrak{K}'_n \ar[r]   & C'_n \ar[r]  & C'_{n-1} \ar[r]  & 0
    }
  \]
  where the vertical arrow of the left is bijective and the two lines are exact ($A'$ being a flat $A$-module).
  We deduce by induction that $s_n$ is bijective for every $n$, because it's true by hypothesis for $n = 0$, and we deduce by application of the five lemma for all $n$.
  That proves the second assertion of (ii).
\end{enumerate}
\end{proof}

\begin{corollary}[16.2.3]
\label{IV.16.2.3}
Let $g: X \to Y$, $u: Y' \to Y$ be two morphisms of preschemes, $X' = X \times_Y Y'$, $g': X' \to Y'$ and $v: X' \to X$ by the canonical projections. Let $f: Y \to X$ by a $Y$-section of $X$ (and therefore an immersion), $f' = f_{(Y')}: Y' \to X'$ the $Y'$-section of $X'$ deduced from $f$ by the base change $u$.
We have:
\begin{enumerate}
  \item[(i)] The morphism $w_n:{Y'}_{f'}^{(n)} \to Y_f^{(n)}$ corresponding to $f$, $f'$, $u$, $v$ \sref{IV.16.2.1} and the canonical morphism $h'_n: {Y'}_{f'}^{(n)} \to X'$ identifies $ {Y'}_{f'}^{(n)}$ with the product $Y_f^{(n)} \times_X X'$.
  \item[(ii)] If we endow $\sh{O}_{Y_f^{(n)}}$ (resp. $\sh{O}_{{Y'}_{f'}^{(n)}}$) with the structure of $\sh{O}_Y$-algebra defined by $g$ (resp. with the structure of $\sh{O}_{Y'}$-algebra defined by $g'$ ) \sref{IV.16.1.5}[(iii)],
  % The original citation is IV.16.1.6, but he clearly meant 16.1.5 item (iii)
  the homomorphism of $\sh{O}_{Y'}$-algebras
  \[
    \label{IV.16.2.3.1}
      \rho^*(\sh{O}_{Y_f^{(n)}})\otimes_{\sh{O}_Y} \sh{O}_{Y'} \to \sh{O}_{{Y'}_{f'}^{(n)}}
    \tag{16.2.3.1}
  \]
  \oldpage[IV-4]{12}
  deduced from the homomorphism $\nu_n$ \sref{IV.16.2.1.2} is bijective.
  Also, the homomorphism of $\sh{O}_{Y'}$-module
  \[
    \label{IV.16.2.3.2}
    \Gr_1(u): \shGr_1(f)\otimes_{\sh{O}_Y} \sh{O}_{Y'} \to \shGr_1(f')
    \tag{16.2.3.2}
  \]
  is bijective.
 \end{enumerate} 
\end{corollary}

\begin{proof}
\begin{enumerate}
  \item[(i)] Let us first note that $f': Y' \to X'$ and $u: Y' \to Y$  identifies $Y'$ with the product $Y \times_X X'$ (via the structural morphisms $f:Y \to X$ and $v: X' \to X$) \sref{IV.14.5.12.1}.
  The conclusion of (i) now follows from \sref{IV.16.2.2}[(ii)], the morphism $g$ being an immersion.
  \item[(ii)] The commutative diagram
  \[
  \xymatrix{
    Y_f^{(n)} \ar[d]^{h_n}  & {Y'}_{f'}^{(n)} \ar[d]^-{h'_n} \ar[l]^{w_n}\\  
    X         \ar[d]^{g}    & X'              \ar[d]^{g'} \ar[l]^v \\  
    Y                       & Y' \ar[l]^u \\  
  }
  \]
  identifies ${Y'}_{f'}^{(n)}$ with the product $Y_f^{(n)} \times_X X'$, so \sref[I]{I.3.3.9} it identifies (via the morphisms $g'\circ h'_n$ and $w_n$) ${Y'}_{f'}^{(n)}$ to the product $Y_f^{(n)} \times_Y Y'$.
  Since $Y_f^{(n)}$ (resp. ${Y'}_{f'}^{(n)}$) is the affine prescheme over $Y$ (resp. over $Y'$) associated with the $\sh{O}_Y$-algebra $\sh{O}_{Y_f^{(n)}}$ (resp. to the $\sh{O}_{Y'}$-algebra $\sh{O}_{{Y'}_{f'}^{(n)}}$), the fact that the canonical homomorphism \sref{IV.16.2.3.1} is bijective results from \sref[II]{II.1.5.2}.
  Finally, the canonical homomorphism \sref{IV.16.2.3.1} is compatible with the augmentations $\sh{O}_{Y_f^{(n)}} \to \sh{O}_Y$ and $\sh{O}_{{Y'}_{f'}^{(n)}} \to \sh{O}_{Y'}$;
  since $\sh{O}_{Y_f^{(n)}}$ is a direct sum (as an $\sh{O}_Y$-module) of $\sh{O}_Y$ and the augmentation ideal $\sh{I}/\sh{I}^{n+1}$, we can therefore see that the canonical homomorphism \sref{IV.16.2.3.1}, restricted to $\sh{I}/\sh{I}^{n+1} \otimes_{\sh{O}_Y} \sh{O}_{Y'}$, is a bijection of the latter onto $\sh{I}'/\sh{I}'^{n+1}$. For $n=1$ this shows that $\Gr_1(u)$ is bijective.
\end{enumerate}
\end{proof}

We note that, under the hypothesis of \sref{IV.16.2.3}, the homomorphisms $\Gr_n(u)$ are \emph{surjective} in view of the above, but are not bijective in general for $n \geq 2$. However:

\begin{corollary}[16.2.4]
\label{IV.16.2.4}
Under the hypothesis of \sref{IV.16.2.3}, suppose that $u: Y' \to Y$ is a flat morphism (resp. that the $\shGr_n(f)$ are flat $\sh{O}_Y$-modules for $n \leq m$).
Then the homomorphism 
\[
  \Gr_n(u):\shGr_n(f) \otimes_{\sh{O}_Y}\sh{O}_{Y'} \to \shGr_n(f')
\]
is bijective for all $n$ (resp. for $n \leq m $).

\end{corollary}

\begin{proof}
If $u$ is flat, then we deduce by base change that the same is true for $v:X' \to X$, and we already know in this case that $\Gr(u)$ is bijective \sref{IV.16.2.2}[(iii)].
If the $\shGr_n(f)$ are flat for $n\leq m$, if first see by induction on $n$ that the same goes for $\sh{I}/\sh{I}^{n+1}$ for $n\leq m$, because of the exact sequences
  \[
    \xymatrix{
      0 \ar[r] & \sh{I}^n/\sh{I}^{n+1} \ar[r] & \sh{I}/\sh{I}^{n+1} \ar[r] & \sh{I}/\sh{I}^{n} \ar[r] & 0
    }
  \]
  \oldpage[IV-4]{13}
  \sref[0]{0.6.1.2};
  in addition, we have the commutative diagram
  \[
    \xymatrix{
      0 \ar[r] & (\sh{I}^n/\sh{I}^{n+1}) \otimes_{\sh{O}_Y}\sh{O}_{Y'} \ar[d]\ar[r] &( \sh{I}/\sh{I}^{n+1}) \otimes_{\sh{O}_Y}\sh{O}_{Y'} \ar[d]\ar[r] & (\sh{I}/\sh{I}^{n}) \otimes_{\sh{O}_Y}\sh{O}_{Y'} \ar[d]\ar[r] & 0 \\
      0 \ar[r] & \sh{I'}^n/\sh{I'}^{n+1} \ar[r] & \sh{I'}/\sh{I'}^{n+1} \ar[r] & \sh{I'}/\sh{I'}^{n} \ar[r] & 0
    }
  \]
  in which the lines are exact (the first from flatness \sref[0]{0.6.1.2}) and the two last vertical arrows are bijectives in the light of \sref{IV.16.2.2}[(ii)];
  from this we conclude.
\end{proof}

\begin{remarks}[16.2.5]
\label{IV.16.2.5}
\begin{enumerate}
  \item[(i)] The reasoning of \sref{IV.16.2.2}[(i)] still applies to \sref{IV.16.2.1.1} when these are morphisms of \emph{locally ringed spaces} \sref[I]{I.1.8.2}.
  \item[(ii)] In \sref{IV.16.2.2}[(ii)], the conclusion is no longer necessairly valid if we only suppose that $v$ and $f$ are morphisms of preschemes ($f$ verifying the condition of \sref{IV.16.1.1}).
  For example (with the notations of the demonstration of \sref{IV.16.2.2}[(ii)]), it can happen that $\mathfrak{I} = 0$ but the kernel $\mathfrak{I}'$ of $A' \to B' = B \otimes_A A' $ is not zero and that $B' \neq 0$, in which case we have $Y^{(n)} = Y$ for all $n$, but ${Y'}^{(n)} \neq Y'$.
  We have an example of this taking $A = \bb{Z}$, $B = \bb{Q}$, $A' = \prod_{h = 1}^\infty (\bb{Z}/m^h\bb{Z})$ where $m>1$.
\end{enumerate}
\end{remarks}

\begin{env}[16.2.6]
\label{IV.16.2.6}
Consider the particular case of \sref{IV.16.2.1.1} where $X' = X$, $v$ being the identity, $X$ a prescheme, $Y$ a subprescheme of $X$, $Y'$ a subprescheme of $Y$, $f$, $u$, $f' = f \circ u$ the canonical injections;
the di-homomorphism \sref{IV.16.2.1.3} gives us, by tensorization $\sh{O}_{Y'}$ over $\rho^*(\sh{O}_Y)$, a homomorphism of graded $\sh{O}_{Y'}$-algebras
\[
  \label{IV.16.2.6.1}
  u^*(\shGr_\bullet(f)) \to \shGr_\bullet(f').
  \tag{16.2.6.1}
\]
On the other hand, we identify $\sh{O}_Y$ to $\psi^*(\sh{O}_X)/\sh{I}_f$ and $\sh{O}_{Y'}$ to $\rho^*(\sh{O}_Y)/\sh{I}_u$;
since $\rho^*$ is an exact functor, we have $\rho^*(\sh{O}_Y) = \rho^*(\psi^*(\sh{O}_X))/\rho^*(\sh{I}_f) = \psi'^*(\sh{O}_X)/\rho^*(\sh{I}_f)$, and since $\sh{O}_{Y'}$ is moreover identified with $\psi'^*{\sh{O}_X}/\sh{I}_{f'}$, we see that $\sh{I}_u = \sh{I}_{f'}/\rho^*(\sh{I}_f)$.
We deduce that for every integer $n$ there is a canonical homomorphism $\sh{I}_{f'}^n/\sh{I}_{f'}^{n+1} \to \sh{I}_{u}^n/\sh{I}_{u}^{n+1}$, from which we have canonical morphism of graded $\sh{O}_{Y'}$-algebras
\[
  \label{IV.16.2.6.2}
  \shGr_\bullet(f') \to \shGr_\bullet(u).
  \tag{16.2.6.2}
\]
\end{env}

\begin{proposition}[16.2.7]
\label{IV.16.2.7}
Let $X$ be a prescheme, $Y$ a subprescheme of $X$, $Y'$ a subprescheme of $Y$, $j:Y' \to Y$ the canonical injection.
We have an exact sequence of conormal sheaves ($\sh{O}_{Y'}$-modules)
\[
  \label{IV.16.2.7.1}
  \xymatrix{
    j^*(\sh{N}_{Y/X}) \ar[r] & \sh{N}_{Y'/X} \ar[r] & \sh{N}_{Y'/Y} \ar[r] & 0
  }
  \tag{16.2.7.1}
\]
where the arrows are the degree $1$ components of the canonical homomorphisms \sref{IV.16.2.6.1} and \sref{IV.16.2.6.2}.
\end{proposition}

\begin{proof}
  The problem being local, we can restrict ourselves to the case where $X = \Spec(A)$, $Y = \Spec(A/\mathfrak{I})$ and $Y' = \Spec(A/\mathfrak{K})$, $\mathfrak{I}$ and $\mathfrak{K}$ being ideals of $A$ such that $\mathfrak{I} \subset \mathfrak{K}$;
  everything reduces to seeing 
  \oldpage[IV-4]{14}
  that the sequence of canonical morphisms $\mathfrak{I}/\mathfrak{K}\mathfrak{I} \to \mathfrak{K}/\mathfrak{K}^2 \to (\mathfrak{K}/\mathfrak{I} )/ (\mathfrak{K}/\mathfrak{I} )^2 \to 0$ is exact, which is immediate given that the image of $\mathfrak{I}/\mathfrak{K}\mathfrak{I} $ in $\mathfrak{K}/\mathfrak{K}^2$ is $(\mathfrak{I} + \mathfrak{K}^2)/\mathfrak{K}^2$ and that $(\mathfrak{K}/\mathfrak{I} )/ (\mathfrak{K}/\mathfrak{I} )^2$ is identified with $\mathfrak{K}/ (\mathfrak{I} + \mathfrak{K}^2)$.
\end{proof}

It is easy to give examples where the sequence \sref{IV.16.2.7.1} extended on the right by $0$ is not exact;
with the previous notation, just take $A = k[T]$, $\mathfrak{I} = AT^2$, $\mathfrak{K} = AT$, because then $(\mathfrak{I} + \mathfrak{K}^2)/\mathfrak{K}^2 = 0$ and $\mathfrak{I}/\mathfrak{K}\mathfrak{I} \neq 0$.
See however \sref{IV.16.9.13} and \sref{IV.19.1.5} for some cases there the extended sequence is indeed exact.

\subsection{Fundamental differential invariants of morphisms of preschemes}
\label{IV.16.3}

\begin{definition}[16.3.1]
\label{IV.16.3.1}
Let $f:X \to S$ be a morphism of preschemes, $\Delta_f: X \to X \times_S X$ the corresponding diagonal morphism, which is an immersion \sref[I]{I.5.3.9}.
We denote by $\sh{P}_f^n$ or $\sh{P}_{X/S}^n$, and call by \emph{sheaf of principal parts of order $n$ of the $S$-prescheme $X$}, the $\sh{O}_X$-augmented sheaf of rings, $n$'th normal invariant of $\Delta_f$ \sref{IV.16.1.2}.
We'll also write $\sh{P}_f^\infty = \sh{P}_{X/S}^\infty = \varprojlim_n \sh{P}_{X/S}^n$, $\shGr_n(\sh{P}_f) = \shGr_n(\sh{P}_{X/S}) = \shGr_n(\Delta_f)$ \sref{IV.16.1.2};
the $\sh{O}_X$-module $\shGr_1(\Delta_f)$, augmentation ideal of $\sh{P}_{X/S}^1$, is denoted by $\Omega_f^1$ or $\Omega_{X/S}^1$, and is called the $\sh{O}_X$-module of $1$-\emph{differentials} of $f$, or of $X$ by $S$, or of the prescheme $X$.
\end{definition}

It results from the definition that $\sh{P}_{X/S}^0$ is canonically identified with $\sh{O}_X$ \sref{IV.16.1.2}.

We have \sref{IV.16.1.2.2} a canonical surjective morphism of graded $\sh{O}_X$-algebras 
\[
  \label{IV.16.3.1.1}
  \bb{S}_{\sh{O}_X}^\bullet(\Omega_{X/S}^1) \to \shGr_\bullet(\sh{P}_{X/S}).
  \tag{16.3.1.1}
\]
And it results from the definition \sref{IV.16.3.1} that for every open set $U$ of $X$ we have $\sh{P}_{f|U}^n = \sh{P}_f^n|U$, $\sh{P}_{f|U}^\infty = \sh{P}_f^\infty|U$, $\shGr_n(\sh{P}_{f|U}) = \shGr_n(\sh{P}_f)|U$, $\Omega_{f|U}^1 = \Omega_f^1|U$ (in other words, the introduced notions are \emph{local} on $X$).

\begin{env}[16.3.2]
\label{IV.16.3.2}
Denote by $p_1$, $p_2$ the two canonical projections of the product $X \times_S X$;
since $\Delta_f$ is an $X$-section of $X \times_S X$ for both $p_1$ and $p_2$, \emph{each} of these morphisms define, for all $n$, a homomorphism of sheaves of rings $\sh{P}_{X/S}^n \to \sh{O}_X$, right inverse of the augmentation $\sh{P}_{X/S}^n \to \sh{O}_X$ \sref{IV.16.1.7};
one can also say that we can define on $\sh{P}_{X/S}^n$ \emph{two} structures of \emph{quasi-coherent augmented $\sh{O}_X$-algebras};
the corresponding structures of $\sh{O}_X$-modules on $\shGr_n(\sh{P}_{X/S}^n)$ are the same. 
We have the same, by passing to the limit, two structures of $\sh{O}_X$-algebras on $\sh{P}_{X/S}^\infty$.	
\end{env}

\begin{env}[16.3.3]
\label{IV.16.3.3}
The morphism $s = (p_2, p_1)_S: X \times_S X \to X \times_S X$ is an \emph{involution} of $X \times_S X$, called the \emph{canonical symmetry}, such that
\[
	\label{IV.16.3.3.1}
  p_1 \circ s = p_2, \qquad p_2 \circ s = p_1, \qquad s \circ \Delta_f = \Delta_f.
	\tag{16.3.3.1}
\]

If we put $s = (\rho, \lambda)$, $p_i = (\pi_i, \mu_i)$ ($i = 1,2$), $\Delta_f = (\delta, \nu)$, $\lambda^\#$ is then an isomorphism of $\rho^*(\pi_1^*(\sh{O}_X))$ onto $\pi_2^*(\sh{O}_X)$, and $\delta^*(\lambda^\#)$ fixes $\delta^*(\sh{O}_{X \times_S X})$ and the kernel $\sh{I}$ of the homomorphism $\nu^\#: \delta^*(\sh{O}_{X \times_S X}) \to \sh{O}_X$. 
Therefore:
\end{env}

\begin{proposition}[16.3.4]
\label{IV.16.3.4}
The homomorphism $\sigma = \delta^*(\lambda^\#)$ deduced from $s$ (and also called \emph{canonical symmetry}) is an involution of the projective system $(\sh{P}_{X/S}^n)$ of $\sh{O}_X$-augmented 
\oldpage[IV-4]{15}
sheaves of rings, and as a consequence also of the projective limit $\sh{P}_{X/S}^\infty$.
This automorphism permutes the structure of $\sh{O}_X$-algebra on $\sh{P}_{X/S}^n$ and $\sh{P}_{X/S}^\infty$.
\end{proposition}

\begin{env}[16.3.5]
\label{IV.16.3.5}
In what follows, the two structure of $\sh{O}_X$-algebras defined on $\sh{P}_{X/S}^n$ and on $\sh{P}_{X/S}^\infty$ will play very different roles:
\emph{we will now agree, unless stressed otherwise, that when $\sh{P}_{X/S}^n$ and $\sh{P}_{X/S}^\infty$ are considered as an $\sh{O}_X$-algebra, it is the algebra structure induced by $p_1$}.
\end{env}

For every open set $U$ of $X$ and every section $t \in \Gamma(U, \sh{O}_X)$, we will simply denote $t.1$ or even $t$ the image of $t$ under the structural morphism $\Gamma(U, \sh{O}_X) \to \Gamma(U, \sh{P}_{X/S}^n)$ (resp. $\Gamma(U, \sh{O}_X) \to \Gamma(U, \sh{P}_{X/S}^n)$) (that is to say, the homomorphism corresponding to $p_1$).

\begin{definition}[16.3.6]
\label{IV.16.3.6}
We denote by $d_f^n$, or $d_{X/S}^n$ (resp. $d_f^\infty$, or $d_{X/S}^\infty$), or simply $d^n$ (resp. $d^\infty$), the homomorphism of sheaves of rings $\sh{O}_x \to \sh{P}_{f}^n = \sh{P}_{X/S}^n$ (resp. $\sh{O}_x \to \sh{P}_{f}^\infty = \sh{P}_{X/S}^\infty$) deduced from $p_2$.
For every open set $U$ of $X$, and every $t \in \Gamma(U, \sh{O}_x)$, $d^nt$ (resp. $d^\infty t$) is called the \emph{principal part of order $n$} (resp. principal part of infinite order) of $t$.
We put $dt = d^1 t - t$, and we say that $dt$ is the differential of $t$ (element of $\Gamma(U, \Omega_{X/S}^1)$, also denoted $d_{X/S}(t)$).
\end{definition}  

It results immediately
\footnote{
[Trans.] This is, locally we have \sref[0]{0.20.1.1}.
}
from the definition that we have
\[
  \label{IV.16.3.6.1}
  d(t_1t_2) = t_1dt_2 + t_2dt_1
  \tag{16.3.6.1} 
\]
for every $t_1$, $t_2$ in $\Gamma(U \sh{O}_x)$, that is, $d$ is a \emph{derivation} of the ring $\Gamma(U, \sh{O}_x)$ on the $\Gamma(U, \sh{O}_x)$-module $\Gamma(U, \Omega_{X/S}^1)$.

On all notations introduced in \sref{IV.16.3.1} and \sref{IV.16.3.6}, we'll sometimes replace $S$ by $A$ when $S = \Spec(A)$.

\begin{env}[16.3.7]
\label{IV.16.3.7}
Suppose in particular that $S = \Spec(A)$ and $X = \Spec(B)$ are affine schemes, $B$ being then an $A$-algebra.
Then $\Delta_f$ correspeonds to the canonical surjective homomorphism $\pi: B \otimes_A B \to B$ such that $\pi(b\otimes b') = bb'$, with kernel $\mathfrak{I} = \mathfrak{I}_{B/A}$ \sref[0]{0.20.4.1};
$\sh{P}_{f}^n$ is the structure sheaf of the prescheme $\Spec(P_{B/A}^n)$, where
\[
	P_{B/A}^n = (B \otimes_A B)/\mathfrak{I}^{n+1};
\]
$\shGr_\bullet(\sh{P}_f)$ is the quasi-coherent $\sh{O}_x$-module corresponding to the graded $B$-module
\[
	\gr_\mathfrak{I}^\bullet(B \otimes_A B) = \bigoplus_{n \geq 0} (\mathfrak{I}^n/\mathfrak{I}^{n+1});
\]
in particular $\Omega_f^1 = \Omega_{X/S}^1$ is the quasi-coherent $\sh{O}_x$-module corresponding to the $B$-module of $1$-differentials of $B$ over $A$, $\Omega_{B/A}^1$ \sref[0]{0.20.4.3}.
The projection morphisms $p_1: X \times_S X \to X$, $p_2: X \times_S X \to X$ corresponding to the two homomorphisms of rings $j_1: B \to B \otimes_A B$, $j_2: B \to B \otimes_A B$ such that $j_1(b) = b \otimes 1$, $j_2(b) = 1 \otimes b$, such that (by convention of \sref{IV.16.3.5}), $P_{B/A}^n$ is always considered as a $B$-algebra via the composed homomorphism $B \xrightarrow{j_1} B \otimes_A B \xrightarrow{} P_{B/A}^n$;
the ring homomorphism $B \xrightarrow{j_2} B \otimes_A B \xrightarrow{} P_{B/A}^n$ is denoted by $d_{B/A}^n$ and corresponds to $d_{X/S}^n$ operating on $\Gamma(X, \sh{O}_x)$;
for every $t \in B$, $dt$ is equal to $d_{B/A}t$, defined on \sref[0]{0.20.4.6}.

If $\pi_n: B \otimes_A B \to P_{B/A}^n$ is the canonical homomorphism, so we have, in light of the preceding definitions,
\[
	\label{IV.16.3.7.1}
	\pi_n(b\otimes b') = b.\pi_n(1\otimes b') = b.d_{B/A}^n(b') \quad \text{for } b \in B, b' \in B.   		
	\tag{16.3.7.1}
\]
\end{env}

\oldpage[IV-4]{16}

\begin{proposition}[16.3.8]
\label{IV.16.3.8}
The image of the canonical homomorphism $d_{X/S}^n: \sh{O}_x \to \sh{P}_{X/S}^n$ generates the $\sh{O}_x$-module $\sh{P}_{X/S}^n$.
\end{proposition}

\begin{proof}
We come back immediately to the case where $X = \Spec(B)$ and $S = \Spec(A)$ are affines and the proposition results from \sref{IV.16.3.7.1} since $\pi_n$ is surjective.
We note that in general $d_{X/S}^n$ \emph{is not surjective} (even for $n = 1$). 
\end{proof}

\begin{proposition}[16.3.9]
\label{IV.16.3.8}
Suppose that $f:X \to S$ is a morphism locally of finite type.
Then the $\sh{P}_{f}^n$ and the $\shGr_n(\sh{P}_{f})$ are quasi-coherent $\sh{O}_x$-modules of finite type.
\end{proposition}

\begin{proof}
This results from \sref{IV.16.1.6} and from the fact that $\Delta_f$ is locally of finite presentation \sref[1]{1.4.3.1}.
\end{proof}

\subsection{Functorial properties of differential invariants}
\label{IV.16.4}

\begin{env}[16.4.1]
\label{IV.16.4.1}
Consider a commutative diagram of morphisms of preschemes
\[
  \label{IV.16.4.1.1}
  \xymatrix{
    X \ar[d]_-{f} & X' \ar[l]_-u \ar[d]^-{f'}\\
    S & S' \ar[l]^-w
  }
  \tag{16.4.1.1}
\]
We deduce a commutative diagram
\[
  \xymatrix{
    X \ar[d]_-{\Delta_f} & X' \ar[l]_-u \ar[d]^-{\Delta_{f'}}\\
    X \times_S X & X' \times_{S'} X' \ar[l]^-v
  }
\]
where $v$ is the composite homomorphism \sref[I]{I.5.3.5} and \sref[I]{I.5.3.15}.
\[
  \label{IV.16.4.1.2}
  X' \times_{S'} X' \xrightarrow{(p'_1, p'_2)_S} X' \times_{S} X' \xrightarrow{u \times_S u} X \times_S X
  \tag{16.4.1.2}
\]

So we deduce from $u$ and $v$, as explained in \sref{IV.16.2.1}, homomorphisms of augmented sheaves of rings
\[
  \label{IV.16.4.1.3}
  \nu_n: \rho^*(\sh{P}_{X/S}^n) \to \sh{P}_{X'/S'}^n
  \tag{16.4.1.3}
\]
(where we put $u = (\rho,\lambda)$);
these homomorphisms form a projective system, and give the limit a homomorphism of sheaves of graded rings
\[
  \label{IV.16.4.1.4}
  \nu_\infty: \rho^*(\sh{P}_{X/S}^\infty) \to \sh{P}_{X'/S'}^\infty;
  \tag{16.4.1.4}
\]
on the other hand, by passing to the quotient, the homomorphisms $\nu_n$ give rise to a di-homomorphism of graded algebras (relative to $\lambda^\#$):
\[
  \label{IV.16.4.1.5}
  \gr(u): \rho^*(\shGr_\bullet(\sh{P}_{X/S})) \to \shGr_\bullet(\sh{P}_{X'/S'}).
  \tag{16.4.1.5}
\]
\end{env}

\oldpage[IV-4]{17}

\begin{env}[16.4.2]
\label{IV.16.4.2}
If we have a commutative diagram
\[
  \xymatrix{
    X \ar[d]_-{f} & X' \ar[l]_-u \ar[d]^-{f'} & X'' \ar[l]_-{u'} \ar[d]^-{f''}\\
    S & S' \ar[l]^-w & S'' \ar[l]^-{w'} 
  }
\]
we deduce a commutative diagram
\[
  \xymatrix{
    X \ar[d]_-{\Delta_f} & X' \ar[l]_-u \ar[d]^-{\Delta_{f'}} & X'' \ar[l]_-{u'} \ar[d]^-{\Delta_{f''}}\\
    X \times_S X & X' \times_{S'} X' \ar[l]^-v & X'' \times_{S''} X'' \ar[l]^-{v'}
  }
\]
where $v'$ is defined from $u'$, $w'$, $f'$, $f''$ as $v$ is from $u$, $w$, $f$, $f'$.
We verify immediately that if $u'' = u \circ u'$, $w'' = w \circ w'$, then the composite homomorphism $v \circ v'$ is equal to the homomorphism $v''$ deduced from $u''$, $v''$, $f$, $f''$ as $v$ is from $u$, $w$, $f$, $f'$.
If we put $u' = (\rho', \lambda')$, $u'' = (\rho'', \lambda'')$ it results \sref{IV.16.2.1} that the homomorphism $\nu''_n: \rho''^*(\sh{P}_{X/S}^n) \to \sh{P}_{X''/S''}^n$ is equal to the composite
\[
  \rho'^*(\rho^*(\sh{P}_{X/S}^n)) \xrightarrow{\rho'^*(\nu_n^\#)} \rho'^*(\sh{P}_{X'/S'}^n) \xrightarrow{\nu'_n} \sh{P}_{X''/S''}^n
\]
and we have transitivity properties analogous to the homomorphisms \sref{IV.16.4.1.4} and \sref{IV.16.4.1.5}, which lets us say that the $\sh{P}_{X/S}^n$, $\sh{P}_{X/S}^\infty$ and $\shGr_\bullet(\sh{P}_{X/S})$ \emph{depend functorially on $f$}. 
\end{env}

\begin{env}[16.4.3]
\label{IV.16.4.3}
We verify immediately (for example, by restricting ourselves to the affine case with aid of \sref{IV.16.3.7}) that with the notation of \sref{IV.16.4.1}, the diagram
\[
  \label{IV.16.4.3.1}
  \xymatrix{
    \rho^*(\sh{O}_X) \ar[r]^-{\lambda^\#} \ar[d] & \sh{O}_{X'} \ar[d] \\
    \rho^*(\sh{P}_{X/S}^n) \ar[r]_-{\nu_n} & \sh{P}_{X'/S'}^n  
  }
  \tag{16.4.3.1}
\]
where the vertical arrows are the ones defining the algebra structure chosen in \sref{IV.16.3.5} (that is to say, the ones coming from the first projections) is commutative;
the same goes for the diagram
\[
  \label{IV.16.4.3.2}
  \xymatrix{
    \rho^*(\sh{O}_X) \ar[r]^-{\lambda^\#} \ar[d]_-{\rho^*(d_{X/S}^n)} & \sh{O}_{X'} \ar[d]^-{d_{X'/S'}^n} \\
    \rho^*(\sh{P}_{X/S}^n) \ar[r]_-{\nu_n} & \sh{P}_{X'/S'}^n  
  }
  \tag{16.4.3.2}
\]
\oldpage[IV-4]{18}
the vertical arrows defining here the algebra structure from the second projection;
besides, if $\sigma$ and $\sigma'$ are the canonical symmetries corresponding to $f$ and $f'$ \sref{IV.16.3.4}, we have
\[
  \nu_n \circ \rho^*(\sigma) = \sigma' \circ \nu_n 
\] 
which switches one diagram with the other.
We deduce from \sref{IV.16.4.3.1} a canonical homomorphism of \emph{augmented $\sh{O}_{X'}$-algebras}
\[
  \label{IV.16.4.3.3}
  P^n(u): u^*(\sh{P}_{X/S}^n) = \sh{P}_{X/S}^n \otimes_{\sh{O}_X} \sh{O}_{X'} \to \sh{P}_{X'/S'}^n 
  \tag{16.4.3.3}
\]
and it results from \sref{IV.16.4.3.2} that the diagram 
\[
  \label{IV.16.4.3.4}
  \xymatrix{
    \sh{O}_{X'} \ar[r]^-{\operatorname{id}} \ar[d]_-{u^*(d_{X/S}^n)}  & \sh{O}_{X'} \ar[d]^-{d_{X'/S'}^n} \\
    u^*(\sh{P}_{X/S}^n) \ar[r]_{P^n(u)} & \sh{P}_{X'/S'}^n
  }
  \tag{16.4.3.4}
\]
is commutative.
We deduce a homomorphism of \emph{graded $\sh{O}_{X'}$-algebras}
\[
  \label{IV.16.4.3.5}
  \Gr_\bullet(u):u^*(\shGr_\bullet(\sh{P}_{X/S})) \to \shGr_\bullet(\sh{P}_{X'/S'})
  \tag{16.4.3.5}
\]
and in particular a homomorphism of $\sh{O}_{X'}$-modules 
\[
  \label{IV.16.4.3.6}
  \Gr_1(u):\Omega_{X/S}^1 \otimes_{\sh{O}_X} \sh{O}_{X'} \to \Omega_{X'/S'}^1
  \tag{16.4.3.6}
\]
giving rise to a commutative diagram
\[
  \label{IV.16.4.3.7}
  \xymatrix{
    \sh{O}_{X'} \ar[r]^-{\operatorname{id}} \ar[d]_-{d_{X/S} \otimes 1}  & \sh{O}_{X'} \ar[d]^-{d_{X'/S'}} \\
    \Omega_{X/S}^1 \otimes_{\sh{O}_X} \sh{O}_{X'} \ar[r] & \Omega_{X'/S'}^1
  }
  \tag{16.4.3.7}
\]
\end{env}

\begin{env}[16.4.4]
\label{IV.16.4.4}
When $S = \Spec(A)$, $S' = \Spec(A')$, $X = \Spec(B)$, $X' = \Spec(B')$ are affines, so that we have a commutative diagram of homomorphisms of rings
\[
  \xymatrix{
    B \ar[r] & B' \\
    A \ar[u] \ar[r] & A' \ar[u]
  }
\]
the image of $\mathfrak{I}_{B/A}$ in $B' \otimes_{A'} B'$ is contained in $\mathfrak{I}_{B'/A'}$, and the homomorphism $\nu_n$ corresponds to the homomorphism of rings $P_{B/A}^n \to P_{B'/A'}^n$ deduced from the homomorphism $B \otimes_A B \to B' \otimes_{A'} B'$ by passing to the quotient.
The homomorphism \sref{IV.16.4.3.6} corresponds to the homomorphism defined in \sref[0]{0.20.5.4.1}, and the commutative diagram \sref{IV.16.4.3.7} to the diagram \sref[0]{0.20.5.4.2}.
\end{env}

\oldpage[IV-4]{19}

\begin{proposition}[16.4.5]
\label{IV.16.4.5}
Suppose that $X' = X \times_S S'$, $f'$ and $u$ being then the canonical projections.
Then the canonical homomorphisms $P^n(u)$ \sref{IV.16.4.3.3} and $\Gr_1(u)$ \sref{IV.16.4.3.6} are bijective.
\end{proposition}

\begin{proof}
Indeed we have $X' \times_{S'} X' = (X \times_S X) \times_S S'$, and it suffices to apply \sref{IV.16.2.3}[(ii)] replacing $g$ by the first $p_1:X\times_S X \to X$ and $f$ by the diagonal $\Delta_f$.
\end{proof}

We note that under the hypothesis of \sref{IV.16.4.5} the homomorphism $\Gr_\bullet(u)$ \sref{IV.16.4.3.5} is \emph{surjective}, but not bijective in general. 
However \sref{IV.16.2.4}:

\begin{corollary}[16.4.6]
\label{IV.16.4.6}
Under the hypothesis of \sref{IV.16.4.5}, suppose also that $w: S \to S'$ is flat (resp. that $\shGr_n(\sh{P}_{X/S}^n)$ are flat $\sh{O}_X$-modules for $n \leq m$);
then the homomorphism
\[
  \Gr_n(u):u^*(\shGr(\sh{P}_{X/S}^n)) \to \shGr(\sh{P}_{X'/S'}^n)
\]
is bijective for each $n$ (resp. for $n \leq m$).
\end{corollary}   

\begin{proof}
Indeed, if $w$ is flat then the same goes to $v: X' \times_{S'} X' \to X \times_S X$, from which the conclusion results from \sref{IV.16.2.4}.
\end{proof}

\begin{env}[16.4.7]
\label{IV.16.4.7}
Let $S$ be a prescheme, $\sh{E}$ a quasi-coherent $\sh{O}_S$-Module and put $X = \bb{V}(\sh{E})$ \sref[II]{II.1.7.8}, vector bundle associated to $\sh{E}$, equal to $\Spec(\bb{S}_{\sh{O}_S}(\sh{E}))$.
Let $f:X \to S$ be the structural morphism.
For every open set $U$ of $S$ and every section $t \in \Gamma(U, \sh{E})$, $t$ is identified with a section of $\bb{S}_{\sh{O}_S}(\sh{E})$ over $U$;
let $t'$ be its image in $\Gamma(f^{-1}(U), \sh{O}_X) = \Gamma(U, f_*(\sh{O}_X)) = \Gamma(U, \bb{S}_{\sh{O}_S}(\sh{E}))$, and put
\[
  \label{IV.16.4.7.1}
  \delta(t) = d_{X/S}^n(t') - t' \in \Gamma(f^{-1}(U), \sh{P}_{X/S}^n);
  \tag{16.4.7.1}
\]
it is clear that $\delta$ is a di-homomorphism of modules (corresponding to the homomorphism of rings $\Gamma(U, \sh{O}_S) \to \Gamma(f^{-1}(U), \sh{O}_X)$) of $\Gamma(U, \sh{E})$ into $\Gamma(f^{-1}(U), \sh{P}_{X/S}^n)$, and therefore the image belongs to the augmentation ideal of $\Gamma(f^{-1}(U), \sh{P}_{X/S}^n)$.
We deduce (by making $U$ vary) a canonical homomorphism of $\sh{O}_X$-algebras
\[
  \label{IV.16.4.7.2}
  f^*(\bb{S}_{\sh{O}_S}(\sh{E})) \to \sh{P}_{X/S}^n
  \tag{16.4.7.2}
\]
and in view of the above remark, if $\sh{K}$ is the ideal kernel of augmentation $\bb{S}_{\sh{O}_S}(\sh{E}) \to \sh{O}_S$, the image of $\sh{K}^{n+1}$ by \sref{IV.16.4.7.2} is zero, so that by factoring by $\sh{K}^{n+1}$, we have finally a canonical homomorphism
\[
  \label{IV.16.4.7.3}
  \delta_n: f^*(\bb{S}_{\sh{O}_S}(\sh{E})/\sh{K}^{n+1}) \to \sh{P}_{X/S}^n.
  \tag{16.4.7.3}
\]
\end{env} 

\begin{proposition}[16.4.8]
\label{IV.16.4.8}
Under the conditions of \sref{IV.16.4.7}, the homomorphisms $\delta_n$ are bijective and form a projective system of isomorphisms;
we deduce an isomorphism of graded $\sh{O}_S$-algebras 
\[
  \label{IV.16.4.8.1}
  f^*(\bb{S}_{\sh{O}_S}^\bullet(\sh{E})) \to \shGr_\bullet(\sh{P}_{X/S}).
  \tag{16.4.8.1}
\]
\end{proposition}

\begin{proof}
The fact that homomorphisms \sref{IV.16.4.7.3} form a projective system results immediately from the definition.
To prove they are isomorphisms, it suffices to 
\oldpage[IV-4]{20}
demonstrate that \sref{IV.16.4.8.1} is an isomorphism, since both filtrations involved in \sref{IV.16.4.7.3} are finite (Bourbaki, \emph{Alg. comm.}, chap.~III, \textsection2, no 8, cor. 3 of th. ~1).
For that, consider the split exact sequence of $\sh{O}_S$-modules
\[
  \label{IV.16.4.8.2}
  \xymatrix{
    0 \ar[r] & \sh{E} \ar[r]^-u & \sh{E} \oplus \sh{E} \ar[r]^-v & \sh{E} \ar[r] & 0
  }
  \tag{16.4.8.2}
\] 
where, for every pair of sections $s, t$ of $\sh{E}$ over an open set $U$ of $S$, we take $u(s) = (-s, s)$ and $v(s,t) = s + t$.
We have
\[
  X \times_S X = \Spec(\bb{S}_{\sh{O}_S}(\sh{E}) \otimes_{\sh{O}_S} \bb{S}_{\sh{O}_S}(\sh{E})) = \Spec(\bb{S}_{\sh{O}_S}(\sh{E} \oplus \sh{E}))
\]
(\sref[II]{II.1.4.6} and \sref[II]{II.1.7.11}), and the diagonal morphism $X \to X \times_S X$ corresponds \sref[II]{II.1.2.7} to the homomorphism of $\sh{O}_X$-algebras $\bb{S}(v): \bb{S}_{\sh{O}_S}(\sh{E} \oplus \sh{E}) \to \bb{S}_{\sh{O}_S}(\sh{E})$ \sref[II]{II.1.7.4}, so that if $\sh{I}$ is the kernel of this homomorphism we have
\[
  \sh{P}_{X/S}^n = f^*(\bb{S}_{\sh{O}_S}(\sh{E} \oplus \sh{E})/\sh{I}^{n+1}).
\]
The proposition now will be a consequence of the following lemma:
\end{proof}

\begin{lemma}[16.4.8.3]
\label{IV.16.4.8.3}
Let Y be a ringed space, $0 \xrightarrow{} \sh{F}'  \xrightarrow{u} \sh{F} \xrightarrow{v} \sh{F}'' \xrightarrow{} 0$ an exact sequence of $\sh{O}_Y$ modules such that each point $y \in Y$ has an open neighborhood $V$ such that the sequence $0 \to \sh{F}'|V \to \sh{F}|V \to \sh{F}''|V \to 0$ is split.
Let $\sh{I}$ be the kernel ideal of $\bb{S}(v)$:
  \[
  \bb{S}_{\sh{O}_Y}(\sh{F}) \to \bb{S}_{\sh{O}_Y}(\sh{F}''),
  \]
and let $\gr_{\sh{I}}^\bullet(\bb{S}_{\sh{O}_Y}(\sh{F}))$ be the graded $\sh{O}_Y$-algebra associated to the $\sh{O}_Y$-algebra $\bb{S}_{\sh{O}_Y}(\sh{F})$ endowed with the filtration $\sh{I}$-preadic.
Then the homomorphism of graded $\sh{O}_Y$-algebras
\[
  \label{IV.16.4.8.4}
  \bb{S}_{\sh{O}_Y}^\bullet(\sh{F}') \otimes_{\sh{O}_Y} \bb{S}_{\sh{O}_Y}^\bullet(\sh{F}'') \to \gr_{\sh{I}}^\bullet(\bb{S}_{\sh{O}_Y}(\sh{F}))
  \tag{16.4.8.4}
\]
(where the first member is the graded tensor product of symmetric $\sh{O}_Y$-algebras endowed with the canonical gradation \sref[II]{II.1.7.4} and \sref[II]{II.2.1.2}), resulting from the canonical injection
\[
  \sh{F}' \to \sh{I} = \gr_{\sh{I}}^1(\bb{S}_{\sh{O}_Y}(\sh{F})),
\]
is bijective.
\end{lemma}

\begin{proof}
The injection $\sh{F}' \to \sh{I}$ indeed canonically gives a homomorphism of graded $\sh{O}_Y$-algebras $\bb{S}_{\sh{O}_Y}^\bullet(\sh{F}') \to \gr_\sh{I}^\bullet(\bb{S}_{\sh{O}_Y}^\bullet(\sh{F}))$, and since the second member is by definition a graded $\bb{S}_{\sh{O}_Y}^\bullet(\sh{F}'')$-algebra, we deduce the canonical homomorphism \sref{IV.16.4.8.4} by tensoring the above by $\bb{S}_{\sh{O}_Y}^\bullet(\sh{F}'')$.
To prove the lemma we can, being a local problem, restrict ourselves to the case where $\sh{F} = \sh{F}' \oplus \sh{F}''$, $u$ and $v$ being the canonical homomorphisms.
Then the graded algebra $\bb{S}_{\sh{O}_Y}^\bullet(\sh{F})$ is identified canonically with the graded tensor product $\bb{S}_{\sh{O}_Y}^\bullet(\sh{F}') \otimes_{\sh{O}_Y} \bb{S}_{\sh{O}_Y}^\bullet(\sh{F}'')$ \sref[II]{II.1.7.4}, and it is immediate that $\sh{I}$ is therefore the ideal $\sh{I}' \otimes_{\sh{O}_Y} \bb{S}_{\sh{O}_Y}^\bullet(\sh{F}'')$, where $\sh{I}'$ is the augmentation ideal of $\bb{S}_{\sh{O}_Y}^\bullet(\sh{F}')$, that is to say the (direct) sum of the $\bb{S}_{\sh{O}_Y}^m(\sh{F}')$ for $m \geq 1$.
We conclude that $\sh{I}^n = \sh{I}'^n \otimes_{\sh{O}_Y} \bb{S}_{\sh{O}_Y}^\bullet(\sh{F}'')$, where this time $\sh{I}'^n$ is the direct sum of the $\bb{S}_{\sh{O}_Y}^m(\sh{F}')$ for $m \geq n$;
we have therefore $\sh{I}^n/\sh{I}^{n+1} = \bb{S}_{\sh{O}_Y}^n(\sh{F}) \otimes_{\sh{O}_Y} \bb{S}_{\sh{O}_Y}^\bullet(\sh{F}'')$, which proves that \sref{IV.16.4.8.4} is bijective.
\end{proof}

\oldpage[IV-4]{21}

Having demonstrated the lemma, it remains to see that the homomorphism \sref{IV.16.4.8.1} is the image by $f^*$ of the homomorphism \sref{IV.16.4.8.4} corresponding to the exact sequence \sref{IV.16.4.8.2};
we can easily see that it results from the definition of $u$ \sref{IV.16.4.8.2} and of $\delta$ \sref{IV.16.4.7.1}, given the definition of the structure of $\sh{O}_X$-algebras of $\sh{P}_{X/S}^n$ and of the $d_{X/S}^n$ \sref{IV.16.3.5} and \sref{IV.16.4.3.6}.

In particular:

\begin{corollary}[16.4.9]
\label{IV.16.4.9}
Under the conditions of \sref{IV.16.4.7}, we have a canonical morphism
\[
  \gr_1(\delta): f^*(\sh{E}) \xrightarrow{\sim} \Omega_{X/S}^1.
\]
\end{corollary}

\begin{corollary}[16.4.10]
\label{IV.16.4.10}
If $S = \Spec(A)$, $\sh{E} = \sh{O}_S^m$, so that
\[
  X = \Spec(A[T_1, \dots, T_m]),
\]
$\sh{P}_{X/S}^n$ is canonically identified to the $\sh{O}_X$-algebra corresponding to the quotient $A[T_1, \dots, T_m]$-algebra \\
 $A[T_1, \dots, T_m, U_1, \dots, U_m]/\mathfrak{K}^{n+1}$, where the $U_i$ ($1 \leq i \leq m$) are $m$ new indeterminates and $\mathfrak{K}$ is the ideal generated by $U_1, \dots, U_m$.
\end{corollary}

In particular we recover in particular the structure of $\Omega_{X/S}^1$ in \sref[0]{0.20.5.13}.

Also note that the $d_{X/S}^n$ makes a polynomial $F(T_1, \dots, T_m)$ correspond to the class modulo $\mathfrak{K}^{n+1}$ of $F(T_1 + U_1, \dots, T_m + U_m)$, which results from the definition \sref{IV.16.4.7.1}.

\begin{proposition}[16.4.11]
\label{IV.16.4.11}  
Let $f:X \to S$ be a morphism, $g: S \to X$ a $S$-section of $X$, $S^{(n)}$ the n'th infinitesimal neighborhood of $S$ by the immersion $g$ \sref{IV.16.1.2}.
Then it exists one and only one isomorphism of $\sh{O}_S$-algebras
\[
  \label{IV.16.4.11.1}
  \bar\omega_n: g^*(\sh{P}_{X/S}^n) \to \sh{O}_{S_g^{(n)}}
  \tag{16.4.11.1}
\]  
(via the structure of $\sh{O}_S$-algebra on $\sh{O}_{S^{(n)} }$ defined by $f$ \sref{IV.16.1.7}), making commutative the diagram
\[
  \label{IV.16.4.11.2}
  \xymatrix{
    \sh{O}_S = g^*(\sh{O}_X) \ar[rr]^-{\lambda_n} \ar[dr]_-{g^*(d_{X/S}^n)} && \sh{O}_{S_g^{(n)}} \\
    & g^*(\sh{P}_{X/S}^n) \ar[ur]_-{\bar\omega_n}
  }
  \tag{16.4.11.2}
\]
(where $\lambda_n$ is the structural morphism).
\end{proposition} 

\begin{proof}
In light of \sref[I]{I.5.3.7}, where we replace $X$, $Y$, $S$ by $S$, $X$, $S$ respectively and $f$ by $g$, the diagrams
\[
  \label{IV.16.4.11.3}
  \xymatrix@=3pc{
    S \ar[r]^-g \ar[d]_-g &  X \ar[d]^-{\Delta_f} & S \ar[r]^-g  \ar[d]_-g & X \ar[d]^-{\Delta_f}  \\
    X \ar[r]_{(g\circ f, 1_X)_S} & X\times_S X & X \ar[r]_{(1_X, g\circ f)_S} & X \times_S X
  }
  \tag{16.4.11.3}
\]
\oldpage[IV-4]{22}
identifies $S$ with the product of the $(X \times_S X)$-preschemes $X$ and $X$ by the morphisms $\Delta_f$ and $(g\circ f, 1_X)_S$ (resp. $(1_X, g\circ f)_S$).
On the other hand, the diagrams
\[
  \label{IV.16.4.11.4}
  \xymatrix@=3pc{
    X \ar[r]^{(g\circ f, 1_X)_S} \ar[d]_-f &  X \ar[d]^-{p_1} &  X \ar[r]^{(1_X, g\circ f)_S} \ar[d]_-f &  X \ar[d]^-{p_2} \\
    S \ar[r]_{g} & X & S \ar[r]_{g} & X
  }
  \tag{16.4.11.4}
\]
identify $X$ to the product of $X$-preschemes $S$ and $X \times_S X$ via the morphisms $g$ and $p_1$ (resp. $p_2$) (particular case of the associativity formula \sref[I]{I.3.3.9.1}).
We can say that $\Delta_f$, considered as an $X$-section of $X \times_S X$ (relative to $p_1$ or $p_2$) plays the role of a \emph{universal section} for the $S$-sections of $X$:
each of these sections $g$ in fact are deduced by \emph{base change} $(g \circ f, I_X)_S: X \to X \times_S X$.
The definition of the homomorphism $\bar\omega_n$ and the fact that it is bijective results from the remarks of \sref{IV.16.2.3}[(ii)] applied to the first diagram \sref{IV.16.4.11.4}.
The commutativity of the first diagram \sref{IV.16.4.11.4} results also from \sref{IV.16.2.3}[(ii)] this time applied to the second diagram \sref{IV.16.11.4}.
To explicit $\bar\omega_n$, we can restrict ourselves to the case where $g$ is a closed immersion:
Indeed, for every $s\in S$, there is an open neighborhood $W$ of $s$ in $S$ such that $g(W)$ is closed in an open set $U$ of $X$, and it is clear that $g|W$ is a $W$-section of the morphism $U \cap f^{-1}(W)$.
We can then suppose that $S$ is a closed subprescheme of $X$ defined by a quasi-coherent ideal $\sh{K}$.
Then the preceding definitions show that if $W$ is an open of $S$, $t$ is a section of $\sh{O}_X$ over $f^{-1}(W)$, $\bar\omega_n(d^nt|W)$ is equal to the canonical image of $t$ in $\Gamma(W, (\sh{O}_X/\sh{K}^{n+1})|W)$. 
The uniqueness of $\bar\omega_n$ then follows since the image of $\sh{O}_X$ by $d_{X/S}^n$ generates the $\sh{O}_X$-module $\sh{P}_{X/S}^n$ \sref{IV.16.3.8}.
\end{proof}

\begin{corollary}[16.4.12]
\label{IV.16.4.12}
Let $k$ be a field, $X$ a $k$-prescheme, $x$ a point of $X$ rational over $k$.
Then $(\sh{P}_{X/S}^n)_x \otimes_{\sh{O}_x} \kres(x)$ is canonically isomorphic (as an augmented $\kres(x)$-algebra) to $\sh{O}_x/\mathfrak{m}_x^{n+1}$.
\end{corollary}

\begin{proof}
It suffices to consider the unique $k$-section $g$ of $X$ such that $g(\Spec(k)) = {x}$.
\end{proof}

\begin{corollary}[16.4.13]
\label{IV.16.4.13}
Let $f: X \to S$ be a morphism, $s$ a point of $S$, $X_s = X \times_S \Spec(\kres(s))$ the fiber of $f$ in $s$.
If $x \in X_s$ is rational over $\kres(s)$, $(\sh{P}_{X/S}^n)_x \otimes_{\sh{O}_s} \kres(s)$ is canonically isomorphic to $\sh{O}_{X_{s}, x}/{\mathfrak{m}'}_x^{n+1}$, is the maximal ideal of $\sh{O}_{X_{s}, x}$;
more precisely, this isomorphism makes $(d^nt)_x \otimes 1$ correspond (where $t$ is a section of $\sh{O}_X$ over an open neighborhood of $x$ in $X$) to the class of $t_x \otimes 1$ modulo ${\mathfrak{m}'}_x^{n+1}$.
\end{corollary}

\begin{proof}
This results from \sref{IV.16.4.5} and \sref{IV.16.4.12}.
\end{proof}

The preceding properties justify the nomenclature ``sheaf of principal parts of order $n$''.

\begin{proposition}[16.4.14]
\label{IV.16.4.14}
Let $\rho:A \to B$ be a morphism of rings, $S$ a multiplicative subset of $B$.
Then the canonical homomorphisms 
\[
	\label{IV.16.4.14.1}
	S^{-1}P_{B/A}^n \to P_{S^{-1}B/A}^n
	\tag{16.4.14.1}
\] 
\oldpage[IV-4]{23}
deduced from the canonical homomorphisms $P_{B/A}^n \to P_{S^{-1}B/A}^n$ \sref{IV.16.4.4}, form a projective system and are bijective.
\end{proposition}

\begin{proof}
It suffices to remark that $S^{-1}((B \otimes_A B)/\mathfrak{I}^{n+1}) = S^{-1}(B \otimes_A B)/(S^{-1}\mathfrak{I})^{n+1}$ by flatness, and that $S^{-1}(B \otimes_A B) = (S^{-1}B)\otimes_A (S^{-1}B) $ \sref[I]{I.1.3.4}.
\end{proof}

\begin{corollary}[16.4.15]
\label{IV.16.4.15}
The notations being that of \sref{IV.16.4.14}, let $R$ be a multiplicative subset of $A$ such that $\rho(A) \subset S$.
Then we have canonical isomorphisms 
\[
	\label{IV.16.4.15.1}
	S^{-1}P_{B/A}^n \xrightarrow{\sim} P_{S^{-1}B/R^{-1}A}^n
	\tag{16.4.15.1}
\]
forming a projective system.
\end{corollary}

\begin{proof}
It evidently suffices to define canonical isomorphisms 
\[
	\label{IV.16.4.15.2}
	P_{S^{-1}B/A}^n \xrightarrow{\sim} P_{S^{-1}B/R^{-1}A}^n
	\tag{16.4.15.2}
\]
that is to say that we're back to the case there $\rho(R)$ is formed of invertible elements of $B$.
But then the isomorphism \sref{IV.16.4.15.2} is simply deduced from the canonical isomorphism $B \otimes_A B \to B \otimes_{R^{-1}A} B$ by passing to the quotient \sref[I]{I.1.5.3}.
\end{proof}

\begin{corollary}[16.4.16]
\label{IV.16.4.16}
Let $f:X \to S$ be a morphism of preschemes, $x$ a point of $X$, $s = f(x)$.
Then we have canonical isomorphisms
\[
	\label{IV.14.4.16.1}
	(\sh{P}_{X/S}^n)_x \xrightarrow{\sim} P_{\sh{O}_x/\sh{O}_s}^n
	\tag{14.4.16.1}
\]
forming a projective system.
\end{corollary}

We deduce from there isomorphisms from the associated graded rings and in particular a canonical isomorphism
\[
	\label{IV.16.4.16.2}
	(\Omega_{X/S}^1)_x \xrightarrow{\sim} \Omega_{\sh{O}_x/\sh{O}_s}^1.
	\tag{16.4.16.2}
\]

\begin{corollary}[16.4.17]
\label{IV.16.4.17}
Let $k$ be a field, $K$ the field of rational functions $k(T_1, \dots, T_r)$.
Then, for every integer $n$, the homomorphism of $K[U_1, \dots, U_r]$ ($U_i$ indeterminates) into $P_{K/k}^n$ which sends $U_i$ to $d^nT_i - T_i.1$ is surjective and defines an isomorphism of the quotient $K[U_1, \dots, U_n]/\mathfrak{m}^{n+1}$  (where $\mathfrak{m}$ is the ideal generated by the $U_i$) onto $P_{K/k}^n$.
\end{corollary}

\begin{proof}
This results from \sref{IV.16.4.8}, \sref{IV.16.4.10}  and \sref{IV.16.4.14}, where we make $A = k$, $B = k[T_1, \dots, T_r]$ and $S = B \setmin \{ 0 \}$.
\end{proof}

We thus recover the fact that the $dT_i$ form a basis of the $K$-vector space $\Omega_{K/k}^1$ \sref[0]{0.20.5.10}.

\begin{env}[16.4.18]
\label{IV.16.4.18}
Let $f:X \to Y$, $g:Y \to Z$ be two morphism of preschemes, and consider the canonical homomorphism of augmented $\sh{O}_X$-algebras \sref{IV.16.4.3.3}
\[
	\label{IV.16.4.18.1}
	g_{X/Y/Z}:\sh{P}_{X/Z}^n \to \sh{P}_{X/Y}^n
	\tag{16.4.18.1}
\]
\[
	\label{IV.16.4.18.2}
	f_{X/Y/Z}:f^*(\sh{P}_{Y/Z}^n) \to \sh{P}_{X/Z}^n.
	\tag{16.4.18.2}
\]
Then $g_{X/Y/Z}$ is surjective, and its kernel is the ideal generated by the image via $f_{X/Y/Z}$ of the augmentation ideal of $f^*(\sh{P}_{X/Z}^n)$.
\end{env}

\oldpage[IV-4]{24}

\begin{proof}
First note that $g_{X/Y/Z}$ corresponds to the case in \sref{IV.16.4.3.3} where $X' = X$, $S' = Y$ and $S = Z$, $u = 1_X$, and $f_{X/Y/Z}$ to the case where we replace $X', X, S, S'$ by $X, Y, Z, Z$ respectively and $u$ $f$ by $f$,$g$ respectively.

We have a commutative diagram \sref[I]{5.3.5}
\[
  \label{IV.16.4.18.3}
  \xymatrix{
    X \ar[r]^-{\Delta_f} \ar[dr]_-f & X \times_Y X \ar[d]^-p \ar[r]^-j & X \times_Z X \ar[d]^-{f \times_z f} \\
    &   Y \ar[r]_-{\Delta_g}  & Y \times_Z Y
  }
  \tag{16.4.18.3}
\] 
where $j = (1_X, 1_X)_Z$ is an immersion, $j \circ \Delta_f = \Delta_{g \circ f}$ and $p$ is the structural morphism.
Since we can restrict ourselves to the case where $X$, $Y$ and $Z$ are affines, we can suppose that the immersions $\Delta_f$, $\Delta_g$ and $j$ are closed, so that $\sh{O}_X$ and $\sh{O}_{X \times_Y X}$ are identified respectively to $\sh{O}_{X \times_Z X} / \sh{I}$ and $\sh{O}_{X \times_Z X}/\sh{L}$, where $\sh{L} \supset \sh{I}$ are two quasi-coherent ideals corresponding respectively to the immersions $\Delta_{g \circ f} $ and $j$.
The $\sh{O}_X$-algebra $\sh{P}_{X/Z}^n$ is identified with $\sh{O}_{X \times_Z X} / \sh{I}^{n+1}$, and $\sh{P}_{X/Y}^n$ is identified with $\sh{O}_{X \times_Y X}/(\sh{I}/\sh{L})^{n+1}$, which is to say with $\sh{O}_{X \times_Z X}/(\sh{I}^{n+1} + \sh{L})$, and therefore with the quotient of $\sh{P}_{X/Z}^n$ by $(\sh{I}^{n+1} + \sh{L})/\sh{I}^{n+1}$.
But we know (\emph{loc. cit}) that if $p$ and $j$ makes of $X \times_Y X$ the product of the $(Y \times_Z Y)$-preschemes $Y$ and $X \times_Z X$, so if $\sh{O}_Y$ is identified to $\sh{O}_{Y \times_Z Y}/\sh{K}$, where $\sh{K}$ is the ideal corresponding to $\Delta_g$, $\sh{L}$ is equal to $(f\times_Z f)^*(\sh{K}).\sh{O}_{X \times_Z X}$ \sref[I]{I.4.4.5}.
Since $(\sh{I}^{n+1} + \sh{L})/\sh{I}^{n+1}$ is the ideal of $\sh{P}_{X/Z}^n$ generated by the image of $\sh{L}$, we deduce the proposition.
\end{proof}

\begin{corollary}[16.4.19]
\label{IV.16.4.19}
With the notation of \sref{IV.16.4.18}, we have an exact sequence of quasi-coherent $\sh{O}_X$-modules 
\[
  \label{IV.16.4.19.1}
  \xymatrix{
    f^*(\Omega_{Y/Z}^1) \ar[r]^-{f_{X/Y/Z}} & \Omega_{X/Z}^1 \ar[r]^-{g_{X/Y/Z}} & \Omega_{X/Y}^1 \ar[r] & 0.
  }
  \tag{16.4.19.1}
\]
\end{corollary}

When $X$, $Y$, $Z$ are affines, we recover the exact sequence \sref[0]{20.5.7.1}.

\begin{env}[16.4.20]
\label{IV.16.4.20}
Let $f:Y \to Z$ be a morphism, $j: X \to Y$ a closed immersion, $\sh{K}$ the quasi-coherent sheaf of ideals of $\sh{O}_Y$ corresponding to $j$.
It follows that $\sh{P}_{X/Y}^n = \sh{O}_X = \sh{O}_Y/\sh{K}$, the canonical homomorphism $j_{X/Y/Z}: j^*(\sh{P}_{Y/Z}^n) \to \sh{P}_{X/Z}^n$ is surjective, and its kernel is the ideal of $j^*(\sh{P}_{Y/Z}^n)$ generated by $j^*(\sh{O}_Y.d_{Y/Z}^n(\sh{K}))$ (it should be noted that $d_{Y/Z}^n(\sh{K})$ is a subsheaf of Abelian groups of $\sh{P}_{X/Z}^n$, but not an $\sh{O}_Y$-module in general).
\end{env}

\begin{proof}
We know \sref[I]{I.5.3.8} that the diagonal $\Delta_j:X \to X \times_Y X$ is an isomorphism, from which the first assertion follows.
If $\bar\omega_1$ and $\bar\omega_2$ are the two canonical homomorphisms of algebras $\sh{O}_Y \to \sh{P}_{Y/Z}^n$ corresponding respectively to the two canonical projections $p_1$, $p_2$ of $Y \times_Z Y \to Y$, recall that by definition (\sref{IV.16.3.5} and \sref{IV.16.3.6}) $\bar\omega_1$ is the structural homomorphism of the $\sh{O}_Y$-algebra $\sh{P}_{Y/Z}^n$ and $\bar\omega_2 = d_{Y/Z}^n$.
The $\sh{O}_X$-algebra $j^*(\sh{P}_{Y/Z}^n)$ is therefore identified with $\sh{P}_{Y/Z}^n/\bar\omega_1(\sh{K})\sh{P}_{Y/Z}^n$ and its quotient by the ideal generated by $j^*(d_{Y/Z}^n(\sh{K}))$ to $\sh{P}_{Y/Z}^n/(\bar\omega_1(\sh{K}) + \bar\omega_2(\sh{K}))\sh{P}_{Y/Z}^n$.
Now note that we have a commutative diagram
\oldpage[IV-4]{25}
\[
  \xymatrix{
  Y \ar[d]_-{\Delta_f} & X \ar[d]^-{\Delta_{f \circ j}} \ar[l]_-{j}  \\
  Y \times_Z Y & X \times_Z X \ar[l]^-{j \times_Z j} 
  }
\]
identifying $X$ to the product of the $(Y \times_Z Y)$-preschemes $Y$ and $X \times_Z X$ \sref[I]{I.5.3.7}.
Since $j \times_Z j$ is an immersion, we therefore deduce of this remark and from \sref{IV.16.2.2} that if $\Delta_{Y/Z}^n$ and $\Delta_{X/Z}^n$ denote the infinitesimal neighborhoods of order $n$ of $Y$ and $X$ by the canonical immersions $\Delta_f$ and $\Delta_{f\circ j}$ respectively, we have a diagram
\[
  \xymatrix{
  \Delta_{Y/Z}^n \ar[d] & \Delta_{X/Z}^n \ar[d] \ar[l]  \\
  Y \times_Z Y & X \times_Z X \ar[l]^-{j \times_Z j} 
  }
\]
making $\Delta_{X/Z}^n$ the product of the $(Y \times_Z Y)$-preschemes $\Delta_{Y/Z}^n$ and $X \times_Z X$.
We can also say that $\sh{P}_{X/Z}^n$ is identified to the sheaf of rings $\sh{P}_{Y/Z}^n \otimes_{\sh{O}_{Y \times_Z Y}} \sh{O}_{X \times_Z X}$.
But we see immediately that (for example, by restricting ourselves to the affine case) that $\sh{O}_{X \times_Z X} = \sh{O}_{Y \times_Z Y}/(p_1^*(\sh{K}) + p_2^*(\sh{K}))\sh{O}_{Y \times_Z Y}$.
Therefore $\sh{P}_{X/Z}^n$ is identified with the quotient of $\sh{P}_{Y/Z}^n$ by the ideal generated by the image in $\sh{P}_{Y/Z}^n$ of $p_1^*(\sh{K}) + p_2^*(\sh{K})$.
But by definition this ideal is generated by $\bar\omega_1(\sh{K}) + \bar\omega_2(\sh{K})$. 
\end{proof}

\begin{corollary}[16.4.21]
\label{IV.16.4.21}
Let $f:Y \to Z$ be a morphism, $j: X \to Y$ an immersion.
We have an exact sequence of quasi-coherent $\sh{O}_X$-modules
\[
  \label{IV.16.4.21.1}
  \xymatrix{
    \sh{N}_{X/Y} \ar[r] & j^*(\Omega_{Y/Z}^1) \ar[r] & \Omega_{X/Z}^1 \ar[r] & 0.
  }
  \tag{16.4.21.1}
\]
\end{corollary}

When $X, Y, Z$ are affines, we recover the exact sequence \sref[0]{0.20.5.12.1}.

\begin{corollary}[16.4.22]
\label{IV.16.4.22}
If $f: X \to S$ is a morphism locally of finite presentation, $\sh{P}_{X/S}^n$ and $\Omega_{X/S}^1$ are quasi-coherent $\sh{O}_X$-modules of finite presentation.
\end{corollary}

\begin{proof}
We are immediately reduced to the case where $S = \Spec(A)$ is affine, $X = \Spec(B)$, where $B = A[T_1, \dots, T_r]/\mathfrak{K}$, $\mathfrak{K}$ being an ideal of finite type of $C = A[T_1, \dots, T_r]$.
Applying \sref{IV.16.4.20} where $Z = S$, $Y = \Spec(C)$ and $\sh{K} = \widetilde{\mathfrak{K}}$.
Then $j^*(\sh{P}_{Y/Z}^n)$ is a free $\sh{O}_X$-module of finite rank \sref{IV.16.4.10} and the hypothesis on $\mathfrak{K}$ implies that $j^*(\sh{O}_Y.d_{Y/Z}^n(\sh{K}))$ generates a quasi-coherent $\sh{O}_X$-module of finite type;
from this we conclude.
\end{proof}

\begin{proposition}[16.4.23]
\label{IV.16.4.23}
Let $X$, $Y$ be two $S$-preschemes, $Z = X \times_S Y$ their product, $p:X \times_S Y \to X$ and $q:X \times_S Y \to Y$ the canonical projections.
Then the canonical homomorphism
\[
  \label{IV.16.4.23.1}
  p_{Z/X/S} \oplus q_{Z/Y/S}: p^*(\Omega_{X/S}^1) \oplus  q^*(\Omega_{Y/S}^1) \to \Omega_{(X\times_S Y)/S}^1
  \tag{16.4.23.1}
\]
is bijective.
\end{proposition}   

\oldpage[IV-4]{26}

\begin{proof}
The commutative diagram
\[
  \xymatrix{
    Y \ar[d]_-g & X \times_S Y \ar[l]_-q \ar[d]_-h & X \times_S Y \ar[l]_-{\operatorname{id}} \ar[d]^-p\\
    S & S \ar[l]^-{\operatorname{id}} & X \ar[l]^-f
  }
\]
gives us a factorization of the canonical \emph{isomorphism} $P^n(p)$ \sref{IV.16.4.5}
\[
  p^*(\sh{P}_{X/S}^n) \to \sh{P}_{Z/S}^n \to \sh{P}_{Z/Y}^n
\]
and similarly, switching $X$ with $Y$, we have a factorization of the \emph{isomorphism} $P^n(q)$
\[
  q^*(\sh{P}_{Y/S}^n) \to \sh{P}_{Z/S}^n \to \sh{P}_{Z/Y}^n.
\]
This proves that the canonical homomorphism \sref{IV.16.4.18.1}
\[
  p_{Z/X/S}:p^*(\sh{P}_{X/S}^n) \to \sh{P}_{Z/S}^n \quad \text{(resp. $q_{Z/X/S}:q^*(\sh{P}_{Y/S}^n) \to \sh{P}_{Z/S}^n$)}
\] 
is \emph{injective}, and that the kernel of the canonical surjective homomorphism  \sref{IV.16.4.18.2}
\[
  \sh{P}_{Z/S}^n \to \sh{P}_{Z/Y}^n \quad \text{(resp. $\sh{P}_{Z/S}^n \to \sh{P}_{Z/X}^n$)}
\] 
is direct summand of the image $p_{Z/X/S}$ (resp. $q_{Z/Y/S}$).
On the other hand, this kernel is, in light of \sref{IV.16.4.18}, generated by the image by $q_{Z/Y/S}$ (resp. $p_{Z/X/S}$) of the augmentation ideal of $q^*(\sh{P}_{Y/S}^n)$ (resp. $p^*(\sh{P}_{X/S}^n)$).
We conclude the proposition considering $n = 1$.
\end{proof}

One generalizes immediately \sref{IV.16.4.23} to the case of a product of any finite number of preschemes.

\begin{remarks}[16.4.24]
\label{IV.16.4.24}
\begin{enumerate}
  \item[(i)] We will see \sref{IV.17.2.3} that when the morphism $f: X \to Y$ in \sref{IV.16.4.18} is \emph{smooth}, the homomorphism $f_{X/Y/Z}$ in \sref{IV.16.4.19.1} is locally \emph{right invertible} and in particular injective.
  Similarly, when the morphism $f \circ j: X \to Z$ of \sref{IV.16.4.20} is \emph{smooth}, the homomorphism on the right in \sref{IV.16.4.21.1} is locally \emph{right invertible} and \textit{a fortiori} injective \sref{IV.17.2.5}.
  In chapter V, we will also give a variant, in the case of modules over the preschemes, the ``imperfection modules'' studied in \sref[0]{0.20.6}, and the exact sequences where they occur.
  \item[(ii)] Let $X$ be a topological space, $\sh{A}$ a sheaf of  rings over $X$ and $\sh{B}$ a $\sh{A}$-algebra over $X$.
  Then it is clear that 
  \[
    U \rightsquigarrow P_{\Gamma(U, \sh{B})/\Gamma(U, \sh{A})}^n \quad \text{($U$ open in $X$)}
  \]
  is a presheaf of augmented $\Gamma(U \sh{B})$-algebras, and therefore the associated sheaf $\sh{P}_{\sh{B}/\sh{A}}^n$ is an augmented $\sh{B}$-algebra. In the particular case where $X$ is a prescheme, $f = (\psi, \theta): X \to S$ a morphism of preschemes, it results easily from \sref{IV.16.4.16} and from the exactness of the functor $\varinjlim$ that $\sh{P}_{X/S}^n$ is canonically isomorphic to $\sh{P}_{\sh{O}_X/\psi^*(\sh{O}_S)}^n$.
  It follows that the formalism developed in the present paragraph could be considered as a
  \oldpage[IV-4]{27}
  particular case of a differential formalism for ringed spaces endowed with a sheaf of algebras over the structure sheaf.
  However, we did not start with this point of view, which is less intuitive and less convenient for applications.
  It also seems that, for various kinds of ``varieties'', the ``global'' constructions of the $\sh{P}^n$ analogous to those we've used here are also better suited for applications.
\end{enumerate}
\end{remarks}
