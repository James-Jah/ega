\section{Meromorphic functions and pseudo-morphisms}
\label{IV.20}

\setcounter{subsection}{-1}
\subsection{Introduction}
\label{IV.20.0}

Most of the concepts and results of \textsection\textsection20 and 21 directly relate to Chapter~I, and hardly depend on Chapters~I and IV, except for the occasional usage of the notion of depth and of a local regular ring (in \sref{IV.20.6}, \sref{IV.21.11}, \sref{IV.21.13}, and \sref{IV.21.15}), of Zariski's ``Main theorem'' in \sref{IV.20.4} and \sref{IV.21.12}, and the properties of transversely regular immersions in \sref{IV.20.6} and \sref{IV.21.15}.

\oldpage[IV-4]{226}
In \textsection20, we introduce several variants of the concept of a rational map, already studied in \sref[I]{I.7} from a point of view still fairly close to the classical point of view, and for this reason quite ill-suited to the case of not necessarily reduced preschemes.
The notions and results of \textsection20 are used to develop in \textsection21 (n\textsuperscript{os}\sref{IV.21.1} and \sref{IV.21.7}) the general notion of a divisor and its most elementary properties.
This notion is especially convenient when the local rings of the preschemes considered are Noetherian and integrally closed, and especially when they are also \emph{factorial} (\sref{IV.21.6} and \sref{IV.21.7}), because of its identification in the latter case with the notion of a \emph{$1$-codimensional cycle} (a linear combination of irreducible subpreschemes of codimension~$1$).
In \sref{IV.21.9},  we determine the divisors on a Notherian prescheme of dimension~$1$ but not necessarily normal, which is useful for various applications.
The \sref{IV.21.11} and \sref{IV.21.12} give two important theorems, due respectively to Auslander--Buchsbaum and Van der Waerden, and relate the notion of a factorial ring (the n\textsuperscript{os}\sref{IV.21.9}, \sref{IV.21.11}, and \sref{IV.21.12} are independent of each other).
In the n\textsuperscript{os}\sref{IV.21.13} and \sref{IV.21.14}, also independent of the previous three, we study a useful variant of the notion of a local factorial ring, that of a local \emph{parafactorial} ring, which is introduce in particular~\cite{IV-41} in the development of comparison theorem of the Picard group of a projective prescheme $X$ over a field $k$ and a ``hyperplane section''.
We will see in \sref{IV.21.14.1} (Ramanujam--Samuel theorem) that the local parafactorial rings are much more numerous than one would have expected \emph{a priori}.

In \sref{IV.20.5}, \sref{IV.20.6}, and \sref{IV.21.15}, we review the previous notions but from a point of view ``relative'' to a fixed base prescheme.
For the moment these notions are used only relatively rarely; in particular, the concept of a relative divisor is hardly used except when it comes to positive divisors, and in this case it is explained advantageously without the help of the notion of a relative meromorphic functions, using the notion of a transversely regular immersion of codimension~$1$.
It will thereofore be advantageous to omit these sections on a first reading.

\subsection{Meromorphic functions}
\label{IV.20.1}

\begin{env}[20.1.1]
\label{IV.20.1.1}
Let $(X,\sh{O}_X)$ be a ringed space, and let $\sh{S}$ be a subsheaf \emph{of sets} of $\sh{O}_X$.
For every open $U$ of $X$, consider the \emph{ring of fractions} $\Gamma(U,\sh{O}_X)[\Gamma(U,\sh{S})^{-1}]$ (Bourbaki, \emph{Alg. comm.}, chap.~II, \textsection2, n\textsuperscript{o}1).
It is immediate that the map $U\mapsto\Gamma(U,\sh{O}_X)[\Gamma(U,\sh{S})^{-1}]$ is a \emph{presheaf of rings} (\sref[0]{0.1.5.1} and \sref[0]{0.1.5.7}).
We denote by $\sh{O}_X[\sh{S}^{-1}]$ the \emph{sheaf of rings} associated to this presheaf and we say that this is the \emph{sheaf of rings of fractions of $\sh{O}_X$ with denominators in $\sh{S}$};
this is a \emph{flat} $\sh{O}_X$-module.
It is immediate that for every $x\in X$, we have a canonical isomorphism
\[
\label{IV.20.1.1.1}
  (\sh{O}_X[\sh{S}^{-1}])_x\isoto\sh{O}_x[\sh{S}_x^{-1}],
  \tag{20.1.1.1}
\]
since the reasoning of \sref[0]{0.1.4.5} generalizes immediately in the case where we have an inductive system $(A_\alpha,\vphi_{\beta\alpha})$ of rings, and for each index $\alpha$ a subset $S_\alpha$ of $A_\alpha$ such that
\oldpage[IV-4]{227}
$\vphi_{\beta\alpha}(S_\alpha)\subset S_\beta$ for $\alpha\leq\beta$;
we then take for $S$ the inductive limit in $A=\varinjlim A_\alpha$ of the inductive limit of the subsets $(S_\alpha)$.
\end{env}

\begin{env}[20.1.2]
Now let $\sh{F}$ be an $\sh{O}_X$-module.
We then set
\[
\label{IV.20.1.2.1}
  \sh{F}[\sh{S}^{-1}]=\sh{F}\otimes_{\sh{O}_X}\sh{O}_X[\sh{S}^{-1}],
  \tag{20.1.2.1}
\]
and we say that this is the \emph{sheaf of modules of fractions of $\sh{F}$ with denominators in $\sh{S}$};
it is immedyare that it is associated to the presheaf of modules $U\mapsto\Gamma(U,\sh{F})[\Gamma(U,\sh{S})^{-1}]$, and that for every $x\in X$, we have a canonical isomorphism
\[
\label{IV.20.1.2.2}
  (\sh{F}[\sh{S}^{-1}])_x\isoto\sh{F}_x[\sh{S}_x^{-1}].
  \tag{20.1.2.2}
\]
\end{env}

\begin{env}[20.1.3]
\label{IV.20.1.3}
We will focus here on the case where $\sh{S}$ is the subsheaf $\sh{S}(\sh{O}_X)$ of $\sh{O}_X$ such that for every open $U$, $\Gamma(U,\sh{S})$ is the \emph{set of regular elements} of the ring $\Gamma(U,\sh{O}_X)$;
it is immediate that it is a sheaf (and not only a presheaf), the regularity of a section of $\sh{O}_X$ over $U$ being verified ``fibre by fibre'' (i.e. meaning that the germ of the section in $x$ is regular in $\sh{O}_X$ for all $x\in U$), in other words $\sh{S}(\sh{O}_X)_x$ is none other then the set of regular elements of $\sh{O}_{X,x}$.
The corresponding sheaf of rings
\[
  \sh{M}_X=\sh{O}_X[\sh{S}^{-1}]
\]
is called the \emph{shead of germs of meromorphic functions on $X$}, and the sections of $\sh{M}_X$ over $X$ are called the \emph{meromorphic functions on $X$};
they form a ring which we denote by $M(X)$.
For every \emph{$\sh{O}_X$-module $\sh{F}$},
\[
  \sh{F}\otimes_{\sh{O}_X}\sh{M}_X=\sh{F}[\sh{S}^{-1}]
\]
is also denoted $\sh{M}_X(\sh{F})$ and called the \emph{sheaf of germs of meromorphic sections of $\sh{F}$};
its sections over $X$ form an $M(X)$-module denoted by $M(X,\sh{F})$, whose elements are called \emph{meromorphic sections of $\sh{F}$ over $X$}.
These definitions imply that for every open $U$ of $X$, we have a canonical isomorphism $\sh{M}_X(\sh{F})|U\isoto\sh{M}_U(\sh{F}|U)$, in particular $\sh{M}_X|U\isoto\sh{M}_U$.
\end{env}

\begin{env}[20.1.3.1]
\label{IV.20.1.3.1}
If $X$ is a \emph{reduced prescheme}, then we note that if an element $s\in\Gamma(U,\sh{O}_X)$ is such that $s_\xi\neq 0$ for every maximal point $\xi$ of $U$, then $s$ is \emph{regular}.
Indeed, if $st=0$ for a $t\in\Gamma(U,\sh{O}_X)$, then we have $s_\xi t_\xi=0$, so $t_\xi=0$ since $\sh{O}_{X,\xi}$ is a field, and say that $t_\xi=0$ for every maximal point $\xi$ of $X$ means that $t=0$: we are immediately reduced to the case where $U$ is affine, and an element of a reduced ring which belongs to all the minimal prime ideals is zero by definition.
The converse is true if the set of irreducible components of $X$ is \emph{locally finite}.
We immediately reduce to the case where $X=\Spec(A)$ is affine;
if $\mathfrak{p}_i$ ($1\leq i\leq n$) are the minimal prime ideals of $A$ and if $s\in\mathfrak{p}_i$ for an index $i$, then there exists $t\in A$ such that $t\in\mathfrak{p}_j$ for $j\neq i$ and $t\not\in\mathfrak{p}_i$ (Bournaki, \emph{Alg. comm.}, chap.~II, \textsection1, n\textsuperscript{o}1, Prop.~1);
so we have $st\in\mathfrak{p}_i$ for all $i$, and as a result $st=0$ since $A$ is reduced;
$s$ is therefore nonregular.
\end{env}

\begin{env}[20.1.4]
\label{IV.20.1.4}
For every open $U$ of $X$, the homomorphism $t\mapsto t/1$ from $\Gamma(U,\sh{O}_X)$ to $\Gamma(U,\sh{O}_X)[\Gamma(U,\sh{S})^{-1}]$ (which is none other than the \emph{total ring of fractions} of
\oldpage[IV-4]{228}
$\Gamma(U,\sh{O}_X)$) is injective;
these homomorphisms thus define a \emph{canonical injective homomorphism}
\[
\label{IV.20.1.4.1}
  i:\sh{O}_X\to\sh{M}_X
  \tag{20.1.4.1}
\]
which allows us to indentify $\sh{O}_X$ with a subsheaf of $\sh{M}_X$.
Given a meromorphic function $\vphi\in M(X)$, we say that $\vphi$ is \emph{defined} on an open $U$ of $X$ if $\vphi|U$ is a \emph{section of $\sh{O}_U$} over $U$;
the axioms of sheaves show that there is, for a given section $\vphi$, a \emph{largest} open on which $\vphi$ is defined;
we call this the \emph{domain of definition} of $\vphi$ and denote it by $\operatorname{dom}(\vphi)$.
\end{env}

\begin{env}[20.1.5]
\label{IV.20.1.5}
For every $\sh{O}_X$-module $\sh{F}$, we obtain from \sref{IV.20.1.4.1} a di-homomorphism consisting of $i$ and the homomorphism of sheaves of additive groups
\[
\label{IV.20.1.5.1}
  1_\sh{F}\otimes i:\sh{F}\to\sh{M}_X(\sh{F})=\sh{F}\otimes_{\sh{O}_X}\sh{M}_X.
  \tag{20.1.5.1}
\]
We note that the latter is not injective in general;
when it is injective, we say that $\sh{F}$ is \emph{strictly torsion-free}: this means that for every open $U$ of $X$ and every section $s\in\Gamma(U,\sh{O}_X)$ which is a regular element in this ring the homothety $z\mapsto sz$ of $\Gamma(U,\sh{F})$ is injective;
this condition is evidently satisfied if $\sh{F}$ is \emph{locally free}.
\end{env}

\begin{proposition}[20.1.6]
\label{IV.20.1.6}
Let $X$ be a locally Noetherian prescheme, $\sh{F}$ a quasi-coherent $\sh{O}_X$-module.
For $\sh{F}$ to be strictly torsion-free, it is necessary and sufficient that $\operatorname{Ass}(\sh{F})\subset\operatorname{Ass}(\sh{O}_X)$.
\end{proposition}
 
\begin{proof}
We immediately reduce to the case where $X=\Spec(A)$ is affine, $\sh{F}=\widetilde{M}$, and we know that the elements $s$ of $A$ belonging to an ideal of $\operatorname{Ass}(M)$ are exactly those for which the homothety $z\mapsto sz$ is not injective (Bourbaki, \emph{Alg. comm.}, chap.~IV, \textsection1, n\textsuperscript{o}1, cor.~2 of prop.~2).
\end{proof}

\begin{env}[20.1.7]
\label{IV.20.1.7}
If $u$ is a section of $\sh{M}_X(\sh{F})$ over $X$, then we say that $u$ is \emph{defined} at a point $x\in X$ if there exists an open neighborhood $V$ of $x$ in $X$ such that $u|V$ is the image of a section of $\sh{F}$ over $V$ under the di-homomorphism \sref{IV.20.1.5.1}.
We say that $u$ is \emph{defined} on an open $U$ of $X$ if it is defined at every point in $U$;
there is still a larger open in which $u$ is defined, called the \emph{domain of definition} of $u$ and denoted $\operatorname{dom}(u)$.
When $\sh{F}$ is strictly torsion-free, such that $\sh{F}$ is identified by \sref{IV.20.1.5.1} with a subsheaf of $\sh{M}_X(\sh{F})$, then saying $u$ is defined on $U$ means that $u|V$ is a \emph{section of $\sh{F}$ over $U$}.
\end{env}

\begin{env}[20.1.8]
\label{IV.20.1.8}
In accordance with the general notation \sref[0\textsubscript{I}]{0.5.4.7}, we denote by $\sh{M}_X^*$ the sheaf of multiplicative groups such that $\Gamma(U,\sh{M}_X^*)$ is (for every open $U$ of $X$) the group of \emph{invertible elements} of $\Gamma(U,\sh{M}_X)$.
THis sheaf is none other than the sheaf $\sh{S}(\sh{M}_X)$ defined in \sref{IV.20.1.3}: indeed, if $s\in\Gamma(U,\sh{S}(\sh{M}_X))$, then for every $x\in U$, there exists an open neighborhood $V\subset U$ of $x$ such that $s|V$ is a regular element in the \emph{total ring of fractions} of $\Gamma(V,\sh{O}_X)$, and we know that such an element is necessarily invertible in this ring of fractions.
We say that the sections of $\sh{M}_X^*$ over $X$ are the \emph{regular meromorphic functions} (note that we are deviating here from the terminology followed by certain authors, who call ``regular'' meromorphic unctions those which are \emph{sections of $\sh{O}_X$}, identified with a subsheaf of $\sh{M}_X$).

Let $\sh{L}$ be an \emph{invertible $\sh{O}_X$-module} \sref[0\textsubscript{I}]{0.5.4.1};
then it is clear that $\sh{M}_X(\sh{L})=\sh{L}\otimes_{\sh{O}_X}\sh{M}_X$
\oldpage[IV-4]{229}
is an \emph{invertible $\sh{M}_X$-module}.
Let $U$ be an open such that $\sh{L}|U$ is isomorphic to $\sh{O}_U$;
as every automorphism of $\sh{M}_U$ is multiplication by an invertible element of $\Gamma(U,\sh{M}_X)$ \sref[0\textsubscript{I}]{0.5.4.7}, it is equivalent to say that a section $s\in\Gamma(U,\sh{M}_X(\sh{L}))$ has an invertible image in $\Gamma(U,\sh{M}_X)$ under \emph{an} isomorphism or by \emph{any} isomorphism on $\Gamma(U,\sh{M}_X)$;
we say in this case that $s$ is a \emph{regular meromorphic section of $\sh{L}$} over $U$;
a section $s$ of $\sh{L}$ over $X$ is called a \emph{regular meromorphic section of $\sh{L}$} if, for every open $U$ such that $\sh{L}|U$ is isomorphic to $\sh{O}_U$, $s|U$ is a regular meromorphic section of $\sh{L}$ over $U$.
We denote by $(\sh{M}_X(\sh{L}))^*$ the subsheaf of $\sh{M}_X(\sh{L})$ such that for every open $U$, $\Gamma(U,(\sh{M}_X(\sh{L}))^*)$ is the set of regular meromorphic sections of $\sh{L}$ over $U$.
Let $s$ be a meromorphic section of $\sh{L}$ over $X$ (i.e. a section of $\sh{M}_X(\sh{L})$);
it defines a homomorphism $h_s:\sh{M}_X\to\sh{M}_X(\sh{L})$ which sends every section $t$ of $\sh{M}_X$ over an open $U$ to $(s|U)t$.
It follows immediately from the above that for $s$ to be \emph{regular}, it is necessary and sufficient for $h_s$ to be \emph{injective}, and in fact $h_s$ is then a \emph{bijective} homomorphism from $\sh{M}_X$ to $\sh{M}_X(\sh{L})$, and its restriction to $\sh{M}_X^*$ is a bijection to $(\sh{M}_X(\sh{L}))^*$.
We conclude that the homothety $t\mapsto ts$ is an isomorphism from $M(X)$ to $M(X,\sh{L})$.
\end{env}

\begin{env}[20.1.9]
\label{IV.20.1.9}
Let $s$ be a regular meromorphic section of an invertible $\sh{O}_X$-module $\sh{L}$ over $X$;
then for every $\sh{O}_X$-module $\sh{F}$, $s$ similarly defines a homomorphism $h_s\otimes 1_\sh{F}:\sh{M}_X(\sh{F})\to\sh{M}_X(\sh{F}\otimes_{\sh{O}_X}\sh{L})$, which is again \emph{bijective}.
\end{env}

\begin{env}[20.1.10]
\label{IV.20.1.10}
Let $s$ be a meromorphic section of an invertible $\sh{O}_X$-module $\sh{L}$ over $X$;
for $s$ to be regular, it is necessary and sufficient for there to exists a meromorphic section $s'$ of $\sh{L}^{-1}$ over $X$ such that the canonical image of $s\otimes s'$ in $\sh{M}_X$ \sref[0\textsubscript{I}]{0.5.4.3} is the unit section, and this section $s'$ is then unique: indeed, the necessity of the local existence of such a section is evident, and its local uniqueness implies its global (and unique) existence;
moreover, the existence of $s'$ is trivially sufficient for $s$ to be regular.
We will take $s'=s^{-1}$.

Finally, if $\sh{L}'$ is a second invertible $\sh{O}_X$-module, $s$ (resp. $s'$) a regular meromorphic section of $\sh{L}$ (resp. $\sh{L}'$) over $X$, then $s\otimes s'$ is evidently a regular meromorphic section of $\sh{L}\otimes\sh{L}'$ over $X$.
\end{env}

\begin{env}[20.1.11]
\label{IV.20.1.11}
If $f:X'\to X$ is a morphism of ringed spaces, then there is in general no natural map sending a meromorphic function on $X$ to a meromorphic function on $X$'.
For example, if $X$ is the spectrum of a local integral domain $A$, $X'$ its residue field $\kres$, then there is no natural homomorphism from the field of fractions $K$ of $A$ to $\kres$, and we can only send an element of $K$ to an element of $\kres$ if it is already in $A$.

In general, if $f=(\psi,\theta)$, then for every open $U$ of $X$, denote by $\sh{S}_f(U)$ the set of \emph{regular} sections $s\in\Gamma(U,\sh{O}_X)$ such that the image of $s$ under
\[
  \Gamma(\theta^\#):\Gamma(U,\sh{O}_X)\to\Gamma(f^{-1}(U),\sh{O}_{X'})
\]
is a \emph{regular} section.
It is immediate that $U\mapsto\sh{S}_f(U)$ is a \emph{subsheaf} of the sheaf of sets $\sh{S}(\sh{O}_X)$, which we denote by $\sh{S}_f$.
We set $\sh{M}_f=\sh{O}_X[\sh{S}_f^{-1}]$;
this is a subsheaf
\oldpage[IV-4]{230}
of rings of $\sh{M}_X$, and we canonically obtain from $\theta^\#:\psi^*(\sh{O}_X)\to\sh{O}_{X'}$ a homomorphism of sheaves of rings $\theta^{\prime\#}:\psi^*(\sh{M}_f)\to\sh{M}_{X'}$ extending $\theta^\#$ (Bourbaki, \emph{Alg. comm.}, chap.~II, \textsection2, n\textsuperscript{o}1, prop.~2);
hence, recalling that $f^*(\sh{M}_f)=\psi^*(\sh{M}_f)\otimes_{\psi^*(\sh{O}_X)}\sh{O}_{X'}$,
we get a canonical homomorphism of $\sh{O}_{X'}$-algebras
\[
\label{IV.20.1.11.1}
  f^*(\sh{M}_f)\to\sh{M}_{X'}.
  \tag{20.1.11.1}
\]
For every meromorphic function $\vphi$ on $X$ that \emph{is a section of $\sh{M}_f$}, $\Gamma(\theta^{\prime\#})(\vphi)$ is a meromorphic function on $X'$, called the \emph{inverse image of $\vphi$ under $f$}, and denoted by $\vphi\circ f$ is there is no cause for confusion.

Similarly, if $\sh{F}$ is an $\sh{O}_X$-module, then we set $\sh{M}_f(\sh{F})=\sh{F}\otimes_{\sh{O}_X}\sh{M}_f$, and we immediately obtain from $\theta^{\prime\#}$ a canonical homomorphism (which is also written as $u\mapsto u\circ f$)
\[
  \Gamma(X,\sh{M}_f(\sh{F}))\to\Gamma(X',\sh{M}_{X'}(f^*(\sh{F}))).
\]
In addition, if $u\in\Gamma(X,\sh{M}_f(\sh{F}))$ is defined \sref{IV.20.1.7} at a point $x$, then $u$ coincides, on a neighborhood $U$ of $x$, with a section of the form $\sum_i h_i\otimes(t_i/s_i)$, where the $h_i$ belong to $\Gamma(U,\sh{F})$, the $t_i$ to $\Gamma(U,\sh{O}_X)$, and the $s_i$ to $\Gamma(U,\sh{S}_f)$.
As by hypothesis the images of the $s_i$ in $\Gamma(f^{-1}(U),\sh{O}_{X'})$ are regular, we see that $u\circ f$ is defined at every point of $f^{-1}(U)$;
in other words, we have
\[
\label{IV.20.1.11.2}
  f^{-1}(\operatorname{dom}(u))\subset\operatorname{dom}(u\circ f).
  \tag{20.1.11.2}
\]
We will see later \sref{IV.20.6.5}[(i)] examples (with $\sh{F}=\sh{O}_X$) where the two sides of \sref{IV.20.1.11.2} can be different.

Consider in particular the case where $\sh{M}_f=\sh{M}_X$;
then, if $\sh{L}$ is an invertible $\sh{O}_X$-module, the image in $\sh{M}_{X'}(f^*(\sh{L}))$, under $\Gamma(\theta^{\prime\#})$, of a \emph{regular} meromorphic section of $\sh{L}$ over $X$ \sref{IV.20.1.8} is a \emph{regular} meromorphic section of $f^*(\sh{L})$ over $X'$, as it follows immediately from the definition of its sections, and from the fact that a homomorphism of rings sends an invertible element to an invertible element.

Let $f':X''\to X'$ be a second morphism of ringed spaces, and suppose that $\sh{M}_f=\sh{M}_X$ and $\sh{M}_{f'}=\sh{M}_{X'}$;
then, if we set $f''=f\circ f'$, we also have $\sh{M}_{f''}=\sh{M}_X$, and we immediately see that for every meromorphic section $u$ of $\sh{F}$ over $X$, we have $u\circ f''=(u\circ f)\circ f'$.
\end{env}

\begin{proposition}[20.1.12]
\label{IV.20.1.12}
If the morphism $f:X'\to X$ is flat \sref[0\textsubscript{I}]{0.6.7.1}, then we have $\sh{M}_f=\sh{M}_X$, and the homomorphism $\vphi\mapsto\vphi\circ f$ is defined on all of $M(X)$.
In addition, if $f$ is a (flat) morphism of locally ringed spaces, then we have $\operatorname{dom}(\vphi\circ f)=f^{-1}(\operatorname{dom}(\vphi))$;
if in addition $f$ is surjective (thus faithfully flat), then the homomorphism $\vphi\mapsto\vphi\circ f$ is injective.
\end{proposition}

\begin{proof}
The first assertion follows from the fact that, if $B$ is an $A$-algebra which is a flat $A$-module, then every element of $A$ not a divisor of $0$ in $A$ is not a divisor of $0$ in $B$ \sref[0\textsubscript{I}]{0.6.3.4}.
To prove the other assertions, note that, for every $x'\in X'$, if $x=f(x')$, then $\sh{O}_{X',x'}$ is a flat $\sh{O}_{X,x}$-module, and as the homomorpism $\sh{O}_{X,x}\to\sh{O}_{X',x'}$ is \emph{local} by hypothesis, it is injective (\sref[0\textsubscript{I}]{0.6.5.1} and \sref[0\textsubscript{I}]{0.6.6.2});
if we set $A=\sh{O}_{X,x}$, $B=\sh{O}_{X',x'}$, such that $A$ identifies with a subring of $B$, then $(f^*(\sh{M}_X))_{x'}$ is equal to $S^{-1}A\otimes_A B=S^{-1}B$, where $S$ is the set of regular elements of $A$, $(\sh{M}_{X'})_{x'}$ is equal to $T^{-1}B$, where $T$ is the set
\oldpage[IV-4]{231}
of regular elements of $B$, and as we have seen that $S\subset T$, the homomorphism $S^{-1}B\to T^{-1}B$ is injective;
in other words, this proves that the homomorphism \sref{IV.20.1.11.1} $f^*(\sh{M}_X)\to\sh{M}_{X'}$ is \emph{injective} (hence the last assertion of the statement).
The quotient $f^*(\sh{M}_X)/\sh{O}_{X'}$ identifies with a $\sh{O}_{X'}$-submodule of $\sh{M}_{X'}/\sh{O}_{X'}$, and $(f^*(\sh{M}_X)/\sh{O}_{X'})_{x'}$ identifies with $(\sh{M}_X/\sh{O}_X)_x\otimes_{\sh{O}_{X,x}}\sh{O}_{X',x'}$.
Then suppose that $x\not\in\operatorname{dom}(\vphi)$;
the image of $\vphi_x$ in $(\sh{M}_X/\sh{O}_X)_x$ is therefore $\neq 0$;
by faithful flatness, we deduce that that it is the same for the image of $(\vphi\circ f)_{x'}$, so $x'\not\in\operatorname{dom}(\vphi\circ f)$, which finishes the proof.
\end{proof}

\begin{remark}[20.1.13]
\label{IV.20.1.13}
Let $X$ be a \emph{reduced} complex analytic space;
then the notion of a meromorphic function on $X$ defined above coincides with the usual notion.
Consider on the other hand a prescheme $Y$, locally of finite type over the field $\bb{C}$;
we then know that we can associate to $Y$ an analytic space $Y^\mathrm{an}$ having the same underlying topological space, and the canonical morphism $f:Y^\mathrm{an}\to Y$ is \emph{flat}~\cite{IV-37};
by virtue of \sref{IV.20.1.12}, the canonical homomorphism $u\mapsto u\circ f$ from $M(Y)$ to $M(Y^\mathrm{an})$ is therefore always defined and is injective;
but it is not \emph{surjective} in general.
For example, when $Y=\bb{V}_0^r$ (\textbf{Err}\textsubscript{\textbf{III}},~14) is the affine space of dimension~$r$ over $\bb{C}$, $M(Y)$ canonically identifies with the field $R(Y)$ of rational functions on $Y$ \sref{IV.20.2.13}[(i)], while $M(Y^\mathrm{an})$ is the field of usual meromorphic functions on $\bb{C}^r$.
Because of this fact, it is often preferable, in algebraic geometry, to abstain from the terminology introduced in this section, and to use the equivalent terminology of ``pseudo-function'' which will be defined below.
\end{remark}

\subsection{Pseudo-morphisms and pseudo-functions}
\label{IV.20.2}

