\section{Smooth morphisms, unramified morphisms, and \'etale morphisms.}
\label{section:IV.17}

In this paragraph, we revisit the concepts studied in (\textbf{0}\textsubscript{III},~\hyperref[section:0.9]{9}), expressed in the geometric language of schemes from a global point of view, for preschemes locally of finite presentation over a given base.

Most of the results (except~\hyperref[subsection:IV.17.7]{17.7},~\hyperref[subsection:IV.17.8]{17.8},~\hyperref[subsection:IV.17.9]{17.9},~\hyperref[subsection:IV.17.13]{17.13}, and~\hyperref[subsection:IV.17.16]{17.16}) are reduced to various properties already encountered in (\textbf{0}\textsubscript{III},~\hyperref[section:0.9]{9}).

For more specific results on \'etale morphisms, the reader should consult~\textsection\hyperref[section:IV.18]{18}.

\subsection{Formally smooth morphisms, formally unramified morphisms, formally \'etale morphisms.}
\label{subsection:IV.17.1}

\begin{definition}[17.1.1]
\label{IV.17.1.1}
Let $f:X\to Y$ be a morphism of preschemes.
We say that $f$ is \emph{formally smooth} (resp. \emph{formally unramified}, resp. \emph{formally \'etale}) if, for all affine schemes $Y'$, all closed subschemes $Y_0'$ of $Y'$ defined by a nilpotent ideal $\sh{J}$ of $\sh{O}_{Y'}$, and every morphism $Y'\to Y$, the map 
\[
\label{IV.17.1.1.1}
  \Hom_Y(Y',X)\to\Hom_Y(Y_0',X)
  \tag{17.1.1.1}
\]
induced by the canonical map $Y_0'\to Y'$, is \emph{surjective} (resp. \emph{injective}, resp. \emph{bijective}).

One also says that $X$ is \emph{formally smooth} (resp. \emph{formally unramified}, resp. \emph{formally \'etale}) over $Y$.

It is clear that for $f$ to be formally \'etale, it is necessary and sufficient for $f$ to be formally smooth and formally unramified.
\end{definition}

\begin{remark}[17.1.2]
\label{IV.17.1.2}
\medskip\noindent
\begin{enumerate}
  \item[(i)] Suppose that $Y=\Spec(A)$ and $X=\Spec(B)$ are affine, so that $f$ comes from a homomorphism of rings $\varphi:A\to B$. 
    According to \sref[0]{0.19.3.1} and \sref[0]{0.19.10.1}, saying that $f$ is formally smooth (resp. formally unramified, resp. formally \'etale) means that, via $\varphi$, $B$ is a \emph{formally smooth} (resp. \emph{formally unramified}, resp. \emph{formally \'etale}) $A$-algebra, for the \emph{discrete} topologies on $A$ and $B$.
  \item[(ii)] To verify that $f$ is formally smooth (resp. formally unramified, resp. formally \'etale), we can, in Definition~\sref{IV.17.1.1}, restrict to the case where $\sh{J}^2=0$.
  To see this, if $f$ satisfies the corresponding condition of Definition~\sref{IV.17.1.1} in the particular case $\sh{J}^2=0$, and if we have $\sh{J}^n=0$, then we consider the closed subscheme $Y_j'$ of $Y'$ defined by the sheaf of ideals $\sh{J}^{j+1}$ for $0\leq j\leq n-1$, so that $Y_j'$ is a closed subscheme of $Y_{j+1}'$ defined by a square-zero sheaf of ideals;
the hypotheses imply that each of the maps
\[
  \Hom_Y(Y_{j+1}',X)\to\Hom_Y(Y_j',X)\quad(0\leq j\leq n-1) 
\]
\oldpage[IV]{57}
is surjective (resp. injective, resp. bijective);
by composition, we conclude that the same holds for \sref{IV.17.1.1.1}.
\item[(iii)] Note that the properties of the morphism $f$ defined in \sref{IV.17.1.1} are properties of the \emph{representable functor} \sref[\textbf{0}\textsuperscript{III}]{0.8.1.8}
\[
  Y'\mapsto\Hom_Y(Y',X) 
\]
from the category of $Y$-preschemes to the category of sets;
they keep a meaning for \emph{any} contravariant functor with the same domain and codomain, representable or not.
\item[(iv)] Assume that the morphism $f$ is formally unramified (resp. formally \'etale);
consider an \emph{arbitrary} $Y$-prescheme $Z$ and a closed subprescheme $Z_0$ of $Z$ defined by a \emph{locally nilpotent} sheaf of ideals $\sh{J}$ of $\sh{O}_Z$. 
Then the map
\[
\label{IV.17.1.2.1}  
  \Hom_Y(Z,X)\to\Hom_Y(Z_0,X)
  \tag{17.1.2.1}
\]
induced by the canonical injection $Z_0\to Z$, is still injective (resp. bijective).
To see this, let $(U_\alpha)$ be an open affine covering of $Z$ such that the sheaves of ideals $\sh{J}|U_\alpha$ are nilpotent, and for each $\alpha$, let $U_\alpha^0$ be the inverse image of $U_\alpha$ in $Z_0$, which is the closed subprescheme of $U_\alpha$ defined by $\sh{J}|U_\alpha$.
Let $f_0:Z_0\to X$ by a $Y$-morphism;
by hypothesis, for each $\alpha$, there is at most one (resp. one and only one) $Y$-morphism $f_\alpha:U_\alpha\to X$ whose restriction to $Z_0$ coincides with $f_0|U_\alpha$.
We immediately conclude that if $f_\alpha$ and $f_\beta$ are defined, then, for each affine open $V\subset U_\alpha\cap U_\beta$, we have $f_\alpha|V=f_\beta|V$, as the restrictions of these morphisms to the inverse image $V_0$ of $V$ in $Z_0$ coincide.
There is therefore at most one (resp. one and only one) $Y$-morphism $f:Z\to X$ whose restriction to $Z_0$ coincides with $f_0$.
\end{enumerate}
\end{remark}

\begin{proposition}[17.1.3]
\label{IV.17.1.3}
\medskip\noindent
\begin{enumerate}
  \item[{\rm(i)}] A monomorphism of preschemes is formally unramified;
    an open immersion is formally \'etale.
  \item[{\rm(ii)}] The composition of two formally smooth (resp. formally unramified, resp. formally \'etale) morphisms is formally smooth (resp. formally unramified, resp. formally \'etale).
  \item[{\rm(iii)}] If $f:X\to Y$ is a formally smooth (resp. formally unramified, resp. formally \'etale) $S$-morphism, then so is $f_{(S')}:X_{(S')}\to Y_{(S')}$ for any base extension $S'\to S$.
  \item[{\rm(iv)}] If $f:X\to X'$ and $g:Y\to Y'$ are two formally smooth (resp. formally unramified, resp. formally \'etale) $S$-morphisms, then so is $f\times_S g:X\times_S Y\to X'\times_S Y'$.
  \item[{\rm(v)}] Let $f:X\to Y$ and $g:Y\to Z$ be two morphisms;
  if $g\circ f$ is formally unramified, then so is $f$.
  \item[{\rm(vi)}] If $f:X\to Y$ is a formally unramified morphism, then so is $f_\red:X_\red\to Y_\red$.
\end{enumerate}
\end{proposition}

\begin{proof}
According to \sref[I]{I.5.5.12}, it suffices to prove (i), (ii), and (iii). 
The assertions in (i) are both trivial.
To prove (ii), consider two morphisms $f:X\to Y$, $g:Y\to Z$, an affine scheme $Z'$, a closed subscheme $Z_0'$ of $Z$ defined by a nilpotent ideal and a morphism $Z'\to Z$. 
Suppose that $f$ and $g$ formally smooth, and consider a $Z$-morphism
\oldpage[IV]{58}
$u_0:Z_0'\to X$;
the hypothesis  on $g$ implies that there exists a $Z$-morphism $v:Z'\to Y$ such that $f\circ u_0=v\circ j$ (where $j:Z_0'\to Z$ is the canonical injection);
the hypothesis on $f$ then implies that there exists a morphism $u:Z'\to X$ such that $f\circ u=v$ and $u\circ j=u_0$, therefore $(g\circ f)\circ u$ is equal to the given morphism $Z'\to Z$ and $u\circ j=u_0$, which proves that $g\circ f$ is formally smooth;
we argue the same way when we suppose that $f$ and $g$ are formally unramified.

Finally, to prove (iii), let $X'=X_{S'}$, $Y'=Y_{S'}$, $f'=f_{S'}$;
consider an affine scheme $Y''$, a closed subscheme $Y_0''$ defined by a nilpotent sheaf of ideals, and a morphism $g:Y''\to Y'$ making $Y''$ a $Y'$-prescheme;
we then know by \sref[I]{I.3.3.8} that $\Hom_{Y'}(Y'',X')$ is canonically identified with $\Hom_Y(Y'',X)$, and $\Hom_{Y'}(Y_0'',X')$ with $\Hom_Y(Y_0'',X)$, and the conclusion follows immediately from Definition~\sref{IV.17.1.1}.
\end{proof}

We note that a \emph{closed immersion} is not necessarily formally smooth.
\begin{proposition}[17.1.4]
\label{IV.17.1.4}
Let $f:X\to Y$ and $g:Y\to Z$ be two morphisms, and suppose that $g$ is formally unramified.
Then, if $g\circ f$ is formally smooth (resp. formally \'etale), so is $f$.
\end{proposition}

\begin{proof}
Let $Y'$ be an affine scheme, $Y_0'$ a closed subscheme of $Y'$ defined by a nilpotent sheaf of ideals, $h:Y'\to Y$ a morphism, $j:Y_0'\to Y'$ the canonical injection, $u_0:Y_0'\to Y$ a $Y$-morphism, such that $f\circ u_0=h\circ j$. 
Suppose that $g\circ f$ is formally smooth;
then there exists a morphism $u:Y'\to X$ such that $u\circ j=u_0$ and $(g\circ f)\circ u=g\circ h$. 
But these two relations imply that $f\circ u$ and $h$ are $Z$-morphisms from $Y'$ to $Y$ such that $(f\circ u)\circ j=h\circ j$;
by virtue of the hypothesis that $g$ is formally unramified, we get that $f\circ u=h$, in other words that $u$ is a $Y$-morphism;
thus $f$ is formally smooth.
Taking into account \sref{IV.17.1.3}[(v)], this proves the proposition.
\end{proof}

\begin{corollary}[17.1.5]
\label{IV.17.1.5}
Suppose that $g$ is formally \'etale;
then, for $g\circ f$ to be formally smooth (resp. formally unramified, resp. formally \'etale), it is necessary and sufficient that $f$ is.
\end{corollary}

\begin{proof}
This follows from \sref{IV.17.1.4} and \sref{IV.17.1.3}[(ii) and (iv)].
\end{proof}

\begin{proposition}[17.1.6]
\label{IV.17.1.6}
Let $f:X\to Y$ be a morphism of preschemes.
\begin{enumerate}
  \item[{\rm(i)}] Let $(U_\alpha)$ be an open covering of $X$ and, for each $\alpha$, let $i_\alpha:U_\alpha\to X$ be the canonical injection.
    For $f$ to be formally smooth (resp. formally unramified, resp. formally \'etale), it is necessary and sufficient that each $f\circ i_\alpha$ is.
  \item[{\rm(ii)}] Let $(V_\lambda)$ be an open covering of $Y$.
    For $f$ to be formally smooth (resp. formally unramified, resp. formally \'etale), it is necessary and sufficient that each of the restrictions $f^{-1}(V_\lambda)\to V_\lambda$ of $f$ is.
\end{enumerate}
\end{proposition}

\begin{proof}
First note that (ii) is a consequence of (i): if $j_\lambda:V_\lambda\to Y$ and $i_\lambda:f^{-1}(V_\lambda)\to X$ are the canonical injections, then the restriction $f_\lambda:f^{-1}(V_\lambda)\to V_\lambda$ of $f$ is such that $j_\lambda\circ f_\lambda=f\circ i_\lambda$;
if $f$ is formally smooth (resp. formally unramified), then so is $f\circ i_\lambda$ since $i_\lambda$ is formally \'etale \sref{IV.17.1.3};
but since $j_\lambda$ is formally \'etale, this means that $f_\lambda$ is formally smooth (resp. formally unramified), by virtue of \sref{IV.17.1.5}.
Conversely, if all the $f_\lambda$ are formally smooth (resp. formally unramified), the same applies to $j_\lambda\circ f_\lambda$ \sref{IV.17.1.3}, so also to $f$ in virtue of (i).

\oldpage[IV]{59}
If we take into account that the $i_\alpha$ are formally \'etale, everything comes down to proving that if the $f\circ i_\alpha$ are formally smooth (resp. formally unramified), then the same applies to $f$.

Therefore let $Y'$ be an affine scheme, $Y_0'$ a closed subscheme of $Y'$ defined by a nilpotent ideal $\sh{J}$, which we may assume to satisfy $\sh{J}^2=0$ \sref{IV.17.1.2}[(ii)], and finally let $g:Y'\to Y$ be a morphism. 
Suppose we are given a $Y$-morphism $u_0:Y_0'\to X$;
denote by $W_\alpha$ (resp. $W_\alpha^0$) the prescheme induced by $Y'$ (resp. $Y_0'$) on the open subset $u_0^{-1}(U_\alpha)$ (we recall that $Y'$ and $Y_0'$ share the \emph{same underlying topological space}). 
Let us first suppose that the $f\circ i_\alpha$ are \emph{formally unramified}, and show that, if $u'$ and $u''$ are two $Y$-morphisms from $Y'$ to $X$ whose restrictions to $Y_0'$ coincide, then we have $u'=u''$. 
Indeed, taking into account \sref{IV.17.1.2}[(iv)], the hypothesis that the $f\circ i_\alpha$ are formally unramified implies that for all $\alpha$, we have $u'|W_\alpha=u''|W_\alpha$, since the restrictions of both $Y$-morphisms to $W_\alpha^0$ coincide. 
Hence the conclusion follows.

Now suppose that the $f\circ i_\alpha$ are \emph{formally smooth} and prove the existence of a $Y$-morphism $u:Y'\to X$ whose restriction to $Y_0'$ is $u_0$.
Now, since $Y'$ is an \emph{affine scheme}, we can apply \sref{IV.16.5.17}, the hypotheses of which are satisfied, and the conclusion of which precisely proves the existence of $u$.
\end{proof}

We can therefore say that the notions introduced in \sref{IV.17.1.1} are \emph{local} on $X$ and $Y$, which always allows, in virtue of \sref{IV.17.1.2}[(i)], to be reduced to the study of formally smooth (resp. formally unramified, resp. formally \'etale) \emph{algebras}.

\subsection{General properties of differentials}
\label{subsection:IV.17.2}

\begin{proposition}[17.2.1]
\label{IV.17.2.1}
For a morphism $f:X\to Y$ to be formally unramified, it is necessary and sufficient that $\Omega_f^1=0$ (what we still write $\Omega_{X/Y}^1=0$ \sref{IV.16.3.1}).  
\end{proposition}

\begin{proof}
Taking into account \sref{IV.17.1.6}, we reduce to the case where $Y=\Spec(A)$ and $X=\Spec(B)$ are affine, and the conclusion then follows from \sref[0]{0.20.7.4} and the interpretation of $\Omega_{X/Y}^1$ in this case \sref{IV.16.3.7}.
\end{proof}

\begin{corollary}[17.2.2]
\label{IV.17.2.2}
Let $f:X\to Y$ and $g:Y\to Z$ be two morphisms. 
For $f$ being formally unramified, it is necessary and sufficient that the canonical morphism \sref{IV.16.4.19}
\[
  f^*(\Omega^1_{Y/Z})\to \Omega^1_{X/Z}
\] is surjective.
\end{corollary}

\begin{proof}
This is an immediate consequence of \sref{IV.17.2.1} and the exact sequence \sref{IV.16.4.19.1}.
\end{proof}

\begin{proposition}[17.2.3]
\label{IV.17.2.3}
Let $f:X\to Y$ be a formally smooth morphism.
\begin{enumerate}
  \item[{\rm(i)}] The $\sh{O}_X$-module $\Omega_{X/Y}^1$ is locally projective \sref{IV.16.10.1}.
    If $f$ is locally of finite type, then $\Omega_{X/Y}^1$ is locally free and of finite type.
  \item[{\rm(ii)}] For all morphisms $g:Y\to Z$, the sequence \sref{IV.16.4.19} of $\sh{O}_X$-modules
\[
\label{IV.17.2.3.1}
  0\to f^*(\Omega_{Y/Z}^1)\to\Omega_{X/Z}^1\to\Omega_{X/Y}^1\to 0
  \tag{17.2.3.1}
\]
is exact; moreover, for each $x\in X$, there exists an open neighborhood $U$ of $x$ such that the restrictions to $U$ of the homomorphisms in \sref{IV.17.2.3.1} form a \emph{split} exact sequence.
\end{enumerate}
\end{proposition}

\oldpage[IV]{60}
\begin{proof}
\begin{enumerate}
  \item[(i)] We know \sref{IV.16.3.9} that if $f$ is locally of finite type, then $\Omega_f^1$ is an $\sh{O}_X$-module of finite type. 
    To prove that, in all cases, it is locally projective, we can reduce, by virtue of \sref{IV.17.1.6}, to the case where $Y=\Spec(A)$ and $X=\Spec(B)$ are affine, and the result follows from the hypothesis on $f$ and from \sref[0]{0.20.4.9} and \sref[0]{0.19.2.1}.
  \item[(ii)] Again, we can restrict to the case where $X$, $Y$, and $Z$ are affine \sref{IV.17.1.6}, and the conclusion in this case follows from the interpretation of the sheaves of modules in the sequence \sref{IV.17.2.3.1} and from \sref[0]{0.20.5.7}.
\end{enumerate}
\end{proof}

\begin{corollary}[17.2.4]
\label{IV.17.2.4}
If $f:X\to Y$ is formally \'etale, then, for all morphisms $g:Y\to Z$, the canonical homomorphism of $\sh{O}_X$-modules
\[
  f^*(\Omega_{Y/Z}^1)\to\Omega_{X/Z}^1
\]
is bijective.
\end{corollary}

\begin{proof}
This follows from the exactness of the sequence \sref{IV.17.2.3.1} and from the fact that we then have $\Omega_{X/Y}^1=0$ \sref{IV.17.2.1}.
\end{proof}

\begin{proposition}[17.2.5]
\label{IV.17.2.5}
Let $f:X\to Y$ be a morphism, $X'$ a subprescheme of $X$ such that the composite morphism $X'\xrightarrow{j}X\xrightarrow{f}Y$ (where $j$ is the canonical injection) is formally smooth.
Then the sequence of $\sh{O}_X$-modules \sref{IV.16.4.21}
\[
\label{IV.17.2.5.1}
  0\to\sh{N}_{X'/X}\to\Omega_{X/Y}^1\otimes_{\sh{O}_X}\sh{O}_{X'}\to\Omega_{X'/Y}^1\to 0
\tag{17.2.5.1}
\]
is exact; moreover, for each $x\in X$, there exists an open neighborhood $U$ of $x$ such that the restrictions to $U$ of the homomorphisms in \sref{IV.17.2.5.1} form a \emph{split} exact sequence.
\end{proposition}

\begin{proof}
By virtue of \sref{IV.17.1.6}, we reduce to the case where $Y=\Spec(A)$ and $X=\Spec(B)$ are affine, and $X'=\Spec(B/\mathfrak{J})$, where $\mathfrak{J}$ is an ideal of $B$. 
The conormal sheaf $\sh{N}_{X'/X}$ then corresponds to the $B$-module $\mathfrak{J}/\mathfrak{J}^2$ \sref{IV.16.1.3}, and the conclusion follows from \sref[0]{0.20.5.14}.
\end{proof}

\begin{proposition}[17.2.6]
\label{IV.17.2.6}
Let $X$ and $Y$ be two preschemes, $f:X\to Y$ a morphism locally of finite type. 
The following conditions are equivalent:
\begin{enumerate}
  \item[{\rm(a)}] $f$ is a monomorphism.
  \item[{\rm(b)}] $f$ is radicial and formally unramified.
  \item[{\rm(c)}] For each $y\in Y$, the fibre $f^{-1}(y)$ is empty or $\kres(y)$-isomorphic to $\Spec(\kres(y))$ (in other words, it is reduced to a single point $z$ such that $\kres(y)\to\sh{O}_z/\mathfrak{m}_y\sh{O}_z$ is an isomorphism).
\end{enumerate}
\end{proposition}

\begin{proof}
The fact that (a) implies (c) follows from \sref{IV.8.11.5.1}. 
It is clear that (c) implies that $f$ is radicial;
let us prove that it also follows from (c) that $\Omega_{X/Y}^1=0$, which will prove that (c) implies (b) \sref{IV.17.2.1}.
Note that the $\sh{O}_X$-module $\Omega_{X/Y}^1$ is quasi-coherent of finite type \sref{IV.16.3.9}.
It follows from \sref[I]{I.9.1.13.1} that, for $(\Omega_{X/Y}^1)_x=0$, it is necessary and sufficient that if we set $Y_1=\Spec(\kres(y))$, $X_1=f^{-1}(y)=X\times_Y Y_1$, then we have $(\Omega_{X_1/Y_1}^1)_x=0$;
but as the morphism $f_1:X_1\to Y_1$ induced by $f$ is formally unramified by virtue of the hypothesis (c) \sref{IV.17.1.3}, the conclusion follows from \sref{IV.17.2.1}.
Finally, let us prove that (b) implies (a);
for this, consider the diagonal morphism $g=\Delta_f:X\to X\times_Y X$;
since $f$ is radicial, $g$ is surjective \sref{IV.1.8.7.1};
on the other hand, $\Omega_{X/Y}^1$ is by definition the conormal sheaf $\shGr_1(g)$ of the immersion $g$ \sref{IV.16.3.1}, and to say that $f$ is formally unramified therefore means that
\oldpage[IV]{61}
$\shGr_1(g)=0$ \sref{IV.17.2.1}. 
In addition, $g$ is locally of finite presentation \sref{IV.1.4.3.1};
therefore the hypothesis $\shGr_1(g)=0$ implies that $g$ is an open immersion \sref{IV.16.1.10};
being surjective, this immersion is an isomorphism, hence $f$ is a monomorphism \sref[I]{I.5.3.8}.
\end{proof}

\subsection{Smooth morphisms, unramified morphisms, \'etale morphisms}
\label{subsection:IV.17.3}

\begin{definition}[17.3.1]
\label{IV.17.3.1}
We say that a morphism $f:X\to Y$ is \emph{smooth} (resp. \emph{unramified}, or \emph{net}
\footnote{The words ``net'' and ``formally net'' seem more preferable to the terminology used in ``unramified'' (resp. formally unramified'') and will be used almost exclusively in Chapter~V.
In this chapter, we have kept the old terminology so as not to conflict with \hyperref[subsection:IV.19.10]{\textbf{0}, 19.10}.}
resp. \emph{\'etale})
if it is locally of finite presentation and formally smooth (resp. formally unramified, resp. formally \'etale).
\end{definition}

We then also say that $X$ is \emph{smooth} (resp. \emph{unramified}, resp. \emph{\'etale}) \emph{over $Y$}.

We will see later \sref{IV.17.5.2} that this definition of a smooth morphism coincides with the definition already given in \sref{IV.6.8.1};
until then, we will exclusively use definition \sref{IV.17.3.1}.

It is clear that saying that $f$ is \'etale means that it is \emph{both} smooth and unramified.

\begin{remark}[17.3.2]
\label{IV.17.3.2}
\medskip\noindent
\begin{enumerate}
  \item[(i)] Note that definition \sref{IV.17.3.1} can be phrased using only the functor
    \[
      Y'\mapsto\Hom_Y(Y',X)
    \]
    considered in \sref{IV.17.1.2}[(iii)] because to say that $f$ is locally of finite presentation is equivalent to saying that the preceding functor \emph{commutes with projective limits of affine schemes} \sref{IV.8.14.2}.
  \item[(ii)] Let $A$ be a ring and $B$ an $A$-algebra. 
   We say that $B$ is a \emph{smooth} (resp. \emph{unramified}, resp. \emph{\'etale}) $A$-algebra if the corresponding morphism $\Spec(B)\to\Spec(A)$ is smooth (resp. unramified, resp. \'etale).
   It is equivalent to say that $B$ is an $A$-algebra \emph{of finite presentation} \sref{IV.1.4.6} that is furthermore formally smooth (resp. formally unramified, resp. formally \'etale) for the discrete topologies.
  \item[(iii)] It follows from \sref{IV.17.1.6} and the definition of a morphism locally of finite presentation \sref{IV.1.4.2} that the notion of a smooth (resp. unramified, resp. \'etale) morphism is \emph{local on $X$ and on $Y$}.
\end{enumerate}
\end{remark}

\begin{proposition}[17.3.3]
\label{IV.17.3.3}
\medskip\noindent
\begin{enumerate}
  \item[{\rm(i)}] An open immersion is \'etale.
    For an immersion to be unramified, it is necessary and sufficient to it be locally of finite presentation.
  \item[{\rm(ii)}] The composition of two smooth (resp. unramified, resp. \'etale) morphisms is smooth (resp. unramified, resp. \'etale).
  \item[{\rm(iii)}] If $f:X\to Y$ is a smooth (resp. unramified, resp. \'etale) $S$-morphism, then so is $f_{(S')}:X_{(S')}\to Y_{(S')}$ for any base extension $S'\to S$. 
  \item[{\rm(iv)}] If $f:X\to X'$ and $g:Y\to Y'$ are smooth (resp. unramified, resp. \'etale) $S$-morphisms, then so is $f\times_S g:X\times_S Y\to X'\times_S Y'$.
\oldpage[IV]{62}
  \item[{\rm(v)}] Let $f:X\to Y$ and $g:Y\to Z$ be two morphisms;
    if $g$ is locally of finite type and if $g\circ f$ is unramified, then $f$ is unramified.
\end{enumerate}
\end{proposition}

\begin{proof}
This follows from \sref{IV.1.4.3} and \sref{IV.17.1.3}.
\end{proof}

\begin{proposition}[17.3.4]
\label{IV.17.3.4}
Let $f:X\to Y$ and $g:Y\to Z$ be two morphisms, and suppose that $g$ is unramified.
Then, if $g\circ f$ is smooth (resp. unramified, resp. \'etale), so is $f$.
\end{proposition}

\begin{proof}
As $g$ and $g\circ f$ are locally of finite presentation, so is $f$ \sref{IV.1.4.3}[(v)];
the conclusion thus follows from \sref{IV.17.1.4} and \sref{IV.17.1.3}[(v)].
\end{proof}

\begin{corollary}[17.3.5]
\label{IV.17.3.5}
Suppose that $g$ is \'etale;
then, for $f$ to be smooth (resp. unramified, resp. \'etale) it is necessary and sufficient that $g\circ f$ is.
\end{corollary}

\begin{proof}
This follows from \sref{IV.17.3.4} and \sref{IV.17.3.3}[(ii)].
\end{proof}

\begin{proposition}[17.3.6]
\label{IV.17.3.6}
Let $g:Y\to S$ and $h:X\to S$ be two morphisms locally of finite presentation.
For an $S$-morphism $f:X\to Y$ to be unramified, it is necessary and sufficient that the canonical homomorphism \sref{IV.16.4.19}
\[
  f^*(\Omega_{Y/S}^1)\to\Omega_{X/S}^1
\]
is surjective.
\end{proposition}

\begin{proof}
As $f$ is locally of finite presentation \sref{IV.1.4.3}[(v)], the proposition follows from \sref{IV.17.2.2}.
\end{proof}

\begin{definition}[17.3.7]
\label{IV.17.3.7}
Let $f:X\to Y$ be a morphism.
We say that $f$ is \emph{smooth} (resp. \emph{unramifed}, resp. \emph{\'etale}) at a point $x\in X$, if there exists an open neighborhood $U$ of $x$ in $X$ such that the restriction $f|U$ is a smooth (resp. unramified, resp. \'etale) morphism from $U$ to $Y$.
\end{definition}

We then also say that $X$ is \emph{smooth} (resp. \emph{unramified}, resp. \emph{\'etale}) \emph{over $Y$ at the point $x$}.

Taking into account remark \sref{IV.17.3.2}[(iii)], it is equivalent to say that $f$ is smooth (resp. unramified, resp. \'etale) and to say that $f$ is smooth (resp. unramified, resp. \'etale) at all points of $X$.

It is clear that the set of points of $X$ at which the morphism $f:X\to Y$ is smooth (resp. unramified, resp. \'etale) is \emph{open} in $X$.

\begin{proposition}[17.3.8]
\label{IV.17.3.8}
For all preschemes $Y$ and all locally free $\sh{O}_Y$-modules $\sh{E}$ of finite type, the vector bundle prescheme $\bb{V}(\sh{E})$ \sref[II]{II.1.7.8} associated to $\sh{E}$ is a smooth $Y$-prescheme.
\end{proposition}

\begin{proof}
Indeed \sref{IV.17.3.2}[(iii)], we can restrict ourselves to the case where $Y=\Spec(A)$ is affine and $\bb{V}(\sh{E})=\Spec(A[T_1,\dots,T_r])$;
as $A[T_1,\dots,T_r]$ is a formally smooth $A$-algebra for the discrete topologies \sref[0]{0.19.3.2}, and of finite presentation, this proves the proposition \sref{IV.17.3.2}[(ii)] 
\end{proof}

\begin{corollary}[17.3.9]
\label{IV.17.3.9}
Under the hypotheses of \sref{IV.17.3.8}, the projective prescheme $\bb{P}(\sh{E})$ \sref[II]{II.4.1.1} is a smooth $Y$-prescheme.
\end{corollary}

\begin{proof}
We can still restrict to the case where $Y=\Spec(A)$ is affine and $\bb{P}(\sh{E})=\bb{P}_Y^r$.
We then know \sref[II]{II.2.3.14} that we have a finite open cover of $\bb{P}^r_{A}$ by the $\operatorname{D}_+(T_i)$ ($0\leq i\leq r$) respectively equal to the spectrum of the ring $S_{(f)}$, where we wrote $S$ for $A[T_1,\dots,T_r]$ and $f$ for $T_i$;
but it follows immediately from the definition of $S_{(f)}$ \sref[II]{II.2.2.1}, that this ring, in this case, is isomorphic to $A[T_0,\dots,T_{i-1},T_{i+1},\dots,T_r]$;
hence the corollary follows by \sref{IV.17.3.8}.
\end{proof}

\subsection{Characterizations of unramified morphisms.}
\label{subection:IV.17.4} 

\oldpage[IV]{63}


