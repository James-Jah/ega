\documentclass[10pt,oneside]{amsart}

\usepackage[utf8]{inputenc}
\usepackage{subfiles}
\usepackage[left=0.85in,right=0.85in,top=0.89in,bottom=1.15in]{geometry}
\usepackage{amsmath,amssymb,amsthm}
\usepackage{mathrsfs}
\usepackage{enumitem}
\usepackage{tikz-cd}
\usepackage{epigraph}
\usepackage[super]{nth}
\usepackage{titlesec,titletoc}
\usepackage{soul}
\usepackage[hidelinks]{hyperref}
\usepackage{calligra}


% if you don't like the font comment out the following two lines
\renewcommand{\rmdefault}{pplx}
\usepackage{eulervm}

\allowdisplaybreaks[1]
%\binoppenalty=9999
%\relpenalty=9999


% section titles

\titleformat{\part}[block]{\huge}{\sc Chapter \thepart. ---\,\,}{0pt}{\filcenter\sc\Huge}[\setcounter{section}{0}]
\titleformat{\section}[block]{\huge\bfseries}{\thesection.\,\,}{0pt}{\filcenter}[\setcounter{subsection}{0}]
\titleformat{\subsection}[hang]{\bfseries}{\thesubsection.\,\,}{0pt}{}


% table of contents

\dottedcontents{part}[0em]{\sc}{2em}{1em}
\dottedcontents{section}[3em]{\bfseries}{2em}{1em}
\dottedcontents{subsection}[6em]{}{2.5em}{1em}

% center the toc heading
\renewcommand{\contentsname}{\centering Contents}


% commands and maths things

% if you are not sure of a translation
\newcommand{\unsure}[2][]{\hl{#2}\marginpar{#1}}
\newcommand{\completelyunsure}{\unsure{[\ldots]}}

% use to mark where original page starts
\newcommand{\oldpage}[1]{\marginpar{\textbf{#1}}}

\newcommand{\asttri}{%
    \begin{center}
        \begin{equation*}
            \arraycolsep=1pt
            \def\arraystretch{0.7}
            \begin{array}{rcl}
                &*&\\
                *&&*
            \end{array}
        \end{equation*}
    \end{center}\vspace{1.3em}}

% spacing 
\newcommand{\NEXT}{\medskip\noindent}

% nilradical
\newcommand{\rad}{\mathfrak{R}}

\newcommand{\RA}{\Longrightarrow}
\newcommand{\LA}{\Longleftarrow}

\newcommand{\su}{\subset}
\newcommand{\us}{\supset}

\newcommand{\emp}{\varnothing}

\renewcommand{\a}{\alpha}
\renewcommand{\b}{\beta}

\newcommand{\mf}{\mathfrak}

% for easy changes to style
\newcommand{\sheaf}{\mathcal}
\newcommand{\bbold}{\mathbf}
\newcommand{\cat}{\mathsf}

% math bold letters
\newcommand{\R}{\bbold{R}}
\newcommand{\C}{\bbold{C}}
\newcommand{\N}{\bbold{N}}
\newcommand{\Q}{\bbold{Q}}
\newcommand{\Z}{\bbold{Z}}

% residue field
\renewcommand{\k}{\bbold{k}}

% category
\newcommand{\K}{\cat{K}}

% structure sheaf
\renewcommand{\O}{\sheaf{O}}

% operators
\newcommand*{\sheafHom}{\mathscr{H}\text{\normalfont\kern -3pt {\calligra\large om}}\,}
\DeclareMathOperator{\Hom}{Hom}
\DeclareMathOperator{\Supp}{Supp}
\DeclareMathOperator{\img}{im}
\DeclareMathOperator{\Spec}{Spec}

% environments and other things

% currently this works as \begin{env}[optional prop/def]{x.y.z}
% assigns label to (chapter number).x.y.z
\makeatletter
\newenvironment{env}[2][\@nil]{%
    \label{\arabic{part}.#2}
    \def\tmp{#1}%
    \ifx\tmp\@nnil
        \par\medskip\noindent\indent\textbf{(#2)}\rmfamily
    \else
        \par\medskip\noindent\indent\textit{#1}~\textbf{(#2)}.\,---\itshape
    \fi}
\makeatother

% reference the label given by env; works as \eref{x.y.z}
\newcommand{\eref}[1]{
  \hyperref[\arabic{part}.#1]{(#1)}
}

% reference out of chapter; use as \eref{chapter number}{x.y.z}
\newcommand{\erefc}[2]{
  \hyperref[#1.#2]{(#2)}
}



\title{EGA I}
\author{A. Grothendieck}
% is this going to be the date it was last updated or the date of the original?
% \date{ }


\begin{document}


\renewcommand{\abstractname}{What this is}
\begin{abstract}
    This is a community translation of Grothendieck's EGA I.
    As it is a work in progress by multiple people, it will probably have a few mistakes --- if you spot any then please feel free to \href{https://github.com/ryankeleti/en.ega.i/issues}{let us know}!
    \thanks{\url{https://github.com/ryankeleti/en.ega.i}}
    
    On est désolés, Grothendieck.

    --- Ryan Keleti, Tim Hosgood
\end{abstract}

\maketitle

\noindent\hspace{0.15\linewidth}
\begin{minipage}{0.7\linewidth}
    \tableofcontents{}
\end{minipage}


\clearpage


\part*{Introduction}
\subfile{sections/intro}


\clearpage


\setcounter{part}{-1}

\part{Preliminaries}

    \section{Rings of Fractions}
    \setcounter{subsection}{-1}

        \subsection{Rings and Algebras}
        \subfile{sections/0-prelim/prelim.1.0}

        \subsection{Root (radical) of an ideal. Nilradical and radical of a ring.}
        \subfile{sections/0-prelim/prelim.1.1}

        \subsection{Modules and rings of fractions.}
        \subfile{sections/0-prelim/prelim.1.2}

        \subsection{Functorial properties.}
        \subfile{sections/0-prelim/prelim.1.3}

        \subsection{Change of multiplicative subset.}
        \subfile{sections/0-prelim/prelim.1.4}

        \subsection{Change of ring.}
        \subfile{sections/0-prelim/prelim.1.5}

        \subsection{Indentification of the module $M_f$ as an inductive limit.}
        \subfile{sections/0-prelim/prelim.1.6}

        \subsection{Support of a module.}
        \subfile{sections/0-prelim/prelim.1.7}

    \section{Irreducible spaces. Noetherian spaces.}

        \subsection{Irreducible spaces.}
        \subfile{sections/0-prelim/prelim.2.1}

        \subsection{Noetherian spaces.}
        \subfile{sections/0-prelim/prelim.2.2}

\clearpage


\setcounter{subsection}{0}
\part{The language of schemes}
    
    \section*{Summary}
    \subfile{sections/1-schemes/schemes.summary}

    \section{Affine schemes}
       
       \subsection{The prime spectrum of a ring}
       \subfile{sections/1-schemes/schemes.1.1}
       
       \subsection{Functorial properties of prime spectra of rings}
       \subfile{sections/1-schemes/schemes.1.2}
       
       \subsection{Sheaf associated to a module}
       \subfile{sections/1-schemes/schemes.1.3}
       
       \subsection{Quasi-coherent sheaves over a prime spectrum}
       \subfile{sections/1-schemes/schemes.1.4}
       
       \subsection{Coherent sheaves over a prime spectrum}
       \subfile{sections/1-schemes/schemes.1.5}
       
       \subsection{Functorial properties of quasi-coherent sheaves over a prime spectrum}
       \subfile{sections/1-schemes/schemes.1.6}
       
       \subsection{Characterisation of morphisms of affine schemes}
       \subfile{sections/1-schemes/schemes.1.7}

    \section{Preschemes and morphisms of preschemes}

        \subsection{Definition of preschemes}
        \subfile{sections/1-schemes/schemes.2.1}

        \subsection{Morphisms of preschemes}
        \subfile{sections/1-schemes/schemes.2.2}

        \subsection{Gluing of preschemes}
        \subfile{sections/1-schemes/schemes.2.3}

        \subsection{Local schemes}
        \subfile{sections/1-schemes/schemes.2.4}

        \subsection{Preschemes over a prescheme}
        \subfile{sections/1-schemes/schemes.2.5}

    \section{Products of preschemes}

    \section{Sub-preschemes and immersion morphisms}

    \section{Reduced preschemes; separation conditions}

    \section{Finiteness conditions}

    \section{Rational maps}

    \section{Chevalley schemes}

    \section{Details on quasi-coherent sheaves}

    \section{Formal schemes}


\end{document}

