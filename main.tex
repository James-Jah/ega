\documentclass[10pt]{amsart}

\usepackage[utf8]{inputenc}
\usepackage{subfiles}
\usepackage[left=0.85in,right=0.85in,top=0.89in,bottom=1.15in]{geometry}
\usepackage{amsmath,amssymb,amsthm}
\usepackage{mathrsfs}
\usepackage{enumitem}
\usepackage{tikz-cd}
\usepackage{epigraph}
\usepackage[super]{nth}

% if you don't like the font comment out the following two lines
\renewcommand{\rmdefault}{pplx}
\usepackage{eulervm}

\allowdisplaybreaks[1]

\title{EGA I}
\author{A. Grothendieck}
% is this going to be the date it was last updated or the date of the original?
% \date{ }

\newcommand{\asttri}{%
    \begin{center}
        \begin{equation*}
            \arraycolsep=1pt
            \def\arraystretch{0.7}
            \begin{array}{rcl}
                &*&\\
                *&&*
            \end{array}
        \end{equation*}
    \end{center}}

\newcommand{\LS}{\begin{enumerate}[label=(\alph*)]}
\newcommand{\LSA}{\begin{enumerate}[label=(\arabic*)]}
\newcommand{\LSI}{\begin{enumerate}[label=(\roman*)]}
\newcommand{\LE}{\end{enumerate}}
\newcommand{\NEXT}{\medskip\noindent}
\newcommand{\bb}{\mathbb}
\newcommand{\scr}{\mathscr}
\newcommand{\cal}{\mathcal}
\newcommand{\mf}{\mathfrak}
\newcommand{\OP}{\operatorname}
\newcommand{\TS}{\textstyle}
\newcommand{\ro}{\mathfrak{r}}
\newcommand{\rad}{\mathfrak{R}}
\newcommand{\RA}{\Rightarrow}
\newcommand{\LA}{\Leftarrow}
\newcommand{\su}{\subset}
\newcommand{\emp}{\varnothing}
%\newcommand{\ssm}{\smallsetminus}

\def\R{\mathbb{R}}
\def\C{\mathbb{C}}
\def\N{\mathbb{N}}
\def\Q{\mathbb{Q}}
\def\Z{\mathbb{Z}}

\def\O{\mathcal{O}}

\DeclareMathOperator*{\Hom}{Hom}
\DeclareMathOperator*{\img}{im}
%\DeclareMathOperator*{\pr}{pr}

% \theoremstyle{plain}
% \newtheorem{thm}{Theorem}[section]
% \newtheorem{prop}[thm]{Proposition}
% \newtheorem{lem}[thm]{Lemma}
% \newtheorem{coro}[thm]{Corollary}

% \theoremstyle{definition}
% \newtheorem{defn}[thm]{Definition}

% currently this works as \begin{cx}[optional prop/def]{x.y.z}
\makeatletter
\newenvironment{cx}[2][\@nil]{%
    \def\tmp{#1}%
    \ifx\tmp\@nnil
        \par\medskip\noindent\indent\textbf{(#2)}\rmfamily
    \else
        \par\medskip\noindent\indent\textit{#1}~\textbf{(#2)}.\,---\itshape
    \fi}{\par\medskip}
\makeatother

\begin{document}
\maketitle
%\setcounter{section}{0}
\tableofcontents{}
%\mbox{}

\section*{What is this?}

This is my poor translation of Grothendieck's EGA I. This
will probably consist of lots of online translations and incorrect grammar.
You have been warned!

S'il te pla\^it pardonne-moi, Grothendieck.

Ryan Keleti :)

\section*{Introduction}
\subfile{sections/intro}

% make this into a chapter heading later?
Chapter 0. --- Preliminaries
\section*{Rings of Fractions}
\subsection*{Rings and Algebras}\mbox{}\\
\subfile{sections/prelim.1.0}
\subsection*{Root (radical) of an ideal. Nilradical and radical of a ring.}\mbox{}\\
\subfile{sections/prelim.1.1}
\subsection*{Modules and rings of fractions}\mbox{}\\
\subfile{sections/prelim.1.2}

\newpage

% make this into a chapter heading later?
Chapter 1. --- The language of schemes
\section*{Preschemes and morphisms of preschemes}
\subsection*{Definition of preschemes}\mbox{}\\
\subfile{sections/schemes.2.1}


\end{document}

