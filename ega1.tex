\documentclass[oneside]{amsart}

\usepackage[all]{xy}
\usepackage[T1]{fontenc}
\usepackage{xstring}
\usepackage{xparse}
\usepackage{xr-hyper}
\usepackage[linktocpage=true,colorlinks=true,hyperindex,citecolor=blue,linkcolor=magenta]{hyperref}
\usepackage[left=0.95in,right=0.95in,top=0.75in,bottom=0.75in]{geometry}
\usepackage[charter,ttscaled=false,greekfamily=didot,greeklowercase=upright]{mathdesign}

\usepackage{Baskervaldx}

\usepackage{enumitem}
\usepackage{longtable}
\usepackage{aurical}

\externaldocument[what-]{what}
\externaldocument[intro-]{intro}
\externaldocument[ega0-]{ega0}
\externaldocument[ega1-]{ega1}
\externaldocument[ega2-]{ega2}
\externaldocument[ega3-]{ega3}
\externaldocument[ega4-]{ega4}

\newtheoremstyle{ega-env-style}%
  {}{}{\rmfamily}{}{\bfseries}{.}{ }{\thmnote{(#3)}}%

\newtheoremstyle{ega-thm-env-style}%
  {}{}{\itshape}{}{\bfseries}{. --- }{ }{\thmname{#1}\thmnote{ (#3)}}%

\newtheoremstyle{ega-defn-env-style}%
  {}{}{\rmfamily}{}{\bfseries}{. --- }{ }{\thmname{#1}\thmnote{ (#3)}}%

\theoremstyle{ega-env-style}
\newtheorem*{env}{---}

\theoremstyle{ega-thm-env-style}
\newtheorem*{thm}{Theorem}
\newtheorem*{prop}{Proposition}
\newtheorem*{lem}{Lemma}
\newtheorem*{cor}{Corollary}

\theoremstyle{ega-defn-env-style}
\newtheorem*{defn}{Definition}
\newtheorem*{exm}{Example}
\newtheorem*{rmk}{Remark}
\newtheorem*{nota}{Notation}

% indent subsections, see https://tex.stackexchange.com/questions/177290/.
% also make section titles bigger.
% also add § to \thesection, https://tex.stackexchange.com/questions/119667/ and https://tex.stackexchange.com/questions/308737/.
\makeatletter
\def\l@subsection{\@tocline{2}{0pt}{2.5pc}{1.5pc}{}}
\def\section{\@startsection{section}{1}%
  \z@{.7\linespacing\@plus\linespacing}{.5\linespacing}%
  {\normalfont\bfseries\Large\scshape\centering}}
\renewcommand{\@seccntformat}[1]{%
  \ifnum\pdfstrcmp{#1}{section}=0\textsection\fi%
  \csname the#1\endcsname.~}
\makeatother

%\allowdisplaybreaks[1]
%\binoppenalty=9999
%\relpenalty=9999

% for Chapter 0, Chapter I, etc.
% credit for ZeroRoman https://tex.stackexchange.com/questions/211414/
% added into scripts/make_book.py
%\newcommand{\ZeroRoman}[1]{\ifcase\value{#1}\relax 0\else\Roman{#1}\fi}
%\renewcommand{\thechapter}{\ZeroRoman{chapter}}

\def\mathcal{\mathscr}
\def\sh{\mathcal}                   % sheaf font
\def\bb{\mathbf}                    % bold font
\def\cat{\mathtt}                   % category font
\def\fk{\mathfrak}                  % mathfrak font
\def\leq{\leqslant}                 % <=
\def\geq{\geqslant}                 % >=
\def\wt#1{{\widetilde{#1}}}         % tilde over
\def\wh#1{{\widehat{#1}}}           % hat over
\def\setmin{-}                      % set minus
\def\rad{\fk{r}}                    % radical
\def\nilrad{\fk{R}}                 % nilradical
\def\emp{\varnothing}               % empty set
\def\vphi{\phi}                     % for switching \phi and \varphi, change if needed
\def\HH{\mathrm{H}}                 % cohomology H
\def\CHH{\check{\HH}}               % Čech cohomology H
\def\RR{\mathrm{R}}                 % right derived R
\def\LL{\mathrm{L}}                 % left derived L
\def\dual#1{{#1}^\vee}              % dual
\def\kres{k}                        % residue field k
\def\C{\cat{C}}                     % category C
\def\op{^\cat{op}}                  % opposite category
\def\Set{\cat{Set}}                 % category of sets
\def\CHom{\cat{Hom}}                % functor category
\def\OO{\sh{O}}                     % structure sheaf O

\def\shHom{\sh{H}\textup{\kern-2.2pt{\Fontauri\slshape om}}\!}   % sheaf Hom
\def\shProj{\sh{P}\textup{\kern-2.2pt{\Fontauri\slshape roj}}\!} % sheaf Proj
\def\shExt{\sh{E}\textup{\kern-2.2pt{\Fontauri\slshape xt}}\!}   % sheaf Ext
\def\red{\mathrm{red}}
\def\rg{{\mathop{\mathrm{rg}}\nolimits}}
\def\gr{{\mathop{\mathrm{gr}}\nolimits}}
\def\Hom{{\mathop{\mathrm{Hom}}\nolimits}}
\def\Proj{{\mathop{\mathrm{Proj}}\nolimits}}
\def\Tor{{\mathop{\mathrm{Tor}}\nolimits}}
\def\Ext{{\mathop{\mathrm{Ext}}\nolimits}}
\def\Supp{{\mathop{\mathrm{Supp}}\nolimits}}
\def\Ker{{\mathop{\mathrm{Ker}}\nolimits}\,}
\def\Im{{\mathop{\mathrm{Im}}\nolimits}\,}
\def\Coker{{\mathop{\mathrm{Coker}}\nolimits}\,}
\def\Spec{{\mathop{\mathrm{Spec}}\nolimits}}
\def\Spf{{\mathop{\mathrm{Spf}}\nolimits}}
\def\grad{{\mathop{\mathrm{grad}}\nolimits}}
\def\dim{{\mathop{\mathrm{dim}}\nolimits}}
\def\dimc{{\mathop{\mathrm{dimc}}\nolimits}}
\def\codim{{\mathop{\mathrm{codim}}\nolimits}}

\renewcommand{\to}{\mathchoice{\longrightarrow}{\rightarrow}{\rightarrow}{\rightarrow}}
\let\mapstoo\mapsto
\renewcommand{\mapsto}{\mathchoice{\longmapsto}{\mapstoo}{\mapstoo}{\mapstoo}}
\def\isoto{\simeq}  % isomorphism

% if unsure of a translation
%\newcommand{\unsure}[2][]{\hl{#2}\marginpar{#1}}
%\newcommand{\completelyunsure}{\unsure{[\ldots]}}
\def\unsure#1{#1 {\color{red}(?)}}
\def\completelyunsure{{\color{red}(???)}}

% use to mark where original page starts
\newcommand{\oldpage}[2][--]{{\marginpar{\textbf{#1}~|~#2}}\ignorespaces}
\def\sectionbreak{\begin{center}***\end{center}}

% for referencing environments.
% use as \sref{chapter-number.x.y.z}, with optional args
% for volume and indices, e.g. \sref[volume]{chapter-number.x.y.z}[i].
\NewDocumentCommand{\sref}{o m o}{%
  \IfNoValueTF{#1}%
    {\IfNoValueTF{#3}%
      {\hyperref[#2]{\normalfont{(\StrGobbleLeft{#2}{2})}}}%
      {\hyperref[#2]{\normalfont{(\StrGobbleLeft{#2}{2},~{#3})}}}}%
    {\IfNoValueTF{#3}%
      {\hyperref[#2]{\normalfont{(\textbf{#1},~\StrGobbleLeft{#2}{2})}}}%
      {\hyperref[#2]{\normalfont{(\textbf{#1},~\StrGobbleLeft{#2}{2},~({#3}))}}}}%
}



\begin{document}
\title{The language of schemes (EGA I)}
\maketitle

\phantomsection
\label{section:ega1}

\tableofcontents

\section*{Summary}

\begin{longtable}{ll}
  \textsection\hyperref[section:I.1]{1}.   & Affine schemes.\\
  \textsection\hyperref[section:I.2]{2}.   & Preschemes and morphisms of preschemes.\\
  \textsection\hyperref[section:I.3]{3}.   & Products of preschemes.\\
  \textsection\hyperref[section:I.4]{4}.   & Subpreschemes and immersion morphisms.\\
  \textsection\hyperref[section:I.5]{5}.   & Reduced preschemes; separation condition.\\
  \textsection\hyperref[section:I.6]{6}.   & Finiteness conditions.\\
  \textsection\hyperref[section:I.7]{7}.   & Rational maps.\\
  \textsection\hyperref[section:I.8]{8}.   & Chevalley schemes.\\
  \textsection\hyperref[section:I.9]{9}.   & Supplement on quasi-coherent sheaves.\\
  \textsection\hyperref[section:I.10]{10}. & Formal schemes.
\end{longtable}
\bigskip

\oldpage[I]{79}
In \textsection\textsection1--8 we do little more than develop a language to be used in what follows.
It should be noted, however, that, in accordance with the general spirit of this treatise, \textsection\textsection7--8 will be used less than the others, and in a less essential way; we speak of Chevalley schemes only for the purpose of relating to the language of Chevalley \cite{I-1} and Nagata \cite{I-9}.
Then \textsection9 gives definitions and results concerning quasi-coherent sheaves, some of which are no longer simply a translation of known notions of commutative algebra into a ``geometric'' language, but are instead already of a global nature; they will be indispensable, in the following chapters, when it comes to the global study of morphisms.
Finally, \textsection10 introduces a generalization of the notion of a scheme, which will be used as an intermediary in Chapter~III to formulate and prove, in a convenient way, the fundamental results of the cohomological study of proper morphisms;
moreover, it should be noted that the notion of formal schemes seems indispensable in expressing certain facts about the ``theory of modules'' (classification problems of algebraic varieties).
The results of \textsection10 will not be used before \textsection3 of Chapter~III, and it is recommended to skip their reading until then.
\bigskip

\setcounter{section}{0}
\section{Affine schemes}
\label{section:1.1}

\setcounter{subsection}{0}
\subsection{The prime spectrum of a ring}
\label{subsection:1.1.1}

\begin{env}[1.1.1]
\label{1.1.1.1}
\oldpage[I]{80}
\emph{Notation}. Let $A$ be a (commutative) ring, and $M$ an $A$-module.
In this chapter and the following, we will constantly use the following notation:
\begin{itemize}
  \item $\Spec(A)=$ \emph{set of prime ideals} of $A$, also called the \emph{prime spectrum} of $A$; for $x\in X=\Spec(A)$, it will often be convenient to write $\mathfrak{j}_x$ instead of $x$.
    For $\Spec(A)$ to be \emph{empty}, it is necessary and sufficient for the ring $A$ to be $0$.
  \item $A_x=A_{\mathfrak{j}_x}=$ \emph{(local) ring of fractions $S^{-1}A$}, where $S=A\setmin\mathfrak{j}_x$.
  \item $\mathfrak{m}_x=\mathfrak{j}_x A_{\mathfrak{j}_x}=$ \emph{maximal ideal of $A$}.
  \item $\kres(x)=A_x/\mathfrak{m}_x=$ \emph{residue field of $A_x$}, canonically isomorphic to the field of fractions of the integral ring $A/\mathfrak{j}_x$, with which we identify it.
  \item $f(x)=$ \emph{class of $f$ mod.~$\mathfrak{j}_x$} in $A/\mathfrak{j}_x\subset\kres(x)$, for $f\in A$ and $x\in X$.
    We also say that $f(x)$ is the \emph{value} of $f$ at a point $x\in\Spec(A)$; the equations $f(x)=0$ and $f\in\mathfrak{j}_x$ are \emph{equivalent}.
  \item $M_x=M\otimes_A A_x=$ \emph{module of fractions with denominators in $A\setmin\mathfrak{j}_x$}.
  \item $\rad(E)=$ \emph{radical of the ideal of $A$ generated by a subset $E$ of $A$}.
  \item $V(E)=$ \emph{set of $x\in X$ such that $E\subset\mathfrak{j}_x$} (or the set of $x\in X$ such that $f(x)=0$ for all $f\in E$), for $E\subset A$.
    So we have
    \[
    \label{eq:1.1.1.1.1}
      \rad(E)=\bigcap_{x\in V(E)}\mathfrak{j}_x.
      \tag{1.1.1.1}
    \]
  \item $V(f)=V(\{f\})$ for $f\in A$.
  \item $D(f)=X\setmin V(f)=$ \emph{set of $x\in X$ where $f(x)\neq 0$}.
\end{itemize}
\end{env}

\begin{proposition}[1.1.2]
\label{1.1.1.2}
We have the following properties:
\begin{enumerate}
  \item[{\rm(i)}] $V(0)=X$, $V(1)=\emp$.
  \item[{\rm(ii)}] The relation $E\subset E'$ implies $V(E)\supset V(E')$.
  \item[{\rm(iii)}] For each family $(E_\lambda)$ of subsets of $A$, $V(\bigcup_\lambda E_\lambda)=V(\sum_\lambda E_\lambda)=\bigcap_\lambda V(E_\lambda)$.
  \item[{\rm(iv)}] $V(EE')=V(E)\cup V(E')$.
  \item[{\rm(v)}] $V(E)=V(\mathfrak{r}(E))$.
\end{enumerate}
\end{proposition}

\begin{proof}
\label{proof-1.1.1.2}
The properties (i), (ii), (iii) are trivial, and (v) follows from (ii) and from equation \sref{1.1.1.1.1}.
It is evident that $V(EE')\supset V(E)\cap V(E')$;
conversely, if $x\not\in V(E)$ and $x\not\in V(E')$, then there exists $f\in E$ and $f'\in E'$ such that $f(x)\neq 0$ and $f'(x)\neq 0$ in $\kres(x)$, hence $f(x)f'(x)\neq 0$, i.e., $x\not\in V(EE')$, which proves (iv).
\end{proof}

Proposition \sref{1.1.1.2} shows, among other things, that sets of the form $V(E)$ (where $E$ varies over the subsets of $A$) are the \emph{closed sets} of a topology on $X$, which we will call the \emph{spectral topology}
\footnote{The introduction of this topology in algebraic geometry is due to Zariski.
So this topology is usually called the ``Zariski topology'' on $X$.};
unless expressly stated otherwise, we always assume that $X=\Spec(A)$ is equipped with the spectral topology.

\begin{env}[1.1.3]
\label{1.1.1.3}
\oldpage[I]{81}
For each subset $Y$ of $X$, we denote by $\mathfrak{j}(Y)$ the set of $f\in A$ such that $f(y)=0$ for all $y\in Y$;
equivalently, $\mathfrak{j}(Y)$ is the intersection of the prime ideals $\mathfrak{j}_y$ for $y\in Y$.
It is clear that the relation $Y\subset Y'$ implies that $\mathfrak{j}(Y)\supset\mathfrak{j}(Y')$ and that we have
\[
  \mathfrak{j}\bigg(\bigcup_\lambda Y_\lambda\bigg)=\bigcap_\lambda\mathfrak{j}(Y_\lambda)
  \tag{1.1.3.1}
\]
for each family $(Y_\lambda)$ of subsets of $X$.
Finally we have
\[
  \mathfrak{j}(\{x\})=\mathfrak{j}_x.
  \tag{1.1.3.2}
\]
\end{env}

\begin{proposition}[1.1.4]
\label{1.1.1.4}
\medskip\noindent
\begin{enumerate}
  \item[{\rm(i)}] For each subset $E$ of $A$, we have $\mathfrak{j}(V(E))=\rad(E)$.
  \item[{\rm(ii)}] For each subset $Y$ of $X$, $V(\mathfrak{j}(Y))=\overline{Y}$, the closure of $Y$ in $X$.
\end{enumerate}
\end{proposition}

\begin{proof}
\label{proof-1.1.1.4}
(i) is an immediate consequence of the definitions and (1.1.1.1);
on the other hand, $V(\mathfrak{j}(Y))$ is closed and contains $Y$;
conversely, if $Y\subset V(E)$, we have $f(y)=0$ for $f\in E$ and all $y\in Y$, so $E\subset\mathfrak{j}(Y)$, $V(E)\supset V(\mathfrak{j}(Y))$, which proves (ii).
\end{proof}

\begin{corollary}[1.1.5]
\label{1.1.1.5}
The closed subsets of $X=\Spec(A)$ and the ideals of $A$ equal to their radicals (in other words, those that are the intersection of prime ideals) correspond bijectively by the inclusion-reversing maps $Y\mapsto\mathfrak{j}(Y)$, $\mathfrak{a}\mapsto V(\mathfrak{a})$;
the union $Y_1\cup Y_2$ of two closed subsets corresponds to $\mathfrak{j}(Y_1)\cap\mathfrak{j}(Y_2)$, and the intersection of any family $(Y_\lambda)$ of closed subsets corresponds to the radical of the sum of the $\mathfrak{j}(Y_\lambda)$.
\end{corollary}

\begin{corollary}[1.1.6]
\label{1.1.1.6}
If $A$ is a Noetherian ring, $X=\Spec(A)$ is a Noetherian space.
\end{corollary}

Note that the converse of this corollary is false, as shown by any non-Noetherian integral ring with a single prime ideal $\neq\{0\}$ (for example a nondiscrete valuation ring of rank $1$).

As an example of ring $A$ whose spectrum is not a Noetherian space, one can consider the ring $\sh{C}(Y)$ of continuous real functions on an infinite compact space $Y$;
we know that, as a set, $Y$ corresponds to the set of maximal ideals of $A$, and it is easy to see that the topology induced on $Y$ by that of $X=\Spec(A)$ is the original topology of $Y$.
Since $Y$ is not a Noetherian space, the same is true for $X$.

\begin{corollary}[1.1.7]
\label{1.1.1.7}
For each $x\in X$, the closure of $\{x\}$ is the set of $y\in X$ such that $\mathfrak{j}_x\subset\mathfrak{j}_y$.
For $\{x\}$ to be closed, it is necessary and sufficient that $\mathfrak{j}_x$ is maximal.
\end{corollary}

\begin{corollary}[1.1.8]
\label{1.1.1.8}
The space $X=\Spec(A)$ is a Kolmogoroff space.
\end{corollary}

\begin{proof}
\label{proof-1.1.1.8}
If $x$, $y$ are two distinct points of $X$, we have either $\mathfrak{j}_x\not\subset\mathfrak{j}_y$ or $\mathfrak{j}_y\not\subset\mathfrak{j}_x$, so one of the points $x$, $y$ does not belong to the closure of the other.
\end{proof}

\begin{env}[1.1.9]
\label{1.1.1.9}
According to Proposition \sref{1.1.1.2}[iv], for two elements $f$, $g$ of $A$, we have
\[
  D(fg)=D(f)\cap D(g).
  \tag{1.1.9.1}
\]
Note also that the equality $D(f)=D(g)$ means, according to Proposition~\sref{1.1.1.4}[i] and Proposition~\sref{1.1.1.2}[v], that $\rad(f)=\rad(g)$, or that the minimal prime ideals containing $(f)$ and $(g)$ are the same;
in particular, it is also the case when $f=ug$, where $u$ is invertible.
\end{env}

\begin{proposition}[1.1.10]
\label{1.1.1.10}
\medskip\noindent
\oldpage[I]{82}
\begin{enumerate}
  \item[{\rm(i)}] When $f$ ranges over $A$, the sets $D(f)$ forms a basis for the topology of $X$.
  \item[{\rm(ii)}] For every $f\in A$, $D(f)$ is quasi-compact.
    In particular, $X=D(1)$ is quasi-compact.
\end{enumerate}
\end{proposition}

\begin{proof}
\label{proof-1.1.1.10}
\medskip\noindent
\begin{enumerate}
  \item[(i)] Let $U$ be an open set in $X$;
    by definition, we have $U=X\setmin V(E)$ where $E$ is a subset of $A$, and $V(E)=\bigcap_{f\in E}V(f)$, hence $U=\bigcup_{f\in E}D(f)$.
  \item[(ii)] By (i), it suffices to prove that, if $(f_\lambda)_{\lambda\in L}$ is a family of elements of $A$ such that $D(f)\subset\bigcup_{\lambda\in L}D(f_\lambda)$, then there exists a finite subset $J$ of $L$ such that $D(f)\subset\bigcup_{\lambda\in J}D(f_\lambda)$.
    Let $\mathfrak{a}$ be the ideal of $A$ generated by the $f_\lambda$;
    we have, by hypothesis, that $V(f)\supset V(\mathfrak{a})$, so $\rad(f)\subset\rad(\mathfrak{a})$;
    since $f\in\rad(f)$, there exists an integer $n\geq 0$ such that $f^n\in\mathfrak{a}$.
    But then $f^n$ belongs to the ideal $\mathfrak{b}$ generated by the finite subfamily $(f_\lambda)_{\lambda\in J}$, and we have $V(f)=V(f^n)\supset V(\mathfrak{b})=\bigcap_{\lambda\in J}V(f_\lambda)$, that is to say, $D(f)\supset\bigcup_{\lambda\in J}D(f_\lambda)$.
\end{enumerate}
\end{proof}

\begin{proposition}[1.1.11]
\label{1.1.1.11}
For each ideal $\mathfrak{a}$ of $A$, $\Spec(A/\mathfrak{a})$ is canonically identified with the closed subspace $V(\mathfrak{a})$ of $\Spec(A)$.
\end{proposition}

\begin{proof}
\label{proof-1.1.1.11}
We know there is a canonical bijective correspondence (respecting the inclusion order structure) between ideals (resp. prime ideals) of $A/\mathfrak{a}$ and ideals (resp. prime ideals) of $A$ containing $\mathfrak{a}$.
\end{proof}

Recall that the set $\nilrad$ of nilpotent elements of $A$ (the \emph{nilradical} of $A$) is an ideal equal to $\rad(0)$, the intersection of all the prime ideals of $A$ \sref[0]{0.1.1.1}.

\begin{corollary}[1.1.12]
\label{1.1.1.12}
The topological spaces $\Spec(A)$ and $\Spec(A/\nilrad)$ are canonically homeomorphic.
\end{corollary}

\begin{proposition}[1.1.13]
\label{1.1.1.13}
For $X=\Spec(A)$ to be irreducible \sref[0]{0.2.1.1}, it is necessary and sufficient that the ring $A/\nilrad$ is integral (or, equivalently, that the ideal $\nilrad$ is prime).
\end{proposition}

\begin{proof}
\label{proof-1.1.1.13}
By virtue of Corollary \sref{1.1.1.12}, we can restrict to the case where $\nilrad=0$.
If $X$ is reducible, then there exist two distinct closed subsets $Y_1$ and $Y_2$ of $X$ such that $X=Y_1\cup Y_2$, so $\mathfrak{j}(X)=\mathfrak{j}(Y_1)\cap\mathfrak{j}(Y_2)=0$, since the ideals $\mathfrak{j}(Y_1)$ and $\mathfrak{j}(Y_2)$ are distinct from $(0)$ \sref{1.1.1.5};
so $A$ is not integral.
Conversely, if there are elements $f\neq 0$, $g\neq 0$ of $A$ such that $fg=0$, we have $V(f)\neq X$, $V(g)\neq X$ (since the intersection of all the prime ideals of $A$ is $(0)$), and $X=V(fg)=V(f)\cup V(g)$.
\end{proof}

\begin{corollary}[1.1.14]
\label{1.1.1.14}
\medskip\noindent
\begin{enumerate}
  \item[{\rm(i)}] In the bijective correspondence between closed subsets of $X=\Spec(A)$ and ideals of $A$ equal to their radicals, the irreducible closed subsets of $X$ correspond to the prime ideals of $A$.
    In particular, the irreducible components of $X$ correspond to the minimal prime ideals of $A$.
  \item[{\rm(ii)}] The map $x\mapsto\overline{\{x\}}$ establishes a bijective correspondence between $X$ and the set of closed irreducible subsets of $X$ (\emph{in other words,} all closed irreducible subsets of $X$ admit exactly one generic point).
\end{enumerate}
\end{corollary}

\begin{proof}
\label{proof-1.1.1.14}
(i) follows immediately from \sref{1.1.1.13} and \sref{1.1.1.11};
and for proving (ii), we can, by \sref{1.1.1.11}, restrict to the case where $X$ is irreducible;
then, according to Proposition~\sref{1.1.1.13}, there exists a smaller prime ideal $\nilrad$ in $A$, which corresponds to the generic point
\oldpage[I]{83}
of $X$;
in addition, $X$ admits at most one generic point since it is a Kolmogoroff space (\sref{1.1.1.8} and \sref[0]{0.2.1.3}).
\end{proof}

\begin{proposition}[1.1.15]
\label{1.1.1.15}
If $\mathfrak{J}$ is an ideal in $A$ containing the radical $\nilrad(A)$, the only
neighborhood of $V(\mathfrak{J})$ in $X=\Spec(A)$ is the whole space $X$.
\end{proposition}

\begin{proof}
\label{proof-1.1.1.15}
Each maximal ideal of $A$ belongs, by definition, to $V(\mathfrak{J})$.
As each ideal $\mathfrak{a}\neq A$ of $A$ is contained in a maximal ideal, we have $V(\mathfrak{a})\cap V(\mathfrak{J})\neq 0$, whence the proposition.
\end{proof}

\subsection{Functorial properties of prime spectra of rings}
\label{subsection:1.1.2}

\begin{env}[1.2.1]
\label{1.1.2.1}
Let $A$, $A'$ be two rings, and
\[
  \vphi:A'\to A
\]
a homomorphism of rings.
For each prime ideal $x=\mathfrak{j}_x\in\Spec(A)=X$, the ring $A'/\vphi^{-1}(\mathfrak{j}_x)$ is canonically isomorphic to a subring of $A/\mathfrak{j}_x$, and so it is integral, or, in other words, $\vphi^{-1}(\mathfrak{j}_x)$ is a prime ideal of $A'$;
we denote it by ${}^a\vphi(x)$, and we have thus defined a map
\[
  {}^a\vphi:X=\Spec(A)\to X'=\Spec(A')
\]
(also denoted $\Spec(\vphi)$), that we call the map \emph{associated} to the homomorphism $\vphi$.
We denote by $\vphi^x$ the injective homomorphism from $A'/\vphi^{-1}(\mathfrak{j}_x)$ to $A/\mathfrak{j}_x$ induced by $\vphi$ by passing to quotients, as well as its canonical extension to a monomorphism of fields
\[
  \vphi^x:\kres({}^a\vphi(x))\to\kres(x);
\]
for each $f'\in A'$, we therefore have, by definition,
\[
  \vphi^x(f'({}^a\vphi(x)))=(\vphi(f'))(x)\qquad(x\in X).
  \tag{1.2.1.1}
\]
\end{env}

\begin{proposition}[1.2.2]
\label{1.1.2.2}
\medskip\noindent
\begin{enumerate}
  \item[{\rm(i)}] For each subset $E'$ of $A'$, we have
    \[
    \label{eq:1.1.2.2.1}
      {}^a\vphi^{-1}(V(E'))=V(\vphi(E')),
      \tag{1.2.2.1}
    \]
    and in particular, for each $f'\in A'$,
    \[
    \label{eq:1.1.2.2.2}
      {}^a\vphi^{-1}(D(f'))=D(\vphi(f')).
      \tag{1.2.2.2}
    \]
  \item[{\rm(ii)}] For each ideal $\mathfrak{a}$ of $A$, we have
    \[
    \label{eq:1.1.2.2.3}
      \overline{{}^a\vphi(V(\mathfrak{a}))}=V(\vphi^{-1}(\mathfrak{a})).
      \tag{1.2.2.3}
    \]
\end{enumerate}
\end{proposition}

\begin{proof}
\label{proof-1.1.2.2}
The relation ${}^a\vphi(x)\in V(E')$ is, by definition, equivalent to $E'\subset\vphi^{-1}(\mathfrak{j}_x)$, so $\vphi(E')\subset\mathfrak{j}_x$, and finally $x\in V(\vphi(E'))$, hence (i).
To prove (ii), we can suppose that $\mathfrak{a}$ is equal to its radical, since $V(\rad(\mathfrak{a}))=V(\mathfrak{a})$ \sref{1.1.1.2}[v] and $\vphi^{-1}(\rad(\mathfrak{a}))=\rad(\vphi^{-1}(\mathfrak{a}))$;
the relation $f'\in\mathfrak{a}'$ is, by definition, equivalent to $f'(x')=0$ for each $x\in{{}^a\vphi(Y)}$, so, by Equation~\hyperref[1.1.2.1]{(1.2.1.1)}, it is also equivalent to $\vphi(f')(x)=0$ for each $x\in Y$, or to $\vphi(f')\in\mathfrak{j}(Y)=\mathfrak{a}$, since $\mathfrak{a}$ is equal to its radical;
hence (ii).
\end{proof}

\begin{corollary}[1.2.3]
\label{1.1.2.3}
The map ${}^a\vphi$ is continuous.
\end{corollary}

We remark that, if $A''$ is a third ring, and $\vphi'$ a homomorphism $A''\to A'$, then we have ${}^a(\vphi'\circ\vphi)={}^a\vphi\circ{}^a\vphi'$;
this result, with Corollary~\sref{1.1.2.3}, says that $\Spec(A)$ is a \emph{contravariant functor} in $A$, from the category of rings to that of topological spaces.

\begin{corollary}[1.2.4]
\label{1.1.2.4}
\oldpage[I]{84}
Suppose that $\vphi$ is such that every $f\in A$ can be written as $f=h\vphi(f')$, where $h$ is invertible in $A$ (\emph{which, in particular, is the case when $\vphi$ is \emph{surjective}}).
Then ${}^a\vphi$ is a homeomorphism from $X$ to ${}^a\vphi(X)$.
\end{corollary}

\begin{proof}
\label{proof-1.1.2.4}
We show that for each subset $E\subset A$, there exists a subset $E'$ of $A'$ such that $V(E)=V(\vphi(E'))$;
according to the ($T_0$) axiom \sref{1.1.1.8} and the formula \hyperref[1.1.2.2]{(1.2.2.1)}, this implies first of all that ${}^a\vphi$ is injective, and then, by \hyperref[1.1.2.2]{(1.2.2.1)}, that ${}^a\vphi$ is a homeomorphism.
But it suffices, for each $f\in E$, to take $f'\in A'$ such that $h\vphi(f')=f$ with $h$ invertible in $A$;
the set $E'$ of these elements $f'$ is exactly what we are searching for.
\end{proof}

\begin{env}[1.2.5]
\label{1.1.2.5}
In particular, when $\vphi$ is the canonical homomorphism from $A$ to a ring quotient $A/\mathfrak{a}$, we again get \sref{1.1.1.12}, and ${}^a\vphi$ is the \emph{canonical injection} of $V(\mathfrak{a})$, identified with $\Spec(A/\mathfrak{a})$, into $X=\Spec(A)$.
\end{env}

Another particular case of \sref{1.1.2.4}:
\begin{corollary}[1.2.6]
\label{1.1.2.6}
If $S$ is a multiplicative subset of $A$, the spectrum $\Spec(S^{-1}A)$ is canonically identified (with its topology) with the subspace of $X=\Spec(A)$ consisting of the $x$ such that $\mathfrak{j}_x\cap S=\emp$.
\end{corollary}

\begin{proof}
\label{proof-1.1.2.6}
We know by \sref[0]{0.1.2.6} that the prime ideals of $S^{-1}A$ are the ideals $S^{-1}\mathfrak{j}_x$ such that $\mathfrak{j}_x\cap S=\emp$, and that we have $\mathfrak{j}_x=(i_A^S)^{-1}(S^{-1}\mathfrak{j}_x)$.
It then suffices to apply Corollary~\sref{1.1.2.4} to the $i_A^S$.
\end{proof}

\begin{corollary}[1.2.7]
\label{1.1.2.7}
For ${}^a\vphi(X)$ to be dense in $X'$, it is necessary and sufficient for each element of the kernel $\Ker\vphi$ to be nilpotent.
\end{corollary}

\begin{proof}
\label{proof-1.1.2.7}
Applying Equation~\hyperref[1.1.2.2]{(1.2.2.3)} to the ideal $\mathfrak{a}=(0)$, we have $\widetilde{{}^a\vphi(X)}=V(\Ker\vphi)$, and for $V(\Ker\vphi)=X$ to hold, it is necessary and sufficient for $\Ker\vphi$ to be contained in all the prime ideals of $A'$, or, equivalent, in the nilradical $\rad'$ of $A'$.
\end{proof}

\subsection{Sheaf associated to a module}
\label{subsection:1.1.3}

\begin{env}[1.3.1]
\label{1.1.3.1}
Let $A$ be a commutative ring, $M$ an $A$-module, $f$ an element of $A$, and $S_f$ the multiplicative set consisting of the $f^n$, where $n\geq 0$.
Recall that we set $A_f=S_f^{-1}A$, $M_f=S_f^{-1}M$.
If $S_f'$ is the saturated multiplicative subset of $A$ consisting of the $g\in A$ which divide an element of $S_f$, we know that $A_f$ and $M_f$ are canonically identified with ${S_f'}^{-1}A$ and ${S_f'}^{-1}M$ \sref[0]{0.1.4.3}.
\end{env}

\begin{lemma}[1.3.2]
\label{1.1.3.2}
The following conditions are equivalent:
\begin{enumerate}
  \item[{\rm(a)}] $g\in S_f'$;
  \item[{\rm(b)}] $S_g'\subset S_f'$;
  \item[{\rm(c)}] $f\in\rad(g)$;
  \item[{\rm(d)}] $\rad(f)\subset\rad(g)$;
  \item[{\rm(e)}] $V(g)\subset V(f)$;
  \item[{\rm(f)}] $D(f)\subset D(g)$.
\end{enumerate}
\end{lemma}

\begin{proof}
\label{proof-1.1.3.2}
This follows immediately from the definitions and \sref{1.1.1.5}.
\end{proof}

\begin{env}[1.3.3]
\label{1.1.3.3}
If $D(f)=D(g)$, then Lemma~\sref{1.1.3.2}[b] shows that $M_f=M_g$.
More generally, if $D(f)\supset D(g)$, then $S_f'\subset S_g'$, and we know \sref[0]{0.1.4.1} that there exists a canonical functorial homomorphism
\[
  \rho_{g,f}:M_f\to M_g,
\]
and if $D(f)\supset D(g)\supset D(h)$, we have \sref[0]{0.1.4.4}
\[
  \rho_{h,g}\circ\rho_{g,f}=\rho_{h,f}.
  \tag{1.3.3.1}
\]
\end{env}

\oldpage[I]{85}
When $f$ ranges over the elements of $A\setmin\mathfrak{j}_x$ (for a given $x$ in $X=\Spec(A)$), the sets $S_f'$ constitute an increasing filtered set of subsets of $A-\mathfrak{j}_x$, since for elements $f$ and $g$ of $A\setmin\mathfrak{j}_x$, $S_f'$ and $S_g'$ are contained in $S_{fg}'$;
since the union of the $S_f'$ over $f\in A\setmin\mathfrak{j}_x$ is $A-\mathfrak{j}_x$, we conclude \sref[0]{0.1.4.5} that the $A_x$-module $M_x$ is canonically identified with the \emph{inductive limit} $\varinjlim M_f$, relative to the family of homomorphisms $(\rho_{g,f})$.
We denote by
\[
  \rho_x^f:M_f\to M_x
\]
the canonical homomorphism for $f\in A\setmin\mathfrak{j}_x$ (or, equivalently, $x\in D(f)$).

\begin{definition}[1.3.4]
\label{1.1.3.4}
We define the structure sheaf of the prime spectrum $X=\Spec(A)$ (resp. the sheaf associated to the $A$-module $M$), denoted by $\widetilde{A}$ or $\sh{O}_X$ (resp. $\widetilde{M}$) as the sheaf of rings (resp. the $\widetilde{A}$-module) associated to the presheaf $D(f)\mapsto A_f$ (resp. $D(f)\mapsto M_f$), defined on the basis $\mathfrak{B}$ of $X$ consisting of the $D(f)$ for $f\in A$ (\sref{1.1.1.10}, \sref[0]{0.3.2.1}, and \sref[0]{0.3.5.6}).
\end{definition}

We saw \sref[0]{0.3.2.4} that the stalk $\widetilde{A}_x$ (resp. $\widetilde{M}_x$) \emph{can be identified with the ring $A_x$} (resp. \emph{the $A_x$-module $M_x$});
we denote by
\[
  \theta_f:A_f\to\Gamma(D(f),\widetilde{A})
\]
\[
  \text{(resp. }\theta_f:M_f\to\Gamma(D(f),\widetilde{M})\text{),}
\]
the canonical map, so that, for all $x\in D(f)$ and all $\xi\in M_f$, we have
\[
  (\theta_f(\xi))_x=\rho_x^f(\xi).
  \tag{1.3.4.1}
\]

\begin{proposition}[1.3.5]
\label{1.1.3.5}
$\widetilde{M}$ is an exact functor, covariant in $M$, from the category of $A$-modules to the category of $\widetilde{A}$-modules.
\end{proposition}

\begin{proof}
\label{proof-1.1.3.5}
Indeed, let $M$, $N$ be two $A$-modules, and $u$ a homomorphism $M\to N$;
for each $f\in A$, $u$ corresponds canonically to a homomorphism $u_f$ from the $A_f$-module $M_f$ to the $A_f$-module $N_f$, and the diagram (for $D(g)\subset D(f)$)
\[
  \xymatrix{
    M_f\ar[r]^{u_f}\ar[d]_{\rho_{g,f}} & N_f\ar[d]^{\rho_{g,f}}\\
    M_g\ar[r]^{u_g} & N_g
  }
\]
is commutative \sref{1.1.4.1};
these homomorphisms then define a homomorphism of $\widetilde{A}$-modules $\widetilde{u}:\widetilde{M}\to\widetilde{N}$ \sref[0]{0.3.2.3}.
In addition, for each $x\in X$, $\widetilde{u}_x$ is the inductive limit of the $u_f$ for $x\in D(f)$ ($f\in A$), and as a result \sref[0]{0.1.4.5}, if we canonically identify $\widetilde{M}_x$ and $\widetilde{N}_x$ with $M_x$ and $N_x$ respectively, then $\widetilde{u}_x$ is identified with the homomorphism $u_x$ canonically induced by $u$.
If $P$ is a third $A$-module, $v$ a homomorphism $N\to P$, and $w=v\circ u$, it is immediate that $w_x=v_x\circ u_x$, so $\widetilde{w}=\widetilde{v}\circ\widetilde{u}$.
We have therefore clearly defined a \emph{covariant (in $M$) functor} $\widetilde{M}$, from the category of $A$-modules to that of $\widetilde{A}$-modules.
\emph{This functor is exact}, since, for each $x\in X$, $M_x$ is an exact functor in $M$ \sref[0]{0.1.3.2};
in addition, we have $\Supp(M)=\Supp(\widetilde{M})$, by definition (\sref[0]{0.1.7.1} and \sref[0]{0.3.1.6}).
\end{proof}

\oldpage[I]{86}
\begin{proposition}[1.3.6]
\label{1.1.3.6}
For each $f\in A$, the open subset $D(f)\subset X$ is canonically identified with the prime spectrum $\Spec(A_f)$, and the sheaf $\widetilde{M_f}$ associated to the $A_f$-module $M_f$ is canonically identified with the restriction $\widetilde{M}|D(f)$.
\end{proposition}

\begin{proof}
\label{proof-1.1.3.6}
The first assertion is a particular case of \sref{1.1.2.6}.
In addition, if $g\in A$ is such that $D(g)\subset D(f)$, then $M_g$ is canonically identified with the module of fractions of $M_f$ whose denominators are the powers of the canonical image of $g$ in $A_f$ \sref[0]{0.1.4.6}.
The canonical identification of $\widetilde{M_f}$ with $\widetilde{M}|D(f)$ then follows from the definitions.
\end{proof}

\begin{theorem}[1.3.7]
\label{1.1.3.7}
For each $A$-module $M$ and each $f\in A$, the homomorphism
\[
  \theta_f:M_f\to\Gamma(D(f),\widetilde{M})
\]
is bijective \emph{(in other words, the presheaf $D(f)\mapsto M_f$ is a \emph{sheaf})}.
In particular, $M$ can be identified with $\Gamma(X,\widetilde{M})$ via $\theta_1$.
\end{theorem}

\begin{proof}
\label{proof-1.1.3.7}
We note that, if $M=A$, then $\theta_f$ is a homomorphism of structure rings;
Theorem~\sref{1.1.3.7} then implies that, if we identify the rings $A_f$ and $\Gamma(D(f),\widetilde{A})$ via $\theta_f$, the homomorphism $\theta_f:M_f\to\Gamma(D(f),\widetilde{M})$ is an isomorphism of \emph{modules}.

We show first that $\theta_f$ is \emph{injective}.
Indeed, if $\xi\in M_f$ is such that $\theta_f(\xi)=0$, then, for each prime ideal $\mathfrak{p}$ of $A_f$, there exists $h\not\in\mathfrak{p}$ such that $h\xi=0$;
as the annihilator of $\xi$ is not contained in any prime ideal of $A_f$, each $A_f$ integral, and so $\xi=0$.

It remains to show that $\theta_f$ is \emph{surjective};
we can restrict to the case where $f=1$ (the general case then following by ``localizing'', using \sref{1.1.3.6}).
Now let $s$ be a section of $\widetilde{M}$ over $X$;
according to \sref{1.1.3.4} and \sref{1.1.1.10}[ii], there exists a \emph{finite} cover $(D(f_i))_{i\in I}$ of $X$ ($f_i\in A$) such that, for each $i\in I$, the restriction $s_i=s|D(f_i)$ is of the form $\theta_{f_i}(\xi_i)$, where $\xi_i\in M_{f_i}$.
If $i$, $j$ are indices of $I$, and if the restrictions of $s_i$ and $s_j$ to $D(f_i)\cap D(f_j)=D(f_i f_j)$ are equal, then it follows, by definition of $M$, that
\[
  \label{1.1.3.7.1}
  \rho_{f_i f_j,f_i}(\xi_i)=\rho_{f_i f_j,f_j}(\xi_j).
  \tag{1.3.7.1}
\]
By definition, we can write, for each $i\in I$, $\xi_i=z_i/f_i^{n_i}$, where $z_i\in M$, and. since $I$ is finite, by multiplying each $z_i$ by a power of $f_i$, we can assume that all the $n_i$ are equal to one single $n$.
Then, by definition, \sref{1.1.3.7.1} implies that there exists an integer $m_{ij}\geq 0$ such that $(f_i f_j)^{m_{ij}}(f_j^n z_i-f_i^n z_j)=0$, and we can moreover suppose that the $m_{ij}$ are equal to the one single $m$;
then replacing the $z_i$ by $f_i^m z_i$, it remains to prove the case where $m=0$, in other words, the case where we have
\[
  \label{1.1.3.7.2}
  f_j^n z_i=f_i^n z_j
  \tag{1.3.7.2}
\]
for any $i$, $j$.
We have $D(f_i^n)=D(f_i)$, and since the $D(f_i)$ form a cover of $X$, the ideal generated by the $f_i^n$ is $A$;
in other words, there exist elements $g_i\in A$ such that $\sum_i g_i f_i^n=1$.
Then consider the element $z=\sum_i g_i z_i$ of $M$;
in \sref{1.1.3.7.2}, we have $f_i^n z=\sum_j g_j f_i^n z_j=(\sum_j g_j f_j^n)z_i=z_i$, where, by definition, $\xi_i=z/1$ in $M_{f_i}$.
We conclude
\oldpage[I]{87}
that the $s_i$ are the restrictions to $D(f_i)$ of $\theta_1(z)$, which proves that $s=\theta_1(z)$ and finishes the proof.
\end{proof}

\begin{corollary}[1.3.8]
\label{1.1.3.8}
Let $M$ and $N$ be $A$-modules;
the canonical homomorphism $u\mapsto\widetilde{u}$ from $\Hom_A(M,N)$ to $\Hom_{\widetilde{A}}(\widetilde{M},\widetilde{N})$ is bijective.
In particular, the equations $M=0$ and $\widetilde{M}=0$ are equivalent.
\end{corollary}

\begin{proof}
\label{proof-1.1.3.8}
Consider the canonical homomorphism $v\mapsto\Gamma(v)$ from $\Hom_{\widetilde{A}}(\widetilde{M},\widetilde{N})$ to $\Hom_{\Gamma(\widetilde{A})}(\Gamma(\widetilde{M}),\Gamma(\widetilde{N}))$;
the latter module is canonically identified with $\Hom_A(M,N)$, by Theorem~\sref{1.1.3.7}.
It remains to show that $u\mapsto\widetilde{u}$ and $v\mapsto\Gamma(v)$ are inverses of each other;
it is evident that $\Gamma(\widetilde{u})=u$ by definition of $\widetilde{u}$;
on the other hand, if we let $u=\Gamma(v)$ for $v\in\Hom_{\widetilde{A}}(\widetilde{M},\widetilde{N})$, then the map $w:\Gamma(D(f),\widetilde{M})\to\Gamma(D(f),\widetilde{N})$ canonically induced from $v$ is such that the diagram
\[
  \xymatrix{
    M\ar[r]^u\ar[d]_{\rho_{f,1}} & N\ar[d]^{\rho_{f,1}}\\
    M_f\ar[r]^w & N_f
  }
\]
is commutative;
so we necessarily have that $w=u_f$ for all $f\in A$ \sref[0]{0.1.2.4}, which shows that $\widetilde{\Gamma(v)}=v$.
\end{proof}

\begin{corollary}[1.3.9]
\label{1.1.3.9}
\medskip\noindent
\begin{enumerate}
  \item[{\rm(i)}] Let $u$ be a homomorphism from an $A$-module $M$ to an $A$-module $N$;
    then the sheaves associated to $\Ker u$, $\Im u$, and $\Coker u$, are $\Ker\widetilde{u}$, $\Im\widetilde{u}$, and $\Coker\widetilde{u}$ (respectively).
    In particular, for $\widetilde{u}$ to be injective (resp. surjective, bijective), it is necessary and sufficient for $u$ to be so too.
  \item[{\rm(ii)}] If $M$ is an inductive limit (resp. direct sum) of a family of $A$-modules $(M_\lambda)$, then $\widetilde{M}$ is the inductive limit (resp. direct sum) of the family $(\widetilde{M_\lambda})$, via a canonical isomorphism.
\end{enumerate}
\end{corollary}

\begin{proof}
\label{proof-1.1.3.9}
\medskip\noindent
\begin{enumerate}
  \item[(i)] It suffices to apply the fact that $\widetilde{M}$ is an exact functor in $M$ \sref{1.1.3.5} to the two exact sequences of $A$-modules
    \[
      0\to\Ker u\to M\to\Im u\to 0,
    \]
    \[
      0\to\Im u\to N\to\Coker u\to 0.
    \]
    The second claim then follows from Theorem~\sref{1.1.3.7}.
  \item[(ii)] Let $(M_\lambda,g_{\mu\lambda})$ be an inductive system of $A$-modules, with inductive limit $M$, and let $g_\lambda$ be the canonical homomorphism $M_\lambda\to M$.
    Since we have $\widetilde{g_{\nu\mu}}\circ\widetilde{g_{\mu\lambda}}=\widetilde{g_{\nu\lambda}}$ and $\widetilde{g_\lambda}=\widetilde{g_\mu}\circ\widetilde{g_{\mu\lambda}}$ for $\lambda\leq\mu\leq\nu$, it follows that $(\widetilde{M_\lambda},\widetilde{g_{\mu\lambda}})$ is an inductive system of sheaves on $X$, and if we denote by $h_\lambda$ the canonical homomorphism $\widetilde{M_\lambda}\to\varinjlim\widetilde{M_\lambda}$, then there is a unique homomorphism $v:\varinjlim\widetilde{M_\lambda}\to\widetilde{M}$ such that $v\circ h_\lambda=\widetilde{g_\lambda}$.
    To see that $v$ is bijective, it suffices to check, for each $x\in X$, that $v_x$ is a bijection from $(\varinjlim\widetilde{M_\lambda})_x$ to $\widetilde{M}_x$;
    but $\widetilde{M}_x=M_x$, and
    \[
      (\varinjlim\widetilde{M_\lambda})_x=\varinjlim(\widetilde{M_\lambda})_x
      =\varinjlim(M_\lambda)_x=M_x\quad\sref[0]{0.1.3.3}.
    \]
    Conversely, it follows from the definitions that $(\widetilde{g_\lambda})_x$ and
    $(h_\lambda)$ are both equal to the canonical map from $(M_\lambda)_x$ to $M_x$;
    since $(\widetilde{g_\lambda})_x=v_x\circ(h_\lambda)_x$, $v_x$ is the identity.
\oldpage[I]{88}
    Finally, if $M$ is the direct sum of two $A$-modules $N$ and $P$, it is immediate that $\widetilde{M}=\widetilde{N}\oplus\widetilde{P}$;
    each direct sum being the inductive limit of finite direct sums, the claims of (ii) are thus proved.
\end{enumerate}
\end{proof}

We note that Corollary~\sref{1.1.3.8} proves that the sheaves isomorphic to the associated sheaves of $A$-modules form an \emph{abelian category} (T, I, 1.4).

We also note that Corollary~\sref{1.1.3.9} implies that, if $M$ is an $A$-module \emph{of finite type} (that is to say, there exists a surjective homomorphism $A^n\to M$) then there exists a surjective homomorphism $\widetilde{A^n}\to\widetilde{M}$, or, in other words, the $\widetilde{A}$-module $\widetilde{M}$ is \emph{generated by a finite family of sections over $X$} \sref[0]{0.5.1.1}, and vice versa.

\begin{env}[1.3.10]
\label{1.1.3.10}
If $N$ is a submodule of an $A$-module $M$, the canonical injection $j:N\to M$ gives, by \sref{1.1.3.9}, an injective homomorphism $\widetilde{N}\to\widetilde{M}$, which allows us to canonically identify $\widetilde{N}$ with an \emph{$\widetilde{A}$-submodule} of $\widetilde{M}$;
we will always assume that we have made this identification.
If $N$ and $P$ are submodules of $M$, then we have
\[
\label{eq:1.1.3.10.1}
  (N+P)^\sim=\widetilde{N}+\widetilde{P},
  \tag{1.3.10.1}
\]
\[
\label{eq:1.1.3.10.2}
  (N\cap P)^\sim=\widetilde{N}\cap\widetilde{P},
  \tag{1.3.10.2}
\]
since $N+P$ and $N\cap P$ are the image of the canonical homomorphism
$N\oplus P\to M$ and the kernel of the canonical homomorphism $M\to(M/N)\oplus(M/P)$ (respectively), and it suffices to apply \sref{1.1.3.9}.

We conclude from \sref{1.1.3.10.1} and \sref{1.1.3.10.2} that, if $\widetilde{N}=\widetilde{P}$, then we have $N=P$.
\end{env}

\begin{corollary}[1.3.11]
\label{1.1.3.11}
On the category of sheaves isomorphic to the associated sheaves of $A$-modules, the functor $\Gamma$ is exact.
\end{corollary}

\begin{proof}
\label{proof-1.1.3.11}
Let $\widetilde{M}\xrightarrow{\widetilde{u}}\widetilde{N}\xrightarrow{\widetilde{v}}\widetilde{P}$ be an exact sequence corresponding to two homomorphisms $u:M\to N$ and $v:N\to P$ of $A$-modules.
If $Q=\Im u$ and $R=\Ker v$, we have $\widetilde{Q}=\Im\widetilde{u}=\Ker\widetilde{v}=\widetilde{R}$ (Corollary~\sref{1.1.3.9}), hence $Q=R$.
\end{proof}

\begin{corollary}[1.3.12]
\label{1.1.3.12}
Let $M$ and $N$ be $A$-modules.
\begin{enumerate}
  \item[{\rm(i)}] The sheaf associated to $M\otimes_A N$ is canonically identified with $\widetilde{M}\otimes_{\widetilde{A}}\widetilde{N}$.
  \item[{\rm(ii)}] If, in addition, $M$ admits a finite presentation, then the sheaf associated to $\Hom_A(M,N)$ is canonically identified with $\shHom_{\widetilde{A}}(\widetilde{M},\widetilde{N})$.
\end{enumerate}
\end{corollary}

\begin{proof}
\label{proof-1.1.3.12}
\medskip\noindent
\begin{enumerate}
  \item[(i)] The sheaf $\sh{F}=\widetilde{M}\otimes_{\widetilde{A}}\widetilde{N}$ is associated to the presheaf
    \[
      U\mapsto\sh{F}(U)=\Gamma(U,\widetilde{M})\otimes_{\Gamma(U,\widetilde{A})}\Gamma(U,\widetilde{N}),
    \]
    with $U$ varying over the basis \sref{1.1.1.10}[i] of $X$ consisting of the $D(f)$, where $f\in A$.
    We know that $\sh{F}(D(f))$ is canonically identified with $M_f\otimes_{A_f}N_f$, by \sref{1.1.3.7} and \sref{1.1.3.6}.
    Moreover, we know that the $A_f$-module $M_f\otimes_{A_f}N_f$ is canonically
    isomorphic to $(M\otimes_A N)_f$ \sref[0]{0.1.3.4}, which is itself canonically isomorphic to $\Gamma(D(f),(M\otimes_A N)^\sim)$ (Theorem~\sref{1.1.3.7} and Proposition~\sref{1.1.3.6}).
    In addition, we see immediately that the canonical isomorphisms
    \[
      \sh{F}(D(f))\isoto\Gamma(D(f),(M\otimes_A N)^\sim)
    \]
\oldpage[I]{89}
    thus obtained satisfy the compatibility conditions with respect to the restriction operations \sref[0]{0.1.4.2}, so they define a canonical functorial isomorphism
    \[
      \widetilde{M}\otimes_{\widetilde{A}}\widetilde{N}\isoto(M\otimes_A N)^\sim.
    \]
  \item[(ii)] The sheaf $\sh{G}=\shHom_{\widetilde{A}}(\widetilde{M},\widetilde{N})$ is associated to the presheaf
    \[
      U\mapsto\sh{G}(U)=\Hom_{\widetilde{A}|U}(\widetilde{M}|U,\widetilde{N}|U),
    \]
    with $U$ varying over the basis of $X$ consisting of the $D(f)$.
    We know that $\sh{G}(D(f))$ is canonically identified with $\Hom_{A_f}(M_f,N_f)$ (Proposition~\sref{1.1.3.6} and
    Corollary~\sref{1.1.3.8}), which, according to the hypotheses on $M$, is canonically identified with $(\Hom_A(M,N))_f$ \sref[0]{0.1.3.5}.
    Finally, $(\Hom_A(M,N))_f$ is canonically identified with $\Gamma(D(f),(\Hom_A(M,N))^\sim)$ (Proposition~\sref{1.1.3.6} and Theorem~\sref{1.1.3.7}), and the canonical isomorphisms $\sh{G}(D(f))\isoto\Gamma(D(f),(\Hom_A(M,N))^\sim)$ thus obtained are compatible with the restriction operations \sref[0]{0.1.4.2};
    they thus define a canonical isomorphism $\shHom_{\widetilde{A}}(\widetilde{M},\widetilde{N})\isoto(\Hom_A(M,N))^\sim$.
\end{enumerate}
\end{proof}

\begin{env}[1.3.13]
\label{1.1.3.13}
Now let $B$ be a (commutative) $A$-algebra; this can be understood by saying that $B$ is an $A$-module such that we have some given element $e\in B$ and an $A$-homomorphism $\vphi:B\otimes_A B\to B$, so that \emph{(a)} the diagrams
\[
  \xymatrix{
    B\otimes_A B\otimes_A B\ar[r]^{\vphi\otimes 1}\ar[d]_{1\otimes\vphi} &
    B\otimes_A B\ar[d]^\vphi & &
    B\otimes_A B\ar[rr]^\sigma\ar[rd]_\vphi & &
    B\otimes_A B\ar[dl]^\vphi\\
    B\otimes_A B\ar[r]^\vphi &
    B & & &
    B
  }
\]
($\sigma$ being the canonical symmetry map) are commutative; and \emph{(b)} $\vphi(e\otimes x)=\vphi(x\otimes e)=x$.
By Corollary~\sref{1.1.3.12}, the homomorphism $\widetilde{\vphi}:\widetilde{B}\otimes_{\widetilde{A}}\widetilde{B}\to\widetilde{B}$ of $\widetilde{A}$-modules satisfies the analogous conditions, and so it defines an \emph{$\widetilde{A}$-algebra} structure on $\widetilde{B}$.
In a similar way, the data of a $B$-module $N$ is the same as the data of an $A$-module $N$ and an $A$-homomorphism $\psi:B\otimes_A N\to N$ such that the diagram
\[
  \xymatrix{
    B\otimes_A B\otimes_A N\ar[r]^{\vphi\otimes 1}\ar[d]_{1\otimes\psi} &
    B\otimes_A N\ar[d]^\psi\\
    B\otimes_A N\ar[r]^\psi &
    N
  }
\]
is commutative and $\psi(e\otimes n)=n$;
the homomorphism $\widetilde{\psi}:\widetilde{B}\otimes_{\widetilde{A}}\widetilde{N}\to\widetilde{N}$ satisfies the analogous condition, and so defines a \emph{$\widetilde{B}$-module} structure on $\widetilde{N}$.

In a similar way, we see that if $u:B\to B'$ (resp. $v:N\to N'$) is a homomorphism of $A$-algebras (resp. of $B$-modules), then $\widetilde{u}$ (resp. $\widetilde{v}$) is a homomorphism of $\widetilde{A}$-algebras (resp. of $\widetilde{B}$-modules), and $\Ker\widetilde{u}$ is a $\widetilde{B}$-ideal (resp. $\Ker\widetilde{v}$, $\Coker\widetilde{v}$, and $\Im\widetilde{v}$ are $\widetilde{B}$-modules).
If $N$ is a $B$-module, then $\widetilde{N}$ is a $\widetilde{B}$-module of finite type if and only if $N$ is a $B$-module of finite type \sref[0]{0.5.2.3}.

\oldpage[I]{90}
If $M$, $N$ are $B$-modules, then the $\widetilde{B}$-module $\widetilde{M}\otimes_{\widetilde{B}}\widetilde{N}$ is canonically identified with $(M\otimes_B N)^\sim$;
similarly $\shHom_{\widetilde{B}}(\widetilde{M},\widetilde{N})$ is canonically identified with $(\Hom_B(M,N))^\sim$ whenever $M$ admits a finite presentation;
the proofs are similar to those for Corollary~\sref{1.1.3.12}.

If $\mathfrak{J}$ is an ideal of $B$, and $N$ is a $B$-module, then we have $(\mathfrak{J}N)^\sim=\widetilde{\mathfrak{J}}\cdot\widetilde{N}$.

Finally, if $B$ is an $A$-algebra \emph{graded} by the $A$-submodules $B_n$ ($n\in\bb{Z}$), then the $\widetilde{A}$-algebra $\widetilde{B}$, the direct sum of the $\widetilde{A}$-modules $\widetilde{B_n}$ \sref{1.1.3.9}, is graded by these $\widetilde{A}$-submodules, the axiom of graded algebras saying that the image of the homomorphism $B_m\otimes B_n\to B$ is contained in $B_{m+n}$.
Similarly, if $M$ is a $B$-module graded by the submodules $M_n$, then $\widetilde{M}$ is a $\widetilde{B}$-module graded by the $\widetilde{M_n}$.
\end{env}

\begin{env}[1.3.14]
\label{1.1.3.14}
If $B$ is an $A$-algebra, and $M$ a submodule of $B$, then the $\widetilde{A}$-subalgebra of $\widetilde{B}$ generated by $\widetilde{M}$ \sref[0]{0.4.1.3} is the $\widetilde{A}$-subalgebra $\widetilde{C}$, where we denote by $C$ the subalgebra of $B$ generated by $M$.
Indeed, $C$ is the direct sum of the submodules of $B$ which are the images of the homomorphisms $\bigotimes^n M\to B$ ($n\geq 0$), so it suffices to apply \sref{1.1.3.9} and \sref{1.1.3.12}.
\end{env}

\subsection{Quasi-coherent sheaves over a prime spectrum}
\label{subsection:1.1.4}

\begin{theorem}[1.4.1]
\label{1.1.4.1}
Let $X$ be the prime spectrum of a ring $A$, $V$ a quasi-compact open subset of $X$, and $\sh{F}$ an $(\sh{O}_X|V)$-module.
The following four conditions are equivalent.
\begin{enumerate}
  \item[{\rm(a)}] There exists an $A$-module $M$ such that $\sh{F}$ is isomorphic to $\widetilde{M}|V$.
  \item[{\rm(b)}] There exists a finite open cover $(V_i)$ of $V$ by sets of the form $D(f_i)$ ($f_i\in A$) contained in $V$, such that, for each $i$, $\sh{F}|V_i$ is isomorphic to a sheaf of the form $\widetilde{M_i}$, where $M_i$ is an $A_{f_i}$-module.
  \item[{\rm(c)}] The sheaf $\sh{F}$ is quasi-coherent \sref[0]{0.5.1.3}.
  \item[{\rm(d)}] The two following properties are satisfied:
    \begin{enumerate}
      \item[{\rm(d1)}] For each $f\in A$ such that $D(f)\subset V$, and for each section $s\in\Gamma(D(f),\sh{F})$, there exists an integer $n\geq 0$ such that $f^n s$ extends to a section of $\sh{F}$ over $V$.
      \item[{\rm(d2)}] For each $f\in A$ such that $D(f)\subset V$ and for each section $t\in\Gamma(V,\sh{F})$ such that the restriction of $t$ to $D(f)$ is $0$, there exists an integer $n\geq 0$ such that $f^n t=0$.
    \end{enumerate}
\end{enumerate}
\end{theorem}
(In the statement of the conditions (d1) and (d2), we have tacitly identified $A$ and $\Gamma(\widetilde{A})$ using Theorem~\sref{1.1.3.7}).

\begin{proof}
\label{proof-1.1.4.1}
The fact that (a) implies (b) is an immediate consequence of Proposition~\sref{1.1.3.6} and the fact that the $D(f_i)$ form a basis for the topology of $X$ \sref{1.1.1.10}.
As each $A$-module is isomorphic to the cokernel of a homomorphism of the form $A^{(I)}\to A^{(J)}$, \sref{1.1.3.6} implies that each sheaf associated to an $A$-module is quasi-coherent;
so (b) implies (c).
Conversely, if $\sh{F}$ is quasi-coherent, each $x\in V$ has a neighborhood of the form $D(f)\subset V$ such that $\sh{F}|D(f)$ is isomorphic to the cokernel of a homomorphism $\widetilde{A_f}^{(I)}\to\widetilde{A_f}^{(J)}$, so also to the sheaf $\widetilde{N}$ associated to the module $N$, the cokernel of the corresponding homomorphism
$A_f^{(I)}\to A_f^{(J)}$ (Corollaries~\sref{1.1.3.8} and \sref{1.1.3.9});
since $V$ is quasi-compact, it is clear that (c) implies (b).

\oldpage[I]{91}
To prove that (b) implies (d1) and (d2), we first assume that $V=D(g)$ for some $g\in A$, and that $\sh{F}$ is isomorphic to the sheaf $\widetilde{N}$ associated to an $A_g$-module $N$;
by replacing $X$ with $V$ and $A$ with $A_g$ \sref{1.1.3.6}, we can reduce to the case where $g=1$.
Then $\Gamma(D(f),\widetilde{N})$ and $N_f$ are canonically identified with one another (Proposition~\sref{1.1.3.6} and Theorem~\sref{1.1.3.7}), so a section $s\in\Gamma(D(f),\widetilde{N})$ is identified with an element of the form $z/f^n$, where $z\in N$;
the section $f^n s$ is identified with the element $z/1$ of $N_f$ and, as a result, is the restriction to $D(f)$ of the section of $\widetilde{N}$ over $X$ that is identified with the element $z\in N$;
hence (d1) in this case.
Similarly, $t\in\Gamma(X,\widetilde{N})$ is identified with an element $z'\in N$, the restriction of $t$ to $D(f)$ is identified with the image $z'/1$ of $z'$ in $N_f$, and to say that this image is zero means that there exists some $n\geq 0$ such that $f^n z'=0$ in $N$, or, equivalently, $f^n t=0$.

To finish the proof, that (b) implies (d1) and (d2), it suffices to establish the following lemma.
\begin{lemma}[1.4.1.1]
\label{1.1.4.1.1}
Suppose that $V$ is the finite union of sets of the form $D(g_i)$, and that all of the sheaves $\sh{F}|D(g_i)$ and $\sh{F}|(D(g_i)\cap D(g_j))=\sh{F}|D(g_i g_j)$ satisfy \emph{(d1)} and \emph{(d2)};
then $\sh{F}$ has the following two properties:
\begin{enumerate}
  \item[{\rm(d'1)}] For each $f\in A$ and for each section $s\in\Gamma(D(f)\cap V,\sh{F})$, there exists an integer $n\geq 0$ such that $f^n s$ extends to a section of $\sh{F}$ over $V$.
  \item[{\rm(d'2)}] For each $f\in A$ and for each section $t\in\Gamma(V,\sh{F})$ such that the restriction of $t$ to $D(f)\cap V$ is $0$, there exists an integer $n\geq 0$ such that $f^n t=0$.
\end{enumerate}
\end{lemma}

We first prove (d$'$2): since $D(f)\cap D(g_i)=D(fg_i)$, there exists, for each $i$, an integer $n_i$ such that the restriction of $(fg_i)^{n_i}t$ to $D(g_i)$ is zero:
since the image of $g_i$ in $A_{g_i}$ is invertible, the restriction of $f^{n_i}t$ to $D(g_i)$ is also zero;
taking $n$ to be the largest of the $n_i$, we have proved (d$'$2).

To show (d$'$1), we apply (d1) to the sheaf $\sh{F}|D(g_i)$: there exists an integer $n_i\geq 0$ and a section $s_i'$ of $\sh{F}$ over $D(g_i)$ extending the restriction of $(fg_i)^{n_i}s$ to $D(fg_i)$;
since the image of $g_i$ in $A_{g_i}$ is invertible, there is a section $s_i$ of $\sh{F}$ over $D(g_i)$ such that $s_i'=g_i^{n_i}s_i$, and $s_i$ extends the
restriction of $f^{n_i}s$ to $D(fg_i)$;
in addition we can suppose that all the $n_i$ are equal to a single integer $n$.
By construction, the restriction of $s_i-s_j$ to $D(f)\cap D(g_i)\cap D(g_j)=D(fg_i g_j)$ is zero;
by (d2) applied to the sheaf $\sh{F}|D(g_i g_j)$, there exists an integer $m_{ij}\geq 0$ such that the restriction to $D(g_i g_j)$ of $(fg_i g_j)^{m_{ij}}(s_i-s_j)$ is zero;
since the image of $g_i g_j$ in $A_{g_i g_j}$ is invertible, the restriction of $f^{m_{ij}}(s_i-s_j)$ to $D(g_i g_j)$ is zero.
We can then assume that all the $m_{ij}$ are equal to a single integer $m$, and so there exists a section $s'\in\Gamma(V,\sh{F})$ extending the $f^m s_i$;
as a result, this section extends $f^{n+m}s$, hence we have proved (d$'$1).

It remains to show that (d1) and (d2) imply (a).
We first show that (d1) and (d2) imply that these conditions are satisfied for each sheaf $\sh{F}|D(g)$, where $g\in A$ is such that $D(g)\subset V$.
It is evident for (d1);
on the other hand, if $t\in\Gamma(D(g),\sh{F})$ is such that its restriction to $D(f)\subset D(g)$ is zero, there exists, by (d1), an integer $m\geq 0$ such that $g^m t$
\oldpage[I]{92}
extends to a section $s$ of $\sh{F}$ over $V$;
applying (d2), we see that there exists an integer $n\geq 0$ such that $f^n g^m t=0$, and as the image of $g$ in $A_g$ is invertible, $f^n t=0$.

That being so, since $V$ is quasi-compact, Lemma~\sref{1.1.4.1.1} proves that
the conditions (d$'$1) and (d$'$2) are satisfied.
Consider then the $A$-module $M=\Gamma(V,\sh{F})$, and define a homomorphism of $\widetilde{A}$-modules $u:\widetilde{M}\to j_*(\sh{F})$, where $j$ is the canonical injection $V\to X$.
Since the $D(f)$ form a basis for the topology of $X$, it suffices, for each $f\in A$, to define a homomorphism $u_f:M_f\to\Gamma(D(f),j_*(\sh{F}))=\Gamma(D(f)\cap V,\sh{F})$, with the usual compatibility conditions \sref[0]{0.3.2.5}.
Since the canonical image of $f$ in $A_f$ is invertible, the restriction homomorphism $M=\Gamma(V,\sh{F})\to\Gamma(D(f)\cap V,\sh{F})$ factors as $M\to M_f\xrightarrow{u_f}\Gamma(D(f)\cap V,\sh{F})$ \sref[0]{0.1.2.4}, and the verification of these compatibility conditions for $D(g)\subset D(f)$ is immediate.
This being so, we show that the condition (d$'$1) (resp. (d$'$2)) implies that each of the $u_f$ are surjective (resp. injective), which proves that $u$ is \emph{bijective}, and as a result that $\sh{F}$ is the restriction to $V$ of an $\widetilde{A}$-module isomorphic to $\widetilde{M}$.
If $s\in\Gamma(D(f)\cap V,\sh{F})$, there exists, by (d$'$1), an integer $n\geq 0$ such that $f^n s$ extends to a section $z\in M$;
we then have $u_f(z/f^n)=s$, so $u_f$ is surjective.
Similarly, if $z\in M$ is such that $u_f(z/1)=0$, this means that the restriction to $D(f)\cap V$ of the section $z$ is zero;
according to (d$'$2), there exists an integer $n\geq 0$ such that $f^n z=0$, hence $z/1=0$ in $M_f$, and so $u_f$ is injective.
\end{proof}

\begin{corollary}[1.4.2]
\label{1.1.4.2}
Each quasi-coherent sheaf over a quasi-compact open subset of $X$ is induced by a quasi-coherent sheaf on $X$.
\end{corollary}

\begin{corollary}[1.4.3]
\label{1.1.4.3}
Every quasi-coherent $\sh{O}_X$-algebra over $X=\Spec(A)$ is isomorphic to an $\sh{O}_X$-algebra of the form $\widetilde{B}$, where $B$ is an algebra over $A$;
every quasi-coherent $\widetilde{B}$-module is isomorphic to a $\widetilde{B}$-module of the form $\widetilde{N}$, where $N$ is a $B$-module.
\end{corollary}

\begin{proof}
\label{proof-1.1.4.3}
Indeed, a quasi-coherent $\sh{O}_X$-algebra is a quasi-coherent $\sh{O}_X$-module, and therefore of the form $\widetilde{B}$, where $B$ is an $A$-module;
the fact that $B$ is an $A$-algebra follows from the characterization of the structure of an $\sh{O}_X$-algebra using the homomorphism $\widetilde{B}\otimes_{\widetilde{A}}\widetilde{B}\to\widetilde{B}$ of $\widetilde{A}$-modules, as well as Corollary~\sref{1.1.3.12}.
If $\sh{G}$ is a quasi-coherent $\widetilde{B}$-module, it suffices to show, in a similar way, that it is also a quasi-coherent $\widetilde{A}$-module to conclude the proof;
since the question is local, we can, by restricting to an open subset of $X$ of the form $D(f)$, assume that $\sh{G}$ is the cokernel of a homomorphism $\widetilde{B}^{(I)}\to\widetilde{B}^{(J)}$ of $\widetilde{B}$-modules (and \emph{a fortiori} of $\widetilde{A}$-modules);
the proposition then follows from Corollaries~\sref{1.1.3.8} and \sref{1.1.3.9}.
\end{proof}

\subsection{Coherent sheaves over a prime spectrum}
\label{subsection:1.1.5}

\begin{theorem}[1.5.1]
\label{1.1.5.1}
Let $A$ be a \emph{Noetherian} ring, $X=\Spec(A)$ its prime spectrum, $V$ an open subset of $X$, and $\sh{F}$ an $(\sh{O}_X|V)$-module.
The following conditions are equivalent.
\begin{enumerate}
  \item[{\rm(a)}] $\sh{F}$ is coherent.
  \item[{\rm(b)}] $\sh{F}$ is of finite type and quasi-coherent.
  \item[{\rm(c)}] There exists an $A$-module $M$ of finite type such that $\sh{F}$ is isomorphic to the sheaf $\widetilde{M}|V$.
\end{enumerate}
\end{theorem}

\begin{proof}
\label{proof-1.1.5.1}
\oldpage[I]{93}
(a) trivially implies (b).
To see that (b) implies (c), note that, since $V$ is quasi-compact \sref[0]{0.2.2.3}, we have previously seen that $\sh{F}$ is isomorphic to a sheaf $\widetilde{N}|V$, where $N$ is an $A$-module \sref{1.1.4.1}.
We have $N=\varinjlim M_\lambda$, where $M_\lambda$ run over the set of $A$-submodules of $N$ of finite type, hence \sref{1.1.3.9} $\sh{F}=\widetilde{N}|V=\varinjlim\widetilde{M_\lambda}|V$;
but since $\sh{F}$ is of finite type, and $V$ is quasi-compact, there exists an index $\lambda$ such that $\sh{F}=\widetilde{M_\lambda}|V$ \sref[0]{0.5.2.3}.

Finally, we show that (c) implies (a).
It is clear that $\sh{F}$ is then of finite type (\sref{1.1.3.6} and \sref{1.1.3.9});
in addition, the question being local, we can restrict to the case where $V=D(f)$, $f\in A$.
Since $A_f$ is Noetherian, we see that it suffices to prove that the kernel of a homomorphism $\widetilde{A^n}\to\widetilde{M}$, where $M$ is an $A$-module, is of finite type.
But such a homomorphism is of the form $\widetilde{u}$, where $u$ is a homomorphism $A^n\to M$ \sref{1.1.3.8}, and if $P=\Ker u$ then we have $\widetilde{P}=\Ker\widetilde{u}$ \sref{1.1.3.9}.
Since $A$ is Noetherian, $P$ is of finite type, which finishes the proof.
\end{proof}

\begin{corollary}[1.5.2]
\label{1.1.5.2}
Under the hypotheses of \sref{1.1.5.1}, the sheaf $\sh{O}_X$ is a quasi-coherent sheaf of rings.
\end{corollary}

\begin{corollary}[1.5.3]
\label{1.1.5.3}
Under the hypotheses of \sref{1.1.5.1}, every coherent sheaf over an open subset of $X$ is induced by a coherent sheaf on $X$.
\end{corollary}

\begin{corollary}[1.5.4]
\label{1.1.5.4}
Under the hypotheses of \sref{1.1.5.1}, every quasi-coherent $\sh{O}_X$-module $\sh{F}$ is the inductive limit of the coherent $\sh{O}_X$-submodules of $\sh{F}$.
\end{corollary}

\begin{proof}
\label{proof-1.1.5.4}
Indeed, $\sh{F}=\widetilde{M}$, where $M$ is an $A$-module, and $M$ is the inductive limit of its submodules of finite type;
we conclude the proof by appealing to \sref{1.1.3.9} and \sref{1.1.5.1}.
\end{proof}

\subsection{Functorial properties of quasi-coherent sheaves over a prime spectrum}
\label{subsection:1.1.6}

\begin{env}[1.6.1]
\label{1.1.6.1}
Let $A$, $A'$ be rings,
\[
  \vphi:A'\to A
\]
a homomorphism, and
\[
  {}^a\vphi:X=\Spec(A)\to X'=\Spec(A')
\]
the continuous map associated to $\vphi$ \sref{1.1.2.1}.
We will define a \emph{canonical homomorphism}
\[
  \widetilde{\vphi}:\sh{O}_{X'}\to{}^a\vphi_*(\sh{O}_X)
\]
of sheaves of rings.
For each $f'\in A'$, we put $f=\vphi(f')$;
we have ${}^a\vphi^{-1}(D(f'))=D(f)$ \sref{1.1.2.2.2}.
The rings $\Gamma(D(f'),\widetilde{A'})$ and $\Gamma(D(f),\widetilde{A})$ are identified  with $A_{f'}'$ and $A_f$ (respectively) (\sref{1.1.3.6} and \sref{1.1.3.7}). The homomorphism $\vphi$ canonically defines a homomorphism $\vphi_{f'}:A_{f'}'\to A_f$ \sref[0]{0.1.5.1}, in other words, we have a homomorphism of rings
\[
  \Gamma(D(f),\widetilde{A'})\to\Gamma({}^a\vphi^{-1}(D(f')),\widetilde{A})
  =\Gamma(D(f'),{}^a\vphi_*(\widetilde{A})).
\]
\oldpage[I]{94}
In addition, these homomorphisms satisfy the usual compatibility conditions: for $D(f')\supset D(g')$, the diagram
\[
  \xymatrix{
    \Gamma(D(f'),\widetilde{A'})\ar[r]\ar[d] &
    \Gamma(D(f'),{}^a\vphi_*(\widetilde{A}))\ar[d]\\
    \Gamma(D(g'),\widetilde{A'})\ar[r] &
    \Gamma(D(g'),{}^a\vphi_*(\widetilde{A})
  }
\]
is commutative \sref[0]{0.1.5.1};
we have thus defined a homomorphism of $\sh{O}_{X'}$-algebras, as the $D(f')$ form a basis for the topology of $X'$ \sref[0]{0.3.2.3}.
The pair $\Phi=({}^a\vphi,\widetilde{\vphi})$ is thus a \emph{morphism} of ringed spaces
\[
  \Phi:(X,\sh{O}_X)\to(X',\sh{O}_{X'}),
\]
\sref[0]{0.4.1.1}.

We also note that, if we put $x'={}^a\vphi(x)$, then the homomorphism $\widetilde{\vphi}_x^\sharp$ \sref[0]{0.3.7.1} is exactly the homomorphism
\[
  \vphi_x:A_{x'}'\to A_x
\]
canonically induced by $\vphi:A'\to A$ \sref[0]{0.1.5.1}.
Indeed, each $z'\in A_{x'}'$ can be written as $g'/f'$, where $f'$, $g'$ are in $A'$ and $f'\not\in\mathfrak{j}_{x'}$;
$D(f')$ is then a neighborhood of $x'$ in $X'$, and the homomorphism $\Gamma(D(f'),\widetilde{A'})\to\Gamma({}^a\vphi^{-1}(D(f')),\widetilde{A})$ induced by $\widetilde{\vphi}$ is exactly $\vphi_{f'}$;
by considering the section $s'\in\Gamma(D(f'),\widetilde{A'})$ corresponding to $g'/f'\in A_{f'}'$, we obtain $\widetilde{\vphi}_x^\sharp(z')=\vphi(g')/\vphi(f')$ in $A_x$.
\end{env}

\begin{example}[1.6.2]
\label{1.1.6.2}
Let $S$ be a multiplicative subset of $A$, and $\vphi$ the canonical homomorphism $A\to S^{-1}A$;
we have already seen \sref{1.1.2.6} that ${}^a\vphi$ is a \emph{homeomorphism} from $Y=\Spec(S^{-1}A)$ to the subspace of $X=\Spec(A)$ consisting of the $x$ such that $\mathfrak{j}_x\cap S=\emp$.
In addition, for each $x$ in this subspace, which is thus of the form ${}^a\vphi(y)$ with $y\in Y$, the homomorphism $\widetilde{\vphi}_y^\sharp:\sh{O}_x\to\sh{O}_y$ is
\emph{bijective} \sref[0]{0.1.2.6};
in other words, $\sh{O}_Y$ is identified with the sheaf on $Y$ induced by $\sh{O}_X$.
\end{example}

\begin{proposition}[1.6.3]
\label{1.1.6.3}
For every $A$-module $M$, there exists a canonical functorial isomorphism from the $\sh{O}_{X'}$-module $(M_{[\vphi]})^\sim$ to the direct image $\Phi_*(\widetilde{M})$.
\end{proposition}

\begin{proof}
\label{proof-1.1.6.3}
For purposes of abbreviation, we write $M'=M_{[\vphi]}$, and for each $f'\in A'$, we put $f=\vphi(f')$.
The modules of sections $\Gamma(D(f'),\widetilde{M'})$ and $\Gamma(D(f),\widetilde{M})$ are identified, respectively, with the modules $M_{f'}'$ and $M_f$ (over $A_{f'}'$ and $A_f$, respectively);
in addition, the $A_{f'}'$-module $(M_f)_{[\vphi_{f'}]}$ is canonically isomorphic to $M_{f'}'$ \sref[0]{0.1.5.2}.
We thus have a functorial isomorphism of $\Gamma(D(f'),\widetilde{A'})$-modules: $\Gamma(D(f'),\widetilde{M'}) \isoto\Gamma({}^a\vphi^{-1}(D(f')),\widetilde{M})_{[\vphi_{f'}]}$ and these isomorphisms satisfy the usual compatibility conditions with the restrictions \sref[0]{0.1.5.6}, thus defining the desired functorial isomorphism.
We note that, in a precise way, if $u:M_1\to M_2$ is a homomorphism of $A$-modules, it can be considered as a homomorphism $(M_1)_{[\vphi]}\to(M_2)_{[\vphi]}$ of $A'$-modules;
if we denote this homomorphism by $u_{[\vphi]}$, then $\Phi_*(\widetilde{u})$ is identified with $(u_{[\vphi]})^\sim$.
\end{proof}

This proof also shows that, for each \emph{$A$-algebra $B$}, the canonical functorial isomorphism
\oldpage[I]{95}
$(B_{[\vphi]})^\sim\isoto\Phi_*(\widetilde{B})$ is an isomorphism of \emph{$\sh{O}_{X'}$-algebras};
if $M$ is a $B$-module, the canonical functorial isomorphism $(M_{[\vphi]})^\sim\isoto\Phi_*(\widetilde{M})$ is an isomorphism of $\Phi_*(\widetilde{B})$-modules.

\begin{corollary}[1.6.4]
\label{1.1.6.4}
The direct image functor $\Phi_*$ is exact on the category of quasi-coherent $\sh{O}_X$-modules.
\end{corollary}

\begin{proof}
\label{proof-1.1.6.4}
Indeed, it is clear that $M_{[\vphi]}$ is an exact functor in $M$ and $\widetilde{M'}$ is an
exact functor in $M'$ \sref{1.1.3.5}.
\end{proof}

\begin{proposition}[1.6.5]
\label{1.1.6.5}
Let $N'$ be an $A'$-module, and $N$ the $A$-module $N'\otimes_{A'}A_{[\vphi]}$;
then there exists a canonical functorial isomorphism from the $\sh{O}_X$-module $\Phi^*(\widetilde{N'})$ to $\widetilde{N}$.
\end{proposition}

\begin{proof}
\label{proof-1.1.6.5}
We first remark that $j:z'\mapsto z'\otimes 1$ is an $A'$-homomorphism from $N'$ to $N_{[\vphi]}$: indeed, by definition, for $f'\in A'$, we have $(f' z')\otimes 1=z'\otimes\vphi(f')=\vphi(f')(z'\otimes 1)$.
We have \sref{1.1.3.8} a homomorphism $\widetilde{j}:\widetilde{N'}\to(N_{[\vphi]})^\sim$ of $\sh{O}_{X'}$-modules, and, thanks to \sref{1.1.6.3}, we can consider $\widetilde{j}$ as mapping $\widetilde{N'}$ to $\Phi_*(\widetilde{N})$.
There canonically corresponds to this homomorphism $\widetilde{j}$ a homomorphism $h=\widetilde{j}^\sharp$ from $\Phi^*(\widetilde{N'})$ to $\widetilde{N}$ \sref[0]{0.4.4.3};
we will see that, for each stalk, $h_x$ is \emph{bijective}.
Put $x'={}^a\vphi(x)$ and let $f'\in A'$ be such that $x'\in D(f')$;
let $f=\vphi(f')$.
The ring $\Gamma(D(f),\widetilde{A})$ is identified with $A_f$, the modules
$\Gamma(D(f),\widetilde{N})$ and $\Gamma(D(f'),\widetilde{N'})$ with $N_f$ and $N_{f'}'$
(respectively);
let $s\in\Gamma(D(f'),\widetilde{N'})$, identified with $n'/{f'}^p$ ($n'\in N'$), and $s$ be its image under $\widetilde{j}$ in $\Gamma(D(f),\widetilde{N})$;
$s$ is identified with $(n'\otimes 1)/f^p$.
On the other hand, let $t\in\Gamma(D(f),\widetilde{A})$, identified with $g/f^q$ ($g\in A$);
then, by definition, we have $h_x(s_x'\otimes t_x)=t_x\cdot s_x$ \sref[0]{0.4.4.3}.
But we can canonically identify $N_f$ with $N_{f'}'\otimes_{A_{f'}'}(A_f)_{[\vphi_{f'}]}$ \sref[0]{0.1.5.4};
$s$ then corresponds to the element $(n'/{f'}^p)\otimes 1$, and the section $y\mapsto t_y\cdot s_y$ with $(n'/{f'}^p)\otimes(g/f^q)$.
The compatibility diagram of \sref[0]{0.1.5.6} show that $h_x$ is exactly the canonical isomorphism
\[
  \label{1.1.6.5.1}
  N_{x'}'\otimes_{A_{x'}'}(A_x)_{[\vphi_{x'}]}\isoto N_x=(N'\otimes_{A'}A_{[\vphi]})_x.
  \tag{1.6.5.1}
\]

In addition, let $v:N_1'\to N_2'$ be a homomorphism of $A'$-modules;
since $\widetilde{v}_{x'}=v_{x'}$ for each $x'\in X'$, it follows immediately from the above that $\Phi^*(\widetilde{v})$ is canonically identified with $(v\otimes 1)^\sim$, which finishes the proof of \sref{1.1.6.5}.
\end{proof}

If $B'$ is an $A'$-algebra, the canonical isomorphism from $\Phi^*(\widetilde{B'})$ to $(B'\otimes_{A'}A_{[\vphi]})^\sim$ is an isomorphism of $\sh{O}_X$-algebras;
if, in addition, $N'$ is a $B'$-module, then the canonical isomorphism from $\Phi^*(\widetilde{N'})$ to $(N'\otimes_{A'}A_{[\vphi]})^\sim$ is an isomorphism of $\Phi^*(\widetilde{B'})$-modules.

\begin{corollary}[1.6.6]
\label{1.1.6.6}
The sections of $\Phi^*(\widetilde{N'})$, the canonical images of the sections $s'$, where $s'$ varies over the $A'$-module $\Gamma(\widetilde{N'})$, generate the $A$-module $\Gamma(\Phi^*(N'))$.
\end{corollary}

\begin{proof}
\label{proof-1.1.6.6}
Indeed, these images are identified with the elements $z'\otimes 1$ of $N$, when we identify $N'$ and $N$ with $\Gamma(\widetilde{N'})$ and $\Gamma(\widetilde{N})$ (respectively) \sref{1.1.3.7}, and $z'$ varies over $N'$.
\end{proof}

\begin{env}[1.6.7]
\label{1.1.6.7}
In the proof of \sref{1.1.6.5}, we had proved in passing that the canonical map (\textbf{0},~4.4.3.2) $\rho:\widetilde{N'}\to\Phi_*(\Phi^*(\widetilde{N'}))$ is exactly the homomorphism $\widetilde{j}$,
\oldpage[I]{96}
where $j:N'\to N'\otimes_{A'}A_{[\vphi]}$ is the homomorphism $z'\mapsto z'\otimes 1$.
Similarly, the canonical map (\textbf{0},~4.4.3.3) $\sigma:\Phi^*(\Phi_*(\widetilde{M}))\to\widetilde{M}$ is exactly $\widetilde{p}$, where $p:M_{[\vphi]}\otimes_{A'}A_{[\vphi]}\to M$ is the canonical homomorphism, which sends each tensor product $z\otimes a$ ($z\in M$, $a\in A$) to $a\cdot z$;
this follows immediately from the definitions (\sref[0]{0.3.7.1}, \sref[0]{0.4.4.3}, and \sref{1.1.3.7}).

We conclude (\sref[0]{0.4.4.3} and (\textbf{0},~3.5.4.4)) that if $v:N'\to M_{[\vphi]}$ is an $A'$-homomorphism, then $\widetilde{v}^\sharp=(v\otimes 1)^\sim$.
\end{env}

\begin{env}[1.6.8]
\label{1.1.6.8}
Let $N_1'$ and $N_2'$ be $A'$-modules, and assume $N_1'$ admits a \emph{finite presentation};
it then follows from \sref{1.1.6.7} and \sref{1.1.3.12}[ii] that the
canonical homomorphism \sref[0]{0.4.4.6}
\[
  \Phi^*(\shHom_{\widetilde{A'}}(\widetilde{N_1'},\widetilde{N_2'}))
  \to\shHom_{\widetilde{A}}(\Phi^*(\widetilde{N_1'}),\Phi^*(\widetilde{N_2'}))
\]
is exactly $\widetilde{\gamma}$, where $\gamma$ denotes the canonical homomorphism of $A$-modules $\Hom_{A'}(N_1',N_2')\otimes_{A'}A\to\Hom_A(N_1'\otimes_{A'}A,N_2'\otimes_{A'}A)$.
\end{env}

\begin{env}[1.6.9]
\label{1.1.6.9}
Let $\mathfrak{J}'$ be an ideal of $A'$, and $M$ an $A$-module;
since, by definition, $\widetilde{\mathfrak{J}'}\widetilde{M}$ is the image of the canonical homomorphism $\Phi^*(\widetilde{\mathfrak{J}'})\otimes_{\widetilde{A}}\widetilde{M}\to\widetilde{M}$, it
follows from Proposition~\sref{1.1.6.5} and Corollary~\sref{1.1.3.12}[i] that
$\widetilde{\mathfrak{J}'}\widetilde{M}$ canonically identifies with $(\mathfrak{J}' M)^\sim$;
in particular, $\Phi^*(\widetilde{\mathfrak{J}'})\widetilde{A}$ is identified with $(\mathfrak{J}' A)^\sim$, and, taking the right exactness of the functor $\Phi^*$ into account,
the $\widetilde{A}$-algebra $\Phi^*((A'/\mathfrak{J}')^\sim)$ is identified with $(A/\mathfrak{J}' A)^\sim$.
\end{env}

\begin{env}[1.6.10]
\label{1.1.6.10}
Let $A''$ be a third ring, $\vphi'$ a homomorphism $A''\to A'$, and write $\vphi''=\vphi\circ\vphi'$.
It follows immediately from the definitions that ${}^a\vphi''=({}^a\vphi')\circ({}^a\vphi)$, and $\widetilde{\vphi''}=\widetilde{\vphi}\circ\widetilde{\vphi'}$ \sref[0]{0.1.5.7}. We conclude that $\Phi''=\Phi'\circ\Phi$;
in other words, $(\Spec(A),\widetilde{A})$ is a \emph{functor} from the category of rings to that of ringed spaces.
\end{env}

\subsection{Characterization of morphisms of affine schemes}
\label{subsection:1.1.7}

\begin{definition}[1.7.1]
\label{1.1.7.1}
We say that a ringed space $(X,\sh{O}_X)$ is an \emph{affine scheme} if it is isomorphic to a ringed space of the form $(\Spec(A),\widetilde{A})$, where $A$ is a ring;
we then say that $\Gamma(X,\sh{O}_X)$, which is canonically identified with the ring $A$ \sref{1.1.3.7}, is the ring of the affine scheme $(X,\sh{O}_X)$, and we denote it by $A(X)$ when there is no chance of confusion.
\end{definition}

By abuse of language, when we speak of an \emph{affine scheme $\Spec(A)$}; it will always be the ringed space $(\Spec(A),\widetilde{A})$.
\begin{env}[1.7.2]
\label{1.1.7.2}
Let $A$ and $B$ be rings, and $(X,\sh{O}_X)$ and $(Y,\sh{O}_Y)$ the affine schemes corresponding to the prime spectra $X=\Spec(A)$, $Y=\Spec(B)$.
We have seen \sref{1.1.6.1} that each ring homomorphism $\vphi:B\to A$ corresponds to a morphism $\Phi=({}^a\vphi,\widetilde{\vphi})=\Spec(\vphi):(X,\sh{O}_X)\to(Y,\sh{O}_Y)$.
We note that $\vphi$ is entirely determined by $\Phi$, since we have, by definition, $\vphi=\Gamma(\widetilde{\vphi}):\Gamma(\widetilde{B})\to\Gamma({}^a\vphi_*(\widetilde{A})=\Gamma(\widetilde{A})$.
\end{env}

\begin{theorem}[1.7.3]
\label{1.1.7.3}
Let $(X,\sh{O}_X)$  $(Y,\sh{O}_Y)$ be affine schemes.
For a morphism of ringed spaces $(\psi,\theta):(X,\sh{O}_X)\to(Y,\sh{O}_Y)$ to be of the form $({}^a\vphi,\widetilde{\vphi})$, where $\vphi$ is a homomorphism of rings $A(Y)\to A(X)$, it is necessary and sufficient that, for each $x\in X$, $\theta_x^\sharp$ is a local homomorphism: $\sh{O}_{\psi(x)}\to\sh{O}_x$.
\end{theorem}

\begin{proof}
\label{proof-1.1.7.3}
\oldpage[I]{97}
Let $A=A(X)$, $B=A(Y)$.
The condition is necessary, since we saw \sref{1.1.6.1} that $\widetilde{\vphi}_x^\sharp$ is the homomorphism from $B_{{}^a\vphi(x)}$ to $A_x$ canonically induced by $\vphi$, and, by definition, of ${}^a\vphi(x)=\vphi^{-1}(\mathfrak{j}_x)$, this homomorphism is local.

We now prove that the condition is sufficient.
By definition, $\theta$ is a homomorphism $\sh{O}_Y\to\psi_*(\sh{O}_X)$, and we canonically obtain a ring homomorphism
\[
  \vphi=\Gamma(\theta):B=\Gamma(Y,\sh{O}_Y)
  \to\Gamma(Y,\psi_*(\sh{O}_X))=\Gamma(X,\sh{O}_X)=A.
\]

The hypotheses on $\theta_x^\sharp$ mean that this homomorphism induces, by passing to quotients, a monomorphism $\theta^x$ from the residue field $\kres(\psi(x))$ to the residue field $\kres(x)$, such that, for each section $f\in\Gamma(Y,\sh{O}_Y)=B$, we have $\theta^x(f(\psi(x)))=\vphi(f)(x)$.
The relation $f(\psi(x))=0$ is therefore equivalent to $\vphi(f)(x)=0$, which means that $\mathfrak{j}_{\psi(x)}=\mathfrak{j}_{{}^a\vphi(x)}$, and we now write $\psi(x)={}^a\vphi(x)$ for each $x\in X$, or $\psi={}^a\vphi$.
We also know that the diagram
\[
  \xymatrix{
    B=\Gamma(Y,\sh{O}_Y)\ar[r]^\vphi\ar[d] &
    \Gamma(X,\sh{O}_X)=A\ar[d]\\
    B_{\psi(x)}\ar[r]^{\theta_x^\sharp} &
    A_x
  }
\]
is commutative \sref[0]{0.3.7.2}, which means that $\theta_x^\sharp$ is equal to the homomorphism $\vphi_x:B_{\psi(x)}\to A_x$ canonically induced by $\vphi$ \sref[0]{0.1.5.1}.
As the data of the $\theta_x^\sharp$ completely characterize $\theta^\sharp$, and as a result $\theta$ \sref[0]{0.3.7.1}, we conclude that we have $\theta=\widetilde{\vphi}$, by the definition of $\widetilde{\vphi}$ \sref{1.1.6.1}.
\end{proof}

We say that a morphism $(\psi,\theta)$ of ringed spaces satisfying the condition of \sref{1.1.7.3} is a \emph{morphism of affine schemes}.

\begin{corollary}[1.7.4]
\label{1.1.7.4}
If $(X,\sh{O}_X)$ and $(Y,\sh{O}_Y)$ are affine schemes, there exists a canonical isomorphism from the set of morphisms of affine schemes $\Hom((X,\sh{O}_X),(Y,\sh{O}_Y))$ to the set of ring homomorphisms from $B$ to $A$, where $A=\Gamma(\sh{O}_X)$ and $B=\Gamma(\sh{O}_Y)$.
\end{corollary}

Furthermore, we can say that the functors $(\Spec(A),\widetilde{A})$ in $A$ and $\Gamma(X,\sh{O}_X)$ in $(X,\sh{O}_X)$ define an \emph{equivalence} between the category of commutative rings and the opposite category of affine schemes (T, I, 1.2).

\begin{corollary}[1.7.5]
\label{1.1.7.5}
If $\vphi:B\to A$ is surjective, then the corresponding morphism $({}^a\vphi,\widetilde{\vphi})$ is a monomorphism of ringed spaces \emph{(cf.~\sref{1.4.1.7})}.
\end{corollary}

\begin{proof}
\label{proof-1.1.7.5}
Indeed, we know that ${}^a\vphi$ is injective \sref{1.1.2.5}, and, since $\vphi$ is
surjective, for each $x\in X$, $\vphi_x^\sharp:B_{{}^a\vphi(x)}\to A_x$, which is induced by $\vphi$ by passing to rings of fractions, is also surjective \sref[0]{0.1.5.1};
hence the conclusion \sref[0]{0.4.1.1}.
\end{proof}

\subsection{Morphisms from locally ringed spaces to affine schemes}
\label{subsection:1.1.8}

\oldpage[II]{217}
Due to a remark by J.~Tate, the statements of Theorem~\sref{1.1.7.3} and Proposition~\sref{1.2.2.4} can be generalized as follows:\footnote{\emph{[Trans.] The following section (I.1.8) was added in the errata of EGA~II, hence the temporary change in page numbers, which refer to EGA~II.}}

\begin{proposition}[1.8.1]
\label{1.1.8.1}
Let $(S,\sh{O}_S)$ be an affine scheme, and $(X,\sh{O}_X)$ a locally ringed space.
Then there is a canonical bijection from the set of ring homomorphisms
\oldpage[II]{218}
$\Gamma(S,\sh{O}_S)\to\Gamma(X,\sh{O}_X)$ to the set of morphisms of ringed spaces $(\psi,\theta):(X,\sh{O}_X)\to(S,\sh{O}_S)$ such that, for each $x\in X$, $\theta_x^\sharp$ is a local homomorphism $\sh{O}_{\psi(x)}\to\sh{O}_x$.
\end{proposition}

\begin{proof}
\label{proof-1.1.8.1}
We note first that if $(X,\sh{O}_X)$ and $(S,\sh{O}_S)$ are any two ringed spaces, then a morphism $(\psi,\theta)$ from $(X,\sh{O}_X)$ to $(S,\sh{O}_S)$ canonically defines a ring homomorphism $\Gamma(\theta):\Gamma(S,\sh{O}_S)\to\Gamma(X,\sh{O}_X)$, hence a first map
\[
  \label{1.1.8.1.1}
  \rho:\Hom((X,\sh{O}_X),(S,\sh{O}_S))\to\Hom(\Gamma(S,\sh{O}_S),\Gamma(X,\sh{O}_X)).
  \tag{1.8.1.1}
\]
Conversely, under the stated hypotheses, we set $A=\Gamma(S,\sh{O}_S)$, and consider a ring homomorphism $\vphi:A\to\Gamma(X,\sh{O}_X)$.
For each $x\in X$, it is clear that the set of the $f\in A$ such that $\vphi(f)(x)=0$ is a \emph{prime ideal} of $A$, since $\sh{O}_x/\mathfrak{m}_x=\kres(x)$ is a field;
it is therefore an element of $S=\Spec(A)$, which we denote by ${}^a\vphi(x)$.
In addition, for each $f\in A$, we have, by definition \sref[0]{0.5.5.2}, that ${}^a\vphi(D(f))=X_f$, which proves that ${}^a\vphi$ is a \emph{continuous map} $X\to S$.
We then define a homomorphism
\[
  \widetilde{\vphi}:\sh{O}_S\to{}^a\vphi_*(\sh{O}_X)
\]
of $\sh{O}_S$-modules;
for each $f\in A$, we have $\Gamma(D(f),\sh{O}_S)=A_f$ \sref{1.1.3.6};
for each $s\in A$, we associate to $s/f\in A_f$ the element $(\vphi(s)|X_f)(\vphi(f)|X_f)^{-1}$ of $\Gamma(X_f,\sh{O}_X)=\Gamma(D(f),{}^a\vphi(\sh{O}_X))$, and we immediately see (by passing from $D(f)$ to $D(fg)$) that this is a well-defined homomorphism of $\sh{O}_S$-modules, hence a morphism $({}^a\vphi,\widetilde{\vphi})$ of ringed spaces.
In addition, with the same notation, and setting $y={}^a\vphi(x)$ for purposes of simplification, we immediately see \sref[0]{0.3.7.1} that we have $\widetilde{\vphi}_x^\sharp(s_y/f_y)=(\vphi(s)_x)(\vphi(f)_x)^{-1}$;
since the relation $s_y\in\mathfrak{m}_y$ is, by definition, equivalent to $\vphi(s)_x\in\mathfrak{m}_x$, we see that $\widetilde{\vphi}_x^\sharp$ is a \emph{local} homomorphism $\sh{O}_y\to\sh{O}_x$, and we have thus defined a second map $\sigma:\Hom(\Gamma(S,\sh{O}_S),\Gamma(X,\sh{O}_X))\to\mathfrak{L}$, where $\mathfrak{L}$ is the set of the morphisms $(\psi,\theta):(X,\sh{O}_X)\to(S,\sh{O}_S)$ such that $\theta_x^\sharp$ is local for each $x\in X$.
It remains to prove that $\sigma$ and $\rho$ (restricted to $\mathfrak{L}$) are inverses of each other;
the definition of $\widetilde{\vphi}$ immediately shows that $\Gamma(\widetilde{\vphi})=\vphi$, and, as a result, that $\rho\circ\sigma$ is the identity.
To see that $\sigma\circ\rho$ is the identity, start with a morphism $(\psi,\theta)\in\mathfrak{L}$ and let $\vphi=\Gamma(\theta)$;
the hypotheses on $\theta_x^\sharp$ mean that this morphism induces, by passing to quotients, a monomorphism $\theta^x:\kres(\psi(x))\to\kres(x)$ such that for each section $f\in A=\Gamma(S,\sh{O}_S)$, we have $\theta^x(f(\psi(x)))=\vphi(f)(x)$;
the equation $f(\vphi(x))=0$ is therefore equivalent to $\vphi(f)(x)=0$, which proves that ${}^a\vphi=\psi$.
On the other hand, the definitions imply that the diagram
\[
  \xymatrix{
    A\ar[r]^\vphi\ar[d] &
    \Gamma(X,\sh{O}_X)\ar[d]\\
    A_{\psi(x)}\ar[r]^{\theta_x^\sharp} &
    \sh{O}_x
  }
\]
is commutative, and it is the same for the analogous diagram where $\theta_x^\sharp$ is replaced by $\widetilde{\vphi}_x^\sharp$, hence $\widetilde{\vphi}_x^\sharp=\theta_x^\sharp$ \sref[0]{0.1.2.4}, and, as a result, $\widetilde{\vphi}=\theta$.
\end{proof}

\begin{env}[1.8.2]
\label{1.1.8.2}
When $(X,\sh{O}_X)$ and $(Y,\sh{O}_Y)$ are \emph{locally} ringed spaces, we will consider the morphisms $(\psi,\theta):(X,\sh{O}_X)\to(Y,\sh{O}_Y)$ such that, for each $x\in X$, $\theta_x^\sharp$ is a \emph{local} homomorphism $\sh{O}_{\psi(x)}\to\sh{O}_x$.
Henceforth when we speak
\oldpage[II]{219}
of a \emph{morphism of locally ringed spaces}, it will always be a morphism like the above;
with this definition of morphisms, it is clear that the locally ringed spaces form a \emph{category};
for any two objects $X$ and $Y$ of this category, $\Hom(X,Y)$ thus denotes the set of morphisms of locally ringed spaces from $X$ to $Y$ (the set denoted $\mathfrak{L}$ in \sref{1.1.8.1});
when we consider the set of \emph{morphisms of ringed spaces} from $X$ to $Y$, we will denote it by $\Hom_\text{rs}(X,Y)$ to avoid any confusion.
The map \sref{1.1.8.1.1} is then written as
\[
  \label{1.1.8.2.1}
  \rho:\Hom_\text{rs}(X,Y)\to\Hom(\Gamma(Y,\sh{O}_Y),\Gamma(X,\sh{O}_X))
  \tag{1.8.2.1}
\]
and its restriction
\[
  \label{1.1.8.2.2}
  \rho':\Hom(X,Y)\to\Hom(\Gamma(Y,\sh{O}_Y),\Gamma(X,\sh{O}_X))
  \tag{1.8.2.2}
\]
is a \emph{functorial} map in $X$ and $Y$ on the category of locally ringed spaces.
\end{env}

\begin{corollary}[1.8.3]
\label{1.1.8.3}
Let $(Y,\sh{O}_Y)$ be a locally ringed space.
For $Y$ to be an affine scheme, it is necessary and sufficient that, for each locally ringed space $(X,\sh{O}_X)$, the map \sref{1.1.8.2.2} be bijective.
\end{corollary}

\begin{proof}
\label{proof-1.1.8.3}
Proposition~\sref{1.1.8.1} shows that the condition is necessary.
Conversely, if we suppose that the condition is satisfied, and if we put $A=\Gamma(Y,\sh{O}_Y)$, then it follows from the hypotheses and from \sref{1.1.8.1} that the functors $X\mapsto\Hom(X,Y)$ and $X\mapsto\Hom(X,\Spec(A))$, from the category of locally ringed spaces to that of sets, are \emph{isomorphic}.
We know that this implies the existence of a canonical isomorphism $X\to\Spec(A)$ (cf.~\textbf{0},~8).
\end{proof}

\begin{env}[1.8.4]
\label{1.1.8.4}
Let $S=\Spec(A)$ be an affine scheme;
denote by $(S',A')$ the ringed space whose underlying space is \emph{a point} and the structure sheaf $A'$ is the (necessarily simple) sheaf on $S'$ defined by the ring $A$.
Let $\pi:S\to S'$ be the unique map from $S$ to $S'$;
on the other hand, we note that, for each open subset $U$ of $S$, we have a canonical map $\Gamma(S',A')=\Gamma(S,\sh{O}_S)\to\Gamma(U,\sh{O}_S)$ which thus defines a \emph{$\pi$-morphism} $\iota:A'\to\sh{O}_S$ of sheaves of rings.
We have thus canonically defined a \emph{morphism of ringed spaces $i=(\pi,\iota):(S,\sh{O}_S)\to(S',A')$}.
For each $A$-module $M$, we denote by $M'$ the simple sheaf on $S'$ defined by $M$, which is evidently an $A'$-module.
It is clear that $i_*(\widetilde{M})=M'$ \sref{1.1.3.7}.
\end{env}

\begin{lemma}[1.8.5]
\label{1.1.8.5}
With the notation of \sref{1.1.8.4}, for each $A$-module $M$, the canonical functorial $\sh{O}_S$-homomorphism \sref[0]{4.4.3.3}
\[
\label{eq:1.1.8.5.1}
  i^*(i_*(\widetilde{M}))\to\widetilde{M}
  \tag{1.8.5.1}
\]
is an isomorphism.
\end{lemma}

\begin{proof}
\label{proof-1.1.8.5}
Indeed, the two parts of \sref{1.1.8.5.1} are right exact (the functor $M\mapsto i_*(\widetilde{M})$ evidently being exact) and commute with direct sums;
by considering $M$ as the cokernel of a homomorphism $A^{(I)}\to A^{(J)}$, we can reduce to proving the lemma for the case where $M=A$, and it is evident in this case.
\end{proof}

\begin{corollary}[1.8.6]
\label{1.1.8.6}
Let $(X,\sh{O}_X)$ be a ringed space, and $u:X\to S$ a morphism of ringed spaces.
\oldpage[II]{220}
For each $A$-module $M$, we have (with the notation of \sref{1.1.8.4}) a canonical functorial isomorphism of $\sh{O}_X$-modules
\[
\label{eq:1.1.8.6.1}
  u^*(\widetilde{M})\isoto u^*(i^*(M')).
  \tag{1.8.6.1}
\]
\end{corollary}

\begin{corollary}[1.8.7]
\label{1.1.8.7}
Under the hypotheses of \sref{1.1.8.6}, we have, for each $A$-module $M$ and each $\sh{O}_X$-module $\sh{F}$, a canonical isomorphism, functorial in $M$ and $\sh{F}$,
\[
\label{eq:1.1.8.7.1}
  \Hom_{\sh{O}_S}(\widetilde{M},u_*(\sh{F}))\isoto\Hom_A(M,\Gamma(X,\sh{F})).
  \tag{1.8.7.1}
\]
\end{corollary}

\begin{proof}
\label{proof-1.1.8.7}
We have, according to \sref[0]{0.4.4.3} and Lemma~\sref{1.1.8.5}, a canonical
isomorphism of bifunctors
\[
  \Hom_{\sh{O}_S}(\widetilde{M},u_*(\sh{F}))\isoto\Hom_{A'}(M',i_*(u_*(\sh{F})))
\]
and it is clear that the right hand side is exactly $\Hom_A(M,\Gamma(X,\sh{F}))$.
We note that the canonical homomorphism \sref{1.1.8.7.1} sends each $\sh{O}_S$-homomorphism $h:\widetilde{M}\to u_*(\sh{F})$ (in other words, each $u$-morphism $\widetilde{M}\to\sh{F}$) to the $A$-homomorphism $\Gamma(h):M\to\Gamma(S,u_*(\sh{F}))=\Gamma(X,\sh{F})$.
\end{proof}

\begin{env}[1.8.8]
\label{1.1.8.8}
With the notation of \sref{1.1.8.4}, it is clear \sref[0]{0.4.1.1} that each morphism of ringed spaces $(\psi,\theta):X\to S'$ is equivalent to the data of a ring homomorphism $A\to\Gamma(X,\sh{O}_X)$.
We can thus interpret Proposition~\sref{1.1.8.1} as defining a canonical bijection $\Hom(X,S)\isoto\Hom(X,S')$ (where we understand that the right-hand side is the collection of morphisms of ringed spaces, since in general $A$ is not a local ring).
More generally, if $X$ and $Y$ are locally ringed spaces, and if $(Y',A')$ is the ringed space whose underlying space is a point and whose sheaf of rings $A'$ is the simple sheaf defined by the ring $\Gamma(Y,\sh{O}_Y)$, we can interpret \sref{1.1.8.2.1} as a map
\[
  \label{1.1.8.8.1}
  \rho:\Hom_\text{rs}(X,Y)\to\Hom(X,Y').
  \tag{1.8.8.1}
\]
The result of Corollary~\sref{1.1.8.3} is interpreted by saying that affine schemes are characterized among locally ringed spaces as those for which the restriction of $\rho$ to
$\Hom(X,Y)$:
\[
  \label{1.1.8.8.2}
  \rho':\Hom(X,Y)\to\Hom(X,Y')
  \tag{1.8.8.2}
\]
is \emph{bijective} for \emph{every} locally ringed space $X$.
In the following chapter, we generalize this definition, which allows us to associate to \emph{any} ringed space $Z$ (and not only to a ringed space whose underlying space is a point) a locally ringed space which we will call $\Spec(Z)$;
this will be the starting point for a ``relative'' theory of preschemes over any ringed space, extending the results of Chapter~I.
\end{env}

\begin{env}[1.8.9]
\label{1.1.8.9}
We can consider the pairs $(X,\sh{F})$ consisting of a locally ringed space $X$ and an $\sh{O}_X$-module $\sh{F}$ as forming a category, a \emph{morphism} in this category being a pair $(u,h)$ consisting of a morphism of locally ringed spaces
\oldpage[II]{221}
$u:X\to Y$ and a $u$-morphism $h:\sh{G}\to\sh{F}$ of modules;
these morphisms (for $(X,\sh{F})$ and $(Y,\sh{G})$ fixed) form a set which we denote by $\Hom((X,\sh{F}),(Y,\sh{G}))$;
the map $(u,h)\mapsto(\rho'(u),\Gamma(h))$ is a canonical map
\[
  \label{1.1.8.9.1}
  \Hom((X,\sh{F}),(Y,\sh{G}))\to\Hom((\Gamma(Y,\sh{O}_Y),\Gamma(Y,\sh{G})),(\Gamma(X,\sh{O}_X),\Gamma(X,\sh{F})))
  \tag{1.8.9.1}
\]
\emph{functorial} in $(X,\sh{F})$ and $(Y,\sh{G})$, the right-hand side being the set of di-homomorphisms corresponding to the rings and modules considered \sref[0]{0.1.0.2}.
\end{env}

\begin{corollary}[1.8.10]
\label{1.1.8.10}
Let $Y$ be a locally ringed space, and $\sh{G}$ an $\sh{O}_Y$-module.
For $Y$ to be an affine scheme and $\sh{G}$ to be a quasi-coherent $\sh{O}_Y$-module, it is necessary and sufficient that, for each pair $(X,\sh{F})$ consisting of a locally ringed space $X$ and an $\sh{O}_X$-module $\sh{F}$, the canonical map \sref{1.1.8.9.1} be bijective.
\end{corollary}

We leave the proof, which is modelled on that of \sref{1.1.8.3}, using \sref{1.1.8.1} and \sref{1.1.8.7}, to the reader.

\begin{remark}[1.8.11]
\label{1.1.8.11}
The statements \sref{1.1.7.3}, \sref{1.1.7.4}, and \sref{1.2.2.4} are particular cases of \sref{1.1.8.1}, as well as the definition in \sref{1.1.6.1};
similarly, \sref{1.2.2.5} follows from \sref{1.1.8.7}.
Corollary~\sref{1.1.8.7} also implies \sref{1.1.6.3} (and, as a result, \sref{1.1.6.4}) as a particular case, since if $X$ is an affine scheme and $\Gamma(X,\sh{F})=N$, then the functors $M\mapsto\Hom_{\sh{O}_S}(\widetilde{M},u_*(\widetilde{N}))$ and $M\mapsto\Hom_{\sh{O}_S}(\widetilde{M},(N_{[\vphi]})^\sim)$ (where $\vphi:A\to\Gamma(X,\sh{O}_X)$ corresponds to $u$) are isomorphic, by Corollaries~\sref{1.1.8.7} and \sref{1.1.3.8}.
Finally, \sref{1.1.6.5} (and, as a result, \sref{1.1.6.6}) follow from \sref{1.1.8.6}, and the fact that, for each $f\in A$, the $A_f$-modules $N'\otimes_{A'}A_f$ and $(N'\otimes_{A'}A)_f$ (with the notation of \sref{1.1.6.5}) are canonically isomorphic.
\end{remark}


\section{Preschemes and morphisms of preschemes}
\label{section:preschemes-and-morphisms}

\subsection{Definition of preschemes}
\label{subsection:preschemes-definition}

\begin{env}[2.1.1]
\label{1.2.1.1}
Given a ringed space $(X,\OO_X)$, we say that an open subset $V$ of $X$ is an \emph{affine open} subset if the ringed space $(V,\OO_X|V)$ is an affine scheme \sref{1.1.7.1}.
\end{env}

\begin{defn}[2.1.2]
\label{1.2.1.2}
We define a prescheme to be a ringed space $(X,\OO_X)$ such that every point of $X$ admits an affine open neighborhood.
\end{defn}

\begin{prop}[2.1.3]
\label{1.2.1.3}
\oldpage[I]{98}
If $(X,\OO_X)$ is a prescheme, then its affine open subsets form a basis for the topology of $X$.
\end{prop}

\begin{proof}
\label{proof-1.2.1.3}
If $V$ is an arbitrary open neighborhood of $x\in X$, then there exists by hypothesis an open neighborhood $W$ of $x$ such that $(W,\OO_X|W)$ is an affine scheme;
we write $A$ to mean its ring.
In the space $W$, $V\cap W$ is an open neighborhood of $x$;
so there exists some $f\in A$ such that $D(f)$ is an open neighborhood of $x$ contained inside $V\cap W$ \sref{1.1.1.10}[i].
The ringed space $(D(f),\OO_X|D(f))$ is thus an affine scheme, isomorphic to $A_f$ \sref{1.1.3.6}, whence the proposition.
\end{proof}

\begin{prop}[2.1.4]
\label{1.2.1.4}
The underlying space of a prescheme is a Kolmogoroff space.
\end{prop}

\begin{proof}
\label{proof-1.2.1.4}
If $x$ and $y$ are two distinct points of a prescheme $X$, then it is clear that there exists an open neighborhood of one of these points that does not contain the other if $x$ and $y$ are not in the same affine open subset; and if they are in the same affine open subset, this is a result of \sref{1.1.1.8}.
\end{proof}

\begin{prop}[2.1.5]
\label{1.2.1.5}
If $(X,\OO_X)$ is a prescheme, then every closed irreducible subset of $X$ admits exactly one generic point, and the map $x\mapsto\overline{\{x\}}$ is thus a bijection of $X$ onto its set of closed irreducible subsets.
\end{prop}

\begin{proof}
\label{proof-1.2.1.5}
If $Y$ is a closed irreducible subset of $X$ and $y\in Y$, and if $U$ is an affine open neighborhood of $y$ in $X$, then $U\cap Y$ is dense in $Y$, and also irreducible (\sref[0]{0.2.1.1} and \sref[0]{0.2.1.4});
thus, by Corollary~\sref{1.1.1.14}, $U\cap Y$ is the closure in $U$ of a point $x$, and so $Y\subset\overline{U}$ is the closure of $x$ in $X$.
The uniqueness of the generic point of $X$ is a result of Proposition~\sref{1.2.1.4} and of \sref[0]{0.2.1.3}.
\end{proof}

\begin{env}[2.1.6]
\label{1.2.1.6}
If $Y$ is a closed irreducible subset of $X$, and $y$ its generic point, then the local ring $\OO_y$ (also written $\OO_{X/Y}$) is called the \emph{local ring of $X$ along $Y$}, or the \emph{local ring of $Y$ in $X$}.

If $X$ itself is irreducible and $x$ its generic point then we say that $\OO_x$ is the \emph{ring of rational functions on $X$} (cf.~\textsection7).
\end{env}

\begin{prop}[2.1.7]
\label{1.2.1.7}
If $(X,\OO_X)$ is a prescheme, then the ringed space $(U,\OO_X|U)$ is a prescheme for every open subset $U$.
\end{prop}

\begin{proof}
\label{proof-1.2.1.7}
This follows directly from Definition~\sref{1.2.1.2} and Proposition~\sref{1.2.1.3}.
\end{proof}

We say that $(U,\OO_X|U)$ is the prescheme \emph{induced} on $U$ by $(X,\OO_X)$, or the \emph{restriction} of $(X,\OO_X)$ to $U$.

\begin{env}[2.1.8]
\label{1.2.1.8}
We say that a prescheme $(X,\OO_X)$ is \emph{irreducible} (resp. \emph{connected}) if the underlying space $X$ is irreducible (resp. connected).
We say that a prescheme is \emph{integral} if it is \emph{irreducible and reduced} (cf. \sref{1.5.1.4}).
We say that a prescheme $(X,\OO_X)$ is \emph{locally integral} if every $x\in X$ admits an open neighborhood $U$ such that the prescheme induced on $U$ by $(X,\OO_X)$ is integral.
\end{env}

\subsection{Morphisms of preschemes}
\label{subsection:prescheme-morphisms}

\begin{defn}[2.2.1]
\label{1.2.2.1}
Given two preschemes, $(X,\OO_X)$ and $(Y,\OO_Y)$, we define a morphism (of preschemes) from $(X,\OO_X)$ to $(Y,\OO_Y)$ to be a morphism of ringed spaces $(\psi,\theta)$ such that, for all $x\in X$, $\theta_x^\sharp$ is a local homomorphism $\OO_{\psi(x)}\to\OO_x$.
\end{defn}

\oldpage[I]{99}
By passing to quotients, the map $\OO_{\psi(x)}\to\OO_x$ gives us a monomorphism $\theta^x:\kres(\psi(x))\to\kres(x)$, which lets us consider $\kres(x)$ as an \emph{extension} of the field $\kres(\psi(x))$.

\begin{env}[2.2.2]
\label{1.2.2.2}
The composition $(\psi'',\theta'')$ of two morphisms $(\psi,\theta)$, $(\psi',\theta')$ of preschemes is also a morphism of preschemes, since it is given by the formula ${\theta''}^\sharp=\theta^\sharp\circ\psi^*({\theta'}^\sharp)$ \sref[0]{0.3.5.5}.
From this we conclude that preschemes form a \emph{category};
using the usual notation, we will write $\Hom(X,Y)$ to mean the set of morphisms from a prescheme $X$ to a prescheme $Y$.
\end{env}

\begin{exm}[2.2.3]
\label{1.2.2.3}
If $U$ is an open subset of $X$, then the canonical injection \sref[0]{0.4.1.2} of the induced prescheme $(U,\OO_X|U)$ into $(X,\OO_X)$ is a morphism of preschemes;
it is further a \emph{monomorphism} of ringed spaces (and \emph{a fortiori} a monomorphism of preschemes), which follows rapidly from \sref[0]{0.4.1.1}.
\end{exm}

\begin{prop}[2.2.4]
\label{1.2.2.4}
Let $(X,\OO_X)$ be a prescheme, and $(S,\OO_S)$ an affine scheme associated to a ring $A$.
Then there exists a canonical bijective correspondence between morphisms of preschemes from $(X,\OO_X)$ to $(S,\OO_S)$ and ring homomorphisms from $A$ to $\Gamma(X,\OO_X)$.
\end{prop}

\begin{proof}
First note that, if $(X,\OO_X)$ and $(Y,\OO_Y)$ are two arbitrary ringed spaces,
a morphism $(\psi,\theta)$ from $(X,\OO_X)$ to $(Y,\OO_Y)$ canonically defines a ring homomorphism $\Gamma(\theta):\Gamma(Y,\OO_Y)\to\Gamma(Y,\psi_*(\OO_X))=\Gamma(X,\OO_X)$.
In the case that we consider, everything boils down to showing that any homomorphism $\vphi:A\to\Gamma(X,\OO_X)$ is of the form $\Gamma(\theta)$ for exactly one $\theta$.
Now, by hypothesis, there is a covering $(V_\alpha)$ of $X$ by affine open subsets;
by composing $\vphi$ with the restriction homomorphism $\Gamma(X,\OO_X)\to\Gamma(V_\alpha,\OO_X|V_\alpha)$, we obtain a homomorphism $\vphi_\alpha:A\to\Gamma(V_\alpha,\OO_X|V_\alpha)$ that corresponds to a unique morphism $(\psi_\alpha,\theta_\alpha)$ from the prescheme $(V_\alpha,\OO_X|V_\alpha)$ to $(S,\OO_S)$, by Theorem~\sref{1.1.7.3}.
Furthermore, for each pair of indices $(\alpha,\beta)$, each point of $V_\alpha\cap V_\beta$ admits an affine open neighborhood $W$ contained inside $V_\alpha\cap V_\beta$ \sref{1.2.1.3};
it is clear that, by composing $\vphi_\alpha$ and $\vphi_\beta$ with the restriction homomorphisms to $W$, we obtain the same homomorphism $\Gamma(S,\OO_S)\to\Gamma(W,\OO_X|W)$, so, with the equation $(\theta_\alpha^\sharp)_x=(\vphi_\alpha)_x$ for all $x\in V_\alpha$ and all $\alpha$ \sref{1.1.6.1}, the restriction to $W$ of the morphisms $(\psi_\alpha,\theta_\alpha)$ and $(\psi_\beta,\theta_\beta)$ coincide.
From this we conclude that there is a morphism $(\psi,\theta):(X,\OO_X)\to(S,\OO_S)$ of ringed spaces, and only one such that its restriction to each $V_\alpha$ is $(\psi_\alpha,\theta_\alpha)$, and it is clear that this morphism is a morphism of preschemes and such that $\Gamma(\theta)=\vphi$.

Let $u:A\to\Gamma(X,\OO_X)$ be a ring homomorphism, and $v=(\psi,\theta)$ the corresponding morphism $(X,\OO_X)\to(S,\OO_S)$.
For each $f\in A$, we have that
\begin{equation*}
  \label{1.2.2.4.1}
  \psi^{-1}(D(f))=X_{u(f)}
  \tag{2.2.4.1}
\end{equation*}
with the notation of \sref[0]{0.5.5.2} relative to the locally free sheaf $\OO_X$.
In fact, it suffices to verify this formula when $X$ itself is affine, and then this is nothing but \sref{1.1.2.2.2}.
\end{proof}

\begin{prop}[2.2.5]
\label{1.2.2.5}
Under the hypotheses of Proposition~\sref{1.2.2.4}, let $\vphi:A\to\Gamma(X,\OO_X)$ be a ring homomorphism, $f:(X,\OO_X)\to(S,\OO_S)$ the corresponding morphism of preschemes, $\sh{G}$ (resp. $\sh{F}$) an $\OO_X$-module (resp. $\OO_S$-module), and $M=\Gamma(S,\sh{F})$.
Then there exists a canonical bijective
\oldpage[I]{100}
correspondence between $f$-morphisms $\sh{F}\to\sh{G}$ \sref[0]{0.4.4.1} and $A$-homomorphisms $M\to(\Gamma(X,\sh{G}))_{[\vphi]}$.
\end{prop}

\begin{proof}
\label{proof-1.2.2.5}
Reasoning as in Proposition~\sref{1.2.2.4}, we reduce to the case where $X$ is affine, and the proposition then follows from Proposition~\sref{1.1.6.3} and from Corollary~\sref{1.1.3.8}.
\end{proof}

\begin{env}[2.2.6]
\label{1.2.2.6}
We say that a morphism of preschemes $(\psi,\theta):(X,\OO_X)\to(Y,\OO_Y)$ is \emph{open} (resp. \emph{closed}) if, for all open subsets $U$ of $X$ (resp. all closed subsets $F$ of $X$), $\psi(U)$ is open (resp. $\psi(F)$ is closed) in $Y$.
We say that $(\psi,\theta)$ is \emph{dominant} if $\psi(X)$ is dense in $Y$, and \emph{surjective} if $\psi$ is surjective.
We note that these conditions rely only on the continuous map $\psi$.
\end{env}

\begin{prop}[2.2.7]
\label{1.2.2.7}
Let
\[
  f=(\psi,\theta):(X,\OO_X)\to(Y,\OO_Y)
\]
and
\[
  g=(\psi',\theta'):(Y,\OO_Y)\to(Z,\OO_Z)
\]
be morphisms of preschemes.
\begin{enumerate}[label=\emph{(\roman*)}]
  \item If $f$ and $g$ are both open (resp. closed, dominant, surjective),
    then so is $g\circ f$.
  \item If $f$ is surjective and $g\circ f$ closed, then $g$ is closed.
  \item If $g\circ f$ is surjective, then $g$ is surjective.
\end{enumerate}
\end{prop}

\begin{proof}
\label{proof-1.2.2.7}
Claims (i) and (iii) are evident.
Write $g\circ f=(\psi'',\theta'')$.
If $F$ is closed in $Y$ then $\psi^{-1}(F)$ is closed in $X$, so $\psi''(\psi^{-1}(F))$ is closed in $Z$;
but since $\psi$ is surjective, $\psi(\psi^{-1}(F))=F$, so $\psi''(\psi^{-1}(F))=\psi'(F)$, which proves (ii).
\end{proof}

\begin{prop}[2.2.8]
\label{1.2.2.8}
Let $f=(\psi,\theta)$ be a morphism $(X,\OO_X)\to(Y,\OO_Y)$, and $(U_\alpha)$ an open cover of $Y$.
For $f$ to be open (resp. closed, surjective, dominant), it is necessary and sufficient for its restriction to each induced prescheme $(\psi^{-1}(U_\alpha),\OO_X|\psi^{-1}(U_\alpha))$, considered as a morphism of preschemes from this induced prescheme to the induced prescheme $(U_\alpha,\OO_Y|U_\alpha)$ to be open (resp. closed, surjective, dominant).
\end{prop}

\begin{proof}
\label{proof-1.2.2.8}
The proposition follows immediately from the definitions, taking into account the fact that a subset $F$ of $Y$ is closed (resp. open, dense) in $Y$ if and only if each of the $F\cap U_\alpha$ are closed (resp. open, dense) in $U_\alpha$.
\end{proof}

\begin{env}[2.2.9]
\label{1.2.2.9}
Let $(X,\OO_X)$ and $(Y,\OO_Y)$ be two preschemes;
suppose that $X$ (resp. $Y$) has a finite number of irreducible components $X_i$ (resp. $Y_i$) ($1\leq i\leq n$);
let $\xi_i$ (resp. $\eta_i$) be the generic point of $X_i$ (resp. $Y_i$) \sref{1.2.1.5}.
We say that a morphism
\[
  f=(\psi,\theta):(X,\OO_X)\to(Y,\OO_Y)
\]
is \emph{birational} if, for all $i$, $\psi^{-1}(\eta_i)=\{\xi_i\}$ and $\theta_{\xi_i}^\sharp:\OO_{\eta_i}\to\OO_{\xi_i}$ is an \emph{isomorphism}.
It is clear that a birational morphism is dominant \sref[0]{0.2.1.8}, and thus it is surjective if it is also closed.
\end{env}

\begin{nota}[2.2.10]
\label{1.2.2.10}
In all that follows, when there is no risk of confusion, we \emph{suppress} the structure sheaf (resp. the morphism of structure sheaves) from the notation of a prescheme (resp. morphism of preschemes).
If $U$ is an open subset of the underlying space $X$ of a prescheme, then whenever we speak of $U$ as a prescheme we always mean the induced prescheme on $U$.
\end{nota}

\subsection{Gluing preschemes}
\label{subsection:gluing-preschemes}

\begin{env}[2.3.1]
\label{1.2.3.1}
\oldpage[I]{101}
It follows from Definition~\sref{1.2.1.2} that every ringed space obtained by \emph{gluing} preschemes \sref[0]{0.4.1.7} is again a prescheme.
In particular, although every prescheme admits, by definition, a cover by affine open sets, we see that every prescheme can actually be obtained by \emph{gluing affine schemes}.
\end{env}

\begin{env}[Example]{2.3.2}
\label{1.2.3.2}
Let $K$ be a field, $B=K[s]$ and $C=K[t]$ polynomial rings in one indeterminate over $K$, and define $X_1=\Spec(B)$ and $X_2=\Spec(C)$, which are isomorphic affine schemes.
In $X_1$ (resp. $X_2$), let $U_{12}$ (resp. $U_{21}$) be the affine open $D(s)$ (resp. $D(t)$) where the ring $B_s$ (resp. $C_t$) is formed of rational fractions of the form $f(s)/s^m$ (resp. $g(t)/t^n$) with $f\in B$ (resp. $g\in C$).
Let $u_{12}$ be the isomorphism of preschemes $U_{21}\to U_{12}$ corresponding \sref{1.2.2.4} to the isomorphism from $B_s$ to $C_t$ that, to $f(s)/s^m$, associates the rational fraction $f(1/t)/(1/t^m)$.
We can glue $X_1$ and $X_2$ along $U_{12}$ and $U_{21}$ by using $u_{12}$, because there is clearly no gluing condition.
We later show that the prescheme $X$ obtained in this manner is a particular case of a general method of construction \sref[II]{2.2.4.3}.
Here we show only that $X$ \emph{is not an affine scheme};
this will follow from the fact that the ring $\Gamma(X,\OO_X)$ is \emph{isomorphic} to $K$, and so its spectrum reduces to a point.
Indeed, a section of $\OO_X$ over $X$ has a restriction over $X_1$ (resp. $X_2$), identified with an affine open of $X$, that is a polynomial $f(s)$ (resp. $g(t)$), and it follows from the definitions that we should have $g(t)=f(1/t)$, which is not possible if $f=g\in K$.
\end{env}

\subsection{Local schemes}
\label{subsection:local-schemes}

\begin{env}[2.4.1]
\label{1.2.4.1}
We say that an affine scheme is a \emph{local scheme} if it is the affine scheme associated to a local ring $A$;
there then exists, in $X=\Spec(A)$, a single \emph{closed point $a\in X$}, and for all other $b\in X$ we have that $a\in\overline{\{b\}}$ \sref{1.1.1.7}.
\end{env}

For all preschemes $Y$ and points $y\in Y$, the local scheme $\Spec(\OO_y)$
is called the \emph{local scheme of $Y$ at the point $y$}.
Let $V$ be an affine open subset of $Y$ containing $y$, and $B$ the ring of the affine scheme $V$;
then $\OO_y$ is canonically identified with $B_y$ \sref{1.1.3.4}, and the canonical homomorphism $B\to B_y$ thus corresponds \sref{1.1.6.1} to a morphism of preschemes $\Spec(\OO_y)\to V$.
If we compose this morphism with the canonical injection $V\to Y$, then we obtain a morphism $\Spec(\OO_y)\to Y$ which is \emph{independent} of the affine open subset $V$ (containing $y$) that we chose: indeed, if $V'$ is some other affine open subset containing $y$, then there exists a third affine open subset $W$ that contains $y$ and is such that $W\subset V\cap V'$ \sref{1.2.1.3};
we can thus assume that $V\subset V'$, and then if $B'$ is the ring of $V'$, so everything relies on remarking that the diagram
\[
  \xymatrix{
    B'\ar[rr]\ar[dr] & &
    B\ar[dl]\\
    & \OO_y
  }
\]
is commutative \sref[0]{0.1.5.1}.
The morphism
\[
  \Spec(\OO_y)\to Y
\]
thus defined is said to be {\it canonical}.

\begin{prop}[2.4.2]
\label{1.2.4.2}
\oldpage[I]{102}
Let $(Y,\OO_Y)$ be a prescheme;
for all $y\in Y$, let $(\psi,\theta)$ be the canonical morphism $(\Spec(\OO_y),\widetilde{\OO}_y)\to(Y,\OO_Y)$.
Then $\psi$ is a homeomorphism from $\Spec(\OO_y)$ to the subspace $S_y$ of $Y$ given by the $z$ such that $y\in\overline{\{z\}}$ (\emph{or, equivalently, the generalizations of $y$ \sref[0]{0.2.1.2}};
furthermore, if $z=\psi(\fk{p})$, then $\theta_z^\sharp:\OO_z\to(\OO_y)_\fk{p}$ is an isomorphism;
$(\psi,\theta)$ is thus a monomorphism of ringed spaces.
\end{prop}

\begin{proof}
\label{proof-1.2.4.2}
Since the unique closed point $a$ of $\Spec(\OO_y)$ is contained in the closure of any point of this space, and since $\psi(a)=y$, the image of $\Spec(\OO_y)$ under the continuous map $\psi$ is contained in $S_y$.
Since $S_y$ is contained in every affine open containing $y$, one can consider just the case where $Y$ is an affine scheme;
but then this proposition follows from \sref{1.1.6.2}.
\end{proof}

\emph{We see \sref{1.2.1.5} that there is a bijective correspondence between $\Spec(\OO_y)$ and the set of closed irreducible subsets of $Y$ containing $y$.}

\begin{cor}[2.4.3]
\label{1.2.4.3}
For $y\in Y$ to be the generic point of an irreducible component of $Y$, it is necessary and sufficient for the only prime ideal of the local ring $\OO_y$ to be its maximal ideal (\emph{in other words, for $\OO_y$ to be of \emph{dimension zero}}).
\end{cor}

\begin{prop}[2.4.4]
\label{1.2.4.4}
Let $(X,\OO_X)$ be a local scheme of some ring $A$, $a$ its unique closed point, and $(Y,\OO_Y)$ a prescheme.
Every morphism $u=(\psi,\theta):(X,\OO_X)\to(Y,\OO_Y)$ then factors uniquely as $X\to\Spec(\OO_{\psi(a)})\to Y$, where the second arrow denotes the canonical morphism, and the first corresponds to a local homomorphism $\OO_{\psi(a)}\to A$.
This establishes a canonical bijective correspondence between the set of morphisms $(X,\OO_X)\to(Y,\OO_Y)$ and the set of local homomorphisms $\OO_y\to A$ for ($y\in Y$).
\end{prop}

Indeed, for all $x\in X$, we have that $a\in\overline{\{x\}}$, so $\psi(a)\in\overline{\{\psi(x)\}}$, which shows that $\psi(X)$ is contained in every affine open subset that contains $\psi(a)$.
So it suffices to consider the case where $(Y,\OO_Y)$ is an affine scheme of ring $B$, and then we have that $u=({}^a\vphi,\tilde{\vphi})$, where $\vphi\in\Hom(B,A)$ \sref{1.1.7.3}.
Further, we have that $\vphi^{-1}(\fk{j}_a)=\fk{j}_{\psi(a)}$, and hence that the image under $\vphi$ of any element of $B\setmin\fk{j}_{\psi(a)}$ is invertible in the local ring $A$;
the factorization in the result follows from the universal property of the ring of fractions \sref[0]{0.1.2.4}.
Conversely, to each local homomorphism $\OO_y\to A$ there is a unique corresponding morphism $(\psi,\theta):X\to\Spec(\OO_y)$ such that $\psi(a)=y$ \sref{1.1.7.3}, and, by composing with the canonical morphism $\Spec(\OO_y)\to Y$, we obtain a morphism $X\to Y$, which proves the proposition.

\begin{env}[2.4.5]
\label{1.2.4.5}
The affine schemes whose ring is a field $K$ have an underlying space that is just a point.
If $A$ is a local ring with maximal ideal $\fk{m}$, then each local homomorphism $A\to K$ has kernel equal to $\fk{m}$, and so factors as $A\to A/\fk{m}\to K$, where the second arrow is a monomorphism.
The morphisms $\Spec(K)\to\Spec(A)$ thus correspond bijectively to monomorphisms of fields $A/\fk{m}\to K$.
\end{env}

Let $(Y,\OO_Y)$ be a prescheme;
for each $y\in Y$ and each ideal $\fk{a}_y$ of $\OO_y$, the canonical homomorphism $\OO_y\to\OO_y/\fk{a}_y$ defines a morphism $\Spec(\OO_y/\fk{a}_y)\to\Spec(\OO_y)$;
if we compose this with the canonical morphism $\Spec(\OO_y)\to Y$, then we obtain a morphism $\Spec(\OO_y/\fk{a}_y)\to Y$, again said to be \textit{canonical}.
For $\fk{a}_y=\fk{m}_y$, this says that $\OO_y/\fk{a}_y=\kres(y)$, and so Proposition~\sref{1.2.4.4} says that:

\begin{cor}[2.4.6]
\label{1.2.4.6}
\oldpage[I]{103}
Let $(X,\OO_X)$ be a local scheme whose ring $K$ is a field, $\xi$ the unique point of $X$, and $(Y,\OO_Y)$ a prescheme.
Then each morphism $u:(X,\OO_X)\to(Y,\OO_Y)$ factors uniquely as $X\to\Spec(\kres(\psi(\xi)))\to Y$, where the second arrow denotes the canonical morphism, and the first corresponds to a monomorphism $\kres(\psi(\xi))\to K$.
This establishes a canonical bijective correspondence between the set of morphisms $(X,\OO_X)\to (Y,\OO_Y)$ and the set of monomorphisms $\kres(y)\to K$ (for $y\in Y$).
\end{cor}

\begin{cor}[2.4.7]
\label{1.2.4.7}
For all $y\in Y$, every canonical morphism $\Spec(\OO_y/\fk{a}_y)\to Y$ is a monomorphism of ringed spaces.
\end{cor}

\begin{proof}
\label{proof-1.2.4.7}
We have already seen this when $\fk{a}_y=0$ \sref{1.2.4.2}, and it suffices to apply Corollary~\sref{1.1.7.5}.
\end{proof}

\begin{rmk}{2.4.8}
\label{1.2.4.8}
Let $X$ be a local scheme, and $a$ its unique closed point.
Since every affine open subset containing $a$ is necessarily equal to the whole of $X$, every \emph{invertible} $\OO_X$-module \sref[0]{0.5.4.1} is necessarily \emph{isomorphic to $\OO_X$} (or, as we say, again, \emph{trivial}).
This property does not hold in general for an arbitrary affine scheme $\Spec(A)$;
we will see in Chapter~V that if $A$ is a normal ring then this is true when $A$ is a unique factorisation domain.
\end{rmk}

\subsection{Preschemes over a prescheme}
\label{subsection:preschemes-over-prescheme}

\begin{defn}[2.5.1]
\label{1.2.5.1}
Given a prescheme $S$, we say that the data of a prescheme $X$ and a morphism of preschemes $\vphi:X\to S$ defines a prescheme $X$ \emph{over the prescheme $S$}, or an \emph{$S$-prescheme};
we say that $S$ is the \emph{base prescheme} of the $S$-prescheme $X$.
The morphism $\vphi$ is called the \emph{structure morphism} of the $S$-prescheme $X$.
When $S$ is an affine scheme of ring $A$, we also say that $X$ endowed with $\vphi$ is a prescheme \emph{over the ring $A$} (or an \emph{$A$-prescheme}).
\end{defn}

It follows from \sref{1.2.2.4} that the data of a prescheme over a ring $A$ is equivalent to the data of a prescheme $(X,\OO_X)$ whose structure sheaf $\OO_X$ is a sheaf of \emph{$A$-algebras}.
\emph{An arbitrary prescheme can always be considered as a $\bb{Z}$-prescheme in a unique way.}

If $\vphi:X\to S$ is the structure morphism of an $S$-prescheme $X$, we
say that a point $x\in X$ is \emph{over a point $s\in S$} if $\vphi(x)=s$. We
say that $X$ \emph{dominates} $S$ if $\vphi$ is a dominant morphism \sref{1.2.2.6}.

\begin{env}[2.5.2]
\label{1.2.5.2}
Let $X$ and $Y$ be $S$-preschemes;
we say that a morphism of preschemes $u:X\to Y$ is a \emph{morphism of preschemes over $S$} (or an \emph{$S$-morphism}) if the diagram
\[
  \xymatrix{
    X \ar[rr]^u \ar[dr] & & Y\ar[dl]\\
    & S &
  }
\]
(where the diagonal arrows are the structure morphisms) is commutative: this ensures that, for all $s\in S$ and $x\in X$ over $s$, $u(x)$ also lies over $s$.
\end{env}

It follows immediately from this definition that the composition of any two $S$-morphisms is an $S$-morphism;
$S$-preschemes thus form a \emph{category}.

We denote by $\Hom_S(X,Y)$ the set of $S$-morphisms from an $S$-prescheme $X$ to an $S$-prescheme $Y$;
the identity morphism of an $S$-prescheme $X$ is denoted by $1_X$.

When $S$ is an affine scheme of ring $A$, we will also say \emph{$A$-morphism} instead of $S$-morphism.

\begin{env}[2.5.3]
\label{1.2.5.3}
\oldpage[I]{104}
If $X$ is an $S$-prescheme, and $v:X'\to X$ a morphism of preschemes, then the composition $X'\to X\to S$ endows $X'$ with the structure of an $S$-prescheme;
in particular, every prescheme induced by an open set $U$ of $X$ can be considered as an $S$-prescheme by the canonical injection.

If $u:X\to Y$ is an $S$-morphism of $S$-preschemes, then the restriction of $u$ to any prescheme induced by an open subset $U$ of $X$ is also an $S$-morphism $U\to Y$.
Conversely, let $(U_\alpha)$ be an open cover of $X$, and for each $\alpha$ let $u_\alpha:U_\alpha\to Y$ be an $S$-morphism;
if, for all pairs of indices $(\alpha,\beta)$, the restrictions of $u_\alpha$ and $u_\beta$ to $U_\alpha\cap U_\beta$ agree, then there exists an $S$-morphism $X\to Y$, and exactly one such that the restriction to each $U_\alpha$ is $u_\alpha$.

If $u:X\to Y$ is an $S$-morphism such that $u(X)\subset V$, where $V$ is an open subset of $Y$, then $u$, considered as a morphism from $X$ to $V$, is also an $S$-morphism.
\end{env}

\begin{env}[2.5.4]
\label{1.2.5.4}
Let $S'\to S$ be a morphism of preschemes;
for all $S'$-preschemes, the composition $X\to S'\to S$ endows $X$ with the structure of an $S$-prescheme.
Conversely, suppose that $S'$ is the induced prescheme of an open subset of $S$;
let $X$ be an $S$-prescheme and suppose that the structure morphism $f:X\to S$ is such that $f(X)\subset S'$;
then we can consider $X$ as an $S'$-prescheme.
In this latter case, if $Y$ is another $S$-prescheme whose structure morphism sends the underlying space to $S'$, then every $S$-morphism from $X$ to $Y$ is also an $S'$-morphism.
\end{env}

\begin{env}[2.5.5]
\label{1.2.5.5}
If $X$ is an $S$-prescheme, with structure morphism $\vphi:X\to S$, we define an \emph{$S$-section of $X$} to be an $S$-morphism from $S$ to $X$, that is to say a morphism of preschemes $\psi:S\to X$ such that $\vphi\circ\psi$ is the identity on $S$.
We denote by $\Gamma(X/S)$ the set of $S$-sections of $X$.
\end{env}


\section{Products of preschemes}
\label{section:I.3}

\subsection{Sums of preschemes}
\label{subsection:I.3.1}

Let $(X_\alpha)$ be any family of preschemes; let $X$ be a topological space which is the
\emph{sum} of the underlying spaces $X_\alpha$; $X$ is then the union of pairwise
disjoint open subspaces $X_\alpha'$, and for each $\alpha$ there is a homomorphism
$\vphi_\alpha$ from $X_\alpha$ to $X_\alpha'$. If we equip each of the $X_\alpha'$ with the
sheaf $(\vphi_\alpha)_*(\sh{O}_{X_\alpha})$, it is clear that $X$ becomes a prescheme, which
we call the \emph{sum} of the family of preschemes $(X_\alpha)$ and which we denote
$\bigsqcup_\alpha X_\alpha$. If $Y$ is a prescheme, then the map $f\mapsto(f\circ\vphi_\alpha)$ is a
\emph{bijection} from the set $\Hom(X,Y)$ to the product set $\prod_\alpha\Hom(X_\alpha,Y)$.
In particular, if the $X_\alpha$ are $S$-preschemes, with structure morphisms $\psi_\alpha$, then $X$ is an $S$-prescheme by the unique morphism $\psi:X\to S$ such that
$\psi\circ\vphi_\alpha=\psi_\alpha$ for each $\alpha$. The sum of two preschemes $X$ and $Y$ is denoted by $X\sqcup Y$.
It is immediate that, if $X=\Spec(A)$ and $Y=\Spec(B)$, then $X\sqcup Y$ is canonically identified with $\Spec(A\times B)$.

\subsection{Products of preschemes}
\label{subsection:I.3.2}

\begin{definition}[3.2.1]
\label{I.3.2.1}
Given $S$-preschemes $X$ and $Y$, we say that a triple $(Z,p_1,p_2)$, consisting of an
$S$-prescheme $Z$, and $S$-morphisms $p_1:Z\to X$ and $p_2:Z\to Y$, is a product of the
\oldpage[I]{105}
$S$-preschemes $X$ and $Y$, if, for each $S$-prescheme $T$, the map
$f\mapsto(p_1\circ f,p_2\circ f)$ is a bijection from the set of $S$-morphisms from $T$ to
$Z$, to the set of pairs consisting of an $S$-morphism $T\to X$ and an $S$-morphism $T\to Y$
(in other words, a bijection
\[
  \Hom_S(T,Z)\isoto\Hom_S(T,X)\times\Hom_S(T,Y)).
\]
\end{definition}

This is the general notion of a \emph{product} of two objects in a category, applied
to the category of $S$-preschemes (T, I, 1.1); in particular, a product of two $S$-preschemes
is \emph{unique} up to a unique $S$-isomorphism. Because of this uniqueness, most of the time
we will denote a product of two $S$-preschemes $X$ and $Y$ by $X\times_S Y$ (or
simply $X\times Y$, when there is no chance of confusion), with the morphisms $p_1$ and $p_2$
(the \emph{canonical projections} of $X\times_S Y$ to $X$ and to $Y$, respectively) being suppressed
in the notation. If $g:T\to X$ and $h:T\to Y$ are $S$-morphisms, we denote by $(g,h)_S$, or
simply $(g,h)$, the $S$-morphism $f:T\to X\times_S Y$ such that $p_1\circ f=g$,
$p_2\circ f=h$. If $X'$ and $Y'$ ar two $S$-preschemes, $p_1'$ and $p_2'$ the canonical
projections of $X'\times_S Y'$ (assumed to exist), and $u:X'\to X$ and $v:Y'\to Y$ $S$-morphisms, then we write $u\times_S v$ (or simply $u\times v$) for the $S$-morphism
$(u\circ p_1',v\circ p_2')_S$ from $X'\times_S Y'$ to $X\times_S Y$.

When $S$ is an affine scheme given by some ring $A$, we often replace $S$
by $A$ is the above notation.

\begin{proposition}[3.2.2]
\label{I.3.2.2}
Let $X$, $Y$, and $S$ be affine schemes, given by rings $B$, $C$, and $A$ (respectively). Let
$Z=\Spec(B\otimes_A C)$, and let $p_1$ and $p_2$ be the $S$-morphisms corresponding \sref{I.2.2.4} to
the canonical $A$-homomorphisms $u:b\mapsto b\otimes 1$ and $v:c\mapsto 1\otimes c$ (respectively) from $B$
and $C$ to $B\otimes_A C$; then $(Z,p_1,p_2)$ is a product of $X$ and $Y$.
\end{proposition}

\begin{proof}
According to \sref{I.2.2.4}, it suffices to check that, if, to each $A$-homomorphism
$f:B\otimes_A C\to L$ (where $L$ is an $A$-algebra), we associate the pair
$(f\circ u,f\circ v)$, then this defines a bijection
$\Hom_A(B\otimes_A C,L)\isoto\Hom_A(B,L)\times\Hom_A(C,L)$,\footnote{The notation $\Hom_A$
denotes here the set of homomorphisms of \emph{$A$-algebras}.} which follows immediately
from the definitions and the fact that $b\otimes c=(b\otimes 1)(1\otimes c)$.
\end{proof}

\begin{corollary}[3.2.3]
\label{I.3.2.3}
Let $T$ be an affine scheme given by some ring $D$, and $\alpha=({}^a\rho,\widetilde{\rho})$
(resp. $\beta=({}^a\sigma,\widetilde{\sigma})$) an $S$-morphism $T\to X$ (resp. $T\to Y$),
where $\rho$ (resp. $\sigma$) is an $A$-homomorphism from $B$ (resp. $C$) to $D$; then
$(\alpha,\beta)_S=({}^a\tau,\widetilde{\tau})$, where $\tau$ is the homomorphism
$B\otimes_A C\to D$ such that $\tau(b\otimes c)=\rho(b)\sigma(c)$.
\end{corollary}

\begin{proposition}[3.2.4]
\label{I.3.2.4}
Let $f:S'\to S$ be a \emph{monomorphism} of preschemes \emph{(T, I, 1.1)}, and let $X$ and $Y$ be
$S'$-preschemes, also considered as $S$-preschemes via $f$. Every product of
the $S$-preschemes $X$ and $Y$ is then a product of the $S'$-preschemes $X$ and $Y$, and vice versa.
\end{proposition}

\begin{proof}
Let $\vphi:X\to S'$ and $\psi:Y\to S'$ be the structure morphisms. If $T$ is an $S$-prescheme, and
$u:T\to X$ and $v:T\to Y$ are $S$-morphisms, then we have, by definition, that
$f\circ\vphi\circ u=f\circ\psi\circ v=\theta$, the structure morphism of $T$; the hypotheses
on $f$ imply that $\vphi\circ u=\psi\circ v=\theta'$, and so we can consider $T$ as an
$S'$-prescheme with structure morphism $\theta'$, and $u$ and $v$ as $S'$-morphisms. The
conclusion of the proposition follows immediately, taking \sref{I.3.2.1} into account.
\end{proof}

\begin{corollary}[3.2.5]
\label{I.3.2.5}
Let $X$ and $Y$ be $S$-preschemes, with structure morphisms $\vphi:X\to S$ and $\psi:Y\to S$, and let
$S'$ be an open subset of $S$ such that $\vphi(X)\subset S'$ and $\psi(Y)\subset S'$. Every product
of the $S$-preschemes $X$ and $Y$ is then also a product of the $S'$-preschemes $X$ and $Y$, and
conversely.
\end{corollary}

\oldpage[I]{106}
It suffices to apply \sref{I.3.2.4} to the canonical injection $S'\to S$.

\begin{theorem}[3.2.6]
\label{I.3.2.6}
Given $S$-preschemes $X$ and $Y$, there exists a product $X\times_S Y$.
\end{theorem}

The proof proceeds in several steps.
\begin{lemma}[3.2.6.1]
\label{I.3.2.6.1}
Let $(Z,p,q)$ be a product of $X$ and $Y$, and $U$ and $V$ open subsets of $X$ and $Y$,
respectively. If we let $W=p^{-1}(U)\cap q^{-1}(V)$, then the triple consisting of $W$ and
the restrictions of $p$ and $q$ to $W$ \emph{(considered as the morphisms $W\to U$ and $W\to V$,
respectively)} is a product of $U$ and $V$.
\end{lemma}

Indeed, if $T$ is an $S$-prescheme, then we can identify the $S$-morphisms $T\to W$ with the
$S$-morphisms $T\to Z$ mapping $T$ to $W$. Then, if $g:T\to U$ and $h:T\to V$ are any two
$S$-morphisms, we can consider them as $S$-morphisms from $T$ to $X$ and to $Y$, respectively,
and, by hypothesis, there is then a unique $S$-morphism $f:T\to Z$ such that $g=p\circ f$ and
$h=q\circ f$. Since $p(f(Y))\subset U$, $q(f(T))\subset V$, we have
\[
  f(T)\subset p^{-1}(U)\cap q^{-1}(V)=W,
\]
hence our claim.

\begin{lemma}[3.2.6.2]
\label{I.3.2.6.2}
Let $Z$ be an $S$-prescheme, $p:Z\to X$ and $q:Z\to Y$ both $S$-morphisms, $(U_\alpha)$ an open
cover of $X$, and $(V_\lambda)$ an open cover of $Y$. Suppose that, for each pair
$(\alpha,\lambda)$, the $S$-prescheme
$W_{\alpha\lambda}=p^{-1}(U_\alpha)\cap q^{-1}(V_\lambda)$ and the restrictions of $p$ and
$q$ to $W_{\alpha\lambda}$ form a product of $U_\alpha$ and $V_\lambda$. Then $(Z,p,q)$ is a
product of $X$ and $Y$.
\end{lemma}

We first show that, if $f_1$ and $f_2$ are $S$-morphisms $T\to Z$, then the equations
$p\circ f_1=p\circ f_2$ and $q\circ f_1=q\circ f_2$ imply that $f_1=f_2$. Indeed, $Z$ is the union
of the $W_{\alpha\lambda}$, so the $f_1^{-1}(W_{\alpha\lambda})$ form an open cover of $T$,
and similarly for $f_2^{-1}(W_{\alpha\lambda})$. In addition, we have
\[
  f_1^{-1}(W_{\alpha\lambda})=f_1^{-1}(p^{-1}(U_\alpha))\cap f_1^{-1}(q^{-1}(V_\lambda))
  =f_2^{-1}(p^{-1}(U_\alpha))\cap f_2^{-1}(q^{-1}(V_\lambda))=f_2^{-1}(W_{\alpha\lambda})
\]
by hypothesis, and it thus reduces to noting that the the restrictions of $f_1$ and $f_2$ to
$f_1^{-1}(W_{\alpha\lambda})=f_2^{-1}(W_{\alpha\lambda})$ are identical for each pair of
indices. But since these restrictions can be considered as $S$-morphisms from
$f_1^{-1}(W_{\alpha\lambda})$ to $W_{\alpha\lambda}$, our claim follows from the
hypotheses and Definition~\sref{I.3.2.1}.

Suppose now that we are given $S$-morphisms $g:T\to X$ and $h:T\to Y$. Let
$T_{\alpha\lambda}=g^{-1}(U_\alpha)\cap h^{-1}(V_\lambda)$; then the $T_{\alpha\lambda}$ form an
open cover of $T$. By hypothesis, there exists an $S$-morphism $f_{\alpha\lambda}$ such that
$p\circ f_{\alpha\lambda}$ and $q\circ f_{\alpha\lambda}$ are the restrictions of
$g$ and $h$ to $T_{\alpha\lambda}$ (respectively). Now, we will show that the restrictions of
$f_{\alpha\lambda}$ and $f_{\beta\mu}$ to $T_{\alpha\lambda}\cap T_{\beta\mu}$ coincide,
which will finish the proof of Lemma~\sref{I.3.2.6.2}. The images of
$T_{\alpha\lambda}\cap T_{\beta\mu}$ under $f_{\alpha\lambda}$ and $f_{\beta\mu}$ are
contained in $W_{\alpha\lambda}\cap W_{\beta\mu}$ by definition. Since
\[
  W_{\alpha\lambda}\cap W_{\beta\mu}
  =p^{-1}(U_\alpha\cap U_\beta)\cap q^{-1}(V_\lambda\cap V_\mu),
\]
it follows from Lemma~\sref{I.3.2.6.1} that $W_{\alpha\lambda}\cap W_{\beta\mu}$ and the
restrictions to this prescheme of $p$ and $q$ form a \emph{product} of $U_\alpha\cap U_\beta$
and $V_\lambda\cap V_\mu$. Since $p\circ f_{\alpha\lambda}$ and $p\circ f_{\beta\mu}$ coincide
on $T_{\alpha\lambda}\cap T_{\beta\mu}$ and similarly for $q\circ f_{\alpha\lambda}$ and
$q\circ f_{\beta\mu}$, we see that $f_{\alpha\lambda}$ and $f_{\beta\mu}$ coincide on
$T_{\alpha\lambda}\cap T_{\beta\mu}$.

\begin{lemma}[3.2.6.3]
\label{I.3.2.6.3}
\oldpage[I]{107}
Let $(U_\alpha)$ be an open cover of $X$, $(V_\lambda)$ an open cover of $Y$, and suppose
that, for each pair $(\alpha,\lambda)$, there exists a product of $U_\alpha$ and $V_\lambda$;
then there exists a product of $X$ and $Y$.
\end{lemma}

Applying Lemma~\sref{I.3.2.6.1} to the open sets $U_\alpha\cap U_\beta$ and
$V_\lambda\cap V_\mu$, we see that there exists a product of $S$-preschemes induced,
respectively, by $X$ and $Y$ on these open sets; in addition, the uniqueness of the product
shows that, if we set $i=(\alpha,\lambda)$ and $j=(\beta,\mu)$, then there is a canonical
isomorphism $h_{ij}$ (resp. $h_{ji}$) from this product to an $S$-prescheme $W_{ij}$
(resp. $W_{ji}$) induced by $U_\alpha\times_S V_\lambda$ (resp. $U_\beta\times_S V_\mu$) on
an open set; then $f_{ij}=h_{ij}\circ h_{ji}^{-1}$ is an isomorphism from $W_{ji}$ to
$W_{ij}$. In addition, for a third pair $k=(\gamma,\nu)$, we have
$f_{ik}=f_{ij}\circ f_{jk}$ on
$W_{ki}\cap W_{kj}$, by applying Lemma \sref{I.3.2.6.1} to
the open sets $U_\alpha\cap U_\beta\cap U_\gamma$ and $V_\lambda\cap V_\mu\cap V_\nu$ in
$U_\beta$ and $V_\mu$, respectively. It follows that we have a prescheme $Z$, an open cover
$(Z_i)$ of the underlying space of $Z$, and, for each $i$, an isomorphism $g_i$ from the
induced prescheme
$Z_i$ to the prescheme $U_\alpha\times_S V_\lambda$, so that, for each pair $(i,j)$, we have
$f_{ij}=g_i\circ g_j^{-1}$ \sref{I.2.3.1}; in addition, we have $g_i(Z_i\cap Z_j)=W_{ij}$.
If $p_i$, $q_i$, and $\theta_i$ are the projections and the structure morphism of the
$S$-prescheme $U_\alpha\times_S V_\lambda$ (respectively), we immediately see that
$p_i\circ g_i=p_j\circ g_j$ on $Z_i\cap Z_j$, and similarly for the two other morphisms. We
can thus define the morphisms of preschemes $p:Z\to X$ (resp. $q:Z\to Y$, $\theta:Z\to S$) by
the condition that $p$ (resp. $q$, $\theta$) coincide with $p_i\circ g_i$
(resp. $q_i\circ g_i$, $\theta_i\circ g_i$) on each of the $Z_i$; $Z$, equipped with
$\theta$, is then an $S$-prescheme. We now show that
$Z_i'=p^{-1}(U_\alpha)\cap q^{-1}(V_\lambda)$ is equal to $Z_i$. For each index
$j=(\beta,\mu)$, we have $Z_j\cap Z_i'=g_j^{-1}(p_j^{-1}(U_\alpha)\cap q_j^{-1}(V_\lambda))$.
We have, by Lemma~\sref{I.3.2.6.1},
\[
  p_j^{-1}(U_\alpha)\cap q_j^{-1}(V_\lambda)
  =p_j^{-1}(U_\alpha\cap U_\beta)\cap q_j^{-1}(V_\lambda\cap V_\mu);
\]
with the restrictions of $p_j$ and $q_j$ to
$p_j^{-1}(U_\alpha)\cap q_j^{-1}(V_\lambda)$ defining, on this $S$-prescheme, the structure of a
product of $U_\alpha\cap U_\beta$ and $V_\lambda\cap V_\mu$; but the uniqueness of the
product then implies that $p_j^{-1}(U_\alpha)\cap q_j^{-1}(V_\lambda)=W_{ji}$. As a result, we
have $Z_j\cap Z_i'=Z_j\cap Z_i$ for each $j$, hence $Z_i'=Z_i$. We then deduce from
Lemma~\sref{I.3.2.6.2} that $(Z,p,q)$ is a product of $X$ and $Y$.

\begin{lemma}[3.2.6.4]
\label{I.3.2.6.4}
Let $\vphi:X\to S$ and $\psi:Y\to S$ be the structure morphisms of $X$ and $Y$, $(S_i)$ an open
cover of $S$, and let $X_i=\vphi^{-1}(S_i)$, $Y_i=\psi^{-1}(S_i)$. If each of the products
$X_i\times_S Y_i$ exists, then $X\times_S Y$ exists.
\end{lemma}

According to Lemma~\sref{I.3.2.6.3}, everything follows from proving that the products
$X_i\times_S Y_i$ exists for any $i$ and $j$. Set
$X_{ij}=X_i\cap X_j=\vphi^{-1}(S_i\cap S_j)$, $Y_{ij}=Y_i\cap Y_j=\psi^{-1}(S_i\cap S_j)$;
by Lemma~\sref{I.3.2.6.1}, the product $Z_{ij}=X_{ij}\times_S Y_{ij}$ exists. We now
note that, if $T$ is an $S$-prescheme, and if $g:T\to X_i$ and $h:T\to Y_j$ are $S$-morphisms,
then we necessarily have that $\vphi(g(T))=\psi(h(T))\subset S_i\cap S_j$ by the
definition of an $S$-morphism, and thus that $g(T)\subset X_{ij}$ and $h(T)\subset Y_{ij}$; it is
then immediate that $Z_{ij}$ is the product of $X_i$ and $Y_j$.

\begin{env}[3.2.6.5]
\label{I.3.2.6.5}
We can now complete the proof of Theorem~\sref{I.3.2.6}. If $S$ is an \emph{affine
scheme}, then there are covers $(U_\alpha)$ and $(V_\lambda)$ of $X$ and $Y$ (respectively)
consisting of affine open subsets; since $U_\alpha\times_S V_\lambda$ exists, by
\sref{I.3.2.2}, $X\times_S Y$ exists similarly, by Lemma~\sref{I.3.2.6.3}. If $S$
is any prescheme, then there is a cover $(S_i)$ of $S$ consisting of affine open subsets. If
$\vphi:X\to S$ and $\psi:Y\to S$ are the structure morphisms, and if we set
$X_i=\vphi^{-1}(S_i)$ and $Y_i=\psi^{-1}(S_i)$, then the products $X_i\times_{S_i}Y_i$ exist, by the
\oldpage[I]{108}
above; but then the products $X_i\times_S Y_i$ also exist \sref{I.3.2.5}, therefore
$X\times_S Y$ exists similarly, by Lemma~\sref{I.3.2.6.4}.\qed
\end{env}

\begin{corollary}[3.2.7]
\label{I.3.2.7}
Let $Z=X\times_S Y$ be the product of two $S$-preschemes, $p$ and $q$ the projections from $Z$
to $X$ and to $Y$ (respectively),and  $\vphi$ (resp. $\psi$) the structure morphism of $X$ (resp. $Y$). Let $S'$ be
an open subset of $S$, and $U$ (resp. $V$) an open subset of $X$ (resp. $Y$) contained in
$\vphi^{-1}(S')$ (resp. $\psi^{-1}(S')$). Then the product $U\times_{S'}V$ is canonically
identified with the prescheme induced on $Z$ by $p^{-1}(U)\cap q^{-1}(V)$ (considered as an
$S'$-prescheme). In addition, if $f:T\to X$ and $g:T\to Y$ are $S$-morphisms such that
$f(T)\subset U$ and $g(T)\subset V$, then the $S'$-morphism $(f,g)_{S'}$ can be identified with the
restriction of $(f,g)_S$ to $p^{-1}(U)\cap q^{-1}(V)$.
\end{corollary}

\begin{proof}
This follows from Corollary~\sref{I.3.2.5} and Lemma~\sref{I.3.2.6.1}.
\end{proof}

\begin{env}[3.2.8]
\label{I.3.2.8}
Let $(X_\alpha)$ and $(Y_\lambda)$ be families of $S$-preschemes, and $X$ (resp. $Y$) the sum
of the family $(X_\alpha)$ (resp. $(Y_\lambda)$) \hyperref[subsection:I.3.1]{(3.1)}.
Then $X\times_S Y$ can be identified with the \emph{sum} of the family $(X_\alpha\times_S Y_\lambda)$; this follows immediately from Lemma~\sref{I.3.2.6.3}.
\end{env}

\begin{env}[3.2.9]
\label{I.3.2.9}
\footnote{\emph{[Trans.] \sref{I.3.2.9} is from the errata of EGA~II, on page 221, whence the change in page numbering.}}
\oldpage[II]{221}
It follows from \sref{I.1.8.1} that we can state \sref{I.3.2.2} in the following
manner: $Z=\Spec(B\otimes_A C)$ is not only a product of $X=\Spec(B)$ and $Y=\Spec(C)$ in the
category of \emph{$S$-preschemes}, but also in the category of \emph{locally ringed spaces
over $S$} (with a definition of $S$-morphisms modelled on that of \sref{I.2.5.2}). The
proof of \sref{I.3.2.6} also proves that, for any two $S$-preschemes $X$ and $Y$, the
prescheme $X\times_S Y$ is not only the product of $X$ and $Y$ in the category of
$S$-preschemes, but also in the category of locally ringed spaces over the prescheme $S$.
\end{env}

\subsection{Formal properties of the product; change of the base prescheme}
\label{subsection:I.3.3}

\begin{env}[3.3.1]
\label{I.3.3.1}
The reader will notice that all the properties stated in this section, except
\sref{I.3.3.13} and \sref{I.3.3.15}, are true without modification in any
category, whenever the products involved in the statements exist (since it is
clear that the notions of an $S$-object and of an $S$-morphism can be defined
exactly as in \hyperref[subsection:I.2.5]{(2.5)} for any object $S$ of the category).
\end{env}

\begin{env}[3.3.2]
\label{I.3.3.2}
First of all, $X\times_S Y$ is a \emph{covariant bifunctor} in $X$ and $Y$ on the
category of $S$-preschemes: it suffices in fact to note that the diagram
\[
  \xymatrix{
    X\times Y\ar[r]^{f\times 1}\ar[d] &
    X'\times Y\ar[r]^{f'\times 1}\ar[d] &
    X''\times Y\ar[d]\\
    X\ar[r]^f &
    X'\ar[r]^{f'} &
    X''
  }
\]
is commutative.
\end{env}

\begin{proposition}[3.3.3]
\label{I.3.3.3}
For each $S$-prescheme $X$, the first (resp. second) projection from
$X\times_S S$ (resp. $S\times_S X$) is a functorial isomorphism from
$X\times_S S$ (resp. $S\times_S X$) to $X$, whose inverse isomorphism is
$(1_X,\vphi)_S$ (resp. $(\vphi,1_X)_S$), where we denote by $\vphi$ the
structure morphism $X\to S$; we can therefore write, up to a canonical
isomorphism,
\[
  X\times_S S=S\times_S X=X.
\]
\end{proposition}

\begin{proof}
It suffices to prove that the triple $(X,1_X,\vphi)$ is a product of $X$ and
$S$. If $T$ is an $S$-prescheme, then the only $S$-morphism from $T$ to $S$ is
necessarily the structure morphism $\psi:T\to S$. If $f$ is an $S$-morphism from
$T$ to $X$, we necessarily have $\psi=\vphi\circ f$, hence our claim.
\end{proof}

\begin{corollary}[3.3.4]
\label{I.3.3.4}
Let $X$ and $Y$ be $S$-preschemes, with structure morphisms $\vphi:X\to S$ and $\psi:Y\to S$. If we canonically identify $X$ with $X\times_S S$, and $Y$
with $S\times_S Y$, then the projections $X\times_S Y\to X$ and $X\times_S Y\to Y$
are identified with $1_X\times\psi$ and $\vphi\times 1_Y$ (respectively).
\end{corollary}

The proof is immediate and is left to the reader.

\begin{env}[3.3.5]
\label{I.3.3.5}
We can define, in a manner similar to \hyperref[subsection:I.3.2]{(3.2)}, the product of a
\oldpage[I]{109}
finite number $n$ of $S$-preschemes, and the existence of these products follows
from \sref{I.3.2.6} by induction on $n$, and by noting that
$(X_1\times_S X_2\times_S\cdots\times_S X_{n-1})\times_S X_n$ satisfies the
definition of a product. The uniqueness of the product implies, as in any
category, its \emph{commutativity} and \emph{associativity} properties. If, for
example, $p_1$, $p_2$, and $p_3$ denote the projections from
$X_1\times_S X_2\times_S X_3$, and if we identify this prescheme with
$(X_1\times_S X_2)\times_S X_3$, then the projection to $X_1\times_S X_2$ is
identified with $(p_1,p_2)_S$.
\end{env}

\begin{env}[3.3.6]
\label{I.3.3.6}
Let $S$ and $S'$ be preschemes, and $\vphi:S'\to S$ a morphism, which lets us consider $S'$ as an
$S$-prescheme. For each $S$-prescheme $X$, consider the product $X\times_S S'$,
and let $p$ and $\pi'$ be the projections to $X$ and to $S'$ (respectively). Equipped
with $\pi'$, this product is an $S'$-prescheme; when we consider it as such, we
denote it by $X_{(S')}$ or $X_{(\vphi)}$, and we say that this is the prescheme
obtained by \emph{base change} (or \emph{a change of base}) from $S$ to $S'$ by means of the morphism
$\vphi$, or the \emph{inverse image} of $X$ by $\vphi$. We note that, if $\pi$ is
the structure morphism of $X$, and $\theta$ the structure morphism of
$X\times_S S'$, considered as an $S$-prescheme, then the diagram
\[
  \xymatrix{
    X\ar[d]_\pi &
    X_{(S')}\ar[l]_p\ar[ld]_\theta\ar[d]^{\pi'}\\
    S &
    S'\ar[l]_\vphi
  }
\]
is commutative.
\end{env}

\begin{env}[3.3.7]
\label{I.3.3.7}
With the notation of \sref{I.3.3.6}, for each $S$-morphism $f:X\to Y$, we
denote by $f_{(S')}$ the $S'$-morphism $f\times_S 1:X_{(S')}\to Y_{(S')}$, and
we say that $f_{(S')}$ is the \emph{base change} (or \emph{inverse image}) of
$f$ by $\vphi$. Therefore $X_{(S')}$ is a \emph{covariant functor} in $X$, from
the category of $S$-preschemes to that of $S'$-preschemes.
\end{env}

\begin{env}[3.3.8]
\label{I.3.3.8}
The prescheme $X_{(S')}$ can be considered as a solution to a \emph{universal
mapping problem}: each $S'$-prescheme $T$ is also an $S$-prescheme via $\vphi$;
each $S$-morphism $g:T\to X$ is then uniquely written as $g=p\circ f$, where $f$
is an $S'$-morphism $T\to X_{(S')}$, as follows from the definition of the
product applied to the $S$-morphisms $f$ and $\psi:T\to S'$ (the structure
morphism of $T$).
\end{env}

\begin{proposition}[3.3.9]
\label{I.3.3.9}
\emph{(``Transitivity of base change'')}. Let $S''$ be a prescheme, and
$\vphi':S''\to S$ a morphism. For each $S$-prescheme $X$, there exists a
canonical functorial isomorphism from the $S''$-prescheme
$(X_{(\vphi)})_{(\vphi')}$ to the $S''$-prescheme $X_{(\vphi\circ\vphi')}$.
\end{proposition}

\begin{proof}
Let $T$ be a $S''$-prescheme, $\psi$ its structure morphism, and $g$ an
$S$-morphism from $T$ to $X$ ($T$ being considered as an $S$-prescheme with
structure morphism $\vphi\circ\vphi'\circ\psi$). Since $T$ is also an $S'$-prescheme
with structure morphism $\vphi'\circ\psi$, we can write $g=p\circ g'$, where
$g'$ is an $S'$-morphism $T\to X_{(\vphi)}$, and then $g'=p'\circ g''$, where
$g''$ is an $S''$-morphism $T\to(X_{(\vphi)})_{(\vphi')}$:
\[
  \xymatrix{
    X\ar[d]_\pi &
    X_{(\vphi)}\ar[l]_p\ar[d]_{\pi'} &
    (X_{(\vphi)})_{(\vphi')}\ar[l]_{p'}\ar[d]^{\pi''}\\
    S &
    S\ar[l]_\vphi &
    S''\ar[l]_{\vphi'}.
  }
\]
\oldpage[I]{110}
So the result follows by the uniqueness of the solution to a universal
mapping problem.
\end{proof}

This result can be written as the equality (up to a canonical isomorphism)
$(X_{(S')})_{(S'')}=X_{(S'')}$ (if there is no chance of confusion), or also as
\[
\label{I.3.3.9.1}
  (X\times_S S')\times_{S'}S''=X\times_S S'';
  \tag{3.3.9.1}
\]
the functorial nature of the isomorphism defined in \sref{I.3.3.9} can
similarly be expressed by the transitivity formula for base change morphisms
\[
\label{I.3.3.9.2}
  (f_{(S')})_{(S'')}=f_{(S'')}
  \tag{3.3.9.2}
\]
for each $S$-morphism $f:X\to Y$.

\begin{corollary}[3.3.10]
\label{I.3.3.10}
If $X$ and $Y$ are $S$-preschemes, then there exists a canonical functorial
isomorphism from the $S'$-prescheme $X_{(S')}\times_{S'}Y_{(S')}$ to the
$S'$-prescheme $(X\times_S Y)_{(S')}$.
\end{corollary}

\begin{proof}
We have, up to canonical isomorphism,
\[
  (X\times_S S')\times_{S'}(Y\times_S S')
  =X\times_S(Y\times_S S')=(X\times_S Y)\times_S S'
\]
according to \sref{I.3.3.9.1} and the associativity of products of $S$-preschemes.
\end{proof}

The functorial nature of the isomorphism defined in
Corollary~\sref{I.3.3.10} can be expressed by the formula
\[
\label{I.3.3.10.1}
  (u_{(S')},v_{(S')})_{S'}=((u,v)_S)_{(S')}
  \tag{3.3.10.1}
\]
for each pair of $S$-morphisms $u:T\to X$, $v:T\to Y$.

In other words, the base change functor $X_{(S')}$ \emph{commutes with
products}; it also commutes with sums \sref{I.3.2.8}.

\begin{corollary}[3.3.11]
\label{I.3.3.11}
Let $Y$ be an $S$-prescheme, and $f:X\to Y$ a morphism which makes $X$ a
$Y$-prescheme (and, as a result, also an $S$-prescheme). The prescheme $X_{(S')}$
is then identified with the product $X\times_Y Y_{(S')}$, the projection
$X\times_Y Y_{(S')}\to Y_{(S')}$ being identified with $f_{(S')}$.
\end{corollary}

\begin{proof}
Let $\psi:Y\to S$ be the structure morphism of $Y$; we have the commutative
diagram
\[
  \xymatrix{
    S'\ar[d] &
    Y_{(S')}\ar[l]\ar[d] &
    X_{(S')}\ar[l]_{f_{(S')}}\ar[d]\\
    S &
    Y\ar[l]_\psi &
    X\ar[l]_f.
  }
\]
We have that $Y_{(S')}$ is identified with $S_{(\psi)}'$, and $X_{(S')}$ with
$S_{(\psi\circ f)}'$; taking \sref{I.3.3.9} and
\sref{I.3.3.4} into account, we thus deduce the corollary.
\end{proof}

\begin{env}[3.3.12]
\label{I.3.3.12}
Let $f:X\to X'$ and $g:Y\to Y'$ be $S$-morphisms which are \emph{monomorphisms}
of preschemes (T, I, 1.1); then $f\times_S g$ is a \emph{monomorphism}. Indeed,
if $p$ and $q$ are the projections of $X\times_S Y$, $p'$ and $q'$ the projections of
$X'\times_S Y'$, and $u$ and $v$ both $S$-morphisms $T\to X\times_S Y$, then the
equation $(f\times_S g)\circ u=(f\times_S g)\circ v$ implies that
$p'\circ(f\times_S g)\circ u=p'\circ(f\times_S g)\circ v$, or, in other words, that
$f\circ p\circ u=f\circ p\circ v$, and since $f$ is a monomorphism,
$p\circ u=p\circ v$; using the fact that $g$ is a monomorphism, we similarly
obtain $q\circ u=q\circ v$, hence $u=v$.

\oldpage[I]{111}
It follows that, for each base change $S'\to S$,
\[
  f_{(S')}:X_{(S')}\to Y_{(S')}
\]
is a monomorphism.
\end{env}

\begin{env}[3.3.13]
\label{I.3.3.13}
Let $S$ and $S'$ be affine schemes of rings $A$ and $A'$ (respectively); a morphism
$S'\to S$ then corresponds to a ring homomorphism $A\to A'$. If $X$ is an
$S$-prescheme, we denote by $X_{(A')}$ or $X\otimes_A A'$ the $S'$-prescheme
$X_{(S')}$; when $X$ is also affine of ring $B$, $X_{(A')}$ is affine of ring
$B_{(A')}=B\otimes_A A'$ obtained by extension of scalars from the $A$-algebra
$B$ to $A'$.
\end{env}

\begin{env}[3.3.14]
\label{I.3.3.14}
With the notation of \sref{I.3.3.6}, for each \emph{$S$-morphism}
$f:S'\to X$, we have that $f'=(f,1_{S'})_S$ is an $S'$-morphism $S'\to X'=X_{(S')}$ such that
$p\circ f'=f$, $\pi'\circ f'=1_{S'}$, or, in other words, an \emph{$S'$-section of
of $X'$}; conversely, if $f'$ is such an $S'$-section, then $f=p\circ f'$ is an
$S$-morphism $S'\to X$. We thus define a canonical
\emph{bijective correspondence}
\[
  \Hom_S(S',X)\isoto\Hom_{S'}(S',X').
\]
We say that $f'$ is the \emph{graph morphism} of $f$, and we denote it by
$\Gamma_f$.
\end{env}

\begin{env}[3.3.15]
\label{I.3.3.15}
Given a prescheme $X$, which we can always consider as a $\bb{Z}$-prescheme,
it follows, in particular, from \sref{I.3.3.14} that the \emph{$X$-sections} of
$X\otimes_\bb{Z}\bb{Z}[T]$ (where $T$ is an indeterminate)
correspond bijectively with \emph{morphisms} $\bb{Z}[T]\to X$. We will show that these
$X$-sections also correspond bijectively with \emph{sections of the structure
sheaf $\sh{O}_X$ over $X$}. Indeed, let $(U_\alpha)$ be a cover of $X$ by
affine open subsets; let $u:X\to X\otimes_\bb{Z}\bb{Z}[T]$ be an $X$-morphism, and let
$u_\alpha$ be its restriction to $U_\alpha$; if $A_\alpha$ is the ring of the
affine scheme $U_\alpha$, then $U_\alpha\otimes_\bb{Z}\bb{Z}[T]$ is an affine
scheme of ring $A_\alpha[T]$ \sref{I.3.2.2}, and $u_\alpha$ canonically
corresponds to an $A_\alpha$-homomorphism $A_\alpha[T]\to A_\alpha$
\sref{I.1.7.3}. Now, since such a homomorphism is completely determined by the
data of the image of $T$ in $A_\alpha$, let
$s_\alpha\in A_\alpha=\Gamma(U_\alpha,\sh{O}_X)$, and if we suppose that the
restrictions of $u_\alpha$ and $u_\beta$ to an affine open subset
$V\subset U_\alpha\cap U_\beta$ coincide, then we see immediately that
$s_\alpha$ and $s_\beta$ coincide on $V$; thus the family $(s_\alpha)$ consists
of the restrictions to $U_\alpha$ of a section $s$ of $\sh{O}_X$ over $X$;
conversely, it is clear that such a section defines a family $(u_\alpha)$ of
morphisms which are the restrictions to $U_\alpha$ of an $X$-morphism
$X\to X\otimes_\bb{Z}\bb{Z}[T]$. This result is generalized in
\sref[II]{II.1.7.12}.
\end{env}

\subsection{Points of a prescheme with values in a prescheme; geometric points}
\label{subsection:I.3.4}

\begin{env}[3.4.1]
\label{I.3.4.1}
Let $X$ be a prescheme; for each prescheme $T$, we then denote by $X(T)$ the set
$\Hom(T,X)$ of morphisms $T\to X$, and the elements of this set are called
\emph{the points of $X$ with values in $T$}. If we associate to each morphism
$f:T\to T'$ the map $u'\mapsto u'\circ f$ from $X(T')$ to $X(T)$, we see,
for fixed $X$, that $X(T)$ is a \emph{contravariant functor in $T$}, from the
category of preschemes to that of sets. In addition, each morphism of preschemes
$g:X\to Y$ defines a functorial homomorphism $X(T)\to Y(T)$, which sends
$v\in X(T)$ to $g\circ v$.
\end{env}

\begin{env}[3.4.2]
\label{I.3.4.2}
Given sets $P$, $Q$, and $R$, and maps $\vphi:P\to R$ and $\psi:Q\to R$, we define the
\emph{fibre product of $P$ and $Q$ over $R$} (relative to $\vphi$ and $\psi$) as the subset
of
\oldpage[I]{112}
the product set $P\times Q$ consisting of the pairs $(p,q)$ such that $\vphi(p)=\psi(q)$; we
denote it by $P\times_R Q$. Definition~\sref{I.3.2.1} of the product
of $S$-preschemes can be interpreted, with the notation of \sref{I.3.4.1},
via the formula
\[
\label{I.3.4.2.1}
  (X\times_S Y)(T)=X(T)\times_{S(T)}Y(T).
  \tag{3.4.2.1}
\]
the maps $X(T)\to S(T)$ and $Y(T)\to S(T)$ corresponding to the structure morphisms
$X\to S$ and $Y\to S$.
\end{env}

\begin{env}[3.4.3]
\label{I.3.4.3}
If we are given a prescheme $S$ and we consider only the $S$-preschemes and $S$-morphisms,
then we will denote by $X(T)_S$ the set $\Hom_S(T,X)$ of $S$-morphisms $T\to X$, and suppress
the subscript $S$ when there is no chance of confusion; we say that the elements of $X(T)_S$
are the \emph{points} (or \emph{$S$-points}, when there is a possibility of confusion)
\emph{of the $S$-prescheme $X$ with values in the $S$-prescheme $T$}. In particular, an
\emph{$S$-section} of $X$ is none other than a \emph{point of $X$ with values in $S$}. The
formula \sref{I.3.4.2.1} can then be written as
\[
\label{I.3.4.3.1}
  (X\times_S Y)(T)_S=X(T)_S\times Y(T)_S;
  \tag{3.4.3.1}
\]
more generally, if $Z$ is an $S$-prescheme, and $X$, $Y$, and $T$ are $Z$-preschemes (thus
\emph{ipso facto} $S$-preschemes), then we have
\[
\label{I.3.4.3.2}
  (X\times_Z Y)(T)_S=X(T)_S\times_{Z(T)_S}Y(T)_S.
  \tag{3.4.3.2}
\]

We note that, to show that a triple $(W,r,s)$ consisting of an $S$-prescheme $W$ and
$S$-morphisms $r:W\to X$ and $s:W\to Y$ is a product of $X$ and $Y$ (over $Z$), it suffices, by
definition, to check that, for \emph{each $S$-prescheme $T$}, the diagram
\[
  \xymatrix{
    W(T)_S\ar[r]^{r'}\ar[d]_{s'} &
    X(T)_S\ar[d]^{\vphi'}\\
    Y(T)_S\ar[r]^{\psi'} &
    Z(T)_S
  }
\]
makes $W(T)_S$ the fibre product of $X(T)_S$ and $Y(T)_S$ over $Z(T)_S$, where $r'$ and $s'$
correspond to $r$ and $s$, and $\vphi'$ and $\psi'$ to the structure morphisms $\vphi:X\to Z$ and $\psi:Y\to Z$.
\end{env}

\begin{env}[3.4.4]
\label{I.3.4.4}
When $T$ (resp. $S$) in the above is an affine scheme of ring $B$ (resp. $A$), we replace
$T$ (resp. $S$) by $B$ (resp. $A$) in the above notation, and we then call the elements of
$X(B)$ the \emph{points of $X$ with values in the ring $B$}, and the elements of $X(B)_A$ the
\emph{points of the $A$-prescheme $X$ with values in the $A$-algebra $B$}. We note that
$X(B)$ and $X(B)_A$ are \emph{covariant} functors in $B$. We similarly write $X(T)_A$ for the
set of points of the $A$-prescheme $X$ with values in the $A$-prescheme $T$.
\end{env}

\begin{env}[3.4.5]
\label{I.3.4.5}
Consider, in particular, the case where $T$ is of the form $\Spec(A)$, where $A$ is a
\emph{local} ring; the elements of $X(A)$ then correspond bijectively to \emph{local}
homomorphisms $\sh{O}_x\to A$ for $x\in X$ \sref{I.2.2.4}; we say that the
point $x$ of the underlying space of $X$ is the \emph{location}\footnote{\emph{[Trans.] We also say that
\emph{the geometric point lies over this $x$}.}} of the point of $X$ with values in $A$ to which it corresponds.

More specifically, we define the \emph{geometric points} of a prescheme $S$ to be the \emph{points of
$X$ with values in a field $K$}: the data of such a point is equivalent to the data of its
\oldpage[I]{113}
location $x$ in the underlying subspace of $X$, and of an \emph{extension} $K$ of $\kres(x)$;
$K$ will be called the \emph{field of values} of the corresponding geometric point, and we
say that this geometric point is \emph{located at $x$}. We also define a map $X(K)\to X$,
sending a geometric point with values in $K$ to its location.

If $S'=\Spec(K)$ is an $S$-prescheme (in other words, if $K$ is considered as an extension
of the residue field $\kres(s)$, where $s\in S$), and if $X$ is an $S$-prescheme, then an
element of $X(K)_S$, or, as we say, a \emph{geometric point of $X$ lying over $s$ with values
in $K$}, consists of the data of a $\kres(s)$-monomorphism from the residue field $\kres(x)$
to $K$, where $x$ is a point of $X$ \emph{lying over $s$} (therefore $\kres(x)$ is an
extension of $\kres(s)$).

In particular, if $S=\Spec(K)=\{\xi\}$, \emph{then the geometric points of $X$ with values in
$K$ can be identified with the points $x\in X$ such that $\kres(x)=K$}; we say that these latter
points are the \emph{$K$-rational points of the $K$-prescheme $X$}; if $K'$ is an extension
of $K$, then the geometric points of $X$ with values in $K'$ bijectively correspond to the
$K'$-rational points of $X'=X_{(K')}$ \sref{I.3.3.14}.
\end{env}

\begin{lemma}[3.4.6]
\label{I.3.4.6}
Let $X_i$ ($1\leq i\leq n$) be $S$-preschemes, $s$ a point of $S$, and $x_i$
($1\leq i\leq n$) points of $X_i$ lying over $s$. Then there exists an extension
$K$ of $\kres(s)$ and a geometric point of the product
$Y=X_1\times_S X_2\times_S\cdots\times_S X_n$, with values in $K$, whose projections to
the $X_i$ are localized at the $x_i$.
\end{lemma}

\begin{proof}
There exist $\kres(s)$-monomorphisms $\kres(x_i)\to K$, all in the same extension $K$ of $\kres(s)$ (Bourbaki, \emph{Alg.}, chap.~V, \textsection4, prop.~2).
The compositions $\kres(s)\to\kres(x_i)\to K$ are all identical, and so the morphisms $\Spec(K)\to X_i$ corresponding to the $\kres(x_i)\to K$ are all $S$-morphisms, and we thus conclude that they define a unique morphism $\Spec(K)\to Y$.
If $y$ is the corresponding point of $Y$, it is clear that its projection in each of the $X_i$ is $x_i$.
\end{proof}

\begin{proposition}[3.4.7]
\label{I.3.4.7}
Let $X_i$ ($1\leq i\leq n$) be $S$-preschemes, and, for each index $i$, let $x_i$ be a point of $X_i$.
For there to exist a point $y$ of $Y=X_1\times_S X_2\times\ldots\times_S X_n$ whose image is $x_i$ under the $i$th projection for each $1\leq i\leq n$, it is necessary and sufficient that the $x_i$ all lie above the same point $s$ of $S$.
\end{proposition}

\begin{proof}
The condition is evidently necessary; Lemma~\sref{I.3.4.6} proves that it is sufficient.
\end{proof}

In other words, if we denote by $(X)$ the underlying set of $X$, we see that we have a canonical \emph{surjective} function $(X\times_S Y)\to(X)\times_{(S)}(Y)$; we must point out that this function \emph{is not injective} in general; in other words, \emph{there can exist multiple distinct points $z$ in $X\times_S Y$ that have the same projections $x\in X$ and $y\in Y$}; we have already seen this when $S$, $X$, and $Y$ are prime spectra of fields $k$, $K$, and $K'$ (respectively), since the tensor product $K\otimes_k K'$ has, in general, multiple distinct prime ideals (cf.~\sref{I.3.4.9}).

\begin{corollary}[3.4.8]
\label{I.3.4.8}
Let $f:X\to Y$ be an $S$-morphism, and $f_{(S')}:X_{(S')}\to Y_{(S')}$ the $S'$-morphism induced by $f$ by an extension $S'\to S$ of the base prescheme.
Let $p$ (\emph{resp. $q$}) be the projection $X_{(S')}\to X$ (\emph{resp. $Y_{(S')}\to Y$}); for every subset $M$ of $X$, we have
\[
  q^{-1}(f(M))=f_{(S')}(p^{-1}(M)).
\]
\end{corollary}

\oldpage[I]{114}
\begin{proof}
Indeed \sref{I.3.3.11}, $X_{(S')}$ can be identified with the product $X\times_Y Y_{(S')}$ thanks to the commutative diagram
\[
  \xymatrix{
    X\ar[d]_f &
    X_{(S')}\ar[l]_p\ar[d]^{f_{(S')}}\\
    Y &
    Y_{(S')}\ar[l]_q
  }
\]
By \sref{I.3.4.7}, the equation $q(y')=f(x)$ for $x\in M$ and $y'\in Y_{(S)}$ is equivalent to the existence of some $x'\in X_{(S')}$ such that $p(x')=x$ and $f_{(S')}(x')=y'$, whence the corollary.
\end{proof}

Lemma~\sref{I.3.4.6} can be made clearer in the following manner:
\begin{proposition}[3.4.9]
\label{I.3.4.9}
Let $X$ and $Y$ be $S$-preschemes, $x$ a point of $X$, and $y$ a point of $Y$, with both $x$ and $y$ lying above the same point $s\in S$.
The set of points of $X\times_S Y$ with projections $x$ and $y$ is in bijective correspondence with the set of \unsure{types of extensions} composed of $\kres(x)$ and $\kres(y)$ considered as extensions of $\kres(s)$ \emph{(Bourbaki, \emph{Alg.}, chap.~VIII, \textsection8, prop.~2)}.
\end{proposition}

\begin{proof}
Let $p$ (resp. $q$) be the projection from $X\times_S Y$ to $X$ (resp. $Y$), and $E$ the subspace $p^{-1}(x)\cap q^{-1}(y)$ of the underlying space of $X\times_S Y$.
First, note that the morphisms $\Spec(\kres(x))\to S$ and $\Spec(\kres(y))\to S$ factor as $\Spec(\kres(x))\to\Spec(\kres(s))\to S$ and $\Spec(\kres(y))\to\Spec(\kres(s))\to S$; since $\Spec(\kres(s))\to S$ is a monomorphism \sref{I.2.4.7}, it follows from \sref{I.3.2.4} that we have
\[
  P=\Spec(\kres(x))\times_S\Spec(\kres(y))=\Spec(\kres(x))\times_{\Spec(\kres(s))}\Spec(\kres(y))=\Spec(\kres(x)\otimes_{\kres(s)}\kres(y)).
\]
We will define two maps, $\alpha:P_0\to E$ and $\beta:E\to P_0$, inverse to one another (where $P_0$ denotes the underlying set of the prescheme $P$).
If $i:\Spec(\kres(x))\to X$ and $j:\Spec(\kres(y))\to Y$ are the canonical morphisms \sref{I.2.4.5}, we take $\alpha$ to be the map of underlying spaces corresponding to the morphism $i\times_S j$.
On the other hand, every $z\in E$ defines, by hypothesis, two $\kres(s)$-monomorphisms, $\kres(x)\to\kres(z)$ and $\kres(y)\to\kres(z)$, and thus a $\kres(s)$-monomorphism $\kres(x)\otimes_{\kres(s)}\kres(y)\to\kres(z)$, and thus a morphism $\Spec(\kres(z))\to P$; $\beta(z)$ will be the image of $z$ in $P_0$ under this morphism.
The verification of the fact that $\alpha\circ\beta$ and $\beta\circ\alpha$ are the identity maps follows from \sref{I.2.4.5} and the definition of the product \sref{I.3.2.1}.
Finally, we know that $P_0$ is in bijective correspondence with the set of \unsure{types of extensions} composed of $\kres(x)$ and $\kres(y)$ (Bourbaki, \emph{Alg.}, chap.~VIII, \textsection8, prop.~1).
\end{proof}

\subsection{Surjections and injections}
\label{subsection:I.3.5}

\begin{env}[3.5.1]
\label{I.3.5.1}
In a general sense, consider a property $\textbf{P}$ of morphisms of preschemes, and the following two propositions:
\begin{enumerate}
  \item[(i)] If $f:X\to X'$ and $g:Y\to Y'$ are $S$-morphisms that have property $\textbf{P}$, then $f\times_S g$ also has property $\textbf{P}$.
  \item[(ii)] If $f:X\to Y$ is an $S$-morphism that has property $\textbf{P}$, then every $S'$-morphism $f_{(S')}:X_{(S')}\to Y_{(S')}$, induced by $f$ by an extension of the base prescheme, also has property $\textbf{P}$.
\end{enumerate}

Since $f_{(S')}=f\times_S 1_{S'}$, we see that, if, for every prescheme $X$, the \emph{identity} $1_X$ has property $\textbf{P}$, then (i) implies (ii); on the other hand, since $f\times_S g$ is the composite morphism
\[
  X\times_S Y\xrightarrow{f\times1_Y}X'\times_S Y\xrightarrow{1_{X'}\times g}X'\times_S Y',
\]
we see that, if the \emph{composition} of two morphisms has property $\textbf{P}$, then so does the product $f\times_S g$, and so (ii) implies (i).
\end{env}

A first application of this remark is
\begin{proposition}[3.5.2]
\label{I.3.5.2}
\medskip\noindent
\begin{enumerate}
  \item[{\rm(i)}] If $f:X\to X'$ and $g:Y\to Y'$ are surjective $S$-morphisms, then $f\times_S g$ is surjective.
  \item[{\rm(ii)}] If $f:X\to Y$ is a surjective $S$-morphism, then $f_{(S')}$ is surjective for every extension $S'$ of the base prescheme.
\end{enumerate}
\end{proposition}

\begin{proof}
The composition of any two surjections being a surjection, it suffices to prove (ii); but this proposition follows immediately from \sref{I.3.4.8} applied to $M=X$.
\end{proof}

\begin{proposition}[3.5.3]
\label{I.3.5.3}
For a morphism $f:X\to Y$ to be surjective, it is necessary and sufficient that, for every field $K$ and every morphism $\Spec(K)\to Y$, there exist an extension $K'$ of $K$ and a morphism $\Spec(K')\to X$ that make the following diagram commute:
\[
  \xymatrix{
    X\ar[d]_f &
    \Spec(K')\ar[l]\ar[d]\\
    Y &
    \Spec(K).\ar[l]
  }
\]
\end{proposition}

\begin{proof}
The condition is sufficient because, for all $y\in Y$, it suffices to apply it to a morphism $\Spec(K)\to Y$ corresponding to a monomorphism $\kres(y)\to K$, with $K$ being an extension of $\kres(y)$ \sref{I.2.4.6}.
Conversely, suppose that $f$ is surjective, and let $y\in Y$ be the image of the unique point of $\Spec(K)$; there exists some $x\in X$ such that $f(x)=y$; we will consider the corresponding monomorphism $\kres(y)\to\kres(x)$ \sref{I.2.2.1}; it then suffices to take $K'$ to be the extension of $\kres(y)$ such that there exist $\kres(y)$-monomorphisms from $\kres(x)$ and $K$ to $K'$ (Bourbaki, \emph{Alg.}, chap.~V, §4, prop.~2); the morphism $\Spec(K')\to X$ corresponding to $\kres(x)\to K'$ is exactly that for which we are searching.
\end{proof}

With the language introduced in \sref{I.3.4.5}, we can say that \emph{every geometric point of $Y$ with values in $K$ comes from a geometric point of $X$ with values in an extension of $K$}.

\begin{definition}[3.5.4]
\label{I.3.5.4}
We say that a morphism $f:X\to Y$ of preschemes is \emph{universally injective}, or a \emph{radicial morphism}, if, for every field $K$, the corresponding map $X(K)\to Y(K)$ is injective.
\end{definition}

It follows also from the definitions that every \emph{monomorphism of preschemes} (T,~1.1) is radicial.

\begin{env}[3.5.5]
\label{I.3.5.5}
For a morphism $f:X\to Y$ to be radicial, it suffices that the condition of Definition~\sref{I.3.5.4} hold for every \emph{algebraically closed} field.
In fact, if $K$ is an arbitrary field, and $K'$ an algebraically-closed extension of $K$, then the diagram
\[
  \xymatrix{
    X(K)\ar[r]^\alpha\ar[d]_\vphi &
    Y(K)\ar[d]^{\vphi'}\\
    X(K')\ar[r]^{\alpha'} &
    Y(K')
  }
\]
commutes, where $\vphi$ and $\vphi'$ come from the morphism $\Spec(K')\to\Spec(K)$, and $\alpha$ and $\alpha'$ corresponding to $f$.
However, $\vphi$ is injective, and so too is $\alpha'$, by hypothesis; hence $\alpha$ is necessarily injective.
\end{env}

\begin{proposition}[3.5.6]
\label{I.3.5.6}
Let $f:X\to Y$ and $g:Y\to Z$ be two morphisms of preschemes.
\begin{enumerate}
  \item[{\rm(i)}] If $f$ and $g$ are radicial, then so is $g\circ f$.
  \item[{\rm(ii)}] Conversely, if $g\circ f$ is radicial, then so is $f$.
\end{enumerate}
\end{proposition}

\begin{proof}
Taking into account Definition~\sref{I.3.5.4}, the proposition reduces to the corresponding claims for the maps $X(K)\to Y(K)\to Z(K)$, and these claims are evident.
\end{proof}

\begin{proposition}[3.5.7]
\label{I.3.5.7}
\medskip\noindent
\begin{enumerate}
  \item[{\rm(i)}] If the $S$-morphisms $f:X\to X'$ and $g:X\to X'$ are radicial, then so is $f\times_S g$.
  \item[{\rm(ii)}] If the $S$-morphism $f:X\to Y$ is radicial, then so is $f_{(S')}:X_{(S')}\to Y_{(S')}$ for every extension $S'\to S$ of the base prescheme.
\end{enumerate}
\end{proposition}

\begin{proof}
Given \sref{I.3.5.1}, it suffices to prove (i).
We have seen \sref{I.3.4.2.1} that
\[
  (X\times_S Y)(K)=X(K)\times_{S(K)}Y(K),
\]
\[
  (X'\times_S Y')(K)=X'(K)\times_{S(K)}Y'(K),
\]
with the map $(X\times_S Y)(K)\to(X'\times_S Y')(K)$ corresponding to $f\times_S g$ thus being identified with $(u,v)\to(f\circ u,g\circ v)$, and the proposition then follows.
\end{proof}

\begin{proposition}[3.5.8]
\label{I.3.5.8}
For a morphism $f=(\psi,\theta):X\to Y$ to be radicial, it is necessary and sufficient for $\psi$ to be injective and for the monomorphism $\theta^x:\kres(\psi(x))\to\kres(x)$ to make $\kres(x)$ a radicial extension of $\kres(\psi(x))$ for every $x\in X$.
\end{proposition}

\begin{proof}
We suppose that $f$ is radicial and first show that the equation $\psi(x_1)=\psi(x_2)=y$ necessarily implies that $x_1=x_2$.
Indeed, there exists a field $K$, and an extension of $\kres(y)$, along with $\kres(y)$-monomorphisms $\kres(x_1)\to K$ and $\kres(x_2)\to K$ (Bourbaki, \emph{Alg.}, chap.~V, §4, prop.~2); the corresponding morphisms $u_1:\Spec(K)\to X$ and $u_2:\Spec(K)\to X$ are then such that $f\circ u_1=f\circ u_2$, and so $u_1=u_2$ by hypothesis, and this implies, in particular, that $x_1=x_2$.
We now consider $\kres(x)$ as the extension of $\kres(\psi(x))$ by means of $\theta^x$: if $\kres(x)$ is not a radicial algebraically-closed extension, then there exist two distinct $\kres(\psi(x))$-monomorphisms from $\kres(x)$ to an algebraically-closed extension $K$ of $\kres(\psi(x))$, and the two corresponding morphisms $\Spec(K)\to X$ would contradict the hypothesis.
Conversely, taking \sref{I.2.4.6} into account, it is immediate that the conditions stated are sufficient for $f$ to be radicial.
\end{proof}

\begin{corollary}[3.5.9]
\label{I.3.5.9}
If $A$ is a ring, and $S$ is a multiplicative set of $A$, then the canonical morphism $\Spec(S^{-1}A)\to\Spec(A)$ is radicial.
\end{corollary}

\begin{proof}
Indeed, this morphism is a monomorphism \sref{I.1.6.2}.
\end{proof}

\begin{corollary}[3.5.10]
\label{I.3.5.10}
Let $f:X\to Y$ be a radicial morphism, $g:Y'\to Y$ a morphism, and $X'=X_{(Y')}=X\times_Y Y'$.
Then the radicial morphism $f_{(Y')}$ \sref{I.3.5.7}[ii] is a bijection from the underlying space of $X$ to $g^{-1}(f(X))$; further, for every field $K$, the set $X'(K)$ can be identified with the subset of $Y'(K)$ given by the inverse image of the map $Y'(K)\to Y(K)$ (corresponding to $g$) from the subset $X(K)$ of $Y(K)$.
\end{corollary}

\begin{proof}
The first claim follows from \sref{I.3.5.8} and \sref{I.3.4.8}; the second, from the commutativity of the following diagram:\oldpage[I]{117}
\[
  \xymatrix{
    X'(K)\ar[r]\ar[d] &
    Y'(K)\ar[d]\\
    X(K)\ar[r] &
    Y(K)
  }
\]
\end{proof}

\begin{remark}[3.5.11]
\label{I.3.5.11}
We say that a morphism $f=(\psi,\theta)$ of preschemes is \emph{injective} if the map $\psi$ is injective.
For a morphism $f=(\psi,\theta):X\to Y$ to be radicial, it is necessary and sufficient that, for every morphism $Y'\to Y$, the morphism $f_{(Y')}:X_{(Y')}\to Y'$ be injective (which justifies the terminology of a \emph{universally injective} morphism).
In fact, the condition is necessary by \sref{I.3.5.7}[ii] and \sref{I.3.5.8}.
Conversely, the condition implies that $\psi$ is injective; if, for some $x\in X$, the monomorphism $\theta^x:\kres(\psi(x))\to\kres(x)$ were not radicial, then there would be an extension $K$ of $\kres(\psi(x))$, and two distinct morphisms $\Spec(K)\to X$ corresponding to the same morphism $\Spec(K)\to Y$ \sref{I.3.5.8}.
But then, setting $Y'=\Spec(K)$, there would be two distinct $Y'$-sections of $X_{(Y')}$ \sref{I.3.3.14}, which contradicts the hypothesis that $f_{(Y')}$ is injective.
\end{remark}

\subsection{Fibres}
\label{subsection:I.3.6}

\begin{proposition}[3.6.1]
\label{I.3.6.1}
Let $f:X\to Y$ be a morphism, $y$ a point of $Y$, and $\mathfrak{a}_y$ an ideal of definition for $\sh{O}_y$ for the $\mathfrak{m}_y$-preadic topology.
Then the projection $p:X\times_Y\Spec(\sh{O}_y/\mathfrak{a}_y)\to X$ is a homeomorphism from the underlying space of the prescheme $X\times_Y\Spec(\sh{O}_y/\mathfrak{a}_y)$ to the fibre $f^{-1}(y)$ equipped with the topology induced from that of the underlying space of $X$.
\end{proposition}

\begin{proof}
Since $\Spec(\sh{O}_y/\mathfrak{a}_y)\to Y$ is radicial (\sref{I.3.5.4} and \sref{I.2.4.7}), since $\Spec(\sh{O}_y/\mathfrak{a}_y)$ is a single point, and since the ideal $\mathfrak{m}_y/\mathfrak{a}_y$ is nilpotent by hypothesis \sref{I.1.1.12}, we already know (\sref{I.3.5.10} and \sref{I.3.3.4}) that $p$ identifies, as \emph{sets}, the underlying space of $X\times_Y\Spec(\sh{O}_y/\mathfrak{a}_y)$ with $f^{-1}(y)$; everything reduces to proving that $p$ is a homeomorphism.
By \sref{I.3.2.7}, the question is local on $X$ and $Y$, and so we can suppose that $X=\Spec(B)$ and $Y=\Spec(A)$, with $B$ being an $A$-algebra.
The morphism $p$ then corresponds to the homomorphism $1\otimes\vphi:B\to B\otimes_A A'$, where $A'=A_y/\mathfrak{a}_y$ and $\vphi$ is the canonical map from $A$ to $A'$.
Then every element of $B\otimes_A A'$ can be written as
\[
  \sum_i b_i\otimes\vphi(a_i)/\vphi(s)=\left(\sum_i(a_ib_i\otimes1)\right)(1\otimes\vphi(s))^{-1},
\]
where $s\not\in\mathfrak{j}_y$, and Proposition~\sref{I.1.2.4} applies.
\end{proof}

\begin{env}[3.6.2]
\label{I.3.6.2}
Throughout the rest of this treatise, whenever we consider a fibre $f^{-1}(y)$ of a morphism as having the structure of a $\kres(y)$-prescheme, \emph{it will always be the prescheme obtained by transporting the structure of $X\times_Y\Spec(\kres(y))$ by the projection to $X$}.
We will also write this (latter) product as $X\times_Y\kres(y)$, or $X\otimes_{\sh{O}_Y}\kres(y)$; more generally, if $B$ is an $\sh{O}_y$-algebra, we will denote by $X\times_Y B$ or $X\otimes_{\sh{O}_Y}B$ the product $X\times_Y\Spec(B)$.
\end{env}

With the preceding convention, it follows from \sref{I.3.5.10} that the points of $X$ with values in an extension $K$ of $\kres(y)$ are identified with the \emph{points of $f^{-1}(y)$ with values in $K$}.

\begin{env}[3.6.3]
\label{I.3.6.3}
Let $f:X\to Y$ and $g:Y\to Z$ be two morphisms, and $h=g\circ f$ their composition; for all $z\in Z$, the fibre $h^{-1}(z)$ is a prescheme isomorphic to
\[
  X\times_Z\Spec(\kres(z))=(X\times_Y Y)\times_Z\Spec(\kres(z))=X\times_Y g^{-1}(z).
\]
In\oldpage[I]{118} particular, if $U$ is an open subset of $X$, then the prescheme induced on $U\cap f^{-1}(y)$ by the prescheme $f^{-1}(y)$ is isomorphic to $f^{-1}_U(y)$ ($f_U$ being the restriction of $f$ to $U$),
\end{env}

\begin{proposition}[3.6.4]
\label{I.3.6.4}
\emph{(Transitivity of fibres)}
Let $f:X\to Y$ and $g:Y'\to Y$ be morphisms; let $X'=X\times_Y Y'=X_{(Y')}$ and $f'=f_{(Y')}:X'\to Y'$.
For every $y'\in Y'$, if we let $y=g(y')$, then the prescheme $f'^{-1}(y')$ is isomorphic to $f^{-1}(y)\otimes_{\kres(y)}\kres(y')$.
\end{proposition}

\begin{proof}
Indeed, it suffices to remark that the two preschemes $(X\otimes_Y\kres(y))\otimes_{\kres(y)}\kres(y')$ and $(X\times_Y Y')\otimes_{Y'}\kres(y')$ are both canonically isomorphic to $X\times_Y\Spec(\kres(y'))$ by \sref{I.3.3.9.1}.
\end{proof}

In particular, if $V$ is an open neighborhood of $y$ in $Y$, and we denote by $f_V$ the restriction of $f$ to the induced prescheme on $f^{-1}(V)$, then the preschemes $f^{-1}(y)$ and $f^{-1}_V(y)$ are canonically identified.

\begin{proposition}[3.6.5]
\label{I.3.6.5}
Let $f:X\to Y$ be a morphism, $y$ a point of $Y$, $Z$ the local prescheme $\Spec(\sh{O}_y)$, and $p=(\psi,\theta)$ the projection $X\times_Y Z\to X$; then $p$ is a homeomorphism from the underlying space of $X\times_Y Z$ to the subspace $f^{-1}(Z)$ of $X$ (\emph{when the underlying space of $Z$ is identified with a subspace of $Y$, cf.~\sref{I.2.4.2}}), and, for all $t\in X\times_Y Z$, letting $z=\psi(t)$, we have that $\theta_t^\sharp$ is an isomorphism from $\sh{O}_x$ to $\sh{O}_t$.
\end{proposition}

\begin{proof}
Since $Z$ (identified as a subspace of $Y$) is contained inside every affine open containing $y$ \sref{I.2.4.2}, we can, as in \sref{I.3.6.1}, reduce to the case where $X=\Spec(A)$ and $Y=\Spec(B)$ are affine schemes, with $A$ being a $B$-algebra.
Then $X\times_Y Z$ is the prime spectrum of $A\otimes_B B_y$, and this ring is canonically identified with $S^{-1}A$, where $S$ is the image of $B\setmin\mathfrak{j}_y$ in $A$ \sref[0]{0.1.5.2}; since $p$ then corresponds to the canonical homomorphism $A\to S^{-1}A$, the proposition follows from \sref{I.1.6.2}.
\end{proof}

\subsection{Application: reduction of a prescheme mod.~$\mathfrak{J}$}
\label{subsection:I.3.7}

\emph{This section, which makes use of notions and results from Chapter~I and Chapter~II, will not be used in what follows in this treatise, and is only intended for readers familiar with classical algebraic geometry}.

\begin{env}[3.7.1]
\label{I.3.7.1}
Let $A$ be a ring, $X$ an $A$-prescheme, and $\mathfrak{J}$ an ideal of $A$; then $X_0=X\otimes_A(A/\mathfrak{J})$ is an $(A/\mathfrak{J})$-prescheme, which we sometimes say is induced from $X$ by \emph{reduction} mod.~$\mathfrak{J}$.
\end{env}

\begin{env}
\label{I.3.7.2}
This terminology is used foremost when $A$ is a \emph{local ring} and $\mathfrak{J}$ its maximal ideal, in such a way that $X_0$ is a prescheme over the residue field $k=A/\mathfrak{J}$ of $A$.

When $A$ is also integral, with field of fractions $K$, we can consider the $K$-prescheme $X'=X\otimes_A K$.
By an abuse of language which we will not use, it has been said, up until now, that $X_0$ is \emph{induced by $X'$} by reduction mod.~$\mathfrak{J}$.
In the case where this language was used, $A$ was a local ring of dimension~$1$ (most often a discrete valuation ring) and it was implied (be it more or less explicitly) that the given $K$-prescheme $X'$ was a closed subprescheme of a $K$-prescheme $P'$ (in fact, a projective space \unsure{of the form} $\bb{P}_K^r$, cf. \sref[II]{II.4.1.1}), itself of the form $P'=P\otimes_A K$, where $P$ is a given $A$-prescheme (in fact, the $A$-scheme $\bb{P}_A^r$, with the notation of \sref[II]{II.4.1.1}).
In our language, the definition of $X_0$ in terms of $X'$ is formulated as follows:

We consider the affine scheme $Y=\Spec(A)$, formed of two points, the unique closed point
\oldpage[I]{119}
$y=\mathfrak{J}$ and the generic point $(0)$, the singleton set $U$ of the generic point being thus an open $U=\Spec(K)$ in $Y$.
If $X$ is an $A$-prescheme (or, in other words, a $Y$-prescheme), then $X\otimes_A K=X'$ is exactly the prescheme induced by $X$ on $\psi^{-1}(U)$, denoting by $\psi$ the structure morphism $X\to Y$.
In particular, if $\vphi$ is the structure morphism $P\to Y$, then a closed subprescheme $X'$ of $P'=\vphi^{-1}(U)$ is a (locally closed) subprescheme of $P$.
If $P$ is Noetherian (for example, if $A$ is Noetherian and $P$ is of finite type over $A$), then there exists a smaller closed subprescheme $X=\overline{X'}$ of $G$ that through which $X'$ factors \sref{I.9.5.10}, and $X'$ is the prescheme induced by $X$ on the open $\vphi^{-1}(U)\cap X$, and so is isomorphic to $X\otimes_A K$ \sref{I.9.5.10}.
\emph{The immersion of $X'$ into $P'=P\otimes_A K$ thus lets us canonically consider $X'$ as being of the form $X'=X\otimes_A K$, where $X$ is an $A$-prescheme.}
We can then consider the reduced mod.~$\mathfrak{J}$ prescheme $X_0=X\otimes_A k$, which is exactly the fibre $\psi^{-1}(y)$ of the closed point $y$.
Up until now, lacking the adequate terminology, we had avoided explicitly introducing the $A$-prescheme $X$.
One ought to, however, note that all the claims normally made about the ``reduced mod.~$\mathfrak{J}$'' prescheme  $X_0$ should be seen as consequences of more complicated claims about $X$ itself, and cannot be satisfactorily formulated or understood except by interpreting them as such.
It seems also that any hypotheses made on $X_0$ always reduce to hypotheses on $X$ itself (independent of the prior data of an immersion of $X'$ in $\bb{P}_K^r$), which lets us give more intrinsic statements.
\end{env}

\begin{env}[3.7.3]
\label{I.3.7.3}
Lastly, we draw attention to a very particular fact, which has undoubtedly contributed to slowing the conceptual clarification of the situation envisaged here: if $A$ is a discrete valuation ring, and if $X$ is \emph{proper} over $A$ (which is indeed the case if $X$ is a closed subprescheme of some $\bb{P}_A^r$, cf. \sref[II]{II.5.5.4}), then the points of $X$ with values in $A$ and the points of $X'$ with values in $k$ are in bijective correspondence \sref[II]{II.7.3.8}.
This is why we often believe that facts about $X'$ have been proved, when in reality we have proved facts about $X$, and these remain valuable (in this form) whenever we no longer suppose that the base local ring is of dimension~$1$.
\end{env}


\section{Subpreschemes and immersion morphisms}
\label{section-subpreschemes-and-immersion-morphisms}

\subsection{Subpreschemes}
\label{subsection-subpreschemes}

\begin{env}[4.1.1]
\label{1.4.1.1}
As the notion of a quasi-coherent sheaf \sref[0]{0.5.1.3} is local, a quasi-coherent $\OO_X$-module $\sh{F}$ over a prescheme $X$ can be defined by the condition that, for each affine open $V$ of $X$, $\sh{F}|V$ is isomorphic to the sheaf associated to a $\Gamma(V,\OO_X)$-module \sref{1.1.4.1}.
It is clear that over a prescheme $X$, the structure sheaf $\OO_X$ is quasi-coherent and that the kernels, cokernels, and images of homomorphisms of quasi-coherent $\OO_X$-modules, as well as inductive limits and direct sums of quasi-coherent $\OO_X$-modules, are also quasi-coherent (Theorem \sref{1.1.3.7} and Corollary \sref{1.1.3.9}).
\end{env}

\begin{prop}[4.1.2]
\label{1.4.1.2}
Let $X$ be a prescheme, $\sh{I}$ a quasi-coherent sheaf of ideals of $\OO_X$.
The support $Y$ of the sheaf $\OO_X/\sh{I}$ is then closed, and if we denote by $\OO_Y$ the restriction of $\OO_X/\sh{I}$ to $Y$, then $(Y,\OO_Y)$ is a prescheme.
\end{prop}

\begin{proof}
\label{proof-1.4.1.2}
\oldpage[I]{120}
It evidently suffices \sref{1.2.1.3} to consider the case where $X$ is an affine scheme, and to show that in this case $Y$ is closed in $X$ and is an \emph{affine scheme}.
Indeed, if $X=\Spec(A)$, then we have $\OO_X=\wt{A}$ and $\sh{I}=\wt{\fk{I}}$, where $\fk{I}$ is an ideal of $A$ \sref{1.1.4.1}; $Y$ is then equal to the closed subset $V(\fk{I})$ of $X$ and identifies with the prime spectrum of the ring $B=A/\fk{I}$ \sref{1.1.1.11};
in addition, if $\vphi$ is the canonical homomorphism $A\to B=A/\fk{I}$, then the direct image ${}^a\vphi_*(\wt{B})$ canonically identifies with the sheaf $\wt{A}/\wt{\fk{I}}=\OO_X/\sh{I}$ (Proposition \sref{1.1.6.3} and Corollary \sref{1.1.3.9}), which finishes the proof.
\end{proof}

We say that $(Y,\OO_Y)$ is the \emph{subprescheme} of $(X,\OO_X)$ \emph{defined by the sheaf of ideals $\sh{I}$}; this is a particular case of the more general notion of a \emph{subprescheme}:

\begin{defn}[4.1.3]
\label{1.4.1.3}
We say that a ringed space $(Y,\OO_Y)$ is a subprescheme of a prescheme $(X,\OO_X)$ if:
\begin{enumerate}
  \item[1st] $Y$ is a locally closed subspace of $X$;
  \item[2nd] if $U$ denotes the largest open subset of $X$ containing $Y$ such that $Y$ is closed in $U$ (\emph{equivalently}, the complement in $X$ of the boundary of $Y$ with respect to $\overline{Y}$), then $(Y,\OO_Y)$ is a subprescheme of $(U,\OO_X|U)$ defined by a quasi-coherent sheaf of ideals of $\OO_X|U$.
\end{enumerate}
We say that the subprescheme $(Y,\OO_Y)$ of $(X,\OO_X)$ is closed if $Y$ is closed in $X$ (in which case $U=X$).
\end{defn}

It follows immediately from this definition and Proposition \sref{1.4.1.2} that the closed subpreschemes of $X$ are in canonical \emph{bijective correspondence} with the quasi-coherent sheaf of ideals $\sh{J}$ of $\OO_X$, since if two such sheaves $\sh{J}$, $\sh{J}'$ have the same (closed) support $Y$ and if the restrictions of $\OO_X/\sh{J}$ and $\OO_X/\sh{J}'$ to $Y$ are \unsure{identical/the identity}, then we have $\sh{J}'=\sh{J}$.

\begin{env}[4.1.4]
\label{1.4.1.4}
Let $(Y,\OO_Y)$ be a subprescheme of $X$, $U$ the largest open subset of $X$ containing $Y$ and in which $Y$ is closed, $V$ an open subset of $X$ contained in $U$; then $V\cap Y$ is closed in $V$. In addition, if $Y$ is defined by the quasi-coherent sheaf of ideals $\sh{J}$ of $\OO_X|U$, then $\sh{J}|V$ is a quasi-coherent sheaf of ideals of $\OO_X|V$, and it is immediate that the prescheme induced by $Y$ on $Y\cap V$ is the closed subprescheme of $V$ defined by the sheaf of ideals $\sh{J}|V$.
Conversely:
\end{env}

\begin{prop}[4.1.5]
\label{1.4.1.5}
Let $(Y,\OO_Y)$ be a ringed space such that $Y$ is a subspace of $X$ and that there exists a cover $(V_\alpha)$ of $Y$ by open subsets of $X$ such that for each $\alpha$, $Y\cap V_\alpha$ is closed in $V_\alpha$ and the ringed space $(Y\cap V_\alpha,\OO_Y|(Y\cap V_\alpha))$ is a closed subprescheme of the prescheme induced on $V_\alpha$ by $X$.
Then $(Y,\OO_Y)$ is a subprescheme of $X$.
\end{prop}

\begin{proof}
\label{proof-1.4.1.5}
The hypotheses imply that $Y$ is locally closed in $X$ and that the largest open $U$ containing $Y$ in which $Y$ is closed contains all the $V_\alpha$; we can thus reduce to the case where $U=X$ and $Y$ is closed in $X$.
We then define a quasi-coherent sheaf of ideals $\sh{J}$ of $\OO_X$ by taking $\sh{J}|V_\alpha$ to be the sheaf of ideals of $\OO_X|V_\alpha$ which define the closed subprescheme $(Y\cap V_\alpha,\OO_Y|(Y\cap V_\alpha))$, and for each open subset $W$ of $X$ not intersecting $Y$, $\sh{J}|W=\OO_X|W$.
We check immediately according to Definition \sref{1.4.1.3} and \sref{1.4.1.4} that there exists a unique sheaf of ideals $\sh{J}$ satisfying these conditions and that define the closed subprescheme $(Y,\OO_Y)$.
\end{proof}

In particular, the \emph{induced} prescheme by $X$ on an \emph{open subset} of $X$ is a \emph{subprescheme} of $X$.

\begin{prop}[4.1.6]
\label{1.4.1.6}
A subprescheme (resp. a closed subprescheme) of a subprescheme
\oldpage[I]{121}
(resp. closed subprescheme) of $X$ canonically identifies with a subprescheme (resp. closed subprescheme) of $X$.
\end{prop}

\begin{proof}
\label{proof-1.4.1.6}
Since a locally closed subset of a locally closed subspace of $X$ is a locally closed subspace of $X$, it is clear \sref{1.4.1.5} that the question is local and that we can thus suppose that $X$ is affine; the proposition then follows from the canonical identification of $A/\fk{J}'$ and $(A/\fk{J})/(\fk{J}'/\fk{J})$ when $\fk{J}$, $\fk{J}'$ are two ideals of a ring $A$ such that $\fk{J}\subset\fk{J}'$.
\end{proof}

We will always make the previous identification.
\begin{env}[4.1.7]
\label{1.4.1.7}
Let $Y$ be a subprescheme of a prescheme $X$, and denote by $\psi$ the canonical injection $Y\to X$ of the \emph{underlying subspaces}; we know that the inverse image $\psi^*(\OO_X)$ is the restriction $\OO_X|Y$ \sref[0]{0.3.7.1}.
If, for each $y\in Y$, we denote by $\omega_y$ the canonical homomorphism $(\OO_X)_y\to(\OO_Y)_y$, then these homomorphisms are the restrictions to stalks of a \emph{surjective} homomorphism $\omega$ of sheaves of rings $\OO_X|Y\to\OO_Y$: indeed, it suffices to check locally on $Y$, that is to say, we can suppose that $X$ is affine and that the subprescheme $Y$ is closed; if in this case $\sh{I}$ is the sheaf of ideals in $\OO_X$ which defines $Y$, then the $\omega_y$ are none other than the restriction to stalks of the homomorphism $\OO_X|Y\to(\OO_X/\sh{I})|Y$.
We have thus defined a \emph{monomorphism of ringed spaces} \sref[0]{0.4.1.1} $j=(\psi,\omega^\flat)$ which is evidently a morphism $Y\to X$ of preschemes \sref{1.2.2.1}, and we call this the \emph{canonical injection morphism}.

If $f:X\to Z$ is a morphism, we then say that the composite morphism $Y\xrightarrow{j}X\xrightarrow{f}Z$ is the \emph{restriction} of $f$ to the subprescheme $Y$.
\end{env}

\begin{env}[4.1.8]
\label{1.4.1.8}
Conforming to the general definitions (T, I, 1.1), we will say that a morphism of preschemes $f:Z\to X$ is \unsure{\emph{majorized}} by the injection morphism $j:Y\to X$ of a subprescheme $Y$ of $X$ if $f$ factors as $Z\xrightarrow{g}Y\xrightarrow{j}X$, where $g$ is a morphism of preschemes; $g$ is necessarily \emph{unique} since $j$ is a monomorphism.
\end{env}

\begin{prop}[4.1.9]
\label{1.4.1.9}
For a morphism $f:Z\to X$ to be majorized by an injection morphism $j:Y\to X$, it is necessary and sufficient that $f(Z)\subset Y$ and that, for all $z\in Z$, letting $y=f(z)$, the homomorphism $(\OO_X)_y\to\OO_z$ corresponding to $f$ factors as $(\OO_Z)_y\to(\OO_Y)_y\to\OO_z$ (\emph{or equivalently, that} the kernel of $(\OO_X)_y\to\OO_z$ contains the kernel of $(\OO_X)_y\to(\OO_Y)_y$).
\end{prop}

\begin{proof}
\label{proof-1.4.1.9}
The conditions are evidently necessary.
To see that they are sufficient, we can reduce to the case where $Y$ is a \emph{closed} subprescheme of $X$, by replacing if needed $X$ by an open $U$ such that $Y$ is closed in $U$ \sref{1.4.1.3}; $Y$ is then defined by a quasi-coherent sheaf of ideals $\sh{I}$ of $\OO_X$.
Set $f=(\psi,\theta)$, and let $\sh{J}$ be the sheaf of ideals of $\psi^*(\OO_X)$, kernel of $\theta^\sharp:\psi^*(\OO_X)\to\OO_Z$; considering the properties of the functor $\psi^*$ \sref[0]{0.3.7.2}, the hypotheses emply that for each $z\in Z$, we have $(\psi^*(\sh{I}))_z\subset\sh{J}_z$, and as a result $\psi^*(\sh{I})\subset\sh{J}$.
Thus $\theta^\sharp$ factors as
\[
  \psi^*(\OO_X)\to\psi^*(\OO_X)/\psi^*(\sh{I})=\psi^*(\OO_X/\sh{I})\xrightarrow{\omega}\OO_Z,
\]
the first arrow being the canonical homomorphism.
Let $\psi'$ be the continuous map $Z\to Y$ coinciding with $\psi$; it is clear that we have ${\psi'}^*(\OO_Y)=\psi^*(\OO_X/\sh{J})$; on the other hand, $\omega$ is evidently a local homomorphism, so $g=(\psi',\omega^\flat)$ is a morphism
\oldpage[I]{122}
of preschemes $Z\to Y$ \sref{1.2.2.1}, which according to the above is such that $f=j\circ g$, hence the proposition.
\end{proof}

\begin{cor}[4.1.10]
\label{1.4.1.10}
For an injection morphism $Z\to X$ to be majorized by the injection morphism $Y\to X$, it is necessary and sufficient for $Z$ to be a subprescheme of $Y$.
\end{cor}

We then write $Z\leqslant Y$, and this condition is evidently an \emph{ordering} on the set of subpreschemes of $X$.

\subsection{Immersion morphisms}
\label{subsection-immersion-morphisms}

\begin{defn}[4.2.1]
\label{1.4.2.1}
We say that a morphism $f:Y\to X$ is an immersion (resp. a closed immersion, an open immersion) if it factors as $Y\xrightarrow{g}Z\xrightarrow{j}X$, where $g$ is an isomorphism, $Z$ is a subprescheme of $X$ (resp. a closed subprescheme, a subprescheme induced by an open set) and $j$ is the injection morphism.
\end{defn}

The subprescheme $Z$ and the isomorphism $g$ are then determined in a \emph{unique} way, since if $Z'$ is a second subprescheme of $X$, $j'$ the injection $Z'\to X$, and $g'$ an isomorphism $Y\to Z'$ such that $j\circ g=j'\circ g'$, then we have $j'=j\circ g\circ{g'}^{-1}$, hence $Z'\leqslant Z$ \sref{1.4.1.10}, and we similarly show that $Z\leqslant Z'$, hence $Z'=Z$, and as $j$ is a monomorphism of preschemes, $g'=g$.

We say that $f=j\circ g$ is the \emph{canonical factorization} of the immersion $f$, and the subprescheme $Z$ and the isomorphism $g$ are those \emph{associated to $f$}.

It is clear that an immersion is a \emph{monomorphism} of preschemes \sref{1.4.1.7} and \emph{a fortiori} a \emph{radicial} morphism \sref{1.3.5.4}.

\begin{prop}[4.2.2]
\label{1.4.2.2}
\medskip\noindent
\begin{enumerate}[label={\rm(\alph*)}]
  \item For a morphism $f=(\psi,\theta):Y\to X$ to be an open immersion, it is necessary and sufficient for $\psi$ to be a homeomorphism between $Y$ and an open subset of $X$, and that for all $y\in Y$, the homomorphism $\theta_y^\sharp:\OO_{\psi(y)}\to\OO_y$ is bijective.
  \item For a morphism $f=(\psi,\theta):Y\to X$ to be an immersion (resp. a closed immersion), it is necessary and sufficient for $\psi$ to be a homeomorphism between $Y$ and a locally closed (resp. closed) subset of $X$, and that for all $y\in Y$, the homomorphism $\theta_y^\sharp:\OO_{\psi(y)}\to\OO_y$ is surjective.
\end{enumerate}
\end{prop}

\begin{proof}
\label{proof-1.4.2.2}
\medskip\noindent
\begin{enumerate}[label=(\alph*)]
  \item The conditions are evidently necessary.
    Conversely, if they are satisfied, then it is clear that $\theta^\sharp$ is an isomorphism from $\OO_Y$ to $\psi^*(\OO_X)$, and $\psi^*(\OO_X)$ is the sheaf induced by ``transport of structure'' via $\psi^{-1}$ from $\OO_X|\psi(Y)$; hence the conclusion.
  \item The conditions are evidently necessary---we prove that they are sufficient.
    Consider first the particular case where we suppose that $X$ is an affine scheme and that $Z=\psi(Y)$ is \emph{closed} in $X$.
    We then know \sref[0]{0.3.4.6} that $\psi_*(\OO_Y)$ has $Z$ for its support and that if we denote by $\OO_Z'$ its restriction ot $Z$, then the ringed space $(Z,\OO_Z')$ is induced from $(Y,\OO_Y)$ by transport of structure via the homeomorphism $\psi$, considered as a map from $Y$ to $Z$.
    Let us show that $f_*(\OO_Y)=\psi_*(\OO_Y)$ is a \emph{quasi-coherent} $\OO_X$-module.
    Indeed, for all $x\not\in Z$, $\psi_*(\OO_Y)$ restricted to a suitable neighborhood of $x$ is zero.
    On the contrary, if $z\in Z$, then we have $x=\psi(y)$ for a well-defined  $y\in Y$; let $V$ be an open affine neighborhood of $y$ in $Y$; $\psi(V)$ is then open in $Z$, so \unsure{trace} over $Z$ of an open subset $U$ of $X$, and the restriction of $U$ to $\psi_*(\OO_Y)$ is identical to the restriction of $U$ to the direct image
\oldpage[I]{123}
    $(\psi_V)_*(\OO_Y|V)$, where $\psi_V$ is the restriction of $\psi$ to $V$.
    The restriction to $(V,\OO_Y|V)$ of the morphism $(\psi,\theta)$ is a morphism from this prescheme to $(X,\OO_X)$, and as a result is of the form $({}^a\vphi,\wt{\vphi})$, where $\vphi$ is the homomorphism from the ring $A=\Gamma(X,\OO_X)$ to the ring $\Gamma(V,\OO_Y)$ \sref{1.1.7.3}; we conclude that $(\psi_V)_*(\OO_Y|V)$ is a quasi-coherent $\OO_X$-module \sref{1.1.6.3}, which proves our assertion, due to the local nature of quasi-coherent sheaves.
    In addition, the hypothesis that $\psi$ is a homeomorphism implies \sref[0]{0.3.4.5} that for all $y\in Y$, $\psi_y$ is an isomorphism $(\psi_*(\OO_Y))_{\psi(y)}\to\OO_y$; as the diagram
    \[
      \xymatrix{
        \OO_{\psi(y)}\ar[r]^{\theta_{\psi(y)}}\ar[d]_{\psi_y\circ\alpha_{\psi(y)}} &
        (\psi_*(\OO_Y))_{\psi(y)}\ar[d]^{\psi_y}\\
        (\psi^*(\OO_X))_y\ar[r]^{\theta_y^\sharp} &
        \OO_y
      }
    \]
    is commutative and the vertical arrows are the isomorphisms \sref[0]{0.3.7.2}, the hypothesis that $\theta_y^\sharp$ is surjective implies that so is $\theta_{\psi(y)}$.
    As the support of $\psi_*(\OO_Y)$ is $Z=\psi(Y)$, $\theta$ is a \emph{surjective} homomorphism from $\OO_X=\wt{A}$ to the quasi-coherent $\OO_X$-module $f_*(\OO_Y)$.
    As a result, there exists a unique isomorphism $\omega$ from a sheaf quotient $\wt{A}/\wt{\fk{J}}$ ($\fk{J}$ an ideal of $A$) to $f_*(\OO_Y)$ which when composed with the canonical homomorphism $\wt{A}\to\wt{A}/\wt{\fk{J}}$ gives $\theta$ \sref{1.1.3.8}; if $\OO_Z$ denotes the restriction of $\wt{A}/\wt{\fk{J}}$ to $Z$, then $(Z,\OO_Z)$ is a subprescheme of $(X,\OO_X)$, and $f$ factors through the canonical injection of this subprescheme into $X$ and the isomorphism $(\psi_0,\omega_0)$, where $\psi_0$ is $\psi$ considered as a map from $Y$ to $Z$, and $\omega_0$ the restriction of $\omega$ to $\OO_Z$.

    We pass to the general case.
    Let $U$ be an affine open subset of $X$ such that $U\cap\psi(Y)$ is closed in $U$ and nonempty.
    By restricting $f$ to the prescheme induced by $Y$ on the open subset $\psi^{-1}(U)$, and by considering it as a morphism from this prescheme to the prescheme induced by $X$ on $U$, we reduce to the first case; the restriction of $f$ to $\psi^{-1}(U)$ is thus a closed immersion $\psi^{-1}(U)\to U$, canonically factoring as $j_U\circ g_U$, where $g_U$ is an isomorphism from the prescheme $\psi^{-1}(U)$ to a subprescheme $Z_U$ of $U$, and $j_U$ is the canonical injection $Z_U\to U$.
    Let $V$ be a second affine open subset of $X$ such that $V\subset U$; as the restriction $Z_V'$ of $Z_U$ to $V$ is a subprescheme of the prescheme $V$, the restriction of $f$ to $\psi^{-1}(V)$ factors as $j_V'\circ g_V'$, where $j_V'$ is the canonical injection $Z_V'\to V$ and $g_V'$ is an isomorphism from $\psi^{-1}(V)$ to $Z_V'$.
    By the uniqueness of the canonical factorization of an immersion \sref{1.4.2.1}, we necessarily have that $Z_V'=Z_V$ and $g_V'=g_V$.
    We conclude \sref{1.4.1.5} that there is a subprescheme $Z$ of $X$ whose underlying space is $\psi(Y)$ and whose restriction to each $U\cap\psi(Y)$ is $Z_U$; the $g_U$ are then the restrictions to $\psi^{-1}(U)$ of an isomorphism $g:Y\to Z$ such that $f=j\circ g$, where $j$ is the canonical injection $Z\to X$.
\end{enumerate}
\end{proof}

\begin{cor}[4.2.3]
\label{1.4.2.3}
Let $X$ be an affine scheme.
For a morphism $f=(\psi,\theta):Y\to X$ to be a closed immersion, it is necessary and sufficicent that $Y$ is an affine scheme and that the homomorphism $\Gamma(\psi):\Gamma(\OO_X)\to\Gamma(\OO_Y)$ is surjective.
\end{cor}

\begin{cor}[4.2.4]
\label{1.4.2.4}
\medskip\noindent
\begin{enumerate}[label={\rm(\alph*)}]
  \item Let $f$ be a morphism $Y\to X$, $(V_\lambda)$ a cover of $f(Y)$ by open subsets of $X$.
    For $f$ to be an immersion (resp. an open immersion), it is necessary and sufficient
\oldpage[I]{124}
    for its restriction to each of the induced preschemes $f^{-1}(V_\lambda)$ to be an immersion (resp. an open immersion) into $V_\lambda$.
  \item Let $f$ be a morphism $Y\to X$, $(V_\lambda)$ an open cover of $X$.
    For $f$ to be a closed immersion, it is necessary and sufficient for its restriction to each of the induced preschemes $f^{-1}(V_\lambda)$ to be a closed immersion into $V_\lambda$.
\end{enumerate}
\end{cor}

\begin{proof}
\label{proof-1.4.2.4}
Let $f=(\psi,\theta)$; in the case (a), $\theta_y^\sharp$ is surjective (resp. bijective) for all $y\in Y$, and in the case (b), $\theta_y^\sharp$ is surjective for all $y\in Y$; it thus suffices to check that $\psi$, in case (a), is a homeomorphism from $Y$ to a locally closed (resp. open) subset of $X$, and in case (b), a homeomorphism from $Y$ to a closed subset of $X$.
Now $\psi$ is evidently injective and sends each neighborhood of $y$ in $Y$ to a neighborhood of $\psi(y)$ is $\psi(Y)$ for all $y\in Y$, by virtue of the hypothesis; in case (a), $\psi(Y)\cap V_\lambda$ is locally closed (resp. open) in $V_\lambda$, so $\psi(Y)$ is locally closed (resp. open) in the union of the $V_\lambda$, and \emph{a fortiori} in $X$; in case (b), $\psi(Y)\cap V_\lambda$ is closed in $V_\lambda$, so $\psi(Y)$ is closed in $X$ since $X=\bigcup_\lambda V_\lambda$.
\end{proof}

\begin{prop}[4.2.5]
\label{1.4.2.5}
The composition of two immersions (resp. of two open immersions, of two closed immersions) is an immersion (resp. an open immersion, a closed immersion).
\end{prop}

\begin{proof}
\label{proof-1.4.2.5}
This follows easily from \sref{1.4.1.6}.
\end{proof}

\subsection{Products of immersions}
\label{subsection-products-of-immersions}

\begin{prop}[4.3.1]
\label{1.4.3.1}
Let $\alpha:X'\to X$, $\beta:Y'\to Y$ be two $S$-morphisms; if $\alpha$ and $\beta$ are immersions (resp. open immersions, closed immersions), then $\alpha\times_S\beta$ is an immersion (resp. an open immersion, a closed immersion).n
In addition, if $\alpha$ (resp. $\beta$) identifies $X'$ (resp. $Y'$) with a subprescheme $X''$ (resp. $Y''$) of $X$ (resp. $Y$), then $\alpha\times_S\beta$ identifies the underlying space of $X'\times_S Y'$ with the subspace $p^{-1}(X'')\cap q^{-1}(Y'')$ of the underlying space of $X\times_S Y$, where $p$ and $q$ denote the projections from $X\times_S Y$ to $X$ and $Y$ respectively.
\end{prop}

\begin{proof}
\label{proof-1.4.3.1}
According to Definition \sref{1.4.2.1}, we can restrict to the case where $X'$ and $Y'$ are subpreschemes, $\alpha$ and $\beta$ the injection morphisms. The proposition has already been proven for the subpreschemes induced by open sets \sref{1.3.2.7}; as each subprescheme is a closed subprescheme of a prescheme induced by an open set \sref{1.4.1.3}, we reduce to the case where $X'$ and $Y'$ are \emph{closed} subpreschemes.

Let us first show that we can assume that $S$ is \emph{affine}.
Indeed, let $(S_\lambda)$ be a cover of $S$ by open affine sets; if $\vphi$ and $\psi$ are the structure morphisms of $X$ and $Y$, then let $X_\lambda=\vphi^{-1}(S_\lambda)$ and $Y_\lambda=\psi^{-1}(S_\lambda)$.
The restriction $X_\lambda'$ (resp. $Y_\lambda'$) of $X'$ (resp. $Y'$) to $X_\lambda\cap X'$ (resp. $Y_\lambda\cap Y'$) is a closed subprescheme of $X_\lambda$ (resp. $Y_\lambda$), the preschemes $X_\lambda$, $Y_\lambda$, $X_\lambda'$, $Y_\lambda'$ can be considered as $S_\lambda$-preschemes and the products $X_\lambda\times_S Y_\lambda$ and $X_\lambda\times_{S_\lambda}Y_\lambda$ (resp. $X_\lambda'\times_S Y_\lambda'$ and $X_\lambda'\times_{S_\lambda}Y_\lambda'$) are identical \sref{1.3.2.5}.
If the proposition is true when $S$ is affine, then the restriction of $\alpha\times_S\beta$ to each of the $X_\lambda'\times_S Y_\lambda'$ is thus an immersion \sref{1.3.2.7}.
As the product $X_\lambda'\times_S Y_\mu'$ (resp. $X_\lambda\times_S Y_\mu$) identifies with $(X_\lambda'\cap X_\mu')\times_S(Y_\lambda'\cap Y_\mu')$ (resp. $(X_\lambda\cap X_\mu)\times_S(Y_\lambda\cap Y_\mu)$) \sref{1.3.2.6.4}, the restriction of $\alpha\times_S\beta$
\oldpage[I]{125}
to each of the $X_\lambda'\times_S Y_\mu'$ is also an immersion; the same is true for $\alpha\times_S\beta$ by \sref{1.4.2.4}.

Second, we prove that we can suppose that $X$ and $Y$ are \emph{affine}.
Indeed, let $(U_i)$ (resp. $(V_j)$) be a cover of $X$ (resp. $Y$) by open affine sets, and let $X_i'$ (resp. $Y_j'$) be the restriction of $X'$ (resp. $Y'$) to $X'\cap U_i$ (resp. $Y'\cap V_j$), which is a closed subprescheme of $U_i$ (resp. $V_j$); $U_i\times_S V_j$ indentifies with the restriction of $X\times_S Y$ to $p^{-1}(U_i)\cap q^{-1}(V_j)$ \sref{1.3.2.7}; similarly, if $p'$ and $q'$ are the projections from $X'\times_S Y'$, then $X_i'\times_S Y_j'$ identifies with the restriction of $X'\times_S Y'$ to ${p'}^{-1}(X_i')\cap{q'}^{-1}(Y_j')$.
Set $\gamma=\alpha\times_S\beta$; we have by definition $p\circ\gamma=\alpha\circ p'$ and $q\circ\gamma=\beta\circ q'$; as $X_i'=\alpha^{-1}(U_i)$ and $Y_j'=\beta^{-1}(V_j)$, we also have ${p'}^{-1}(X_i')=\gamma^{-1}(p^{-1}(U_i))$ and ${q'}^{-1}(Y_j')=\gamma^{-1}(q^{-1}(V_j))$, hence
\[
  {p'}^{-1}(X_i')\cap{q'}^{-1}(Y_j')=\gamma^{-1}(p^{-1}(U_i)\cap q^{-1}(V_j))=\gamma^{-1}(U_i\times_S V_j),
\]
we conclude as in the first part.

So suppose $X$, $Y$, $S$ are affine, and let $B$, $C$, $A$ be their respective rings.
Then $B$ and $C$ are $A$-algebras, $X'$ and $Y'$ are affine schemes whose rings are quotient algebras $B'$ and $C'$ of $B$ and $C$ respectively.
In addition, we have $\alpha=({}^a\rho,\wt{\rho})$ and $\beta=({}^a\sigma,\wt{\sigma})$, where $\rho$ and $\sigma$ are respectively the canonical homomorphisms $B\to B'$ and $C\to C'$ \sref{1.1.7.3}.
This being so, we know that $X\times_S Y$ (resp. $X'\times_S Y'$) is an affine scheme with ring $B\otimes_A C$ (resp. $B'\otimes_A C'$), and $\alpha\times_S\beta=({}^a\tau,\wt{\tau})$, where $\tau$ is the homomorphism $\rho\otimes\sigma$ from $B\otimes_A C$ to $B'\otimes_A C'$ (\sref{1.3.2.2} and \sref{1.3.2.3}); as this homomorphism is surjective, $\alpha\times_S\beta$ is an immersion.
In addition, if $\fk{b}$ (resp. $\fk{c}$) is the kernel of $\rho$ (resp. $\sigma$), then the kernel of $\tau$ is $u(\fk{b})+v(\fk{c})$, where $u$ (resp. $v$) is the homomorphism $b\mapsto b\otimes 1$ (resp. $c\mapsto 1\otimes c$).
As $p=({}^a u,\wt{u})$ and $q=({}^a v,\wt{v})$, this kernel corresponds, in the prime spectrum of $B\otimes_A C$, with the closed set $p^{-1}(X')\cap q^{-1}(Y')$ ((1.2.2.1) and \sref{1.1.1.2}[iii]), which finishes the proof.
\end{proof}

\begin{cor}[4.3.2]
\label{1.4.3.2}
If $f:X\to Y$ is an immersion (resp. an open immersion, a closed immersion) and an $S$-morphism, then $f_{(S')}$ is an immersion (resp. an open immersion, a closed immersion) for every extension $S'\to S$ of the base prescheme.
\end{cor}

\subsection{Inverse images of a subprescheme}
\label{subsection-inverse-image-of-subprescheme}

\begin{prop}[4.4.1]
\label{1.4.4.1}
Let $f:X\to Y$ be a morphism, $Y'$ a subprescheme (resp. a closed subprescheme, a prescheme induced by an open set) of $Y$, and $j:Y'\to Y$ the injection morphism.
Then the projection $p:X\times_Y Y'\to X$ is an immersion (resp. a closed immersion, an open immersion); the subprescheme of $X$ associated to $p$ has the underlying space $f^{-1}(Y')$; in addition, if $j'$ is the injection morphism of this subprescheme into $X$, then for a morphism $h:Z\to X$ to be such that $f\circ h:Z\to Y$ is \unsure{majorized} by $j$, it is necessary and sufficient that $h$ be \unsure{majorized} by $j'$.
\end{prop}

\begin{proof}
\label{proof-1.4.4.1}
As $p=1_X\times_Y j$ \sref{1.3.3.4}, the first assertion follows from \sref{1.4.3.1}; the second is a particular case of \sref{1.3.5.10} (where one has to switch $X$ and $Y'$).
Finally, if we have $f\circ h=j\circ h'$, where $h'$ is a morphism $Z\to Y'$, then it follows from the definition of the product that we have $h=p\circ u$, where $u$ is a morphism $Z\to X\times_Y Y'$, hence the latter assertion.
\end{proof}

We say that the subprescheme of $X$ thus defined is the \emph{inverse image} of the subprescheme $Y'$ of $Y$ under the morphism $f$, terminology which is consistent with that introduced
\oldpage[I]{126}
more generally in \sref{1.3.3.6}.
When we speak of $f^{-1}(Y')$ as a subprescheme of $X$, this will always be the subprescheme we mean.

When the preschemes $f^{-1}(Y')$ and $X$ are identical, $j'$ is the identity and each morphism $h:Z\to X$ is thus majorized by $j'$, so \emph{the morphism $f:X\to Y$ then factors as $X\xrightarrow{g}Y'\xrightarrow{j}Y$}.

When $y$ is a \emph{closed} point of $Y$ and $Y'=\Spec(\kres(y))$ is the smallest closed subprescheme of $Y$ having $\{y\}$ as its underlying space \sref{1.4.1.9}, the closed subprescheme $f^{-1}(Y')$ is canonically isomorphic to the \emph{fibre} $f^{-1}(y)$ defined in \sref{1.3.6.2}, with which we identify it.

\begin{cor}[4.4.2]
\label{1.4.4.2}
Let $f:X\to Y$ and $g:Y\to Z$ be two morphisms, $h=g\circ f$ their composition.
For each subprescheme $Z'$ of $Z$, the subpreschemes $f^{-1}(g^{-1}(Z'))$ and $h^{-1}(Z')$ of $X$ are identical.
\end{cor}

\begin{proof}
\label{proof-1.4.4.2}
This follows from the existence of the canonical isomorphism $X\times_Y(Y\times_Z Z')\isoto X\times_Z Z'$ (3.3.9.1).
\end{proof}

\begin{cor}[4.4.3]
\label{1.4.4.3}
Let $X'$ and $X''$ be two subpreschemes of $X$, $j':X'\to X$ and $j'':X''\to X$ their injection morphisms; then ${j'}^{-1}(X'')$ and ${j''}^{-1}(X')$ are both equal to the greatest lower bound $\inf(X',X'')$ of $X'$ and $X''$ for the ordering $\leqslant$ on subpreschemes, and is canonically isomorphic to $X'\times_X X''$.
\end{cor}

\begin{proof}
\label{proof-1.4.4.3}
This follows immediately from Proposition \sref{1.4.4.1} and Corollary \sref{1.4.1.10}.
\end{proof}

\begin{cor}[4.4.4]
\label{1.4.4.4}
Let $f:X\to Y$ be a morphism, $Y'$ and $Y''$ two subpreschemes of $Y$; then we have $f^{-1}(\inf(Y',Y''))=\inf(f^{-1}(Y'),f^{-1}(Y''))$.
\end{cor}

\begin{proof}
\label{proof-1.4.4.4}
This follows from the existence of the canonical isomorphism between $(X\times_Y Y')\times_X(X\times_Y Y'')$ and $X\times_Y(Y'\times_Y Y'')$ (3.3.9.1).
\end{proof}

\begin{prop}[4.4.5]
\label{1.4.4.5}
Let $f:X\to Y$ be a morphism, $Y'$ a closed subprescheme of $Y$ defined by a quasi-coherent sheaf of ideals $\sh{K}$ of $\OO_Y$ \sref{1.4.1.3}; the closed subprescheme $f^{-1}(Y')$ of $X$ is then defined by the quasi-coherent sheaf of ideals $f^*(\sh{K})\OO_X$ of $\OO_X$.
\end{prop}

\begin{proof}
\label{proof-1.4.4.5}
The statement is evidently local on $X$ and $Y$; it thus suffices to note that if $A$ is a $B$-algebra and $\fk{K}$ an ideal of $B$. then we have $A\otimes_B(B/\fk{K})=A/\fk{K}A$, and to apply \sref{1.1.6.9}.
\end{proof}

\begin{cor}[4.4.6]
\label{1.4.4.6}
Let $X'$ be a closed subprescheme of $X$ defined by a quasi-coherent sheaf of ideals $\sh{J}$ of $\OO_X$, $i$ the injection $X'\to X$; for the restriction $f\circ i$ of $f$ to $X'$ to be majorized by the injection $j:Y'\to Y$ \emph{(in other words it factors as $j\circ g$, where $g$ is a morphism $X'\to Y'$)}, it is necessary and sufficient that $f^*(\sh{K})\subset\sh{J}$.
\end{cor}

\begin{proof}
\label{proof-1.4.4.6}
It suffices to apply Proposition \sref{1.4.4.1} to $i$, taking into account Proposition \sref{1.4.4.5}.
\end{proof}

\subsection{Local immersions and local isomorphisms}
\label{subsection-local-immersions-isomorphisms}

\begin{defn}[4.5.1]
\label{1.4.5.1}
Let $f:X\to Y$ be a morphism of preschemes.
We say that $f$ is a local immersion at a point $x\in X$ if there exists an open neighborhood $U$ of $x$ in $X$ and an open neighborhood $V$ of $f(x)$ in $Y$ such that the restriction of $f$ to the induced prescheme $U$ is a closed immersion of $U$ into the induced prescheme $V$.
We say that $f$ is a local immersion if $f$ is a local immersion for each point of $X$.
\end{defn}

\begin{defn}[4.5.2]
\label{1.4.5.2}
We say that a morphism $f:X\to Y$ is a local isomorphism at
\oldpage[I]{127}
a point $x\in X$ if there exists an open neighborhood $U$ of $x$ in $X$ such that the restriction of $f$ to the induced prescheme $U$ is an open immersion of $U$ into $Y$.
We say that $f$ is a local isomorphism if $f$ is a local isomorphism for each point of $X$.
\end{defn}

\begin{env}[4.5.3]
\label{1.4.5.3}
An immersion (resp. a closed immersion) $f:X\to Y$ can be characterized as a local immersion such that $f$ is a homeomorphism from the underlying space of $X$ to a subset (resp. a closed subset) of $Y$.
An open immersion $f$ can be characterized as an \emph{injective} local isomorphism.
\end{env}

\begin{prop}[4.5.4]
\label{1.4.5.4}
Let $X$ be an irreducible prescheme, $f:X\to Y$ a dominant injective morphism.
If $f$ is a local immersion, then $f$ is an immersion and $f(X)$ is open in $Y$.
\end{prop}

\begin{proof}
\label{proof-1.4.5.4}
Let $x\in X$, and let $U$ be an open neighborhood of $x$, $V$ an open neighborhood of $f(x)$ in $Y$ such that the restriction of $f$ to $U$ is a closed immersion into $V$; as $U$ is dense in $X$, $f(U)$ is dense in $Y$ by hypothesis, so $f(U)=V$ and $f$ is a homeomorphism from $U$ to $V$; the hypothesis that $f$ is injective implies that $f^{-1}(V)=U$, hence the proposition.
\end{proof}

\begin{prop}[4.5.5]
\label{1.4.5.5}
\medskip\noindent
\begin{enumerate}[label=\emph{(\roman*)}]
  \item The composition of two local immersions (resp. of two local isomorphisms) is a local immersion (resp. a local isomorphism).
  \item Let $f:X\to X'$ and $g:Y\to Y'$ be two $S$-morphisms.
    If $f$ and $g$ are local immersions (resp. local isomorphisms), then so is $f\times_S g$.
  \item If an $S$-morphism $f$ is a local immersion (resp. a local isomorphism), then so is $f_{(S')}$ for every extension $S'\to S$ of the base prescheme.
\end{enumerate}
\end{prop}

\begin{proof}
\label{proof-1.4.5.5}
According to \sref{1.3.5.1}, it suffices to prove (i) and (ii).

(i) follows immediately from the transitivity of closed (resp. open) immersions \sref{1.4.2.4} and from the fact that if $f$ is a homeomorphism from $X$ to a closed subset of $Y$, then for every open $U\subset X$, $f(U)$ is open in $f(X)$, so there exists an open subset $V$ of $Y$ such that $f(U)=V\cap f(X)$, and as a result $f(U)$ is closed in $V$.

To prove (ii), let $p$ and $q$ be the projections from $X\times_X Y$, $p'$ and $q'$ the projections from $X'\times_S Y'$.
There exists by hypothesis open neighborhoods $U$, $U'$, $V$, $V'$ of $x=p(z)$, $x'=p'(z')$, $y=q(z)$, $y'=q'(z')$ respectively, such that the restrictions of $f$ and $g$ to $U$ and $V$ respectively are closed (resp. open) immersions into $U'$ and $V'$ respectively.
As the underlying space of $U\times_S V$ and that of $U'\times_S V'$ identify with the open neighborhoods $p^{-1}(U)\cap q^{-1}(V)$ and ${p'}^{-1}(U')\cap{q'}^{-1}(V')$ of $z$ and $z'$ respectively \sref{1.3.2.7}, the proposition follows from Proposition \sref{1.4.3.1}.
\end{proof}


\section{Reduced preschemes; separation condition}
\label{section:reduced-preschemes-and-separation-condition}

\subsection{Reduced preschemes}
\label{subsection:reduced-preschemes}

\begin{prop}[5.1.1]
\label{1.5.1.1}
Let $(X,\OO_X)$ be a prescheme, and $\sh{B}$ a quasi-coherent $\OO_X$-algebra.
Then there exists a unique quasi-coherent $\OO_X$-module $\sh{N}$ whose stalk $\sh{N}_x$ at any $x\in X$ is the nilradical of the ring $\sh{B}_x$.
When $X$ is affine, and, consequently, $\sh{B}=\wt{B}$, where $B$ is an algebra over $A(X)$, then we have $\sh{N}=\wt{\nilrad}$, where $\nilrad$ is the nilradical of $B$.
\end{prop}

\begin{proof}
\label{proof-1.5.1.1}
\oldpage[I]{128}
The statement is local, so it suffices to show the latter claim.
We know that $\wt{\nilrad}$ is a quasi-coherent $\OO_X$-module \sref{1.1.4.1}, and that its stalk at a point $x\in X$ is the ideal $\nilrad_x$ of the ring of fractions $B_x$;
it remains to prove that the nilradical of $B_x$ is contained in $\nilrad_x$, the converse inclusion being evident.
Let $z/s$ be an element of the nilradical of $B_x$, with $z\in B$, and $s\not\in\fk{j}_x$;
by hypothesis, there exists an integer $k$ such that $(z/s)^k=0$, which implies that there exists some $t\not\in\fk{j}_x$ such that $tz^k=0$.
We conclude that $(tz)^k=0$, and, as a result, that $z/s=(tz)/(ts)\in\nilrad_x$.
\end{proof}

We say that the quasi-coherent $\OO_X$-module $\sh{N}$ thus defined is the \emph{nilradical} of the $\OO_X$-algebra $\sh{B}$; in particular, we denote by $\sh{N}_X$ the nilradical of $\OO_X$.

\begin{cor}[5.1.2]
\label{1.5.1.2}
Let $X$ be a prescheme;
the closed subprescheme of $X$ defined by the sheaf of ideals $\sh{N}_X$ is the only reduced subprescheme \sref[0]{0.4.1.4} of $X$ that has $X$ as its underlying space;
it is also the smallest subprescheme of $X$ that has $X$ as its underlying space.
\end{cor}

\begin{proof}
\label{proof-1.5.1.2}
Since the structure sheaf of the closed subprescheme of $Y$ defined by $\sh{N}_X$ is $\OO_X/\sh{N}_X$, it is immediate that $Y$ is reduced and has $X$ as its underlying space, because $\sh{N}_x\neq\OO_x$ for any $x\in X$.
To show the other claims, note that a subprescheme $Z$ of $X$ that has $X$ as its underlying space is defined by a sheaf of ideals $\sh{I}$ \sref{1.4.1.3} such that $\sh{I}_x\neq\OO_x$ for any $x\in X$.
We can restrict to the case where $X$ is affine, say $X=\Spec(A)$ and $\sh{I}=\wt{\fk{I}}$, where $\fk{I}$ is an ideal of $A$;
then, for every $x\in X$, we have $\fk{I}_x\subset\fk{j}_x$, and so $\fk{I}$ is contained in every prime ideal of $A$, and so also in their intersection $\nilrad$, the nilradical of $A$.
This proves that $Y$ is the small subprescheme of $X$ that has $X$ as its underlying space \sref{1.4.1.9};
furthermore, if $Z$ is distinct from $Y$, we necessarily have $\sh{I}_x\neq\sh{N}_x$ for at least one $x\in X$, and so \sref{1.5.1.1} $Z$ is not reduced.
\end{proof}

\begin{defn}[5.1.3]
\label{1.5.1.3}
We define the reduced prescheme associated to a prescheme $X$, denoted by $X_\red$, to be the unique reduced subprescheme of $X$ that has $X$ as its underlying space.
\end{defn}

Saying that a prescheme $X$ is reduced thus implies that $X=X_\red$.

\begin{prop}[5.1.4]
\label{1.5.1.4}
For the prime spectrum of a ring $A$ to be a reduced (resp. integral) prescheme \sref{1.2.1.7}, it is necessary and sufficient for $A$ to be a reduced (resp. integral) ring.
\end{prop}

\begin{proof}
\label{proof-1.5.1.4}
Indeed, it follows immediately from \sref{1.5.1.1} that the condition $\sh{N}=(0)$ is necessary and sufficient for $X=\Spec(A)$ to be reduced;
the claim corresponding to integral rings is then a consequence of \sref{1.1.1.13}.
\end{proof}

Since every ring of fractions $\neq\{0\}$ of an integral ring is integral, it follows from \sref{1.5.1.4} that, for every \emph{locally integral} prescheme $X$, $\OO_x$ is an \emph{integral} ring for every $x\in X$.
The converse is true whenever the underlying space of $X$ is \emph{locally Noetherian}:
indeed, $X$ is then reduced, and if $U$ is an affine open subset of $X$, which is a Noetherian space, then $U$ has only a finite number of irreducible components, and so its ring $A$ has only a finite number of minimal prime ideals \sref{1.1.1.14}.
If two of the irreducible components of $U$ had a common point $x$, then $\OO_x$ would have at least two distinct minimal prime ideals, and would thus not be integral;
the components of $U$ are thus open subsets that are pairwise disjoint, and each of them is thus integral.

\begin{env}[5.1.5]
\label{1.5.1.5}
Let $f=(\psi,\theta):X\to Y$ be a morphism of preschemes;
\oldpage[I]{129}
the homomorphism $\theta_x^\sharp:\OO_{\psi(x)}\to\OO_x$ sends each nilpotent element of $\OO_{\psi(x)}$ to a nilpotent element of $\OO_x$;
by passing to the quotients, $\theta^\sharp$ induces a homomorphism
\[
  \omega:\psi^*(\OO_Y/\sh{N}_Y)\to\OO_X/\sh{N}_X.
\]
It is clear that, for every $x\in X$, $\omega_x:\OO_{\psi(x)}/\sh{N}_{\psi(x)}\to\OO_x/\sh{N}_x$ is a local homomorphism, and so $(\psi,\omega^\flat)$ is a morphism of preschemes $X_\red\to Y_\red$, which we denote by $f_\red$, and call the \emph{reduced} morphism associated to $f$.
It is immediate that, for morphisms $f:X\to Y$ and $g:Y\to Z$, we have $(g\circ f)_\red=g_\red\circ f_\red$, and so we have defined $X_\red$ as a \emph{functor}, \emph{covariant} in $X$.

The preceding definition shows that the diagram
\[
  \xymatrix{
    X_\red\ar[r]^{f_\red}\ar[d] &
    Y_\red\ar[d]\\
    X\ar[r]^f &
    Y
  }
\]
is commutative, where the vertical arrows are the injection morphisms;
in other words, $X_\red\to X$ is a \emph{functorial} morphism.
We note in particular that, if $X$ is reduced, then every morphism $f:X\to Y$ factors as $X\xrightarrow{f_\red}Y_\red\to Y$;
in other words, $f$ factors through the injection morphism $Y_\red\to Y$.
\end{env}

\begin{prop}[5.1.6]
\label{1.5.1.6}
Let $f:X\to Y$ be a morphism;
if $f$ is surjective (resp. radicial, an immersion, a closed immersion, an open immersion, a local immersion, a local isomorphism), then so too is $f_\red$.
Conversely, if $f_\red$ is surjective (resp. radicial), then so too is $f$.
\end{prop}

\begin{proof}
\label{proof-1.5.1.6}
The proposition is trivial if $f$ is surjective;
if $f$ is radicial, then the proposition follows from the fact that, for every $x\in X$, the field $\kres(x)$ is the same for the preschemes $X$ and $X_\red$ \sref{1.3.5.8}.
Finally, if $f=(\psi,\theta)$ is an immersion, a closed immersion, or a local immersion (resp. an open immersion, or a local isomorphism), then the proposition follows from the fact that, if $\theta_x^\sharp$ is surjective (resp. bijective), then so too is the homomorphism obtained by passing to the quotients by the nilradicals $\OO_{\psi(x)}$ and $\OO_x$ (\sref{1.5.1.2} and \sref{1.4.2.2}) (cf. \sref{1.5.5.12}).
\end{proof}

\begin{prop}[5.1.7]
\label{1.5.1.7}
If $X$ and $Y$ are $S$-preschemes, then the preschemes $X_\red\times_{S_\red}Y_\red$ and $X_\red\times_S Y_\red$ are identical, and canonically identified with a subprescheme of $X\times_S Y$ that has the same underlying subspace as the two aforementioned products.
\end{prop}

\begin{proof}
\label{proof-1.5.1.7}
The canonical identification of $X_\red\times_S Y_\red$ with a subprescheme of $X\times_S Y$ that has the same underlying space follows from \sref{1.4.3.1}.
Furthermore, if $\vphi$ and $\psi$ are the structure morphisms $X_\red\to S$ and $Y_\red\to S$ (respectively), then they factor through $S_\red$ \sref{1.5.1.5}, and since $S_\red\to S$ is a monomorphism, the first claim of the proposition follows from \sref{1.3.2.4}.
\end{proof}

\begin{cor}[5.1.8]
\label{1.5.1.8}
The preschemes $(X\times_S Y)_\red$ and $(X_\red\times_{S_\red}Y_\red)_\red$ are canonically identified with one another.
\end{cor}

\begin{proof}
\label{proof-1.5.1.8}
This follows from \sref{1.5.1.2} and \sref{1.5.1.7}.
\end{proof}

We note that, even if $X$ and $Y$ are reduced preschemes, $X\times_S Y$ might not be reduced, because the tensor product of two reduced algebras can have nilpotent elements.

\begin{prop}[5.1.9]
\label{1.5.1.9}
\oldpage[I]{130}
Let $X$ be a prescheme, and $\sh{I}$ a quasi-coherent sheaf of ideals of $\OO_X$ such that $\sh{I}^n=0$ for some integer $n>0$.
Let $X_0$ be the closed subprescheme $(X,\OO_X/\sh{I})$ of $X$;
for $X$ to be an affine scheme, it is necessary and sufficient for $X_0$ to be an affine scheme.
\end{prop}

The condition is clearly necessary, so we will show that it is sufficient.
If we set $X_k=(X,\OO_X/\sh{I}^{k+1})$, it is enough to prove by induction on $k$ that $X_k$ is affine, and so we are led to consider the base case, where $\sh{I}^2=0$.
We set
\begin{align*}
  A&=\Gamma(X,\OO_X)\\
  A_0&=\Gamma(X_0,\OO_{X_0})=\Gamma(X,\OO_X/\sh{I}).
\end{align*}
The canonical homomorphism $\OO_X\to\OO_X/\sh{I}$ induces a homomorphism of rings $\vphi:A\to A_0$.
We will see below that $\vphi$ is \emph{surjective}, which implies that
\begin{equation*}
\label{1.5.1.9.1}
  0\to\Gamma(X,\sh{I})\to\Gamma(X,\OO_X)\to\Gamma(X,\OO_X/\sh{I})\to 0\tag{5.1.9.1}
\end{equation*}
is an \emph{exact} sequence.
We now prove, assuming that this is true, the proposition.
Note that $\fk{K}=\Gamma(X,\sh{I})$ is an ideal whose square is zero in $A$, and thus a module over $A_0=A/\fk{K}$.
By hypothesis, we have $X_0=\Spec(A)$, and, since the underlying spaces of $X_0$ and $X$ are identical, $\fk{K}=\Gamma(X_0,\sh{I})$;
Additionally, since $\sh{I}^2=0$, $\sh{I}$ is a quasi-coherent $(\OO_X/\sh{I})$-module, so we have $\sh{I}\cong\wt{\fk{K}}$ and $\fk{K}_x=\sh{I}_x$ for all $x\in X_0$ \sref{1.1.4.1}.
With this in mind, let $X'=\Spec(A)$, and consider the morphism $f=(\psi,\theta):X\to X'$ of preschemes that corresponds to the identity map $A\to\Gamma(X,\OO_X)$ \sref{1.2.2.4}.
For every affine open subset $V$ of $X$, the diagram
\[
  \xymatrix{
    A\ar[r]\ar[d] &
    \Gamma(V,\OO_X|V)\ar[d]\\
    A_0=A/\fk{K}\ar[r] &
    \Gamma(V,\OO_{X_0}|V)
  }
\]
commutes, whence the diagram
\[
  \xymatrix{
    X' &
    X\ar[l]_f\\
    X'_0\ar[u]^{j'} &
    X_0\ar[l]_{f_0}\ar[u]_j
  }
\]
also commutes ($X'_0$ being the closed subprescheme of $X'$ defined by the quasi-coherent sheaf of ideals $\wt{R}$, and $j$ and $j'$ being the canonical injection morphisms).
But since $X_0$ is affine, $f_0$ is an isomorphism, and since the underlying continuous maps of $j$ and $j'$ are identity maps, we see straight away that $\psi:X\to X'$ is a homeomorphism.
Furthermore, the equation $\fk{K}_x=\sh{I}_x$ shows that the restriction of $\theta^\sharp:\psi^*(\OO_{X'})\to\OO_X$ is an \emph{isomorphism} from $\psi^*(\wt{\fk{K}})$ to $\sh{I}$;
additionally, by passing to the quotients, $\theta^\sharp$ gives an \emph{isomorphism} $\psi^*(\OO_X/\wt{\fk{K}})\to\OO_X/\sh{I}$, because $f_0$ is an isomorphism;
we thus immediately conclude, by the 5 lemma (M,~I,~1.1), that $\theta^\sharp$ is itself an isomorphism, and thus that $f$ is an \emph{isomorphism}, and thus that $X$ is affine.
So everything reduces to proving the exactitude of \sref{1.5.1.9.1}, which will follow from showing that $\HH^1(X,\sh{I})=0$.
\oldpage[I]{131}
But $\HH^1(X,\sh{I})=\HH^1(X_0,\sh{I})$, and we have seen that $\sh{I}$ is a quasi coherent $\OO_{X_0}$-module.
Our proof will thus follow from
\begin{lem}[5.1.9.2]
\label{1.5.1.9.2}
If $Y$ is an affine scheme, and $\sh{F}$ a quasi-coherent $\OO_Y$-module, then $\HH^1(Y,\sh{F})=0$.
\end{lem}

\begin{proof}
\label{proof-1.5.1.9}
This lemma will be proven in Chapter~III, §1, as a consequence of the more general theorem that $\HH^i(Y,\sh{F})=0$ for all $i>0$.
To give an independent proof, note that $\HH^1(Y,\sh{F})$ can be identified with the module $\Ext_{\OO_Y}^1(Y;\OO_Y,\sh{F})$ of extensions classes of the $\OO_Y$-module $\OO_Y$ by the $\OO_Y$-module $\sh{F}$ (T,~4.2.3);
so everything reduces to proving that such an extension $\sh{G}$ is trivial.
But, for all $y\in Y$, there is a neighbourhood $V$ of $y$ in $Y$ such that $\sh{G}|V$ is isomorphic to $\sh{F}|Y\oplus\OO_Y|V$ \sref[0]{0.5.4.9};
from this we conclude that $\sh{G}$ is a \emph{quasi-coherent} $\OO_Y$-module.
If $A$ is the ring of $Y$, then we have $\sh{F}=\wt{M}$ and $\sh{G}=\wt{N}$, where $M$ and $N$ are $A$-modules, and, by hypothesis, $N$ is an extension of the $A$-module $A$ by the $A$-module $M$ \sref{1.1.3.11}.
Since this extension is necessarily trivial, the lemma is proven, and thus so too is \sref{1.5.1.9}.
\end{proof}

\begin{cor}[5.1.10]
\label{1.5.1.10}
Let $X$ be a prescheme such that $\sh{N}_X$ is nilpotent.
For $X$ to be an affine scheme, it is necessary and sufficient for $X_\red$ to be an affine scheme.
\end{cor}

\subsection{Existence of a subprescheme with a given underlying space}
\label{subsection:existence-of-a-subprescheme-with-a-given-underlying-space}

\begin{prop}[5.2.1]
\label{1.5.2.1}
For every locally closed subspace $Y$ of the underlying space of a prescheme $X$, there exists exactly one reduced subprescheme of $X$ that has $Y$ as its underlying space.
\end{prop}

\begin{proof}
\label{proof-1.5.2.1}
The uniqueness follows from \sref{1.5.1.2}, so it remains only to show the existence of the prescheme in question.

If $X$ is affine, given by some ring $A$, and $Y$ closed in $X$, then the proposition is immediate:
$\fk{j}(Y)$ is the largest ideal $\fk{a}\subset A$ such that $V(\fk{a})=Y$, and it is equal to its radical \sref{1.1.1.4}[i], so $A/\fk{j}(Y)$ is a reduced ring.

In the general case, for every affine open $U\subset X$ such that $U\cap Y$ is closed in $U$, consider the closed subprescheme $Y_U$ of $U$ defined by the sheaf of ideals associated to the ideal $\fk{j}(U\cap Y)$ of $A(U)$, which is reduced.
We can show that, if $V$ is an affine open subset of $X$ contained in $U$, then $Y_V$ is \emph{induced} by $Y_U$ on $V\cap Y$;
but this induced prescheme is a closed subprescheme (of $V$) which is reduced and has $V\cap Y$ as its underlying space;
the uniqueness of $Y_V$ thus implies our claim.
\end{proof}

\begin{prop}[5.2.2]
\label{1.5.2.2}
Let $X$ be a reduced subprescheme of a prescheme $Y$; if $Z$ is the closed reduced subprescheme of $Y$ that has $\overline{X}$ as its underlying space, then $X$ is a subprescheme induced on an open subset of $Z$.
\end{prop}

\begin{proof}
\label{proof-1.5.2.2}
\oldpage[I]{132}
There is indeed an open subset $U$ of $Y$ such that $X=U\cap\overline{X}$;
since, by \sref{1.5.2.2}, $X$ is a reduced subprescheme of $Z$, the subprescheme $X$ is induced by $Z$ on the open subspace $X$ by uniqueness \sref{1.5.2.1}.
\end{proof}

\begin{cor}[5.2.4]
\label{1.5.2.4}
Let $f:X\to Y$ be a morphism, and $X'$ (resp. $Y'$) a closed subprescheme of $X$ (resp. $Y$) defined by a quasi-coherent sheaf of ideals $\sh{J}$ (resp. $\sh{K}$) of $\OO_X$ (resp. $\OO_Y$).
Suppose that $X'$ is reduced, and that $f(X')\subset Y'$.
Then $f^*(\sh{K})\OO_X\subset\sh{J}$.
\end{cor}

\begin{proof}
\label{proof-1.5.2.4}
Since, by \sref{1.5.2.2}, the restriction of $f$ to $X'$ factors as $X'\to Y'\to Y$, it suffices to apply \sref{1.4.4.6}.
\end{proof}

\subsection{Diagonal; graph of a morphism}
\label{subsection:diagonal-graph-of-a-morphism}

\begin{env}[5.3.1]
\label{1.5.3.1}
Let $X$ be an $S$-prescheme;
we define the \emph{diagonal morphism} of $X$ in $X\times_S X$, denoted by $\Delta_{X|S}$, or $\Delta_X$, or even $\Delta$ if no confusion may arise, to be the $S$-morphism $(1_X,1_X)_S$, or, in other words, the unique $S$-morphism $\Delta_X$ such that
\[
    p_1\circ\Delta_X=p_2\circ\Delta_X=1_X\tag{5.3.1.1}
\]
where $p_1$ and $p_2$ are the projections of $X\times_S X$ (Definition~\sref{1.3.2.1}).
If $f:T\to X$ and $g:T\to Y$ are $S$-morphisms, we immediately have that
\[
    (f,g)_S=(f\times_S g)\circ\Delta_{T|S}.\tag{5.3.1.2}
\]

The reader will note that the preceding definition and the results stated in \sref{1.5.3.1} to \sref{1.5.3.8} are true in any category, \emph{provided that the products used within exist in the category}.
\end{env}

\begin{prop}[5.3.2]
\label{1.5.3.2}
Let $X$ and $Y$ be $S$-preschemes;
if we make the canonical identification between $(X\times Y)\times(X\times Y)$ and $(X\times X)\times(Y\times Y)$, then the morphism $\Delta_{X\times Y}$ is identified with $\Delta_X\times\Delta_Y$.
\end{prop}

\begin{proof}
\label{proof-1.5.3.2}
Indeed, if $p_1:X\times X\to X$ and $q_1:Y\times Y\to Y$ are the projections onto the first component, then the projection onto the first component $(X\times Y)\times(X\times Y)\to X\times Y$ is identified with $p_1\times q_1$, and we have
\[
    (p_1\times q_1)\circ(\Delta_X\times\Delta_Y)=(p_1\circ\Delta_X)\times(q_1\circ\Delta_Y)=1_{X\times Y}
\]
and we can argue similarly for the projections onto the second component.
\end{proof}

\begin{cor}[5.3.4]
\label{1.5.3.4}
For every extension $S'\to S$ of the base prescheme, $\Delta_{X_{S'}}$ is canonically identified with $(\Delta_X)_{(S')}$.
\end{cor}

\begin{proof}
\label{proof-1.5.3.4}
It suffices to remark that $(X\times_S X)_{(S')}$ is canonically identified with $X_{(S')}\times_{S'}X_{(S')}$ \sref{1.3.3.10}.
\end{proof}

\begin{prop}[5.3.5]
\label{1.5.3.5}
Let $X$ and $Y$ be $S$-preschemes, and $\vphi:S\to T$ a morphism of preschemes, which lets us consider every $S$-prescheme as a $T$-prescheme.
Let $f:X\to S$ and $g:Y\to S$ be the structure morphisms, $p$ and $q$ the projections of $X\times_S Y$, and $\pi=f\circ p=g\circ q$ the structure morphism $X\times_S Y\to S$.
Then the diagram
\[
  \xymatrix{
    X\times_S Y\ar[r]^{(p,q)_T}\ar[d]_\pi &
    X\times_T Y\ar[d]^{f\times_T g}\\
    S\ar[r]^{\Delta_{S|T}} &
    S\times_T S
  }
  \tag{5.3.5.1}
\]
\oldpage[I]{133}
commutes, and identifies $X\times_S Y$ with the product of the $(S\times_T S)$-preschemes $S$ and $X\times_T Y$, and the projections with $\pi$ and $(p,q)_T$.
\end{prop}

\begin{proof}
\label{proof-1.5.3.5}
By \sref{1.3.4.3}, we are led to proving the corresponding proposition \emph{in the category of sets}, replacing $X$, $Y$, and $S$ by $X(Z)_T$, $Y(Z)_T$, and $S(Z)_T$ (respectively), with $Z$ being an arbitrary $T$-prescheme.
But, in the category of sets, the proof is immediate and left to the reader.
\end{proof}

\begin{cor}[5.3.6]
\label{1.5.3.6}
The morphism $(p,q)_T$ can be identified (letting $P=S\times_T S$) with $1_{X\times_T Y}\times_P\Delta_S$.
\end{cor}

\begin{proof}
\label{proof-1.5.3.6}
This follows from \sref{1.5.3.5} and \sref{1.3.3.4}.
\end{proof}

\begin{cor}[5.3.7]
\label{1.5.3.7}
If $f:X\to Y$ is an $S$-morphism, then the diagram
\[
  \xymatrix{
    X\ar[r]^{(1_X,f)_S}\ar[d]_f &
    X\times_S Y\ar[d]^{f\times_S 1_Y}\\
    Y\ar[r]^{\Delta_Y} &
    Y\times_S Y
  }
\]
commutes, and identifies $X$ with the product of the $(Y\times_S Y)$-preschemes $Y$ and $X\times_S Y$.
\end{cor}

\begin{proof}
\label{proof-1.5.3.7}
It suffices to apply \sref{1.5.3.5}, replacing $S$ by $Y$, and $T$ by $S$, and noting that $X\times_Y Y=X$ \sref{1.3.3.3}.
\end{proof}

\begin{prop}[5.3.8]
\label{1.5.3.8}
For $f:X\to Y$ to be a monomorphism of preschemes, it is necessary and sufficient for $\Delta_{X|Y}$ to be an isomorphism from $X$ to $X\times_Y X$.
\end{prop}

\begin{proof}
\label{1.5.3.8}
Indeed, to say that $f$ is a monomorphism implies that, for every $Y$-prescheme $Z$, the corresponding map $f':X(Z)_Y\to Y(Z)_Y$ is an injection, and, since $Y(Z)_Y$ consists of a single element, this implies that $X(Z)_Y$ consists of a single element as well.
But this can also be expressed by saying that $X(Z)_Y\times X(Z)_Y$ is canonically isomorphic to $X(Z)_Y$; the former is exactly the set $(X\times_Y X)(Z)_Y$ \hyperref[1.3.4.3]{(3.4.3.1)}, which implies that $\Delta_{X|Y}$ is an isomorphism.
\end{proof}

\begin{prop}[5.3.9]
\label{1.5.3.9}
The diagonal morphism $\Delta_X$ is an immersion from $X$ to $X\times_S X$.
\end{prop}

\begin{proof}
\label{proof-1.5.3.9}
Indeed, since the continuous maps $p_1$ and $\Delta_X$ from the underlying spaces are such that $p_1\circ\Delta_X$ is the identity, $\Delta_X$ is a homeomorphism from $X$ to $\Delta_X(X)$.
Similarly, the composite homomorphism $\OO_x\to\OO_{\Delta_X(x)}\to\OO_x$ (composed of the homomorphisms corresponding to $p_1$ and $\Delta_X$) is the identity, which means that the homomorphism corresponding to $\Delta_X$ is surjective;
the proposition thus follows from \sref{1.4.2.2}.
\end{proof}

We say that the subprescheme of $X\times_S X$ associated to the immersion $\Delta_X$ \sref{1.4.2.1} is \emph{the diagonal} of $X\times_S X$.

\begin{cor}[5.3.10]
\label{1.5.3.10}
Under the hypotheses of \sref{1.5.3.5}, $(p,q)_T$ is an immersion.
\end{cor}

\begin{proof}
\label{proof-1.5.3.10}
This follows from \sref{1.5.3.6} and \sref{1.4.3.1}.
\end{proof}

We say (under the hypotheses of \sref{1.5.3.5}) that $(p,q)_T$ is the \emph{canonical immersion} of $X\times_S Y$ into $X\times_T Y$.

\begin{cor}[5.3.11]
\label{1.5.3.11}
Let $X$ and $Y$ be $S$-preschemes, and $f:X\to Y$ an $S$-morphism;
then the graph morphism $\Gamma_f=(1_X,f)_S$ of $f$ \sref{1.3.3.14} is an immersion of $X$ into $X\times_S Y$.
\end{cor}

\begin{proof}
\label{proof-1.5.3.11}
This is a particular case of Corollary \sref{1.5.3.10}, where we replace $S$ by $Y$, and $T$ by $S$ (cf. \sref{1.5.3.7}).
\end{proof}

\oldpage[I]{134}
The subprescheme of $X\times_S Y$ associated to the immersion $\Gamma_f$ \sref{1.4.2.1} is called \emph{the graph} of the morphism $f$;
the subpreschemes of $X\times_S Y$ that are graphs of morphisms $X\to Y$ are characterised by the property that the restriction to such a subprescheme $G$ of the projection $p_1:X\times_S Y\to X$ is an \emph{isomorphism} $g$ from $G$ to $X$:
$G$ is the the graph of the morphism $p_2\circ g^{-1}$, where $p_2$ is the projection $X\times_S Y\to Y$.

When we take, in particular, $X=S$, then the $S$-morphisms $S\to Y$ (which are exactly the \emph{$S$-sections} of $Y$ \sref{1.2.5.5}) are equal to their graph morphisms;
the subpreschemes of $Y$ that are the graphs of $S$-sections (in other words, those that are isomorphic to $S$ by the restriction of the structure morphism $Y\to S$) are then also called the \emph{images} of these sections, or, by an abuse of language, the \emph{$S$-sections} of $Y$.

\begin{cor}[5.3.12]
\label{1.5.3.12}
With the hypotheses and notation of \sref{1.5.3.11}, for every morphism $g:S'\to S$, let $f'$ be the inverse image of $f$ under $g$ \sref{1.3.3.7};
then $\Gamma_{f'}$ is the inverse image of $\Gamma_f$ under $g$.
\end{cor}

\begin{proof}
\label{proof-1.5.3.12}
This is a particular case of \sref{1.3.3.10.1}.
\end{proof}

\begin{cor}[5.3.13]
\label{1.5.3.13}
Let $f:X\to Y$ and $g:Y\to Z$ be morphisms;
if $g\circ f$ is an immersion (resp. a local immersion), then so too is $f$.
\end{cor}

\begin{proof}
\label{proof-1.5.3.13}
Indeed, $f$ factors as $X\xrightarrow{\Gamma_f}X\times_Z Y\xrightarrow{p_2}Y$.
Furthermore, $p_2$ can be identified with $(g\circ f)\times_Z 1_Y$ \sref{1.3.3.4};
if $g\circ f$ is an immersion (resp. a local immersion), then so too is $p_2$ (\sref{1.4.3.1} and \sref{1.4.5.5}), and since $\Gamma_f$ is an immersion \sref{1.5.3.11}, we are done, by \sref{1.4.2.4} (resp. \sref{1.4.5.5}).
\end{proof}

\begin{cor}[5.3.14]
\label{1.5.3.14}
Let $j:X\to Y$ and $g:Z\to Z$ be $S$-morphisms.
If $j$ is an immersion (resp. a local immersion), then so too is $(j,g)_S$.
\end{cor}

\begin{proof}
\label{proof-1.5.3.14}
Indeed, if $p:Y\times_S Z\to Y$ is the projection onto the first component, then we have $j=p\circ(j,g)_S$, and it suffices to apply \sref{1.5.3.13}.
\end{proof}

\begin{prop}[5.3.15]
\label{1.5.3.15}
If $f:X\to Y$ is an $S$-morphism, then the diagram
\begin{equation*}
\label{1.5.3.15.1}
  \xymatrix{
    X\ar[r]^{\Delta_X}\ar[d]_f &
    X\times_S X\ar[d]^{f\times_S f}\\
    Y\ar[r]^{\Delta_Y} &
    Y\times_S Y
  }\tag{5.3.15.1}
\end{equation*}
commutes \emph{(in other words, $\Delta_X$ is a functorial morphism in the category of preschemes)}.
\end{prop}

\begin{proof}
\label{proof-1.5.3.15}
The proof is immediate and left to the reader.
\end{proof}

\begin{cor}[5.3.16]
\label{1.5.3.16}
If $X$ is a subprescheme of $Y$, then the diagonal $\Delta_X(X)$ can be identified with a subprescheme of $\Delta_Y(Y)$, and the underlying space can be identified with
\[
  \Delta_Y(Y)\cap p_1^{-1}(X)=\Delta_Y\cap p_2^{-1}(X)
\]
($p_1$ and $p_2$ being the projections of $Y\times_S Y$).
\end{cor}

\begin{proof}
\label{proof-1.5.3.16}
Applying \sref{1.5.3.15} to the injection morphism $f:X\to Y$, we see that $f\times_S f$ is an immersion that identifies the underlying space of $X\times_S X$ with the subspace $p_1^{-1}(X)\cap p_2^{-1}(X)$ of $Y\times_S Y$ \sref{1.4.3.1};
further, if $z\in\Delta_Y(Y)\cap p_1^{-1}(X)$, then we have $z=\Delta_Y(y)$
\oldpage[I]{135}
and $y=p_1(z)\in X$, so $y=f(y)$, and $z=\Delta_Y(f(y))$ belongs to $\Delta_X(X)$ by the commutativity of \sref{1.5.3.15.1}.
\end{proof}

\begin{cor}[5.3.17]
\label{1.5.3.17}
Let $f_1:Y\to X$ and $f_2:Y\to X$ be $S$-morphisms, and $y$ a point of $Y$ such that $f_1(y)=f_2(y)=x$, and such that the homomorphisms $\kres(x)\to\kres(y)$ corresponding to $f_1 $ and $f_2$ are identical.
Then, if $f=(f_1,f_2)_S$, the point $f(y)$ belongs to the diagonal $\Delta_{X|S}(X)$.
\end{cor}

\begin{proof}
\label{proof-1.5.3.17}
The two homomorphisms $\kres(x)\to\kres(y)$ corresponding to $f_i$ ($i=1,2$) define two $S$-morphisms $g_i:\Spec(\kres(y))\to\Spec(\kres(x))$ such that the diagrams
\[
  \xymatrix{
    \Spec(\kres(y))\ar[r]^{g_i}\ar[d] &
    \Spec(\kres(x))\ar[d]\\
    Y\ar[r]^{f_i} &
    X
  }
\]
commute.
The diagram
\[
  \xymatrix{
    \Spec(\kres(y))\ar[rr]^{(g_1,g_2)_S}\ar[d] & &
    \Spec(\kres(x))\times_S\Spec(\kres(x))\ar[d]\\
    Y\ar[rr]^{(f_1,f_2)_S} & &
    X\times_S X
  }
\]
thus also commutes.
But it follows from the equality $g_1=g_2$ that the image under $(g_1,g_2)_S$ of the unique point of $\Spec(\kres(y))$ belongs to the diagonal of $\Spec(\kres(x))\times_S\Spec(\kres(x))$;
the conclusion then follows from \sref{1.5.3.15}.
\end{proof}

\subsection{Separated morphisms and separated preschemes}
\label{subsection:separated-morphism-and-separated-preschemes}

\begin{defn}[5.4.1]
\label{1.5.4.1}
We say that a morphism of preschemes $f:X\to Y$ is separated if the diagonal morphism $X\to X\times_Y X$ is a \emph{closed} immersion;
we then also say that $X$ is a \emph{separated prescheme over $Y$}, or a \emph{$Y$-scheme}.
We say that a prescheme $X$ is separated if it is separated over $\Spec(\bb{Z})$;
we then also say that $X$ is a \emph{scheme}\footnote{\emph{[Trans.] We repeat here the warning given at the very start of this translation: the early versions of the EGA use \emph{prescheme} to mean is now usually called a scheme, and \emph{scheme} for what is now usually called a separated scheme.}. Grothendieck himself later said that the more modern terminology was preferable, but we have decided to keep this translation `historically accurate' by using the older nomenclature.} \emph{(cf. \sref{1.5.5.7})}.
\end{defn}

By \sref{1.5.3.9}, for $X$ to be separated over $Y$, it is necessary and sufficient for $\Delta_X(X)$ to be a \emph{closed subspace} of the underlying space of $X\times_Y X$.

\begin{prop}[5.4.2]
\label{1.5.4.2}
Let $S\to T$ be a separated morphism.
If $X$ and $Y$ are $S$-preschemes, then the canonical immersion $X\times_S Y\to X\times_T Y$ \sref{1.5.3.10} is closed.
\end{prop}

\begin{proof}
\label{proof-1.5.4.2}
Indeed, if we refer to the diagram in \sref{1.5.3.5.1}, we see that $(p,q)_T$ can be considered as being obtained from $\Delta_{S|T}$ by the extension $f\times_T g:X\times_T Y\to S\times_T S$ of the base prescheme $S\times_T S$;
the proposition then follows from \sref{1.4.3.2}.
\end{proof}

\begin{cor}[5.4.3]
\label{1.5.4.3}
Let $Y$ be an $S$-scheme, and $f:X\to Y$ an $S$-morphism.
Then the graph morphism $\Gamma_f:X\to X\times_S Y$ \sref{1.5.3.11} is a closed immersion.
\end{cor}

\begin{proof}
\label{proof-1.5.4.3}
This is a particular case of \sref{1.5.4.2}, where we replace $S$ by $Y$, and $T$ by $S$.
\end{proof}

\begin{cor}[5.4.4]
\label{1.5.4.4}
Let $f:X\to Y$ and $g:Y\to Z$ be morphisms, with $g$ separated.
If $g\circ f$ is a closed immersion, then so too is $f$.
\end{cor}

\begin{proof}
\label{proof-1.5.4.4}
The proof using \sref{1.5.4.3} is the same as that of \sref{1.5.3.13} using \sref{1.5.3.11}.
\end{proof}

\oldpage[I]{136}
\begin{cor}[5.4.5]
\label{1.5.4.5}
Let $Z$ be an $S$-scheme, and $j:X\to Y$ and $g:X\to Z$ $S$-morphisms.
If $j$ is a closed immersion, then so too is $(j,g)_S:X\to Y\times_S Z$.
\end{cor}

\begin{proof}
\label{proof-1.5.4.5}
The proof using \sref{1.5.4.4} is the same as that of \sref{1.5.3.14} using \sref{1.5.3.13}.
\end{proof}

\begin{cor}[5.4.6]
\label{1.5.4.6}
If $X$ is an $S$-scheme, then every $S$-section of $X$ \sref{1.2.5.5} is a closed immersion.
\end{cor}

\begin{proof}
\label{proof-1.5.4.6}
If $\vphi:X\to S$ is the structure morphism, and $\psi:S\to X$ an $S$-section of $X$, it suffices to apply \sref{1.5.4.5} to $\vphi\circ\psi=1_S$.
\end{proof}

\begin{cor}[5.4.7]
\label{1.5.4.7}
Let $X$ be an integral prescheme with generic point $s$, and $X$ an $S$-scheme.
If two $S$-sections $f$ and $g$ are such that $f(s)=g(s)$, then $f=g$.
\end{cor}

\begin{proof}
\label{proof-1.5.4.7}
Indeed, if $x=f(s)=g(s)$, then the homomorphisms $\kres(x)\to\kres(s)$ corresponding to $f$ and $g$ are necessarily identical.
If $h=(f,g)_S$, we thus deduce \sref{1.5.3.17} that $h(s)$ belongs to the diagonal $Z=\Delta_X(X)$;
but since $S=\overline{\{s\}}$, and since $Z$ is closed by hypothesis, we have $h(S)\subset Z$.
It then follows from \sref{1.5.2.2} that $h$ factors as $S\to Z\to X\times_S X$, and we thus conclude that $f=g$, by definition of the diagonal.
\end{proof}

\begin{rmk}[5.4.8]
\label{1.5.4.8}
If we suppose, conversely, that the conclusion of \sref{1.5.4.3} is true when $f=1_Y$, then we can conclude that $Y$ is separated over $S$;
similarly, if we suppose that the conclusion of \sref{1.5.4.5} applies to the two morphisms $Y\xrightarrow{\Delta_Y}Y\times_Z Y\xrightarrow{p_1}Y$, then we can conclude that $\Delta_Y$ is a closed immersion, and thus that $Y$ is separated over $Z$;
finally, if we assume that the conclusion of \sref{1.5.4.6} is true for the $Y$ section $\Delta_Y$ of the $Y$-prescheme $Y\times_S Y\to Y$, then this implies that $Y$ is separated over $S$.
\end{rmk}

\subsection{Separation criteria}
\label{subsection:separation-criteria}

\begin{prop}[5.5.1]
\label{1.5.5.1}
\medskip\noindent
\begin{enumerate}[label=\emph{(\roman*)}]
    \item Every monomorphism of preschemes (and, in particular, every immersion) is a separated morphism.
    \item The composition of any two separated morphisms is separated.
    \item If $f:X\to X'$ and $g:Y\to Y'$ are two separated $S$-morphisms, then $f\times_S g$ is separated.
    \item If $f:X\to Y$ is a separated $S$-morphism, then the $S'$-morphism $f_{(S')}$ is separated, for every extension $S'\to S$ of the base prescheme.
    \item If the composition $g\circ f$ is separated, then $f$ is separated.
    \item For a morphism $f$ to be separated, it is necessary and sufficient for $f_\red$ \sref{1.5.1.5} to be separated.
\end{enumerate}
\end{prop}

\begin{proof}
\label{proof-1.5.5.1}
Note that (i) is an immediate consequence of \sref{1.5.3.8}.
If $f:X\to Y$ and $g:Y\to Z$ are morphisms, then the diagram
\[
  \xymatrix{
    X\ar[rr]^{\Delta_{X|Z}}\ar[dr]_{\Delta_{X|Y}} & &
    X\times_Z X\\
    & X\times_Y X\ar[ur]_j
  }
  \tag{5.5.1.1}
\]
where $j$ denotes the canonical immersion \sref{1.5.3.10} is commutative, as can be immediately verified.
If $f$ and $g$ are separated, then $\Delta_{X|Y}$ is a closed immersion by definition, and $j$ is a closed immersion by \sref{1.5.4.2}, whence $\Delta_{X|Z}$ is a closed immersion by \sref{1.4.2.4}, which
\oldpage[I]{137}
proves (ii).
Given (i) and (ii), (iii) and (iv) are equivalent \sref{1.3.5.1}, so it suffices to prove (iv).
But $X_{(S')}\times_{Y_{(S')}}X_{(S')}$ is canonically identified with $(X\times_Y X)\times_Y Y_{(S')}$ by \sref{1.3.3.11} and \hyperref[1.3.3.9]{(3.3.9.1)}, and we immediately see that the diagonal morphism $\Delta_{X_{(S')}}$ can then be identified with $\Delta_X\times_Y 1_{Y_{(S')}}$;
the proposition then follows from \sref{1.4.3.1}.

To prove (v), consider, as in \sref{1.5.3.13}, the factorisation $X\xrightarrow{\Gamma_f}X\times_Z Y\xrightarrow{p_2}Y$ of $f$, noting that $p_2=(g\circ f)\times_Z 1_Y$;
the hypothesis (that $g\circ f$ is separated) implies that $g_2$ is separated, by (iii) and (i), and since $\Gamma_f$ is an immersion, $\Gamma_f$ is separated, by (i), whence $f$ is separated, by (ii).
Finally, to prove (vi), recall that the preschemes $X_\red\times_{Y_\red}X_\red$ and $X_\red\times_Y X_\red$ are canonically identified \sref{1.5.1.7};
if we denote by $j$ the injection $X_\red\to X$, then the diagram
\[
  \xymatrix{
    X_\red\ar[r]^{\Delta_{X_\red}}\ar[d]_j &
    X_\red\times_Y X_\red\ar[d]^{j\times_Y j}\\
    X\ar[r]^{\Delta_X} &
    X\times_Y X
  }
\]
commutes \sref{1.5.3.15}, and the proposition follows from the fact that the vertical arrows are the homeomorphisms of the underlying spaces \sref{1.4.3.1}.
\end{proof}

\begin{cor}[5.5.2]
\label{1.5.5.2}
If $f:X\to Y$ is separated, then the restriction of $f$ to any subprescheme of $X$ is separated.
\end{cor}

\begin{proof}
\label{proof-1.5.5.2}
This follows from \hyperref[1.5.5.1]{(5.5.1, (i) and (ii))}.
\end{proof}

\begin{cor}[5.5.3]
\label{1.5.5.3}
If $X$ and $Y$ are $S$-preschemes such that $Y$ is separated over $S$, then $X\times_S Y$ is separated over $X$.
\end{cor}

\begin{proof}
\label{proof-1.5.5.3}
This is a particular case of \sref{1.5.5.1}[iv].
\end{proof}

\begin{prop}[5.5.4]
\label{1.5.5.4}
Let $X$ be a prescheme, and assume that its underlying space is a \emph{finite} union of closed subsets $X_k$ ($1\leq k\leq n$);
for each $k$, consider the reduced subprescheme of $X$ that has $X_k$ as its underlying space \sref{1.5.2.1}, and denote this again by $X_k$.
Let $f:X\to Y$ be a morphism, and for each $k$, let $Y_k$ be a closed subset of $Y$ such that $f(X_k)\subset Y_k$;
we again denote by $Y_k$ the reduced subprescheme of $Y$ that has $Y_k$ as its underlying space, so that the restriction $X_k\to Y$ of $f$ to $X_k$ factors as $X_k\xrightarrow{f_k}Y_k\to Y$ \sref{1.5.2.2}.
For $f$ to be separated, it is necessary and sufficient for the $f_k$ to be separated.
\end{prop}

\begin{proof}
\label{proof-1.5.5.4}
The necessity follows from \hyperref[1.5.5.1]{(5.5.1, (i), (ii), and (v))}.
Conversely, if the condition of the statement is satisfied, then each of the restrictions $X_k\to Y$ of $f$ is separated \hyperref[1.5.5.1]{(5.5.1, (i) and (ii))};
if $p_1$ and $p_2$ are the projections of $X\times_Y X$, then the subspace $\Delta_{X_k}(X_k)$ is identified with the subspace $\Delta_X(X)\cap p_1^{-1}(X_k)$ of the underlying space of $X\times_Y X$ \sref{1.5.3.16};
these subspaces are closed in $X\times_Y X$, and thus so too is their union $\Delta_X(X)$.
\end{proof}

Suppose, in particular, that the $X_k$ are the \emph{irreducible components} of $X$;
then we can suppose that the $Y_k$ are the irreducible components of $Y$ \sref[0]{0.2.1.5};
prop.~\sref{1.5.5.4} then, in this case, leads to the idea of separation in the case of \emph{integral} preschemes \sref{1.2.1.7}.

\oldpage[I]{138}
\begin{prop}[5.5.5]
\label{1.5.5.5}
Let $(Y_\lambda)$ be an open cover of a prescheme $Y$;
for a morphism $f:X\to Y$ to be separated, it is necessary and sufficient for each of its restrictions $f^{-1}(Y_\lambda)\to Y_\lambda$ to be separated.
\end{prop}

\begin{proof}
\label{proof-1.5.5.5}
If we set $X_\lambda=f^{-1}(Y_\lambda)$, everything relies, taking \sref{1.4.2.4}[b] and the identification of the products $X_\lambda\times_Y X_\lambda$ and $X_\lambda\times_{Y_\lambda}X_\lambda$, on proving that the $X_\lambda\times_Y X_\lambda$ form a cover of $X\times_Y X$.
But if we set $Y_{\lambda\mu}=Y_\lambda\cap Y_\mu$ and $X_{\lambda\mu}=X_\lambda\cap X_\mu=f^{-1}(Y_{\lambda\mu})$, then $X_\lambda\times_Y X_\mu$ can be identified with the product $X_{\lambda\mu}\times_{Y_{\lambda\mu}}X_{\lambda\mu}$ \hyperref[1.3.2.6]{(3.2.6.4)}, and so also with $X_{\lambda\mu}\times_Y X_{\lambda\mu}$ \sref{1.3.2.5}, and finally with an open subset of $X_\lambda\times_Y X_\lambda$, which proves our claim \sref{1.3.2.7}.
\end{proof}

Prop.~\sref{1.5.5.4} allows us, by taking an affine open cover of $Y$, to restrict our study of separated morphisms to only those that take values in affine schemes.

\begin{prop}[5.5.6]
\label{1.5.5.6}
Let $Y$ be an \emph{affine} scheme, $X$ a prescheme, and $(U_\alpha)$ a cover of $X$ by affine open subsets.
For a morphism $f:X\to Y$ to be separated, it is necessary and sufficient for $U_\alpha\cap U_\beta$ to be an affine open subset for every pair of indices $(\alpha,\beta)$, and for the ring $\Gamma(U_\alpha\cap U_\beta,\OO_X)$ to be generated by the union of the canonical images of the rings $\Gamma(U_\alpha,\OO_X)$ and $\Gamma(U_\beta,\OO_X)$.
\end{prop}

\begin{proof}
\label{proof-1.5.5.6}
The $U_\alpha\times_Y U_\beta$ form an open cover of $X\times_Y X$ \sref{1.3.2.7};
denoting the projections of $X\times_Y X$ by $p$ and $q$, we have
\begin{align*}
  \Delta_X^{-1}(U_\alpha\times_Y U_\beta)&=\Delta_X^{-1}(p^{-1}(U_\alpha)\cap q^{-1}(U_\beta))\\
                                         &=\Delta_X^{-1}(p^{-1}(U_\alpha))\cap\Delta_X^{-1}(q^{-1}(U_\beta))=U_\alpha\cap U_\beta;
\end{align*}
so everything relies on proving that the restriction of $\Delta_X$ to $U_\alpha\cap U_\beta$ is a closed immersion into $U_\alpha\times_Y U_\beta$.
But this restriction is exactly $(j_\alpha,j_\beta)_Y$, where $j_\alpha$ (resp. $j_\beta$) denotes the injection morphism from $U_\alpha\cap U_\beta$ to $U_\alpha$ (resp. $U_\beta$), as follows from the definitions.
Since $U_\alpha\times_Y U_\beta$ is an affine scheme whose ring is canonically isomorphic to $\Gamma(U_\alpha,\OO_X)\otimes_{\Gamma(Y,\OO_Y)}\Gamma(U_\beta,\OO_X)$ \sref{1.3.2.2}, we see that $U_\alpha\cap U_\beta$ must be an affine scheme, and that the map $h_\alpha\otimes h_\beta\mapsto h_\alpha h_\beta$ from the ring $A(U_\alpha\times_Y U_\beta)$ to $\Gamma(U_\alpha\cap U_\beta,\OO_X)$ must be surjective \sref{1.4.2.3}, which finishes the proof.
\end{proof}

\begin{cor}[5.5.7]
\label{1.5.5.7}
An affine scheme is separated \emph{(and is thus a \emph{scheme}, which justifies the terminology of \sref{1.5.4.1})}.
\end{cor}

\begin{cor}[5.5.8]
\label{1.5.5.8}
Let $Y$ be an affine scheme;
for $f:X\to Y$ to be a separated morphism, it is necessary and sufficient for $X$ to be separated \emph{(in other words, for $X$ to be a \emph{scheme})}.
\end{cor}

\begin{proof}
\label{proof-1.5.5.8}
We see indeed that the criteria of \sref{1.5.5.6} does not depend on $f$.
\end{proof}

\begin{cor}[5.5.9]
\label{1.5.5.9}
For a morphism $f:X\to Y$ to be separated, it is necessary and sufficient for the induced prescheme $f^{-1}(U)$ to be separated, for every open subset of $U$ on which $Y$ induces a separated prescheme, and it is sufficient for it to be the case for every affine open subset $U\subset Y$.
\end{cor}

\begin{proof}
\label{proof-1.5.5.9}
The necessity of the condition follows from \sref{1.5.5.4} and \sref{1.5.5.1}[ii];
the sufficiency follows from \sref{1.5.5.4} and \sref{1.5.5.8}, taking into account the existence of affine open covers of $Y$.
\end{proof}

In particular, if $X$ and $Y$ are affine schemes, then \emph{every} morphism $X\to Y$ is separated.

\begin{prop}[5.5.10]
\label{1.5.5.10}
Let $Y$ be a \emph{scheme}, and $f:X\to Y$ a morphism.
For every affine open subset $U$ of $X$ and every affine open subset $V$ of $Y$, $U\cap f^{-1}(V)$ is affine.
\end{prop}

\begin{proof}
\label{proof-1.5.5.10}
Let $p_1$ and $p_2$ be the projections of $X\times_\bb{Z} Y$;
the subspace $U\cap f^{-1}(V)$ is the image of $\Gamma_f(X)\cap p_1^{-1}(U)\cap p_2^{-1}(V)$ under $p_1$.
But $p_1^{-1}(U)\cap p_2^{-1}(V)$ can be identified with the underlying space of the
\oldpage[I]{139}
prescheme $U\times_\bb{Z} V$ \sref{1.3.2.7}, and is thus an affine scheme \sref{1.3.2.2};
since $\Gamma_f(X)$ is closed in $X\times_\bb{Z} Y$ \sref{1.5.4.3}, $\Gamma_f(X)\cap p_1^{-1}(U)\cap p_2^{-1}(V)$ is closed in $U\times_\bb{Z} V$, and so the prescheme induced by the subprescheme of $X\times_\bb{Z} Y$ associated to $\Gamma_f$ \sref{1.4.2.1} on the open subset $\Gamma_f(X)\cap p_1^{-1}(U)\cap p_2^{-1}(V)$ of its underlying space is a closed subprescheme of an affine scheme, and thus an affine scheme \sref{1.4.2.3}.
The proposition then follows from the fact that $\Gamma_f$ is an immersion.
\end{proof}

\begin{exm}[5.5.11]
\label{1.5.5.11}
The prescheme from example~\sref{1.2.3.2} (``the projective line over a field $K$'') is \emph{separated}, because, for the cover $(X_1,X_2)$ of $X$ by affine open subsets, $X_1\cap X_2=U_{12}$ is affine, and $\Gamma(U_{12},\OO_X)$, the ring of rational fractions of the form $f(s)/s^m$ with $f\in K[s]$, is generated by $K[s]$ and $1/s$, so the conditions of \sref{1.5.5.6} are satisfied.

With the same choice of $X_1$, $X_2$, $U_{12}$, and $U_{21}$ as in example~\sref{1.2.3.2}, now take $u_{12}$ to be the isomorphism which sends $f(s)$ to $f(t)$;
we now obtain, by gluing, a \emph{non-separated integral} prescheme $X$, because the first condition of \sref{1.5.5.6} is satisfied, but not the second.
It is immediate here that $\Gamma(X,\OO_X)\to\Gamma(X_1,\OO_X)=K[s]$ is an isomorphism;
the inverse isomorphism defines a morphism $f:X\to\Spec(K[s])$ that is surjective, and for every $y\in\Spec(K[s])$ such that $\fk{j}_y\neq(0)$, $f^{-1}(y)$ reduces to a single point, but for $\fk{j}_y=(0)$, $f^{-1}(y)$ consists of \emph{two} distinct points (we say that $X$ is the ``affine line over $K$, where the point $0$ is doubled'').

We can also give examples where \emph{neither} of the two conditions of \sref{1.5.5.6} are satisfied.
First note that, in the prime spectrum $Y$ of the ring $A=K[s,t]$ of polynomials in two indeterminates over a field $K$, the open subset $U$ given by the union of $D(s)$ and $D(t)$ is \emph{not an affine open subset}.
Indeed, if $z$ is a section of $\OO_Y$ over $U$, there exist two integers $m,n\geq0$ such that $s^mz$ and $t^nz$ are the restrictions of polynomials in $s$ and $t$ to $U$ \sref{1.1.4.1}, which is clearly possible only if the section $z$ extends to a section over the whole of $Y$, identified with a polynomial in $s$ and $t$.
If $U$ were an affine open subset, then the injection morphism $U\to Y$ would be an isomorphism \sref{1.1.7.3}, which is a contradiction, since $U\neq Y$.

With the above in mind, take two affine schemes $Y_1$ and $Y_2$, prime spectra of the rings $A_1=K[s_1,t_2]$ and $A_2=K[s_2,t_2]$ (respectively);
take $U_{12}=D(s_1)\cup D(t_1)$ and $U_{21}=D(s_2)\cup D(t_2)$, and take $u_{12}$ to be the restriction of an isomorphism $Y_2\to Y_1$ to $U_{21}$ corresponding to the isomorphism of rings that sends $f(s_1,t_1)$ to $f(s_2,t_2)$;
we then have an example where the conditions of \sref{1.5.5.6} are not satisfied (the integral prescheme thus obtained is called ``the affine plane over $K$, where the point $0$ is doubled'').
\end{exm}

\begin{rmk}[5.5.12]
\label{1.5.5.12}
Given some property \textbf{P} of morphisms of preschemes, consider the following propositions.
\begin{enumerate}[label=(\roman*)]
    \item \emph{Every closed immersion has property~\textbf{P}.}
    \item \emph{The composition of any two morphisms that both have property~\textbf{P} also has property~\textbf{P}.}
    \item \emph{If $f:X\to X'$ and $g:Y\to Y'$ are $S$-morphisms that have property~\textbf{P}, then $f\times_S g$ has property~\textbf{P}.}
\oldpage[I]{140}
    \item \emph{If $f:X\to Y$ is an $S$-morphism that has property~\textbf{P}, then every $S'$-morphism $f_{(S')}$ obtained by an extension $S'\to S$ of the base prescheme also has property~\textbf{P}.}
    \item \emph{If the composition $g\circ f$ of two morphisms $f:X\to Y$ and $g:Y\to Z$ has property~\textbf{P}, and $g$ is separated, then $f$ has property~\textbf{P}.}
    \item \emph{If a morphism $f:X\to Y$ has property~\textbf{P}, then so too does $f_\red$ \sref{1.5.1.5}.}
\end{enumerate}
If we suppose that (i) and (ii) are both true, then (iii) and (iv) are \emph{equivalent}, and (v) and (vi) are \emph{consequences} of (i), (ii), and (iii).

The first claim has already been shown \sref{1.3.5.1}.
Consider the factorisation \sref{1.5.3.13} $X\xrightarrow{\Gamma_f}X\times_Z Y\xrightarrow{p_2}Y$ of $f$;
the relation $p_2=(g\circ f)\times_Z 1_Y$ shows that, if $g\circ f$ has property~\textbf{P}, then so too is $p_2$, by (iii);
if $g$ is separated, then $\Gamma_f$ is a closed immersion \sref{1.5.4.3}, and so also has property~\textbf{P}, by (i);
finally, by (ii), $f$ has property~\textbf{P}.

Finally, consider the commutative diagram
\[
  \xymatrix{
    X_\red\ar[r]^{f_\red}\ar[d] &
    Y_\red\ar[d]\\
    X\ar[r]^f &
    Y,
  }
\]
where the vertical arrows are the closed immersions \sref{1.5.1.5}, and so have property~\textbf{P}, by (i).
The hypothesis that $f$ has property~\textbf{P} implies, by (ii), that $X_\red\xrightarrow{f_\red}Y_\red\to Y$ has property~\textbf{P};
finally, since a closed immersion is separated \sref{1.5.5.1}[i], $f_\red$ has property~\textbf{P}, by (v).

Note that, if we consider the propositions
\begin{enumerate}
  \item[(i')] \emph{Every immersion has property~\textbf{P}};
  \item[(v')] \emph{If $g\circ f$ has property~\textbf{P}, then so too does $f$};
\end{enumerate}
then the above arguments show that (v') is a consequence of (i'), (ii), and (iii).
\end{rmk}

\begin{env}[5.5.13]
\label{1.5.5.13}
Note that (v) and (vi) are again consequences of (i), (iii), and
\begin{enumerate}
  \item[(ii')] \emph{If $j:X\to Y$ is a closed immersion, and $g:Y\to Z$ is a morphism that has property~\textbf{P}, then $g\circ j$ has property~\textbf{P}}.
\end{enumerate}
Similarly, (v') is a consequence of (i'), (iii), and
\begin{enumerate}
  \item[(ii'')] \emph{If $j:X\to Y$ is an immersion, and $g:Y\to Z$ is a morphism that has property~\textbf{P}, then $g\circ j$ has property \textbf{P}}.
\end{enumerate}
This follows immediately from the arguments of \sref{1.5.5.12}.
\end{env}

\section{Finiteness conditions}
\label{section:I.6}

\subsection{Noetherian and locally Noetherian preschemes}
\label{subsection:I.6.1}

\begin{definition}[6.1.1]
\label{I.6.1.1}
We say that a prescheme $X$ is Noetherian (resp. locally Noetherian) if it is a finite union (resp. union) of affine open $V_\alpha$ in such a way that the ring of the of the induced scheme on each of the $V_\alpha$ is Noetherian.
\end{definition}

It follows immediately from \sref{I.1.5.2} that, if $X$ is locally Noetherian, then the structure sheaf $\sh{O}_X$ is a \emph{coherent sheaf of rings}, the question being a local one.
Every \emph{quasi-coherent $\sh{O}_X$-submodule}
\oldpage[I]{141}
(resp. quasi-coherent quotient $\sh{O}_X$-module) of a \emph{coherent} $\sh{O}_X$-module $\sh{F}$ is \emph{coherent}, as the question is once again a local one, and it suffices to apply \sref{I.1.5.1}, \sref{I.1.4.1}, and \sref{I.1.3.10}, combined with the fact that a submodule (resp. quotient module) of a module of finite type over a Noetherian ring is of finite type.
In particular, every \emph{quasi-coherent sheaf of ideals} of $\sh{O}_X$ is \emph{coherent}.

If a prescheme $X$ is a finite union (resp. union) of open subsets $W_\lambda$ in such a way that the preschemes induced on the $W_\lambda$ are Noetherian (resp. locally Noetherian), it is clear that $X$ is Noetherian (resp. locally Noetherian).

\begin{proposition}[6.1.2]
\label{I.6.1.2}
For a prescheme $X$ to be Noetherian, it is necessary and sufficient for it to be locally Noetherian and have a quasi-compact underlying space.
The underlying space itself is then also Noetherian.
\end{proposition}

\begin{proof}
\label{proof-I.6.1.2}
The first claim follows immediately from the definitions and \sref{I.1.1.10}[ii].
The second follows from \sref{I.1.1.6} and the fact that every space that is a finite union of Noetherian subspaces is itself Noetherian \sref[0]{0.2.2.3}.
\end{proof}

\begin{proposition}[6.1.3]
\label{I.6.1.3}
Let $X$ be an affine scheme given by a ring $A$.
The following conditions are equivalent:
\emph{(a)} $X$ is Noetherian;
\emph{(b)} $X$ is locally Noetherian;
\emph{(c)} $A$ is Noetherian.
\end{proposition}

\begin{proof}
\label{proof-I.6.1.3}
The equivalence between (a) and (b) follows from \sref{I.6.1.2} ant the fact that the underlying space of every affine scheme is quasi-compact \sref{I.1.1.10}; it is furthermore clear that (c) implies (a).
To see that (a) implies (c), we remark that there is a finite cover $(V_i)$ of $X$ by affine open subsets such that the ring $A_i$ of the prescheme induced on $V_i$ is Noetherian.
So let $(\mathfrak{a}_n)$ be an increasing sequence of ideals of $A$; by a canonical bijective correspondence, there is a corresponding sequence $(\widetilde{\mathfrak{a}}_n)$ of sheaves of ideals in $\widetilde{A}=\sh{O}_X$;
to see that the sequence $(\mathfrak{a}_n)$ is \unsure{stable}, it suffices to prove that the sequence $(\widetilde{\mathfrak{a}}_n)$ is.
But the restriction $\widetilde{\mathfrak{a}}_n|V_i$ is a quasi-coherent sheaf of ideals in $\sh{O}_X|V_i$, being the inverse image of $\widetilde{\mathfrak{a}}_n$ under the canonical injection $V_i\to X$ \sref[0]{0.5.1.4};
$\widetilde{\mathfrak{a}}_n|V_i$ is thus of the form $\widetilde{\mathfrak{a}}_{ni}$, where $\mathfrak{a}_{ni}$ is an ideal of $A_i$ \sref{I.1.3.7}.
Since $A_i$ is Noetherian, the sequence $(\mathfrak{a}_{ni})$ is stable for all $i$, whence the proposition.
\end{proof}

We note that the above argument proves also that \emph{if $X$ is a Noetherian prescheme, then every increasing sequence of coherent sheaves of ideals of $\sh{O}_X$ is \unsure{stable}}.

\begin{proposition}[6.1.4]
\label{I.6.1.4}
Every subprescheme of a Noetherian (resp. locally Noetherian) prescheme is Noetherian (resp. locally Noetherian).
\end{proposition}

\begin{proof}
\label{proof-I.6.1.4}
If suffices to give a proof for a Noetherian prescheme $X$;
further, by definition~\sref{I.6.1.1}, we can also restrict to the case where $X$ is an affine scheme.
Since every subprescheme of $X$ is a closed subprescheme of a prescheme induced on an open subset \sref{I.4.1.3}, we can restrict to the case of a subprescheme $Y$, either closed or induced on an open subset of $X$.
The proof in the case where $Y$ is closed is immediate, since if $A$ is the ring of $X$, we know that $Y$ is an affine scheme given by the ring $A/\mathfrak{J}$, where $\mathfrak{J}$ is an ideal of $A$ \sref{I.4.2.3};
since $A$ is Noetherian \sref{I.6.1.3}, so too is $A/\mathfrak{J}$.

Now suppose that $Y$ is open in $X$;
the underlying space of $Y$ is Noetherian \sref{I.6.1.2}, hence quasi-compact, and thus a finite union of open subsets $D(f_i)$ ($f_i\in A$);
everything reduces to showing the proposition in the case where $Y=D(f)$ with $f\in A$.
But then $Y$ is an affine scheme whose ring is isomorphic to $A_f$ \sref{I.1.3.6};
since $A$ is Noetherian \sref{I.6.1.3}, so too is $A_f$.
\end{proof}

\begin{env}[6.1.5]
\label{I.6.1.5}
We note that the \emph{product} of two Noetherian $S$-preschemes is not necessarily Noetherian, even if the preschemes are affine, since the tensor product of two Noetherian algebras in not necessarily a Noetherian ring (cf. \sref{I.6.3.8}).
\end{env}

\begin{proposition}[6.1.6]
\label{I.6.1.6}
If $X$ is a Noetherian prescheme, the nilradical $\sh{N}_X$ of $\sh{O}_X$ is nilpotent.
\end{proposition}

\begin{proof}
\label{proof-I.6.1.6}
We can in fact cover $X$ with a finite number of affine open subsets $U_i$, and it suffices to prove that there exists whole numbers $n_i$ such that $(\sh{N}_X|U_i)^{n_i}=0$;
if $n$ is the largest of the $n_i$, then we will have $\sh{N}_X^n=0$.
We can thus restrict to the case where $X=\Spec(A)$ is affine, with $A$ a Noetherian ring;
by \sref{I.5.1.1} and \sref{I.1.3.13}, it suffices to observe that the nilradical of $A$ is nilpotent (\cite[p.~127, cor.~4]{I-11}).
\end{proof}

\begin{corollary}[6.1.7]
\label{I.6.1.7}
Let $X$ be a Noetherian prescheme;
for $X$ to be an affine scheme, it is necessary and sufficient that $X_\red$ be affine.
\end{corollary}

\begin{proof}
\label{proof-I.6.1.7}
This follows from \sref{I.6.1.6} and \sref{I.5.1.10}.
\end{proof}

\begin{lemma}[6.1.8]
\label{I.6.1.8}
Let $X$ be a topological space, $x$ a point of $X$, and $U$ an open neighbourhood of $x$ having only a finite number of irreducible components.
Then there exists a neighbourhood $V$ of $x$ such that every open neighbourhood of $x$ contained in $V$ is connected.
\end{lemma}

\begin{proof}
\label{proof-I.6.1.8}
Let $U_i$ ($1\leq i\leq m$) be the irreducible components of $U$ not containing $x$;
the complement (in $U$) of the union of the $U_i$ is an open neighbourhood $V$ of $X$ inside $U$, and thus so too in $X$;
it is also, incidentally, the complement (in $X$) of the union of the irreducible components of $X$ that do not contain $x$ \sref[0]{0.2.1.6}.
So let $W$ be an open neighbourhood of $X$ contained in $V$.
The irreducible components of $W$ are the intersections of $W$ with the irreducible components of $U$ \sref[0]{0.2.1.6}, so these components contain $x$;
since they are connected, so too is $W$.
\end{proof}

\begin{corollary}[6.1.9]
\label{I.6.1.9}
A locally Noetherian topological space is locally connected (which implies, amongst other things, that its connected components are open).
\end{corollary}

\begin{proposition}[6.1.10]
\label{I.6.1.10}
Let $X$ be a locally Noetherian topological space.
The following conditions are equivalent.
\begin{enumerate}
  \item[{\rm(a)}] The irreducible components of $X$ are open.
  \item[{\rm(b)}] The irreducible components of $X$ are exactly its connected components.
  \item[{\rm(c)}] The connected components of $X$ are irreducible.
  \item[{\rm(d)}] Two distinct irreducible components of $X$ have an empty intersection.
\end{enumerate}
Finally, if $X$ is a prescheme, then these conditions are also equivalent to
\begin{enumerate}
  \item[{\rm(e)}] For every $x\in X$, $\Spec(\sh{O}_x)$ is irreducible (or, in other words, the nilradical of $\sh{O}_x$ is prime).
\end{enumerate}
\end{proposition}

\begin{proof}
\label{proof-I.6.1.10}
It is immediate that (a) implies (b), because an irreducible space is connected, and (a) implies that the irreducible components of $X$ are the sets that are both open and closed.
It is trivial that (b) implies (c); conversely, a closed set $F$ containing
\oldpage[I]{143}
a connected component $C$ of $X$, with $C$ distinct from $F$, cannot be irreducible, because not being connected means that $F$ is the union of two disjoint nonempty sets that are both open and closed in $F$, and thus closed in $X$; as a result, (c) implies (b).
We immediately conclude from this that (c) implies (d), since two distinct connected components have no points in common.

We have not yet used the fact that $X$ is locally Noetherian.
Suppose now that this is indeed the case, and we will show that (d) implies (a): by \sref[0]{0.2.1.6}, we can restrict ourselves to the case where the space $X$ is Noetherian, and so has only a finite number of irreducible components.
Since they are closed and pairwise disjoint, they are open.

Finally, the equivalence between (d) and (e) holds true even without the assumption that the underlying space of the prescheme $X$ is locally Noetherian.
We can in fact restrict ourselves to the case where $X=\Spec(A)$ is affine, by \sref[0]{0.2.1.6};
to say that $x$ is contained in only one single irreducible component of $X$ is to say that $\mathfrak{j}_x$ contains only one single minimal ideal of $A$ \sref{I.1.1.14}, which is equivalent to saying that $\mathfrak{j}_x\sh{O}_x$ contains only one single minimal ideal of $\sh{O}_x$, whence the conclusion.
\end{proof}

\begin{corollary}[6.1.11]
\label{I.6.1.11}
Let $X$ be a locally Noetherian space.
For $X$ to be irreducible, it is necessary and sufficient that $X$ be connected and nonempty, and that any two distinct irreducible components of $X$ have an empty intersection.
If $X$ is a prescheme, this latter condition is equivalent to asking that $\Spec(\sh{O}_x)$ be irreducible for all $x\in X$.
\end{corollary}

\begin{proof}
\label{proof-I.6.1.11}
The second claim has already been shown in \sref{I.6.1.10};
the only thing thus remaining to show is that the conditions in the first claim are sufficient.
But by \sref{I.6.1.10}, these conditions imply that the irreducible components of $X$ are exactly its connected components, and since $X$ is connected and nonempty, it is irreducible.
\end{proof}

\begin{corollary}[6.1.12]
\label{I.6.1.12}
Let $X$ be a locally Noetherian prescheme.
For $X$ to be integral, it is necessary and sufficient that $X$ be connected and that $\sh{O}_x$ be integral for all $x\in X$.
\end{corollary}

\begin{proposition}[6.1.13]
\label{I.6.1.13}
Let $X$ be a locally Noetherian prescheme, and let $x\in X$ be a point such that the nilradical $\sh{N}_x$ of $\sh{O}_x$ is prime (resp. such that $\sh{O}_x$ is reduced, resp. integral);
then there exists an open neighbourhood $U$ of $x$ that is irreducible (resp. reduced, resp. integral).
\end{proposition}

\begin{proof}
\label{proof-I.6.1.13}
It suffices to consider two cases: where $\sh{N}_x$ is prime, and where $\sh{N}_x=0$; the third hypotheses is a combination of the first two.
If $\sh{N}_x$ is prime, then $x$ belongs to only one single irreducible component $Y$ of $X$ \sref{I.6.1.10};
the union of the irreducible components of $X$ that do not contain $x$ is closed (the set of these components being locally finite), and the complement $U$ of this union is thus open and contained in $Y$, and thus irreducible \sref[0]{0.2.1.6}
If $\sh{N}_x=0$, we also have $\sh{N}_y=0$ for any $y$ in a neighbourhood of $x$, because $\sh{N}$ is quasi-coherent \sref{I.5.1.1}, and thus coherent, since $X$ is locally Noetherian, and the conclusion then follows from \sref[0]{0.5.2.2}.
\end{proof}

\subsection{Artinian preschemes}
\label{subsection:I.6.2}

\begin{definition}[6.2.1]
\label{I.6.2.1}
We say that a prescheme is \emph{Artinian} if it is affine, and given by an Artinian ring.
\end{definition}

\begin{proposition}[6.2.2]
\label{I.6.2.2}
Given
\oldpage[I]{144}
a prescheme $X$, the following conditions are equivalent:
\begin{enumerate}
  \item[{\rm(a)}] $X$ is an Artinian scheme;
  \item[{\rm(b)}] $X$ is Noetherian and its underlying space is discrete;
  \item[{\rm(c)}] $X$ is Noetherian and the points of its underlying space are closed \emph{(the $\mathrm{T}_1$ condition)}.
\end{enumerate}
When any of the above hold, the underlying space of $X$ is finite, and the ring $A$ of $X$ is the direct sum of local (Artinian) rings of points of $X$.
\end{proposition}

\begin{proof}
\label{proof-I.6.2.2}
We know that (a) implies the last claim (\cite[p.~205, th.~3]{I-13}), so every prime ideal of $A$ is thus maximal and is the inverse image of a maximal ideal of one of the local components of $A$, and so the space $X$ is finite and discrete;
(a) thus implies (b), and (b) clearly implies (c).
To see that (c) implies (a), we first show that $X$ is then finite;
we can indeed restrict to the case where $X$ is affine, and we know that a Noetherian ring whose prime ideals are all maximal is Artinian (\cite[p.~203]{I-13}), whence our claim.
The underlying space $X$ is then discrete, the topological sum of a finite number of points $x_i$, and the local rings $\sh{O}_{x_i}=A_i$ are Artinian;
it is clear that $X$ is isomorphic to the prime spectrum affine scheme of the ring $A$ (the direct sum of the $A_i$) \sref{I.1.7.3}.
\end{proof}

\subsection{Morphisms of finite type}
\label{subsection:I.6.3}

\begin{definition}[6.3.1]
\label{I.6.3.1}
We say that a morphism $f:X\to Y$ is \emph{of finite type} if $Y$ is the union of a family $(V_\alpha)$ of affine open subsets having the following property:
\begin{enumerate}
  \item[(P)] $f^{-1}(V_\alpha)$ is a finite union of affine open subsets $U_{\alpha i}$ that are such that each ring $A(U_{\alpha i})$ is an algebra of finite type over $A(V_\alpha)$.
\end{enumerate}
We then say that $X$ is a prescheme of finite type over $Y$, or a $Y$-prescheme of finite type.
\end{definition}

\begin{proposition}[6.3.2]
\label{I.6.3.2}
If $f:X\to Y$ is a morphism of finite type, then every affine open subset $W$ of $Y$ satisfies property \emph{(P)} of \sref{I.6.3.1}.
\end{proposition}

We first show
\begin{lemma}[6.3.2.1]
\label{I.6.3.2.1}
If $T\subset Y$ is an affine open subset, satisfying property \emph{(P)}, then, for every $g\in A(T)$, $D(g)$ also satisfies property \emph{(P)}.
\end{lemma}

\begin{proof}
\label{proof-I.6.3.2.1}
By hypothesis, $f^{-1}(T)$ is a finite union of affine open subsets $Z_j$, that are such that $A(Z_j)$ is an algebra of finite type over $A(T)$;
let $\vphi_j:A(T)\to A(Z_j)$ be the homomorphism of rings corresponding to the restriction of $f$ to $Z_j$ \sref{I.2.2.4}, and set $g_j=\vphi_j(g)$;
we then have $f^{-1}(D(g))\cap Z_j=D(g_j)$ (1.2.2.2).
But $A(D(g_j))=A(Z_j)_{g_j}=A(Z_j)[1/g_j]$ is of finite type over $A(Z_j)$, and \emph{a fortiori} over $A(T)$ by the hypothesis, and so also over $A(D(g))=A(T)[1/g]$, which proves the lemma.
\end{proof}

\begin{proof}
\label{proof-I.6.3.2}
With the above lemma, since $W$ is quasi-compact \sref{I.1.1.10}, there exists a finite covering of $W$ by sets of the form \erratum[II]{$D(g_i)\subset W$}, where each $g_i$ belongs to a ring $A(V_{\alpha(i)})$.
Each $D(g_i)$, being quasi-compact, is a finite union of sets $D(h_{ik})$, where $h_{ik}\in A(W)$;
if $\vphi_i:A(W)\to A(D(g_i))$ is the canonical map, then we have $D(h_{ik})=D(\vphi_i(h_{ik}))$ by (1.2.2.2).
By \sref{I.6.3.2.1}, each of the $f^{-1}(D(h_{ik}))$ admits a finite covering by affine open subsets $U_{ijk}$, that are such that the $A(U_{ijk})$ are algebras of finite type over $A(D(h_{ik}))=A(W)[1/h_{ik}]$, whence the proposition.
\end{proof}

We
\oldpage[I]{145}
can thus say that the notion of a prescheme of finite type over $Y$ is \emph{local on $Y$}.

\begin{proposition}[6.3.3]
\label{I.6.3.3}
Let $X$ and $Y$ be affine schemes;
for $X$ to be of finite type over $Y$, it is necessary and sufficient that $A(X)$ be an algebra of finite type over $A(Y)$.
\end{proposition}

\begin{proof}
\label{proof-I.6.3.3}
Since the condition clearly suffices, we show that it is necessary.
Set $A=A(Y)$ and $B=A(X)$;
by \sref{I.6.3.2}, there exists a finite affine open cover $(V_i)$ of $X$ such that each of the rings $A(V_i)$ is an $A$-algebra of finite type.
Further, since the $V_i$ are quasi-compact, we can cover each of them with a finite number of open subsets of the form $D(g_{ij})\subset V_i$, where $g_{ij}\in B$;
if $\vphi_i$ is a homomorphism $B\to A(V_i)$ that corresponds to the canonical injection $V_i\to X$, then we have $B_{g_{ij}}=(A(V_i))_{\vphi_i(g_{ij})}=A(V_i)[1/\vphi_i(g_{ij})]$, so $B_{g_{ij}}$ is an $A$-algebra of finite type.
We can thus restrict to the case where $V_i=D(g_i)$ with $g_i\in B$.
By hypothesis, there exists a finite subset $F_i$ of $B$ and an integer $n_i\geq0$ such that $B_{g_i}$ is the algebra generated over $A$ by the elements $b_i/g_i^{n_i}$, where the $b_i$ run over all of $F_i$.
Since there are only finitely many of the $g_i$, we can assume that all the $n_i$ are equal to the same integer $n$.
Further, since the $D(g_i)$ form a cover of $X$, the ideal generated in $B$ by the $g_i$ is equal to $B$, or, in other words, there exist $h_i\in B$ such that $\sum_i h_ig_i=1$.
So let $F$ be the finite subset of $B$ given by the union of the $F_i$, the set of the $g_i$, and the set of the $h_i$; we will show that the subring $B'=A[F]$ of $B$ is equal to $B$.
By hypothesis, for every $b\in B$ and every $i$, the canonical image of $b$ in $B_{g_i}$ is of the form $b'_i/g_i^{m_i}$, where $b'_i\in B'$;
by multiplying the $b'_i$ by suitable powers of the $g_i$, we can again assume that all the $m_i$ are equal to the same integer $m$.
By the definition of the ring of fractions, there is thus an integer $N$ (dependant on $b$) such that $N\geq m$ and $g_i^Nb\in B'$ for all $i$;
but, \emph{in the ring $B'$}, the $g_i^N$ generate the ideal $B'$, because the $g_i$ do (and the $h_i$ belong to $B'$);
there are thus $c_i\in B'$ such that $\sum_i c_ig_i^N=1$, whence $b=\sum_ic_ig_i^Nb\in B'$, Q.E.D.
\end{proof}

\begin{proposition}[6.3.4]
\label{I.6.3.4}
\medskip\noindent
\begin{enumerate}
  \item[{\rm(i)}] Every closed immersion is of finite type.
  \item[{\rm(ii)}] The composition of any two morphisms of finite type is of finite type.
  \item[{\rm(iii)}] If $f:X\to X'$ and $g:Y\to Y'$ are $S$-morphisms of finite type, then $f\times_S g$ is of finite type.
  \item[{\rm(iv)}] If $f:X\to Y$ is an $S$-morphism of finite type, then $f_{(S')}$ is of finite type for any extension $g:S'\to S$ of the base prescheme.
  \item[{\rm(v)}] If the composition $g\circ f$ of two morphisms is of finite type, with $g$ separated, then $f$ is of finite type.
  \item[{\rm(vi)}] If a morphism $f$ is of finite type, then $f_\red$ is of finite type.
\end{enumerate}
\end{proposition}

\begin{proof}
\label{proof-I.6.3.4}
By \sref{I.5.5.12}, it suffices to prove (i), (ii), and (iv).

To show (i), we can restrict to the case of a canonical injection $X\to Y$, with $X$ being a closed subprescheme of $Y$;
further \sref{I.6.3.2}, we can assume that $Y$ is affine, in which case $X$ is also affine \sref{I.4.2.3} and its ring is isomorphic to a quotient ring $A/\mathfrak{J}$, where $A$ is the ring of $Y$ and $\mathfrak{J}$ is an ideal of $A$;
since $A/\mathfrak{J}$ is of finite type over $A$, the conclusion follows.

Now we show (ii).
Let $f:X\to Y$ and $g:Y\to Z$ be two morphisms of finite type, and let $U$ be an affine open subset of $Z$;
$g^{-1}$ admits a finite covering by affine open subsets $V_i$ that are such that each $A(V_i)$ is an algebra of finite type over $A(U)$ \sref{I.6.3.2};
similarly. each
\oldpage[I]{146}
of the $f^{-1}$ admits a finite cover by affine open subsets $W_{ij}$ that are such that each $A(W_{ij})$ is an algebra of finite type over $A(V_i)$, and so also an algebra of finite type over $A(U)$, whence the conclusion.

Finally, to show (iv), we can restrict to the case where $S=Y$;
then $f_{(S')}$ is also equal to $f_{Y_{(S')}}$, where we consider $f$ as a $Y$-morphism, and the base extension is $Y_{(S')}\to Y$ \sref{I.3.3.9}.
So let $p$ and $q$ be the projections $X_{(S')}\to X$ and $X_{(S')}\to S'$.
Let $V$ be an affine open subset of $S$;
$f^{-1}(V)$ is a finite union of affine open subsets $W_i$, each of which is such that $A(W_i)$ is an algebra of finite type over $A(V)$ \sref{I.6.3.2}.
Let $V'$ be an affine open subset of $S'$ contained in $g^{-1}(V)$;
since $f\circ p=g\circ q$, $q^{-1}(V')$ is contained in the union of the $p^{-1}(W_i)$;
on the other hand, the intersection $p^{-1}(W_i)\cap q^{-1}(V')$ can be identified with the product $W_i\times V V'$ \sref{I.3.2.7}, which is an affine scheme whose ring is isomorphic to $A(W_i)\otimes_{A(V)}A(V')$ \sref{I.3.2.2};
this ring is, by hypothesis, an algebra of finite type over $A(V')$, which proves the proposition.
\end{proof}

\begin{corollary}[6.3.5]
\label{I.6.3.5}
Let $f:X\to Y$ be an immersion morphism.
If the underlying space of $Y$ (resp. $X$) is locally Noetherian (resp. Noetherian), then $f$ is of finite type.
\end{corollary}

\begin{proof}
\label{proof-I.6.3.5}
We can always assume that $Y$ is affine \sref{I.6.3.2};
if the underlying space of $Y$ is locally Noetherian, then we can further assume that it is Noetherian, and then the underlying space of $X$, which is a subspace, is also Noetherian.
In other words, we can assume that $Y$ is affine and that the underlying space of $X$ is Noetherian;
there then exists a covering of $X$ by a finite number of affine open subsets $D(g_i)\subset Y$, where $g_i\in A(Y)$, that are such that the $X\cap D(g_i)$ are closed in $D(g_i)$ (and thus affine schemes \sref{I.4.2.3}), because $X$ is locally closed in $Y$ \sref{I.4.1.3}.
Then $A(X\cap D(g_i))$ is an algebra of finite type over $A(D(g_i))$, by \sref{I.6.3.4}[i] and \sref{I.6.3.3}, and $A(D(g_i))=A(Y)_{g_i}=A(Y)[1/g_i]$ is of finite type over $A(Y)$, which finishes the proof.
\end{proof}

\begin{corollary}[6.3.6]
\label{I.6.3.6}
Let $f:X\to Y$ and $g:Y\to Z$ be morphisms.
If $g\circ f$ if of finite type, with either $X$ Noetherian or $X\times_Z Y$ locally Noetherian, then $f$ is of finite type.
\end{corollary}

\begin{proof}
\label{proof-I.6.3.6}
This follows immediately from the proof of \sref{I.5.5.12} and from \sref{I.6.3.5} applied to the immersion morphism $\Gamma_f$.
\end{proof}

\begin{proposition}[6.3.7]
\label{I.6.3.7}
Let $f:X\to Y$ be a morphism of finite type;
if $Y$ is Noetherian (resp. locally Noetherian), then $X$ is Noetherian (resp. locally Noetherian).
\end{proposition}

\begin{proof}
\label{proof-I.6.3.7}
We can restrict to proving the proposition for when $Y$ is Noetherian.
Then $Y$ is a finite union of affine open subsets $V_i$ that are such that the $A(V_i)$ are Noetherian rings.
By \sref{I.6.3.2}, each of the $f^{-1}(V_i)$ is a finite union of affine open subsets $W_{ij}$ that are such that the $A(W_{ij})$ are algebras of finite type over $A(V_i)$, and thus Noetherian rings;
this proves that $X$ is Noetherian.
\end{proof}

\begin{corollary}[6.3.8]
\label{I.6.3.8}
Let $X$ be a prescheme of finite type over $S$.
For every base extension $S'\to S$ with $S'$ Noetherian (resp. locally Noetherian), $X_{(S')}$ is Noetherian (resp. locally Noetherian).
\end{corollary}

\begin{proof}
\label{proof-I.6.3.8}
This follows from \sref{I.6.3.7}, since $X_{(S')}$ is of finite type over $S'$ by \sref{I.6.3.4}[iv].
\end{proof}

We can also says that, for a product $X\times_S Y$ of $S$-preschemes, if \emph{one} of the factors
\oldpage[I]{147}
$X$ or $Y$ is \emph{of finite type} over $S$ and \emph{the other is Noetherian} (resp. \emph{locally Noetherian}), then $X\times_S Y$ is \emph{Noetherian} (resp. \emph{locally Noetherian}).

\begin{corollary}[6.3.9]
\label{I.6.3.9}
Let $X$ be a prescheme of finite type over a locally Noetherian prescheme $S$.
Then every $S$-morphism $f:X\to Y$ is of finite type.
\end{corollary}

\begin{proof}
\label{proof-I.6.3.9}
In fact, we can assume that $S$ is Noetherian;
if $\vphi:X\to S$ and $\psi:Y\to S$ are the structure morphisms, then we have $\vphi=\psi\circ f$, and $X$ is Noetherian by \sref{I.6.3.7};
$f$ is thus of finite type by \sref{I.6.3.6}.
\end{proof}

\begin{proposition}[6.3.10]
\label{I.6.3.10}
Let $f:X\to Y$ be a morphism of finite type.
For $f$ to be surjective, it is necessary and sufficient that, for every \emph{algebraically closed} field $\Omega$, the map $X(\Omega)\to Y(\Omega)$ that corresponds to $f$ \sref{I.3.4.1} be surjective.
\end{proposition}

\begin{proof}
\label{proof-I.6.3.10}
The condition suffices, as we can see by considering, for all $y\in Y$, an algebraically closed extension $\Omega$ of $\kres(y)$, and the commutative diagram
\[
  \xymatrix{
    X\ar[dd]_f\\
    & \Spec(\Omega)\ar[ul]\ar[dl]\\
    Y
  }
\]
(cf. \sref{I.3.5.3}).
Conversely, suppose that $f$ is surjective, and let $g:\{\xi\}=\Spec(\Omega)\to Y$ be a morphism, where $\Omega$ is an algebraically closed field.
If we consider the diagram
\[
  \xymatrix{
    X\ar[d]_f &
    X_{(\Omega)}\ar[l]\ar[d]^{f_{(\Omega)}}\\
    Y &
    \Spec(\Omega),\ar[l]
  }
\]
then it suffices to show that there exists a \emph{rational point over $\Omega$} in $X_{(\Omega)}$ (\sref{I.3.3.14}, \sref{I.3.4.3}, and \sref{I.3.4.4}).
Since $f$ is surjective, $X_{(\Omega)}$ is nonempty \sref{I.3.5.10}, and since $f$ is of finite type, so too is $f_{(\Omega)}$ \sref{I.6.4.3}[iv];
thus $X_{(\Omega)}$ contains a nonempty affine open subset $Z$ such that $A(Z)$ is an non-null algebra of finite type over $\Omega$.
By Hilbert's Nullstellensatz~\cite{I-21}, there exists an $\Omega$-homomorphism $A(Z)\to\Omega$, and thus a section of $X_{(\Omega)}$ over $\Spec(\Omega)$, which proves the proposition.
\end{proof}

\subsection{Algebraic preschemes}
\label{subsection:I.6.4}

\begin{definition}[6.4.1]
\label{I.6.4.1}
Given a field $K$, we define an \emph{algebraic $K$-prescheme} to be a prescheme $X$ of finite type over $K$; $K$ is called the base field of $X$.
If in addition $X$ is a scheme \emph{(or if $X$ is a \emph{$K$-scheme}, which is equivalent \sref{I.5.5.8})}, we say that $X$ is an \emph{algebraic $K$-scheme}.
\end{definition}

Every algebraic $K$-prescheme is \emph{Noetherian} \sref{I.6.3.7}.

\begin{proposition}[6.4.2]
\label{I.6.4.2}
Let $X$ be an algebraic $K$-prescheme.
For a point $x\in X$ to be closed, it is necessary and sufficient that $\kres(x)$ be an algebraic extension of $K$ of finite degree.
\end{proposition}

\begin{proof}
\label{proof-I.6.4.2}
We can assume that $X$ is affine, with the ring $A$ of $X$ being a $K$-algebra of finite type.
Indeed, the affine open subsets $U$ of $X$ such that $A(U)$ is a $K$-algebra of finite type form a finite cover of $X$ \sref{I.6.3.1}.
The closed points of $X$ are thus the points such that $\mathfrak{j}_x$ is a
\oldpage[I]{148}
maximal ideal of $A$, or in other words, such that $A/\mathfrak{j}_x$ is a field (necessarily equal to $\kres(x)$).
Since $A/\mathfrak{j}_x$ is a $K$-algebra of finite type, we see that if $x$ is closed, then $\kres(x)$ is a field that is an algebra of finite type over $K$, and so necessarily a $K$-algebra of \emph{finite rank}~\cite{I-21}.
Conversely, if $\kres(x)$ is of finite rank over $K$, then so is $A/\mathfrak{j}_x\subset\kres(x)$, and since every integral ring that is also a $K$-algebra of finite rank is a field, we have that $A/\mathfrak{j}_x=\kres(x)$, and hence $x$ is closed.
\end{proof}

\begin{corollary}[6.4.3]
\label{I.6.4.3}
Let $K$ be an algebraically-closed field, and $X$ an algebraic $K$-prescheme; the closed points of $X$ are then the rational points over $K$ \sref{I.3.4.4} and can be canonically identified with the points of $X$ with values in $K$.
\end{corollary}

\begin{proposition}[6.4.4]
\label{I.6.4.4}
Let $X$ be an algebraic prescheme over a field $K$.
The following properties are equivalent.
\begin{enumerate}
  \item[{\rm(a)}] $X$ is Artinian.
  \item[{\rm(b)}] The underlying space of $X$ is discrete.
  \item[{\rm(c)}] The underlying space of $X$ has only a finite number of closed points.
  \item[{\rm(c')}] The underlying space of $X$ is finite.
  \item[{\rm(d)}] The points of $X$ are closed.
  \item[{\rm(e)}] $X$ is isomorphic to $\Spec(A)$, where $A$ is a $K$-algebra of finite rank.
\end{enumerate}
\end{proposition}

\begin{proof}
\label{proof-I.6.4.4}
Since $X$ is Noetherian, it follows from \sref{I.6.2.2} that the conditions (a), (b), and (d) are equivalent, and imply (c) and (c$'$);
it is also clear that (e) implies (a).
It remains to see that (c) implies (d) and (e);
we can restrict to the case where $X$ is affine.
Then $A(X)$ is a $K$-algebra of finite type \sref{I.6.3.3}, and thus a Jacobson ring (\cite[p.~3-11 and 3-12]{I-1}), in which there are, by hypothesis, only a finite number of maximal ideals.
Since a finite intersection of prime ideals can only be a prime ideal if it is equal to one of the prime ideals being intersected, every prime ideal of $A(X)$ is thus maximal, whence (d).
Further, we then know \sref{I.6.2.2} that $A(X)$ is an Artinian $K$-algebra of finite type, and so necessarily of \emph{finite rank}~\cite{I-21}.
\end{proof}

\begin{env}[6.4.5]
\label{I.6.4.5}
When the conditions of \sref{I.6.4.4} are satisfied, we say that $X$ is a scheme \emph{finite over $K$} (cf. \sref[II]{II.6.1.1}), or a \emph{finite $K$-scheme}, of \emph{rank} $[A:K]$, which we also denote by $\rg_K(X)$;
if $X$ and $Y$ are finite $K$-schemes, we have
\[
  \rg_K(X\sqcup Y)=\rg_K(X)+\rg_K(Y),
  \tag{6.4.5.1}
\]
\[
  \rg_K(X\times_K Y)=\rg_K(X)\rg_K(Y),
  \tag{6.4.5.2}
\]
as a result of \sref{I.3.2.2}.
\end{env}

\begin{corollary}[6.4.6]
\label{I.6.4.6}
Let $X$ be a finite $K$-scheme.
For every extension $K'$ of $K$, $X\otimes_K K'$ as a finite $K'$-scheme, and its rank over $K'$ is equal to the rank of $X$ over $K$.
\end{corollary}

\begin{proof}
\label{proof-I.6.4.6}
If $A=A(X)$, then we have $[A\otimes_K K':K']=[A:K]$.
\end{proof}

\begin{corollary}[6.4.7]
\label{I.6.4.7}
Let $X$ be a scheme finite over a field $K$;
we let $n=\sum_{x\in X}[\kres(X):K]_S$ \emph{(we recall that if $K'$ is an extension of $K$, then $[K':K]_S$ is the \emph{separable rank} of $K'$ over $k$, the rank of the largest algebraic separable extension of $K$ contained in $K'$)};
then
\oldpage[I]{149}
for every algebraically closed extension $\Omega$ of $K$, the underlying space of $X\otimes_K\Omega$ has exactly $n$ points, which can be identified with the points of $X$ with values in $\Omega$.
\end{corollary}

\begin{proof}
\label{proof-I.6.4.7}
We can clearly restrict to the case where the ring $A=A(X)$ is \emph{local} \sref{I.6.2.2};
let $\mathfrak{m}$ be its maximal ideal, and $L=A/\mathfrak{m}$ its residue field, an algebraic extension of $K$.
The points of $X$ with values in $\Omega$ then correspond, bijectively, to the $\Omega$-sections of $X\otimes_K\Omega$ (\sref{I.3.4.1} and \sref{I.3.3.14}), and also to the $K$-homomorphisms from $L$ to $\Omega$ \sref{I.1.7.3}, whence the proposition (Bourbaki, \emph{Alg.}, chap.~V, §7, n\textsuperscript{o}~5, prop.~8), taking \sref{I.6.4.3} into account.
\end{proof}

\begin{env}[6.4.8]
\label{I.6.4.8}
The number $n$ defined in \sref{I.6.4.7} is called the \emph{separable rank} of $A$ (or of $X$) over $K$, or also the \emph{geometric number of points} of $X$;
it is equal to the number of elements of $X(\Omega)_K$.
It follows immediately from this definition that, for every extension $K'$ of $K$, $X\otimes_K K'$ has the same geometric number of points as $X$.
If we denote this number by $n(X)$, it is clear that, if $X$ and $Y$ are two schemes, finite over $K$, then
\[
    n(X\sqcup Y)=n(X)+n(Y).\tag{6.4.8.1}
\]
Under the same hypotheses, we also have
\[
    n(X\times_K Y)=n(X)n(Y)\tag{6.4.8.2}
\]
because of the interpretation of $n(X)$ as the number of elements of $X(\Omega)_K$ and Equation~\hyperref[1.3.4.3]{(3.4.3.1)}.
\end{env}

\begin{proposition}[6.4.9]
\label{I.6.4.9}
Let $K$ be a field, $X$ and $Y$ algebraic $K$-preschemes, $f:X\to Y$ a $K$-morphism, and $\Omega$ an algebraically closed extension of $K$ of infinite transcendence degree over $K$.
For $f$ to be surjective, it is necessary and sufficient that the map $X(\Omega)_K\to Y(\Omega)_K$ that corresponds to $f$ \sref{I.3.4.1} be surjective.
\end{proposition}

\begin{proof}
\label{proof-I.6.4.9}
The necessity follows from \sref{I.6.3.10}, noting that $f$ is necessarily of finite type \sref{I.6.3.9}.
To see that the condition is sufficient, we argue as in \sref{I.6.3.10}, noting that, for every $y\in Y$, $\kres(y)$ is an extension of $K$ of finite type, and so is $K$-isomorphic to a subfield of $\Omega$.
\end{proof}

\begin{remark}[6.4.10]
\label{I.6.4.10}
We will see in chapter~IV that the conclusion of \sref{I.6.4.9} still holds without the hypothesis on the transcendence degree of $\Omega$ over $K$.
\end{remark}

\begin{proposition}[6.4.11]
\label{I.6.4.11}
If $f:X\to Y$ is a morphism of finite type, then, for every $y\in Y$, the fibre $f^{-1}(y)$ is an algebraic prescheme over the residue field $\kres(y)$, and for every $x\in f^{-1}(y)$, $\kres(x)$ is an extension of $\kres(x)$ of finite type.
\end{proposition}

\begin{proof}
\label{proof-I.6.4.11}
Since $f^{-1}(y)=X\otimes_Y\kres(y)$ \sref{I.6.3.6}, the proposition follows from \sref{I.6.3.4}[iv] and \sref{I.6.3.3}.
\end{proof}

\begin{proposition}[6.4.12]
\label{I.6.4.12}
Let $f:X\to Y$ and $g:Y'\to Y$ be morphisms;
set $X'=X\times_Y Y'$, and let $f'=f_{(Y')}:X'\to Y'$.
Let $y'\in Y'$ and set $y=g(y')$;
if the fibre $f^{-1}(y)$ is a finite algebraic scheme over $\kres(y)$, then the fibre $f'^{-1}(y)$ is a finite algebraic scheme over $\kres(y')$, and has the same rank and geometric number of points as $f^{-1}(y)$ does.
\end{proposition}

\begin{proof}
\label{proof-I.6.4.12}
Taking into account the transitivity of fibres \sref{I.3.6.5}, this follows immediately from \sref{I.6.4.6} and \sref{I.6.4.8}.
\end{proof}

\begin{env}[6.4.13]
\label{I.6.4.13}
Proposition~\sref{I.6.4.11} shows that the morphisms of finite type that correspond, intuitively, to the ``algebraic families of algebraic varieties'', with the points of $Y$ playing the
\oldpage[I]{150}
role of ``parameters'', which gives these morphisms a ``geometric'' meaning.
The morphisms which are not of finite type will show up in the following mostly in questions of ``changing the base prescheme'', by localisation or completion, for example.
\end{env}

\subsection{Local determination of a morphism}
\label{subsection:I.6.5}

\begin{proposition}[6.5.1]
\label{I.6.5.1}
Let $X$ and $Y$ be $S$-preschemes, with $Y$ of finite type over $S$;
let $x\in X$ and $y\in Y$ lie over the same point $s\in S$.
\begin{enumerate}
  \item[{\rm(i)}] If two $S$-morphisms $f=(\psi,\theta)$ and $f'=(\psi',\theta')$ from $X$ to $Y$ are such that $\psi(x)=\psi'(x)=y$, and the (local) $\sh{O}_s$-homomorphisms $\theta_x^\sharp$ and ${\theta'}_x^\sharp$ from $\sh{O}_y$ to $\sh{O}_x$ are identical, then $f$ and $f'$ agree on an open neighbourhood of $x$.
  \item[{\rm(ii)}] Suppose further that $S$ is locally Noetherian.
    For every local $\sh{O}_s$-homomorphism $\vphi:\sh{O}_y\to\sh{O}_x$, there exists an open neighbourhood $U$ of $x$ in $X$, and an $S$-morphism $f=(\psi,\theta)$ from $U$ to $Y$ such that $\psi(x)=y$ and $\theta_x^\sharp=\vphi$.
\end{enumerate}
\end{proposition}

\begin{proof}
\label{proof-I.6.5.1}
\medskip\noindent
\begin{enumerate}
  \item[(i)] Since the question is local on $S$, $X$, and $Y$, we can assume that $S$, $X$, and $Y$ are affine, given by rings $A$, $B$, and $C$ (respectively), and with $f$ and $f'$ of the form $({}^a\vphi,\widetilde{\vphi})$ and $({}^a\vphi',\widetilde{\vphi}')$ (respectively), where $\vphi$ and $\vphi'$ are $A$-homomorphisms from $C$ to $B$ such that $\vphi^{-1}(\mathfrak{j}_x)=\vphi'^{-1}(\mathfrak{j}_x)=\mathfrak{j}_y$, and the homomorphisms $\vphi_x$ and $\vphi'_x$ from $C_y$ to $B_x$, induced by $\vphi$ and $\vphi'$, are identical;
    we can further suppose that $C$ is an $A$-algebra \emph{of finite type}.
    Let $c_i$ ($1\leq i\leq n$) be the generators of the $A$-algebra $C$, and set $b_i=\vphi(c_i)$ and $b'_i=\vphi'(c_i)$;
    by hypothesis, we have $b_i/1=b'_i/1$ in the ring of fractions $B_x$ ($1\leq i\leq n$).
    This implies that there exist elements $s_i\in B\setmin\mathfrak{j}_x$ such that $s_i(b_i-b'_i)=0$ for $1\leq i\leq n$, and we can clearly assume that all the $s_i$ are equal to a single element $g\in B\setmin\mathfrak{j}_x$.
    From this, we conclude that we have $b_i/1=b'_i/1$ for $1\leq i\leq n$ in the ring of fractions $B_g$;
    if $i_g$ is the canonical homomorphism $B\to B_g$, we then have $i_g\circ\vphi=i_g\circ\vphi'$;
    so the restrictions of $f$ and $f'$ to $D(g)$ are identical.
  \item[(ii)] We can restrict to the situation as in (i), and further assume that the ring $A$ is Noetherian.
    Let $c_i$ ($1\leq i\leq n$) be the generators of the $A$-algebra $C$, and let $\alpha:A[X_1,\ldots,X_n]\to C$ be the homomorphism of polynomial algebras that sends $X_i$ to $c_i$ for $1\leq i\leq n$.
    Also let $i_y$ be the canonical homomorphism $C\to C_y$, and consider the composite homomorphism
    \[
      \beta:A[X_1,\ldots,X_n] \xrightarrow{\alpha} C \xrightarrow{i_y} C_y \xrightarrow{\vphi} B_x.
    \]
    We denote by $\mathfrak{a}$ the kernel of $\beta$;
    since $A$ is Noetherian, so too is $A[X_1,\ldots,X_n]$, and so $\mathfrak{a}$ admits a finite system of generators $Q_j(X_1,\ldots,X_n)$ ($1\leq j\leq m$).
    Furthermore, each of the elements $\vphi(i_y(c_i))$ can be written in the form $b_i/s_i$, where $b_i\in B$ and $s_i\not\in\mathfrak{j}_x$;
\oldpage[I]{151}
    we can further assume that all of the $s_i$ are equal to a single element $g\in B\setmin\mathfrak{j}_x$.
    With this, by hypothesis, we have $Q_j(b_1/g,\ldots,b_n/g)=0$ in $B_x$; set
    \[
      Q_j(X_1/T,\ldots,X_n/T)=P_j(X_1,\ldots,X_n,T)/T^{k_j}
    \]
    where $P_j$ is homogeneous of degree $k_j$.
    Then let $d_j=P_j(b_1,\ldots,b_n,g)\in B$.
    By hypothesis, we have $t_jd_j=0$ for some $t_j\in B\setmin\mathfrak{j}_x$ ($1\leq j\leq m$), and we can clearly assume that all the $t_j$ are equal to a single element $h\in B\setmin\mathfrak{j}_x$;
    from this we conclude that $P_j(hb_1,\ldots,hb_n,hg)=0$ for $1\leq j\leq m$.
    With this, consider the homomorphism $\rho$ from $A[X_1,\ldots,X_n]$ to the ring of fractions $B_{hg}$ which sends $X_i$ to $hb_i/hg$ ($1\leq i\leq n$);
    the image of $\mathfrak{a}$ under this homomorphism is $0$, and is \emph{a fortiori} the same as the image of the kernel $\alpha^{-1}(0)$ under $\rho$.
    So $\rho$ factors as $A[X_1,\ldots,X_n]\xrightarrow{\alpha}C\xrightarrow{\gamma}B_{hg}$, with $\gamma(c_i)=hb_i/hg$, and it is clear that, if $i_x$ is the canonical homomorphism $B_{hg}\to B_x$, then the diagram
    \[
    \label{I.6.5.1.1}
      \xymatrix{
        C\ar[r]^\gamma\ar[d]_{i_y} &
        B_{hg}\ar[d]^{i_x}\\
        C_y\ar[r]^\vphi &
        B_x
      }
      \tag{6.5.1.1}
    \]
    is commutative; we thus have $\vphi=\gamma_x$, and since $\vphi$ is a local homomorphism, $^a\gamma(x)=y$;
    $f=({}^a\gamma,\widetilde{\gamma})$ is thus an $S$-morphism from the neighbourhood $D(hg)$ of $x$ to $Y$ as claimed in the proposition.
\end{enumerate}
\end{proof}

\begin{corollary}[6.5.2]
\label{I.6.5.2}
Under the hypotheses of \sref{I.6.5.1}[ii], if, further, $X$ is of finite type over $S$, then we can assume that the morphism $f$ is of finite type.
\end{corollary}

\begin{proof}
\label{proof-I.6.5.2}
This follows from Corollary~\sref{I.6.3.6}.
\end{proof}

\begin{corollary}[6.5.3]
\label{I.6.5.3}
Suppose that the hypotheses of Proposition~\sref{I.6.5.1}[ii], and suppose further that $Y$ is integral, and that $\vphi$ is an injective homomorphism.
Then we can assume that $f=({}^a\gamma,\widetilde{\gamma})$, where $\gamma$ is injective.
\end{corollary}

\begin{proof}
\label{proof-I.6.5.3}
Indeed, we can assume $C$ to be integral \sref{I.5.1.4}, hence $i_y$ injective; it then follows from the diagram (6.5.1.1) that $\gamma$ is injective.
\end{proof}

\begin{proposition}[6.5.4]
\label{I.6.5.4}
Let $f=(\psi,\theta):X\to Y$ be a morphism of finite type, $x$ a point of $X$, and $y=\psi(x)$.
\begin{enumerate}
  \item[{\rm(i)}] For $f$ to be a local immersion at the point $x$ \sref{I.4.5.1}, it is necessary and sufficient that $\theta_x^\sharp:\sh{O}_y\to\sh{O}_x$ be surjective.
  \item[{\rm(ii)}] Assume further that $Y$ is locally Noetherian.
    For $f$ to be a local isomorphism at the point $x$ \sref{I.4.5.2}, it is necessary and sufficient that $\theta_x^\sharp$ be an isomorphism.
\end{enumerate}
\end{proposition}

\begin{proof}
\label{proof-I.6.5.4}
\medskip\noindent
\begin{enumerate}
  \item[(ii)] By \sref{I.6.5.1}, there exists an open neighbourhood $V$ of $Y$ and a morphism $g:V\to X$ such that $g\circ f$ (resp. $f\circ g$) is defined and agrees with the identity on a neighbourhood of $x$ (resp. $y$), whence we can easily see that $f$ is a local isomorphism.
  \item[(i)] Since the question is local on $X$ and $Y$, we can assume that $X$ and $Y$ are affine, given by rings $A$ and $B$ (respectively);
    we have $f=({}^a\vphi,\widetilde{\vphi})$, where $\vphi$ is a homomorphism of rings $B\to A$ that makes $A$ a $B$-algebra of finite type;
    we have $\vphi^{-1}(\mathfrak{j}_x)=\mathfrak{j}_y$, and the homomorphism $\vphi_x:B_y\to A_x$ induced by $\vphi$ is \emph{surjective}.
    Let $(t_i)$ ($1\leq i\leq n$) be a system of generators of the $B$-algebra $A$;
    the hypothesis on $\vphi_x$ implies that there exist $b_i\in B$ and some $c\in B\setmin\mathfrak{j}_x$ such that, in the ring of fractions $A_x$, we have $t_i/1=\vphi(b_i)/\vphi(c)$ for $1\leq i\leq n$.
    Then \sref{I.1.3.3} there exists some $a\in A\setmin\mathfrak{j}_x$ such that, if we let $g=a\vphi(c)$, we also have $t_i/1=a\vphi(b_i)/g$ \emph{in the ring of fractions $A_g$}.
    With this, there exists, by hypothesis, a polynomial $Q(X_1,\ldots,X_n)$, with coefficients in the ring $\vphi(B)$, such that $a=Q(t_1,\ldots,t_n)$;
    \oldpage[I]{152}
    let $Q(X_1/T,\ldots,X_n/T)=P(X_1,\ldots,X_n,T)/T^m$, where $P$ is homogeneous of degree $m$.
    In the ring $A_g$, we have
    \[
      a/1=a^mP(\vphi(b_1),\ldots,\vphi(b_n),\vphi(c))/g^m=a^m\vphi(d)/g^m
    \]
    where $d\in B$.
    Since, in $A_g$, $g/1=(a/1)(\vphi(c)/1)$ is invertible by definition, so too are $a/1$ and $\vphi(c)/1$, and we can thus write $a/1=(\vphi(d)/1)(\vphi(c)/1)^{-m}$.
    From this we conclude that $\vphi(d)/1$ is also invertible in $A_g$.
    So let $h=cd$;
    since $\vphi(h)/1$ is invertible in $A_g$, the composite homomorphism $B\xrightarrow{\vphi}A\to A_g$ factors as $B\to B_h\xrightarrow{\gamma}A_g$ \sref[0]{0.1.2.4}.
    We will show that $\gamma$ is \emph{surjective};
    it suffices to show that the image of $B_h$ in $A_g$ contains the $t_i/1$ and $(g/1)^{-1}$.
    But we have $(g/1)^{-1}=(\vphi(c)/1)^{m-1}(\vphi(d)/1)^{-1}=\gamma(c^m/h)$, and $a/1=\gamma(d^{m+1}/h^m)$, so $(a\vphi(b_i))/1=\gamma(b_id^{m+1}/h^m)$, and since $t_i/1=(a\vphi(b_i)/1)(g/1)^{-1}$, our claim is proved.
    The choice of $h$ implies that $\psi(D(g))\subset D(h)$, and we also know that the restriction of $f$ to $D(g)$ is equal to $({}^a\gamma,\widetilde{\gamma})$;
    since $\gamma$ is surjective, this restriction is a closed immersion of $D(g)$ into $D(h)$ \sref{I.4.2.3}.
\end{enumerate}
\end{proof}

\begin{corollary}[6.5.5]
\label{I.6.5.5}
Let $f=(\psi,\theta):X\to Y$ be a morphism of finite type.
Assume that $X$ is irreducible, and denote by $x$ its generic point, and let $y=\psi(x)$.
\begin{enumerate}
  \item[{\rm(i)}] For $f$ to be a local immersion at any point of $X$, it is necessary and sufficient that $\theta_x^\sharp:\sh{O}_y\to\sh{O}_x$ be surjective.
  \item[{\rm(ii)}] Assume further that $Y$ is irreducible and locally Noetherian.
    For $f$ to be a local isomorphism at any point of $X$, it is necessary and sufficient that $y$ be the generic point of $Y$ \emph{(or, equivalently \sref[0]{0.2.1.4}, that $f$ be a \emph{dominant} morphism)} and that $\theta_x^\sharp$ be an isomorphism \emph{(in other words, that $f$ be \emph{birational} \sref{I.2.2.9})}.
\end{enumerate}
\end{corollary}

\begin{proof}
\label{proof-I.6.5.5}
It is clear that (i) follows from \sref{I.6.5.4}[i], taking into account the fact that every nonempty open subset of $X$ contains $x$;
similarly, (ii) follows from \sref{I.6.5.4}[ii].
\end{proof}

\subsection{Quasi-compact morphisms and morphisms locally of finite type}
\label{subsection:I.6.6}

\begin{definition}[6.6.1]
\label{I.6.6.1}
We say that a morphism $f:X\to Y$ is \emph{quasi-compact} if the inverse image of any quasi-compact open subset of $Y$ under $f$ is quasi-compact.
\end{definition}

Let $\mathfrak{B}$ be a base of the topology of $Y$ consisting of quasi-compact open subsets (for example, affine open subsets);
for $f$ to be quasi-compact, it is necessary and sufficient that the inverse image of every set of $\mathfrak{B}$ under $f$ be quasi-compact (or, equivalently, a \emph{finite} union of affine open subsets), because every quasi-compact open subset of $Y$ is a finite union of sets of $\mathfrak{B}$.
For example, if $X$ is \emph{quasi-compact} and $Y$ \emph{affine}, then \emph{every} morphism $f: X\to Y$ is quasi-compact:
indeed, $X$ is a finite union of affine open subsets $U_i$, and for every affine open subset $V$ of $Y$, $U_i\cap f^{-1}(V)$ is affine \sref{I.5.5.10}, and so quasi-compact.

If $f: X\to Y$ is a quasi-compact morphism, it is clear that, for every open subset $V$ of $Y$, the restriction of $f$ to $f^{-1}(V)$ is a quasi-compact morphism $f^{-1}(V)\to V$.
Conversely, if $(U_\alpha)$ is an open cover of $Y$, and $f: X\to Y$ a morphism such that the restrictions $f^{-1}(U_\alpha)\to U_\alpha$ are quasi-compact, then $f$ is quasi-compact.

\begin{definition}[6.6.2]
\label{I.6.6.2}
We say that a morphism $f: X\to Y$ is \emph{locally of finite type} if, for every $x\in X$, there exists an open neighbourhood $U$ of $x$ and an open neighbourhood $V\supset f(U)$ of $y$ such that the restriction of $f$ to $U$ is a morphism of finite type from $U$ to $V$.
\oldpage[I]{153}
We then also say that $X$ is a prescheme locally of finite type over $Y$, or a $Y$-prescheme locally of finite type.
\end{definition}

It follows immediately from \sref{I.6.3.2} that, if $f$ is locally of finite type, then, for every open subset $W$ of $Y$, the restriction of $f$ to $f^{-1}(W)$ is a morphism $f^{-1}(W)\to W$ that is locally of finite type.

If $Y$ is locally Noetherian and $X$ locally of finite type over $Y$, then $X$ is locally Noetherian thanks to \sref{I.6.3.7}.

\begin{proposition}[6.6.3]
\label{I.6.6.3}
For a morphism $f: X\to Y$ to be of finite type, it is necessary and sufficient that it be quasi-compact and locally of finite type.
\end{proposition}

\begin{proof}
\label{proof-I.6.6.3}
The necessity of the conditions is immediate, given \sref{I.6.3.1} and the remark following \sref{I.6.6.1}.
Conversely, suppose that the conditions are satisfied, and let $U$ be an affine open subset of $Y$, given by some ring $A$;
for all $x\in f^{-1}(U)$, there is, by hypothesis, a neighbourhood $V(x)\subset f^{-1}(U)$ of $x$, and a neighbourhood $W(x)\subset U$ of $y=f(x)$ containing $f(V(x))$, and such that the restriction of $f$ to $V(x)$ is a morphism $V(x)\to W(x)$ of finite type.
Replacing $W(x)$ with a neighbourhood $W_1(x)\subset W(x)$ of $x$ of the form $D(g)$ (with $g\in A$), and $V(x)$ with $V(x)\cap f^{-1}(W_1(x))$, we can assume that $W(x)$ is of the form $D(g)$, and thus of finite type over $U$ (because its ring can be written as $A[1/g]$);
so $V(x)$ is of finite type over $U$.
Further, $f^{-1}(U)$ is quasi-compact by hypothesis, and so the finite union of open subsets $V(x_i)$, which finishes the proof.
\end{proof}

\begin{proposition}[6.6.4]
\label{I.6.6.4}
\medskip\noindent
\begin{enumerate}
  \item[{\rm(i)}] An immersion $X\to Y$ is quasi-compact if it is closed, or if the underlying space of $Y$ is locally Noetherian, or if the underlying space of $X$ is Noetherian.
  \item[{\rm(ii)}] The composition of any two quasi-compact morphisms is quasi-compact.
  \item[{\rm(iii)}] If $f: X\to Y$ is a quasi-compact $S$-morphism, then so too is $f_{(S')}: X_{(S')}\to Y_{(S')}$ for any extension $g: S\to S'$ of the base prescheme.
  \item[{\rm(iv)}] If $f:X\to X'$ and $g:Y\to Y'$ are two quasi-compact $S$-morphisms, then $f\times_S g$ is quasi-compact.
  \item[{\rm(v)}] If the composition of any two morphisms $f: X\to Y$ and $g:Y\to Z$ is quasi-compact, and if either $g$ is separated or the underlying space of $X$ is locally Noetherian, then $f$ is quasi-compact.
  \item[{\rm(vi)}] For a morphism $f$ to be quasi-compact, it is necessary and sufficient that $f_\red$ be quasi-compact.
\end{enumerate}
\end{proposition}

\begin{proof}
\label{proof-I.6.6.4}
We note that (vi) is evident because the property of being quasi-compact, for a morphism, depends only on the corresponding continuous map of underlying spaces.
We will similarly prove the part of (v) corresponding to the case where the underlying space of $X$ is locally Noetherian.
Set $h=g\circ f$, and let $U$ be a quasi-compact open subset of $Y$;
$g(U)$ is quasi-compact (but not necessarily open) in $Z$, and so contained in a finite union of quasi-compact open subsets $V_j$ \sref{I.2.1.3}, and $f^{-1}(U)$ is thus contained in the union of the $h^{-1}(V_j)$, which are quasi-compact subspaces of $X$, and thus Noetherian subspaces.
We thus conclude \sref[0]{0.2.2.3} that $f^{-1}(U)$ is a Noetherian space, and \emph{a fortiori} quasi-compact.

To prove the other claims, it suffices to prove (i), (ii), and (iii) \sref{I.5.5.12}.
But (ii) is evident, and (i) follows from \sref{I.6.3.5} whenever the space $Y$ is locally Noetherian or the space $X$ is Noetherian, and is evident for a closed immersion.
\oldpage[I]{154}
To show (iii), we can restrict to the case where $S=Y$ \sref{I.3.3.11};
let $f'=f_{(S')}$, and let $U'$ be a quasi-compact open subset of $S'$.
For every $s'\in U'$, let $T$ be an affine open neighbourhood of $g(s')$ in $S$, and let $W$ be an affine open neighbourhood of $s'$ contained in $U'\cap g^{-1}(T)$;
it will suffice to show that $f'^{-1}(W)$ is quasi-compact;
in other words, we can restrict to showing that, when $S$ and $S'$ are \emph{affine}, the underlying space of $X\times_S S'$ is quasi-compact.
But since $X$ is then, by hypothesis, a finite union of affine open subsets $V_j$, $X\times_S S'$ is a union of the underlying spaces of the affine schemes $V_j\times_S S'$ (\sref{I.3.2.2} and \sref{I.3.2.7}), which proves the proposition.
\end{proof}

We note also that, if $X=X'\sqcup X''$ is the sum of two preschemes, a morphism $f:X\to Y$ is quasi-compact if and only if its restrictions to both $X'$ and $X''$ are quasi-compact.

\begin{proposition}[6.6.5]
\label{I.6.6.5}
Let $f:X\to Y$ be a quasi-compact morphism.
For $f$ to be dominant, it is necessary and sufficient that, for every generic point $y$ of an irreducible component of $Y$, $f^{-1}(y)$ contain the generic point of an irreducible component of $X$.
\end{proposition}

\begin{proof}
\label{proof-I.6.6.5}
It is immediate that the condition is sufficient (even without assuming that $f$ is quasi-compact).
To see that it is necessary, consider an affine open neighbourhood $U$ of $y$;
$f^{-1}(U)$ is quasi-compact, and so a \emph{finite} union of affine open subsets $V_i$, and the hypothesis that $f$ be dominant implies that $y$ belongs to the closure \emph{in $U$} of one of the $f(V_i)$.
We can clearly assume $X$ and $Y$ to be reduced;
since the closure in $X$ of an irreducible component of $V_i$ is an irreducible component on $X$ \sref[0]{0.2.1.6}, we can replace $X$ by $V_i$, and $Y$ by the closed reduced subprescheme of $U$ that has $\overline{f(V_i)}\cap U$ as its underlying space \sref{I.5.2.1}, and we are thus led to proving the proposition when $X=\Spec(A)$ and $Y=\Spec(B)$ are affine and reduced.
Since $f$ is dominant, $B$ is a subring of $A$ \sref{I.1.2.7}, and the proposition then follows from the fact that every minimal prime ideal of $B$ is the intersection of $B$ with a minimal prime ideal of $A$ \sref[0]{0.1.5.8}.
\end{proof}

\begin{proposition}[6.6.6]
\label{I.6.6.6}
\medskip\noindent
\begin{enumerate}
  \item[{\rm(i)}] Every local immersion is locally of finite type.
  \item[{\rm(ii)}] If two morphisms $f:X\to Y$ and $g:Y\to Z$ are locally of finite type, then so too is $g\circ f$.
  \item[{\rm(iii)}] If $f:X\to Y$ is an $S$-morphism locally of finite type, then $f_{(S')}:X_{(S')}\to Y_{(S')}$ is locally of finite type for any extension $S'\to S$ of the base prescheme.
  \item[{\rm(iv)}] If $f:X\to X'$ and $g:Y\to Y'$ are $S$-morphisms locally of finite type, then $f\times_S g$ is locally of finite type.
  \item[{\rm(v)}] If the composition $g\circ f$ of two morphisms is locally of finite type, then $f$ is locally of finite type.
  \item[{\rm(vi)}] If a morphism $f$ is locally of finite type, then so too is $f_\red$.
\end{enumerate}
\end{proposition}

\begin{proof}
\label{proof-I.6.6.6}
By \sref{I.5.5.12}, it suffices to prove (i), (ii), and (iii).
If $j:X\to Y$ is a local immersion then, for every $x\in X$, there is an open neighbourhood $V$ of $j(x)$ in $Y$ and an open neighbourhood $U$ of $x$ in $X$ such that the restriction of $j$ to $U$ is a closed immersion $U\to V$ \sref{I.4.5.1}, and so this restriction is of finite type.
To prove (ii), consider a point $x\in X$;
by hypothesis, there is an open neighbourhood $W$ of $g(f(x))$ and an open neighbourhood $V$ of $f(x)$ such that $g(V)\subset W$ and such that $V$ is of of finite type over $W$;
furthermore, $f^{-1}(V)$ is locally of finite type over $V$ \sref{I.6.6.2}, so there is an open neighbourhood $U$ of $x$ that is contained in $f^{-1}(V)$ and of is finite type over $V$;
\oldpage[I]{155}
thus we have $g(f(U))\subset W$, and that $U$ is of finite type over $W$ \sref{I.6.3.4}[ii].
Finally, to prove (iii), we can restrict to the case where $Y=S$ \sref{I.3.3.11};
for every $x'\in X'=X_{(S')}$, let $x$ be the image of $x'$ in $X$, $s$ the image of $x$ in $S$, $T$ an open neighbourhood of $s$, $T'$ the inverse image of $T$ in $S'$, and $U$ an open neighbourhood of $x$ that is of finite type over $T$ and whose image is contained in $T$;
then $U\times_S T'=U\times_T T'$ is an open neighbourhood of $x'$ \sref{I.3.2.7} that is of finite type over $T'$ \sref{I.6.3.4}[iv].
\end{proof}

\begin{corollary}[6.6.7]
\label{I.6.6.7}
Let $X$ and $Y$ be $S$-preschemes that are locally of finite type over $S$.
If $S$ is locally Noetherian, then $X\times_S Y$ is locally Noetherian.
\end{corollary}

\begin{proof}
\label{proof-I.6.6.7}
Indeed, $X$ being locally of finite type over $S$ means that it is locally Noetherian, and that $X\times_S Y$ is locally of finite type over $X$, and so $X\times_S Y$ is also locally Noetherian.
\end{proof}

\begin{remark}[6.6.8]
\label{I.6.6.8}
Proposition~\sref{I.6.3.10} and its proof extend immediately to the case where we suppose only that the morphism $f$ is locally of finite type.
Similarly, propositions~\sref{I.6.4.2} and \sref{I.6.4.9} hold true when we suppose only that the preschemes $X$ and $Y$ in the claim are locally of finite type over the field $K$.
\end{remark}


\section{Rational maps}
\label{section:1.7}

\subsection{Rational maps and rational functions}
\label{subsection:1.7.1}

\begin{env}[7.1.1]
\label{1.7.1.1}
Let $X$ and $Y$ be preschemes, $U$ and $V$ dense open subsets of $X$, and $f$ (resp. $g$) a morphism from $U$ (resp. $V$) to $Y$; we say that $f$ and $g$ are \emph{equivalent} if they agree on a dense open subset of $U\cap V$.
Since a finite intersection of dense open subsets of $X$ is a dense open subset of $X$, it is clear that this relation is an \emph{equivalence relation}.
\end{env}

\begin{definition}[7.1.2]
\label{1.7.1.2}
Given preschemes $X$ and $Y$, we define a rational map from $X$ to $Y$ to be an equivalence class of morphisms from a dense open subset of $X$ to a dense open subset of $Y$, under the equivalence relation defined in \sref{1.7.1.1}.
If $X$ and $Y$ are $S$-preschemes, we say that such a class is a rational $S$-map if there exists a representative of the class that is also an $S$-morphism.
We define a rational $S$-section of $X$ to be any rational $S$-map from $S$ to $X$.
We define a rational function on a prescheme $X$ to be any rational $X$-section on the $X$-prescheme $X\otimes_{\bb{Z}}\bb{Z}[T]$ (where $T$ is an indeterminate).

By an abuse of language, whenever we are discussing only $S$-preschemes, we will say ``rational map'' instead of ``rational $S$-map'' if no confusion may arise.

Let $f$ be a rational map from $X$ to $Y$, and $U$ an open subset of $X$; if $f_1$ and $f_2$ are two morphisms belonging to the class of $f$, defined (respectively) on dense open subsets $V$ and $W$ of $X$, then the restrictions $f_1|(U\cap V)$ and $f_2|(U\cap W)$ agree on $U\cap V\cap W$, which is dense in $U$; the class $f$ of morphisms thus defines a rational map from $U$ to $Y$, called the \emph{restriction of $f$ to $U$}, and denoted by $f|U$.

If, to every $S$-morphism $f:X\to Y$, we take the corresponding rational $S$-map to which $f$ belongs, we obtain a canonical map from $\Hom_S(X,Y)$ to the set of rational $S$-maps from $X$ to $Y$.
We denote by $\Gamma_\mathrm{rat}(X/Y)$ the set of rational $Y$-sections on $X$, and we thus have a canonical map $\Gamma(X/Y)\to\Gamma_\mathrm{rat}(X/Y)$.
It is also clear that, if $X$ and $Y$ are $S$-preschemes, then the set of rational $S$-maps from $X$ to $Y$ is canonically identified with $\Gamma_\mathrm{rat}((X\times_S Y)/X)$ \sref{1.3.3.14}.
\end{definition}

\begin{env}[7.1.3]
\label{1.7.1.3}
It also follows from \sref{1.7.1.2} and \sref{1.3.3.14} that the rational functions on $X$ are canonically identified with \emph{equivalence classes of sections of the structure sheaf $\sh{O}_X$} over dense open subsets of $X$, where two such sections are equivalent if the agree on some dense open subset of $X$ contained inside the intersection of the subsets on which they are defined.
In particular, it follows that the rational functions on $X$ form a \emph{ring} $R(X)$.
\end{env}

\begin{env}[7.1.4]
\label{1.7.1.4}
When $X$ is an \emph{irreducible} prescheme, every nonempty open subset of $X$ is dense in $X$; so we can say that the nonempty open subsets of $X$ are the \emph{open neighbourhoods of the generic point} $x$ of $X$.
To say that two morphisms from nonempty open subsets of $X$ to $Y$ are equivalent thus means, in this case, that they have the \emph{same germ} at the point $x$.
In other words, the rational maps (resp. rational $S$-maps) $X\to Y$ are identified with the \emph{germs of morphisms} (resp. \emph{$S$-morphisms}) from nonempty open subsets of $X$ to $Y$ at the generic point $x$ of $X$.
In particular:
\end{env}

\begin{proposition}[7.1.5]
\label{1.7.1.5}
If $X$ is an irreducible prescheme, then the ring $R(X)$ of rational maps on $X$ is canonically identified with the local ring $\sh{O}_x$ of the generic point $x$ of $X$.
It is a local ring of dimension~0, and thus a local Artinian ring when $X$ is Noetherian; it is a field when $X$ is integral, and, when $X$ is further an affine scheme, it is identified with the field of fractions of $A(X)$.
\end{proposition}

\begin{proof}
\label{proof-1.7.1.5}
Given the above, and the identification of rational functions with sections of $\sh{O}_X$ over a dense open subset of $X$, the first claim is nothing but the definition of the fibre of a sheaf above a point.
For the other claims, we can reduce to the case where $X$ is affine, given by some ring $A$; then $\mathfrak{j}_x$ is the nilradical of $A$, and $\sh{O}_x$ is thus of dimension~0; if $A$ is integral, then $\mathfrak{j}_x=(0)$, and $\sh{O}_x$ is thus the field of fractions of $A$.
Finally, if $A$ is Noetherian, we know (\cite[p.~127, cor.~4]{I-11}) that $\mathfrak{j}_x$ is nilpotent, and $\sh{O}_x=A_x$ Artinian.
\end{proof}

If $X$ is \emph{integral}, then the ring $\sh{O}_z$ is integral for all $z\in X$; every affine open subset $U$ containing $z$ also contains $x$, and $R(U)$, being equal to the field of fractions of $A(U)$, is identified with $R(X)$; we thus conclude that $R(X)$ can also be identified with the \emph{field of fractions of $\sh{O}_z$}: the canonical identification of $\sh{O}_z$ to a subring of $R(X)$ consists of associating, to every germ of a section $s\in\sh{O}_z$, the unique rational function on $X$, class of a section of $\sh{O}_X$, (necessarily defined on a dense open subset of $X$) having $s$ as its germ at the point $z$.

\begin{env}[7.1.6]
\label{1.7.1.6}
Now suppose that $X$ has a \emph{finite} number of irreducible components $X_i$ ($1\leq i\leq n$) (which will be the case whenever the underlying space of $X$ is \emph{Noetherian}); let $X'_i$ be the open subset of $X$ given by the complement of the $X_j\cap X_i$ for $j\neq i$ inside $X_i$; $X'_i$ is irreducible, its generic point $x_i$ is the generic point of $X_i$, and the $X'_i$ are pairwise disjoint, with their union being dense in $X$ \sref[0]{0.2.1.6}.
For every dense open subset $U$ of $X$, $U_i=U\cap X'_i$ is a nonempty dense open subset of $X'_i$, with the $U_i$ being pairwise disjoint, and so $U'=\bigcup_i U'_i$ is dense in $X$.
Giving a morphism from $U'$ to $Y$ consists of giving (arbitrarily) a morphism from each of the $U_i$ to $Y$.
\end{env}

Thus:
\begin{proposition}[7.1.7]
\label{1.7.1.7}
Let $X$ and $Y$ be two preschemes (resp. $S$-preschemes) such that $X$ has a finite number of irreducible components $X_i$, with generic points $x_i$ ($1\leq i\leq n$).
If $R_i$ is the set of germs of morphisms (resp. $S$-morphisms) from open subsets of $X$ to $Y$ at the point $x_i$, then the set of rational maps (resp. rational $S$-maps) from $X$ to $Y$ can be identified with the product of the $R_i$ ($1\leq i\leq n$).
\end{proposition}

\begin{corollary}[7.1.8]
\label{1.7.1.8}
Let $X$ be a Noetherian prescheme.
The ring of rational functions on $X$ is an Artinian ring, whose local components are the rings $\sh{O}_{x_i}$ of the generic points $x_i$ of the irreducible components of $X$.
\end{corollary}

\begin{corollary}[7.1.9]
\label{1.7.1.9}
Let $A$ be a Noetherian ring, and $X=\Spec(A)$.
If $Q$ is the complement of the union of the minimal prime ideals of $A$, then the ring of rational functions on $X$ can be canonically identified with the ring of fractions $Q^{-1}A$.
\end{corollary}

This will follow from the following lemma:
\begin{lemma}[7.1.9.1]
\label{1.7.1.9.1}
For an element $f\in A$ to be such that $D(f)$ is dense in $X$, it is necessary and sufficient that $f\in Q$; every dense open subset of $X$ contains an open subset of the form $D(f)$, where $f\in Q$.
\end{lemma}

\begin{proof}
\label{proof-1.7.1.9.1}
To show \hyperref[1.7.1.9.1]{(7.1.9.1)}, we again denote by $X_i$ ($1\leq i\leq n$) the irreducible components of $X$; if $D(f)$ is dense in $X$ then $D(f)\cap X_i\neq\emp$ for $1\leq i\leq n$, and vice-versa; but this means that $f\not\in\mathfrak{p}_i$ for $1\leq i\leq n$, where we set $\mathfrak{p}_i=\mathfrak{j}(X_i)$, and since the $\mathfrak{p}_i$ are the minimal prime ideals of $A$ \hyperref[1.1.1.14]{(1.1.14)}, the conditions $f\not\in\mathfrak{p}_i$ ($1\leq i\leq n$) are equivalent to $f\in Q$, whence the first claim of the lemma.
For the other claim, if $U$ is a dense open subset of $X$, the complement of $U$ is a set of the form $V(\mathfrak{a})$, where $\mathfrak{a}$ is an ideal which is not contained in any of the $\mathfrak{p}_i$; it is thus not contained in their union (\cite[p.~13]{I-10}), and there thus exists some $f\in\mathfrak{a}$ belonging to $Q$; whence $D(f)\subset U$, which finishes the proof.
\end{proof}

\begin{env}[7.1.10]
\label{1.7.1.10}
Suppose again that $X$ is irreducible, with generic point $x$.
Since every nonempty open subset $U$ of $X$ contains $x$, and thus also contains every $z\in X$ such that $x\in\overline{\{z\}}$, every morphism $U\to Y$ can be composed with the canonical morphism $\Spec(\sh{O}_x)\to X$ \sref{1.2.4.1}; and any two morphisms into $Y$ from two nonempty open subsets of $X$ which agree on a nonempty open subset of $X$ give, by composition, the same morphism $\Spec(\sh{O}_x)\to Y$.
In other words, to every rational map from $X$ to $Y$ there is a corresponding well-defined morphism $\Spec(\sh{O}_x)\to Y$.
\end{env}

\begin{proposition}[7.1.11]
\label{1.7.1.11}
Let $X$ and $Y$ be two $S$-preschemes; suppose that $X$ is irreducible with generic point $x$, and that $Y$ is of finite type over $S$.
Any two rational $S$-maps from $X$ to $Y$ that correspond to the same $S$-morphism $\Spec(\sh{O}_x)\to Y$ are then identical.
If we further suppose $S$ to be locally Noetherian, then every $S$-morphism from $\Spec(\sh{O}_x)$ to $Y$ corresponds to exactly one rational $S$-map from $X$ to $Y$.
\end{proposition}

\begin{proof}
\label{proof-1.7.1.11}
Taking into account that every nonempty subset of $X$ is dense in $X$, this follows from \sref{1.6.5.1}.
\end{proof}

\begin{corollary}[7.1.12]
\label{1.7.1.12}
Suppose that $S$ is locally Noetherian, and that the other hypotheses of \sref{1.7.1.11} are satisfied.
The rational $S$-maps from $X$ to $Y$ can then be identified with points of the $S$-prescheme $Y$, with values in the $S$-prescheme $\Spec(\sh{O}_x)$.
\end{corollary}

\begin{proof}
\label{proof-1.7.1.12}
This is nothing but \sref{1.7.1.11}, with the terminology introduced in \sref{1.3.4.1}.
\end{proof}

\begin{corollary}[7.1.13]
\label{1.7.1.13}
Suppose that the conditions of \sref{1.7.1.12} are satisfied.
Let $s$ be the image of $x$ in $S$.
The data of a rational $S$-map from $X$ to $Y$ is equivalent to the data of a point $y$ of $Y$ over $s$ along with a local $\sh{O}_s$-homomorphism $\sh{O}_y\to\sh{O}_x=R(X)$.
\end{corollary}

\begin{proof}
\label{proof-1.7.1.13}
This follows from \sref{1.7.1.11} and \sref{1.2.4.4}.
\end{proof}

In particular:
\begin{corollary}[7.1.14]
\label{1.7.1.14}
Under the conditions of \sref{1.7.1.12}, rational $S$-maps from $X$ to $Y$ depend only (for any given $Y$) on the $S$-prescheme $\Spec(\sh{O}_x)$, and, in particular, remain the same whenever $X$ is replaced by $\Spec(\sh{O}_z)$, for any $z\in X$.
\end{corollary}

\begin{proof}
\label{proof-1.7.1.14}
Since $z\in\overline{\{x\}}$, $x$ is the generic point of $Z=\Spec(\sh{O}_z)$, and $\sh{O}_{X,x}=\sh{O}_{Z,z}$.
\end{proof}

When $X$ is integral, $R(X)=\sh{O}_x=\kres(x)$ is a field \sref{1.7.1.5}; the preceding corollaries then specialize to the following:
\begin{corollary}[7.1.15]
\label{1.7.1.15}
Suppose that the conditions of \sref{1.7.1.12} are satisfied, and further that $X$ is integral.
Let $s$ be the image of $x$ in $S$.
Then rational $S$-maps from $X$ to $Y$ can be identified with the geometric points of $Y\otimes_S\kres(s)$ with values in the extension $R(X)$ of $\kres(s)$, or, in other words, every such map is equivalent to the data of a point $y\in Y$ above $s$ along with a $\kres(s)$-monomorphism from $\kres(y)$ to $\kres(x)=R(X)$.
\end{corollary}

\begin{proof}
\label{proof-1.7.1.15}
The points of $Y$ above $s$ are identified with the points of $Y\otimes_S\kres(s)$ \sref{1.3.6.3}, and the local $\sh{O}_s$-homomorphisms $\sh{O}_y\to R(X)$ with the $\kres(s)$-monomorphisms $\kres(y)\to R(X)$.
\end{proof}

More precisely:
\begin{corollary}[7.1.16]
\label{1.7.1.16}
Let $k$ be a field, and $X$ and $Y$ two algebraic preschemes over $k$ \sref{1.6.4.1}; suppose further that $X$ is integral.
Then the rational $k$-maps from $X$ to $Y$ can be identified with the geometric points of $Y$ with values in the extension $R(X)$ of $k$ \sref{1.3.4.4}.
\end{corollary}

\subsection{Domain of definition of a rational map}
\label{subsection:1.7.2}

\begin{env}[7.2.1]
\label{1.7.2.1}
Let $X$ and $Y$ be preschemes, and $f$ rational map from $X$ to $Y$.
We say that $f$ is \emph{defined at a point $x\in X$} if there exists a dense open subset $U$ of $X$ that contains $x$, and a morphism $U\to Y$ belonging to the equivalence class of $f$.
The set of points $x\in X$ where $f$ is defined is called the \emph{domain of definition} of $f$; it is clear that it is an open dense subset of $X$.
\end{env}

\begin{proposition}[7.2.2]
\label{1.7.2.2}
Let
\oldpage[I]{159}
$X$ and $Y$ be $S$-preschemes such that $X$ is reduced and $Y$ is separated over $S$.
Let $f$ be a rational $S$-map from $X$ to $Y$, with domain of definition $U_0$.
Then there exists exactly one $S$-morphism $U_0\to Y$ belonging to the class of $f$.
\end{proposition}

%\label{proof-1.7.2.2}
Since, for every morphism $U\to Y$ belonging to the class of $f$, we necessarily have $U\subset U_0$, it is clear that the proposition will be a consequence of the following:

\begin{lemma}[7.2.2.1]
\label{1.7.2.2.1}
Under the hypotheses of \sref{1.7.2.2}, let $U_1$ and $U_2$ be two dense open subsets of $X$, and $f_i:U_i\to Y$ ($i=1,2$) two $S$-morphisms such that there exists an open subset $V\subset U_1\cap U_2$, dense in $X$, and on which $f_1$ and $f_2$ agree.
Then $f_1$ and $f_2$ agree on $U_1\cap U_2$.
\end{lemma}

\begin{proof}
\label{proof-1.7.2.2.1}
We can clearly restrict to the case where $X=U_1=U_2$.
Since $X$ (and thus $V$) is reduced, $X$ is the smallest closed subprescheme of $X$ containing $V$ \sref{1.5.2.2}.
Let $g=(f_1,f_2)_S:X\to Y\times_S Y$; since, by hypothesis, the diagonal $T=\Delta_Y(Y)$ is a closed subprescheme of $Y\times_S Y$, $Z=g^{-1}(T)$ is a closed subprescheme of $X$ \sref{1.4.4.1}.
If $h:V\to Y$ is the common restriction of $f_1$ and $f_2$ to $V$, then the restriction of $g$ to $V$ is $g'=(h,h)_S$, which factors as $g'=\Delta_Y\circ h$; since $\Delta_Y^{-1}(T)=Y$, we have that $g'^{-1}(T)=V$, and so $Z$ is a closed subprescheme of $X$ inducing $V$, thus containing $V$, which implies that $Z=X$.
From the equation $g^{-1}(T)=X$, we deduce \sref{1.4.4.1} that $g$ factors as $\Delta_Y\circ f$, where $f$ is a morphism $X\to Y$, which implies, by the definition of the diagonal morphism, that $f_1=f_2=f$.
\end{proof}

It is clear that the morphism $U_0\to Y$ defined in \sref{1.7.2.2} is the unique morphism of the class $f$ that \emph{cannot be extended} to a morphism from an open subset of $X$ that strictly contains $U_0$.
\emph{Under the hypotheses of \sref{1.7.2.2}}, we can thus \emph{identify} the rational maps from $X$ to $Y$ with the \emph{non-extendible} (to strictly larger open subsets) morphisms from dense open subsets of $X$ to $Y$.
With this identification, Proposition~\sref{1.7.2.2} implies:
\begin{corollary}[7.2.3]
\label{1.7.2.3}
With the hypotheses from \sref{1.7.2.2} on $X$ and $Y$, let $U$ be a dense open subset of $X$.
Then there exists a canonical bijective correspondence between $S$-morphisms from $U$ to $Y$ and rational $S$-maps from $X$ to $Y$ that are defined at all points of $U$.
\end{corollary}

\begin{proof}
\label{proof-1.7.2.3}
By \sref{1.7.2.2}, for every $S$-morphism $f$ from $U$ to $Y$, there exists exactly one rational $S$-map $\overline{f}$ from $X$ to $Y$ which extends $f$.
\end{proof}

\begin{corollary}[7.2.4]
\label{1.7.2.4}
Let $S$ be a scheme, $X$ a reduced $S$-prescheme, $Y$ an $S$-scheme, and $f:U\to Y$ an $S$-morphism from a dense open subset $U$ of $X$ to $Y$.
If $\overline{f}$ is the rational $\bb{Z}$-map from $X$ to $Y$ that extends $f$, then $\overline{f}$ is an $S$-morphism (and thus the rational $S$-map from $X$ to $Y$ that extends $f$).
\end{corollary}

\begin{proof}
\label{proof-1.7.2.4}
Indeed, if $\vphi:X\to S$ and $\psi:Y\to S$ are the structure morphisms, $U_0$ the domain of definition of $\overline{f}$, and $j$ the injection $U_0\to X$, then it suffices to show that $\psi\circ\overline{f}=\vphi\circ j$, but this follows from \sref{1.7.2.2.1}, since $f$ is an $S$-morphism.
\end{proof}

\begin{corollary}[7.2.5]
\label{1.7.2.5}
Let $X$ and $Y$ be two $S$-preschemes; suppose that $X$ is reduced, and that $X$ and $Y$ are separated over $S$.
Let $p:Y\to X$ be an $S$-morphism (making $Y$ an $X$-prescheme), $U$ a dense open subset of $X$, and $f$ a $U$-section of $Y$; then the rational map $\overline{f}$ from $X$ to $Y$ extending $f$ is a rational $X$-section of $Y$.
\end{corollary}

\begin{proof}
\label{proof-1.7.2.5}
We
\oldpage[I]{160}
have to show that $p\circ\overline{f}$ is the identity on the domain of definition of $\overline{f}$; since $X$ is separated over $S$, this again follows from \sref{1.7.2.2.1}.
\end{proof}

\begin{corollary}[7.2.6]
\label{1.7.2.6}
Let $X$ be a reduced prescheme, and $U$ a dense open subset of $X$.
Then there is a canonical bijective correspondence between sections of $\sh{O}_X$ over $U$ and rational functions on $X$ defined at every point of $U$.
\end{corollary}

\begin{proof}
\label{proof-1.7.2.6}
Taking \sref{1.7.2.3}, \sref{1.7.1.2}, and \sref{1.7.1.3} into account, it suffices to note that the $X$-prescheme $X\otimes_{\bb{Z}}\bb{Z}[T]$ is separated over $X$ \sref{1.5.5.1}[iv].
\end{proof}

\begin{corollary}[7.2.7]
\label{1.7.2.7}
Let $Y$ be a reduced prescheme, $f:X\to Y$ a separated morphism, $U$ a dense open subset of $Y$, $g:U\to f^{-1}(U)$ a $U$-section of $f^{-1}(U)$, and $Z$ the reduced subprescheme of $X$ that has $\overline{g(U)}$ as its underlying space \sref{1.5.2.1}.
For $g$ to be the restriction of a $Y$-section of $X$ \emph{(in other words \sref{1.7.2.5}, for the rational map from $Y$ to $X$ extending $g$ to be defined everywhere)}, it is necessary and sufficient for the restriction of $f$ to $Z$ to be an isomorphism from $Z$ to $Y$.
\end{corollary}

\begin{proof}
\label{proof-1.7.2.7}
The restriction of $f$ to $f^{-1}(U)$ is a separated morphism \sref{1.5.5.1}[i], so $g$ is a closed immersion \sref{1.5.4.6}, and so $g(U)=Z\cap f^{-1}(U)$, and the subprescheme induced by $Z$ on the open subset $g(U)$ of $Z$ is identical to the closed subprescheme of $f^{-1}(U)$ associated to $g$ \sref{1.5.2.1}.
It is then clear that the stated condition is sufficient, because, if satisfied, and if $f_Z:Z\to Y$ is the restriction of $f$ to $Z$, and $\overline{g}:Y\to Z$ is the inverse isomorphism, then $\overline{g}$ extends $g$.
Conversely, if $g$ is the restriction to $U$ of a $Y$-section $h$ of $X$, then $h$ is a closed immersion \sref{1.5.4.6}, and so $h(Y)$ is closed, and, since it is contained in $Z$, is equal to $Z$, and it follows from \sref{1.5.2.1} that $h$ is necessarily an isomorphism from $Y$ to the closed subprescheme $Z$ of $X$.
\end{proof}

\begin{env}[7.2.8]
\label{1.7.2.8}
Let $X$ and $Y$ be two $S$-preschemes, with $X$ reduced, and $Y$ separated over $S$.
Let $f$ be a rational $S$-map from $X$ to $Y$, and let $x$ be a point of $X$; we can compose $f$ with the canonical $S$-morphism $\Spec(\sh{O}_x)\to X$ \sref{1.2.4.1} provided that the intersection of $\Spec(\sh{O}_x)$ with the domain of definition of $f$ is dense in $\Spec(\sh{O}_x)$ (identified with the set of $z\in X$ such that $x\in\overline{\{z\}}$ \sref{1.2.4.2}).
This will happen in the follow cases:
\begin{enumerate}
  \item[1st.] $X$ is \emph{irreducible} (and thus \emph{integral}), because then the generic point $\xi$ of $X$ is the generic point of $\Spec(\sh{O}_x)$; since the domain of definition $U$ of $f$ contains $\xi$, $U\cap\Spec(\sh{O}_x)$ contains $\xi$, and so is dense in $\Spec(\sh{O}_x)$.
  \item[2nd.] $X$ is \emph{locally Noetherian}; our claim then follows from:
\end{enumerate}
\end{env}

\begin{lemma}[7.2.8.1]
\label{1.7.2.8.1}
Let $X$ be a prescheme whose underlying space is locally Noetherian, and $x$ a point of $X$.
The irreducible components of $\Spec(\sh{O}_x)$ are the intersections of $\Spec(\sh{O}_x)$ with the irreducible components of $X$ containing $x$.
For an open subset $U\subset X$ to be such that $U\cap\Spec(\sh{O}_x)$ is dense $\Spec(\sh{O}_x)$, it is necessary and sufficient for it to have a nonempty intersection with the irreducible components of $X$ that contain $x$ \emph{(which will be the case whenever $U$ is \emph{dense} in $X$)}.
\end{lemma}

\begin{proof}
\label{proof-1.7.2.8.1}
It suffices to show just the first claim, since the second then follows.
Since $\Spec(\sh{O}_x)$ is contained in every affine open subset $U$ that contains $x$, and since the irreducible components of $U$ that contain $x$ are the intersections of $U$ with the irreducible components of $X$ containing $x$ \sref[0]{0.2.1.6}, we can suppose that $X$ is affine, given by some ring $A$.
Since the prime ideals of $A_x$ correspond bijectively to the prime ideals of $A$ that are
\oldpage[I]{161}
contained in $\mathfrak{j}_x$ \sref{0.2.1.6}, the minimal prime ideals of $A_x$ correspond to the minimal prime ideals of $A$ that are contained in $\mathfrak{j}_x$, hence the lemma.
\end{proof}

With this in mind, suppose that we are in one of the two cases mentioned in \sref{1.7.2.8}.
If $U$ is the domain of definition of the rational $S$-map $f$, then we denote by $f'$ the rational map from $\Spec(\sh{O}_x)$ to $Y$ which agrees (taking~\sref{1.2.4.2} into account) with $f$ on $U\cap\Spec(\sh{O}_x)$; we say that this rational map is \emph{induced by $f$}.

\begin{proposition}[7.2.9]
\label{1.7.2.9}
Let $S$ be a locally Noetherian prescheme, $X$ a reduced $S$-prescheme, and $Y$ an $S$-scheme of finite type.
Suppose further that $X$ is either irreducible or locally Noetherian.
Then let $f$ be a rational $S$-map from $X$ to $Y$, and $x$ a point of $X$.
For $f$ to be defined at a point $x$, it is necessary and sufficient for the rational map $f'$ from $\Spec(\sh{O}_x)$ to $Y$, induced by $f$ \sref{1.7.2.8}, to be a morphism.
\end{proposition}

\begin{proof}
\label{proof-1.7.2.9}
The condition clearly being necessary (since $\Spec(\sh{O}_x)$ is contained in every open subset containing $x$), we show that it is sufficient.
By~\sref{1.6.5.1}, there exists an open neighbourhood $U$ of $x$ in $X$, and an $S$-morphism $g$ from $U$ to $Y$ that induces $f'$ on $\Spec(\sh{O}_x)$.
If $X$ is irreducible, then $U$ is dense in $X$, and, by \sref{1.7.2.3}, we can suppose that $g$ is a rational $S$-map.
Further, the generic point of $X$ belongs to both $\Spec(\sh{O}_x)$ and the domain of definition of $f$, and so $s$ and $g$ agree at this point, and thus on a nonempty open subset of $X$ \sref{1.6.5.1}.
But since $f$ and $g$ are rational $S$-maps, they are identical \sref{1.7.2.3}, and so $f$ is defined at $x$.

If we now suppose that $X$ is locally Noetherian, then we can suppose that $U$ is Noetherian; then there are only a finite number of irreducible components $X_i$ of $X$ that contain $x$ \sref{1.7.2.8.1}, and we can suppose that they are the only ones that have a nonempty intersection with $U$, by replacing, if needed, $U$ with a smaller open subset (since there are only a finite number of irreducible components of $X$ that have a nonempty intersection with $U$, because $U$ is Noetherian).
We then have, as above, that $f$ and $g$ agree on a nonempty open subset of each of the $X_i$.
Taking into account the fact that each of the $X_i$ is contained in $\overline{U}$, we consider the morphism $f_1$, defined on a dense open subset of $U\cup(X\setmin\overline{U})$, equal to $g$ on $U$, and to $f$ on the intersection of $X\setmin\overline{U}$ with the domain of definition of $f$.
Since $U\cup(X\setmin\overline{U})$ is dense in $X$, $f_1$ and $f$ agree on a dense open subset of $X$, and since $f$ is a rational map, $f$ is an extension of $f_1$ \sref{1.7.2.3}, and is thus defined at the point $x$.
\end{proof}

\subsection{Sheaf of rational functions}
\label{subsection:1.7.3}

\begin{env}[7.3.1]
\label{1.7.3.1}
Let $X$ be a prescheme.
For every open subset $U\subset X$, we denote by $R(U)$ the ring of rational functions on $U$ \sref{1.7.1.3}; this is a $\Gamma(U,\sh{O}_X)$-algebra.
Further, if $V\subset U$ is a second open subset of $X$, then every section of $\sh{O}_X$ over a dense (in $X$) open subset of $V$ gives, by restriction to $V$, a section over a dense (in $X$) open subset of $V$, and if two sections agree on a dense (in $X$) open subset of $U$, then their restrictions to $V$ agree on a dense (in $X$) open subset of $V$.
We can thus define a di-homomorphism of algebras $\rho_{V,U}:R(U)\to R(V)$, and it is clear that, if $U\supset V\supset W$ are open subsets of $X$, then we have $\rho_{W,U}=\rho_{W,V}\circ\rho_{V,U}$; the $R(U)$ thus define a \emph{presheaf} of algebras on $X$.
\end{env}

\begin{definition}[7.3.2]
\label{1.7.3.2}
We
\oldpage[I]{162}
define the sheaf of rational functions on a prescheme $X$, denoted by $\sh{R}(X)$, to be the $\sh{O}_X$-algebra associated to the presheaf defined by the $R(U)$.
\end{definition}

For every prescheme $X$ and open subset $U\subset X$, it is clear that the induced sheaf $\sh{R}(X)|U$ is exactly $\sh{R}(U)$.

\begin{proposition}[7.3.3]
\label{1.7.3.3}
Let $X$ be a prescheme such that the family $(X_\lambda)$ of its irreducible components is locally finite (which is the case whenever the underlying space of $X$ is locally Noetherian).
Then the $\sh{O}_X$-module $\sh{R}(X)$ is quasi-coherent, and for every open subset $U$ of $X$ that has a nonempty intersection with only finitely many of the components $X_\lambda$, $R(U)$ is equal to $\Gamma(U,\sh{R}(X))$, and can be canonically identified with the direct sum of the local rings of the generic points of the $X_\lambda$ such that $U\cap X_\lambda\neq\emp$.
\end{proposition}

\begin{proof}
\label{proof-1.7.3.3}
We can evidently restrict to the case where $X$ has only a finite number of irreducible components $X_i$, with generic points $x_i$ ($1\leq i\leq n$).
The fact that $R(U)$ can be canonically identified with the direct sum of the $\sh{O}_{x_i}=R(X_i)$ such that $U\cap X_i\neq\emp$ then follows from \sref{1.7.1.7}.
We will show that the presheaf $U\to R(U)$ satisfies the sheaf axioms, which will prove that $R(U)=\Gamma(U,\sh{R}(X))$.
Indeed, it satisfies (F1) by what has already been discussed.
To see that it satisfies (F2), consider a cover of an open subset $U$ of $X$ by open subsets $V_\alpha\subset U$; if the $s_\alpha\in R(V_\alpha)$ are such that the restrictions of $s_\alpha$ and $s_\beta$ to $V_\alpha\cap V_\beta$ agree for every pair of indices, then we can conclude that, for every index $i$ such that $U\cap X_i\neq\emp$, the components in $R(X_i)$ of all the $s_\alpha$ such that $V_\alpha\cap X_i\neq\emp$ are all the same; denoting this component by $t_i$, it is clear that the element of $R(U)$ that has the $t_i$ as its components has $s_\alpha$ as its restriction to each $V_\alpha$.
Finally, to see that $\sh{R}(X)$ is quasi-coherent, we can restrict to the case where $X=\Spec(A)$ is affine; by taking $U$ to be an affine open subset of the form $D(f)$, where $f\in A$, it follows from the above and from Definition~\sref{1.1.3.4} that we have $\sh{R}(X)=\widetilde{M}$, where $M$ is the direct sum of the $A$-modules $A_{x_i}$.
\end{proof}

\begin{corollary}[7.3.4]
\label{1.7.3.4}
Let $X$ be a reduced prescheme that has only a finite number of irreducible components, and let $X_i$ ($1\leq i\leq n$) be the closed reduced preschemes of $X$ that have the irreducible components of $X$ as their underlying spaces \sref{1.5.2.1}.
If $h_i$ is the canonical injection $X_i\to X$, then $\sh{R}(X)$ is the direct sum of the $\sh{O}_X$-algebras $(h_i)_*(\sh{R}(X_i))$.
\end{corollary}

\begin{corollary}[7.3.5]
\label{1.7.3.5}
If $X$ is irreducible, then every quasi-coherent $\sh{R}(X)$-module $\sh{F}$ is a simple sheaf.
\end{corollary}

\begin{proof}
\label{proof-1.7.3.5}
It suffices to show that every $x\in X$ admits a neighbourhood $U$ such that $\sh{F}|U$ is a simple sheaf \sref[0]{0.3.6.2}; in other words, we are led to considering the case where $X$ is affine; we can further suppose that $\sh{F}$ is the cokernel of a homomorphism $(\sh{R}(X))^{I}\to(\sh{R}(X))^{J}$ \sref[0]{0.5.1.3}, and everything then follows from showing that $\sh{R}(X)$ is a simple sheaf; but this is evident, because $\Gamma(U,\sh{R}(X))=R(X)$ for every nonempty open subset $U$, where $U$ contains the generic point of $X$.
\end{proof}

\begin{corollary}[7.3.6]
\label{1.7.3.6}
If $X$ is irreducible, then, for every quasi-coherent $\sh{O}_X$-module $\sh{F}$, $\sh{F}\otimes_{\sh{O}_X}\sh{R}(X)$ is a simple sheaf; if, further, $X$ is reduced (and thus integral), then $\sh{F}\otimes_{\sh{O}_X}\sh{R}(X)$ is isomorphic to a sheaf of the form $(\sh{R}(X))^{(I)}$.
\end{corollary}

\begin{proof}
\label{proof-1.7.3.6}
The second claim follows from the fact that $R(X)$ is a field.
\end{proof}

\begin{proposition}[7.3.7]
\label{1.7.3.7}
Suppose that the prescheme $X$ is locally integral or locally Noetherian.
\oldpage[I]{163}
Then $\sh{R}(X)$ is a quasi-coherent $\sh{O}_X$-algebra; if, further, $X$ is reduced (which will be the case whenever $X$ is locally integral), then the canonical homomorphism $\sh{O}_X\to\sh{R}(X)$ is injective.
\end{proposition}

\begin{proof}
\label{proof-1.7.3.7}
The question being local, the first claim follows from \sref{1.7.3.3}; the second follows from \sref{1.7.2.3}.
\end{proof}

\begin{env}[7.3.8]
\label{1.7.3.8}
\footnote{\emph{[Trans.] This paragraph was changed entirely in the Errata of EGA~II.}}
Let $X$ and $Y$ be two \emph{integral} preschemes, which implies that $\sh{R}(X)$ (resp. $\sh{R}(Y)$) is a quasi-coherent $\sh{O}_X$-module (resp. $\sh{O}_Y$-module) \sref{1.7.3.3}.
Let $f:X\to Y$ be a \emph{dominant} morphism; then there exists a canonical homomorphism of $\sh{O}_X$-modules
\[
\label{1.7.3.8.1}
  \tau:f^*(\sh{R}(Y))\to\sh{R}(X).
  \tag{7.3.8.1}
\]
\end{env}

\begin{proof}
\label{proof-1.7.3.8}
Suppose first that $X=\Spec(A)$ and $Y=\Spec(B)$ are affine, given by integral rings $A$ and $B$, with $f$ thus corresponding to an injective homomorphism $B\to A$ which extends to a monomorphism $L\to K$ from the field of fractions $L$ of $B$ to the field of fractions $K$ of $A$.
The homomorphism \sref{1.7.3.8.1} then corresponds to the canonical homomorphism $L\otimes_B A\to K$ \sref{1.1.6.5}.
In the general case, for each pair of nonempty affine open sets $U\subset X$ and $V\subset Y$ such that $f(U)\subset V$, we define, as above, a homomorphism $\tau_{U,V}$ and we immediately have that, if $U'\subset U$, $V'\subset V$, $f(U')\subset V'$, then $\tau_{U,V}$ extends $\tau_{U',V'}$, and hence our assertion.
If $x$ and $y$ are the generic points of $X$ and $Y$ respectively, then we have $f(x)=y$,
\[
  (f^*(\sh{R}(Y)))_x=\sh{O}_y\otimes_{\sh{O}_y}\sh{O}_x=\sh{O}_x
\]
\sref[0]{0.4.3.1} and $\tau_x$ is thus an \emph{isomorphism}.
\end{proof}

\subsection{Torsion sheaves and torsion-free sheaves}
\label{subsection:1.7.4}

\begin{env}[7.4.1]
\label{1.7.4.1}
Let $X$ be an \emph{integral} scheme.
For every $\sh{O}_X$-module $\sh{F}$, the canonical homomorphism $\sh{O}_X\to\sh{R}(X)$ defines, by tensoring, a homomorphism (again said to be \emph{canonical}) $\sh{F}\to\sh{F}\otimes_{\sh{O}_X}\sh{R}(X)$, which, on each fibre, is exactly the homomorphism $z\to z\otimes1$ from $\sh{F}_x$ to $\sh{F}_x\otimes_{\sh{O}_X}R(X)$.
The \emph{kernel $\sh{T}$} of this homomorphism is a $\sh{O}_X$-submodule of $\sh{F}$, called the \emph{torsion sheaf} of $\sh{F}$; it is quasi-coherent if $\sh{F}$ is quasi-coherent (\sref{1.4.1.1} and \sref{1.7.3.6}).
We say that $\sh{F}$ is \emph{torsion free} if $\sh{T}=0$, and that $\sh{F}$ is a \emph{torsion sheaf} if $\sh{T}=\sh{F}$.
For every $\sh{O}_X$-module $\sh{F}$, $\sh{F}/\sh{T}$ is torsion free.
We deduce from \sref{1.7.3.5} that:
\end{env}

\begin{proposition}[7.4.2]
\label{1.7.4.2}
If $X$ is an integral prescheme, then every torsion-free quasi-coherent $\sh{O}_X$-module $\sh{F}$ is isomorphic to a subsheaf $\sh{G}$ of a simple sheaf of the form $(\sh{R}(X))^{(I)}$, generated (as a $\sh{R}(X)$-module) by $\sh{G}$.
\end{proposition}

The cardinality of $I$ is called the \emph{rank} of $\sh{F}$; for every nonempty affine open subset $U$ of $X$, the rank of $\sh{F}$ is equal to the rank of $\Gamma(U,\sh{F})$ as a $\Gamma(U,\sh{O}_X)$-module, as we see by considering the generic point of $X$, contained in $U$.
In particular:
\begin{corollary}[7.4.3]
\label{1.7.4.3}
On an integral prescheme $X$, every torsion-free quasi-coherent $\sh{O}_X$-module of rank $1$ (in particular, every invertible $\sh{O}_X$-module) is isomorphic to a $\sh{O}_X$-submodule of $\sh{R}(X)$, and vice versa.
\end{corollary}

\begin{corollary}[7.4.4]
\label{1.7.4.4}
Let $X$ be an integral prescheme, $\sh{L}$ and $\sh{L}'$ torsion-free $\sh{O}_X$-modules, and $f$ (resp. $f'$) a section of $\sh{L}$ (resp. $\sh{L}'$) over $X$.
In order to have $f\otimes f'=0$, it is necessary and sufficient for one of the sections $f$ and $f'$ to be zero.
\end{corollary}

\begin{proof}
\label{proof-1.7.4.4}
Let $x$ be the generic point of $X$; we have, by hypothesis, that $(f\otimes f')_x=f_x\otimes f'_x=0$.
Since $\sh{L}_x$ and $\sh{L}'_x$ can be identified with $\sh{O}_X$-submodules of the field $\sh{O}_x$, the above equation leads to $f_x=0$ or $f'_x=0$, and thus $f=0$ or $f'=0$, since $\sh{L}$ and $\sh{L}'$ are torsion free \sref{1.7.3.5}.
\end{proof}

\begin{proposition}[7.4.5]
\label{1.7.4.5}
Let $X$ and $Y$ be integral preschemes, and $f:X\to Y$ a dominant morphism.
For every torsion-free quasi-coherent $\sh{O}_X$-module $\sh{F}$, $f_*(\sh{F})$ is a torsion-free $\sh{O}_Y$-module.
\end{proposition}

\begin{proof}
\label{proof-1.7.4.5}
Since
\oldpage[I]{164}
$f_*$ is left exact \sref[0]{0.4.2.1}, it suffices, by \sref{1.7.4.2}, to prove the proposition in the case where $\sh{F}=(\sh{R}(X))^{(I)}$.
But every nonempty open subset $U$ of $Y$ contains the generic point of $Y$, so $f^{-1}(U)$ contains the generic point of $X$ \sref[0]{0.2.1.5}, so we have that $\Gamma(U,f_*(\sh{F}))=\Gamma(f^{-1}(U),\sh{F})=(R(X))^{(I)}$; in other words, $f_*(\sh{F})$ is the simple sheaf with fibre $(R(X))^{(I)}$, considered as a $\sh{R}(Y)$-module, and it is clearly torsion free.
\end{proof}

\begin{proposition}[7.4.6]
\label{1.7.4.6}
Let $X$ be an integral prescheme, and $x$ its generic point.
For every quasi-coherent $\sh{O}_X$-module $\sh{F}$ of finite type, the following conditions are equivalent: \emph{(a)} $\sh{F}$ is a torsion sheaf; \emph{(b)} $\sh{F}_x=0$; \emph{(c)} $\Supp(\sh{F})\neq X$.
\end{proposition}

\begin{proof}
\label{proof-1.7.4.6}
By \sref{1.7.3.5} and \sref{1.7.4.1}, the equations $\sh{F}_x=0$ and $\sh{F}\otimes_{\sh{O}_X}\sh{R}(X)=0$ are equivalent, so (a) and (b) are equivalent; then $\Supp(\sh{F})$ is closed in $X$ \sref[0]{0.5.2.2}, and since every nonempty open subset of $X$ contains $x$, (b) and (c) are equivalent.
\end{proof}

\begin{env}[7.4.7]
\label{1.7.4.7}
We generalise (by an abuse of language) the definitions of \sref{1.7.4.1} to the case where $X$ is a \emph{reduced} prescheme having only a \emph{finite} number of irreducible components; it then follows from \sref{1.7.3.4} that the equivalence between \emph{a)} and \emph{c)} in \sref{1.7.4.6} still holds true for such a prescheme.
\end{env}


\section{Chevalley schemes}
\label{section:chevalley-schemes}

\subsection{Allied local rings}
\label{subsection:allied-local-rings}

For each local ring $A$, we denote by $\fk{m}(A)$ the maximal ideal of $A$.

\begin{lem}[8.1.1]
\label{1.8.1.1}
Let $A$ and $B$ be two local rings such that $A\subset B$;
Then the following conditions are equivalent.
\begin{enumerate}[label=\emph{(\roman*)}]
    \item $\fk{m}(B)\cap A=\fk{m}(A)$.
    \item $\fk{m}(A)\subset\fk{m}(B)$.
    \item $1$ is not an element of
the ideal of $B$ generated by $\fk{m}(A)$.
\end{enumerate}
\end{lem}

\begin{proof}
\label{proof-1.8.1.1}
It is evident that (i) implies (ii), and (ii) implies (iii);
lastly, if (iii) is true, then $\fk{m}(B)\cap A$ contains $\fk{m}(A)$, and does not contain $1$, and is thus equal to $\fk{m}(A)$.

When the equivalent conditions of \sref{1.8.1.1} are satisfied, we say that $B$ \emph{dominates} $A$;
this is equivalent to saying that the injection $A\to B$ is a \emph{local} homomorphism.
It is clear that, in the set of local subrings of a ring $R$, the relation given by domination is an order.
\end{proof}

\begin{env}[8.1.2]
\label{1.8.1.2}
Now consider a \emph{field} $R$.
For all subrings $A$ of $R$, we denote by $L(A)$ the set of local rings $A_\fk{p}$, where $\fk{p}$ ranges over the prime spectrum of $A$;
such local rings are identified with the subrings of $R$ containing $A$.
Since $\fk{p}=(\fk{p}A_\fk{p})\cap A$, the map $\fk{p}\mapsto A_\fk{p}$ from $\Spec(A)$ to $L(A)$ is bijective.
\end{env}

\begin{lem}[8.1.3]
\label{1.8.1.3}
Let $R$ be a field, and $A$ a subring of $R$.
For a local subring $M$ of $R$ to dominate a ring $A_\fk{p}\in L(A)$, it is necessary and sufficient that $A\subset M$;
the local ring $A_\fk{p}$ dominated by $M$ is then unique, and corresponds to $\fk{p}=\fk{m}(M)\cap A$.
\end{lem}

\begin{proof}
\label{proof-1.8.1.3}
If $M$ dominates $A_\fk{p}$, then $\fk{m}(M)\cap A_\fk{p}=\fk{p}A_\fk{p}$, by \sref{1.8.1.1}, whence the uniqueness of $\fk{p}$;
on the other hand, if $A\subset M$, then $\fk{m}M\cap A=\fk{p}$ is prime in $A$, and since $A-\fk{p}\subset M$, we have that $A_\fk{p}\subset M$ and $\fk{p}A_\fk{p}\subset\fk{m}(M)$, so $M$ dominates $A_\fk{p}$.
\end{proof}

\begin{lem}[8.1.4]
\label{1.8.1.4}
\oldpage[I]{165}
Let $R$ be a field, $M$ and $N$ local subrings of $R$, and $P$ the subring of $R$ generated by $M\cup N$.
Then the following conditions are equivalent.
\begin{enumerate}[label=\emph{(\roman*)}]
  \item There exists a prime ideal $\fk{p}$ of $P$ such that $\fk{m}(M)=\fk{p}\cap M$ and $\fk{m}(N)=\fk{p}\cap N$.
  \item The ideal $\fk{a}$ generated in $P$ by $\fk{m}(M)\cup\fk{m}(N)$ is distinct from $P$.
  \item There exists a local subring $Q$ of $R$ simultaneously dominating both $M$ and $N$.
\end{enumerate}
\end{lem}

\begin{proof}
\label{proof-1.8.1.4}
It is clear that (i) implies (ii);
conversely, if $\fk{a}\neq P$, then $\fk{a}$ is contained in a maximal ideal $\fk{n}$ of $P$, and since $1\not\in\fk{n}$, $\fk{n}\cap M$ contains $\fk{m}(M)$ and is distinct from $M$, so $\fk{n}\cap M=\fk{m}(M)$, and similarly $\fk{n}\cap N=\fk{m}(N)$.
It is clear that, if $Q$ dominates both $M$ and $N$, then $P\subset Q$ and $\fk{m}(M)=\fk{m}(Q)\cap M=(\fk{m}(Q)\cap P)\cap M$, and $\fk{m}(N)=(\fk{m}(Q)\cap P)\cap N$, so (iii) implies (i);
the converse is evident when we take $Q=P_\fk{p}$.
\end{proof}

When the conditions of \sref{1.8.1.4} are satisfied, we say, with C.~Chevalley, that the local rings $M$ and $N$ are \emph{allied}.

\begin{prop}[8.1.5]
\label{1.8.1.5}
Let $A$ and $B$ be subrings of a field $R$, and $C$ the subring of $R$ generated by $A\cup B$.
Then the following conditions are equivalent.
\begin{enumerate}[label=\emph{(\roman*)}]
  \item For every local ring $Q$ containing $A$ and $B$, we have that $A_\fk{p}=B_\fk{q}$, where $\fk{p}=\fk{m}(Q)\cap A$ and $\fk{q}=\fk{m}(Q)\cap B$.
  \item For all prime ideals $\fk{r}$ of $C$, we have that $A_\fk{p}=B_\fk{q}$, where $\fk{p}=\fk{r}\cap A$ and $\fk{q}=\fk{r}\cap B$.
  \item If $M\in L(A)$ and $N\in L(B)$ are allied, then they are identical.
  \item $L(A)\cap L(B)=L(C)$.
\end{enumerate}
\end{prop}

\begin{proof}
\label{proof-1.8.1.5}
Lemmas~\sref{1.8.1.3} and \sref{1.8.1.4} prove that (i) and (iii) are equivalent;
it is clear that (i) implies (ii) by taking $Q=C_\fk{r}$;
conversely, (ii) implies (i), because if $Q$ contains $A\cup B$ then it contains $C$, and if $\fk{r}=\fk{m}(Q)\cap C$, then $\fk{p}=\fk{r}\cap A$ and $\fk{q}=\fk{r}\cap B$, by \sref{1.8.1.3}.
It is immediate that (iv) implies (i), because if $Q$ contains $A\cup B$ then it dominates a local ring $C_\fk{r}\in L(C)$ by \sref{1.8.1.3};
by hypothesis we have that $C_\fk{r}\in L(A)\cap L(B)$, and \sref{1.8.1.1} and \sref{1.8.1.3} prove that $C_\fk{r}=A_\fk{p}=B_\fk{q}$.
We prove finally that (iii) implies (iv).
Let $Q\in L(C)$;
$Q$ dominates some $M\in L(A)$ and some $N\in L(B)$ \sref{1.8.1.3}, so $M$ and $N$, being allied, are identical by hypothesis.
As we then have that $C\subset M$, we know that $M$ dominates some $Q'\in L(C)$ \sref{1.8.1.3}, so $Q$ dominates $Q'$, whence necessarily \sref{1.8.1.3} $Q=Q'=M$, so $Q\in L(A)\cap L(B)$.
Conversely, if $Q\in L(A)\cap L(B)$, then $C\subset Q$, so \sref{1.8.1.3} $Q$ dominates some $Q''\in L(C)\subset L(A)\cap L(B)$;
$Q$ and $Q''$, being allied, are identical, so $Q''=Q\in L(C)$, which completes the proof.
\end{proof}

\subsection{Local rings of an integral scheme}
\label{subsection:local-rings-of-integral-scheme}

\begin{env}[8.2.1]
\label{1.8.2.1}
Let $X$ be an \emph{integral} prescheme, and $R$ its field of rational functions, identical to the local ring of the generic point $a$ of $X$;
for all $x\in X$, we know that $\OO_x$ can be canonically identified with a subring of $R$ \sref{1.7.1.5}, and for every rational function $f\in R$, the domain of definition $\delta(f)$ of $f$ is the open set of $x\in X$ such that
$f\in\OO_x$.
It thus follows, from \sref{1.7.2.6}, that, for every open $U\subset X$, we have
\begin{equation*}
  \label{1.8.2.1.1}
  \Gamma(U,\OO_X)=\bigcap_{x\in U}\OO_x.
  \tag{8.2.1.1}
\end{equation*}
\end{env}

\begin{prop}[8.2.2]
\label{1.8.2.2}
\oldpage[I]{166}
Let $X$ be an integral prescheme, and $R$ its field of rational fractions. For $X$ to be a scheme, it is necessary and sufficient for the relation ``$\OO_x$ and $\OO_y$ are allied'' \sref{1.8.1.4}, for points $x$ and $y$ of $X$, to imply that $x=y$.
\end{prop}

\begin{proof}
\label{proof-1.8.2.2}
We suppose that this condition is satisfied, and aim to show that $X$ is separated.
Let $U$ and $V$ be two distinct affine open subsets of $X$, given by rings $A$ and $B$ (respectively), identified with subrings of $R$;
$U$ (resp. $V$) is thus identified \sref{1.8.1.2} with $L(A)$ (resp. $L(B)$), and the hypotheses tell us \sref{1.8.1.5} that $C$ is the subring of $R$ generated by $A\cup B$, and $W=U\cap V$ is identified with $L(A)\cap L(B)=L(C)$.
Furthermore, we know (\cite{I-1}, p.~5-03, 4~\emph{bis}) that every subring $E$ of $R$ is equal to the intersection of the local rings belonging to $L(E)$;
$C$ is thus identified with the intersection of the rings $\OO_z$ for $z\in W$, or, equivalently \sref{1.8.2.1.1}, with $\Gamma(W,\OO_X)$.
So consider the subprescheme induced by $X$ on $W$;
to the identity morphism $\vphi: C\to\Gamma(W,\OO_X)$ there corresponds \sref{1.2.2.4} a morphism $\Phi=(\psi,\theta):W\to\Spec(C)$;
we will see that $\Phi$ is an \emph{isomorphism} of preschemes, whence $W$ is an \emph{affine} open subset.
The identification of $W$ with $L(C)=\Spec(C)$ shows that $\psi$ is \emph{bijective}.
On the other hand, for all $x\in W$, $\theta_x^\sharp$ is the injection $C_\fk{r}\to\OO_x$, where $\fk{r}=\fk{m}_x\cap C$, and, by definition, $C_\fk{r}$ is identified with $\OO_x$, so $\theta_x^\sharp$ is bijective.
It thus remains to show that $\psi$ is a \emph{homeomorphism}, or, in other words, that for every closed subset $F\subset W$, $\psi(F)$ is closed in
$\Spec(C)$.
But $F$ is the intersection of $W$ with a closed subspace of $U$ of the form $V(\fk{a})$, where $\fk{a}$ is an ideal of $A$;
we will show that $\psi(F)=V(\fk{a}C)$, which proves our claim.
In fact, the prime ideals of $C$ containing $\fk{a}C$ are the prime ideals of $C$ containing $\fk{a}$, and so are the ideals of the form $\psi(x)=\fk{m}_x\cap C$, where $\fk{a}\subset\fk{m}_x$ and $x\in W$;
since $\fk{a}\subset\fk{m}_x$ is equivalent to $x\in V(\fk{a})=W\cap F$ for $x\in U$, we do indeed have that $\psi(F)=V(\fk{a}C)$.

It follows that $X$ is separated, because $U\cap V$ is affine and its ring $C$ is generated by the union $A\cup B$ of the rings of $U$ and $V$ \sref{1.5.5.6}.

Conversely, suppose that $X$ is separated, and let $x$ and $y$ be points of $X$ such that $\OO_x$ and $\OO_y$ are allied.
Let $U$ (resp. $V$) be an affine open subset containing $x$ (resp. $y$), of ring $A$ (resp. $B$);
we then know that $U\cap V$ is affine and that its ring $C$ is generated by $A\cup B$ \sref{1.5.5.6}.
If $\fk{p}=\fk{m}_x\cap A$ and $\fk{q}=\fk{m}_y\cap B$, then $A_\fk{p}=\OO_x$ and $B_\fk{q}=\OO_y$, and since $A_\fk{p}$ and $B_\fk{q}$ are allied, there exists a prime ideal $\fk{r}$ of $C$ such that $\fk{p}=\fk{r}\cap A$ and $\fk{q}=\fk{r}\cap B$ \sref{1.8.1.4}.
But then there exists a point $z\in U\cap V$ such that $\fk{r}=\fk{m}_z\cap C$, since $U\cap V$ is affine, and so evidently $x=z$ and $y=z$, whence $x=y$.
\end{proof}

\begin{cor}[8.2.3]
\label{1.8.2.3}
Let $X$ be an integral scheme, and $x$ and $y$ points of $X$.
In order for $x\in\overline{\{y\}}$, it is necessary and
sufficient for $\OO_x\subset\OO_y$, or, equivalently, for every rational
function defined at $x$ to also be defined at $y$.
\end{cor}

\begin{proof}
\label{proof-1.8.2.3}
The condition is evidently necessary because the domain of definition $\delta(f)$ of a rational function $f\in R$ is open;
we now show that it is sufficient.
If $\OO_x\subset\OO_y$, then there exists a prime ideal $\fk{p}$ of $\OO_x$ such that $\OO_y$ dominates $(\OO_x)_\fk{p}$ \sref{1.8.1.3};
but \sref{1.2.4.2} there exists some $z\in X$ such that $x\in\overline{\{z\}}$ and $\OO_z=(\OO_x)_\fk{p}$;
since $\OO_z$ and $\OO_y$ are allied, we have that $z=y$ by \sref{1.8.2.2}, whence the corollary.
\end{proof}

\begin{cor}[8.2.4]
\label{1.8.2.4}
If $X$ is an integral scheme then the map $x\to\OO_x$ is injective; equivalently, if $x$ and $y$ are two distinct points of $X$, then there exists a rational function defined at one of these points but not the other.
\end{cor}

\begin{proof}
\label{proof-1.8.2.4}
\oldpage[I]{167}
This follows from \sref{1.8.2.3} and the axiom ($T_0$) \sref{1.2.1.4}.
\end{proof}

\begin{cor}[8.2.5]
\label{1.8.2.5}
Let $X$ be an integral scheme whose underlying space is Noetherian;
letting $f$ range over the field $R$ of rational functions on $X$, the sets $\delta(f)$ generate the topology of $X$.
\end{cor}

In fact, every closed subset of $X$ is thus a finite union of irreducible closed subsets, or, in other words, of the form $\overline{\{y\}}$ \sref{1.2.1.5}.
But, if $x\not\in\overline{\{y\}}$, then there exists a rational function $f$ defined at $x$ but not at $y$ \sref{1.8.2.3}, or, equivalently, we have that $x\in\delta(f)$ and that $\delta(f)$ is not contained in $\overline{\{y\}}$.
The complement of $\overline{\{y\}}$ is thus a union of sets of the form $\delta(f)$, and, by virtue of the first remark, every open subset of $X$ is the union of finite intersections of open sets of the form $\delta(f)$.

\begin{env}[8.2.6]
\label{1.8.2.6}
Corollary~\sref{1.8.2.5} shows that the topology of $X$ is entirely characterised by the data of the local rings $(\OO_x)_{x\in X}$ that have $R$ as their field of fractions.
It is equivalent to say that the closed subsets of $X$ are defined in the following manner: given a finite subset $\{x_1,\ldots,x_n\}$ of $X$, consider the set of $y\in X$ such that $\OO_y\subset\OO_{x_i}$ for at least one index $i$, and these sets (over all choices of $\{x_1,\ldots,x_n\}$) are the closed subsets of $X$.
Further, once the topology on $X$ is known, the structure sheaf $\OO_X$ is also determined by the family of the $\OO_x$, since $\Gamma(U,\OO_X)=\bigcap_{x\in U}\OO_x$, by \sref{1.8.2.1.1}.
The family $(\OO_X)_{x\in X}$ thus completely determines the prescheme $X$ when $X$ is an integral scheme whose underlying space is Noetherian.
\end{env}

\begin{prop}[8.2.7]
\label{1.8.2.7}
Let $X$ and $Y$ be integral schemes, $f:X\to Y$ a dominant morphism \sref{1.2.2.6}, and $K$ (resp.$L$) the field of rational functions on $X$ (resp.$Y$).
Then $L$ can be identified with a subfield of $K$, and, for all $x\in X$, $\OO_{f(x)}$ is the unique local ring of $Y$ dominated by $\OO_x$.
\end{prop}

\begin{proof}
\label{proof-1.8.2.7}
If $f=(\psi,\theta)$ and $a$ is the generic point of $X$, then $\psi(a)$ is the generic point of $Y$ \sref[0]{0.2.1.5};
$\theta_a^\sharp$ is then a monomorphism of fields, from $L=\OO_{\psi(a)}$ to $K=\OO_a$.
Since every nonempty affine open subset $U$ of $Y$ contains $\psi(a)$, it follows from \sref{1.2.2.4} that the homomorphism $\Gamma(U,\OO_Y)\to\Gamma(\psi^{-1}(U),\OO_X)$ corresponding to $f$ is the restriction of $\theta_a^\sharp$ to $\Gamma(U,\OO_Y)$.
So, for every $x\in X$, $\theta_x^\sharp$ is the restriction to $\OO_{\psi(a)}$ of $\theta_a^\sharp$, and is thus a monomorphism.
We also know that $\theta_x^\sharp$ is a local homomorphism, so, if we identify $L$ with a subfield of $K$ by $\theta_a^\sharp$, $\OO_{\psi(x)}$ is dominated by $\OO_x$ \sref{1.8.1.1};
it is also the only local ring of $Y$ dominated by $\OO_x$, since two local rings of $Y$ that are allied are identical \sref{1.8.2.2}.
\end{proof}

\begin{prop}[8.2.8]
\label{1.8.2.8}
Let $X$ be an \emph{irreducible} prescheme, $f:X\to Y$ a local immersion (\emph{resp.} local isomorphism), and suppose further that $f$ is separated. Then $f$ is an immersion (\emph{resp.} an open immersion).
\end{prop}

\begin{proof}
\label{proof-1.8.2.8}
Let $f=(\psi,\theta)$;
it suffices, in both cases, to prove that $\psi$ is a \emph{homeomorphism} from $X$ to $\psi(X)$ \sref{1.4.5.3}.
Replacing $f$ by $f_\text{red}$ (\sref{1.5.1.6} and \sref{1.5.5.1}[vi]), we can assume that $X$ and $Y$ are \emph{reduced}.
If $Y'$ is the closed reduced subprescheme of $Y$ that has $\overline{\psi(X)}$ as its underlying space, then $f$ factors as $X\xrightarrow{f'}Y'\xrightarrow{j}Y$, where $j$ is the canonical injection \sref{1.5.2.2}.
It follows from \sref{1.5.5.1}[v] that $f'$ is again a separated morphism; further, $f'$ is again
\oldpage[I]{168}
a local immersion (resp. a local isomorphism), because, since the condition is local on $X$ and $Y$, we can restrict to the case where $f$ is a closed immersion (resp. open immersion), and our claim then follows immediately from \sref{1.4.2.2}.

We can thus suppose that $f$ is a \emph{dominant} morphism, which leads to the fact that $Y$ is, itself, irreducible \sref[0]{0.2.1.5}, and so $X$ and $Y$ are both \emph{integral}.
Further, the condition being local on $Y$, we can suppose that $Y$ is an affine scheme;
since $f$ is separated, $X$ is a scheme \sref{1.5.5.1}[ii], and we are finally at the hypotheses of Proposition~\sref{1.8.2.7}.
Then, for all $x\in X$, $\theta_x^\sharp$ is injective;
but the hypothesis that $f$ is a local immersion implies that $\theta_x^\sharp$ is surjective \sref{1.4.2.2}, so $\theta_x^\sharp$ is bijective, or, equivalently (with the identification of Proposition~\sref{1.8.2.7}) we have that $\OO_{\psi(x)}=\OO_x$.
This implies, by Corollary~\sref{1.8.2.4}, that $\psi$ is an \emph{injective} map, which already proves the proposition when $f$ is a local isomorphism \sref{1.4.5.3}.
When we suppose that $f$ is only a local immersion, for all $x\in X$ there exists an open neighborhood $U$ of $x$ in $X$ and an open neighborhood $V$ of $\psi(x)$ in $Y$ such that the restriction of $\psi$ to $U$ is a homeomorphism from $U$ to a \emph{closed} subset of $V$.
But $U$ is dense in $X$, so $\psi(U)$ is dense in $Y$ and \emph{a fortiori} in $V$, which proves that $\psi(U)=V$;
since $\psi$ is injective, $\psi^{-1}(V)=U$ and this proves that $\psi$ is a homeomorphism from $X$ to $\psi(X)$.
\end{proof}

\subsection{Chevalley schemes}
\label{subsection:chevalley-schemes}

\begin{env}[8.3.1]
\label{1.8.3.1}
Let $X$ be a \emph{Noetherian} integral scheme, and $R$ its field of rational functions;
we denote by $X'$ the set of local subrings $\OO_x\subset R$, where $x$ ranges over all points of $X$.
The set $X'$ satisfies the following three conditions.
\begin{enumerate}
  \item[(Sch.~1)] For all $M\in X'$, $R$ is the field of fractions of $M$.
  \item[(Sch.~2)] There exists a finite set of Noetherian subrings $A_i$ of $R$ such that $X'=\bigcup_i L(A_i)$, and, for all pairs of indices $i$, $j$, the subring $A_{ij}$ of $R$ generated by $A_i\cup A_j$ is an algebra of finite type over $A_i$.
  \item[(Sch.~3)] Any two elements $M$ and $N$ of $X'$ that are allied are identical.
\end{enumerate}
\end{env}

We have seen in \sref{1.8.2.1} that (Sch.~1) is satisfied, and (Sch.~3) follows from \sref{1.8.2.2}.
To show (Sch.~2), it suffices to cover $X$ by a finite number of affine open subsets $U_i$ whose rings are Noetherian, and to take $A_i=\Gamma(U_i,\OO_X)$;
the hypothesis that $X$ is a scheme implies that $U_i\cap U_j$ is affine, and also that $\Gamma(U_i\cap U_j,\OO_X)=A_{ij}$ \sref{1.5.5.6};
further, since the space $U_i$ is Noetherian, the immersion $U_i\cap U_j\to U_i$ is of finite type \sref{1.6.3.5}, so $A_{ij}$ is an $A_i$-algebra of finite type \sref{1.6.3.3}.

\begin{env}[8.3.2]
\label{1.8.3.2}
The structures whose axioms are (Sch.~1), (Sch.~2), and (Sch.~3) generalise ``schemes'', in the sense of C.~Chevalley, who additionally supposes that $R$ is an extension of finite type of a field $K$, and that the $A_i$ are $K$-algebras of finite type (which renders a part of (Sch.~2) useless) \cite{I-1}.
Conversely, if we have such a structure on a set $X'$, then we can associate to it an integral scheme $X$ by using the remarks from \sref{1.8.2.6}: the underlying space of $X$ is equal to $X'$ endowed with the topology defined in \sref{1.8.2.6}, and with the sheaf $\OO_X$ such that $\Gamma(U,\OO_X)=\bigcap_{x\in U}\OO_x$ for all open $U\subset X$, with the evident definition of restriction homomorphisms.
We leave to the reader the task of verifying that we thus obtain an integral scheme, whose local rings are the elements of $X'$;
we will not use this result in what follows.
\end{env}


\section{Supplement on quasi-coherent sheaves}
\label{section:supplement-on-qcoh}

\subsection{Tensor product of quasi-coherent sheaves}
\label{subsection:tensor-product-of-qcoh}

\begin{prop}[9.1.1]
\label{1.9.1.1}
\oldpage[I]{169}
Let $X$ be a prescheme (resp. a locally Noetherian prescheme). Let $\sh{F}$ and
$\sh{G}$ be quasi-coherent (resp. coherent) $\OO_X$-modules; then
$\sh{F}\otimes_{\OO_X}\sh{G}$ is quasi-coherent (resp. coherent); it is further
of finite type if both $\sh{F}$ and $\sh{G}$ are of finite type. If
$\sh{F}$ admits a finite presentation and if $\sh{G}$ is quasi-coherent
(resp. coherent), then $\shHom(\sh{F},\sh{G})$ is quasi-coherent
(resp. coherent).
\end{prop}

\begin{proof}
\label{proof-1.9.1.1}
Being a local proposition, we can suppose that $X$ is affine (resp. Noetherian
affine); further, if $\sh{F}$ is coherent, then we can assume that it is the
cokernel of a homomorphism $\OO_X^m\to\OO_X^n$. The claims pertaining to
quasi-coherent sheaves then follow from Corollaries~\sref{1.1.3.12} and \sref{1.1.3.9}; the
claims pertaining to coherent sheaves follow from Theorem~\sref{1.1.5.1} and from the fact
that if $M$ and $N$ are modules of finite type over a Noetherian ring $A$
then $M\otimes_A N$ and $\Hom_A(M,N)$ are both $A$-modules of finite type.
\end{proof}

\begin{defn}[9.1.2]
\label{1.9.1.2}
Let $X$ and $Y$ be $S$-preschemes, $p$ and
$q$ the projections of $X\times_S Y$, and $\sh{F}$ (resp.$\sh{G}$) a
quasi-coherent $\OO_X$-module (resp. quasi-coherent $\OO_Y$-module). We define the
tensor product of $\sh{F}$ and $\sh{G}$ over $\OO_S$ (\emph{or} over $S$),
denoted by $\sh{F}\otimes_{\OO_S}\sh{G}$ (\emph{or}
$\sh{F}\otimes_S\sh{G}$) to be the tensor product
$p^*(\sh{F})\otimes_{\OO_{X\times_S Y}}q^*(\sh{G})$ over the
prescheme $X\times_S Y$.
\end{defn}

If $X_i$ ($1\leq i\leq n$) are $S$-preschemes, and $\sh{F}_i$ ($1\leq i\leq n$) are quasi-coherent
$\OO_{X_i}$-modules, then we similarly define the tensor product
$\sh{F}_1\otimes_S\sh{F}_2\otimes_S\cdots\otimes_S\sh{F}_n$ over the
prescheme $Z=X_1\times_S X_2\times_S\cdots\times_S X_n$; it is a
\emph{quasi-coherent} $\OO_Z$-module by virtue of \sref{1.9.1.1} and
\sref[0]{0.5.1.4}; it is further \emph{coherent} if all the $\sh{F}_i$ are coherent \emph{and}
$Z$ is \emph{locally Noetherian}, by virtue of \sref{1.9.1.1},
\sref[0]{0.5.3.11}, and \sref{1.6.1.1}.

Note that, if we take $X=Y=S$, then Definition~\sref{1.9.1.2} gives us back the tensor
product of $\OO_S$-modules. Furthermore, since $q^*(\OO_Y)=\OO_{X\times_S Y}$
\sref[0]{0.4.3.4}, the product $\sh{F}\otimes_S\OO_Y$ is canonically
identified with $p^*(\sh{F})$, and, in the same way,
$\OO_X\otimes_S\sh{G}$ is canonically identified with $q^*(\sh{G})$. In
particular, if we take $Y=S$ and denote by $f$ the structure morphism $X\to Y$,
then we have that $\OO_X\otimes_Y\sh{G}=f^*(\sh{G})$: the ordinary tensor
product and the inverse image thus appear as particular cases of the general
tensor product.

Definition~\sref{1.9.1.2} leads immediately to the fact that, for fixed $X$ and
$Y$, $\sh{F}\otimes_S\sh{G}$ is a \emph{right-exact additive covariant bifunctor} in $\sh{F}$ and $\sh{G}$.

\begin{prop}[9.1.3]
\label{1.9.1.3}
Let $S$, $X$, and $Y$ be affine schemes of rings
$A$, $B$, and $C$ (respectively), with $B$ and $C$ being $A$-algebras. Let $M$
(resp. $N$) be a $B$-module (resp. $C$-module), and
$\sh{F}=\wt{M}$ (resp. $\sh{G}=\wt{N}$) the
associated quasi-coherent sheaf; then $\sh{F}\otimes_S\sh{G}$ is
canonically isomorphic to the sheaf associated to the $(B\otimes_A C)$-module
$M\otimes_A N$.
\end{prop}

\begin{proof}
\label{proof-1.9.1.3}
\oldpage[I]{170}
According to Proposition~\sref{1.1.6.5}, $\sh{F}\otimes_S\sh{G}$
is canonically isomorphic to the sheaf associated to the $(B\otimes_A C)$-module
\[
  \big(M\otimes_B(B\otimes_A C)\big)\otimes_{B\otimes_A C}\big((B\otimes_A C)\otimes_C N\big)
\]
and, by the canonical isomorphisms between tensor
products, this latter module is isomorphic to
\[
  M\otimes_B(B\otimes_A C)\otimes_C N=(M\otimes_B B)\otimes_A(C\otimes_C N)=M\otimes_A N.\qedhere
\]
\end{proof}

\begin{prop}[9.1.4]
\label{1.9.1.4}
Let $f:T\to X$ and $g:T\to Y$ be $S$-morphisms, and $\sh{F}$ (resp. $\sh{G}$) a quasi-coherent
$\OO_X$-module (resp. quasi-coherent $\OO_Y$-module). Then
\[
  (f,g)^*_S(\sh{F}\otimes_S\sh{G})=f^*(\sh{F})\otimes_{\OO_T}g^*(\sh{G}).
\]
\end{prop}

\begin{proof}
\label{proof-1.9.14}
If $p$, $q$ are the projections of $X\times_S Y$, then the formula follows
from the equalities $(f,g)^*_S\circ p^*=f^*$ and
$(f,g)^*_S\circ q^*=g^*$ \sref[0]{0.3.5.5}, and the fact that the inverse
image of a tensor product of algebraic sheaves is the tensor product of their inverse
images \sref[0]{0.4.3.3}.
\end{proof}

\begin{cor}[9.1.5]
\label{1.9.1.5}
Let $f:X\to X'$ and $g:Y\to Y'$ be
$S$-morphisms, and $\sh{F}'$ (resp. $\sh{G}'$) a quasi-coherent
$\OO_{X'}$-module (resp. quasi-coherent $\OO_{Y'}$-module). Then
\[
  (f,g)^*_S(\sh{F}'\otimes_S\sh{G}')=f^*(\sh{F}')\otimes_S g^*(\sh{G}')
\]
\end{cor}

\begin{proof}
\label{proof-1.9.1.5}
This follows from \sref{1.9.1.4} and the fact that $f\times_S g=(f\circ p, g\circ q)_S$,
where $p$ and $q$ are the projections of $X\times_S Y$.
\end{proof}

\begin{cor}[9.1.6]
\label{1.9.1.6}
Let $X$, $Y$, and $Z$ be $S$-preschemes, and $\sh{F}$ (resp. $\sh{G}$, $\sh{H}$) a
quasi-coherent $\OO_X$-module (resp. quasi-coherent $\OO_Y$-module, quasi-coherent
$\OO_Z$-module); then the sheaf $\sh{F}\otimes_S\sh{G}\otimes_S\sh{H}$ is the inverse image
of $(\sh{F}\otimes_S\sh{G})\otimes_S\sh{H}$ by the canonical isomorphism from
$X\times_S Y\times_S Z$ to $(X\times_S Y)\times_S Z$.
\end{cor}

\begin{proof}
\label{proof-1.9.16}
This isomorphism is given by $(p_1,p_2)_S\times_S p_3$, where $p_1$, $p_2$, and $p_3$
are the projections of $X\times_S Y\times_S Z$.

Similarly, the inverse image of $\sh{G}\otimes_S\sh{F}$ under the canonical isomorphism from
$X\times_S Y$ to $Y\times_S X$ is $\sh{F}\otimes_S\sh{G}$.
\end{proof}

\begin{cor}[9.1.7]
\label{1.9.1.7}
If $X$ is an $S$-prescheme, then every quasi-coherent $\OO_X$-module $\sh{F}$ is the inverse
image of $\sh{F}\otimes_S\OO_S$ by the canonical isomorphism from $X$ to $X\times_S S$
\sref{1.3.3.3}.
\end{cor}

\begin{proof}
\label{proof-1.9.1.7}
This isomorphism is $(1_X,\vphi)_S$, where $\vphi$ is the structure morphism
$X\to S$, and the corollary follows from \sref{1.9.1.4} and the fact that
$\vphi^*(\OO_S)=\OO_X$.
\end{proof}

\begin{env}[9.1.8]
\label{1.9.1.8}
Let $X$ be an $S$-prescheme, $\sh{F}$ a quasi-coherent
$\OO_X$-module, and $\vphi:S'\to S$ a morphism; we denote by
$\sh{F}_{(\vphi)}$ or $\sh{F}_{(S')}$ the quasi-coherent sheaf
$\sh{F}\otimes_S\OO_{S'}$ over $X\times_S S'=X_{(\vphi)}=X_{(S')}$; so
$\sh{F}_{(S')}=p^*(\sh{F})$, where $p$ is the projection $X_{(S')}\to X$.
\end{env}

\begin{prop}[9.1.9]
\label{1.9.1.9}
Let $\vphi'':S''\to S'$ be a morphism.
For every quasi-coherent $\OO_X$-module $\sh{F}$ on the $S$-prescheme $X$,
$(\sh{F}_{(\vphi)})_{(\vphi')}$ is the inverse image of
$\sh{F}_{(\vphi\circ\vphi')}$ by the canonical isomorphism
$(X_{(\vphi)})_{(\vphi')}\isoto X_{(\vphi\circ\vphi')}$
\sref{1.3.3.9}.
\end{prop}

\begin{proof}
\label{proof-1.9.1.9}
This follows immediately from the definitions and from \sref{1.3.3.9}, and is
written
\[
  (\sh{F}\otimes_S\OO_{S'})\otimes_{S'}\OO_{S''}=\sh{F}\otimes_S\OO_{S''}.
  \tag{9.1.9.1}\qedhere
\]
\end{proof}

\begin{prop}[9.1.10]
\label{1.9.1.10}
Let $Y$ be an $S$-prescheme, and $f:X\to Y$ an $S$-morphism.
For every quasi-coherent $\OO_Y$-module $\sh{G}$ and every morphism
$S'\to S$, we have that
$(f_{(S')})^*(\sh{G}_{(S')})=(f^*(\sh{G}))_{(S')}$.
\end{prop}

\begin{proof}
\label{proof-1.9.1.10}
This follows immediately from the commutativity of the diagram
\oldpage[I]{171}
\[
  \xymatrix{
    X_{(S')}\ar[r]^{f_{(S')}}\ar[d] &
    Y_{(S')}\ar[d]\\
    X\ar[r]^f &
    Y.
  }
\]
\end{proof}

\begin{cor}[9.1.11]
\label{1.9.1.11}
Let $X$ and $Y$ be $S$-preschemes, and
$\sh{F}$ (resp. $\sh{G}$) a quasi-coherent $\OO_X$-module
(resp. quasi-coherent $\OO_Y$-module). Then the inverse image of the sheaf
$(\sh{F}_{(S')})\otimes_{(S')}(\sh{G}_{(S')})$ by the canonical isomorphism
$(X\times_S Y)_{(S')}\isoto(X_{(S')})\times_{S'}(Y_{(S')})$
\sref{1.3.3.10} is equal to $(\sh{F}\otimes_S\sh{G})_{(S')}$.
\end{cor}

\begin{proof}
\label{proof-1.9.1.11}
If $p$ and $q$ are the projections of $X\times_S Y$, then the isomorphism in question
is nothing but $(p_{(S')}, q_{(S')})_{S'}$; the corollary then follows from
Propositions~\sref{1.9.1.4} and \sref{1.9.1.10}.
\end{proof}

\begin{prop}[9.1.12]
\label{1.9.1.12}
With the notation from Definition~\sref{1.9.1.2}, let $z$ be
a point of $X\times_S Y$, and let $x=p(z)$, and $y=q(z)$; the stalk
$(\sh{F}\otimes_S\sh{G})_z$ is isomorphic to
$(\sh{F}_x\otimes_{\OO_x}\OO_z)\otimes_{\OO_z}(\sh{G}_y\otimes_{\OO_y}\OO_z)
  =\sh{F}_x\otimes_{\OO_x}\OO_z\otimes_{\OO_y}\otimes\sh{G}_y$.
\end{prop}

\begin{proof}
\label{proof-1.9.1.12}
Since we can reduce to the affine case, the proposition follows from
Equation~\sref{1.1.6.5.1}.
\end{proof}

\begin{cor}[9.1.13]
\label{1.9.1.13}
If $\sh{F}$ and $\sh{G}$ are of finite type, then
\[
  \Supp(\sh{F}\otimes_S\sh{G})=p^{-1}(\Supp(\sh{F}))\cap q^{-1}(\Supp(\sh{G})).
\]
\end{cor}

\begin{proof}
\label{proof-1.9.1.13}
Since $p^*(\sh{F})$ and $q^*(\sh{G})$ are both of finite type over
$\OO_{X\times_S Y}$, we reduce, by Proposition~\sref{1.9.1.12} and
by \sref[0]{0.1.7.5}, to the case where $\sh{G}=\OO_Y$, that
is, it remains to prove the following equation:
\begin{equation*}
\label{1.9.1.13.1}
  \Supp(p^*(\sh{F}))=p^{-1}(\Supp(\sh{F})).
  \tag{9.1.13.1}
\end{equation*}

The same reasoning as in \sref[0]{0.1.7.5} leads us to prove that, for all
$z\in X\times_S Y$, we have $\OO_z/\fk{m}_x\OO_z\neq0$ (with $x=p(z)$),
which follows from the fact that the homomorphism $\OO_x\to\OO_z$ is \emph{local},
by hypothesis.
\end{proof}

We leave it to the reader to extend the results in this section to the more
general case of an arbitrary (but finite) number of factors, instead of just two.

\subsection{Direct image of a quasi-coherent sheaf}
\label{subsection:direct-image-of-qcoh}

\begin{prop}[9.2.1]
\label{1.9.2.1}
Let $f:X\to Y$ be a morphism of
preschemes. We suppose that there exists a cover $(Y_\alpha)$ of $Y$ by affine
opens having the following property: every $f^{-1}(Y_\alpha)$ admits a
\emph{finite} cover $(X_{\alpha i})$ by affine opens contained in
$f^{-1}(Y_\alpha)$ such that every intersection $X_{\alpha i}\cap X_{\alpha j}$
is itself a \emph{finite} union of affine opens. With these hypotheses, for
every quasi-coherent $\OO_X$-module $\sh{F}$, $f_*(\sh{F})$ is a
quasi-coherent $\OO_Y$-module.
\end{prop}

\begin{proof}
\label{proof-1.9.2.1}
Since this is a local condition on $Y$, we can assume that $Y$ is equal to one
of the $Y_\alpha$, and thus omit the indices $\alpha$.
\begin{enumerate}[label=(\alph*)]
  \item First, suppose that the $X_i\cap X_j$
    are themselves \emph{affine} opens. We set $\sh{F}_i=\sh{F}|X_i$ and
    $\sh{F}_{ij}=\sh{F}|(X_i\cap X_j)$, and let $\sh{F}'_i$ and
    $\sh{F}'_{ij}$ be the images of $\sh{F}_i$ and $\sh{F}_{ij}$
    (respectively) by the restriction of $f$ to $X_i$ and $X_i\cap X_j$
    (respectively); we know that the $\sh{F}'_i$ and $\sh{F}'_{ij}$ are
    quasi-coherent \sref{1.1.6.3}. Set $\sh{G}=\bigoplus_i\sh{F}'_i$ and
    $\sh{H}=\bigoplus_{i,j}\sh{F}'_{ij}$; $\sh{G}$ and $\sh{H}$ are
    quasi-coherent $\OO_Y$-modules; we will define a homomorphism
    $u:\sh{G}\to\sh{H}$ such that $f_*(\sh{F})$ is the
    \emph{kernel} of $u$; it will follow from this that $f_*(\sh{F})$ is
    quasi-coherent \sref{1.1.3.9}. It suffices to define $u$ as
\oldpage[I]{172}
    a homomorphism of presheaves; taking into account the definitions of $\sh{G}$
    and $\sh{H}$, it thus suffices, for every open subset $W\subset Y$, to define a
    homomorphism
    \[
      u_W:\bigoplus_i\Gamma(f^{-1}(W)\cap X_i,\sh{F})\to\bigoplus_{i,j}\Gamma(f^{-1}(W)\cap X_i\cap X_j,\sh{F})
    \]
    in such a way that it satisfies the usual compatibility conditions when $W$ varies.
    If for every section $s_i\in\Gamma(f^{-1}(W)\cap X_i,\sh{F})$ we denote by $s_{i|j}$ its restriction to $f^{-1}(W)\cap X_i\cap X_j$, then we set
    \[
      u_W\big((s_i)\big)=(s_{i|j}-s_{j|i})
    \]
    and the compatibility conditions are clearly satisfied.
    To prove that the kernel $\sh{R}$ of $u$ is $f_*(\sh{F})$, we define a homomorphism from $f_*(\sh{F})$ to $\sh{R}$ by sending each section $s\in\Gamma(f^{-1}(W),\sh{F})$ to the family $(s_i)$, where $s_i$ is the restriction of $s$ to $f^{-1}(W)\cap X_i$; the axioms (F1) and (F2) of sheaves (G, II, 1.1) tell us that this homomorphism is \emph{bijective}, which finishes the proof in this case.
  \item In the general case, the same reasoning applies once we have established that the $\sh{F}_{ij}$ are quasi-coherent.
    But, by hypothesis, $X_i\cap X_j$ is a finite union of affine opens $X_{ijk}$; and since the $X_{ijk}$ are affine opens \emph{in a scheme}, the intersection of any two of them is again an affine open \sref{1.5.5.6}. We are thus led to the first case, and so we have proved Proposition \sref{1.9.2.1}.
\end{enumerate}
\end{proof}

\begin{cor}[9.2.2]
\label{1.9.2.2}
The conclusion of Proposition \sref{1.9.2.1} holds true in each of the following cases:
\begin{enumerate}[label=\emph{(\alph*)}]
  \item $f$ is separated and quasi-compact.
  \item $f$ is separated and of finite type.
  \item $f$ is quasi-compact and the underlying space of $X$ is locally Noetherian.
\end{enumerate}
\end{cor}

\begin{proof}
\label{proof-1.9.2.2}
In the case (a), the $X_{\alpha i}\cap X_{\alpha j}$ are affine \sref{1.5.5.6}.
Case (b) is a particular case of (a) \sref{1.6.6.3}. Finally, in case (c), we can reduce to the case where $Y$ is affine and the underlying space of $X$ is Noetherian; then $X$ admits a finite cover of affine opens $(X_i)$, and the $X_i\cap X_j$, being quasi-compact, are finite unions of affine opens \sref{1.2.1.3}.
\end{proof}

\subsection{Extension of sections of quasi-coherent sheaves}
\label{subsection:extension-of-sections-of-qcoh}

\begin{thm}[9.3.1]
\label{1.9.3.1}
Let $X$ be a prescheme whose underlying space is Noetherian, or a scheme whose underlying
space is quasi-compact. Let $\sh{L}$ be an invertible $\OO_X$-module \sref[0]{0.5.4.1}, $f$ a
section of $\sh{L}$ over $X$, $X_f$ the open set of $x\in X$ such that $f(x)\neq0$
\sref[0]{0.5.5.1}, and $\sh{F}$ a quasi-coherent $\OO_X$-module.
\begin{enumerate}[label=\emph{(\roman*)}]
  \item If $s\in\Gamma(X,\sh{F})$ is such that $s|X_f=0$, then there exists an integer $n>0$ such that $s\otimes f^{\otimes n}=0$.
  \item For every section $s\in\Gamma(X_f,\sh{F})$, there exists an integer $n>0$ such that $s\otimes f^{\otimes n}$ extends to a section of $\sh{F}\otimes\sh{L}^{\otimes n}$ over $X$.
\end{enumerate}
\end{thm}

\begin{proof}
\label{proof-1.9.3.1}
\medskip\noindent
\begin{enumerate}[label=(\roman*)]
  \item Since the underlying space of $X$ is quasi-compact, and thus the union of
    finitely-many affine opens $U_i$ with $\sh{L}|U_i$ is isomorphic to
    $\OO_X|U_i$, we can reduce to the case where $X$ is affine and $\sh{L}=\OO_X$.
    In this case, $f$ is identified with an element of $A(X)$, and we have that
    $X_f=D(f)$; $s$ is identified with an element of an $A(X)$-module $M$, and
    $s|X_f$ to the corresponding element of $M_f$, and the result is then trivial,
    recalling the definition of a module of fractions.
\oldpage[I]{173}
  \item Again, $X$ is a finite union of affine opens $U_i$ ($1\leq i\leq r$)
    such that $\sh{L}|U_i\cong\OO_X|U_i$, and for every $i$,
    $(s\otimes f^{\otimes n})|(U_i\cap X_f)$ is identified (by the aforementioned
    isomorphism) with $(f|(U_i\cap X_f))^n(s|(U_i\cap X_f))$. We then know
    \sref{1.1.4.1} that there exists an integer $n>0$ such that, for all
    $i$, $(s\otimes f^{\otimes n})|(U_i\cap X_f)$ extends to a section $s_i$ of
    $\sh{F}\otimes\sh{L}^{\otimes n}$ over $U_i$. Let $s_{i|j}$ be the restriction
    of $s_i$ to $U_i\cap U_j$; by definition we have that $s_{i|j}-s_{j|i}=0$ on
    $X_f\cap U_i\cap U_j$. But, if $X$ is a Noetherian space, then $U_i\cap U_j$ is
    quasi-compact; if $X$ is a scheme, then $U_i\cap U_j$ is an affine open
    \sref{1.5.5.6}, and so again quasi-compact. By virtue of (i), there thus
    exists an integer $m$ (independent of $i$ and $j$) such that
    $(s_{i|j}-s_{j|i})\otimes f^{\otimes m}=0$. It immediately follows that there
    exists a section $s'$ of $\sh{F}\otimes\sh{L}^{\otimes(n+m)}$ over $X$,
    restricting to $s_i\otimes f^{\otimes m}$ over each $U_i$, and restricting to
    $s\otimes f^{\otimes(n+m)}$ over $X_f$.
\end{enumerate}
\end{proof}

The following corollaries give an interpretation of Theorem \sref{1.9.3.1} in a more
algebraic language:
\begin{cor}[9.3.2]
\label{1.9.3.2}
With the hypotheses of \sref{1.9.3.1}, consider the graded ring $A_*=\Gamma_*(\sh{L})$
and the graded $A_*$-module $M_*=\Gamma_*(\sh{L},\sh{F})$ \sref[0]{0.5.4.6}. If $f\in A_n$,
where $n\in\bb{Z}$, then there is a canonical isomorphism
$\Gamma(X_f,\sh{F})\isoto((M_*)_f)_0$ (\emph{the subgroup of the module of
fractions $(M_*)_f$ consisting of elements of degree $0$}).
\end{cor}

\begin{cor}[9.3.3]
\label{1.9.3.3}
Suppose that the hypotheses of \sref{1.9.3.1} are satisfied, and suppose further that
$\sh{L}=\OO_X$. Then, setting $A=\Gamma(X,\OO_X)$ and $M=\Gamma(X,\sh{F})$, the $A_f$-module
$\Gamma(X_f,\sh{F})$ is canonically isomorphic to $M_f$.
\end{cor}

\begin{prop}[9.3.4]
\label{1.9.3.4}
Let $X$ be a Noetherian prescheme, $\sh{F}$ a coherent $\OO_X$-module, and $\sh{J}$ a
coherent sheaf of ideals in $\OO_X$, such that the support of $\sh{F}$ is contained in that
of $\OO_X|\sh{J}$. Then there exists a whole number $n>0$ such that $\sh{J}^n\sh{F}=0$.
\end{prop}

\begin{proof}
\label{proof-1.9.3.4}
Since $X$ is a union of finitely-many affine opens whose rings are Noetherian, we can suppose
that $X$ is affine of Noetherian ring $A$; then $\sh{F}=\wt{M}$, where
$M=\Gamma(X,\sh{F})$ is an $A$-module of finite type, and $\sh{J}=\wt{\fk{J}}$,
where $\fk{J}=\Gamma(X,\sh{J})$ is an ideal of $A$ (\sref{1.1.4.1} and
\sref{1.1.5.1}). Since $A$ is Noetherian, $\fk{J}$ admits a finite system of
generators $f_i$ ($1\leq i\leq m$). By hypothesis, every section of $\sh{F}$ over
$X$ is zero on each of the $D(f_i)$; if $s_j$ ($1\leq j\leq q$) are sections of
$\sh{F}$ generating $M$, then there exists a whole number $h$, independent of $i$ and $j$,
such that $f_i^h s_j=0$ \sref{1.1.4.1}, whence $f_i^h s=0$ for all $s\in M$. We thus
conclude that if $n=mh$ then $\fk{J}^n M=0$, and so the corresponding $\OO_X$-module
$\sh{J}^n\sh{F}=\wt{\fk{J}^n M}$ \sref{1.1.3.13} is zero.
\end{proof}

\begin{cor}[9.3.5]
\label{1.9.3.5}
With the hypotheses of \sref{1.9.3.4}, there exists a closed subprescheme $Y$ of $X$,
whose underlying space is the support of $\OO_X/\sh{J}$, such that, if $j:Y\to X$ is the
canonical injection, then $\sh{F}=j_*(j^*(\sh{F}))$.
\end{cor}

\begin{proof}
\label{proof-1.9.3.5}
First, note that the supports of $\OO_X/\sh{J}$ and $\OO_X/\sh{J}^n$ are the same,
since, if $\sh{J}_x=\OO_x$, then $\sh{J}_x^n=\OO_x$, and we also have that
$\sh{J}_x^n\subset\sh{J}_x$ for all $x\in X$. We can, thanks to \sref{1.9.3.4}, thus
suppose that $\sh{J}\sh{F}=0$; we can then take $Y$ to be the closed subprescheme of $X$
defined by $\sh{J}$, and since $\sh{F}$ is then an $(\OO_X/\sh{J})$-module, the conclusion
follows immediately.
\end{proof}

\subsection{Extension of quasi-coherent sheaves}
\label{subsection:extension-of-qcoh}

\begin{env}[9.4.1]
\label{1.9.4.1}
Let
\oldpage[I]{174}
$X$ be a topological space, $\sh{F}$ a sheaf of sets (resp. of groups, of rings) on $X$, $U$
an open subset of $X$, $\psi:U\to X$ the canonical injection, and $\sh{G}$ a subsheaf of
$\sh{F}|U=\psi^*(\sh{F})$. Since $\psi_*$ is left exact, $\psi_*(\sh{G})$ is a subsheaf of
$\psi_*(\psi^*(\sh{F}))$; if we denote by $\rho$ the canonical homomorphism
$\sh{F}\to\psi_*(\psi^*(\sh{F}))$ \sref[0]{0.3.5.3}, then we denote by $\overline{\sh{G}}$
the subsheaf $\rho^{-1}(\psi_*(\sh{G}))$ of $\sh{F}$. It follows immediately from the
definitions that, for every open subset $V$ of $X$, $\Gamma(V,\overline{\sh{G}})$ consists of
sections $s\in\Gamma(V,\sh{F})$ whose restriction to $V\cap U$ is a section of $\sh{G}$ over
$V\cap U$. We thus have that $\overline{\sh{G}}|U=\psi^*(\overline{\sh{G}})=\sh{G}$, and that
$\overline{\sh{G}}$ is the \emph{largest} subsheaf of $\sh{F}$ that restricts to $\sh{G}$
over $U$; we say that $\overline{\sh{G}}$ is the \emph{canonical extension} of the subsheaf
$\sh{G}$ of $\sh{F}|U$ to a subsheaf of $\sh{F}$.
\end{env}

\begin{prop}[9.4.2]
\label{1.9.4.2}
Let $X$ be a prescheme, $U$ an open subset of $X$ such that the canonical injection
$j:U\to X$ is a quasi-compact morphism \emph{(which will be the case for \emph{all} $U$ if
the underlying space of $X$ is \emph{locally Noetherian}
\sref{1.6.6.4}[i])}. Then:
\begin{enumerate}[label=\emph{(\roman*)}]
  \item For every quasi-coherent $(\OO_X|U)$-module $\sh{G}$, $j_*(\sh{G})$
    is a quasi-coherent $\OO_X$-module, and $j_*(\sh{G})|U=j^*(j_*(\sh{G}))=\sh{G}$.
  \item For every quasi-coherent $\OO_X$-module $\sh{F}$ and every quasi-coherent
    $(\OO_X|U)$-submodule $\sh{G}$, the canonical extension
    $\overline{\sh{G}}$ of $\sh{G}$ \sref{1.9.4.1} is a
    quasi-coherent $\OO_X$-submodule of $\sh{F}$.
\end{enumerate}
\end{prop}

\begin{proof}
\label{proof-1.9.4.2}
If $j=(\psi,\theta)$ ($\psi$ being the injection $U\to X$ of underlying spaces), then by
definition we have that $j_*(\sh{G})=\psi_*(\sh{G})$ for every $(\OO_X|U)$-module $\sh{G}$,
and, further, that $j^*(\sh{H})=\psi^*(\sh{H})=\sh{H}|U$ for every $\OO_X$-module $\sh{H}$,
by definition of the prescheme induced over an open subset. So (i) is thus a particular case
of (\sref{1.9.2.2}, (a)); for the same reason, $j_*(j^*(\sh{F}))$ is quasi-coherent, and
since $\overline{\sh{G}}$ is the inverse image of $j_*(\sh{G})$ by the homomorphism
$\rho:\sh{F}\to j_*(j^*(\sh{F}))$, (ii) follows from \sref{1.4.1.1}.
\end{proof}

Note that the hypothesis that the morphism $j:U\to X$ is quasi-compact
holds whenever the open subset $U$ is \emph{quasi-compact} and $X$ is a
\emph{scheme}: indeed, $U$ is then a union of finitely-many affine opens $U_i$,
and for every affine open $V$ of $X$, $V\cap U_i$ is an affine open \sref{1.5.5.6}, and
thus quasi-compact.

\begin{cor}[9.4.3]
\label{1.9.4.3}
Let $X$ be a prescheme, $U$ a quasi-compact open subset of $X$ such that the injection
morphism $j:U\to X$ is quasi-compact. Suppose as well that every quasi-coherent
$\OO_X$-module is the inductive limit of its quasi-coherent $\OO_X$-submodules of finite type
\emph{(which will be the case if $X$ is an \emph{affine scheme})}. Then let $\sh{F}$ be a
quasi-coherent $\OO_X$-module, and $\sh{G}$ a quasi-coherent $(\OO_X|U)$-submodule \emph{of
finite type} of $\sh{F}|U$. Then there exists a quasi-coherent $\OO_X$-submodule $\sh{G}'$ of
$\sh{F}$ \emph{of finite type} such that $\sh{G}'|U=\sh{G}$.
\end{cor}

\begin{proof}
\label{proof-1.9.4.3}
We have $\sh{G}=\overline{\sh{G}}|U$, and $\overline{\sh{G}}$ is quasi-coherent, from
\sref{1.9.4.2}, and so the inductive limit of its quasi-coherent $\OO_X$-submodules
$\sh{H}_\lambda$ of finite type. It follows that $\sh{G}$ is the inductive limit of the
$\sh{H}_\lambda|U$, and thus equal to one of the $\sh{H}_\lambda|U$ since it is of finite
type \sref[0]{0.5.2.3}.
\end{proof}

\begin{rmk}[9.4.4]
\label{1.9.4.4}
Suppose that for \emph{every} affine open $U\subset X$, the injection morphism $U\to X$ is
quasi-compact. Then, if the conclusion of \sref{1.9.4.3} holds for every affine open $U$
and every quasi-coherent $(\OO_X|U)$-submodule $\sh{G}$ of $\sh{F}|U$ of finite type, it
follows
\oldpage[I]{175}
that $\sh{F}$ is the inductive limit of its quasi-coherent $\OO_X$-submodules of finite type.
Indeed, for every affine open $U\subset X$, we have that $\sh{F}|U=\wt{M}$, where $M$
is an $A(U)$-module, and since the latter is the inductive limit of its quasi-coherent
submodules of finite type, $\sh{F}|U$ is the inductive limit of its $(\OO_X|U)$-submodules of
finite type \sref{1.1.3.9}. But, by hypothesis, each of these submodules is induced on $U$
by a quasi-coherent $\OO_X$-submodule $\sh{G}_{\lambda,U}$ of $\sh{F}$ of finite type. The
finite sums of the $\sh{G}_{\lambda,U}$ are again quasi-coherent $\OO_X$-modules of finite
type, because the property is local, and the case where $X$ is affine was covered in
\sref{1.1.3.10}; it is clear then that $\sh{F}$ is the inductive limit of these finite
sums, whence our claim.
\end{rmk}

\begin{cor}[9.4.5]
\label{1.9.4.5}
Under the hypotheses of Corollary \sref{1.9.4.3}, for every quasi-coherent $(\OO_X|U)$-module
$\sh{G}$ of finite type, there exists a quasi-coherent $\OO_X$-module $\sh{G}'$ of finite
type such that $\sh{G}'|U=\sh{G}$.
\end{cor}

\begin{proof}
\label{proof-1.9.4.5}
Since $\sh{F}=j_*(\sh{G})$ is quasi-coherent \sref{1.9.4.2} and $\sh{F}|U=\sh{G}$, it
suffices to apply Corollary \sref{1.9.4.3} to $\sh{F}$.
\end{proof}

\begin{lem}[9.4.6]
\label{1.9.4.6}
Let $X$ be a prescheme, $L$ a well-ordered set, $(V_\lambda)_{\lambda\in L}$ a cover of $X$
by affine opens, and $U$ an open subset of $X$; for all $\lambda\in L$, we set
$W_\lambda=\bigcup_{\mu<\lambda}V_\mu$. Suppose that: (1) for every $\lambda\in L$,
$V_\lambda\cap W_\lambda$ is quasi-compact; (2) the immersion morphism $U\to X$ is
quasi-compact. Then, for every quasi-coherent $\OO_X$-module $\sh{F}$ and every
quasi-coherent $(\OO_X|U)$-submodule $\sh{G}$ of $\sh{F}|U$ \emph{of finite type}, there
exists a quasi-coherent $\OO_X$-submodule $\sh{G}'$ of $\sh{F}$ \emph{of finite type} such
that $\sh{G}'|U=\sh{G}$.
\end{lem}

\begin{proof}
\label{proof-1.9.4.6}
Let $U_\lambda=U\cup W_\lambda$; we will define a family $(\sh{G}'_\lambda)$ by induction,
where $\sh{G}'_\lambda$ is a quasi-coherent $(\OO_X|U_\lambda)$-submodule of
$\sh{F}|U_\lambda$ of finite type, such that $\sh{G}'_\lambda|U_\mu=\sh{G}'_\mu$ for
$\mu<\lambda$ and $\sh{G}'_\lambda|U=\sh{G}$. The unique $\OO_X$-submodule $\sh{G}'$ of
$\sh{F}$ such that $\sh{G}'|U_\lambda=\sh{G}'$ for all $\lambda\in L$ \sref[0]{0.3.3.1} gives
us what we want. So suppose that the $\sh{G}'_\mu$ are defined and have the preceding
properties for $\mu<\lambda$; if $\lambda$ does not have a predecessor then we take for
$\sh{G}'_\lambda$ the unique $(\OO_X|U_\lambda)$-submodule of $\sh{F}|U_\lambda$ such that
$\sh{G}'_\lambda|U_\mu=\sh{G}'_\mu$ for all $\mu<\lambda$, which is allowed since the $U_\mu$
with $\mu<\lambda$ then form a cover of $U_\lambda$. If, conversely, $\lambda=\mu+1$, then
$U_\lambda=U_\mu\cup V_\mu$, and it suffices to define a quasi-coherent
$(\OO_X|V_\mu)$-submodule $\sh{G}''_\mu$ of $\sh{F}|V_\mu$ of finite type such that
\[
  \sh{G}''_\mu|(U_\mu\cap V_\mu)=\sh{G}'_\mu|(U_\mu\cap V_\mu);
\]
and then to take for $\sh{G}'_\lambda$ the $(\OO_X|U_\lambda)$-submodule of
$\sh{F}|U_\lambda$ such that $\sh{G}'_\lambda|U_\mu=\sh{G}'_\mu$ and
$\sh{G}'_\lambda|V_\mu=\sh{G}''_\mu$ \sref[0]{0.3.3.1}. But, since $V_\mu$ is affine, the
existence of $\sh{G}''_\mu$ is guaranteed by \sref{1.9.4.3} as soon as we show that
$U_\mu\cap V_\mu$ is quasi-compact; but $U_\mu\cap V_\mu$ is the union of $U\cap V_\mu$ and
$W_\mu\cap V_\mu$, which are both quasi-compact by virtue of the hypothesis.
\end{proof}

\begin{thm}[9.4.7]
\label{1.9.4.7}
Let $X$ be a prescheme, and $U$ an open set of $X$. Suppose that one of the following
conditions is verified:
\begin{enumerate}[label=\emph{(\alph*)}]
  \item the underlying space of $X$ is locally Noetherian;
  \item $X$ is a quasi-compact scheme and $U$ is a quasi-compact open.
\end{enumerate}
Then, for every quasi-coherent $\OO_X$-module $\sh{F}$ and every quasi-coherent
$(\OO_X|U)$-submodule $\sh{G}$ of $\sh{F}|U$ \emph{of finite type}, there exists a
quasi-coherent $\OO_X$-submodule $\sh{G}'$ of $\sh{F}$ \emph{of finite type} such that
$\sh{G}'|U=\sh{G}$.
\end{thm}

\begin{proof}
\label{proof-1.9.4.7}
Let
\oldpage[I]{176}
$(V_\lambda)_{\lambda\in L}$ be a cover of $X$ by affine opens, with $L$ assumed to be finite
in case (b); since $L$ is equipped with the structure of a well-ordered set, it suffices to
check that the conditions of \sref{1.9.4.6} are satisfied. It is clear in the case of (a),
as the spaces $V_\lambda$ are Noetherian. For case (b), the $V_\lambda\cap\lambda_\mu$ are
affine \sref{1.5.5.6}, and thus quasi-compact, and since $L$ is finite,
$V_\lambda\cap W_\lambda$ is quasi-compact. Whence the theorem.
\end{proof}

\begin{cor}[9.4.8]
\label{1.9.4.8}
Under the hypotheses of \sref{1.9.4.7}, for every quasi-coherent $(\OO_X|U)$-module
$\sh{G}$ of finite type, there exists a quasi-coherent $\OO_X$-module $\sh{G}'$ of finite
type such that $\sh{G}'|U=\sh{G}$.
\end{cor}

\begin{proof}
\label{proof-1.9.4.8}
It suffices to apply \sref{1.9.4.7} to $\sh{F}=j_*(\sh{G})$, which is quasi-coherent
\sref{1.9.4.2} and such that $\sh{F}|U=\sh{G}$.
\end{proof}

\begin{cor}[9.4.9]
\label{1.9.4.9}
Let $X$ be a prescheme whose underlying space is locally Noetherian, or a quasi-compact
scheme. Then every quasi-coherent $\OO_X$-module is the inductive limit of its quasi-coherent
$\OO_X$-submodules of finite type.
\end{cor}

\begin{proof}
\label{proof-1.9.4.9}
This follows from Theorem \sref{1.9.4.7} and Remark \sref{1.9.4.4}.
\end{proof}

\begin{cor}[9.4.10]
\label{1.9.4.10}
Under the hypotheses of \sref{1.9.4.9}, if a quasi-coherent $\OO_X$-module $\sh{F}$ is
such that every quasi-coherent $\OO_X$-submodule of finite type of $\sh{F}$ is generated by
its sections over $X$, then $\sh{F}$ is generated by its sections over $X$.
\end{cor}

\begin{proof}
\label{proof-1.9.4.10}
Let $U$ be an affine open neighborhood of a point $x\in X$, and let $s$ be a
section of $\sh{F}$ over $U$; the $\OO_X$-submodule $\sh{G}$ of $\sh{F}|U$ generated by $s$
is quasi-coherent and of finite type, so there exists a quasi-coherent $\OO_X$-submodule
$\sh{G}'$ of $\sh{F}$ of finite type such that $\sh{G}'|U=\sh{G}$ \sref{1.9.4.7}. By
hypothesis, there is thus a finite number of sections $t_i$ of $\sh{G}'$ over $X$ and of
sections $a_i$ of $\OO_X$ over a neighborhood $V\subset U$ of $x$ such that
$s|V=\sum_i a_i(t_i|V)$, which proves the corollary.
\end{proof}

\subsection{Closed image of a prescheme; closure of a subprescheme}
\label{subsection:closed-image-and-closure}

\begin{prop}[9.5.1]
\label{1.9.5.1}
Let $f:X\to Y$ be a morphism of preschemes such that $f_*(\OO_X)$ is a quasi-coherent
$\OO_Y$-module (which will be the case if $f$ is quasi-compact and if in addition $f$ is
either separated or $X$ is locally Noetherian \sref{1.9.2.2}). Then there exists a smaller
subprescheme $Y'$ of $Y$ such that $f$ factors through the canonical injection $j:Y'\to Y$
(\emph{or, equivalently \sref{1.4.4.1}, such that the subprescheme $f^{-1}(Y')$ of $X$ is
\emph{identical} to $X$}).
\end{prop}

More precisely:
\begin{cor}[9.5.2]
\label{1.9.5.2}
Under the conditions of \sref{1.9.5.1}, let $f=(\psi,\theta)$, and let $\sh{J}$ be the
(quasi-coherent) kernel of the homomorphism $\theta:\OO_Y\to f_*(\OO_X)$. Then the closed
subprescheme $Y'$ of $Y$ defined by $\sh{J}$ satisfies the conditions of \sref{1.9.5.1}.
\end{cor}

\begin{proof}
\label{proof-1.9.5.2}
Since the functor $\psi^*$ is exact, the canonical factorization
$\theta:\OO_Y\to\OO_Y/\sh{J}\xrightarrow{\theta'}\psi_*(\OO_X)$ gives (\textbf{0},~3.5.4.3)
a factorization
$\theta^\sharp:\psi^*(\OO_Y)\to\psi^*(\OO_Y)/\psi^*(\sh{J})
  \xrightarrow{{\theta'}^\sharp}\OO_X$; since $\theta_x^\sharp$ is a local homomorphism for
every $x\in X$, the same is true of ${\theta_x'}^\sharp$; if we denote by $\psi_0$ the
continuous map $\psi$ considered as a map from $X$ to $X'$, and by $\theta_0$ the restriction
$\theta'|X':(\OO_Y/\sh{J})|X'\to\psi_*(\OO_X)|X'=(\psi_0)_*(\OO_X)$, then we see that
$f_0=(\psi_0,\theta_0)$ is a morphism of preschemes $X\to X'$ \sref{1.2.2.1} such that
$f=j\circ f_0$. Now, if $X''$ is
\oldpage[I]{177}
a second closed subprescheme of $Y$, defined by a quasi-coherent sheaf of ideals $\sh{J}'$ of
$\OO_Y$, such that $f$ factors through the injection $j':X''\to Y$, then we should
immediately have that $\psi(X)\subset X''$, and so $X'\subset X''$, since $X''$ is closed.
Furthermore, for all $y\in X''$, $\theta$ should factorize as
$\OO_y\to\OO_y/\sh{J}'_y\to(\psi_*(\OO_X))_y$, which by definition leads to
$\sh{J}'_y\subset\sh{J}_y$, and thus $X'$ is a closed subprescheme of $X''$
\sref{1.4.1.10}.
\end{proof}

\begin{defn}[9.5.3]
\label{1.9.5.3}
Whenever there exists a smaller subprescheme $Y'$ of $Y$ such that $f$ factors
through the canonical injection $j:Y'\to Y$, we say that $Y'$ is the
\emph{closed image} prescheme of $X$ under the morphism $f$.
\end{defn}

\begin{prop}[9.5.4]
\label{1.9.5.4}
If $f_*(\OO_X)$ is a quasi-coherent $\OO_Y$-module, then the underlying space of
the closed image of $X$ under $f$ is the closure $\overline{f(X)}$ in $Y$.
\end{prop}

\begin{proof}
\label{proof-1.9.5.4}
As the support of $f_*(\OO_X)$ is contained in $\overline{f(X)}$, we have (with
the notation of \sref{1.9.5.2}) $\sh{J}_y=\OO_y$ for
$y\not\in\overline{f(X)}$, thus the support of $\OO_Y/\sh{J}$ is contained in
$\overline{f(X)}$. In addition, this support is closed and contains $f(X)$:
indeed, if $y\in f(X)$, the unit element of the ring $(\psi_*(\OO_X))_y$ is not
zero, being the germ at $y$ of the section
\[
  1\in\Gamma(X,\OO_X)=\Gamma(Y,\psi_*(\OO_X));
\]
as it is the image under $\theta$ of the unit element of $\OO_y$, the latter
does not belong to $\sh{J}_y$, hence $\OO_y/\sh{J}_y\neq 0$; this finishes the
proof.
\end{proof}

\begin{prop}[9.5.5]
\label{1.9.5.5}
\emph{(Transitivity of closed images)}. Let $f:X\to Y$ and $g:Y\to Z$ be two morphisms of
preschemes; we suppose that the closed image $Y'$ of $X$ under $f$ exists, and that if $g'$
is the restriction of $g$ to $Y'$, then the closed image $Z'$ of $Y'$ under $g'$ exists. Then
the closed image of $X$ under $g\circ f$ exists and is equal to $Z'$.
\end{prop}

\begin{proof}
\label{proof-1.9.5.5}
It suffices \sref{1.9.5.1} to show that $Z'$ is the smallest closed
subprescheme $Z_1$ of $Z$ such that the closed subprescheme $(g\circ f)^{-1}(Z_1)$ of $X$
(equal to $f^{-1}(g^{-1}(Z_1))$ by Corollary \sref{1.4.4.2}) is equal to
$X$; it is equivalent to say that $Z'$ is the smallest closed subprescheme of $Z$ such that
$f$ \unsure{factors} through the injection $g^{-1}(Z_1)\to Y$ \sref{1.4.4.1}. By
virtue of the existence of the closed image $Y'$, every $Z_1$ with this property is such
that $g^{-1}(Z_1)$ \unsure{factors} through $Y'$, which is equivalent to saying that $j^{-1}(g^{-1}(Z_1))=g'^{-1}(Z_1)=Y'$, denoting by $j$ the injection $Y'\to Y$.
By the definition of $Z'$, we indeed conclude that $Z'$ is the smallest closed
subprescheme of $Z$ satisfying the preceding condition.
\end{proof}

\begin{cor}[9.5.6]
\label{1.9.5.6}
Let $f:X\to Y$ be an $S$-morphism such that $Y$ is the closed image of $X$ under $f$.
Let $Z$ be an $S$-scheme; if two $S$-morphisms $g_1$, $g_2$ from $Y$ to $Z$ are such that $g_1\circ f=g_2\circ f$ then $g_1=g_2$.
\end{cor}

\begin{proof}
\label{proof-1.9.5.6}
Let $h=(g_1,g_2)_S:Y\to Z\times_S Z$; since the diagonal $T=\Delta_Z(Z)$ is a closed subprescheme of $Z\times_S Z$, $Y'=h^{-1}(T)$ is a closed subprescheme of $Y$ \sref{1.4.4.1}.
Let $u=g_1\circ f=g_2\circ f$; we then have, by definition of the product, $h'=h\circ f=(u,u)_S$, so $h\circ f=\Delta_Z\circ u$; since $\Delta_Z^{-1}(T)=Z$, we have $h'^{-1}(T)=u^{-1}(Z)=X$, so $f^{-1}(Y')=X$.
From this, we conclude \sref{1.4.4.1} that the canonical injection $Y'\to Y$ \unsure{factors} through $f$, so $Y'=Y$ by hypothesis; it then follows \sref{1.4.4.1} that $h$ factorizes as $\Delta_Z\circ v$, where $v$ is a morphism $Y\to Z$, which implies that $g_1=g_2=v$.
\end{proof}

\begin{rmk}[9.5.7]
\label{1.9.5.7}
If $X$ and $Y$ are $S$-schemes, proposition~\sref{1.9.5.6} implies that, when\oldpage[I]{178} $Y$ if the closed image of $X$ under $f$, $f$ is an \emph{epimorphism} in the category of \emph{$S$-schemes} (T,~1.1).
We will show in Chapter~V that, conversely, if the closed image $Y'$ of $X$ under $f$ exists and if $f$ is an epimorphism of $S$-schemes, then we necessarily have $Y'=Y$.
\end{rmk}

\begin{prop}[9.5.8]
\label{1.9.5.8}
Suppose that the hypotheses of \sref{1.9.5.1} are satisfied, and let $Y'$ be the closed image of $X$ under $f$.
For every open $V$ of $Y$, let $f_V:f^{-1}(V)\to V$ be the restriction of $f$; then the closed image of $f^{-1}(V)$ under $f_V$ in $V$ exists and is equal to the prescheme induced by $Y'$ on the open $V\cap Y'$ of $Y'$ \emph{(in other words, to the subprescheme $\inf(V,Y)$) of $Y$ \sref{1.4.4.3}}.
\end{prop}

\begin{proof}
\label{proof-1.9.5.8}
Let $X'=f^{-1}(V)$; since the direct image of $\OO_{X'}$ by $f_V$ is exactly the restriction of $f_*(\OO_X)$ to $V$, it is clear that the kernel $\sh{J}'$ of the homomorphism $\OO_V\to(f_V)_*(\OO_{X'})$ is the restriction of $\sh{J}$ to $V$, from where the proposition quickly follows.
\end{proof}

We will see that this result can be understood as saying that taking the closed image commutes with an extension $Y_1\to Y$ of the base prescheme, which is an \emph{open immersion}.
We will see in Chapter~IV that it is the same for an extension $Y_1\to Y$ which is a \emph{flat} morphism, provided that $f$ is separated and quasi-compact.

\begin{prop}[9.5.9]
\label{1.9.5.9}
Let $f:X\to Y$ be a morphism such that the closed image $Y'$ of $X$ under $f$ exists.
\begin{enumerate}[label=\emph{(\roman*)}]
  \item If $X$ is reduced, then so is $Y'$.
  \item If the hypotheses of Proposition \sref{1.9.5.1} are satisfied and $X$ is irreducible (resp. \unsure{integral}), then so is $Y'$.
\end{enumerate}
\end{prop}

\begin{proof}
\label{proof-1.9.5.9}
By hypothesis, the morphism $f$ factors as $X\xrightarrow{g}Y'\xrightarrow{j}Y$, where $j$ is the canonical injection.
As $X$ is reduced, $g$ factors as $X\xrightarrow{h} {Y'}_\red\xrightarrow{j'}Y'$, where $j'$ is the canonical injection \sref{1.5.2.2}, and it then follows from the definition of $Y'$ that $Y'_\red=Y'$.
If moreover the conditions of Proposition \sref{1.9.5.1} are satisfied, then it follows from \sref{1.9.5.4} that $f(X)$ is dense in $Y'$; if $X$ is irreducible, then so is $Y'$ \sref[0]{0.2.1.5}.
The claim about integral preschemes follows from the conjunction of the two others.
\end{proof}

\begin{prop}[9.5.10]
\label{1.9.5.10}
Let $Y$ be a subprescheme of a prescheme $X$, such that the canonical injection $i:Y\to X$ is a quasi-compact morphism.
Then there exists a smaller closed subprescheme $\overline{Y}$ of $X$ \unsure{containing} $Y$; its underlying space is the closure of that of $Y$; the latter is open in its closure, and the prescheme $Y$ is induced on this open by $\overline{Y}$.
\end{prop}

\begin{proof}
\label{proof-1.9.5.10}
It suffices to apply Proposition~\sref{1.9.5.1} to the injection $j$, which is separated \sref{1.5.5.1} and quasi-compact by hypothesis; \sref{1.9.5.1} thus proves the existence of $\overline{Y}$ and \sref{1.9.5.4} shows that its underlying space is the closure of $Y$ in $X$; since $Y$ is locally closed in $X$, it is open in $\overline{Y}$, and the last claim comes from \sref{1.9.5.8} applied to an open $V$ of $X$ such that $Y$ is closed in $V$.
\end{proof}

With the above notation, if the injection $V\to X$ is quasi-compact, and if $\sh{J}$ is the quasi-coherent sheaf of ideals of $\OO_X|V$ defining the closed subprescheme $Y$ of $V$, it follows from Proposition~\sref{1.9.5.1} that the quasi-coherent sheaf of ideals of $\OO_X$ defining $\overline{Y}$ is the canonical extension \sref{1.9.4.1} $\overline{\sh{J}}$ of $\sh{J}$, because it is evidently the largest quasi-coherent subsheaf of ideals of $\OO_X$ inducing $\sh{J}$ on $V$.

\begin{cor}[9.5.11]
\label{1.9.5.11}
Under
\oldpage[I]{179}
the hypotheses of Proposition~\sref{1.9.5.10}, every section of $\OO_{\overline{Y}}$ over an open $V$ of $\overline{Y}$ that is zero on $V\cap Y$ is zero.
\end{cor}

\begin{proof}
\label{proof-1.9.5.11}
By Proposition~\sref{1.9.5.8}, we can reduce to the case where $V=\overline{Y}$.
If we take into account that the sections of $\OO_{\overline{Y}}$ over $\overline{Y}$ canonically correspond to the $\overline{Y}$-sections of $\overline{Y}\otimes_Z Z[T]$ \sref{1.3.3.15} and that the latter is separated over $\overline{Y}$, the corollary appears as a specific case of \sref{1.9.5.6}.
\end{proof}

When there exists a smaller closed subprescheme $Y'$ of $X$ majorizing a subprescheme $Y$ of $X$, we say that $Y'$ is the \emph{closure} of $Y$ in $X$, when there is little cause for confusion.

\subsection{Quasi-coherent sheaves of algebras; change of structure sheaf}
\label{subsection:qcoh-algs-and-change-of-str-sheaf}

\begin{prop}[9.6.1]
\label{1.9.6.1}
Let $X$ be a prescheme, $\sh{B}$ a quasi-coherent $\OO_X$-algebra \sref[0]{0.5.1.3}.
For a $\sh{B}$-module $\sh{F}$ to be quasi-coherent (on the ringed space $(X,\sh{B})$) it is necessary and sufficient that $\sh{F}$ be a quasi-coherent $\OO_X$-module.
\end{prop}

\begin{proof}
\label{proof-1.9.6.1}
Since the question is a local, we can assume $X$ to be affine, given by the ring $A$, and thus $\sh{B}=\wt{B}$, where $B$ is an $A$-algebra \sref{1.1.4.3}.
If $\sh{F}$ is quasi-coherent on the ringed space $(X,\sh{B})$ then we can also assume that $\sh{F}$ is the cokernel of a $\sh{B}$-homomorphism $\sh{B}^{(I)}\to\sh{B}^{(J)}$; since this homomorphism is also an $\OO_X$-homomorphism of $\OO_X$-modules, and $\sh{B}^{(I)}$ and $\sh{B}^{(J)}$ are quasi-coherent $\OO_X$-modules \sref{1.1.3.9}[ii], $\sh{F}$ is also a quasi-coherent $\OO_X$-module \sref{1.1.3.9}[i].

Conversely, if $\sh{F}$ is a quasi-coherent $\OO_X$-module, then $\sh{F}=\wt{M}$, where $M$ is a $B$-module \sref{1.1.4.3}; $M$ is isomorphic to the cokernel of a $B$-homomorphism $B^{(I)}\to B^{(J)}$, so $\sh{F}$ is a $\sh{B}$-module isomorphic to the cokernel of the corresponding homomorphism $\sh{B}^{(I)}\to\sh{B}^{(J)}$ \sref{1.1.3.13}, which finishes the proof.
\end{proof}

In particular, if $\sh{F}$ and $\sh{G}$ are two quasi-coherent $\sh{B}$-modules, $\sh{F}\otimes_{\sh{B}}\sh{G}$ is a quasi-coherent $\sh{B}$-module; similarly for $\shHom(\sh{F},\sh{G})$ whenever we further suppose that $\sh{F}$ admits a finite presentation \sref{1.1.3.13}.

\begin{env}[9.6.2]
\label{1.9.6.2}
Given a prescheme $X$, we say that a quasi-coherent $\OO_X$-algebra $\sh{B}$ is of \emph{finite type} if, for all $x\in X$, there exists an open \emph{affine} neighborhood $U$ of $x$ such that $\Gamma(U,\sh{B})=B$ is an algebra of finite type over $\Gamma(U,\OO_X)=A$.
We then have that $\sh{B}|U=\wt{B}$ and, for all $f\in A$, the induced $(\OO_X|D(f))$-algebra $\sh{B}|D(f)$ is of finite type, because it is isomorphic to $(B_f)^{\sim}$, and $B_f=B\otimes_A A_f$ is clearly an algebra of finite type over $A_f$.
Since the $D(f)$ form a basis for the topology of $U$, we thus conclude that if $\sh{B}$ is a quasi-coherent $\OO_X$-algebra of finite type then, for every open $V$ of $X$, $\sh{B}|V$ is a quasi-coherent $(\OO_X|V)$-algebra of finite type.
\end{env}

\begin{prop}[9.6.3]
\label{1.9.6.3}
Let $X$ be a locally Noetherian prescheme.
Then every quasi-coherent $\OO_X$-algebra $\sh{B}$ of finite type is a coherent sheaf of rings \sref[0]{0.5.3.7}.
\end{prop}

\begin{proof}
\label{proof-1.9.6.3}
We can once again restrict to the case where $X$ is an affine scheme given by a Noetherian ring $A$, and where $\sh{B}=\wt{B}$, $B$ being an $A$-algebra of finite type; $B$ is then a Noetherian ring.
With this, it remains to prove that the kernel $\sh{N}$ of a $\sh{B}$-homomorphism $\sh{B}^m\to\sh{B}$ is a $\sh{B}$-module\oldpage[I]{180} of finite type; but it is isomorphic (as a $\sh{B}$-module) to $\wt{N}$, where $N$ is the kernel of the corresponding homomorphism of $B$-modules $B^m\to B$ \sref{1.1.3.13}.
Since $B$ is Noetherian, the submodule $N$ of $B^m$ is a $B$-module of finite type, so there exists a homomorphism $B^p\to B^m$ with image $N$; since the sequence $B^p\to B^m\to B$ is exact, so is the corresponding sequence $\sh{B}^p\to\sh{B}^m\to\sh{B}$ \sref{1.1.3.5} and since $\sh{N}$ is the image of $\sh{B}^p\to\sh{B}^m$ \sref{1.1.3.9}[i], the proposition is proved.
\end{proof}

\begin{cor}[9.6.4]
\label{1.9.6.4}
Under the hypotheses of \sref{1.9.6.3}, for a $\sh{B}$-module $\sh{F}$ to be coherent, it is necessary and sufficient that it be a quasi-coherent $\OO_X$-module and a $\sh{B}$-module of finite type.
If this is the case, and if $\sh{G}$ is a $\sh{B}$-submodule or a quotient module of $\sh{F}$, then in order for $\sh{G}$ to be a coherent $\sh{B}$-module, it is necessary and sufficient that it is a quasi-coherent $\OO_X$-module.
\end{cor}

\begin{proof}
\label{proof-1.9.6.4}
Taking into account \sref{1.9.6.1}, the conditions on $\sh{F}$ are clearly necessary; we will show that they are sufficient.
We can restrict to the case where $X$ is affine given by a Noetherian ring $A$, $\sh{B}=\wt{B}$, where $B$ is an $A$-algebra of finite type, $\sh{F}=\wt{M}$, where $M$ is a $B$-module, and where there exists a surjective $\sh{B}$-homomorphism $\sh{B}^m\to\sh{F}\to0$.
We then have the corresponding exact sequence $B^m\to M\to0$, so $M$ is a $B$-module of finite type; further, the kernel $P$ of the homomorphism $B^m\to M$ is then a $B$-module of finite type, since $B$ is Noetherian.
We thus conclude \sref{1.1.3.13} that $\sh{F}$ is the cokernel of a $\sh{B}$-homomorphism $\sh{B}^m\to\sh{B}^n$, and is thus coherent, since $\sh{B}$ is a coherent sheaf of rings \sref[0]{0.5.3.4}.
The same reasoning shows that a quasi-coherent $\sh{B}$-submodule (resp. a quotient $\sh{B}$-module) of $\sh{F}$ is of finite type, from whence the second part of the corollary.
\end{proof}

\begin{prop}[9.6.5]
\label{1.9.6.5}
Let $X$ be a quasi-compact scheme or a prescheme whose underlying space is Noetherian.
For all quasi-compact $\OO_X$-algebras $\sh{B}$ of finite type, there exists a quasi-coherent $\OO_X$-submodule $\sh{E}$ of $\sh{B}$ of finite type such that $\sh{E}$ generates \sref[0]{0.4.1.3} the $\OO_X$-algebra $\sh{B}$.
\end{prop}

\begin{proof}
\label{proof-1.9.6.5}
In fact, by the hypothesis there exists a finite cover $(U_i)$ of $X$ consisting of affine opens such that $\Gamma(U_i,\sh{B})=B_i$ is an algebra of finite type over $\Gamma(U_i,\OO_X)=A_i$; let $E_i$ be a $A_i$-submodule of $B_i$ of finite type that generates the $A_i$-algebra $B_i$; thanks to \sref{1.9.4.7}, there exists a $\OO_X$-submodule $\sh{E}_i$ of $\sh{B}$, quasi-coherent and of finite type, such that $\sh{E}_i|U_i=\wt{E_i}$.
It is clear that the sum $\sh{E}$ of the $\sh{E}_i$ is the desired object.
\end{proof}

\begin{prop}[9.6.6]
\label{1.9.6.6}
Let $X$ be a prescheme whose underlying space is locally Noetherian, or a quasi-compact scheme.
Then every quasi-coherent $\OO_X$-algebra $\sh{B}$ is the inductive limit of its quasi-coherent $\OO_X$-subalgebras of finite type.
\end{prop}

\begin{proof}
\label{proof-1.9.6.6}
In fact, it follows from \sref{1.9.4.9} that $\sh{B}$ is the inductive limit (as an $\OO_X$-module) of its quasi-coherent $\OO_X$-submodules of finite type; the latter generating quasi-coherent \emph{$\OO_X$-subalgebras} of $\sh{B}$ of finite type \sref{1.1.3.14}, and so $\sh{B}$ is \emph{a fortiori} their inductive limit.
\end{proof}

\section{Formal schemes}
\label{section:1.10}

\subsection{Formal affine schemes}
\label{subsection:1.10.1}

\begin{env}[10.1.1]
\label{1.10.1.1}
Let $A$ be an \emph{admissible} topological ring \sref[0]{0.7.1.2}; for each ideal of definition $\mathfrak{J}$ of $A$, $\Spec(A/\mathfrak{J})$ identifies with the closed subspace $V(\mathfrak{J})$ of $\Spec(A)$ \sref{1.1.1.11}, the set of \emph{open} prime ideals of $A$; this topological space does not depend
\oldpage[I]{181}
on the ideal of definition $\mathfrak{J}$ considered; we denote this topological space by $\mathfrak{X}$. Let $(\mathfrak{J}_\lambda)$ be a fundamental system of neighborhoods of $0$ in $A$, consisting of ideals of definition, and for each $\lambda$, let $\OO_\lambda$ be the structure sheaf of $\Spec(A/\mathfrak{J}_\lambda)$; this sheaf is induced on $\mathfrak{X}$ by $\wt{A}/\wt{\mathfrak{J}_\lambda}$ (which is zero outside of $\mathfrak{X}$).
For $\mathfrak{J}_\mu\subset\mathfrak{J}_\lambda$, the canonical homomorphism $A/\mathfrak{J}_\mu\to A/\mathfrak{J}_\lambda$ thus defines a homomorphism $u_{\lambda\mu}:\OO_\mu\to\OO_\lambda$ of sheaves of rings \sref{1.1.6.1}, and $(\OO_\lambda)$ is a \emph{projective system of sheaves of rings} for these homomorphisms.
As the topology of $\mathfrak{X}$ admits a basis consisting of quasi-compact open subsets, we can associate to each $\OO_\lambda$ a \emph{pseudo-discrete sheaf of topological rings} \sref[0]{0.3.8.1} which have $\OO_\lambda$ as the underlying (without topology) sheaf of rings, and that we denote also by $\OO_\lambda$; and the $\OO_\lambda$ give again a \emph{projective system of sheaves of topological rings} \sref[0]{0.3.8.2}.
We denote by $\OO_\mathfrak{X}$ the \emph{sheaf of topological rings} on $\mathfrak{X}$, the projective limit of the system $(\OO_\lambda)$; for each \emph{quasi-compact} open subset $U$ of $\mathfrak{X}$, $\Gamma(U,\OO_\mathfrak{X})$ is a topological ring, the projective limit of the system of \emph{discrete} rings $\Gamma(U,\OO_\lambda)$ \sref[0]{0.3.2.6}.
\end{env}

\begin{defn}[10.1.2]
\label{1.10.1.2}
Given an admissible topological ring $A$, we define the formal spectrum of $A$, and denote it by $\Spf(A)$, to be the closed subspace $\mathfrak{X}$ of $\Spec(A)$ consisting of the open prime ideals of $A$.
We say that a topologically ringed space is a formal affine scheme if it is isomorphic to a formal spectrum $\Spf(A)=\mathfrak{X}$ equipped with a sheaf of topological rings $\OO_\mathfrak{X}$ which is the projective limit of sheaves of psuedo-discrete topological rings $(\wt{A}/\wt{\mathfrak{J}_\lambda})|\mathfrak{X}$, where $\mathfrak{J}_\lambda$ varies over the filtered set of ideals of definition for $A$.
\end{defn}

When we speak of a \emph{formal spectrum $\mathfrak{X}=\Spf(A)$} as a formal affine scheme, it will always be as the topologically ringed space $(\mathfrak{X},\OO_\mathfrak{X})$ where $\OO_\mathfrak{X}$ is defined as above.

We note that every \emph{affine scheme} $X=\Spec(A)$ can be considered as a formal affine scheme in only one way, by considering $A$ as a discrete topological ring: the topological rings $\Gamma(U,\OO_X)$ are then discrete whenever $U$ is quasi-compact (but not, in general, when $U$ is an arbitrary open subset of $X$).

\begin{prop}[10.1.3]
\label{1.10.1.3}
If $\mathfrak{X}=\Spf(A)$, where $A$ is an admissible ring, then $\Gamma(\mathfrak{X},\OO_X)$ is topologically isomorphic to $A$.
\end{prop}

\begin{proof}
\label{proof-1.10.1.3}
Indeed, since $\mathfrak{X}$ is closed in $\Spec(A)$, it is quasi-compact, and so $\Gamma(\mathfrak{X},\OO_\mathfrak{X})$ is topologically isomorphic to the projective limit of the discrete rings $\Gamma(\mathfrak{X},\OO_\lambda)$; but $\Gamma(\mathfrak{X},\OO_\lambda)$ is isomorphic to $A/\mathfrak{J}_\lambda$ \sref{1.1.3.7}; since $A$ is separated and complete, it is topologically isomorphic to $\varprojlim A/\mathfrak{J}_\lambda$ \sref[0]{0.7.2.1}, whence the proposition.
\end{proof}

\begin{prop}[10.1.4]
\label{1.10.1.4}
Let $A$ be an admissible ring, $\mathfrak{X}=\Spf(A)$, and, for every $f\in A$, let $\mathfrak{D}(f)=D(f)\cap\mathfrak{X}$; then the topologically ringed space $(\mathfrak{D}(f),\OO_\mathfrak{X}|\mathfrak{D}(f))$ is isomorphic to the formal affine spectrum $\Spf(A_{\{f\}}$ \sref[0]{0.7.6.15}.
\end{prop}

\begin{proof}
\label{proof-1.10.1.4}
For every ideal of definition $\mathfrak{J}$ of $A$, the discrete ring $S_f^{-1}A/ S_f^{-1}\mathfrak{J}$ is canonically identified with $A_{\{f\}}/\mathfrak{J}_{\{f\}}$ \sref[0]{0.7.6.9}, so, by \sref{1.1.2.5} and \sref{1.1.2.6}, the topological space $\Spf(A_{\{f\}})$ is canonically identified with $\mathfrak{D}(f)$.
Further, for every quasi-compact open subset $U$ of $\mathfrak{X}$ contained in $\mathfrak{D}(f)$, $\Gamma(U,\OO_\lambda)$ can be identified with the module of sections of the structure sheaf of $\Spec(S_f^{-1}A/ S_f^{-1}\mathfrak{J}_\lambda)$ over $U$ \sref{1.1.3.6}, so, setting $\mathfrak{Y}=\Spf(A_{\{f\}})$, $\Gamma(U,\OO_\mathfrak{X})$ can be identified with the module of sections $\Gamma(U,\OO_\mathfrak{Y})$, which proves the proposition.
\end{proof}

\begin{env}[10.1.5]
\label{1.10.1.5}
As a sheaf of rings \emph{without topology}, the structure sheaf $\OO_\mathfrak{X}$ of $\Spf(A)$ admits, for every $x\in\mathfrak{X}$, a fibre which, by \sref{1.10.1.4}, can be identified with the inductive limit $\varinjlim A_{\{f\}}$ for the $f\not\in\mathfrak{j}_x$.
Then, by \sref[0]{0.7.6.17} and \sref[0]{0.7.6.18}:
\end{env}

\begin{prop}[10.1.6]
\label{1.10.1.6}
For every $x\in\mathfrak{X}=\Spf(A)$, the fibre $\OO_x$ is a local ring whose residue field is isomorphic to $\kres(x)=A_x/\mathfrak{j}_xA_x$.
If, further, $A$ is adic and Noetherian, then $\OO_x$ is a Noetherian ring.
\end{prop}

Since $\kres(x)$ is not reduced at $0$, we conclude from this result that the \emph{support} of the ring of sheaves $\OO_\mathfrak{X}$ is \emph{equal to $\mathfrak{X}$}.

\subsection{Morphisms of formal affine schemes}
\label{subsection:1.10.2}

\begin{env}[10.2.1]
\label{1.10.2.1}
Let $A$, $B$ be two admissible rings, and let $\vphi:B\to A$ be a \emph{continuous} morphism.
The continuous map ${}^a\vphi:\Spec(A)\to\Spec(B)$ \sref{1.1.2.1} then maps $\mathfrak{X}=\Spf(A)$ to $\mathfrak{Y}=\Spf(B)$, since the inverse image under $\vphi$ of an open prime ideal of $A$ is an open prime ideal of $B$.
On the other hand, for all $g\in B$, $\vphi$ defines a continuous homomorphism $\Gamma(\mathfrak{D}(g),\OO_\mathfrak{Y})\to\Gamma(\mathfrak{D}(\vphi(g)),\OO_\mathfrak{X})$ according to \sref{1.10.1.4}, \sref{1.10.1.3}, and \sref[0]{0.7.6.7}; as these homomorphisms satisfy the compatibility conditions for the restrictions corresponding to the change from $g$ to a multiple of $g$, and as $\mathfrak{D}(\vphi(g))={}^a\vphi^{-1}(\mathfrak{D}(g))$, they define a \emph{continuous} homomorphisms of sheaves of topological rings $\OO_\mathfrak{Y}\to{}^a\vphi_*(\OO_\mathfrak{X})$ \sref[0]{0.3.2.5}, that we denote by $\wt{\vphi}$; we have thus defined a morphism $\Phi=({}^a\vphi,\wt{\vphi})$ of topologically ringed spaces $\mathfrak{X}\to\mathfrak{Y}$.
We note that as a homomorphism of sheaves without topology, $\wt{\vphi}$ defines a homomorphism $\wt{\vphi}_x^\sharp:\OO_{{}^a\vphi(x)}\to\OO_x$ on the stalks, for all $x\in\mathfrak{X}$.
\end{env}

\begin{prop}[10.2.2]
\label{1.10.2.2}
Let $A$, $B$ be two admissible topological rings, and let $\mathfrak{X}=\Spf(A)$, $\mathfrak{Y}=\Spf(B)$.
For a morphism $u=(\psi,\theta):\mathfrak{X}\to\mathfrak{Y}$ of topologically ringed spaces to be of the form $({}^a\vphi,\wt{\vphi})$, where $\vphi$ is a continuous ring homomorphism $B\to A$, it is necessary and sufficient that for all $x\in\mathfrak{X}$, $\theta_x^\sharp$ is a local homomorphism $\OO_{\vphi(x)}\to\OO_x$.
\end{prop}

\begin{proof}
\label{proof-1.10.2.2}
The condition is necessary: let $\mathfrak{p}=\mathfrak{j}_x\in\Spf(A)$, and let $\mathfrak{q}=\vphi^{-1}(\mathfrak{j}_x)$; if $g\not\in\mathfrak{q}$, then we have $\vphi(g)\not\in\mathfrak{p}$, and it is immediate that the homomorphism
$B_{\{g\}}\to A_{\{\vphi(g)\}}$ induced by $\vphi$ \sref[0]{0.7.6.7} sends $\mathfrak{q}_{\{g\}}$ to a subset of $\mathfrak{p}_{\{\vphi(g)\}}$; by passing to the inductive limit, we see (taking into account \sref{1.10.1.5} and \sref[0]{0.7.6.17}) that $\wt{\vphi}_x^\sharp$ is a local homomorphism.

Conversely, let $(\psi,\theta)$ be a morphism satisying the condition in the statement; according to \sref{1.10.1.3}, $\theta$ defines a continuous ring homomorphism
\[
  \vphi=\Gamma(\theta):B=\Gamma(\mathfrak{Y},\OO_\mathfrak{Y})\to\Gamma(\mathfrak{X},\OO_\mathfrak{X})=A.
\]
By virtue of the hypothesis on $\theta$, for the section $\vphi(g)$ of $\OO_\mathfrak{X}$ over $\mathfrak{X}$ to be an invertible germ at the point $x$, it is necessary and sufficient that $g$ is an invertible germ at the point $\psi(x)$.
But according to \sref[0]{0.7.6.17}, the sections of $\OO_\mathfrak{X}$ (resp. $\OO_\mathfrak{Y}$) over $\mathfrak{X}$ (resp. $\OO_\mathfrak{Y}$) whose germ is not invertible at the point $x$ (resp. $\psi(x)$) are exactly the elements of
$\mathfrak{j}_x$
\oldpage[I]{183}
(resp. $\mathfrak{j}_{\psi(x)}$); the above remark thus shows that ${}^a\vphi=\psi$.
Finally, for all $g\in B$ the diagram
\[
  \xymatrix{
    B=\Gamma(\mathfrak{Y},\OO_\mathfrak{Y})\ar[r]^\vphi\ar[d] &
    \Gamma(\mathfrak{X},\OO_\mathfrak{X})=A\ar[d]\\
    B_{\{g\}}=\Gamma(\mathfrak{D}(g),\OO_\mathfrak{Y})\ar[r]^{\Gamma(\theta_{\mathfrak{D}(g)})} &
    \Gamma(\mathfrak{D}(\vphi(g)),\OO_\mathfrak{X})=A_{\{\vphi(g)\}}
  }
\]
is commutative; by the universal property of completed rings of fractions \sref[0]{0.7.6.6}, $\theta_{\mathfrak{D}(g)}$ is equal to $\wt{\vphi}_{\mathfrak{D}(g)}$ for all $g\in B$, so \sref[0]{0.3.2.5} we have $\theta=\wt{\vphi}$.
\end{proof}

We say that a morphism $(\psi,\theta)$ of topologically ringed spaces satisfying the condition of Proposition \sref{1.10.2.2} is a \emph{morphism of formal affine schemes}.
We can say that the functors $\Spf(A)$ in $A$ and $\Gamma(\mathfrak{X},\OO_\mathfrak{X})$ in $\mathfrak{X}$ define an \emph{equivalence} between the cateory of admissible rings and the opposite category of formal affine schemes (T, I, 1.2).

\begin{env}[10.2.3]
\label{1.10.2.3}
As a particular case of \sref{1.10.2.2}, note that for $f\in A$, the canonical injection of the formal affine scheme induced by $\mathfrak{X}$ on $\mathfrak{D}(f)$ corresponds to the continuous canonical homomorphism $A\to A_{\{f\}}$.
Under the hypotheses of Proposition \sref{1.10.2.2}, let $h$ be an element of $B$, $g$ an element of $A$, multiple of $\vphi(h)$; we then have $\psi(\mathfrak{D}(g))\subset\mathfrak{D}(h)$; the restriction of $u$ to $\mathfrak{D}(g)$, considered as a morphism from $\mathfrak{D}(g)$ to $\mathfrak{D}(h)$, is the unique morphism $v$ making the diagram
\[
  \xymatrix{
    \mathfrak{D}(g)\ar[r]^v\ar[d] &
    \mathfrak{D}(h)\ar[d]\\
    \mathfrak{X}\ar[r]^u &
    \mathfrak{Y}
  }
\]
commutative.

This morphism corresponds to the unique continuous homomorphism $\vphi':B_{\{h\}}\to A_{\{g\}}$ \sref[0]{0.7.6.7} making the diagram
\[
  \xymatrix{
    A\ar[d] &
    B\ar[l]_\vphi\ar[d]\\
    A_{\{g\}} &
    B_{\{h\}}\ar[l]_{\vphi'}
  }
\]
commutative.
\end{env}

\subsection{Ideals of definition for a formal affine scheme}
\label{subsection:1.10.3}

\begin{env}[10.3.1]
\label{1.10.3.1}
Let $A$ be an admissible ring, $\mathfrak{J}$ an open ideal of $A$, $\mathfrak{X}$ the formal affine scheme $\Spf(A)$.
Let $(\mathfrak{J}_\lambda)$ be the set of the ideals of definition for $A$ contained in $\mathfrak{J}$; then $\wt{\mathfrak{J}}/\wt{\mathfrak{J}}_\lambda$ is a sheaf of ideals of $\wt{A}/\wt{\mathfrak{J}}_\lambda$.
Denote by $\mathfrak{J}^\Delta$ the projective limit of the induced sheaves on $\mathfrak{X}$ by $\wt{\mathfrak{J}}/\wt{\mathfrak{J}}_\lambda$, which identifies with a \emph{sheaf of ideals} of $\OO_\mathfrak{X}$ \sref[0]{0.3.2.6}.
For every $f\in A$, $\Gamma(\mathfrak{D}(f),\mathfrak{J}^\Delta)$ is the projective limit of the $S_f^{-1}\mathfrak{J}/S_f^{-1}\mathfrak{J}_\lambda$, in other words, it identifies with the open ideal $\mathfrak{J}_{\{f\}}$ of the ring $A_{\{f\}}$ \sref[0]{0.7.6.9}, and in particular $\Gamma(\mathfrak{X},\mathfrak{J}^\Delta)=\mathfrak{J}$; we conclude (the $\mathfrak{D}(f)$ forming a basis for the topology of $\mathfrak{X}$) that we have
\[
  \mathfrak{J}^\Delta|\mathfrak{D}(f)=(\mathfrak{J}_{\{f\}})^\Delta.
  \tag{10.3.1.1}
\]
\end{env}

\begin{env}[10.3.2]
\label{1.10.3.2}
\oldpage[I]{184}
With the notation of \sref{1.10.3.1}, for all $f\in A$, the canonical map from $A_{\{f\}}=\Gamma(\mathfrak{D}(f),\OO_\mathfrak{X})$ to $\Gamma(\mathfrak{D}(f),(\wt{A}/\wt{\mathfrak{J}})|\mathfrak{X})=S_f^{-1}A/S_f^{-1}\mathfrak{J}$ is \emph{surjective} and has for its kernel $\Gamma(\mathfrak{D}(f),\mathfrak{J}^\Delta)=\mathfrak{J}_{\{f\}}$ \sref[0]{0.7.6.9}; these maps thus define a \emph{surjective} continuous homomorphism, said to be \emph{canonical}, from the sheaf of topological rings $\OO_\mathfrak{X}$ to the sheaf of discrete rings $(\wt{A}/\wt{\mathfrak{J}})|\mathfrak{X}$, whose kernel is $\mathfrak{J}^\Delta$; this homomorphism is none other than $\wt{\vphi}$ \sref{1.10.2.1}, where $\vphi$ is the continuous homomorphism $A\to A/\mathfrak{J}$; the morphism $({}^a\vphi,\wt{\vphi}):\Spec(A/\mathfrak{J})\to\mathfrak{X}$ of formal affine schemes (where ${}^a\vphi$ is the identity homeomorphism from $\mathfrak{X}$ to itself) is also called \emph{canonical}.
We thus have, according to the above, a \emph{canonical isomorphism}
\[
  \OO_\mathfrak{X}/\mathfrak{J}^\Delta\isoto(\wt{A}/\wt{\mathfrak{J}})|\mathfrak{X}.
  \tag{10.3.2.1}
\]

It is clear (since $\Gamma(\mathfrak{X},\mathfrak{J}^\Delta)=\mathfrak{J}$) that the map $\mathfrak{J}\to\mathfrak{J}^\Delta$ is \emph{strictly increasing}; according to the above, for $\mathfrak{J}\subset\mathfrak{J}'$, the sheaf ${\mathfrak{J}'}^\Delta/\mathfrak{J}^\Delta$ is canonically isomorphic to $\wt{\mathfrak{J}'}/\wt{\mathfrak{J}}=(\mathfrak{J}'/\mathfrak{J})^\sim$.
\end{env}

\begin{env}[10.3.3]
\label{1.10.3.3}
The hypotheses and notation being the same as those of \sref{1.10.3.1}, we say that a sheaf of ideals $\sh{J}$ of $\OO_\mathfrak{X}$ is a \emph{sheaf of ideals of definition} for $\mathfrak{X}$ (or an \emph{ideal sheaf of definition} for $\mathfrak{X}$) if, for all $x\in\mathfrak{X}$, there exists an open neighborhood of $x$ of the form $\mathfrak{D}(f)$, where $f\in A$, such that $\sh{J}|\mathfrak{D}(f)$ is of the form $\mathfrak{H}^\Delta$, where $\mathfrak{H}$ is an ideal of definition for $A_{\{f\}}$.
\end{env}

\begin{prop}[10.3.4]
\label{1.10.3.4}
For all $f\in A$, each sheaf of ideals of definition for $\mathfrak{X}$ induces a sheaf of ideals of definition for $\mathfrak{D}(f)$.
\end{prop}

\begin{proof}
\label{proof-1.10.3.4}
This follows from (10.3.1.1).
\end{proof}

\begin{prop}[10.3.5]
\label{1.10.3.5}
If $A$ is an admissible ring, then every sheaf of ideals of definition for $\mathfrak{X}=\Spf(A)$ is of the form $\mathfrak{J}^\Delta$, where $\mathfrak{J}$ is an ideal of definition for $A$, uniquely determined.
\end{prop}

\begin{proof}
\label{proof-1.10.3.5}
Let $\sh{J}$ be a sheaf of ideals of definition of $\mathfrak{X}$; by hypothesis, and since $\mathfrak{X}$ is quasi-compact, there is a finite number of elements $f_i\in A$ such that the $\mathfrak{D}(f_i)$ cover $\mathfrak{X}$ and that $\sh{J}|\mathfrak{D}(f_i)=\mathfrak{H}_i^\Delta$, where $\mathfrak{H}_i$ is an ideal of definition for $A_{\{f_i\}}$.
For each $i$, there exists an open ideal $\mathfrak{K}_i$ of $A$ such that $(\mathfrak{K}_i)_{\{f_i\}}=\mathfrak{H}_i$ \sref[0]{0.7.6.9}; let $\mathfrak{K}$ be an ideal of definition for $A$ containing all the $\mathfrak{K}_i$.
The canonical image of $\sh{J}/\mathfrak{K}^\Delta$ in the structure sheaf $(A/\mathfrak{K})^\sim$ of $\Spec(A/\mathfrak{K})$ \sref{1.10.3.2} is thus such that its restriction to $\mathfrak{D}(f_i)$ is equal to its restriction to $(\mathfrak{K}_i/\mathfrak{K})^\sim$;
we conclude that this canonical image is a \emph{quasi-coherent} sheaf on $\Spec(A/\mathfrak{K})$, so it is of the form $(\mathfrak{J}/\mathfrak{K})^\sim$, where $\mathfrak{J}$ is an ideal of definition for $A$ containing $\mathfrak{K}$ \sref{1.1.4.1} hence $\sh{J}=\mathfrak{J}^\Delta$ \sref{1.10.3.2};
in addition, as for each $i$ there exists an integer $n_i$ such that $\mathfrak{H}_i^{n_i}\subset\mathfrak{K}_{\{f_i\}}$, we will have, by setting $n$ to be the largest of the $n_i$, $(\sh{J}/\mathfrak{K}^\Delta)^n=0$, and as a result \sref{1.10.3.2} $((\mathfrak{J}/\mathfrak{K})^\sim)^n=0$, so finally $(\mathfrak{J}/\mathfrak{K})^n=0$ \sref{1.1.3.13}, which prove that $\mathfrak{J}$ is an ideal of definition for $A$ \sref[0]{0.7.1.4}.
\end{proof}

\begin{prop}[10.3.6]
\label{1.10.3.6}
Let $A$ be an adic ring, $\mathfrak{J}$ an ideal of definition for $A$ such that $\mathfrak{J}/\mathfrak{J}^2$ is an $(A/\mathfrak{J})$-module of finite type. For any integer $n>0$, we then have $(\mathfrak{J}^\Delta)^n=(\mathfrak{J}^n)^\Delta$.
\end{prop}

\begin{proof}
\label{proof-1.10.3.6}
For all $f\in A$, we have (since $\mathfrak{J}^n$ is an open ideal)
\[
  (\Gamma(\mathfrak{D}(f),\mathfrak{J}^\Delta))^n=(\mathfrak{J}_{\{f\}})^n=(\mathfrak{J}^n)_{\{f\}}=\Gamma(\mathfrak{D}(f^n),(\mathfrak{J}^n)^\Delta)
\]
\oldpage[I]{185}
according to (10.3.1.1) and \sref[0]{0.7.6.12}.
As $(\mathfrak{J}^\Delta)^n$ is associated to the presheaf $U\mapsto(\Gamma(U,\mathfrak{J}^\Delta))^n$ \sref[0]{0.4.1.6}, the result follows, since the $\mathfrak{D}(f)$ form a basis for the topology of $\mathfrak{X}$.
\end{proof}

\begin{env}[10.3.7]
\label{1.10.3.7}
We say that a family $(\sh{J}_\lambda)$ of sheaves of ideals of definition for $\mathfrak{X}$ is a \emph{fundamental system of sheaves of ideals of definition} if each sheaf of ideals of definition for $\mathfrak{X}$ contains one of the $\sh{J}_\lambda$; as $\sh{J}_\lambda=\mathfrak{J}_\lambda^\Delta$, it is equivalent to say that the $\mathfrak{J}_\lambda$ for a \emph{fundamental system of neighborhoods of $0$} in $A$.
Let $(f_\alpha)$ be a family of elements of $A$ such that the $\mathfrak{D}(f_\alpha)$ cover $\mathfrak{X}$.
If $(\sh{J}_\lambda)$ is a filtered decreasing family of sheaves of ideals of $\OO_\mathfrak{X}$ such that for each $\alpha$, the family $(\sh{J}_\lambda|\mathfrak{D}(f_\alpha))$ is a fundamental system of sheaves of ideals of definition for $\mathfrak{D}(f_\alpha)$, then $(\sh{J}_\lambda)$ is a fundamental system of sheaves of ideals of definition for $\mathfrak{X}$.
Indeed, for each sheaf of ideals of definition $\sh{J}$ for $\mathfrak{X}$, there is a finite cover of $\mathfrak{X}$ by $\mathfrak{D}(f_i)$ such that, for each $i$, $\sh{J}_{\lambda_i}|\mathfrak{D}(f_i)$ is a sheaf of ideals of definition for $\mathfrak{D}(f_i)$ contained in $\sh{J}|\mathfrak{D}(f_i)$.
If $\mu$ is an index such that $\sh{J}_\mu\subset\sh{J}_{\lambda_i}$ for all $i$, then it follows from \sref{1.10.3.3} that $\sh{J}_\mu$ is a sheaf of ideals of definition for $\mathfrak{X}$, evidently contained in $\sh{J}$, hence our assertion.
\end{env}

\subsection{Formal preschemes and morphisms of formal preschemes}
\label{subsection:1.10.4}

\begin{env}[10.4.1]
\label{1.10.4.1}
Given a topologically ringed space $\mathfrak{X}$, we say that an open $U\subset\mathfrak{X}$ is an \emph{formal affine open} (resp.~an \emph{formal adic affine open}, resp.~an \emph{formal Noetherian affine open}) if the topologically ringed space induced on $U$ by $\mathfrak{X}$ is a formal affine scheme (resp.~a scheme whose ring is adic, resp.~adic and Noetherian).
\end{env}

\begin{defn}[10.4.2]
\label{1.10.4.2}
A \emph{formal prescheme} is a topologically ringed space $\mathfrak{X}$ which admits a formal affine open neighborhood for each point.
We say that the formal prescheme $\mathfrak{X}$ is adic (resp. locally Noetherian) if each point of $\mathfrak{X}$ admits a formal adic (resp. Noetherian) open neighborhood.
We say that $\mathfrak{X}$ is Noetherian if it is locally Noetherian and if its underlying space is quasi-compact (hence Noetherian).
\end{defn}

\begin{prop}[10.4.3]
\label{1.10.4.3}
If $\mathfrak{X}$ is a formal prescheme (resp. a locally Noetherian formal prescheme), then the formal affine (resp. Noetherian affine) open sets form a basis for the topology of $\mathfrak{X}$.
\end{prop}

\begin{proof}
\label{proof-1.10.4.3}
This follows from Definition \sref{1.10.4.2} and Proposition \sref{1.10.1.4} by taking into account that if $A$ is an adic Noetherian ring, then so if $A_{\{f\}}$ for all $f\in A$ \sref[0]{0.7.6.11}.
\end{proof}

\begin{cor}[10.4.4]
\label{1.10.4.4}
If $\mathfrak{X}$ is a formal prescheme (resp. a locally Noetherian formal prescheme, resp. a Noetherian formal prescheme), then the topologically ringed space induced on each open set of $\mathfrak{X}$ a formal prescheme (resp. a locally Noetherian formal prescheme, resp. a Noetherian formal prescheme).
\end{cor}

\begin{defn}[10.4.5]
\label{1.10.4.5}
Given two formal preschemes $\mathfrak{X}$ and $\mathfrak{Y}$, a morphism (of formal preschemes) from $\mathfrak{X}$ to $\mathfrak{Y}$ is a morphism $(\psi,\theta)$ of topologically ringed spaces such that, for all $x\in\mathfrak{X}$, $\theta_x^\sharp$ is a local homomorphism $\OO_{\psi(x)}\to\OO_x$.
\end{defn}

It is immediate that the composition of any two morphisms of formal preschemes is again a morphism of formal preschemes; the formal preschemes thus form a \emph{category}, and we denote by $\Hom(\mathfrak{X},\mathfrak{Y})$ the set of morphisms from a formal prescheme $\mathfrak{X}$ to a formal prescheme $\mathfrak{Y}$.

\oldpage[I]{186}
If $U$ is an open subset of $\mathfrak{X}$, then the canonical injection into $\mathfrak{X}$ of the formal prescheme induced on $U$ by $\mathfrak{X}$ is a morphism of formal preschemes (and similarly a \emph{momomorphism} of topologically ringed spaces \sref[0]{0.4.1.1}).

\begin{prop}[10.4.6]
\label{1.10.4.6}
Let $\mathfrak{X}$ be a formal prescheme, $\mathfrak{S}=\Spf(A)$ a formal affine scheme.
There exists a canonical bijective equivalence between the morphisms from a formal prescheme $\mathfrak{X}$ to the formal prescheme $\mathfrak{S}$ and the continuous homomorphisms from the ring $A$ to the topological ring $\Gamma(\mathfrak{X}.\OO_\mathfrak{X})$.
\end{prop}

\begin{proof}
\label{proof-1.10.4.6}
The proof is similar to that of \sref{1.2.2.4}, by replacing ``homomorphism'' by ``continuous homomorphism'', ``affine open'' by ``formal affine open'', and by using Proposition \sref{1.10.2.2} instead of Theorem \sref{1.1.7.3}; we leave the details to the reader.
\end{proof}

\begin{env}[10.4.7]
\label{1.10.4.7}
Given a formal prescheme $\mathfrak{S}$, we say that the data of a formal prescheme $\mathfrak{X}$ and a morphism $\vphi:\mathfrak{X}\to\mathfrak{S}$ defines a formal prescheme \emph{$\mathfrak{X}$ over $\mathfrak{S}$} or an \emph{formal $\mathfrak{S}$-prescheme}, $\vphi$ being called the \emph{structure morphism} of the $\mathfrak{S}$-prescheme $\mathfrak{X}$.
If $\mathfrak{S}=\Spf(A)$, where $A$ is an admissible ring, then we also say that the formal $\mathfrak{S}$-prescheme $\mathfrak{X}$ is a \emph{formal $A$-prescheme} or a formal prescheme \emph{over $A$}.
An arbitrary formal prescheme can be considered as a formal prescheme over $\bb{Z}$ (equipped with the discrete topology).

If $\mathfrak{X}$ and $\mathfrak{Y}$ are two formal $\mathfrak{S}$-preschemes, we say that a morphism $u:\mathfrak{X}\to\mathfrak{Y}$ is a \emph{$\mathfrak{S}$-morphism} if the diagram
\[
  \xymatrix{
    \mathfrak{X}\ar[rr]^u\ar[rd] & &
    \mathfrak{Y}\ar[ld]\\
    & \mathfrak{S}
  }
\]
(where the downwards arrows are the structure morphisms) is commutative.
With this definition, the formal $\mathfrak{S}$-preschemes (for $\mathfrak{S}$ fixed) forms a \emph{category}.
We denote by $\Hom_\mathfrak{S}(\mathfrak{X},\mathfrak{Y})$ the set of $\mathfrak{S}$-morphisms from a formal $\mathfrak{S}$-prescheme $\mathfrak{X}$ to a formal $\mathfrak{S}$-prescheme $\mathfrak{Y}$.
When $\mathfrak{S}=\Spf(A)$, we also say \emph{$A$-morphism} instead of \emph{$\mathfrak{S}$-morphism}.
\end{env}

\begin{env}[10.4.8]
\label{1.10.4.8}
As each affine scheme can be considered as a formal affine scheme \sref{1.10.1.2}, each (usual) prescheme can be considered as a formal prescheme.
In addition, it follows from Definition \sref{1.10.4.5} that for the \emph{usual} preschemes, the morphisms (resp. $S$-morphisms) of \emph{formal} preschemes coincide with the morphisms (resp. $S$-morphisms) defined in \textsection2.
\end{env}

\subsection{Sheaves of ideals of definition for formal preschemes}
\label{subsection:1.10.5}

\begin{env}[10.5.1]
\label{1.10.5.1}
Let $\mathfrak{X}$ be a formal prescheme; we say that an $\OO_\mathfrak{X}$-ideal $\sh{J}$ is a \emph{sheaf of ideals of definition} (or an \emph{ideal sheaf of definition}) for $\mathfrak{X}$ if every $x\in\mathfrak{X}$ has a formal affine open neighborhood $U$ such that $\sh{J}|U$ is a sheaf of ideals of definition for the formal affine scheme induced on $U$ by $\mathfrak{X}$ \sref{1.10.3.3}; according to (10.3.1.1) and Proposition \sref{1.10.4.3}, for each open $V\subset\mathfrak{X}$, $\sh{J}|V$ is then a sheaf of ideals of definition for the formal prescheme induced on $V$ by $\mathfrak{X}$.

We say that a family $(\sh{J}_\lambda)$ of sheaves of ideals of definition for $\mathfrak{X}$ is a \emph{fundamental system}
\oldpage[I]{187}
\emph{of sheaves of ideals of definition} if there exists a cover $(U_\alpha)$ of $\mathfrak{X}$ by formal affine open sets such that, for each $\alpha$, the family of the $\sh{J}_\lambda|U_\alpha$ is a fundamental system of sheaves of ideals of definition \sref{1.10.3.6} for the formal affine scheme induced on $U_\alpha$ by $\mathfrak{X}$.
It follows from the last remark of \sref{1.10.3.7} that when $\mathfrak{X}$ is a formal affine scheme, this definition coincides with the definition given in \sref{1.10.3.7}.
For an open subset $V$ of $\mathfrak{X}$, the restrictions $\sh{J}_\lambda|V$ then form a fundamental system of sheaves of ideals of definition for the formal prescheme induced on $V$, according to (10.3.1.1).
If $\mathfrak{X}$ is a \emph{locally Noetherian} formal prescheme, and $\sh{J}$ is a sheaf of ideals of definition for $\mathfrak{X}$, then it follows from Proposition \sref{1.10.3.6} that the powers $\sh{J}^n$ form a fundamental system of sheaves of ideals of definition for $\mathfrak{X}$.
\end{env}

\begin{env}[10.5.2]
\label{1.10.5.2}
Let $\mathfrak{X}$ be a formal prescheme, $\sh{J}$ a sheaf of ideals of definition for $\mathfrak{X}$.
Then the ringed space $(\mathfrak{X},\OO_\mathfrak{X}/\sh{J})$ is a (usual) \emph{prescheme}, which is affine (resp. locally Noetherian, resp. Noetherian) when $\mathfrak{X}$ is a formal affine scheme (resp. a locally Noetherian formal scheme, resp. a Noetherian formal scheme);  we can reduce to the affine case, and then the proposition has already been proven in \sref{1.10.3.2}.
In addition, if $\theta:\OO_\mathfrak{X}\to\OO_\mathfrak{X}/\sh{J}$ is the canonical homomorphism, then $u=(1_\mathfrak{X},\theta)$ is a \emph{morphism} (said to be \emph{canonical}) of formal preschemes $(\mathfrak{X},\OO_\mathfrak{X}/\sh{J})\to(\mathfrak{X},\OO_\mathfrak{X})$, because again, this was proven in the affine case \sref{1.10.3.2}, to which it is immediately reduced.
\end{env}

\begin{prop}[10.5.3]
\label{1.10.5.3}
Let $\mathfrak{X}$ be a formal prescheme, $(\sh{J}_\lambda)$ a fundamental system of sheaves of ideals of definition for $\mathfrak{X}$.
Then the sheaf of topological rings $\OO_\mathfrak{X}$ is the projective limit of the pseudo-discrete sheaves of rings \sref[0]{0.3.8.1} $\OO_\mathfrak{X}/\sh{J}_\lambda$.
\end{prop}

\begin{proof}
\label{proof-1.10.5.3}
As the topology of $\mathfrak{X}$ admits a basis of formal quasi-compact affine open sets \sref{1.10.4.3}, we reduce to the affine case, where the proposition is a consequence of Proposition \sref{1.10.3.5}, \sref{1.10.3.2}, and the definition \sref{1.10.1.1}.
\end{proof}

It is not true that any formal prescheme admits a sheaf of ideals of definition.
However:
\begin{prop}[10.5.4]
\label{1.10.5.4}
Let $\mathfrak{X}$ be a locally Noetherian formal prescheme.
There exists a largest sheaf of ideals of definition $\sh{T}$ for $\mathfrak{X}$; this is the unique sheaf of ideals of definition $\sh{J}$ such that the prescheme $(\mathfrak{X},\OO_\mathfrak{X}/\sh{J})$ is reduced.
If $\sh{J}$ is a sheaf of ideals of definition for $\mathfrak{X}$, then $\sh{T}$ is the inverse image under $\OO_\mathfrak{X}\to\OO_\mathfrak{X}/\sh{J}$ of the nilradical of $\OO_\mathfrak{X}/\sh{J}$.
\end{prop}

\begin{proof}
\label{proof-1.10.5.4}
Suppose first that $\mathfrak{X}=\Spf(A)$, where $A$ is an adic Noetherian ring.
The existence and the properties of $\sh{T}$ follow immediately from Propositions \sref{1.10.3.5} and \sref{1.5.1.1}, taking into account the existence and the properties of the largest ideal of definition for $A$ (\sref[0]{0.7.1.6} and \sref[0]{0.7.1.7}).

To prove the existence and the properties of $\sh{T}$ in the general case, it suffices to show that if $U\supset V$ are two Noetherian formal affine open subsets of $X$, then the largest sheaf of ideals of definition $\sh{T}_U$ for $U$ induces the largest sheaf of ideals of definition $\sh{T}_V$ for $V$; but as $(V,(\OO_\mathfrak{X}|V)/(\sh{T}_U|V))$ is reduced, this follows from the above.
\end{proof}

We denote by $\mathfrak{X}_\text{red}$ the (usual) reduced prescheme $(\mathfrak{X},\OO_\mathfrak{X}/\sh{T}$).

\begin{cor}[10.5.5]
\label{1.10.5.5}
Let $\mathfrak{X}$ be a locally Noetherian formal prescheme, $\sh{T}$ the largest sheaf of ideals of definition for $\mathfrak{X}$; for each open subset $V$ of $\mathfrak{X}$, $\sh{T}|V$ is the largest sheaf of ideals of definition for the formal prescheme induced on $V$ by $\mathfrak{X}$.
\end{cor}

\begin{prop}[10.5.6]
\label{1.10.5.6}
Let $\mathfrak{X}$ and $\mathfrak{Y}$ be two formal preschemes, $\sh{J}$ (resp. $\sh{K}$) be a sheaf of ideals of definition for $\mathfrak{X}$ (resp. $\mathfrak{Y}$), $f:\mathfrak{X}\to\mathfrak{Y}$ a morphism of formal preschemes.
\begin{enumerate}[label=\emph{(\roman*)}]
  \item If $f^*(\sh{K})\OO_\mathfrak{X}\subset\sh{J}$, then there exists a unique morphism $f':(\mathfrak{X},\OO_\mathfrak{X}/\sh{J})\to(\mathfrak{Y},\OO_\mathfrak{Y}/\sh{K})$ of usual preschemes making the diagram
    \[
      \xymatrix{
        (\mathfrak{X},\OO_\mathfrak{X})\ar[r]^f &
        (\mathfrak{Y},\OO_\mathfrak{Y})\\
        (\mathfrak{X},\OO_\mathfrak{X}/\sh{J})\ar[r]^{f'}\ar[u] &
        (\mathfrak{Y},\OO_\mathfrak{Y}/\sh{K})\ar[u]
      }
      \tag{10.5.6.1}
    \]
    commutative, where the vertical arrows are the canonical morphisms.
  \item Suppose that $\mathfrak{X}=\Spf(A)$ and $\mathfrak{Y}=\Spf(B)$ are two formal affine schemes, $\sh{J}=\mathfrak{J}^\Delta$ and $\sh{K}=\mathfrak{K}^\Delta$, where $\mathfrak{J}$ (resp. $\mathfrak{K}$) is an ideal of definition for $A$ (resp. $B$), and $f=({}^a\vphi,\wt{\vphi})$, where $\vphi:B\to A$ is a continuous homomorphism;
    for $f^*(\sh{K})\OO_\mathfrak{X}\subset\sh{J}$ to hold, it is necessary and sufficient that $\vphi(\mathfrak{K})\subset\mathfrak{J}$, and $f'$ is then the morphism $({}^a\vphi',\wt{\vphi'})$, where $\vphi':B/\mathfrak{K}\to A/\mathfrak{J}$ is the homomorphism induced from $\vphi$ by passing to quotients.
\end{enumerate}
\end{prop}

\begin{proof}
\label{proof-1.10.5.6}
\medskip\noindent
\begin{enumerate}[label=(\roman*)]
  \item If $f=(\psi,\theta)$, then the hypotheses imply that the image under $\theta^\sharp:\psi^*(\OO_\mathfrak{Y})\to\OO_\mathfrak{X}$ of the sheaf of ideals $\psi^*(\sh{K})$ of $\psi^*(\OO_\mathfrak{Y})$ is contained in $\sh{J}$ \sref[0]{0.4.3.5}.
    By passing to quotients, we thus induce from $\theta^\sharp$ a homomorphism of sheaves of rings
    \[
      \omega:\psi^*(\OO_\mathfrak{Y}/\sh{K})=\psi^*(\OO_\mathfrak{Y})/\psi^*(\sh{K})\to\OO_\mathfrak{X}/\sh{J};
    \]
    in addition, as for all $x\in\mathfrak{X}$, $\theta_x^\sharp$ is a \emph{local} homomorphism, so is $\omega_x$.
    The morphism of ringed spaces $(\psi,\omega^\flat)$ is thus \sref{1.2.2.1} the unique morphism $f'$ of ringed spaces which we need.
  \item The canonical functorial correspondence between morphisms of formal affine schemes and continuous homomorphisms of rings \sref{1.10.2.2} shows that in the case considered, the relation $f^*(\sh{K})\OO_\mathfrak{X}\subset\mathfrak{J}$ impliex that we have $f'=({}^a\vphi',\wt{\vphi'})$, where $\vphi':B/\mathfrak{K}\to A/\mathfrak{J}$ is the unique homomorphism making the diagram
    \[
      \xymatrix{
        B\ar[r]^\vphi\ar[d] &
        A\ar[d]\\
        B/\mathfrak{K}\ar[r]^{\vphi'} &
        A/\mathfrak{J}
      }
      \tag{10.5.6.2}
    \]
    commutative.
    The existence of $\vphi'$ thus implies that $\vphi(\mathfrak{K})\subset\mathfrak{J}$.
    Conversely, if this condition is satisfied, then denoting by $\vphi'$ the unique homomorphism making the diagram (10.5.6.2) commutative and setting $f'=({}^a\vphi',\wt{\vphi'})$, it is clear that the diagram (10.5.6.1) is commutative; the consideration of the homomorphisms ${}^a\vphi^*(\OO_\mathfrak{Y})\to\OO_\mathfrak{X}$ and ${}^a{\vphi'}^*(\OO_\mathfrak{Y}/\sh{K})\to\OO_\mathfrak{X}/\sh{J}$ corresponding to $f$ and $f'$ respectively then shows that this implies the relation $f^*(\sh{K})\OO_\mathfrak{X}\subset\sh{J}$.
\end{enumerate}
\end{proof}

It is clear that the correspondence $f\mapsto f'$ defined above is \emph{functorial}.

\subsection{Formal preschemes as inductive limits of preschemes}
\label{subsection:1.10.6}

\begin{env}[10.6.1]
\label{1.10.6.1}
Let $\mathfrak{X}$ be a formal prescheme, $(\sh{J}_\lambda)$ a fundamental system of sheaves of ideals of definition for $\mathfrak{X}$; for each $\lambda$, let $f_\lambda$ be the canonical morphism $(\mathfrak{X},\OO_\mathfrak{X}/\sh{J}_\lambda)\to\mathfrak{X}$ \sref{1.10.5.2}; for $\sh{J}_\mu\subset\sh{J}_\lambda$, the canonical morphism $\OO_\mathfrak{X}/\sh{J}_\mu\to\OO_\mathfrak{X}/\sh{J}_\lambda$ defines a canonical morphism
\oldpage[I]{189}
$f_{\mu\lambda}:(\mathfrak{X},\OO_\mathfrak{X}/\sh{J}_\lambda)\to(\mathfrak{X},\OO_\mathfrak{X}/\sh{J}_\mu)$ of (usual) preschemes such that we have $f_\lambda=f_\mu\circ f_{\mu\lambda}$.
The preschemes $X_\lambda=(\mathfrak{X},\OO_\mathfrak{X}/\sh{J}_\lambda)$ and the morphisms $f_{\mu\lambda}$ thus form (according to \sref{1.10.4.8}) a \emph{inductive system} in the category of formal preschemes.
\end{env}

\begin{prop}[10.6.2]
\label{1.10.6.2}
With the notation of \sref{1.10.6.1}, the formal prescheme $\mathfrak{X}$ and the morphisms $f_\lambda$ form an inductive limit (T, I, 1.8) of the system $(X_\lambda,f_{\mu\lambda})$ in the category of formal preschemes.
\end{prop}

\begin{proof}
\label{proof-1.10.6.2}
Let $\mathfrak{Y}$ be a formal prescheme, and for each index $\lambda$, let
\[
  g_\lambda=(\psi_\lambda,\theta_\lambda):X_\lambda\to\mathfrak{Y}
\]
be a morphism such that we have $g_\lambda=g_\mu\circ f_{\mu\lambda}$ for $\sh{J}_\mu\subset\sh{J}_\lambda$.
This latter condition and the definition of the $X_\lambda$ imply first that the $\psi_\lambda$ are identical to a continuous map $\psi:\mathfrak{X}\to\mathfrak{Y}$ of the underlying spaces; in addition, the homomorphism $\theta_\lambda^\sharp:\psi^*(\OO_\mathfrak{Y})\to\OO_{X_i}=\OO_\mathfrak{X}/\sh{J}_\lambda$ form a \emph{projective system} of homomorphisms of sheaves of rings.
By passing to the projective limit, we thus induce a homomorphism $\omega:\psi^*(\OO_\mathfrak{Y})\to\varprojlim\OO_\mathfrak{X}/\sh{J}_\lambda=\OO_\mathfrak{X}$, and it is clear that the morphism $g=(\psi,\omega^\flat)$ of \emph{ringed spaces} is the \emph{unique} morphism making the diagrams
\[
  \xymatrix{
    X_\lambda\ar[rr]^{g_\lambda}\ar[rd]_{f_\lambda} & &
    \mathfrak{Y}\\
    & \mathfrak{X}\ar[ru]_g
  }
  \tag{10.6.2.1}
\]
commutative.
It remains to prove that $g$ is a morphism of \emph{formal preschemes}; the question is local on $\mathfrak{X}$ and $\mathfrak{Y}$, so we can assume $\mathfrak{X}=\Spf(A)$ and $\mathfrak{Y}=\Spf(B)$, $A$ and $B$ admissible rings, with $\sh{J}_\lambda=\mathfrak{J}_\lambda^\Delta$, where $(\mathfrak{J}_\lambda)$ is a fundamental system of ideal of definition for $A$ \sref{1.10.3.5}; as $A=\varprojlim A/\mathfrak{J}_\lambda$, the existence of a morphism $g$ of formal affine schemes making the diagrams (10.6.2.1) commutative then follows from the bijective correspondence \sref{1.10.2.2} between morphisms of formal affine schemes and continuous ring homomorphisms, and from the definition of the projective limit.
But the uniqueness of $g$ as a morphism of ringed spaces shows that it coincides with the morphism in the beginning of the proof.
\end{proof}

The following proposition establishes, under certain additional conditions, the existence of the inductive limit of a given inductive system of (usual) preschemes in the category of formal preschemes:
\begin{prop}[10.6.3]
\label{1.10.6.3}
Let $\mathfrak{X}$ be a topological space, $(\OO_i,u_{ji})$ a projective system of sheaves of rings on $\mathfrak{X}$, with $\bb{N}$ for its set of indices.
Let $\sh{J}_i$ be the kernel of $u_{0i}:\OO_i\to\OO_0$.
Suppose that:
\begin{enumerate}[label=\emph{(\alph*)}]
  \item The ringed space $(\mathfrak{X},\OO_i)$ is a prescheme $X_i$.
  \item For all $x\in\mathfrak{X}$ and all $i$, there exists an open neighborhood $U_i$ of $x$ in $\mathfrak{X}$ such that the restriction $\sh{J}_i|U_i$ is nilpotent.
  \item The homomorphisms $u_{ji}$ are surjective.
\end{enumerate}

\oldpage[I]{190}
Let $\OO_\mathfrak{X}$ be the sheaf of topological rings formed as the projective limit of the pseudo-discrete sheaves of rings $\OO_i$, and let $u_i:\OO_\mathfrak{X}\to\OO_i$ be the canonica homomorphism.
Then the topologically ringed space $(\mathfrak{X},\OO_\mathfrak{X})$ is a formal prescheme; the homomorphisms $u_i$ are surjective; their kernels $\sh{J}^{(i)}$ form a fundamental system of sheaves of ideals of definition for $\mathfrak{X}$, and $\sh{J}^{(0)}$ is the projective limit of the sheaves of ideals $\sh{J}_i$.
\end{prop}

\begin{proof}
\label{proof-1.10.6.3}
We first note that on each stalk, $u_{ji}$ is a surjective homomorphism and \emph{a fortiori} a local homomorphism; thus $v_{ij}=(1_\mathfrak{X},u_{ji})$ is a morphism of preschemes $X_j\to X_i$ ($i\geq j$) \sref{1.2.2.1}.
Suppose first that each $X_i$ is an affine scheme woth ring $A_i$.
There exists a \emph{ring} homomorphism $\vphi_{ji}:A_i\to A_j$ such that $u_{ji}=\wt{\vphi_{ji}}$ \sref{1.1.7.3}; as a result \sref{1.1.6.3}, the sheaf $\OO_j$ is a quasi-coherent $\OO_i$-module over $X_i$ (for the external law defined by $u_{ji}$), associated to $A_j$ considered as an $A_i$-module $\vphi_{ji}$.
For all $f\in A_i$, let $f'=\vphi_{ji}(f)$; by hypothesis, the open sets $D(f)$ and $D(f')$ are identical in $\mathfrak{X}$, and the homomorphism from $\Gamma(D(f),\OO_i)=(A_i)_f$ to $\Gamma(D(f),\OO_j)=(A_j)_{f'}$ corresponding to $u_{ji}$ is none other than $(\vphi_{ji})_f$ \sref{1.1.6.1}.
But when we consider $A_j$ as an $A_i$-module, $(A_j)_{f'}$ is the $(A_i)_f$-module $(A_j)_f$, so we also have $u_{ji}=\wt{\vphi_{ji}}$, where $\vphi_{ji}$ is now considered as a homomorphism of \emph{$A_i$-modules}.
Then as $u_{ji}$ is surjective, we conclude that $\vphi_{ji}$ is also surjective \sref{1.1.3.9} and if $\mathfrak{J}_{ji}$ is the kernel of $\vphi_{ji}$, then the kernel of $u_{ji}$ is a quasi-coherent $\OO_i$-module equal to $\wt{\mathfrak{J}_{ji}}$.
In particular, we have $\sh{J}_i=\wt{\mathfrak{J}_i}$, where $\mathfrak{J}_i$ is the kernel of $\vphi_{0i}:A_i\to A_0$.
The hypothesis (b) implies that $\sh{J}_i$ is \emph{nilpotent}: indeed, as $\mathfrak{X}$ is quasi-compact, we can cover $\mathfrak{X}$ by a finite number of open sets $U_k$ such that $(\sh{J}_i|U_k)^{n_k}=0$, and by setting $n$ to be the largest of the $n_k$, we have $\sh{J}_i^n=0$.
We conclude that $\mathfrak{J}_i$ is nilpotent \sref{1.1.3.13}.
Then the ring $A=\varprojlim A_i$ is admissible \sref[0]{0.7.2.2}, the canonical homomorphism $\vphi_i:A\to A_i$ is surjective, and its kernel $\mathfrak{J}^{(i)}$ is equal to the projective limit of the $\mathfrak{J}_{ik}$ for $k\geq i$; the $\mathfrak{J}^{(i)}$ form a fundamental system of neighborhoods of $0$ in $A$.
The assertions of Proposition \sref{1.10.6.3} follow in this case from \sref{1.10.1.1} and \sref{1.10.3.2}, $(\mathfrak{X},\OO_\mathfrak{X})$ being $\Spf(A)$.

In this particular case, we note that if $f=(f_i)$ is an element of the projective limit $A=\varprojlim A_i$, then all the open sets $D(f_i)$ (affine open sets in $X_i$) identify with the open subset $\mathfrak{D}(f)$ of $\mathfrak{X}$, the prescheme induced on $\mathfrak{D}(f)$ by $X_i$ thus identifying with the affine scheme $\Spec((A_i)_{f_i})$.

In the general case, we remark first that for every quasi-compact open subset $U$ of $\mathfrak{X}$, each of the $\sh{J}_i|U$ is nilpotent, as shown by the above reasoning.
We will see that for every $x\in\mathfrak{X}$, there exists an open neighborhood $U$ of $x$ in $\mathfrak{X}$ which is an \emph{affine open set} for \emph{all} the $X_i$.
Indeed, we take $U$ to be an affine open set for $X_0$, and observe that $\OO_{X_0}=\OO_{X_i}/\sh{J}_i$.
As $\sh{J}_i|U$ is nilpotent according to the above, $U$ is also an affine open set for each $X_i$ by Proposition \sref{1.5.1.9}.
This being so, for each $U$ satsifying the preceding conditions, the study of the affine case as above shows that $(U,\OO_X|U)$ is a formal prescheme whose $\sh{J}^{(i)}|U$ for a fundamental system of sheaves of ideals of definition, and $\sh{J}^{(0)}|U$ is the projective limit of the $\sh{J}_i|U$; hence the conclusion.
\end{proof}

\begin{cor}[10.6.4]
\label{1.10.6.4}
Suppose that for $i\geq j$, the kernel of $u_{ji}$ is $\sh{J}_i^{j+1}$ and that $\sh{J}_1/\sh{J}_1^2$
\oldpage[I]{191}
is of finite type over $\OO_0=\OO_1/\sh{J}_1$.
Then $\mathfrak{X}$ is an adic formal prescheme, and if $\sh{J}^{(n)}$ is the kernel of $\OO_\mathfrak{X}\to\OO_n$, then we have $\sh{J}^{(n)}=\sh{J}^{n+1}$ and $\sh{J}/\sh{J}^2$ is isomorphic to $\sh{J}_1$.
If in addition $X_0$ is locally Noetherian (resp. Noetherian), then $\mathfrak{X}$ is locally Noetherian (resp. Noetherian).
\end{cor}

\begin{proof}
\label{proof-1.10.6.4}
As the underlying spaces of $\mathfrak{X}$ and $X_0$ are the same, the question is local, and we can suppose that all the $X_i$ are affine; taking into account the relations $\sh{J}_{ij}=\wt{\mathfrak{J}_{ji}}$ (with the notation of Proposition \sref{1.10.3.6}), we immediately reduce to the corresponding assertions of Proposition \sref[0]{0.7.2.7} and Corollary \sref[0]{0.7.2.8}, by noting that $\mathfrak{J}_1/\mathfrak{J}_1^2$ is then an $A_0$-module of finite type \sref{1.1.3.9}.
\end{proof}

In particular, \emph{every locally Noetherian formal prescheme $\mathfrak{X}$} is the inductive limit of a sequence $(X_n)$ of locally Noetherian (usual) preschemes satisfying the conditions of Proposition \sref{1.10.3.6} and Corollary \sref{1.10.6.4}: it suffices to consider a sheaf of ideals of definition $\sh{J}$ for $\mathfrak{X}$ \sref{1.10.5.4} and by setting $X_n=(\mathfrak{X},\OO_\mathfrak{X}/\sh{J}^{n+1})$ (\sref{1.10.5.1} and Proposition \sref{1.10.6.2}).

\begin{cor}[10.6.5]
\label{1.10.6.5}
Let $A$ be an admissible ring.
For the formal affine scheme $\mathfrak{X}=\Spf(A)$ to be Noetherian, it is necessary and sufficient for $A$ to be adic and Noetherian.
\end{cor}

\begin{proof}
\label{proof-1.10.6.5}
The condition is evidently sufficient.
Conversely, suppose that $\mathfrak{X}$ is Noetherian, and let $\mathfrak{J}$ be an ideal of definition for $A$, $\sh{J}=\mathfrak{J}^\Delta$ the corresponding sheaf of ideals of definition for $\mathfrak{X}$.
The (usual) preschemes $X_n=(\mathfrak{X},\OO_\mathfrak{X}/\sh{J}^{n+1})$ are then affine and Noetherian, so the rings $A_n=A/\mathfrak{J}^{n+1}$ are Noetherian \sref{1.6.1.3}, hence we conclude that $\mathfrak{J}/\mathfrak{J}^2$ is an $A/\mathfrak{J}$-module of finite type.
As the $\sh{J}^n$ form a fundamental system of sheaves of ideals of definition for $\mathfrak{X}$ \sref{1.10.5.1}, we have $\OO_\mathfrak{X}=\varprojlim\OO_\mathfrak{X}/\sh{J}^n$ \sref{1.10.5.3}; we conclude \sref{1.10.1.3} that $A$ is topologically isomorphic to $\varprojlim A/\mathfrak{J}^n$, which is adic and Noetherian \sref[0]{0.7.2.8}.
\end{proof}

\begin{rmk}[10.6.6]
\label{1.10.6.6}
With the notation of Proposition \sref{1.10.6.3}, let $\sh{F}_i$ be an $\OO_i$-module, and suppose we are given, for $i\geq i$, a $v_{ij}$-morphism $\theta_{ji}:\sh{F}_i\to\sh{F}_j$, such that $\theta_{kj}\circ\theta_{ji}=\theta_{ki}$ for $k\leq j\leq i$.
As the continuous underlying map of $v_{ij}$ is the identity, $\theta_{ji}$ is a homomorphism of sheaves of abelian groups on the space $\mathfrak{X}$; in addition, if $\sh{F}$ is the projective limit of the projective system $(\sh{F}_i)$ of sheaves of abelian groups, the fact that the $\theta_{ji}$ are $v_{ij}$-morphism allows one to define on $\sh{F}$ an $\OO_\mathfrak{X}$-module structure by passing to the projective limit; equipped with this structure, we say that $\sh{F}$ is the \emph{projective limit} (with respect to the $\theta_{ji}$) of the system of $\OO_i$-modules $(\sh{F}_i)$.
In the particular case where $v_{ij}^*(\sh{F}_i)=\sh{F}_j$, and where $\theta_{ji}$ is the \emph{identity}, we say that $\sh{F}$ is the projective limit of a system $(\sh{F}_i)$ such that $v_{ij}^*(\sh{F}_i)=\sh{F}_j$ for $j\leq i$ (without mentioning the $\theta_{ji}$).
\end{rmk}

\begin{env}[10.6.7]
\label{1.10.6.7}
Let $\mathfrak{X}$ and $\mathfrak{Y}$ be two formal preschemes, $\sh{J}$ (resp. $\sh{K}$) a sheaf of ideals of definition for $\mathfrak{X}$ (resp. $\mathfrak{Y}$), $f:\mathfrak{X}\to\mathfrak{Y}$ a morphism such that $f^*(\sh{K})\OO_\mathfrak{X}\subset\sh{J}$.
We then have for every integer $n>0$, $f^*(\sh{K}^n)\OO_\mathfrak{X}=(f^*(\sh{K})\OO_\mathfrak{X})^n\subset\sh{J}^n$; we can thus \sref{1.10.5.6} induce from $f$ a morphism of (usual) preschemes $f_n:X_n\to Y_n$, by setting $X_n=(\mathfrak{X},\OO_\mathfrak{X}/\sh{J}^{n+1})$ and $Y_n=(\sh{Y},\OO_\mathfrak{Y}/\sh{K}^{n+1})$, and it immediately follows from the definitions that the diagrams
\[
  \xymatrix{
    X_m\ar[r]^{f_m}\ar[d] &
    Y_m\ar[d]\\
    X_n\ar[r]^{f_n} &
    Y_n
  }
  \tag{10.6.7.1}
\]
\oldpage[I]{192}
are commutative for $m\leq n$; in other words, $(f_n)$ is an \emph{inductive system} of morphisms.
\end{env}

\begin{env}[10.6.8]
\label{1.10.6.8}
Conversely, let $(X_n)$ (resp. $(Y_n)$) be an inductive system of (usual) preschemes satisfying conditions (b) and (c) of Proposition \sref{1.10.6.3}, and let $\mathfrak{X}$ (resp. $\mathfrak{Y}$) its inductive limit.
By definition of the inductive limit, each sequence $(f_n)$ of morphisms $X_n\to Y_n$ form an inductive system admitting an \emph{inductive limit $f:\mathfrak{X}\to\mathfrak{Y}$}, which is the unique morphism of formal preschemes making the diagrams
\[
  \xymatrix{
    X_n\ar[r]^{f_n}\ar[d] &
    Y_n\ar[d]\\
    \mathfrak{X}\ar[r]^f &
    \mathfrak{Y}
  }
\]
commutative.
\end{env}

\begin{prop}[10.6.9]
\label{1.10.6.9}
Let $\mathfrak{X}$ and $\mathfrak{Y}$ be locally Notherian formal preschemes, $\sh{J}$ (resp. $\sh{K}$) be a sheaf of ideals of definition for $\mathfrak{X}$ (resp. $\mathfrak{Y}$); the map $f\mapsto(f_n)$ defined in \sref{1.10.6.7} is a bijection from the set of morphisms $f:\mathfrak{X}\to\mathfrak{Y}$ such that $f^*(\sh{K})\OO_\mathfrak{X}\subset\sh{J}$ to the set of sequences $(f_n)$ of morphisms making the diagrams (10.6.7.1) commutative.
\end{prop}

\begin{proof}
\label{proof-1.10.6.9}
If $f$ is the inductive limit of this sequence, then it is necessary to show that $f^*(\sh{K})\OO_\mathfrak{X}\subset\sh{J}$.
The statement being local on $\mathfrak{X}$ and $\mathfrak{Y}$, we can reduce to the case where $\mathfrak{X}=\Spf(A)$ and $\mathfrak{Y}=\Spf(B)$ are affine, $A$ and $B$ adic Noetherian rings, $\sh{J}=\mathfrak{J}^\Delta$ and $\sh{K}=\mathfrak{K}^\Delta$, where $\mathfrak{J}$ (resp. $\mathfrak{K}$) is an ideal of definition for $A$ (resp. $B$).
We then have $X_n=\Spec(A_n)$ and $Y_n=\Spec(B_n)$, with $A_n=A/\mathfrak{J}^{n+1}$ and $B_n=B/\mathfrak{K}^{n+1}$, according to Proposition \sref{1.10.3.6} and \sref{1.10.3.2}; $f_n=({}^a\vphi_n,\wt{\vphi_n})$, where the homomorphisms $\vphi_n:B_n\to A_n$ forms a projective system, thus $f=({}^a\vphi,\wt{\vphi})$, so $f=({}^a\vphi,\wt{\vphi})$, where $\vphi=\varprojlim\vphi_n$.
The commutativity of the diagram (10.6.7.1) for $m=0$ then gives the condition $\vphi_n(\mathfrak{K}/\mathfrak{K}^{n+1})\subset\mathfrak{J}/\mathfrak{J}^{n+1}$ for all $n$, so by passing to the projective limit we have $\vphi(\mathfrak{K})\subset\mathfrak{J}$, and this implies that $f^*(\sh{K})\OO_\mathfrak{X}\subset\sh{J}$ \sref{1.10.5.6}[ii].
\end{proof}

\begin{cor}[10.6.10]
\label{1.10.6.10}
Let $\mathfrak{X}$ and $\mathfrak{Y}$ be two locally Noetherian formal preschemes, $\sh{T}$ the largest sheaf of ideals of definition for $\mathfrak{X}$ \sref{1.10.5.4}.
\begin{enumerate}[label=\emph{(\roman*)}]
  \item For every sheaf of ideals of definition $\sh{K}$ for $\mathfrak{Y}$ and every morphism $f:\mathfrak{X}\to\mathfrak{Y}$, we have $f^*(\sh{K})\OO_\mathfrak{X}\subset\sh{T}$.
  \item There is a canonical bijective correspondence between $\Hom(\mathfrak{X},\mathfrak{Y})$ and the set of sequences $(f_n)$ of morphisms making the diagrams (10.6.7.1) commutative, where $X_n=(\mathfrak{X},\OO_\mathfrak{X}/\sh{T}^{n+1})$ and $Y_n=(\mathfrak{Y},\OO_\mathfrak{Y}/\sh{K}^{n+1})$.
\end{enumerate}
\end{cor}

\begin{proof}
\label{proof-1.10.6.10}
(ii) follows immediately from (i) and Proposition \sref{1.10.6.9}.
To prove (i), we can reduce to the case where $\mathfrak{X}=\Spf(A)$ and $\mathfrak{Y}=\Spf(B)$, $A$ and $B$ Noetherian, $\sh{T}=\mathfrak{T}^\Delta$ and $\sh{K}=\mathfrak{K}^\Delta$, where $\mathfrak{T}$ is the largest ideal of definition for $A$ and $\mathfrak{K}$ is an ideal of definition for $B$.
Let $f=({}^a\vphi,\wt{\vphi})$, where $\vphi:B\to A$ is a continuous homomorphism; as the elements of $\mathfrak{K}$ are topologically nilpotent \sref[0]{0.7.1.4}[ii], so are those of $\vphi(\mathfrak{K})$, so $\vphi(\mathfrak{K})\subset\mathfrak{T}$ since $\mathfrak{T}$ is the set of topologically nilpotent elements of $A$ \sref[0]{0.7.1.6}; hence, by Proposition \sref{1.10.5.6}[ii], we are done.
\end{proof}

\begin{cor}[10.6.11]
\label{1.10.6.11}
Let $\mathfrak{S}$, $\mathfrak{X}$, $\mathfrak{Y}$ be locally Noetherian formal preschemes, $f:\mathfrak{X}\to\mathfrak{S}$ and $g:\mathfrak{Y}\to\mathfrak{S}$ the morphisms making $\mathfrak{X}$ and $\mathfrak{Y}$ formal $\mathfrak{S}$-preschemes.
Let $\sh{J}$ (resp. $\sh{K}$, $\sh{L}$) be a sheaf of ideals of definition for $\mathfrak{S}$ (resp. $\mathfrak{X}$, $\mathfrak{Y}$), and suppose that $f^*(\sh{J})\OO_\mathfrak{X}\subset\sh{K}$ and $g^*(\sh{J})\OO_\mathfrak{Y}=\sh{L}$; set $S_n=(\mathfrak{S},\OO_\mathfrak{S}/\sh{J}^{n+1})$, $X_n=(\mathfrak{X},\OO_\mathfrak{X}/\sh{K}^{n+1})$, and $Y_n=(\mathfrak{Y},\OO_\mathfrak{Y}/\sh{L}^{n+1})$.
Then there exists a canonical bijective correspondence
\oldpage[I]{193}
between $\Hom_\mathfrak{S}(\mathfrak{X},\mathfrak{Y})$ and the set of sequences $(u_n)$ of $S_n$-morphisms $u_n:X_n\to Y_n$ making the diagrams (10.6.7.1) commutative.
\end{cor}

\begin{proof}
\label{proof-1.10.6.11}
For each $\mathfrak{S}$-morphism $u:\mathfrak{X}\to\mathfrak{Y}$, we have by definition that $f=g\circ u$, so
\[
  u^*(\sh{L})\OO_\mathfrak{X}=u^*(g^*(\sh{J}\OO_\mathfrak{Y})\OO_\mathfrak{X}=f^*(\sh{J})\OO_\mathfrak{X}\subset\sh{K},
\]
and the corollary follows from Proposition \sref{1.10.6.9}.
\end{proof}

We note that for $m\leq n$, the data of a morphism $f_n:X_n\to Y_n$ determines a unique morphism $f_m:X_m\to Y_m$ making the diagram (10.6.7.1) commutative, as we immediately see that we can reduce to the affine case; we thus have defined a map $\vphi_{mn}:\Hom_{S_n}(X_n,Y_n)\to\Hom_{S_m}(X_m,Y_m)$ and the $\Hom_{S_n}(X_n,Y_n)$ form with the $\vphi_{mn}$ a \emph{projective system of sets}; Corollary \sref{1.10.6.11} says that there is a canonical bijection
\[
  \Hom_\mathfrak{S}(\mathfrak{X},\mathfrak{Y})\isoto\varprojlim_n\Hom_{S_n}(X_n,Y_n).
\]

\subsection{Products of formal preschemes}
\label{subsection:1.10.7}

\begin{env}[10.7.1]
\label{1.10.7.1}
Let $\mathfrak{S}$ be a formal prescheme; the formal $\mathfrak{S}$-preschemes form a category, and we can define a notion of a \emph{product} of formal $\mathfrak{S}$-preschemes.
\end{env}

\begin{prop}[10.7.2]
\label{1.10.7.2}
Let $\mathfrak{X}=\Spf(B)$ and $\mathfrak{Y}=\Spf(C)$ be two formal affine schemes over a formal affine scheme $\mathfrak{S}=\Spf(A)$.
Let $\mathfrak{Z}=\Spf(B\wh{\otimes}_A C)$, $p_1$ and $p_2$ the $\mathfrak{S}$-morphisms corresponding \sref{1.10.2.2} to the canonical (continuous) $A$-homomorphisms $\rho$ and $\sigma$ from $B$ and $C$ to $B\wh{\otimes}_A C$; then $(\mathfrak{Z},p_1,p_2)$ is a product of the formal affine $\mathfrak{S}$-schemes $\mathfrak{X}$ and $\mathfrak{Y}$.
\end{prop}

\begin{proof}
\label{proof-1.10.7.2}
According to Proposition \sref{1.10.4.6}, it suffices to check that if we associate to each continuous $A$-homomorphism $\vphi:B\wh{\otimes}_A C\to D$, where $D$ is an admissible ring which is a topological $A$-algebra, the pair $(\vphi\circ\rho,\vphi\circ\sigma)$, then we define a bijection
\[
  \Hom_A(B\wh{\otimes}_A C,D)\isoto\Hom_A(B,D)\times\Hom_A(C,D),
\]
which is none other than the universal property of the completed tensor product \sref[0]{0.7.7.6}.
\end{proof}

\begin{prop}[10.7.3]
\label{1.10.7.3}
Given two formal $\mathfrak{S}$-preschemes $\mathfrak{X}$ and $\mathfrak{Y}$, the produc $\mathfrak{X}\times_\mathfrak{S}\mathfrak{Y}$ exists.
\end{prop}

\begin{proof}
\label{proof-1.10.7.3}
The proof is similar to that of Theorem \sref{1.3.2.6}, by replacing affine schemes (resp. affine open sets) by formal affine schemes (resp. formal affine open sets), and Proposition \sref{1.3.2.2} by Proposition \sref{1.10.7.2}.
\end{proof}

All the formal properties of the product of preschemes (\sref{1.3.2.7} and \sref{1.3.2.8}, \sref{1.3.3.1} and \sref{1.3.3.12}) are valid without modification for the product of formal preschemes.

\begin{env}[10.7.4]
\label{1.10.7.4}
Let $\mathfrak{S}$, $\mathfrak{X}$, $\mathfrak{Y}$ be three formal preschemes and let $f:\mathfrak{X}\to\mathfrak{S}$ and  $g:\mathfrak{Y}\to\mathfrak{S}$ two morphisms.
Suppose that there exist in $\mathfrak{S}$, $\mathfrak{X}$, $\mathfrak{Y}$ respectively three fundamental systems of sheaves of ideals of definitions $(\sh{J}_\lambda)$, $(\sh{K}_\lambda)$, $(\sh{L}_\lambda)$, having the same set of indices $I$, such that $f^*(\sh{J}_\lambda)\OO_\mathfrak{X}\subset\sh{K}_\lambda$ and $g^*(\sh{J}_\lambda)\OO_\mathfrak{Y}\subset\sh{L}_\lambda$ for all $\lambda$.
Set $S_\lambda=(\mathfrak{S},\OO_\mathfrak{S}/\sh{J}_\lambda)$, $X_\lambda=(\mathfrak{X},\OO_\mathfrak{X}/\sh{K}_\lambda)$, $Y_\lambda=(\mathfrak{Y},\OO_\mathfrak{Y}/\sh{L}_\lambda)$; for $\sh{J}_\mu\subset\sh{J}_\lambda$, $\sh{K}_\mu\subset\sh{K}_\lambda$, $\sh{L}_\mu\subset\sh{L}_\lambda$, note that $S_\lambda$ (resp. $X_\lambda$, $Y_\lambda$) is a closed subprescheme of $S_\mu$ (resp. $X_\mu$, $Y_\mu$) having the \emph{same}
\oldpage[I]{194}
underlying space \sref{1.10.6.1}.
As $S_\lambda\to S_\mu$ is a monomorphism of preschemes, we see first that the products $X_\lambda\times_{S_\lambda}Y_\lambda$ and $X_\lambda\times_{S_\mu}Y_\lambda$ are identical \sref{1.3.2.4}, since $X_\lambda\times_{S_\mu}Y_\lambda$ identifies with a closed subprescheme of $X_\mu\times_{S_\mu}Y_\mu$ having the \emph{same} underlying space \sref{1.4.3.1}.
This being so, the product $\mathfrak{X}\times_\mathfrak{S}\mathfrak{Y}$ is the \emph{inductive limit} of the usual preschemes $X_\lambda\times_{S_\lambda}Y_\lambda$: indeed, as we see in Proposition \sref{1.10.6.2}, we can reduce to the case where $\mathfrak{S}$, $\mathfrak{X}$, and $\mathfrak{Y}$ are formal affine schemes.
Taking into account Proposition \sref{1.10.5.6}[ii] and the hypotheses on the fundamental systems of sheaves of ideals of definition for $\mathfrak{S}$, $\mathfrak{X}$, and $\mathfrak{Y}$, we immediately see that our assertion follows from the definition of the completed tensor product of two algebras \sref[0]{0.7.7.1}.

In addition, let $\mathfrak{Z}$ be a formal $\mathfrak{S}$-prescheme, $(\sh{M}_\lambda)$ a fundamental system of ideals of definition for $\mathfrak{Z}$ having $I$ for its set of indices, $u:\mathfrak{Z}\to\mathfrak{X}$ and $v:\mathfrak{Z}\to\mathfrak{Y}$ two $\mathfrak{S}$-morphisms such that $u^*(\sh{K}_\lambda)\OO_\mathfrak{Z}\subset\sh{M}_\lambda$ and $v^*(\sh{L}_\lambda)\OO_\mathfrak{Z}\subset\sh{M}_\lambda$ for all $\lambda$.
If we set $Z_\lambda=(\mathfrak{Z},\OO_\mathfrak{Z}/\sh{M}_\lambda)$, and if $u_\lambda:Z_\lambda\to X_\lambda$ and $v_\lambda:Z_\lambda\to Y_\lambda$ are the $S_\lambda$-morphisms corresponding to $u$ and $v$ \sref{1.10.5.6}, then we have immediately that $(u,v)_\mathfrak{S}$ is the inductive limit of the $S_\lambda$-morphisms $(u_\lambda,v_\lambda)_{S_\lambda}$.

The considerations of this section apply in particular when $\mathfrak{S}$, $\mathfrak{X}$, and $\mathfrak{Y}$ are locally Noetherian, taking for the fundamental systems of sheaves of ideals of definition the systems consisting of the powers of a sheaf of ideals of definition \sref{1.10.5.1}.
But we note that $\mathfrak{X}\times_\mathfrak{S}\mathfrak{Y}$ is not necessarily locally Noetherian (see however \sref{1.10.13.5}).
\end{env}

\subsection{Formal completion of a prescheme along a closed subset}
\label{subsection:1.10.8}

\begin{env}[10.8.1]
\label{1.10.8.1}
Let $X$ be a \emph{locally Noetherian} (usual) prescheme, $X'$ a closed subset of the underlying space of $X$; we denote by $\Phi$ the set of \emph{coherent} sheaves of ideals $\sh{J}$ of $\OO_X$ such that the support of $\OO_X/\sh{J}$ is $X'$.
The set $\Phi$ is nonempty (\sref{1.5.2.1}, \sref{1.4.1.4}, \sref{1.6.1.1}); we order it by the relation $\supset$.
\end{env}

\begin{lem}[10.8.2]
\label{1.10.8.2}
The ordered set $\Phi$ is filtered; if $X$ is Noetherian, then for all $\sh{J}_0\in\Phi$, the set of powers $\sh{J}_0^n$ ($n>0$) is cofinal in $\Phi$.
\end{lem}

\begin{proof}
\label{proof-1.10.8.2}
If $\sh{J}_1$ and $\sh{J}_2$ are in $\Phi$, and if we set $\sh{J}=\sh{J}_1\cap\sh{J}_2$, then $\sh{J}$ is coherent since $\OO_X$ is coherent (\sref{1.6.1.1} and \sref[0]{0.5.3.4}), and we have $\sh{J}_x=(\sh{J}_1)_x\cap(\sh{J}_2)_x$ for all $x\in X$, so $\sh{J}_x=\OO_x$ for $x\not\in X'$ and $\sh{J}_x\neq\OO_x$ for $x\in X'$, which proves that $\sh{J}\in\Phi$.
On the other hand, if $X$ is Noetherian and if $\sh{J}_0$ and $\sh{J}$ are in $\Phi$, then there exists an integer $n>0$ such that $\sh{J}_0^n(\OO_X/\sh{J})=0$ \sref{1.9.3.4}, which implies that $\sh{J}_0^n\subset\sh{J}$.
\end{proof}

\begin{env}[10.8.3]
\label{1.10.8.2}
Now let $\sh{F}$ be a \emph{coherent} $\OO_X$-module; for all $\sh{J}\in\Phi$, we have that $\sh{F}\otimes_{\OO_X}(\OO_X/\sh{J})$ is a coherent $\OO_X$-module \sref{1.9.1.1} with support contained in $X'$, and we will usually identify it with its restriction to $X'$.
When $\sh{J}$ varies over $\Phi$, these sheaves form a \emph{projective system} of sheaves of abelian groups.
\end{env}

\begin{defn}[10.8.4]
\label{1.10.8.4}
Given a closed subset $X'$ of a locally Noetherian prescheme $X$ and a coherent $\OO_X$-module $\sh{F}$, we call the \emph{completion of $\sh{F}$ along $X'$}, and denote it by $\sh{F}_{/X'}$ (or $\wh{\sh{F}}$ when there is little chance of confusion), the restriction to $X'$ of the sheaf
\oldpage[I]{195}
$\varprojlim_\Phi(\sh{F}\otimes_{\OO_X}(\OO_X/\sh{J}))$; we say that its sections over $X'$ are the \emph{formal sections of $\sh{F}$ along $X'$}.
\end{defn}

It is immediate that for every open $U\subset X$, we have $(\sh{F}|U)_{/(U\cap X')}=(\sh{F}_{/X'})|(U\cap X')$.

By passing to the projective limit, it is clear that the $(\OO_X)_{/X'}$ is a sheaf of rings, and that $\sh{F}_{/X'}$ can be considered as an $(\OO_X)_{/X'}$-module.
In addition, as there exists a basis for the topology of $X'$ consisting of quasi-compat open sets, we can consider $(\OO_X)_{/X'}$ (resp. $\sh{F}_{/X'}$) as a \emph{sheaf of topological rings} (resp. of \emph{topological groups}), the projective limit of the \emph{pseudo-discrete} sheaves of rings (resp. groups) $\OO_X/\sh{F}$ (resp. $\sh{F}\otimes_{\OO_X}(\OO_X/\sh{F})=\sh{F}/\sh{J}\sh{F}$), and by passing to the projective limit, $\sh{F}_{/X'}$ then becomes a \emph{topological $(\OO_X)_{/X'}$-module} (\sref[0]{0.3.8.1} and \sref[0]{3.8.2}); recall that for every \emph{quasi-compact} open $U\subset X$, $\Gamma(U\cap X',(\OO_X)_{/X'})$ (resp. $\Gamma(U\cap X',\sh{F}_{/X'})$) is then the projective limit of the discrete rings (resp. groups) $\Gamma(U,\OO_X/\sh{J})$ (resp. $\Gamma(U,\sh{F}/\sh{J}\sh{F})$).

Now if $u:\sh{F}\to\sh{G}$ is a homomorphism of $\OO_X$-modules, then we canonically induce homomorphisms $u_\sh{J}:\sh{F}\otimes_{\OO_X}(\OO_X/\sh{J})\to\sh{G}\otimes_{\OO_X}(\OO_X/\sh{J})$ for all $\sh{J}\in\Phi$, and these homomorphisms form a projective system.
By passing to the projective limit and restricting to $X'$, they give a continuous $(\OO_X)_{/X'}$-homomorphism $\sh{F}_{/X'}\to\sh{G}_{/X'}$, denoted $u_{/X'}$ or $\wh{u}$, and we call it the \emph{completion} of the homomorphism $u$ along $X'$.
It is clear that if $v:\sh{G}\to\sh{H}$ is a second homomorphism of $\OO_X$-modles, then we have $(v\circ u)_{/X'}=(v_{/X'})\circ(u_{/X'})$, hence $\sh{F}_{/X'}$ is a \emph{covariant additive functor} in $\sh{F}$ from the category of coherent $\OO_X$-modules to the category pf topological $(\OO_X)_{/X'}$-modules.

\begin{prop}[10.8.5]
\label{1.10.8.5}
The support of $(\OO_X)_{/X'}$ is $X'$; the topologically ringed space $(X',(\OO_X)_{/X'})$ is a locally Noetherian formal prescheme, and if $\sh{J}\in\Phi$, then $\sh{J}_{/X'}$ is a sheaf of ideals of definition for this formal prescheme.
If $X=\Spec(A)$ is an affine scheme with Noetherian ring $A$, $\sh{J}=\wt{\mathfrak{J}}$, where $\mathfrak{J}$ is an ideal of $A$, and $X'=V(\mathfrak{J})$, then $(X',(\OO_X)_{/X'})$ canonically identifies with $\Spf(\wh{A})$, where $\wh{A}$ is the separated completion of $A$ with respect to the $\mathfrak{J}$-preadic topology.
\end{prop}

\begin{proof}
\label{proof-1.10.8.5}
We can evidently reduce to proving the latter assertion.
We know \sref[0]{0.7.3.3} that the separated completion $\wh{\mathfrak{J}}$ of $\mathfrak{J}$ with respect to the $\mathfrak{J}$-preadic topology identifies with the ideal $\mathfrak{J}\wh{A}$ of $\wh{A}$, where $\wh{A}$ is the Noetherian $\wh{\mathfrak{J}}$-adic ring such that $\wh{A}/\wh{\mathfrak{J}}^n=A/\mathfrak{J}^n$ \sref[0]{0.7.2.6}.
This latter relation shows that the open prime ideals of $\wh{A}$ are the ideals $\wh{\mathfrak{p}}=\mathfrak{p}\wh{\mathfrak{J}}$, where $\mathfrak{p}$ is a prime ideal of $A$ containing $\mathfrak{J}$, and that we have $\wh{\mathfrak{p}}\cap A=\mathfrak{p}$, hence $\Spf(\wh{A})=X'$.
As $\OO_X/\sh{J}^n=(A/\mathfrak{J}^n)^\sim$, the proposition immediately follows from the definitions.
\end{proof}

We say that the formal prescheme defined above is the \emph{completion of $X$ along $X'$}, and we denote it by $X_{/X'}$ or $\wh{X}$ when there is little chance of confusion.
When we take $X'=X$, we can set $\sh{J}=0$, and thus we have $X_{/X}=X$.

It is clear that if $U$ is a subprescheme induced on an open subset of $X$, then $U_{/(U\cap X')}$ canonically identifies with the formal subprescheme induced on $X_{/X'}$ by the open subset $U\cap X'$ of $X'$.

\begin{cor}[10.8.6]
\label{1.10.8.6}
The (usual) prescheme $\wh{X}_\red$ is the unique reduced subprescheme of $X$ having $X'$ for its underlying space \sref{1.5.2.1}.
For $\wh{X}$ to be Noetherian, it is necessary and sufficient that $\wh{X}_\red$ is, and it suffices that $X$ is.
\end{cor}

\begin{proof}
\label{proof-1.10.8.6}
Since the determination of $\wh{X}_\red$ is local \sref{1.10.5.4}, we can assume that $X$ is an affine scheme with Noetherian ring $A$; with the notation of Proposition \sref{1.10.8.5}, the ideal $\mathfrak{T}$ of topologically nilpotent elements of $\wh{A}$ is the inverse image under the canonical map $\wh{A}\to\wh{A}/\wh{\mathfrak{J}}=A/\mathfrak{J}$ of the nilradical of $A/\mathfrak{J}$ \sref[0]{0.7.1.3}, so $\wh{A}/\mathfrak{T}$ is isomorphic to the quotient of $A/\mathfrak{J}$ by its nilradical.
The first assertion then follows from Propositions \sref{1.10.5.4} and \sref{1.5.1.1}.
If $\wh{X}_\red$ is Noetherian, then its underlying space $X'$ is as well, so the $X_n'=\Spec(\OO_X/\sh{J}^n)$ are Noetherian \sref{1.6.1.2} and so is $\wh{X}$ \sref{1.10.6.4}; the converse is immediate by Proposition \sref{1.6.1.2}.
\end{proof}

\begin{env}[10.8.7]
\label{1.10.8.7}
The canonical homomorphisms $\OO_X\to\OO_X/\sh{J}$ (for $\sh{J}\in\Phi$) form a projective system and give, by passing to the projective limit, a homomorphism of sheaves of rings $\theta:\OO_X\to\psi_*((\OO_X)_{/X'})=\varprojlim_\Phi(\OO_X/\sh{J})$, denoting by $\psi$ the canonical injection $X'\to X$ of the underlying spaces.
We denote by $i$ (or $i_X$) the morphism (said to be \emph{canonical})
\[
  (\psi,\theta):X_{/X'}\to X
\]
of ringed spaces.

By taking tensor products, for every coherent $\OO_X$-module $\sh{F}$, the canonical homomorphisms $\OO_X\to\OO_X/\sh{J}$ give homomorphisms $\sh{F}\to\sh{F}\otimes_{\OO_X}(\OO_X/\sh{J})$ of $\OO_X$-modules which form a projective system and thus give, by passing to the projective limit, a canonical functorial homomorphism $\gamma:\sh{F}\to\psi_*(\sh{F}_{/X'})$ of $\OO_X$-modules.
\end{env}

\begin{prop}[10.8.8]
\label{1.10.8.8}
\medskip\noindent
\begin{enumerate}[label=\emph{(\roman*)}]
  \item The functor $\sh{F}_{/X'}$ (in $\sh{F}$) is exact.
  \item The functorial homomorphism $\gamma^\sharp:i^*(\sh{F})\to\sh{F}_{/X'}$ of $(\OO_X)_{/X'}$-modules in an isomorphism.
\end{enumerate}
\end{prop}

\begin{proof}
\label{proof-1.10.8.8}
\medskip\noindent
\begin{enumerate}[label=(\roman*)]
  \item It suffices to prove that if $0\to\sh{F}'\to\sh{F}\to\sh{F}''\to 0$ is an exact sequence of coherent $\OO_X$-modules, and if $U$ is an affine open subset $X$ with Noetherian ring $A$, then the sequence
\[
  0\to\Gamma(U\cap X',\sh{F}_{/X'}')\to\Gamma(U\cap X',\sh{F}_{/X'})\to\Gamma(U\cap X',\sh{F}_{/X'}'')\to 0
\]
is exact.
We have that $\sh{F}|U=\wt{M}$, $\sh{F}'|U=\wt{M'}$, $\sh{F}''|U=\wt{M''}$, where $M$, $M'$, $M''$ are three $A$-modules of finite type such that the sequence $0\to M'\to M\to M''\to 0$ is exact (\sref{1.1.5.1} and \sref{1.1.3.11}); let $\sh{J}\in\Phi$ and let $\mathfrak{J}$ be an ideal of $A$ such that $\sh{J}|U=\wt{\mathfrak{J}}$.
We then have
\[
  \Gamma(U\cap X',\sh{F}\otimes_{\OO_X}\OO_X/\sh{J}^n)=M\otimes_A(A/\mathfrak{J}^n)
\]
\sref{1.3.12}; so, by definition of the projective limit, we have
\[
  \Gamma(U\cap X',\sh{F}_{/X'})=\varprojlim_n(M\otimes_A(A/\mathfrak{J}^n))=\wh{M},
\]
the separated completion of $M$ with respect to the $\mathfrak{J}$-preadic topology, and similarly
\[
  \Gamma(U\cap X',\sh{F}_{/X'}')=\wh{M'},\ \Gamma(U\cap X',\sh{F}_{/X'}'')=\wh{M''};
\]
our assertion then follows since $A$ is Noetherian, and the functor $\wh{M}$ in $M$ is exact on the category of $A$-modules of finite type \sref[0]{0.7.3.3}.
\oldpage[I]{197}
  \item The question is local, so we can assume that we have an exact sequence $\OO_X^m\to\OO_X^n\to\sh{F}\to 0$ \sref[0]{0.5.3.2}; as $\gamma^\sharp$ is functorial, and the functors $i^*(\sh{F})$ and $\sh{F}_{/X'}$ are right exact (by (i) and \sref[0]{0.4.3.1}), we have the commutative diagram
\[
  \xymatrix{
    i^*(\OO_X^m)\ar[r]\ar[d]_{\gamma^\sharp} &
    i^*(\OO_X^n)\ar[r]\ar[d]_{\gamma^\sharp} &
    i^*(\sh{F})\ar[r]\ar[d]_{\gamma^\sharp} &
    0\\
    (\OO_X^m)_{/X'}\ar[r] &
    (\OO_X^n)_{/X'}\ar[r] &
    \sh{F}_{/X'}\ar[r] &
    0
  }
  \tag{10.8.8.1}
\]
whose rows are exact.
In addition, the two functors $i^*(\sh{F})$ and $\sh{F}_{/X'}$ commute with finite direct sums (\sref[0]{0.3.2.6} and \sref[0]{0.4.3.2}), and thus we reduce to proving our assertion for $\sh{F}=\OO_X$.
We have $i^*(\OO_X)=(\OO_X)_{/X'}=\OO_\wh{X}$ \sref[0]{0.4.3.4}, and $\gamma^\sharp$ is a homomophism of \emph{$\OO_\wh{X}$-modules}; so it suffices to check that $\gamma^\sharp$ sends the unit section of $\OO_\wh{X}$ over an open subset of $X'$ to itself, which is immediate and shows in this case that $\gamma^\sharp$ is the identity.
\end{enumerate}
\end{proof}

\begin{cor}[10.8.9]
\label{1.10.8.9}
The morphism of ringed spaces $i:X_{/X'}\to X$ is flat.
\end{cor}

\begin{proof}
\label{proof-1.10.8.9}
This follows from \sref[0]{0.6.7.3} and Proposition \sref{1.10.8.8}[i].
\end{proof}

\begin{cor}[10.8.10]
\label{1.10.8.10}
If $\sh{F}$ and $\sh{G}$ are two coherent $\OO_X$-modules, then there exist canonical functorial isomorphisms (in $\sh{F}$ and $\sh{G}$)
\[
  (\sh{F}_{/X'})\otimes_{(\OO_X)_{/X'}}(\sh{G}_{/X'})\isoto(\sh{F}\otimes_{\OO_X}\sh{G})_{/X'},
  \tag{10.8.10.1}
\]
\[
  (\shHom_{\OO_X}(\sh{F},\sh{G}))_{/X'}\isoto\shHom_{(\OO_X)_{/X'}}(\sh{F}_{/X'},\sh{G}_{/X'}).
  \tag{10.8.10.2}
\]
\end{cor}

\begin{proof}
\label{proof-1.10.8.10}
This follows from the canonical identification of $i^*(\sh{F})$ and of $\sh{F}_{/X'}$; the existence of the first isomorphism is then a valid result for all morphisms of ringed spaces (\textbf{0},~4.3.3.1), and the second is a valid result for all flat morphism \sref[0]{0.6.7.6}, by Corollary \sref{1.10.8.9}.
\end{proof}

\begin{prop}[10.8.11]
\label{1.10.8.11}
For every coherent $\OO_X$-module $\sh{F}$, the kernel of the canonical homomorphism $\Gamma(X,\sh{F})\to\Gamma(X',\sh{F}_{/X'})$ induced by $\sh{F}\to\sh{F}_{/X'}$ consists of the zero sections in a neighborhood of $X'$.
\end{prop}

\begin{proof}
\label{proof-1.10.8.11}
It follows from the definition of $\sh{F}_{/X'}$ that the canonical image of such a section is zero.
Conversely, if $s\in\Gamma(X,\sh{F})$ has a zero image in $\Gamma(X',\sh{F}_{/X'})$, then it suffices to see that every $x\in X'$ admits a neighborhood in $X$ in which $s$ is zero, and we can thus reduce to the case where $X=\Spec(A)$ is affine, $A$ Noetherian, $X'=V(\mathfrak{J})$, where $\mathfrak{J}$ is an ideal of $A$, and $\sh{F}=\wt{M}$, where $M$ is an $A$-module of finite type.
Then $\Gamma(X',\sh{F}_{/X'})$ is the separated completion $\wh{M}$ of $M$ for the $\mathfrak{J}$-preadic topology, and the homomorphism $\Gamma(X,\sh{F})\to\Gamma(X',\sh{F}_{/X'})$ is the canonical homomorphism $M\to\wh{M}$.
We know \sref[0]{0.7.3.7} that the kernel of this homomorphism is the set of the $z\in M$ killed by an element of $1+\mathfrak{J}$.
So we have $(1+f)s=0$ for some $f\in\mathfrak{J}$; for every $x\in X'$ we have $(1_x+f_x)s_x=0$, and as $1_x+f_x$ is invertible in $\OO_x$ ($\mathfrak{J}_x\OO_x$ being contained in the maximal ideal of $\OO_x$), we have $s_x=0$, which proves the proposition.
\end{proof}

\begin{cor}[10.8.12]
\label{1.10.8.12}
The support of $\sh{F}_{/X'}$ is equal to $\Supp(\sh{F})\cap X'$.
\end{cor}

\begin{proof}
\label{proof-1.10.8.12}
It is clear that $\sh{F}_{/X'}$ is an $(\OO_X)_{/X'}$-module of finite type (\sref{1.10.8.8}[ii] and \sref[0]{0.5.2.4}),
\oldpage[I]{198}
so its support is closed \sref[0]{0.5.2.2} and evidently contained in $\Supp(\sh{F})\cap X'$.
To show that it is equal to the latter set, we immediately reduce to proving that the relation $\Gamma(X',\sh{F}_{/X'})=0$ implies that $\Supp(\sh{F})\cap X'=\emp$; this follows from Proposition \sref{1.10.8.11} and from Theorem \sref{1.1.4.1}.
\end{proof}

\begin{cor}[10.8.13]
\label{1.10.8.13}
Let $u:\sh{F}\to\sh{G}$ be a homomorphism of coherent $\OO_X$-modules.
For $u_{/X'}:\sh{F}_{/X'}\to\sh{G}_{/X'}$ to be zero, it is necessary and sufficient for $u$ to be zero on a neighborhood of $X'$.
\end{cor}

\begin{proof}
\label{proof-1.10.8.13}
By Proposition \sref{1.10.8.8}[ii], $u_{/X'}$ identifies with $i^*(u)$, so if we consider $u$ as a section over $X$ of the sheaf $\sh{H}=\shHom_{\OO_X}(\sh{F},\sh{G})$, then $u_{/X'}$ is the section over $X'$ of $i^*(\sh{H})=\sh{H}_{/X'}$ to which it canonically corresponds ((10.8.10.2) and \sref[0]{0.4.4.6}).
It thus suffices to apply Proposition \sref{1.10.8.11} to the coherent $\OO_X$-module $\sh{H}$.
\end{proof}

\begin{cor}[10.8.14]
\label{1.10.8.14}
Let $u:\sh{F}\to\sh{G}$ be a homomorphism of coherent $\OO_X$-modules.
For $u_{/X'}$ to be a monomorphism (resp. an epimorphism), it is necessary and sufficient for $u$ to be a monomorphism (resp. an epimorphism) on a neighborhood of $X'$.
\end{cor}

\begin{proof}
\label{proof-1.10.8.14}
Let $\sh{P}$ and $\sh{N}$ be the cokernel and kernel of $u$, such that we have the exact sequence $0\to\sh{N}\xrightarrow{v}\sh{F}\xrightarrow{u}\sh{G}\xrightarrow{w}\sh{P}\to 0$, hence \sref{1.10.8.8}[i] the exact sequence
\[
  0\to\sh{N}_{/X'}\xrightarrow{v_{/X'}}\sh{F}_{/X'}\xrightarrow{u_{/X'}}\sh{G}_{/X'}\xrightarrow{w_{/X'}}\sh{P}_{/X'}\to 0.
\]
If $u_{/X'}$ is a monomorphism (resp. an epimorphism), then we have $v_{/X'}=0$ (resp. $w_{/X'}=0$), so there exists a neighborhood of $X'$ on which $v=0$ (resp. $w=0$) by Corollary \sref{1.10.8.13}.
\end{proof}

\subsection{Extension of morphisms to completions}
\label{subsection:1.10.9}

\begin{env}[10.9.1]
\label{1.10.9.1}
Let $X$ and $Y$ be two locally Noetherian (usual) preschemes, $f:X\to Y$ a morphism, $X'$ (resp. $Y'$) a closed subset of the underlying space $X$ (resp. $Y$), such that $f(X')\subset Y'$.
Let $\sh{J}$ (resp. $\sh{K}$) be a sheaf of ideals of $\OO_X$ (resp. $\OO_Y$) such that the support of $\OO_X/\sh{J}$ (resp. $\OO_Y/\sh{K}$) is $X'$ (resp. $Y'$) and $f^*(\sh{K})\OO_X\subset\sh{J}$; we note that there always exist such sheaves of ideals, since for example, we can take $\sh{J}$ to be the largest sheaf of ideals of $\OO_X$ defining a subprescheme of $X$ with underlying space $X'$ \sref{1.5.2.1}, and the hypothesis $f(X')\subset Y'$ implies that $f^*(\sh{K})\OO_X\subset\sh{J}$ \sref{1.5.2.4}.
For every integer $n>0$ we have $f^*(\sh{K}^n)\OO_X\subset\sh{J}^n$ \sref[0]{0.4.3.5}; as a result \sref{1.4.4.6}, if we set $X_n'=(X',\OO_X/\sh{J}^{n+1})$ and $Y_n'=(Y',\OO_Y/\sh{K}^{n+1})$, then $f$ induces a morphism $f_n:X_n'\to Y_n'$, and it is immediate that the $f_n$ form an inductive system.
We denote its inductive limit \sref{1.10.6.8} by $\wh{f}:X_{/X'}\to Y_{/Y'}$, and we say (by abuse of language) that $\wh{f}$ is the \emph{extension of $f$ to the completions of $X$ and $Y$ along $X'$ and $Y'$}.
It is immediate to check that this morphism does not depend on the choice of sheaves of ideals $\sh{J}$ and $\sh{K}$ satisfying the above conditions.
It suffices to see that when $X$ and $Y$ are Noetherian affine schemes with rings $A$ and $B$; then $\sh{J}=\wt{\mathfrak{J}}$ and $\sh{K}=\wt{\mathfrak{K}}$, where $\mathfrak{J}$ (resp. $\mathfrak{K}$) is an ideal of $A$ (resp. $B$), $f$ corresponds to a ring homomorphism $\vphi:B\to A$ such that $\vphi(\mathfrak{K})\subset\mathfrak{J}$ (\sref{1.4.4.6} and \sref{1.1.7.4}); \emph{$\wh{f}$ is then the morphism corresponding \sref{1.10.2.2} to the continuous homomorphism $\wh{\vphi}:\wh{B}\to\wh{A}$}, where $\wh{A}$ (resp. $\wh{B}$) is the separated completion of $A$ (resp. $B$) with respect to the $\mathfrak{J}$-preadic (resp. $\mathfrak{K}$-preadic) topology \sref{1.10.6.8}; we know that if we replace $\sh{J}$ by another
\oldpage[I]{199}
sheaf of ideals $\sh{J}'=\wt{\mathfrak{J}'}$ such that the support of $\OO_X/\sh{J}'$ is $X'$, then the $\mathfrak{J}$-preadic and $\mathfrak{J}'$-preadic topologies on $A$ are the same \sref{1.10.82}.

We note that by definition the continuous map $X'\to Y'$ of the underlying spaces of $X_{/X'}$ and $Y_{/Y'}$ correspondign to $\wh{f}$ is none other than the restriction to $X'$ of $f$.
\end{env}

\begin{env}[10.9.2]
\label{1.10.9.2}
It follows immediately from the above definition that the diagram of morphisms of ringed spaces
\[
  \xymatrix{
    \wh{X}\ar[r]^{\wh{f}}\ar[d]_{i_X} &
    \wh{Y}\ar[d]^{i_Y}\\
    X\ar[r]^f &
    Y
  }
\]
is commutative, the vertical arrows being the canonical morphisms \sref{1.10.8.7}.
\end{env}

\begin{env}[10.9.3]
\label{1.10.9.3}
Let $Z$ be a third prescheme, $g:Y\to Z$ a morphism, and $Z'$ a closed subset of $Z$ such that $g(Y')\subset Z'$.
If $\wh{g}$ denotes the completion of the morphism $g$ along $Y'$ and $Z'$, then it immediately follows from \sref{1.10.9.1} that we have $(g\circ f)^\wedge=\wh{g}\circ\wh{f}$.
\end{env}

\begin{prop}[10.9.4]
\label{1.10.9.4}
Let $X$ and $Y$ be two locally Noetherian $S$-preschemes, $Y$ being of finite type over $S$.
Let $f$ and $g$ be two $S$-morphisms from $X$ to $Y$ such that $f(X')\subset Y'$ and $g(X')\subset Y'$.
For $\wh{f}=\wh{g}$ to hold, it is necessary and sufficient for $f$ and $g$ to coincide on a neighborhood of $X'$.
\end{prop}

\begin{proof}
\label{proof-1.10.9.4}
The condition is evidently sufficient (without the finiteness hypothesis on $Y$).
To see that it is necessary, we remark first that the hypothesis $\wh{f}=\wh{g}$ implies that $f(x)=g(x)$ for all $x\in X'$.
On the other hand, the question being local, we can assume that $X$ and $Y$ are affine open neighborhoods of $x$ and $y=f(x)=g(x)$ respectively, with Noetherian rings, that $S$ is affine, and that $\Gamma(Y,\OO_Y)$ is a $\Gamma(S,\OO_S)$-algebra of finite type \sref{1.6.3.3}.
Then $f$ and $g$ correspond to two $\Gamma(S,\OO_S)$-homomorphisms $\rho$ and $\sigma$ from $\Gamma(Y,\OO_Y)$ to $\Gamma(X,\OO_X)$ \sref{1.1.7.3}, and by hypotheseis, the extensions by continuity of these homomorphisms to the separated completion of $\Gamma(Y,\OO_Y)$ are the same.
We conclude from Proposition \sref{1.10.8.11} that for every section $s\in\Gamma(Y,\OO_Y)$, the sections $\rho(s)$ and $\sigma(s)$ coincide on a neighborhood of $X'$ (depending on $s$); as $\Gamma(Y,\OO_Y)$ is an algebra of finite type over $\Gamma(S,\OO_S)$, we have that there exists a neighborhood $V$ of $X'$ such that $\rho(s)$ and $\sigma(s)$ coincide on $V$ for \emph{every} section $s\in\Gamma(Y,\OO_Y)$.
If $h\in\Gamma(X,\OO_X)$ is such that $D(h)$ is a neighborhood of $x$ contained in $V$, then we conclude from the above and from Theorem \sref{1.1.4.1}[d] that $f$ and $g$ coincide on $D(h)$.
\end{proof}

\begin{prop}[10.9.5]
\label{1.10.9.5}
Under the hypotheses of \sref{1.10.9.1}, for every coherent $\OO_Y$-module $\sh{G}$, there exists a canonical functorial isomorphism of $(\OO_X)_{/X'}$-modules
\[
  (f^*(\sh{G}))_{/X'}\isoto\wh{f}^*(\sh{G}_{/Y'}).
\]
\end{prop}

\begin{proof}
\label{proof-1.10.9.5}
If we canonically identify $(f^*(\sh{G}))_{/X'}$ with $i_X^*(f^*(\sh{G}))$ and $\wh{f}^*(\sh{G}_{/Y'})$ with $\wh{f}^*(i_Y^*(\sh{G}))$ \sref{1.10.8.8}, then the proposition immediately follows from the commutativity of the diagram in \sref{1.10.9.2}.
\end{proof}

\begin{env}[10.9.6]
\label{1.10.9.6}
Now let $\sh{F}$ be a coherent $\OO_X$-module, and let $\sh{G}$ be a coherent $\OO_Y$-module.
If $u:\sh{G}\to\sh{F}$ is an $f$-morphism from $\sh{G}$ to $\sh{F}$, then it corresonds to an $\OO_X$-homomorphism $u^\sharp:f^*(\sh{G})\to\sh{F}$, thus by completion a continuous $(\OO_X)_{/X'}$-homomorphism $(u^\sharp)_{/X'}:(f^*(\sh{G}))_{/X'}\to\sh{F}_{/X'}$, and by Proposition \sref{1.10.9.5}, there exists a unique $\wh{f}$-morphism $v:\sh{G}_{/Y'}\to\sh{F}_{/X'}$
\oldpage[I]{200}
such that $v^\sharp=(u^\sharp)_{/X'}$.
If we conider the triples $(\sh{F},X,X')$ ($\sh{F}$ being a coherent $\OO_X$-module and $X'$ a closed subset of $X$) as a \emph{category}, the morphisms $(\sh{F},X,X')\to(\sh{G},Y,Y')$ consisting of a morphism of preschemes $f:X\to Y$ such that $f(X')\subset Y'$ and an $f$-morphism $u:\sh{G}\to\sh{F}$, then we can say that $(X_{/X'},\sh{F}_{/X'})$ is a \emph{functor} in $(\sh{F},X,X')$ with values in the category of pairs $(\mathfrak{Z},\sh{H})$ consisting of a locally Noetherian formal prescheme $\mathfrak{Z}$ and an $\OO_\mathfrak{Z}$-module $\sh{H}$, the morphisms of the latter category being the pairs consisting of a morphism $g$ of formal preschemes and a $g$-morphism.
\end{env}

\begin{prop}[10.9.7]
\label{1.10.9.7}
Let $S$, $X$, and $Y$ be three locally Noetherian preschemes, $g:X\to S$ and $h:Y\to S$ two morphisms, $S'$ a closed subset of $S$, $X'$ (resp. $Y'$) a closed subset of $X$ (resp. $Y$) such that $g(X')\subset S'$ (resp. $h(Y')\subset S'$); let $Z=X\times_S Y$; suppose $Z$ is locally Noetherian, and let $Z'=p^{-1}(X')\cap q^{-1}(Y')$, where $p$ and $q$ are the projections of $X\times_S Y$.
With these conditions, the completion $Z_{/Z'}$ identifies with the product of formal $S_{/S'}$-preschemes $(X_{/X'})\times_{S_{/S'}}(Y_{/Y'})$, the structure morphisms identify with $\wh{g}$ and $\wh{h}$, and the projections with $\wh{p}$ and $\wh{q}$.
\end{prop}

\begin{proof}
\label{proof-1.10.9.7}
It is immediate that the question is local for $S$, $X$, and $Y$, and we thus reduce to the case where $S=\Spec(A)$, $X=\Spec(B)$, $Y=\Spec(C)$, $S'=V(\mathfrak{J})$, $X'=V(\mathfrak{K})$, and $Y'=V(\mathfrak{L})$, where $\mathfrak{J}$, $\mathfrak{K}$, and $\mathfrak{L}$ are three ideals such that $\vphi(\mathfrak{J})\subset\mathfrak{K}$ and $\psi(\mathfrak{J})\subset\mathfrak{L}$, where we denote by $\vphi$ and $\psi$ the homomorphisms $A\to B$ and $A\to C$ which correspond to $g$ and $h$.
We know that $Z=\Spec(B\otimes_A C)$ and that $Z'=V(\mathfrak{M})$, where $\mathfrak{M}$ is the ideal $\Im(\mathfrak{K}\otimes_A C)+\Im(B\otimes_A\mathfrak{L})$.
The conclusion follows \sref{1.10.7.2} from the fact that the completed tensor product $(\wh{B}\otimes_\wh{A}\wh{C})^\wedge$ (where $\wh{A}$, $\wh{B}$, and $\wh{C}$ are respectively the separated completions of $A$, $B$, and $C$ with respect to the $\mathfrak{J}$-, $\mathfrak{K}$-, and $\mathfrak{L}$-preadic topologies) is the separated completion of the tensor product $B\otimes_A C$ with respect to the $\mathfrak{M}$-preadic topology \sref[0]{0.7.7.2}.
\end{proof}

In addition, we note that if $T$ is a locally Noetherian $S$-prescheme, $u:T\to X$ and $v:T\to Y$ two $S$-morphisms, $T'$ a closed subset of $T$ such that $u(T')\subset X'$ and $v(T')\subset Y'$, then the extension to the completion $((u,v)_S)^\wedge$ identifies with $(\wh{u},\wh{v})_{S_{/S'}}$.

\begin{cor}[10.9.8]
\label{1.10.9.8}
Let $X$ and $Y$ be two locally Noetherian $S$-preschemes such that $X\times_S Y$ is locally Noetherian; let $S'$ be a closed subset of $S$, $X'$ (resp. $Y'$) a closed subset of $X$ (resp. $Y$) whose image in $S$ is contained in $S'$.
For every $S$-morphism $f:X\to Y$ such that $f(X')\subset Y'$, the graph morphism $\Gamma_\wh{f}$ identifies with the extension $(\Gamma_f)^\wedge$ of the graph morphism of $f$.
\end{cor}

\begin{cor}[10.9.9]
\label{1.10.9.9}
Let $X$ and $Y$ be two locally Noetherian preschemes, $f:X\to Y$ a morphism, $Y'$ a closed subset of $Y$, and $X'=f^{-1}(Y')$.
Then the prescheme $X_{/X'}$ is identified, by the commutative diagram
\[
  \xymatrix{
    X\ar[d]_f &
    X_{/X'}\ar[l]\ar[d]^{\wh{f}}\\
    Y &
    Y_{/Y'}\ar[l]
  }
\]
with the product $X\times_Y(Y_{/Y'})$ of formal preschemes.
\end{cor}

\begin{proof}
\label{proof-1.10.9.9}
It suffices to apply Proposition \sref{1.10.9.7}, replacing $S$ and $S'$ by $Y$, $X$ and $X'$ by $X$.
\end{proof}

\begin{rmk}[10.9.10]
\label{1.10.9.10}
\oldpage[I]{201}
If $S$ is the sum $X_1\sqcup X_2$ (3.1), $X'$ the union $X_1'\cup X_2'$, where $X_i'$ is a closed subset of $X_i$ ($i=1,2$), then we have $X_{/X'}={X_1}_{/X_1'}\sqcup{X_2}_{/X_2'}$.
\end{rmk}

\subsection{Application to coherent sheaves on formal affine schemes}
\label{subsection:1.10.10}

\begin{env}[10.10.1]
\label{1.10.10.1}
In this paragraph, $A$ denotes an \emph{adic Noetherian ring}, $\mathfrak{J}$ an ideal of definition for $A$.
Let $X=\Spec(A)$, $\mathfrak{X}=\Spf(A)$, which identifies with the closed subset $V(\mathfrak{J})$ of $X$ \sref{1.10.1.2}.
In addition, the Definitions \sref{1.10.1.2} and \sref{1.10.8.4} show that the \emph{formal affine scheme $\mathfrak{X}$} is identical the completion $X_{/\mathfrak{X}}$ of the affine scheme $X$ along the closed subset $\mathfrak{X}$ of its underling space.
To every coherent $\OO_X$-module $\sh{F}$ corresponds an $\OO_\mathfrak{X}$-module of finite type $\sh{F}_{/\mathfrak{X}}$, which is a sheaf of topological modules over over the sheaf of topological rings $\OO_\mathfrak{X}$.
Every coherent $\OO_X$-module $\sh{F}$ is of the form $\wt{M}$, where $M$ is an $A$-module of finite type \sref{1.1.5.1}; we set $(\wt{M})_{/X}=M^\Delta$.
In addition, if $u:M\to N$ is an $A$-homomorphism of $A$-modules of finite type, then it corresponds to a homomorphism $\wt{u}:\wt{M}\to\wt{N}$, and as a result to a continuous homomorphism $\wt{u}_{/X'}:(\wt{M})_{/X'}\to(\wt{N})_{/X'}$, which we denote by $u^\Delta$.
It is immediate that $(v\circ u)^\Delta=v^\Delta\circ u^\Delta$; we have thus defined a \emph{covariant additive functor $M^\Delta$} from the category of $A$-modules of finite type to the category of $\OO_\mathfrak{X}$-modules of finite type.
When $A$ is a \emph{discrete} ring, we have $M^\Delta=\wt{M}$.
\end{env}

\begin{prop}[10.10.2]
\label{1.10.10.2}
\medskip\noindent
\begin{enumerate}[label=\emph{(\roman*)}]
  \item $M^\Delta$ is an exact functor in $M$, and there exists a canonical functorial isomorphism of $A$-modules $\Gamma(\mathfrak{X},M^\Delta)\isoto M$.
  \item If $M$ and $N$ are two $A$-modules of finite type, then there exist canonical functorial isomorphisms
    \[
      (M\otimes_A N)^\Delta\isoto M^\Delta\otimes_{\OO_\mathfrak{X}}N^\Delta,
      \tag{10.10.2.1}
    \]
    \[
      (\Hom_A(M,N))^\Delta\isoto\shHom_{\OO_\mathfrak{X}}(M^\Delta,N^\Delta).
      \tag{10.10.2.2}
    \]
  \item The map $u\mapsto u^\Delta$ is a functorial isomorphism
    \[
      \Hom_A(M,N)\isoto\Hom_{\OO_X}(M^\Delta,N^\Delta).
      \tag{10.10.2.3}
    \]
\end{enumerate}
\end{prop}

\begin{proof}
\label{proof-1.10.10.2}
The exactness of $M^\Delta$ follows from the exactness of the functors $\wt{M}$ \sref{1.1.3.5} and $\sh{F}_{/X'}$ \sref{1.10.8.8}.
By definition, $\Gamma(X,M^\Delta)$ is the separated completion of the $A$-module $\Gamma(X,\wt{M})=M$ with respect to the $\mathfrak{J}$-preadic topology; but as $A$ is complete and $M$ is of finite type, we know \sref[0]{0.7.3.6} that $M$ is separated and complete, which proves (i).
The isomorphism (10.10.2.1) (resp. (10.10.2.2)) coms from the composition of the isomorphisms \sref{1.1.3.12}[i] and (10.8.10.1) (resp. \sref{1.1.3.12}[ii] and (10.8.10.2)).
Finally, as $\Hom_A(M,N)$ is an $A$-module of finite type, we can apply (i), which identifies $\Gamma(\mathfrak{X},(\Hom_A(M,N))^\Delta)$ with $\Hom_A(M,N)$, and we use (10.10.2.2), which proves that the homomorphism (10.10.2.3) is an isomorphism.
\end{proof}

We deduce from Proposition~\sref{1.10.10.2} a series of analogous results to those of Theorem~\sref{1.1.3.7} and Corollary~\sref{1.1.3.12}, which we leave to the reader to formulate.

\oldpage[I]{202}
We note that the exactness property of $M^\Delta$, applied to the exact sequence $0\to\mathfrak{J}\to A\to A/\mathfrak{J}\to 0$ shows that the sheaf of ideals of $\OO_\mathfrak{X}$ denoted here by $\mathfrak{J}^\Delta$ coincides with the one described similarly in \sref{1.10.3.1}, by \sref{1.10.3.2}.

\begin{prop}[10.10.3]
\label{1.10.10.3}
Under the hypotheses of \sref{1.10.10.1}, $\OO_\mathfrak{X}$ is a coherent sheaf of rings.
\end{prop}

\begin{proof}
\label{proof-1.10.10.3}
If $f\in A$, then we know that $A_{\{f\}}$ is an adic Noetherian ring \sref[0]{0.7.6.11}, and as the question is local, we reduce \sref{1.10.1.4} to proving that the kernel of the homomorphism $v:\OO_\mathfrak{X}^n\to\OO_\mathfrak{X}$ is an $\OO_\mathfrak{X}$-module of finite type.
We then have $v=u^\Delta$, where $u$ is an $A$-homomorphism $A^n\to A$ \sref{1.10.10.2}; as $A$ is Noetherian, the kernel of $u$ is of finite type, in other words we have a homomorphism $A^m\xrightarrow{w}A^n$ such that the sequence $A^m\xrightarrow{w}A^n\xrightarrow{u}A$ is exact.
We conclude \sref{1.10.10.2} that the sequence $\OO_\mathfrak{X}^m\xrightarrow{w^\Delta}\OO_\mathfrak{X}^n\xrightarrow{v}\OO_\mathfrak{X}$ is exact, which proves that the kernel of $v$ is of finite type.
\end{proof}

\begin{env}[10.10.4]
\label{1.10.10.4}
With the above notation, set $A_n=A/\mathfrak{J}^{n+1}$, and let $S_n$ be the affine scheme $\Spec(A_n)=(\mathfrak{X},\OO_\mathfrak{X}/\sh{J}^{n+1})$, $\sh{J}=\mathfrak{J}^\Delta$ being the sheaf of ideals of definition for $\OO_\mathfrak{X}$ corresponding to the ideal $\mathfrak{J}$.
Let $u_{mn}$ be the morphism of preschemes $X_m\to X_n$ corresponding to the canonical homomorphism $A_n\to A_m$ for $m\leq n$; the formal scheme $\mathfrak{X}$ is the inductive limit of the $X_n$ with respect to the $u_{mn}$ \sref{1.10.6.3}.
\end{env}

\begin{prop}[10.10.5]
\label{1.10.10.5}
Under the hypothesis of \sref{1.10.10.1}, let $\sh{F}$ be an $\OO_\mathfrak{X}$-module.
The following conditions are equivalent:
\begin{enumerate}[label=\emph{(\alph*)}]
  \item $\sh{F}$ is a coherent $\OO_\mathfrak{X}$-module.
  \item $\sh{F}$ is isomorphic to the projective limit \sref{1.10.6.6} of a sequence $(\sh{F}_n)$ of coherent $\OO_{X_n}$-modules such that $u_{mn}^*(\sh{F}_n)=\sh{F}_m$.
  \item There exists an $A$-module of finite type $M$ (determined up to a canonical isomorphism by Proposition \sref{1.10.10.2}[i]) such that $\sh{F}$ is isomorphic to $M^\Delta$.
\end{enumerate}
\end{prop}

\begin{proof}
\label{proof-1.10.10.5}
We first show that (b) implies (c).
We have $\sh{F}_n=\wt{M_n}$, where $M_n$ is an $A_n$-module of finite type, and the hypotheses imply that $M_m=M_n\otimes_{A_n}A_m$ for $m\leq n$ \sref{1.1.6.5}; the $M_n$ thus form a projective system for the canonical di-homomorphisms $M_n\to M_m$ ($m\leq n$), and it follows immediately from the definition of the $A_n$ that this projective system satisfies the conditions of \sref[0]{0.7.2.9}; as a result, its projective limit $M$ is an $A$-module of finite type such that $M_n=M\otimes_A A_n$ for all $n$.
We deduce that $\sh{F}_n$ is induced over $X_n$ by $\wt{M}\otimes_{\OO_X}(\OO_X/\wt{\mathfrak{J}}^{n+1})$, so $\sh{F}=M^\Delta$ by Definition \sref{1.10.8.4}.

Conversely, (c) implies (b); indeed, if $u_n$ is the immersion morphism $X_n\to X$, then $u_n^*(\wt{M})=(M\otimes_A A_n)^\sim$ is induced over $X_n$ by $\wt{M}\otimes_{\OO_X}(\OO_X/\wt{\mathfrak{J}}^{n+1})$, and $M^\Delta=\varprojlim u_n^*(\wt{M})$ by Definition \sref{1.10.8.4}; as $u_m=u_n\circ u_{mn}$ for $m\leq n$, the $\sh{F}_n=u_n^*(\wt{M})$ satisfy the conditions of (b), hence our assertion.

We now show that (c) implies (a): indeed, we have by definition that $\OO_\mathfrak{X}=A^\Delta$; as $M$ is the cokernel of a homomorphism $A^m\to A^n$, it follows from Proposition \sref{1.10.10.2} that $M^\Delta$ is the cokernel of a homomorphism $\OO_\mathfrak{X}^m\to\OO_\mathfrak{X}^n$, and as the sheaf of rings $\OO_\mathfrak{X}$ is coherent \sref{1.10.10.3}, so is $M^\Delta$ \sref[0]{0.5.3.4}.

\oldpage[I]{203}
Finally, (a) implies (b).
Considered as an $\OO_\mathfrak{X}$-module, we have that $\OO_{X_n}=\OO_\mathfrak{X}/\sh{J}^{n+1}=A_n^\Delta$; $\sh{F}_n=\sh{F}\otimes_{\OO_\mathfrak{X}}\OO_{X_n}$ is a coherent $\OO_\mathfrak{X}$-module \sref[0]{0.5.3.5}, and as it is also an $\OO_{X_n}$-modules and $\sh{J}^{n+1}$ is coherent, we conclude that $\sh{F}_n$ is a coherent $\OO_{X_n}$-module \sref[0]{0.5.3.10}, and it is imediate that $u_{mn}^*(\sh{F}_n)=\sh{F}_m$ for $m\leq n$ (recalling that the continuous map $X_m\to X_n$ of the underlying spaces in the identity on $\mathfrak{X}$).
The sheaf $\sh{G}=\varprojlim\sh{F}_n$ is thus a coherent $\OO_\mathfrak{X}$-module, since we have seen that (b) implies (a). The canonical homomorphisms $\sh{F}\to\sh{F}_n$ form a projective system, which by passing to the limit gives a canonical homomorphism $w:\sh{F}\to\sh{G}$, and it remains to prove that $w$ is bijective.
The question is now \emph{local}, so we can reduce to the case where $\sh{F}$ is the cokernel of a homomorphism $\OO_\mathfrak{X}^p\to\OO_\mathfrak{X}^q$; as this homomorphism is of the form $v^\Delta$, where $v$ is a homomorphism $A^m\to A^n$ \sref{1.10.10.2}, $\sh{F}$ is isomorphic to $M^\Delta$, where $M=\Coker{v}$ \sref{1.10.10.2}.
We then have by Proposition \sref{1.10.10.2} that $\sh{F}_n=M^\Delta\otimes_{\OO_\mathfrak{X}}A_n^\Delta=(M\otimes_A A_n)^\Delta$, and as the $\mathfrak{J}$-adic topology on $M\otimes_A A_n$ is discrete, we have $(M\otimes_A A_n)^\Delta=(M\otimes_A A_n)^\sim$ (as an $\OO_{X_n}$-module); we have seen above that $M^\Delta=\varprojlim\sh{F}_n$, and $w$ is thus the identity in this case.
Q.E.D.
\end{proof}

\begin{cor}[10.10.6]
\label{1.10.10.6}
If $\sh{F}$ satisfies condition \emph{(b)} of Proposition \sref{1.10.10.5}, then the projective system $(\sh{F}_n)$ is isomorphic to the system of the $\sh{F}\otimes_{\OO_\mathfrak{X}}\OO_{X_n}$.
\end{cor}

\begin{env}[10.10.7]
\label{1.10.10.7}
Now let $A$ and $B$ be two adic Noetherian rings, $\vphi:B\to A$ a continuous homomorphism; we denote by $\mathfrak{J}$ (resp. $\mathfrak{K}$) an ideal of definition for $A$ (resp. $B$) such that $\vphi(\mathfrak{K})\subset\mathfrak{J}$, and we set $X=\Spec(A)$, $Y=\Spec(B)$, $\mathfrak{X}=\Spf(A)$, and $\mathfrak{Y}=\Spf(B)$.
Let $f:X\to Y$ be the morphism of preschemes corresponding to $\vphi$ \sref{1.1.6.1}, $\wh{f}:\mathfrak{X}\to\mathfrak{Y}$ its extension to the completions \sref{1.10.9.1}, which is also a morphism of formal preschemes corresponding to $\vphi$ \sref{1.10.2.2}.
\end{env}

\begin{prop}[10.10.8]
\label{1.10.10.8}
For every $B$-module $N$ of finite type, there exists a canonical functorial isomorphism of $\OO_\mathfrak{X}$-modules
\[
  \wh{f}^*(N^\Delta)\isoto(N\otimes_B A)^\Delta.
\]
\end{prop}

\begin{proof}
\label{proof-1.10.10.8}
Denoting by $i_X:\mathfrak{X}\to X$ and $i_Y:\mathfrak{Y}\to Y$ the canonical morphisms, we have \sref{1.10.8.8}, up to canonical functorial isomorphisms, $N^\Delta=i_Y^*(\wt{N})$ and
\[
  (N\otimes_B A)^\Delta=i_X^*((N\otimes_B A)^\sim)=i_X^*(f^*(\wt{N}))
\]
\sref{1.1.6.5}; the proposition then follows from the commutativity of the diagram in \sref{1.10.9.2}.
\end{proof}

\begin{cor}
\label{1.10.10.9}
For every ideal $\mathfrak{b}$ of $B$, we have $\wh{f}^*(\mathfrak{b}^\Delta)\OO_\mathfrak{X}=(\mathfrak{b}A)^\Delta$.
\end{cor}

\begin{proof}
\label{proof-1.10.10.9}
Let $j$ be the canonical injection $\mathfrak{b}\to B$, to which corresponds the canonical injection $j^\Delta:\mathfrak{b}^\Delta\to\OO_\mathfrak{Y}$ of sheaves of $\OO_\mathfrak{Y}$-modules; by definition, $\wh{f}^*(\mathfrak{b}^\Delta)\OO_\mathfrak{X}$ is the image of the homomorphism $\wh{f}^*(j^\Delta):\wh{f}^*(\mathfrak{b}^\Delta)\to\OO_\mathfrak{X}=\wh{f}^*(\OO_\mathfrak{Y})$; but this homomorphism identifies with $(j\otimes 1)^\Delta:(\mathfrak{b}\otimes_B A)^\Delta\to\OO_\mathfrak{X}=(B\otimes_B A)^\Delta$ by Proposition \sref{1.10.10.8}.
As the image of $j\otimes 1$ is the ideal $\mathfrak{b}A$ of $A$, the image of $(j\otimes 1)^\Delta$ is thus $(\mathfrak{b}A)^\Delta$ by Proposition \sref{1.10.10.2}, hence the conclusion.
\end{proof}

\subsection{Coherent sheaves on formal preschemes}
\label{subsection:1.10.11}

\begin{prop}[10.11.1]
\label{1.10.11.1}
\oldpage[I]{204}
If $\mathfrak{X}$ is a locally Noetherian formal prescheme, then the sheaf of rings $\OO_\mathfrak{X}$ is coherent and every sheaf of ideals of definition for $\mathfrak{X}$ is coherent.
\end{prop}

\begin{proof}
\label{proof-1.10.11.1}
The question is local, so we can reduce to the case of a Noetherian affine formal scheme, and the proposition follows from Propositions \sref{1.10.10.3} and \sref{1.10.10.5}.
\end{proof}

\begin{env}[10.11.2]
\label{1.10.11.2}
Let $\mathfrak{X}$ be a locally Noetherian formal prescheme, $\sh{J}$ a sheaf of ideals of definition for $\mathfrak{X}$, and $X_n$ the locally Noetherian (usual) prescheme $(\mathfrak{X},\OO_\mathfrak{X}/\sh{J}^{n+1})$, such that $\mathfrak{X}$ is the \emph{inductive limit} of the sequence $(X_n)$ with respect to the canonical morphisms $u_{mn}:X_m\to X_n$ \sref{1.10.6.3}.
With this notation:
\end{env}

\begin{thm}[10.11.3]
\label{1.10.11.3}
For an $\OO_\mathfrak{X}$-module $\sh{F}$ to be coherent, it is necessary and sufficient for it to be isomorphic to a projective limit of a sequence $(\sh{F}_n)$, where $\sh{F}_n$ is a coherent $\OO_{X_n}$-module such that $u_{mn}^*(\sh{F}_n)=\sh{F}_m$ for $m\leq n$ \sref{1.10.6.6}.
The projective system $(\sh{F}_n)$ is then isomorphic to the system of the $u_n^*(\sh{F})=\sh{F}\otimes_{\OO_\mathfrak{X}}\OO_{X_n}$, where $u_n$ is the canonical morphism $X_n\to\mathfrak{X}$.
\end{thm}

\begin{proof}
\label{proof-1.10.11.3}
The question is local, so we can reduce to the case where $\mathfrak{X}$ is a Noetherian affine formal scheme, and the theorem then is a consequence of Proposition \sref{1.10.10.5} and Corollary \sref{1.10.10.6}.
\end{proof}

We can thus say that \emph{the data of a coherent $\OO_\mathfrak{X}$-module is equivalent to the data of a projective system $(\sh{F}_n)$ of coherent $\OO_{X_n}$-modules such that $u_{mn}(\sh{F}_n)=\sh{F}_m$ for $m\leq n$}.

\begin{cor}[10.11.4]
\label{1.10.11.4}
If $\sh{F}$ and $\sh{G}$ are two coherent $\OO_\mathfrak{X}$-modules, then we can (with the notation of Theorem \sref{1.10.11.3}) define a canonical functorial isomorphism
\[
  \Hom_{\OO_\mathfrak{X}}(\sh{F},\sh{G})\isoto\varprojlim_n\Hom_{\OO_{X_n}}(\sh{F}_n,\sh{G}_n).
  \tag{10.11.4.1}
\]
\end{cor}

\begin{proof}
\label{proof-1.10.11.4}
The projective limit in the right hand side is understood to be with respect to the maps $\theta_n\mapsto u_{mn}^*(\theta_n)$ ($m\leq n$) from $\Hom_{\OO_{X_n}}(\sh{F}_n,\sh{G}_n)$ to $\Hom_{\OO_{X_m}}(\sh{F}_m,\sh{F}_m)$.
The homomorphism (10.11.4.1) sends an element $\theta\in\Hom_{\OO_\mathfrak{X}}(\sh{F},\sh{G})$ to the sequence $(u_n^*(\theta))$; we see that we have defined an inverse homomorphism of the above by sending a projective system $(\theta_n)\in\varprojlim_n\Hom_{\OO_{X_n}}(\sh{F}_n,\sh{G}_n)$ to its projective limit in $\Hom_{\OO_\mathfrak{X}}(\sh{F},\sh{G})$, taking into account Theorem \sref{1.10.11.3}.
\end{proof}

\begin{cor}[10.11.5]
\label{1.10.11.5}
For a homomorphism $\theta:\sh{F}\to\sh{G}$ to be surjective, it is necessary and sufficient for the corresponding homomorphism $\theta_0=u_0^(\theta):\sh{F}_0\to\sh{G}_0$ to be surjective.
\end{cor}

\begin{proof}
\label{proof-1.10.11.5}
The question is local, so we reduce to the case where $\mathfrak{X}=\Spf(A)$, where $A$ is an adic Noetherian ring, $\sh{F}=M^\Delta$, $\sh{G}=N^\Delta$, and $\theta=u^\Delta$, where $M$ and $N$ are two $A$-modules of finite type and $u$ is a homomorphism $M\to N$; we then have that $\theta_0=\wt{u_0}$, where $u_0$ is the homomorphism $u\otimes 1:M\otimes_A A/\mathfrak{J}\to N\otimes_A A/\mathfrak{J}$; the conclusion follows from the fact that $u$ and $u_0$ are simultaneously surjective \sref[0]{0.7.1.14}.
\end{proof}

\begin{env}[10.11.6]
\label{1.10.11.6}
Theorem \sref{1.10.11.3} shows that we can consider every coherent $\OO_\mathfrak{X}$-module $\sh{F}$ as a \emph{topological $\OO_\mathfrak{X}$-module}, considering it as a projective limit of \emph{pseudo-discrete} sheaves of groups $\sh{F}_n$ \sref[0]{0.3.8.1}.
It then follows from Corollary \sref{1.10.11.4} that every homomorphism $u:\sh{F}\to\sh{G}$ of coherent $\OO_\mathfrak{X}$-modules is automatically \emph{continuous}
\oldpage[I]{205}
\sref[0]{0.3.8.2}.
In addition, if $\sh{H}$ is a coherent $\OO_\mathfrak{X}$-submodule of a coherent $\OO_\mathfrak{X}$-module $\sh{F}$, then for every open $U\subset\mathfrak{X}$, $\Gamma(U,\sh{H})$ is a \emph{closed} subgroup of the topological group $\Gamma(U,\sh{F})$, since the functor $\Gamma$ is left exact, $\Gamma(U,\sh{H})$ the kernel of the homomorphism $\Gamma(U,\sh{F})\to\Gamma(U,\sh{F}/\sh{H})$, which is \emph{continuous} by the above, since $\sh{F}/\sh{G}$ is coherent \sref[0]{0.5.3.4}; our assertion follows from the fact that $\Gamma(U,\sh{F}/\sh{H})$ is a separated topological group.
\end{env}

\begin{prop}[10.11.7]
\label{1.10.11.7}
Let $\sh{F}$ and $\sh{G}$ be two coherent $\OO_\mathfrak{X}$-modules.
We can define (with the notation of Theorem \sref{1.10.11.3}) two canonical functorial isomorphisms of topological $\OO_\mathfrak{X}$-modules \sref{1.10.11.6}
\[
  \sh{F}\otimes_{\OO_\mathfrak{X}}\sh{G}\isoto\varprojlim_n(\sh{F}_n\otimes_{\OO_{X_n}}\sh{G}_n),
  \tag{10.11.7.1}
\]
\[
  \shHom_{\OO_\mathfrak{X}}(\sh{F},\sh{G})\isoto\varprojlim_n\shHom_{\OO_{X_n}}(\sh{F}_n,\sh{G}_n).
  \tag{10.11.7.2}
\]
\end{prop}

\begin{proof}
\label{proof-1.10.11.7}
The existence of the isomorphism (10.11.7.1) follows from the formula
\[
  \sh{F}_n\otimes_{\OO_{X_n}}\sh{G}_n=(\sh{F}\otimes_{\OO_\mathfrak{X}}\OO_{X_n})\otimes_{\OO_{X_n}}(\sh{G}\otimes_{\OO_\mathfrak{X}}\OO_{X_n})=(\sh{F}\otimes_{\OO_\mathfrak{X}}\sh{G})\otimes_{\OO_\mathfrak{X}}\OO_{X_n}
\]
and from Theorem \sref{1.10.11.3}.
The isomorphism (10.11.7.2) where the two sodes are considered as sheaves of modules without topology, follows from the definition of the sections of $\shHom_{\OO_\mathfrak{X}}(\sh{F},\sh{G})$ and $\shHom_{\OO_{X_n}}(\sh{F}_n,\sh{G}_n)$ and the existence of the isomorphism (10.11.4.1), mapping a prescheme induced on an arbitrary Noetherian formal affine open set to $\mathfrak{X}$.
It rwmains to prove that the isomorphism (10.11.7.2) is bicontinuous over a quasi-compact set, and we can thus reduce to the case where $\mathfrak{X}=\Spf(A)$, $A$ an adic Noetherian ring, hence \sref{1.10.10.5} $\sh{F}=M^\Delta$ and $\sh{G}=N^\Delta$, where $M$ and $N$ are $A$-modules of finite type; taking into account (10.10.2.1), (10.10.2.3), and Corollary \sref{1.1.3.12}[ii], we reduce to showing that the canonical isomorphism $\Hom_A(M,N)\isoto\varprojlim_n\Hom_{A_n}(M_n,N_n)$ (with $M_n=M\otimes_A A_n$ and $N_n=N\otimes_A A_n$) is continuous, which has been proved in \sref[0]{0.7.8.2}.
\end{proof}

\begin{env}[10.11.8]
\label{1.10.11.8}
As $\Hom_{\OO_\mathfrak{X}}(\sh{F},\sh{G})$ is the group of sections of the sheaf of topological groups $\shHom_{\OO_\mathfrak{X}}(\sh{F},\sh{G})$, it is equipped with a group topology.
If $\mathfrak{X}$ is \emph{Noetherian}, then it follows from (10.11.7.2) that the subgroups $\Hom_{\OO_\mathfrak{X}}(\sh{F},\sh{J}^n\sh{G})$ ($n$ arbitrary) form a fundamental system of neighborhoods of $0$ in this group.
\end{env}

\begin{prop}[10.11.9]
\label{1.10.11.9}
Let $\mathfrak{X}$ be a Noetherian formal prescheme, $\sh{F}$ and $\sh{G}$ two coherent $\OO_\mathfrak{X}$-modules.
In the topological group $\Hom_{\OO_\mathfrak{X}}(\sh{F},\sh{G})$, the surjective (resp. injective, bijective) homomorphisms form an open set.
\end{prop}

\begin{proof}
\label{proof-1.10.11.9}
By Corollary \sref{1.10.11.5}, the set of surjective homomorphisms in $\Hom_{\OO_\mathfrak{X}}(\sh{F},\sh{G})$ is the inverse image under the continuous map $\Hom_{\OO_\mathfrak{X}}(\sh{F},\sh{G})\to\Hom_{\OO_{X_0}}(\sh{F}_0,\sh{G}_0)$ of a subset of the discrete group $\Hom_{\OO_{X_0}}(\sh{F}_0,\sh{G}_0)$, hence the first assertion.
To show the second, we cover $\mathfrak{X}$ by a finite number of Noetherian formal affine subsets $U_i$.
For $\theta\in\Hom_{\OO_\mathfrak{X}}(\sh{F},\sh{G})$ to be injective, it is necessary and sufficient for all of the images under the (continuous) restriction maps $\Hom_{\OO_\mathfrak{X}}(\sh{F},\sh{G})\to\Hom_{\OO_\mathfrak{X}|U_i)}(\sh{F}|U_i,\sh{G}|U_i)$ to be injective; we can thus reduce to the affine case, and then this has already been proved in \sref[0]{0.7.8.3}.
\end{proof}

\subsection{Adic morphisms of formal preschemes}
\label{subsection:1.10.12}

\begin{env}[10.12.1]
\label{1.10.12.1}
\oldpage[I]{206}
Let $\mathfrak{X}$ and $\mathfrak{S}$ be two \emph{locally Noetherian} formal preschemes;
we say that a morphism $f:\mathfrak{X}\to\mathfrak{S}$ is \emph{adic} if there exists an ideal of definition $\sh{J}$ of $\mathfrak{S}$ such that $\sh{K}=f^*(\sh{J})\OO_\mathfrak{X}$ is an ideal of definition of $\mathfrak{X}$;
we then also say that $\mathfrak{X}$ is an \emph{adic $\mathfrak{S}$-prescheme} (for $f$).
Whenever this is the case, for \emph{every} ideal of definition $\sh{J}_1$ of $\mathfrak{S}$, $\sh{K}_1=f^*(\sh{J}_1)\OO_\mathfrak{X}$ is an ideal of definition of $\mathfrak{X}$.
Indeed, the question being local, we can assume that $\mathfrak{X}$ and $\mathfrak{S}$ are Noetherian and affine;
there then exists a whole number $n$ such that $\sh{J}^n\subset\sh{J}_1$ and $\sh{J}_1^n\subset\sh{J}$ (\sref{1.10.3.6} and \sref[0]{0.7.1.4}), whence $\sh{K}^n\subset\sh{K}_1$ and $\sh{K}_1^n\subset\sh{K}$.
The first of these equalities shows that $\sh{K}_1=\mathfrak{K}_1^\Delta$, where $\mathfrak{K}_1$ is an open ideal of $A=\Gamma(\mathfrak{X},\OO_\mathfrak{X})$, and the second shows that $\mathfrak{K}_1$ is an ideal of definition of $A$ \sref[0]{0.7.1.4}, whence our claim.
\end{env}

It follows immediately from the above that, if $\mathfrak{X}$ and $\mathfrak{Y}$ are adic $\mathfrak{S}$-preschemes, then \emph{every $\mathfrak{S}$-morphism $u:\mathfrak{X}\to\mathfrak{Y}$ is adic}:
indeed, if $f:\mathfrak{X}\to\mathfrak{S}$ and $g:\mathfrak{Y}\to\mathfrak{S}$ are the structure morphisms, and $\sh{J}$ is an ideal of definition of $\mathfrak{S}$, then we have $f=g\circ u$, and so $u^*(g^*(\sh{J})\OO_\mathfrak{Y})\OO_\mathfrak{X}=f^*(\sh{J})\OO_\mathfrak{X}$ is an ideal of definition of $\mathfrak{X}$, and, by hypothesis, $g^*(\sh{J})\OO_\mathfrak{Y}$ is an ideal of definition of $\mathfrak{Y}$.

\begin{env}[10.12.2]
\label{1.10.12.2}
In what follows, we suppose that we have some fixed locally Noetherian formal prescheme $\mathfrak{S}$, and some ideal of definition $\sh{J}$ of $\mathfrak{S}$;
we set $S_n=(\mathfrak{S},\OO_\mathfrak{S}/\sh{J}^{n+1})$.
The (locally Noetherian) adic $\mathfrak{S}$-preschemes clearly form a \emph{category}.
We say that an inductive system $(X_n)$ of locally Noetherian (usual) $S_n$-preschemes is an \emph{adic inductive $(S_n)$-system} if the structure morphisms $f_n:X_n\to S_n$ are such that, for $m\leq n$, the diagrams
\[
  \xymatrix{
    X_n \ar[d]_{f_n}
    & X_m \ar[l] \ar[d]^{f_m}\\
    S_n
    & S_m \ar[l]
  }
\]
commute and \emph{identify $X_m$ with the product $X_n\times_{S_n}S_m=(X_n)_{(S_m)}$}.
The adic inductive systems form a \emph{category}:
it suffices in fact to define a morphism $(X_n)\to(Y_n)$ of such systems to be an \emph{inductive system of $S_n$-morphisms $u_n:X_n\to Y_n$} such that $u_m$ is identified with $(u_n)_{(S_m)}$ for $m\leq n$.
With this in mind:
\end{env}

\begin{thm}[10.12.3]
\label{1.10.12.3}
There is a canonical equivalence between the category of adic $\mathfrak{S}$-preschemes and the category of adic inductive $(S_n)$-systems.
\end{thm}

The equivalence in question is obtained in the following way:
if $\mathfrak{X}$ is an adic $\mathfrak{S}$-prescheme, and $f:\mathfrak{X}\to\mathfrak{S}$ is the structure morphism, then $\sh{K}=f^*(\sh{J})\OO_\mathfrak{X}$ is an ideal of definition of $\mathfrak{X}$, and we associate to $\mathfrak{X}$ the inductive system of the $X_n=(\mathfrak{X},\OO_\mathfrak{X}/\sh{K}^{n+1})$, with the stucure morphism $f_n:X_n\to S_n$ corresponding to $f$ \sref{1.10.5.6}.
We first show that $(X_n)$ is an \emph{adic inductive system}:
if $f=(\psi,\theta)$, we have $\psi^*(\sh{J})\OO_\mathfrak{X}=\sh{K}$, so $\psi^*(\sh{J}^n)\OO_\mathfrak{X}=\sh{K}^n$ for all $n$, and (by exactness of the functor $\psi^*$) $\sh{K}^{m+1}/\sh{K}^{n+1}=\psi^*(\sh{J}^{m+1}/\sh{J}^{n+1})(\OO_\mathfrak{X}/\sh{K}^{n+1})$ for $m\leq n$;
our conclusion thus follows from \sref{1.4.4.5}.
Furthermore, it can be immediately verified that a $\mathfrak{S}$-morphism $u:\mathfrak{X}\to\mathfrak{Y}$ of adic $\mathfrak{S}$-preschemes corresponds (with the obvious notation)
\oldpage[I]{207}
to an inductive system of $S_n$-morphisms $u_n:X_n\to Y_n$ such that $u_m$ is identified with $(u_n)_{(S_m)}$ for $m\leq n$.

The fact that this equivalence is well defined will follow from the more-precise following proposition.

\begin{prop}[10.12.3.1]
\label{1.10.12.3.1}
Let $(X_n)$ be an inductive system of $S_n$-preschemes;
suppose that the structure morphisms $f_n:X_n\to S_n$ are such that the diagrams in \hyperref[1.10.12.2]{(10.12.2.1)} commute and identify $X_m$ with $X_n\times_{S_n}S_m$ for $m\leq n$.
Then the inductive system $(X_n)$ satisfies conditions \emph{(b)} and \emph{(c)} of \sref{1.10.6.3};
let $\mathfrak{X}$ be the inductive limit, and $f:\mathfrak{X}\to\mathfrak{S}$ the morphism given by the inductive limit of the inductive system $(f_n)$.
Then, if $X_0$ is locally Noetherian, $\mathfrak{X}$ is locally Noetherian, and $f$ is an adic morphism.
\end{prop}

\begin{proof}
\label{proof-1.10.12.3.1}
Since the sheaf of ideals of $\OO_{S_n}$ that defines the subprescheme $S_m$ of $S_n$ is nilpotent, by \sref{1.4.4.5} so is the sheaf of ideals of $\OO_{X_n}$ that defines the subprescheme $X_m$ of $X_n$, so the conditions of \sref{1.10.6.3} are satisfied.
The question being local on $\mathfrak{X}$ and $\mathfrak{S}$, we can assume that $\mathfrak{S}=\Spf(A)$, $\sh{J}=\mathfrak{J}^\Delta$ (with $A$ a Noetherian $\mathfrak{J}$-adic ring), and $X_n=\Spec(B_n)$;
if $A_n=A/\mathfrak{J}^{n+1}$, then the hypothesis implies that $B_0$ is Noetherian, and if we set $\mathfrak{J}_n=\mathfrak{J}/\mathfrak{J}^{n+1}$, then $B_m=B_n/\mathfrak{J}_n^{m+1}B_n$.
The kernel of $B_n\to B_0$ is thus $\mathfrak{K}_n=\mathfrak{J}_nB_n$, and the kernel of $B_n\to B_m$ is $\mathfrak{K}_n^{m+1}$ for $m\leq n$;
further, since $A_1$ is Noetherian, $\mathfrak{J}_1$ is of finite type over $A_1$, and so $\mathfrak{K}_1=\mathfrak{K}_1/\mathfrak{K}_1^2$ is of finite type over $B_1$, and \emph{a fortiori} of finite type over $B_0=B_1/\mathfrak{K}_1$;
the fact that $\mathfrak{X}$ is Noetherian then follows from \sref{1.10.6.4};
if $B=\varprojlim B_n$, then we have $\mathfrak{X}=\Spf(B)$, and if $\mathfrak{K}$ is the kernel of $B\to B_0$, then $B_n=B/\mathfrak{K}^{n+1}$.
If $\rho_n:A/\mathfrak{J}^{n+1}\to B/\mathfrak{K}^{n+1}$ is the homomorphism corresponding to $f_n$, then we have that
\[
    \mathfrak{K}/\mathfrak{K}^{n+1} = (B/\mathfrak{K}^{n+1})\rho_n(\mathfrak{J}/\mathfrak{J}^{n+1})
\]
since the homomorphism $\rho:A\to B$ corresponding to $f$ is equal to $\varprojlim\rho_n$, and that the ideal $\mathfrak{J}B$ of $B$ is dense in $\mathfrak{K}$, and since every ideal of $B$ is closed \sref[0]{0.7.3.5}, we also have that $\mathfrak{K}=\mathfrak{J}B$.
If $\sh{K}=\mathfrak{K}^\Delta$, the equality $f^*(\sh{J})\OO_\mathfrak{X}=\sh{K}$ then follows from \sref{1.10.10.9}, and finishes the proof.
\end{proof}

\begin{env}[10.12.3.2]
\label{1.10.12.3.2}
The above equivalence gives, for adic $\mathfrak{S}$-preschemes $\mathfrak{X}$ and $\mathfrak{Y}$, a \emph{canonical bijection}
\[
  \Hom_\mathfrak{S}(\mathfrak{X},\mathfrak{Y})\isoto\varprojlim_n\Hom_{S_n}(X_n,Y_n)
\]
where the projective limit is relative to the maps $u_n\to(u_n)_{(S_m)}$ for $m\leq n$.
\end{env}

\subsection{Morphisms of finite type}
\label{subsection:1.10.13}

\begin{prop}[10.13.1]
\label{1.10.13.1}
Let $\mathfrak{Y}$ be a locally Noetherian formal prescheme, $\sh{K}$ an ideal of definition of $\mathfrak{Y}$, and $f:\mathfrak{X}\to\mathfrak{Y}$ a morphism of formal preschemes.
Then the following conditions are equivalent.
\begin{enumerate}[label=\emph{(\alph*)}]
  \item $X$ is locally Noetherian, $f$ is an adic morphism \sref{1.10.12.1}, and, if we set $\sh{J}=f^*(\sh{K})\OO_\mathfrak{X}$, the morphism $f_0:(\mathfrak{X},\OO_\mathfrak{X}/\sh{J})\to(\mathfrak{Y},\OO_\mathfrak{Y}/\sh{K})$ induced by $f$ is of finite type.
  \item $\mathfrak{X}$ is locally Noetherian, and is the inductive limit of an adic inductive $(Y_n)$-system $(X_n)$ such that the morphism $X_0\to Y_0$ is of finite type.
\oldpage[I]{208}
  \item Every point of $\mathfrak{Y}$ has a Noetherian formal affine open neighbourhood $V$ that has the following property:
\begin{enumerate}
  \item[(\textbf{Q})] $f^{-1}(V)$ is a finite union of Noetherian formal affine open subsets $U_i$ such that the Noetherian adic ring $\Gamma(U_i,\OO_\mathfrak{X})$ is topologically isomorphic to the quotient of a formal series algebra, restricted \sref[0]{0.7.5.1} to $\Gamma(V,\OO_\mathfrak{Y})$, by an ideal (necessarily closed).
\end{enumerate}
\end{enumerate}
\end{prop}

\begin{proof}
\label{proof-1.10.13.1}
It is immediate that (a) implies (b) by \sref{1.10.12.3}.
To show that (b) implies (c), we can, since the question is local on $\mathfrak{Y}$, assume that $\mathfrak{Y}=\Spf(B)$, where $B$ is Noetherian and adic;
let $\sh{K}=\mathfrak{K}^\Delta$, with $\mathfrak{K}$ an ideal of definition of $B$.
Since, by hypothesis, $X_0$ is of finite type over $Y_0$, $X_0$ is a finite union of affine open subsets $U_i$ such that the ring $A_{i0}$ of the affine scheme induced by $X_0$ on $U_i$ is an algebra of finite type over the ring $B/\mathfrak{K}$ of $Y_0$ \sref{1.6.3.2}.
By \sref{1.5.1.9}, $U_i$ is also an affine open subset in each of the Noetherian preschemes $X_n$, and if $A_{in}$ is the ring of the affine scheme induced by $X_n$ on $U_i$, then hypothesis (b) implies that, for $m\leq n$, $A_{im}$ is isomorphic to $A_{in}/\mathfrak{K}^{m+1}A_{in}$.
Consequently, the formal prescheme induced by $\mathfrak{X}$ on $U_i$ is isomorphic to $\Spf(A_i)$, where $A_i=\varprojlim_n A_{in}$ \sref{1.10.6.4};
$A_i$ is a $\mathfrak{K}A_i$-adic ring, and $A_i/\mathfrak{K}A_i$, isomorphic to $A_{i0}$, is an algebra of finite type over $B/\mathfrak{K}$.
We thus conclude \sref[0]{0.7.5.5} that $A_i$ is topologically isomorphic to a quotient of a formal series algebra restricted to $B$ (by a necessarily-closed ideal, because such an algebra is Noetherian \sref[0]{0.7.5.4}).

To show that (c) implies (a), we can restrict to the case where $\mathfrak{X}=\Spf(A)$ is also affine, with $A$ a Noetherian adic ring that is isomorphic to a quotient of a formal series algebra, restricted to $B$, by a closed ideal.
Then \sref[0]{0.7.5.5} $A/\mathfrak{K}A$ is an algebra of finite type over $B/\mathfrak{K}$, and $\mathfrak{K}A=\mathfrak{J}$ is an ideal of definition of $A$, and so, by \sref{1.10.10.9}, the conditions of (a) are satisfied.
\end{proof}

We note that, if the conditions of Proposition \sref{1.10.13.1} are satisfied, then property~(a) holds true for \emph{any} ideal of definition $\sh{K}$ of $\mathfrak{Y}$ (by \emph{(c)}), and so, in property~(b), \emph{all} the $f_n$ are morphisms of finite type.

\begin{cor}[10.13.2]
\label{1.10.13.2}
If the conditions of \sref{1.10.13.1} are satisfied, then every Noetherian formal affine open subset $V$ of $\mathfrak{Y}$ has property~(\textbf{Q}), and if $\mathfrak{Y}$ is Noetherian, then so is $\mathfrak{X}$.
\end{cor}

\begin{proof}
\label{proof-1.10.13.2}
This follows immediately from \sref{1.10.13.1} and \sref{1.6.3.2}.
\end{proof}

\begin{defn}[10.13.3]
\label{1.10.13.3}
When the equivalent properties (a), (b), and (c) of \sref{1.10.13.1} are satisfied, we say that the morphism $f$ is of finite type, or that $\mathfrak{X}$ is a formal $\mathfrak{Y}$-prescheme of finite type, or a formal prescheme of finite type over $\mathfrak{Y}$.
\end{defn}

\begin{cor}[10.13.4]
\label{1.10.13.4}
Let $\mathfrak{X}=\Spf(A)$ and $\mathfrak{Y}=\Spf(B)$ be Noetherian formal affine schemes;
for $\mathfrak{X}$ to be of finite type over $\mathfrak{Y}$, it is necessary and sufficient for the Noetherian adic ring $A$ to be isomorphic to the quotient of a formal series algebra, restricted to $B$, by some closed ideal.
\end{cor}

\begin{proof}
\label{proof-1.10.13.4}
With the notation of \sref{1.10.13.1}, if $\mathfrak{X}$ is of finite type over $\mathfrak{Y}$, then $A/\mathfrak{K}A$ is a $(B/\mathfrak{K})$-algebra of finite type by \sref{1.6.3.3}, and $\mathfrak{K}A$ is an ideal of definition of $A$ \sref{1.10.10.9}.
We are then done, by \sref[0]{0.7.5.5}.
\end{proof}

\begin{prop}[10.13.5]
\medskip\noindent
\begin{enumerate}[label=\emph{(\roman*)}]
    \item The composition of any two morphisms (of formal preschemes) of finite type is again of finite type.
\oldpage[I]{209}
    \item Let $\mathfrak{X}$, $\mathfrak{S}$, and $\mathfrak{S}'$ be locally Noetherian (resp. Noetherian) formal preschemes, and $f:\mathfrak{X}\to\mathfrak{S}$ and $\mathfrak{X}\to\mathfrak{S}'$ morphisms.
        If $f$ is of finite type, then $\mathfrak{X}\times_\mathfrak{S}\mathfrak{S}'$ is locally Noetherian (resp. Noetherian) and of finite type over $\mathfrak{S}'$.
    \item Let $\mathfrak{S}$ be a locally Noetherian formal prescheme, and $\mathfrak{X}'$ and $\mathfrak{Y}'$ formal $\mathfrak{S}$-preschemes such that $\mathfrak{X}'\times_\mathfrak{S}\mathfrak{Y}'$ is locally Noetherian.
    If $\mathfrak{X}$ and $\mathfrak{Y}$ are locally Noetherian formal $\mathfrak{S}$-preschemes, and $f:\mathfrak{X}\to\mathfrak{X}'$ and $g:\mathfrak{Y}\to\mathfrak{Y}'$ are $\mathfrak{S}$-morphisms of finite type, then $\mathfrak{X}\times_\mathfrak{S}\mathfrak{Y}$ is locally Noetherian, and $f\times_\mathfrak{S}g$ is a $\mathfrak{S}$-morphism of finite type.
\end{enumerate}
\end{prop}

\begin{proof}
\label{proof-1.10.13.5}
By the formal argument of \sref{1.3.5.1}, (iii) follows from (i) and (ii), so it suffices to prove (i) and (ii).

Let $\mathfrak{X}$, $\mathfrak{Y}$, and $\mathfrak{Z}$ be locally Noetherian formal preschemes, and $f:\mathfrak{X}\to\mathfrak{Y}$ and $g:\mathfrak{Y}\to\mathfrak{Z}$ morphisms of finite type.
If $\sh{L}$ is an ideal of definition of $\mathfrak{Z}$, then $\sh{K}=g^*(\sh{L})\OO_\mathfrak{Y}$ is an ideal of definition of $\mathfrak{Y}$, and $\sh{J}=f^*(g^*(\sh{L}))\OO_\mathfrak{X}$ is an ideal of definition for $\mathfrak{X}$.
Let $X_0=(\mathfrak{X},\OO_\mathfrak{X}/\sh{J})$, $Y_0=(\mathfrak{Y},\OO_\mathfrak{Y}/\sh{K})$, and $Z_0=(\mathfrak{Z},\OO_\mathfrak{Z}/\sh{L})$, and let $f_0:X_0\to Y_0$ and $g_0:Y_0\to Z_0$ be the morphisms corresponding to $f$ and $g$ (respectively).
Since, by hypothesis, $f_0$ and $g_0$ are of finite type, so is $g_0\circ f_0$ \sref{1.6.3.4}, which corresponds to $g\circ f$;
thus $g\circ f$ is of finite type, by \sref{1.10.13.1}.

Under the conditions of (ii), $\mathfrak{S}$ (resp. $\mathfrak{X}$, $\mathfrak{S}'$) is the inductive limit of a sequence $(S_n)$ (resp. $(X_n)$, $(S'_n)$) of locally Noetherian preschemes, and we can assume \sref{1.10.13.1} that $X_m=X_n\times_{S_n}S_m$ for $m\leq n$.
The formal prescheme $\mathfrak{X}\times_\mathfrak{S}\mathfrak{S}'$ is then the inductive limit of the preschemes $X_n\times_{S_n}S'_n$ \sref{1.10.7.4}, and we have
\[
    X_m\times_{S_m}S'_m = (X_n\times_{S_n}S_m)\times_{S_m}S'_m = (X_n\times_{S_n}S'_n)\times_{S'_n}S'_m.
\]
Furthermore, $X_0\times_{S_0}S'_0$ is locally Noetherian because $X_0$ of finite type over $S_0$ \sref{1.6.3.8}.
We thus conclude \sref{1.10.12.3.1}, first of all, that $\mathfrak{X}\times_\mathfrak{S}\mathfrak{S}'$ is locally Noetherian;
then, since $X_0\times_{S_0}S'_0$ is of finite type over $S'_0$ \sref{1.6.3.8}, it follows from \hyperref[1.10.12.3]{(10.12.3.1)} and \sref{1.10.13.1} that $\mathfrak{X}\times_\mathfrak{S}\mathfrak{S}'$ is of finite type over $\mathfrak{S}'$, which proves (ii) (the claim about Noetherian preschemes being an immediate consequence of \sref{1.6.3.8}).
\end{proof}

\begin{cor}[10.13.6]
\label{1.10.13.6}
Under the hypotheses of \sref{1.10.9.9}, if $f$ is a morphism of finite type, then so is its extension $\wh{f}$ to the completions.
\end{cor}

\subsection{Closed subpreschemes of formal preschemes}
\label{subsection:1.10.14}

\begin{prop}[10.14.1]
\label{1.10.14.1}
Let $\mathfrak{X}$ be a locally Noetherian formal preschemes, and $\sh{A}$ a coherent sheaf of ideals of $\OO_\mathfrak{X}$.
If $\mathfrak{Y}$ if the (closed) support of $\OO_\mathfrak{X}/\sh{A}$, then the topologically ringed space $(\mathfrak{Y},(\OO_\mathfrak{X}/\sh{A})|\mathfrak{Y})$ is a locally Noetherian formal prescheme that is Noetherian if $\mathfrak{X}$ is.
\end{prop}

\begin{proof}
\label{proof-1.10.14.1}
Note that $\OO_\mathfrak{X}/\mathfrak{A}$ is coherent by \sref{1.10.10.3} and \sref[0]{0.5.3.4}, so its support $\mathfrak{Y}$ is closed \sref[0]{0.5.2.2}.
Let $\sh{J}$ be an ideal of definition of $\mathfrak{X}$, and let $X_n=(\mathfrak{X}/\OO_\mathfrak{X}/\sh{J}^{n+1})$;
the sheaf of rings $\OO_\mathfrak{X}/\sh{A}$ is the projective limit of the sheaves $\OO_\mathfrak{X}/(\sh{A}+\sh{J}^{n+1})=(\OO_\mathfrak{X}/\sh{A})\otimes_{\OO_\mathfrak{X}}(\OO_\mathfrak{X}/\sh{J}^{n+1})$ \sref{1.10.11.3}, which all have support $\mathfrak{Y}$.
The sheaf $(\sh{A}+\sh{J}^{n+1})/\sh{J}^{n+1}$ is a coherent $\OO_\mathfrak{X}$-module, since $\sh{J}^{n+1}$ is coherent, and so $(\sh{A}+\sh{J}^{n+1})/\sh{J}^{n+1}$ is also a coherent $(\OO_\mathfrak{X}/\sh{J}^{n+1})$-module \sref[0]{0.5.3.10};
if $Y_n$ is the closed subprescheme of $X_n$ defined by this sheaf of ideals, it is immediate that $(\mathfrak{Y},(\OO_\mathfrak{X}/\sh{A})|\mathfrak{Y})$
\oldpage[I]{210}
is the formal prescheme given by the inductive limit of the $Y_n$, and since the conditions of \sref{1.10.6.4} are satisfied, this proves that this formal prescheme is locally Noetherian, and Noetherian if $\mathfrak{X}$ is (since then $Y_0$ is, by \sref{1.6.1.4}).
\end{proof}

\begin{defn}[10.14.2]
\label{1.10.14.2}
We define a closed subprescheme of a formal prescheme $\mathfrak{X}$ to be any formal prescheme of the form $(\mathfrak{Y},(\OO_\mathfrak{X}/\sh{A})|\mathfrak{Y})$ with $\sh{A}$ a coherent $\OO_\mathfrak{X}$-module;
we say that this prescheme is the subprescheme defined by $\sh{A}$.
\end{defn}

It is clear that the correspondence thus defined between coherent $\OO_\mathfrak{X}$-modules and closed subpreschemes of $\mathfrak{X}$ is bijective.

The morphism of topologically ringed spaces $j=(\psi,\theta):\mathfrak{Y}\to\mathfrak{X}$, where $\psi$ is the injection $\mathfrak{Y}\to\mathfrak{X}$ and $\theta^\sharp$ the canonical homomorphism $\OO_\mathfrak{X}\to\OO_\mathfrak{X}/\sh{A}$, is evidently \sref{1.10.4.5} a morphism of formal preschemes, and we call it the \emph{canonical injection} from $\mathfrak{Y}$ to $\mathfrak{X}$.
Note that, if $\mathfrak{X}=\Spf(A)$, or if $A$ is Noetherian and adic, then we have $\sh{A}=\mathfrak{a}^\Delta$, where $\mathfrak{a}$ is an ideal of $A$ \sref{1.10.10.5}, and it immediately follows from the above that we then have $\mathfrak{Y}=\Spf(A/\mathfrak{a})$, up to isomorphism, and $j$ corresponds \sref{1.10.2.2} to the canonical homomorphism $A\to A/\mathfrak{a}$.

We say that a morphism $f:\mathfrak{Z}\to\mathfrak{X}$ of locally Noetherian formal preschemes is a \emph{closed immersion} if it factors as $\mathfrak{Z}\xrightarrow{g}\mathfrak{Y}\xrightarrow{j}\mathfrak{X}$, where $g$ is an isomorphism from $\mathfrak{Z}$ to a closed subprescheme $\mathfrak{Y}$ of $\mathfrak{X}$, and $j$ the canonical injection.
Since $j$ is a monomorphism of ringed spaces, $g$ and $\mathfrak{Y}$ are necessarily \emph{unique}.

\begin{prop}[10.14.3]
\label{1.10.14.3}
A closed immersion is a morphism of finite type.
\end{prop}

\begin{proof}
\label{proof-1.10.14.3}
We can immediately restrict to the case where $\mathfrak{X}$ is a formal affine scheme $\Spf(A)$ and $\mathfrak{Y}=\Spf(A/\mathfrak{a})$;
the proposition then follows from \sref{1.10.13.1}[\emph{c}].
\end{proof}

\begin{lem}[10.14.4]
\label{1.10.14.4}
Let $f:\mathfrak{Y}\to\mathfrak{X}$ be a morphism of locally Noetherian formal preschemes, and let $(U_\alpha)$ be a cover of $f(\mathfrak{Y})$ by Noetherian formal affine open subsets of $\mathfrak{X}$ such that the $f^{-1}(U_\alpha)$ are Noetherian formal affine open subsets of $\mathfrak{Y}$.
For $f$ to be a closed immersion, it is necessary and sufficient for $f(\mathfrak{Y})$ to be a closed subset of $\mathfrak{X}$ and, for all $\alpha$, for the restriction of $f$ to $f^{-1}(U_\alpha)$ to correspond \sref{1.10.4.6} to a surjective homomorphism $\Gamma(U_\alpha,\OO_\mathfrak{X})\to\Gamma(f^{-1}(U_\alpha),\OO_\mathfrak{Y})$.
\end{lem}

\begin{proof}
\label{proof-1.10.14.4}
The conditions are clearly necessary.
Conversely, if the conditions are satisfied, and if we denote by $\mathfrak{a}_\alpha$ the kernel of $\Gamma(U_\alpha,\OO_\mathfrak{X})\to\Gamma(f^{-1}(U_\alpha),\OO_\mathfrak{Y})$, we can define a coherent sheaf of ideals $\sh{A}$ of $\OO_\mathfrak{X}$ by setting $\sh{A}|U_\alpha=\mathfrak{a}_\alpha^\Delta$, and taking $\sh{A}$ to be zero on the complement of the union of the $U_\alpha$.
Because $f(\mathfrak{Y})$ is closed, and the support of $\mathfrak{a}_\alpha^\Delta$ is $U_\alpha\cap f(\mathfrak{Y})$, everything relies on proving that $\mathfrak{a}_\alpha^\Delta$ and $\mathfrak{a}_\beta^\Delta$ induce the same sheaf on any Noetherian formal affine open subset $V\subset U_\alpha\cap U_\beta$.
But the restriction of $f$ to $f^{-1}(U_\alpha)$ is a closed immersion of this formal prescheme into $U_\alpha$, $f^{-1}(V)$ is a Noetherian formal affine open subsets of $f^{-1}(U_\alpha)$, and the restriction of $f$ to $f^{-1}(V)$ is a closed immersion;
if $\mathfrak{b}$ is the kernel of the surjective homomorphism $\Gamma(V,\OO_\mathfrak{X})\to\Gamma(f^{-1}(V),\OO_\mathfrak{Y})$ corresponding to this restriction, then it is immediate \sref{1.10.10.2} that $\mathfrak{a}_\alpha^\Delta$ induces $\mathfrak{b}^\Delta$ on $V$.
The sheaf of ideals $\sh{A}$ being thus defined, it is then clear that $f=g\circ j$, where $j:\mathfrak{Z}\to\mathfrak{X}$ is the canonical injection of the closed subprescheme $\mathfrak{Z}$ of $\mathfrak{X}$ defined by $\sh{A}$, and $g$ is an isomorphism from $\mathfrak{Y}$ to $\mathfrak{Z}$.
\end{proof}

\begin{prop}[10.14.5]
\label{1.10.14.5}
\medskip\noindent
\begin{enumerate}[label=\emph{(\roman*)}]
  \item If $f:\mathfrak{Z}\to\mathfrak{Y}$ and $g:\mathfrak{Y}\to\mathfrak{X}$ are closed immersions of locally Noetherian formal preschemes, then $g\circ f$ is a closed immersion.
\oldpage[I]{211}
  \item Let $\mathfrak{X}$, $\mathfrak{Y}$, and $\mathfrak{S}$ be locally Noetherian formal preschemes, $f:\mathfrak{X}\to\mathfrak{S}$ a closed immersion, and $g:\mathfrak{Y}\to\mathfrak{S}$ a morphism.
    Then the morphism $\mathfrak{X}\times_\mathfrak{S}\mathfrak{Y}\to\mathfrak{Y}$ is a closed immersion.
  \item Let $\mathfrak{S}$ be a locally Noetherian formal prescheme, and $\mathfrak{X}'$ and $\mathfrak{Y}'$ locally Noetherian formal $\mathfrak{S}$-preschemes such that $\mathfrak{X}'\times_\mathfrak{S}\mathfrak{Y}'$ is locally Noetherian.
    If $\mathfrak{X}$ and $\mathfrak{Y}$ are locally Noetherian $\mathfrak{S}$-preschemes, and $f:\mathfrak{X}\to\mathfrak{X}'$ and $g:\mathfrak{Y}\to\mathfrak{Y}'$ are $\mathfrak{S}$-morphisms that are closed immersions, then $f\times_\mathfrak{S}g$ is a closed immersion.
\end{enumerate}
\end{prop}

\begin{proof}
\label{proof-1.10.14.5}
By \sref{1.3.5.1}, it again suffices to prove (i) and (ii).

To prove (i), we can assume that $\mathfrak{Y}$ (resp. $\mathfrak{Z}$) is a closed subprescheme of $\mathfrak{X}$ (resp. $\mathfrak{Y}$) defined by a coherent sheaf $\sh{J}$ (resp. $\sh{K}$) of ideals of $\OO_\mathfrak{X}$ (resp. $\OO_\mathfrak{Y}$);
if $\psi$ is the injection $\mathfrak{Y}\to\mathfrak{X}$ of underlying spaces, then $\psi_*(\sh{K})$ is a coherent sheaf of ideals of $\psi_*(\OO_\mathfrak{Y})=\OO_\mathfrak{X}/\sh{J}$ \sref[0]{0.5.3.12}, and thus also a coherent $\OO_\mathfrak{X}$-module \sref[0]{0.5.3.10};
the kernel $\sh{K}_1$ of $\OO_\mathfrak{X}\to(\OO_\mathfrak{X}/\sh{J})/\psi_*(\sh{K})$ is thus a coherent sheaf of ideals of $\OO_\mathfrak{X}$ \sref[0]{0.5.3.4}, and $\OO_\mathfrak{X}/\sh{K}_1$ is isomorphic to $\psi_*(\OO_\mathfrak{Y}/\sh{K})$, which proves that $\mathfrak{Z}$ is an isomorphism to a closed subprescheme of $\mathfrak{X}$.

To prove (ii), it is immediate that we can restrict to the case where $\mathfrak{S}=\Spf(A)$, $\mathfrak{X}=\Spf(B)$, and $\mathfrak{Y}=\Spf(C)$, with $A$ a Noetherian $\mathfrak{J}$-adic ring, $B=A/\mathfrak{a}$ (where $\mathfrak{a}$ is an ideal of $A$), and $C$ a Noetherian topological adic $A$-algebra.
Everything then relies on proving that the homomorphism $C\to C\wh{\otimes}_A(A/\mathfrak{a})$ is \emph{surjective}:
but $A/\mathfrak{a}$ is an $A$-module of finite type, and its topology is the $\mathfrak{J}$-adic topology;
it then follows from \sref[0]{0.7.7.8} that $C\wh{\otimes}_A(A/\mathfrak{a})$ can be identified with $C\otimes_A(A/\mathfrak{a})=C/\mathfrak{a}C$, whence our claim.
\end{proof}

\begin{cor}[10.14.6]
\label{1.10.14.6}
Under the hypotheses of \sref{1.10.14.5}[ii], let $p:\mathfrak{X}\times_\mathfrak{S}\mathfrak{Y}\to\mathfrak{X}$ and $q:\mathfrak{X}\times_\mathfrak{S}\mathfrak{Y}\to\mathfrak{Y}$ be the projections, so that the diagram
\[
  \xymatrix{
    \mathfrak{X}\ar[d]_f &
    \mathfrak{X}\times_\mathfrak{S}\mathfrak{Y}\ar[l]_p\ar[d]^q\\
    \mathfrak{S} &
    \mathfrak{Y}\ar[l]_g
  }
\]
commutes.
For every coherent $\OO_\mathfrak{X}$ module $\sh{F}$, we then have a canonical isomorphism of $\OO_\mathfrak{Y}$-modules
\[
  u:g^*(f_*(\sh{F}))\isoto q_*(p^*(\sh{F})).
  \tag{10.14.6.1}
\]
\end{cor}

\begin{proof}
\label{proof-1.10.14.6}
We know that defining a homomorphism $g^*(f_*(\sh{F})) \to q_*(p^*(\sh{F}))$ is equivalent to defining a homomorphism $f_*(\sh{F}) \to g_*(q_*(p^*(\sh{F}))) = f_*(p_*(p^*(\sh{F})))$ \sref[0]{0.4.4.3}:
we take $u=f_*(\rho)$, where $\rho$ is the canonical homomorphism $\sh{F}\to p_*(p^*(\sh{F}))$ \sref[0]{0.4.4.3}.
To see that $u$ is an isomorphism, we can immediately restrict to the case where $\mathfrak{S}$, $\mathfrak{X}$, and $\mathfrak{Y}$ are formal spectra of Noetherian adic rings $A$, $B$, and $C$ (respectively), with the conditions above in \sref{1.10.14.5}[ii];
we then have $\sh{F}=M^\Delta$, where $M$ is an $(A/\mathfrak{a})$-module of finite type \sref{1.10.10.5}, and the two sides of \hyperref[1.10.14.6]{(10.14.6.1)} are then identified, respectively, by \sref{1.10.10.8}, with $(C\otimes_A M)^\Delta$ and $((C/\mathfrak{a}C)\otimes_{A/\mathfrak{a}}M)^\Delta$, whence the corollary, because $(C/\mathfrak{a}C)\otimes_{A/\mathfrak{a}}M = (C\otimes_A(A/\mathfrak{a}))\otimes_{A/\mathfrak{a}}M$ is canonically identified with $C\otimes_A M$.
\end{proof}

\oldpage[I]{212}
\begin{cor}[10.14.7]
\label{1.10.14.7}
Let $X$ be a locally Noetherian usual prescheme, $Y$ a closed subprescheme of $X$, $j$ the canonical injection $Y\to X$, $X'$ a closed subset of $X$, and $Y'=Y\cap X'$;
then $\wh{j}:Y_{/Y'}\to X_{/X'}$ is a closed immersion, and, for every coherent $\OO_Y$-module $\sh{F}$, we have
\[
    \wh{j}_*(\sh{F}_{/Y'}) = (j_*(\sh{F}))_{/X'}.
\]
\end{cor}

\begin{proof}
\label{proof-1.10.14.7}
Since $Y'=j^{-1}(X')$, it suffices to use \sref{1.10.9.9} and apply \sref{1.10.14.5} and \sref{1.10.14.6}.
\end{proof}

\subsection{Separated formal preschemes}
\label{subsection:1.10.15}

\begin{defn}[10.15.1]
\label{1.10.15.1}
Let $\mathfrak{S}$ be a formal prescheme, $\mathfrak{X}$ a formal $\mathfrak{S}$-prescheme, and $f:\mathfrak{X}\to\mathfrak{S}$ the structure morphism.
We define the diagonal morphism $\Delta_{\mathfrak{X}|\mathfrak{S}}:\mathfrak{X}\to\mathfrak{X}\times_\mathfrak{S}\mathfrak{X}$ (also denoted by $\Delta_\mathfrak{X}$) to be the morphism $(1_\mathfrak{X},1_\mathfrak{X})_\mathfrak{S}$.
We say that $\mathfrak{X}$ is separated over $\mathfrak{S}$, or is a formal $\mathfrak{S}$-scheme, or that $f$ is a separated morphism, if the image of the underlying space of $\mathfrak{X}$ under $\Delta_\mathfrak{X}$ is a closed subset of the underlying space of $\mathfrak{X}\times_\mathfrak{S}\mathfrak{X}$.
We say that a formal prescheme $\mathfrak{X}$ is separated, or is a formal schemes, if it is separated over $\bb{Z}$.
\end{defn}

\begin{prop}[10.15.2]
\label{1.10.15.2}
Suppose that the formal preschemes $\mathfrak{S}$ and $\mathfrak{X}$ are inductive limits of sequences $(S_n)$ and $(X_n)$ (respectively) of usual preschemes, and that the morphism $f:\mathfrak{X}\to\mathfrak{S}$ is the inductive limit of a sequence of morphisms $f_n:X_n\to S_n$.
For $f$ to be separated, it is necessary and sufficient for the morphism $f_0:X_0\to S_0$ to be separated.
\end{prop}

\begin{proof}
\label{proof-1.10.15.2}
Indeed, $\Delta_{\mathfrak{X}|\mathfrak{S}}$ is then the inductive limit of the sequence of morphisms $\Delta_{X_n|S_n}$ \sref{1.10.7.4}, and the image of the underlying space of $\mathfrak{X}$ (resp. of $\mathfrak{X}\times_\mathfrak{S}\mathfrak{X}$ under $\Delta_{\mathfrak{X}|\mathfrak{S}}$) is identical to the image of the underlying space of $X_0$ (resp. of $X_0\times_{S_0}X_0$) under $\Delta_{X_0|S_0}$;
whence the conclusion.
\end{proof}

\begin{prop}[10.15.3]
\label{1.10.15.3}
Suppose that all the formal preschemes (resp. morphisms of formal preschemes) in what follows are inductive limits of sequences of usual preschemes (resp. of morphisms of usual preschemes).
\begin{enumerate}[label=\emph{(\roman*)}]
    \item The composition of any two separated morphisms is separated.
    \item If $f:\mathfrak{X}\to\mathfrak{X}'$ and $g:\mathfrak{Y}\to\mathfrak{Y}'$ are separated $\mathfrak{S}$-morphisms, then $f\times_\mathfrak{S}g$ is separated.
    \item If $f:\mathfrak{X}\to\mathfrak{Y}$ is a separated $\mathfrak{S}$-morphism, then the $\mathfrak{S}'$-morphism $f_{(\mathfrak{S}')}$ is separated for every extension $\mathfrak{S}'\to\mathfrak{S}$ of the base formal prescheme.
    \item If the composition $g\circ f$ of two morphisms is separated, then $f$ is separated.
\end{enumerate}
\emph{(In the above, it is implicit that if the same formal prescheme $\mathfrak{Z}$ is mentioned more than once in the same proposition, we consider it as the inductive limit of the \emph{same} sequence $(Z_n)$ of usual preschemes wherever it is mentioned, and the morphisms from $\mathfrak{Z}$ to another formal prescheme (resp. from a formal prescheme to $\mathfrak{Z}$) as inductive limits of morphisms from $Z_n$ to some usual preschemes (resp. from some usual preschemes to $Z_n$)).}
\end{prop}

\begin{proof}
\label{proof-1.10.15.3}
With the notation of \sref{1.10.15.2}, we in fact have $(g\circ f)_0=g_0\circ f_0$, and $(f\times_\mathfrak{S}g)_0=f_0\times_{S_0}g_0$;
the claims of \sref{1.10.15.3} are then immediate consequences of \sref{1.10.15.2} and the corresponding claims in \sref{1.5.5.1} for usual preschemes.
\end{proof}

We leave it to the reader to state, for the same type of formal preschemes and morphisms as in \sref{1.10.15.3}, the propositions corresponding to \sref{1.5.5.5}, \sref{1.5.5.9}, and \sref{1.5.5.10}
\oldpage[I]{213}
(by replacing ``affine open subset'' by ``formal affine open subset satisfying condition (b) of \sref{1.10.6.3}'').

A similar argument also shows that every \emph{Noetherian} formal affine scheme is separated, which justifies the terminology.

\begin{prop}[10.15.4]
\label{1.10.15.4}
Let $\mathfrak{S}$ be a locally Noetherian formal prescheme, and $\mathfrak{X}$ and $\mathfrak{Y}$ locally Noetherian formal $\mathfrak{S}$-preschemes such that $\mathfrak{X}$ or $\mathfrak{Y}$ is of finite type over $\mathfrak{S}$ (so that $\mathfrak{X}\times_\mathfrak{S}\mathfrak{Y}$ is locally Noetherian) and such that $\mathfrak{Y}$ is separated over $\mathfrak{S}$.
Let $f:\mathfrak{X}\to\mathfrak{Y}$ be an $\mathfrak{S}$-morphism; then the graph morphism $\Gamma_f(1_\mathfrak{X},f)_\mathfrak{S}:\mathfrak{X}\to\mathfrak{X}\times_\mathfrak{S}\mathfrak{Y}$ is a closed immersion.
\end{prop}

\begin{proof}
\label{proof-1.10.15.4}
We can assume that $\mathfrak{S}$ is the inductive limit of a sequence $(S_n)$ of locally Noetherian preschemes, $\mathfrak{X}$ (resp. $\mathfrak{Y}$) the inductive limit of a sequence $(X_n)$ (resp. $(Y_n)$) of $S_n$-preschemes, and $f$ the inductive limit of a sequence $(f_n:X_n\to Y_n)$ of $S_n$-morphisms;
then $\mathfrak{X}\times_\mathfrak{S}\mathfrak{Y}$ is the inductive limit of the sequence $(X_n\times_{S_n}Y_n)$, and $\Gamma_f$ the inductive limit of the sequence $(\Gamma_{f_n})$ \sref{1.10.7.4};
by hypothesis, $Y_0$ is separated over $S_0$ \sref{1.10.15.2}, so the space $\Gamma_{f_0}(X_0)$ is a closed subspace of $X_0\times_{S_0}Y_0$;
since the underlying spaces of $\mathfrak{X}\times_\mathfrak{S}\mathfrak{Y}$ (resp. $\Gamma_f(\mathfrak{X})$) and $X_0\times_{S_0}Y_0$ (resp. $\Gamma_{f_0}(X_0)$) are the same, we already see that $\Gamma_f(\mathfrak{X})$ is a \emph{closed} subspace of $\mathfrak{X}\times_\mathfrak{S}\mathfrak{Y}$.
Now note that, when $(U,V)$ runs over the set of pairs consisting of a Noetherian formal affine open subset $U$ (resp. $V$) of $\mathfrak{X}$ (resp $\mathfrak{Y}$) such that $f(U)\subset V$, the open subsets $U\times_S V$ form a cover of $\Gamma_f(\mathfrak{X})$ in $\mathfrak{X}\times_\mathfrak{S}\mathfrak{Y}$, and if $f_U:U\to V$ is the restriction of $f$ to $U$, then $\Gamma_{f_U}:U\to U\times_\mathfrak{S}V$ is the restriction of $\Gamma_f$ to $U$.
If we show that $\Gamma_{f_U}$ is a closed immersion, then $\Gamma_f$ will be a closed immersion \sref{1.10.14.4}, or, in other words, we are led to consider the case where $\mathfrak{S}=\Spf(A)$, $\mathfrak{X}=\Spf(B)$, and $\mathfrak{Y}=\Spf(C)$ are affine (with $A$, $B$, and $C$ Noetherian adics), with $f$ corresponding to a continuous $A$-homomorphism $\vphi:C\to B$;
$\Gamma_f$ then corresponds to the unique continuous homomorphism $\omega: B\wh{\otimes}_A C\to B$ which, when composed with the canonical homomorphisms $B\to B\wh{\otimes}_A C$ and $C\to B\wh{\otimes}_A C$, gives (respectively) the identity and $\vphi$.
But it is clear that $\omega$ is \emph{surjective}, whence our claim.
\end{proof}

\begin{cor}[10.15.5]
\label{1.10.15.5}
Let $\mathfrak{S}$ be a locally Noetherian formal prescheme, and $\mathfrak{X}$ a $\mathfrak{S}$-prescheme of finite type;
for $\mathfrak{X}$ to be separated over $\mathfrak{S}$, it is necessary and sufficient for the diagonal morphism $\mathfrak{X}\to\mathfrak{X}\times_\mathfrak{S}\mathfrak{X}$ to be a closed immersion.
\end{cor}

\begin{prop}[10.15.6]
\label{1.10.15.6}
A closed immersion $j:\mathfrak{Y}\to\mathfrak{X}$ of locally Noetherian formal preschemes is a separated morphism.
\end{prop}

\begin{proof}
\label{proof-1.10.15.6}
With the notation of \sref{1.10.14.2}, $j_0:Y_0\to X_0$ is a closed immersion, thus a separated morphism, and so it suffices to apply \sref{1.10.15.2}.
\end{proof}

\begin{prop}[10.15.7]
\label{1.10.15.7}
Let $X$ be a locally Noetherian (usual) prescheme, $X'$ a closed subset of $X$, and $\wh{X}=X_{/X'}$.
For $\wh{X}$ to be separated, it is necessary and sufficient for $\wh{X}_\red$ to be separated, and it is sufficient that $X$ be separated.
\end{prop}

\begin{proof}
\label{proof-1.10.15.7}
With the notation of \sref{1.10.8.5}, for $\wh{X}$ to be separated, it is necessary and sufficient for $X'_0$ to be separated \sref{1.10.15.2}, and since $\wh{X}_\red=(X'_0)_\red$, it is equivalent to ask for $\wh{X}_\red$ to be separated \sref{1.5.5.1}[vi].
\end{proof}


\bibliography{the}
\bibliographystyle{amsalpha}

\end{document}

