\documentclass[../main.tex]{subfiles}

\begin{document}

\begin{cx}{1.7.1}
Given an $A$-module $M$, we call the \emph{support} of $M$ and denote by $\Supp(M)$
the set of prime ideals $\mf{p}$ of $A$ such that $M_\mf{p}\neq 0$. For $M=0$, it
is necessary and sufficient that $\Supp(M)=\emp$, because if $M_\mf{p}=0$ for all
$\mf{p}$, the annihilator of an element $x\in M$ cannot be contained in any prime
ideal of $A$, so $A$ is \unsure{total}.
\end{cx}

\begin{cx}{1.7.2}
If $0\to N\to M\to P\to 0$ is an exact sequence of $A$-modules, we have
\[
  \Supp(M)=\Supp(N)\cup\Supp(P)
\]
because for every prime ideal $\mf{p}$ of $A$, the sequence
${0\to N_\mf{p}\to M_\mf{p}\to P_\mf{p}\to 0}$ is exact (1.3.2) and for
$M_\mf{p}=0$, it is necessary and sufficient that $N_\mf{p}=P_\mf{p}=0$.
\end{cx}

\begin{cx}{1.7.3}
If $M$ is the sum of a family $(M_\lambda)$ of submodules, $M_\mf{p}$ is the sum
of $(M_\lambda)_\mf{p}$ for every prime ideal $\mf{p}$ of $A$ (1.3.3 and 1.3.2),
so $\Supp(M)=\bigcup_\lambda\Supp(M_\lambda)$.
\end{cx}

\begin{cx}{1.7.4}
If $M$ is an $A$-module \emph{of finite type}, $\Supp(M)$ is the set of prime
ideals \emph{containing the annihilator of} $M$. Indeed, if $M$ is cyclic and
generated by $x$, say that $M_\mf{p}=0$ means that there exists $s\not\in\mf{p}$
such that $s\cdot x=0$, so that $\mf{p}$ does not contain the annihilator of $x$.
If now $M$ admits a finite system $(x_i)_{1\leq i\leq n}$ of generators and if
$\mf{a}_i$ is the annihilator of $x_i$, it follows from (1.7.3) that $\Supp(M)$
is th set of $\mf{p}$ containing one of $\mf{a}_i$, or, equivalently, the
set of $\mf{p}$ containing $\mf{a}=\bigcap_i\mf{a}_i$, which is the annihilator
of $M$.
\end{cx}

\begin{cx}{1.7.5}
If $M$ and $N$ are two $A$-modules \emph{of finite type}, we have
\[
  \Supp(M\otimes_A N)=\Supp(M)\cap\Supp(N).
\]
It can be seen that if $\mf{p}$ is a prime ideal of $A$, the condition
$M_\mf{p}\otimes_{A_\mf{p}}N_\mf{p}\neq 0$ is equivalent to
``$M_\mf{p}\neq 0$ and $N_\mf{p}\neq 0$'' (taking into account (1.3.4)). In
other words, it is about seeing that if $P$, $Q$ are two modules of finite type
over a \emph{local} ring $B$, not reduced to $0$, then $P\otimes_B Q\neq 0$. Let
$\mf{m}$ be the maximal ideal of $B$. By virtue of Nakayama's lemma, the vector
spaces $P/\mf{m}P$ and $Q/\mf{m}Q$ are not reduced to $0$, so it is the same with
the tensor product
$(P/\mf{m}P)\otimes_{B/\mf{m}}(Q/\mf{m}Q)=(P\otimes_B Q)\otimes_B(B/\mf{m})$,
hence the conclusion.

In particular, if $M$ is an $A$-module of finite type, $\mf{a}$ an ideal of $A$,
$\Supp(M/\mf{a}M)$ is the set of prime ideals containing both $\mf{a}$ and the
annihilator $\mf{n}$ of $M$ (1.7.4), that is, the set of prime ideals containing
$\mf{a}+\mf{n}$.
\end{cx}

\end{document}

