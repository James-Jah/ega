\documentclass[../main.tex]{subfiles}

\begin{document}

\begin{env}{4.1.1}
A \emph{ringed space} (resp. topologically ringed space) is a couple $(X,\sheaf{A})$
consisting of a topological space $X$ and a sheaf of rings (not necessarily commutative)
(resp. of a sheaf of topological rings) $\sheaf{A}$ on $X$; we say that $X$ is the
\emph{underlying} topological space of the ringed space $(X,\sheaf{A})$, and $\sheaf{A}$
the \emph{structure sheaf}. The latter is denoted $\O_X$, and its stalk at a point
$x\in X$ is denotes $\O_{X,x}$ or simply $\O_x$ when there is no chance of confusion.

We denote by $1$ or $e$ the \emph{unit section} of $\O_X$ over $X$ (the unit element
of $\Gamma(X,\O_X)$).

As in this Treaty we will have to consider in particular sheaves of \emph{commutative}
rings, it will be understood, when we speak of a ringed space $(X,\sheaf{A})$ without
specification, that $\sheaf{A}$ is a sheaf of commutative rings.

The ringed spaces with with not necessarily commutative structure sheaves
(resp. the topologically ringed spaces) form a \emph{category}, where we define
a \emph{morphism} $(X,\sheaf{A})\to(Y,\sheaf{B})$ as a couple $(\psi,\theta)=\Psi$
consisting of a continuous map $\psi\colon X\to Y$ and a \emph{$\psi$-morphism}
$\theta\colon\sheaf{G}\to\sheaf{F}$ (3.5.1) of sheaves of rings (resp. of sheaves of
topological rings); the \emph{composition} of a second morphism
$\Psi'=(\psi',\theta')\colon(Y,\sheaf{B})\to(Z,\sheaf{C})$ and of $\Psi$, denoted
$\Psi''=\Psi'\circ\Psi$, is the morphism $(\psi'',\theta'')$ where $\psi''=\psi'\circ\psi$,
and $\theta''$ is the composition of $\theta$ and $\theta'$ (equal to
$\psi_\ast'(\theta)\circ\theta'$, cf. \eref{3.5.2}). For ringed spaces, remember that we
then have ${\theta''}^\#=\theta^\#\circ\psi^\ast({\theta'}^\#)$ \eref{3.5.5}; therefore
if ${\theta'}^\#$ and $\theta^\#$ are the \emph{injective} (resp. \emph{surjective}),
the same is true of ${\theta''}^\#$, taking into account that $\psi_x\circ\rho_{\psi(x)}$
is an isomorphism for all $x\in X$ \eref{3.7.2}. We verify immediately, thanks to the
above, that when $\psi$ is an \emph{injective} continuous map and $\theta^\#$ is
a \emph{surjective} homomophism of sheaves of rings, the morphism $(\psi,\theta)$ is
a \emph{momomorphism} (T, 1.1) in the category of ringed spaces.

By abuse of language, we will often replace $\psi$ by $\Psi$ in notation, for
example in writing $\Psi^{-1}(U)$ in place of $\psi^{-1}(U)$ for a subset $U$ of $Y$,
when the is no risk of confusion.
\end{env}

\begin{env}{4.1.2}
For each subset $M$ of $X$, the pair $(M,\sheaf{A}|M)$ is evidently a ringed space, said
to be \emph{induced} on $M$ by the ringed space $(X,\sheaf{A})$ (and is still called
the \emph{restriction} of $(X,\sheaf{A})$ to $M$). If $j$ is the canonical injection
$M\to X$ and $\omega$ is the identity map of $\sheaf{A}|M$, $(j,\omega^\flat)$ is a
monomorphism $(M,\sheaf{A}|M)\to(X,\sheaf{A})$ of ringed spaces, called the
\emph{canonical injection}. The composition of a morphism
$\Psi\colon(X,\sheaf{A})\to(Y,\sheaf{B})$ and this injection is called the \emph{restriction}
of $\Psi$ to $M$.
\end{env}

\begin{env}{4.1.3}
We will not revisit the defintions of \emph{$\sheaf{A}$-modules} or
\emph{algebraic sheaves} on a ringed space $(X,\sheaf{A})$ (G, II, 2.2);
when $\sheaf{A}$ is a sheaf of not necessarily commutative rings, by $\sheaf{A}$-module
it will always mean ``left $\sheaf{A}$-module'' unless expressly stated otherwise. The
$\sheaf{A}$-submodules of $\sheaf{A}$ will be called \emph{sheaves of ideals} (left,
right, or two-sided) in $\sheaf{A}$ or \emph{$\sheaf{A}$-ideals}.

When $\sheaf{A}$ is a sheaf of commutative rings, and in the definition of
$\sheaf{A}$-modules, we replace everywhere the \emph{module} structure by that of
an \emph{algebra}, we obtain the definition of an \emph{$\sheaf{A}$-algebra} on $X$.
It is the same to say that an $\sheaf{A}$-algebra (not necessarily commutative) is
a $\sheaf{A}$-module $\sheaf{C}$, given with a homomorphism of $\sheaf{A}$-modules
$\varphi\colon\sheaf{C}\otimes_{\sheaf{A}}\sheaf{C}\to\sheaf{C}$ and a section $e$ over $X$,
such that: \nth{1} the diagram
\[
  \begin{tikzcd}
    \sheaf{C}\otimes_{\sheaf{A}}\sheaf{C}\otimes_{\sheaf{A}}\sheaf{C}
    \ar[r,"\varphi\otimes 1"]\ar[d,"1\otimes\varphi"'] &
    \sheaf{C}\otimes_{\sheaf{A}}\sheaf{C}\ar[d,"\varphi"]\\
    \sheaf{C}\otimes_{\sheaf{A}}\sheaf{C}\ar[r,"\varphi"] & \sheaf{C}
  \end{tikzcd}
\]
is commutative; \nth{2} for each open $U\subset X$ and each section $s\in\Gamma(U,\sheaf{C})$,
we have $\varphi((e|U)\otimes s)=\varphi(s\otimes(e|U))=s$. We say that $\sheaf{C}$ is a
commutative $\sheaf{A}$-algebra if the diagram
\[
  \begin{tikzcd}
    \sheaf{C}\otimes_{\sheaf{A}}\sheaf{C}\ar[rr,"\sigma"]\ar[rd,"\varphi"']
    & & \sheaf{C}\otimes_{\sheaf{A}}\sheaf{C}\ar[ld,"\varphi"]\\
   & \sheaf{C}
  \end{tikzcd}
\]
is commutative, $\sigma$ denoting the canonical symmetry (twist) map of the tensor product
$\sheaf{C}\otimes_{\sheaf{A}}\sheaf{C}$.

The homomorphisms of $\sheaf{A}$-algebras are also defined as the homomorphisms of $\sheaf{A}$-modules
in (G, II, 2.2), but naturally no longer form an abelian group.

If $\sheaf{M}$ is an $\sheaf{A}$-submodule of an $\sheaf{A}$-algebra $\sheaf{C}$, the
\emph{$\sheaf{A}$-subalgebra of $\sheaf{C}$ generated by $\sheaf{M}$} is the sum of the images
of the homomorphisms $\bigotimes^n\sheaf{M}\to\sheaf{C}$ (for each $n\geq 0$). This is also the sheaf
associated to the presheaf $U\mapsto\sheaf{B}(U)$ of algebras, $\sheaf{B}(U)$ being the subalgebra
of $\Gamma(U,\sheaf{C})$ generated by the submodule $\Gamma(U,\sheaf{M})$.
\end{env}

\begin{env}{4.1.4}
We say that a sheaf of rings $\sheaf{A}$ on a topological space $X$ is \emph{reduced at a point $x$ in $X$}
if the stalk $\sheaf{A}_x$ is a \emph{reduced} ring \eref{1.1.1}; we say that $\sheaf{A}$ is \emph{reduced}
if it is reduced at all point of $X$. Recall that a ring $A$ is called \emph{regular} if each of the local
rings $A_{\mathfrak{p}}$ (where $\mathfrak{p}$ runs through the set of prime ideals of $A$) is a regular local ring;
we will say that a sheaf of rings $\sheaf{A}$ on $X$ is \emph{regular at a point $x$} (resp. \emph{regular})
if the stalk $\sheaf{A}_x$ is a regular ring (resp. if $\sheaf{A}$ is regular at each point). Finally, we will
say that a sheaf of rings $\sheaf{A}$ on $X$ is \emph{normal at a point $x$} (resp. \emph{normal}) if the
stalk $\sheaf{A}_x$ is an \emph{integral domain and integrally closed} (resp. if $\sheaf{A}$ is normal at each
point). We will say that a ringed space $(X,\sheaf{A})$ has any of these preceeding properties if the sheaf
of rings $\sheaf{A}$ has that property.

A \emph{graded} sheaf of rings $\sheaf{A}$ is by definition a sheaf of rings that is the direct sum
(G, II, 2,7) of a family $(\sheaf{A}_n)_{n\in\bbold{Z}}$ of sheaves of abelian groups with the conditions
$\sheaf{A}_m\sheaf{A}_n\subset\sheaf{A}_{m+n}$; a \emph{graded $\sheaf{A}$-module} is an $\sheaf{A}$-module
$\sheaf{F}$ that is the direct sum of a family $(\sheaf{F}_n)_{n\in\bbold{Z}}$ of sheaves of abelian groups,
satisfying the conditions $\sheaf{A}_m\sheaf{F}_n\subset\sheaf{F}_{m+n}$. It is equivalent to say that
$(\sheaf{A}_m)_x(\sheaf{A}_n)_x\subset(\sheaf{A}_{m+n})_x$ (resp. $(\sheaf{A}_m)_x(\sheaf{F}_n)_x\subset(\sheaf{F}_{m+n})_x$)
for each point $x$.
\end{env}

\begin{env}{4.1.5}
Given a ringed space $(X,\sheaf{A})$ (not necessarily commutative), we will not recall here the definitions of the
bifunctors $\sheaf{F}\otimes_{\sheaf{A}}\sheaf{G}$, $\sheafHom_{\sheaf{A}}(\sheaf{F},\sheaf{F})$, and
$\Hom_{\sheaf{A}}(\sheaf{F},\sheaf{G})$ (G, II, 2.8~and~2.2) in the categories of right or left (depending on
the case) $\sheaf{A}$-modules, with values in the category of sheaves of abelian groups (or more generally
of $\sheaf{C}$-modules, if $\sheaf{C}$ is the center of $\sheaf{A}$). The stalk $(\sheaf{F}\otimes_{\sheaf{A}}\sheaf{G})_x$
for each point $x\in X$ identifies canonically with $\sheaf{F}_x\otimes_{\sheaf{A}_x}\sheaf{G}_x$ and we define a
canonical and functorial homomorphism $(\sheafHom_{\sheaf{A}}(\sheaf{F},\sheaf{G}))_x\to\Hom_{\sheaf{A}_x}(\sheaf{F}_x,\sheaf{G}_x)$
which is in general not injective or surjective. The bifunctors considered above are additive and in particular,
commute with finite direct limits; $\sheaf{F}\otimes_{\sheaf{A}}\sheaf{G}$ is right exact in $\sheaf{F}$ and in $\sheaf{G}$,
commutes with inductive limits, and $\sheaf{A}\otimes_{\sheaf{A}}\sheaf{G}$ (resp. $\sheaf{F}\otimes_{\sheaf{A}}\sheaf{A}$)
identifies canonically with $\sheaf{G}$ (resp. $\sheaf{F}$). The functors $\sheafHom_{\sheaf{A}}(\sheaf{F},\sheaf{G})$ and
$\Hom_{\sheaf{A}}(\sheaf{F},\sheaf{G})$ are \emph{left exact} in $\sheaf{F}$ and $\sheaf{G}$; more precisely,
if we have an exact sequence of the form $0\to\sheaf{G}'\to\sheaf{G}\to\sheaf{G}''$, the sequence
\[
  \begin{tikzcd}
    0\ar[r] &\sheafHom_{\sheaf{A}}(\sheaf{F},\sheaf{G}')\ar[r] &
    \sheafHom_{\sheaf{A}}(\sheaf{F},\sheaf{G})\ar[r] &
    \sheafHom_{\sheaf{A}}(\sheaf{F},\sheaf{G}'')
  \end{tikzcd}
\]
is exact, and if we have an exact sequence of the form $\sheaf{F}'\to\sheaf{F}\to\sheaf{F}''\to 0$, the sequence
\[
  \begin{tikzcd}
    0\ar[r] &\sheafHom_{\sheaf{A}}(\sheaf{F}'',\sheaf{G})\ar[r] &
    \sheafHom_{\sheaf{A}}(\sheaf{F},\sheaf{G})\ar[r] &
    \sheafHom_{\sheaf{A}}(\sheaf{F}',\sheaf{G})
  \end{tikzcd}
\]
is exact, with the analagous properties for the functor $\Hom$.

\end{env}

\unsure{TODO}

\end{document}

