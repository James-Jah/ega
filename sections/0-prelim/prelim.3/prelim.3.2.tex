\documentclass[../main.tex]{subfiles}

\begin{document}

\begin{env}{3.2.1}
We will restrict to the following categories $\K$ admitting
\emph{projective limits} (generalized, that is, corresponding to not necessarily filtered preordered sets,
cf. (T, 1.8)). Let $X$ be a topological space, $\mf{B}$ an open basis
for the topology of $X$. We will call a \emph{presheaf on $\mf{B}$, with values in $\K$,} to be
a family of objects $\F(U)\in\K$, corresponding to each $U\in\mf{B}$, and a family of morphisms
$\rho_U^V:\F(V)\to\F(U)$ defined for any pair $(U,V)$ of elements of $\mf{B}$ such that $U\su V$,
\oldpage{26}with the conditions $\rho_U^U=$ identity and $\rho_U^W=\rho_U^V\circ\rho_V^W$ if $U$, $V$, $W$ in
$\mf{B}$ are such that $U\su V\su W$. We can associate a \emph{presheaf with values in} $\K$ : $U\mapsto\F(U)$
in the ordinary sense, taking for all open $U$, $\F'(U)=\varprojlim\F(V)$, where $V$ runs through
the ordered set (for $\su$, \emph{not filtered} in general) of $V\in\mf{B}$ sets such that $V\su U$, since
the $\F(V)$ form a projective system for the $\rho_V^W$ ($V\su W\su U$, $V\in\mf{B}$, $W\in\mf{B}$).
Indeed, if $U$, $U'$ are two open sets of $X$ such that $U\su U'$, we define ${\rho'}_U^{U'}$ as the
projective limit (for $V\su U$) of the canonical morphisms $\F'(U')\to\F(V)$, in other words
the unique morphism y $\F'(U')\to\F'(U)$, which, when composed with the canonical morphisms
$\F'(U)\to\F(V)$, gives the canonical morphisms $\F'(U')\to\F(V)$; the verification of
the transitivity of ${\rho'}_U^{U'}$ is then immediate. Moreover, if $U\in\mf{B}$,
the canonical morphism $\F'(U)\to\F(U)$ is an isomorphism, allowing to identify these two objects
\footnote{
If $X$ is a \emph{Noetherian} space, we can still define $\F'(U)$ and show that it is a presheaf (in the
ordinary sense) when one supposes only that $\K$ admits projective limits for \emph{finite} projective systems. Indeed,
if $U$ is any open set of $X$, there is a \emph{finite} covering $(V_i)$ of $U$ formed by sets of $\mf{B}$; for every
couple $(i,j)$ of indices, let $(V_{ijk})$ be a finite covering of $V_i\cap V_j$ formed by sets of $\mf{B}$. Let $I$ be the set
of $i$ and triples $(i,j,k)$, ordered only by the relations $i>(i,j,k)$, $j>(i,j,k)$; we then take $\F'(U)$ to be
the projective limit of the system of $\F(V_i)$ and $\F(V_{ijk})$; it is easy to verify that this does not depend on the
coverings $(V_i)$ and $(V_{ijk})$ and that $U\mapsto\F'(U)$ is a presheaf.}.
\end{env}

\begin{env}{3.2.2}
For the presheaf $\F'$ thus defined to be a \emph{sheaf}, it is necessary and sufficient
that the presheaf $\F$ on $\mf{B}$ satisfies the condition:\\

  (F$_0$) \emph{For any covering $(U_\a)$ of $U\in\mf{B}$ by sets $U_\a\in\mf{B}$
  contained in $U$, and for any object $T\in\K$, the map which takes $f\in\Hom(T,\F(U))$
  to the family $(\rho_{U_\a}^U\circ f)\in\prod_\a\Hom(T,\F(U_\a))$ is a bijection of
  $\Hom(T,\F(U))$ on the set of all $(f_\a)$ such that $\rho_V^{U_\a}\circ f_\a=\rho_V^{U_\b}\circ f_\b$
  for any pair of indices $(\a,\b)$ and any $V\in\mf{B}$ such that $V\su U_\a\cap U_\b$
  \footnote{It also means that the pair formed by $\F(U)$ and the $\rho_\a=\rho_{U_\a}^U$
  is a \emph{solution to the universal problem} defined in \eref{3.1.1} by the data of $A_\a=\F(U_\a)$,
  $A_{\a\b}=\prod\F(V)$ (for $V\in\mf{B}$ such that $V\su U_\a\cap U_\b$) and
  $\rho_{\a\b}=(\rho_V''):\F(U_\a)\to\prod\F(V)$ defined by the condition that for
  $V\in\mf{B}$, $V'\in\mf{B}$, $W\in\mf{B}$, $V\cup V'\su U_\a\cap U_\b$,
  $W\su V\cap V'$, $\rho_W^V\circ\rho_V''=\rho_W^{V'}\circ\rho_{V'}''$.}.}\\

The condition is obviously necessary. To show that it is sufficient,
consider first a second basis $\mf{B}'$ of the topology of $X$, \emph{contained in} $\mf{B}$, and
show that if $\F''$ denotes the presheaf induced by the subfamily $(\F(V))_{V\in\mf{B}'}$, $\F''$ is
\emph{canonically isomorphic} to $\F'$. Indeed, firstly the projective limit (for $V\in\mf{B}'$,
$V\su U$) canonical morphisms ${\F'(U)\to\F(V)}$ is a morphism ${\F'(U)\to\F''(U)}$
for all open $U$. If $U\in\mf{B}$, this morphism is an isomorphism, because by hypothesis
the canonical morphisms ${\F''(U)\to\F(V)}$ for $V\in\mf{B}'$, $V\su U$, factorize into
${\F''(U)\to\F(U)\to\F(V)}$, and it is immediate to see that the composition of morphisms
${\F(U)\to\F''(U)}$ and ${\F''(U)\to\F(U)}$ thus defined are the identities. This being so, for
all open $U$, the morphisms ${\F''(U)\to\F''(W)=\F(W)}$ for $W\in\mf{B}$ and $W\su U$ satisfy
the conditions characterizing the projective limit of $\F(W)$ ($W\in\mf{B}$, $W\su U$), which demonstrates
our assertion given the uniqueness of a projective limit up to isomorphism.

This posed, let $U$ be any open set of $X$, $(U_\a)$ a covering of $U$ by
the open sets contained in $U$, and $\mf{B}'$ the subfamily of $\mf{B}$ formed by the sets
\oldpage{27}of $\mf{B}$ contained in at least $U_\a$; it is clear that $\mf{B}'$ is still a basis
of the topology of $X$, so $\F'(U)$ (resp. $\F''(U_\a)$) is the projective limit of $\F(V)$ for $V\in\mf{B}'$
and $V\su U$ (resp., $V\su U_\a$), the axiom (F) is then immediately verified by virtue of the definition of the
projective limit.

When (F$_0$) is satisfied, we will say by abuse of language that the presheaf $\F$
on the basis $\mf{B}$ is a sheaf.
\end{env}

\begin{env}{3.2.3}
Let $\F$, $\G$ be two presheaves on a basis $\mf{B}$, with values in $\K$; we define a \emph{morphism}
$u:\F\to\G$ as a family $(u_V)_{V\in\mf{B}}$ of morphisms $u_V:\F(V)\to\G(V)$ satisfying the usual
compatibility conditions with the restriction morphisms $\rho_V^W$. With the notation of \eref{3.2.1},
we have a morphism $u':\F'\to\G'$ of (ordinary) presheaves by taking for $u_U'$ the projective limit
of the $u_V$ for $V\in\mf{B}$ and $V\su U$; the verification of the compatibility conditions with
the ${\rho'}_U^{U'}$ follows from the functorial properties of the projective limit.
\end{env}

\begin{env}{3.2.4}
If the category $\K$ admits inductive limits, and if $\F$ is a presheaf on the basis $\mf{B}$, with
values in $\K$, for each $x\in X$ the neighborhoods of $x$ belonging to $\mf{B}$ form a cofinal set
(for $\us$) in the set of neighborhoods of $x$, therefore, if $\F'$ is the (ordinary) presheaf
corresponding to $\F$, the fiber $\F_x'$ is equal to $\varinjlim_{\mf{B}}\F(V)$ over the set of
$V\in\mf{B}$ containing $x$. If $u:\F\to\G$ is morphism of presheaves on $\mf{B}$ with values in
$\K$, $u':\F'\to\G'$ the corresponding morphism of ordinary presheaves, $u_x'$ is likewise the
inductive limit of the morphisms $u_V:\F(V)\to\G(V)$ for $V\in\mf{B}$, $x\in V$.
\end{env}

\end{document}

