\documentclass[../main.tex]{subfiles}

\begin{document}

\begin{env}{3.2.1}
We will restrict to the following categories $\K$ admitting
\emph{projective limits} (generalized, that is, corresponding to not necessarily filtered preordered sets,
cf. (T, 1.8)). Let $X$ be a topological space, $\mathfrak{B}$ an open basis
for the topology of $X$. We will call a \emph{presheaf on $\mathfrak{B}$, with values in $\K$,} to be
a family of objects $\sheaf{F}(U)\in\K$, corresponding to each $U\in\mathfrak{B}$, and a family of morphisms
$\rho_U^V\colon\sheaf{F}(V)\to\sheaf{F}(U)$ defined for any pair $(U,V)$ of elements of $\mathfrak{B}$ such that $U\subset V$,
\oldpage{26}with the conditions $\rho_U^U=$ identity and $\rho_U^W=\rho_U^V\circ\rho_V^W$ if $U$, $V$, $W$ in
$\mathfrak{B}$ are such that $U\subset V\subset W$. We can associate a \emph{presheaf with values in} $\K$: $U\mapsto\sheaf{F}(U)$
in the ordinary sense, taking for all open $U$, $\sheaf{F}'(U)=\varprojlim\sheaf{F}(V)$, where $V$ runs through
the ordered set (for $\subset$, \emph{not filtered} in general) of $V\in\mathfrak{B}$ sets such that $V\subset U$, since
the $\sheaf(V)$ form a projective system for the $\rho_V^W$ ($V\subset W\subset U$, $V\in\mathfrak{B}$, $W\in\mathfrak{B}$).
Indeed, if $U$, $U'$ are two open sets of $X$ such that $U\subset U'$, we define ${\rho'}_U^{U'}$ as the
projective limit (for $V\subset U$) of the canonical morphisms $\sheaf{F}'(U')\to\sheaf{F}(V)$, in other words
the unique morphism y $\sheaf{F}'(U')\to\sheaf{F}'(U)$, which, when composed with the canonical morphisms
$\sheaf{F}'(U)\to\sheaf{F}(V)$, gives the canonical morphisms $\sheaf{F}'(U')\to\sheaf{F}(V)$; the verification of
the transitivity of ${\rho'}_U^{U'}$ is then immediate. Moreover, if $U\in\mathfrak{B}$,
the canonical morphism $\sheaf{F}'(U)\to\sheaf{F}(U)$ is an isomorphism, allowing to identify these two objects
\footnote{
If $X$ is a \emph{Noetherian} space, we can still define $\sheaf{F}'(U)$ and show that it is a presheaf (in the
ordinary sense) when one supposes only that $\K$ admits projective limits for \emph{finite} projective systems. Indeed,
if $U$ is any open set of $X$, there is a \emph{finite} covering $(V_i)$ of $U$ formed by sets of $\mathfrak{B}$; for every
couple $(i,j)$ of indices, let $(V_{ijk})$ be a finite covering of $V_i\cap V_j$ formed by sets of $\mathfrak{B}$. Let $I$ be the set
of $i$ and triples $(i,j,k)$, ordered only by the relations $i>(i,j,k)$, $j>(i,j,k)$; we then take $\sheaf{F}'(U)$ to be
the projective limit of the system of $\sheaf{F}(V_i)$ and $\sheaf{F}(V_{ijk})$; it is easy to verify that this does not depend on the
coverings $(V_i)$ and $(V_{ijk})$ and that $U\mapsto\sheaf{F}'(U)$ is a presheaf.}.
\end{env}

\begin{env}{3.2.2}
For the presheaf $\sheaf{F}'$ thus defined to be a \emph{sheaf}, it is necessary and sufficient
that the presheaf $\sheaf{F}$ on $\mathfrak{B}$ satisfies the condition:\\

  (F$_0$) \emph{For any covering $(U_\alpha)$ of $U\in\mathfrak{B}$ by sets $U_\alpha\in\mathfrak{B}$
  contained in $U$, and for any object $T\in\K$, the map which takes $f\in\Hom(T,\sheaf{F}(U))$
  to the family $(\rho_{U_\alpha}^U\circ f)\in\prod_\alpha\Hom(T,\sheaf{F}(U_\alpha))$ is a bijection of
  $\Hom(T,\sheaf{F}(U))$ on the set of all $(f_\alpha)$ such that $\rho_V^{U_\alpha}\circ f_\alpha=\rho_V^{U_\beta}\circ f_\beta$
  for any pair of indices $(\alpha,\beta)$ and any $V\in\mathfrak{B}$ such that $V\subset U_\alpha\cap U_\beta$
  \footnote{It also means that the pair formed by $\sheaf{F}(U)$ and the $\rho_\alpha=\rho_{U_\alpha}^U$
  is a \emph{solution to the universal problem} defined in \eref{3.1.1} by the data of $A_\alpha=\sheaf{F}(U_\alpha)$,
  $A_{\alpha\beta}=\prod\sheaf{F}(V)$ (for $V\in\mathfrak{B}$ such that $V\subset U_\alpha\cap U_\beta$) and
  $\rho_{\alpha\beta}=(\rho_V'')\colon\sheaf{F}(U_\alpha)\to\prod\sheaf{F}(V)$ defined by the condition that for
  $V\in\mathfrak{B}$, $V'\in\mathfrak{B}$, $W\in\mathfrak{B}$, $V\cup V'\subset U_\alpha\cap U_\beta$,
  $W\subset V\cap V'$, $\rho_W^V\circ\rho_V''=\rho_W^{V'}\circ\rho_{V'}''$.}.}\\

The condition is obviously necessary. To show that it is sufficient,
consider first a second basis $\mathfrak{B}'$ of the topology of $X$, \emph{contained in} $\mathfrak{B}$, and
show that if $\sheaf{F}''$ denotes the presheaf induced by the subfamily $(\sheaf{F}(V))_{V\in\mathfrak{B}'}$, $\sheaf{F}''$ is
\emph{canonically isomorphic} to $\sheaf{F}'$. Indeed, firstly the projective limit (for $V\in\mathfrak{B}'$,
$V\subset U$) canonical morphisms ${\sheaf{F}'(U)\to\sheaf{F}(V)}$ is a morphism ${\sheaf{F}'(U)\to\sheaf{F}''(U)}$
for all open $U$. If $U\in\mathfrak{B}$, this morphism is an isomorphism, because by hypothesis
the canonical morphisms ${\sheaf{F}''(U)\to\sheaf{F}(V)}$ for $V\in\mathfrak{B}'$, $V\subset U$, factorize into
${\sheaf{F}''(U)\to\sheaf{F}(U)\to\sheaf{F}(V)}$, and it is immediate to see that the composition of morphisms
${\sheaf{F}(U)\to\sheaf{F}''(U)}$ and ${\sheaf{F}''(U)\to\sheaf{F}(U)}$ thus defined are the identities. This being so, for
all open $U$, the morphisms ${\sheaf{F}''(U)\to\sheaf{F}''(W)=\sheaf{F}(W)}$ for $W\in\mathfrak{B}$ and $W\subset U$ satisfy
the conditions characterizing the projective limit of $\sheaf{F}(W)$ ($W\in\mathfrak{B}$, $W\subset U$), which demonstrates
our assertion given the uniqueness of a projective limit up to isomorphism.

This posed, let $U$ be any open set of $X$, $(U_\alpha)$ a covering of $U$ by
the open sets contained in $U$, and $\mathfrak{B}'$ the subfamily of $\mathfrak{B}$ formed by the sets
\oldpage{27}of $\mathfrak{B}$ contained in at least $U_\alpha$; it is clear that $\mathfrak{B}'$ is still a basis
of the topology of $X$, so $\sheaf{F}'(U)$ (resp. $\sheaf{F}''(U_\alpha)$) is the projective limit of $\sheaf{F}(V)$ for $V\in\mathfrak{B}'$
and $V\subset U$ (resp., $V\subset U_\alpha$), the axiom (F) is then immediately verified by virtue of the definition of the
projective limit.

When (F$_0$) is satisfied, we will say by abuse of language that the presheaf $\sheaf{F}$ on the basis $\mathfrak{B}$ is a sheaf.
\end{env}

\begin{env}{3.2.3}
Let $\sheaf{F}$, $\sheaf{G}$ be two presheaves on a basis $\mathfrak{B}$, with values in $\K$; we define a \emph{morphism}
$u\colon\sheaf{F}\to\sheaf{G}$ as a family $(u_V)_{V\in\mathfrak{B}}$ of morphisms $u_V\colon\sheaf{F}(V)\to\sheaf{G}(V)$ satisfying the usual
compatibility conditions with the restriction morphisms $\rho_V^W$. With the notation of \eref{3.2.1},
we have a morphism $u'\colon\sheaf{F}'\to\sheaf{G}'$ of (ordinary) presheaves by taking for $u_U'$ the projective limit
of the $u_V$ for $V\in\mathfrak{B}$ and $V\subset U$; the verification of the compatibility conditions with
the ${\rho'}_U^{U'}$ follows from the functorial properties of the projective limit.
\end{env}

\begin{env}{3.2.4}
If the category $\K$ admits inductive limits, and if $\sheaf{F}$ is a presheaf on the basis $\mathfrak{B}$, with
values in $\K$, for each $x\in X$ the neighborhoods of $x$ belonging to $\mathfrak{B}$ form a cofinal set
(for $\supset$) in the set of neighborhoods of $x$, therefore, if $\sheaf{F}'$ is the (ordinary) presheaf
corresponding to $\sheaf{F}$, the stalk $\sheaf{F}_x'$ is equal to $\varinjlim_{\mathfrak{B}}\sheaf{F}(V)$ over the set of
$V\in\mathfrak{B}$ containing $x$. If $u\colon\sheaf{F}\to\sheaf{G}$ is morphism of presheaves on $\mathfrak{B}$ with values in
$\K$, $u'\colon\sheaf{F}'\to\sheaf{G}'$ the corresponding morphism of ordinary presheaves, $u_x'$ is likewise the
inductive limit of the morphisms $u_V\colon\sheaf{F}(V)\to\sheaf{G}(V)$ for $V\in\mathfrak{B}$, $x\in V$.
\end{env}

\begin{env}{3.2.5}
We return to the general conditions of \eref{3.2.1}. If $\sheaf{F}$ is an ordinary
\emph{sheaf} with values in $\K$, $\sheaf{F}_1$ the sheaf \emph{on} $\mathfrak{B}$ obtained
by the restriction of $\sheaf{F}$ to $\mathfrak{B}$, the ordinary sheaf $\sheaf{F}_1'$
obtained from $\sheaf{F}_1$ by the procedure of \eref{3.2.1} is canonically
isomorphic to $\sheaf{F}$, by virtue of the condition (F) and the uniqueness properties
of the projective limit. We identify the ordinary sheaf $\sheaf{F}$ with $\sheaf{F}_1'$.

If $\sheaf{G}$ is a second (ordinary) sheaf on $X$ with values in $\K$, and
$u\colon\sheaf{F}\to\sheaf{G}$ a morphism, the preceding remark shows that the data of
the $u_V\colon\sheaf{F}(V)\to\sheaf{G}(V)$ \emph{for only the $V\in\mathfrak{B}$} completely
determine $u$; conversely, it is sufficient, the $u_V$ being given for $V\in\mathfrak{B}$,
to verify that the commutative diagram with the restriction morphisms $\rho_V^W$ for
$V\in\mathfrak{B}$, $W\in\mathfrak{B}$, and $V\subset W$, for
\end{env}

\unsure{TODO}

\end{document}

