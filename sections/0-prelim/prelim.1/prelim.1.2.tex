\documentclass[../main.tex]{subfiles}

\begin{document}

\begin{env}{1.2.1}
We say that a subset $S$ of a ring $A$ is \emph{multiplicative} if $1\in S$ and if the product of two elements of
\oldpage{13}We say that a part $S$ of a ring $A$ is \emph{multiplicative} if $1\in S$ and if the product of two elements of
$S$ is in $S$. The examples which will be the most important for the following are:
\nth{1} the set $S_f$ of powers $f^n$ ($n\geq 0$) of an element $f\in A$;
\nth{2} the complement $A-\mathfrak{p}$ of a \emph{prime} ideal $\mathfrak{p}$ of $A$.
\end{env}

\begin{env}{1.2.2}
Let $S$ be a multiplicative subset of a ring $A$, $M$ an $A$-module; in the set $M\times S$, the relation between
couples $(m_1,s_1)$, $(m_2,s_2)$:
\[
   ``\text{There exists } s\in S\text{ such that } s(s_1 m_2-s_2 m_1)=0"
\]
is an equivalence relation. We denote by $S^{-1}M$ the quotient set of $M\times S$ by this relation, by $m/s$ the canonical
image in $S^{-1}M$ of the pair $(m,s)$; we call the \emph{canonical} mapping of $M$ in $S^{-1}M$ the mapping $i_M^S\colon m\mapsto m/1$
(also denoted $i^S$). This mapping is generally neither injective nor surjective; its kernel is the set of $m\in M$ such that there
exists an $s\in S$ for which $sm=0$.

On $S^{-1}M$ we define an additive group law by taking
\[
  (m_1/s_1)+(m_2/s_2)=(s_2 m_1+s_1 m_2)/(s_1 s_2)
\]
(we check that it is independent of the expressions of the elements of $S^{-1}M$ considered). On $S^{-1}A$ we further define
a multiplicative law by taking $(a_1/s_1)(a_2/s_2)=(a_1 a_2)/(s_1 s_2)$, and finally an external law on $S^{-1}M$, having
$S^{-1}A$ as a set of operators, by setting $(a/s)(m/s')=(am)/(ss')$. It is thus verified that $S^{-1}A$ is provided with a
ring structure (called \emph{the ring of fractions of} $A$, \emph{with denominators in} $S$) and $S^{-1}M$ the structure of
an $S^{-1}A$-module (called \emph{the  module of fractions of} $M$, \emph{with denominators in} $S$); for all $s\in S$,
$s/1$ is invertible in $S^{-1}A$, its inverse being $1/s$. The canonical mapping $i_A^S$ (resp. $i_M^S$) is a homomorphism
of rings (resp. a homomorphism of $A$-modules, $S^{-1}M$ being considered $A$-module by means of the homomorphism
$i_A^S\colon A\to S^{-1}A$).
\end{env}

\begin{env}{1.2.3}
If $S_f=\{f^n\}_{n\geq 0}$ for a $f\in A$, we write $A_f$ and $M_f$ instead of $S_f^{-1}A$ and $S_f^{-1}M$; when $A_f$ is
considered as algebra over $A$, we can write $A_f=A[1/f]$. $A_f$ is isomorphic to the quotient algebra $A[T]/(fT-1)A[T]$.
When $f=1$, $A_f$ and $M_f$ identify canonically with $A$ and $M$; if $f$ is niipotent, $A_f$ and $M_f$ are reduced to $0$.
When $S=A-\mathfrak{p}$, where $\mathfrak{p}$ is a prime ideal of $A$, we write $A_\mathfrak{p}$ and $M_\mathfrak{p}$ instead of $S^{-1}A$ and $S^{-1}M$;
$A_\mathfrak{p}$ is a \emph{local ring} whose maximal ideal $\mathfrak{q}$ is generated by $i_A^S(\mathfrak{p})$, and we have
$(i_A^S)^{-1}(\mathfrak{q})=\mathfrak{p}$; by passing to the quotients, $i_A^S$ gives a monomorphism of the integral domains $A/\mathfrak{p}$ into the
field $A_\mathfrak{p}/\mathfrak{q}$, which identifies with the field of fractions of $A/\mathfrak{p}$.
\end{env}

\begin{env}{1.2.4}
The ring of fractions $S^{-1}A$ and the canonical homomorphism $i_A^S$ are a solution of a \emph{universal mapping problem}:
any homomorphism $u$ of $A$ into a ring $B$ such that $u(S)$ is composed of \emph{invertible} elements in $B$ factorizes in
one way
\[
  u\colon A\xrightarrow{i_A^S}S^{-1}A\xrightarrow{u^\ast}B
\]
\oldpage{14}where $u^\ast$ is a ring homomorphism. Under the same hypotheses, let $M$ be an
$A$-module, $N$ a $B$-module, $v\colon M\to N$ a homomorphism of $A$-modules (for the
$B$-module structure on $N$ defined by $u\colon A\to B$); then $v$ is factorizes in a single
way
\[
  v\colon M\xrightarrow{i_M^S}S^{-1}M\xrightarrow{v^\ast}N
\]
where $v^\ast$ is a homomorphism of $S^{-1}A$-modules (for the $S^{-1}A$-module structure 
on $N$ defined by $u^\ast$).
\end{env}

\begin{env}{1.2.5}
We define a canonical isomorphism $S^{-1}A\otimes_A M\xrightarrow{\sim}S^{-1}M$ of $S^{-1}A$-
modules, sending the element $(a/s)\otimes m$ to the element $(am)/s$, the isomorphism
reciprocally applying $m/s$ to $(1/s)\otimes m$.
\end{env}

\begin{env}{1.2.7}
When $A$ is an \emph{integral domain}, for which $K$ denotes its field of fractions, the canonical mapping $i_A^S\colon A\to S^{-1}A$
is injective for any multiplicative subset $S$ not containing $0$, and $S^{-1}A$ then identifies canonically with a subring of $K$
containing $A$. In particular, for every prime ideal $\mathfrak{p}$ of $A$ , $A_\mathfrak{p}$ is a local ring containing $A$,
with maximal ideal $\mathfrak{p}A_\mathfrak{p}$, and we have $\mathfrak{p}A_\mathfrak{p}\cap A=\mathfrak{p}$.
\end{env}

\begin{env}{1.2.8}
If $A$ is a \emph{reduced} ring \eref{1.1.1}, so is $S^{-1}A$: indeed, if $(x/s)^n=0$ for $x\in A$, $s\in S$, it means that
there exists $s'\in S$ such that $s'x^n=0$, hence $(s'x)^n=0$, which, by hypothesis, entails $s'x=0$, so $x/s=0$.
\end{env}

\end{document}

