\documentclass[../main.tex]{subfiles}

\begin{document}

\begin{env}{1.3.1}
Let $M$, $N$ be two $A$-modules, $u$ an $A$-homomorphism $M\to N$. If $S$ is a multiplicative subset of $A$,
we define a $S^{-1}A$-homomorphism $S^{-1}M\to S^{-1}N$, denoted by $S^{-1}u$, by putting
$S^{-1}u(m/s)=u(m)/s$; if $S^{-1}M$ and $S^{-1}N$ are canonically identified with $S^{-1}A\otimes_A M$ and
$S^{-1}A\otimes_A N$ \eref{1.2.5}, $S^{-1}u$ is considered as $1\otimes u$. If $P$ is a third $A$-module,
$v$ an $A$-homomorphism $N\to P$, we have $S^{-1}(v\circ u)=(S^{-1}v)\circ(S^{-1}u)$; in other words,
$S^{-1}M$ is a \emph{covariant functor in} $M$, of the category of $A$-modules into that of $S^{-1}A$-modules
($A$ and $S$ being fixed).
\end{env}

\begin{env}{1.3.2}
The functor $S^{-1}M$ is \emph{exact}; in other words, if the following
\[
  M\xrightarrow{u}N\xrightarrow{v}P
\]
is exact, so is the following
\[
  S^{-1}M\xrightarrow{S^{-1}u}S^{-1}N\xrightarrow{S^{-1}v}S^{-1}P.
\]
In particular, if $u\colon M\to N$ is injective (resp. surjective), the same is true for $S^{-1}u$;
\oldpage{15} if $N$ and $P$ are two
submodules of $M$, $S^{-1}N$ and $S^{-1}P$ identify canonically with submodules of $S^{-1}M$, and we have
\[
  S^{-1}(N+P)=S^{-1}N+S^{-1}P\quad\text{and}\quad S^{-1}(N\cap P)=(S^{-1}N)\cap(S^{-1}P).
\]
\end{env}

\begin{env}{1.3.3}
Let $(M_\alpha,\varphi_{\beta\alpha})$ be an inductive system of $A$-modules; then
$(S^{-1}M_\alpha,S^{-1}\varphi_{\beta\alpha})$ is an inductive system of $S^{-1}A$-modules.
Expressing the $S^{-1}M_\alpha$ and $S^{-1}\varphi_{\beta\alpha}$ as tensor products (\eref{1.2.5} and \eref{1.3.1}),
it follows from the permutability of tensor product and inductive limit operations that we have a canonical isomorphism
\[
  S^{-1}\varinjlim M_\alpha\xrightarrow{\sim}\varinjlim S^{-1}M_\alpha
\]
which is further expressed by saying that the functor $S^{-1}M$ (in $M$) \emph{commutes with inductive limits}.
\end{env}

\begin{env}{1.3.4}
Let $M$, $N$ be two $A$-modules; there is a canonical \emph{functorial} isomorphism (in $M$ and $N$)
\[
  (S^{-1}M)\otimes_{S^{-1}A}(S^{-1}N)\xrightarrow{\sim}S^{-1}(M\otimes_A N)
\]
which transforms $(m/s)\otimes(n/t)$ into $(m\otimes n)/st$.
\end{env}

\begin{env}{1.3.5}
We also have a \emph{functorial} homomorphism (in $M$ and $N$)
\[
  S^{-1}\Hom_A(M,N)\longrightarrow\Hom_{S^{-1}A}(S^{-1}M,S^{-1}N)
\]
which, at $u/s$, corresponds to the homomorphism $m/t\mapsto u(m)/st$. When $M$ has a finite presentation, the
preceding homomorphism is an \emph{isomorphism}: it is immediate when $M$ is of the form $A^r$, and goes on to the general
case starting from the following exact sequence $A^p\to A^q\to M\to 0$, and using the exactness of the functor $S^{-1}M$ and
the left-exactness of the functor $\Hom_A(M,N)$ in $M$. Note that this case always occurs when $A$ is \emph{Noetherian} and the
$A$-module $M$ is \emph{of finite type}.
\end{env}

\end{document}

