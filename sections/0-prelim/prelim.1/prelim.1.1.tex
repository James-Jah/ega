
\begin{env}{1.1.1}
\label{env-0.1.1.1}
Let $\mathfrak{a}$ be an ideal of a ring $A$; the \emph{radical} of $\mathfrak{a}$, denoted
by $\rad(\mathfrak{a})$, is the set of $x\in A$ such that $x^n\in\mathfrak{a}$ for
an integer $n>0$; it is an ideal containing $\mathfrak{a}$. We have
$\rad(\mathfrak{r}(\mathfrak{a}))=\rad(\mathfrak{a})$; the relation
$\mathfrak{a}\subset\mathfrak{b}$ leads to $\rad(\mathfrak{a})\subset\rad(\mathfrak{b})$;
the radical of a finite intersection of ideals is the intersection of their radicals. If $\varphi$
is a homomorphism of a ring $A'$ into $A$, then we have
$\rad(\varphi^{-1}(\mathfrak{a}))=\varphi^{-1}(\rad(\mathfrak{a}))$
for any ideal $\mathfrak{a}\subset A$. For an ideal to be the radical of an ideal,
it is necessary and sufficient that it be an intersection of prime ideals. The radical of an
ideal $\mathfrak{a}$ is the intersection of the
\emph{minimal} prime ideals among those containing $\mathfrak{a}$; if $A$ is
Noetherian, these minimal prime ideals are finite in number.

The radical of the ideal $(0)$ is also called the \emph{nilradical} of $A$; it is the set
$\nilrad$ of the nilpotent elements of $A$. It is said that the ring $A$ is \emph{reduced} if
$\nilrad=(0)$; for every ring $A$, the quotient $A/\nilrad$ of $A$ by its nilradical is a
reduced ring.
\end{env}

\begin{env}{1.1.2}
\label{env-0.1.1.2}
Recall that the \emph{nilradical} $\nilrad(A)$ of a ring $A$ (not necessarily commutative) is the
intersection of the maximal left ideals of $A$ (and also the intersection of maximal
right ideals). The nilradical of $A/\nilrad(A)$ is $(0)$.
\end{env}

