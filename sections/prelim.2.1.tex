\documentclass[../main.tex]{subfiles}

\begin{document}

\begin{cx}{2.1.1}
\oldpage{21}We say that a topological space $X$ is \emph{irreducible} if it is nonempty and
if it is not a union of two distinct closed subspaces of $X$. It is the same to say
that $X\neq\emp$ and that the intersection of two nonempty open sets (and consequently
of a finite number of open sets) of $X$ is nonempty, or that every nonempty open set
is everywhere dense, or that any closed subset is \unsure{\emph{rare}}, or finally
that all open sets of $X$ are \emph{connected}.
\end{cx}

\begin{cx}{2.1.2}
For a subspace $Y$ of a topological spave $X$ to be irreducible, it is necessary and
sufficient that its closure $\overline{Y}$ be irreducible. In particular, any subspace
which is the closure $\overline{\{x\}}$ of a singleton is irreducible;
we will express the relation $y\in\overline{\{x\}}$ (equivalent to
$\overline{\{y\}}\su\overline{\{x\}}$) by saying that there is a \emph{specialization of}
$x$ or that there is a \emph{generalization of} $y$. When there exists in an irreducible
space $X$ a point $x$ such that $X=\overline{\{x\}}$, we will say that $x$ is a
\emph{generic point} of $X$. Any nonempty open subset of $X$ then contains $x$, and any
subspace containing $x$ admits $x$ for a generic point.
\end{cx}

\begin{cx}{2.1.3}
Recall that a \emph{Kolmogoroff space} is a topological space $X$ satisfying the axiom
of separation:

$(T_0)$ If $x\neq y$ are any two points of $X$, there is an open set containing one of
the points $x$, $y$ and not the other.

If an irreducible Kolmogoroff space admits a generic point, it admits \emph{only} one
since a nonempty open set contains any generic point.

Recall that a topological space $X$ is said to be \emph{quasi-compact} if, from any
collection of open sets of $X$, one can extract a finite cover of $X$ (or, equivalently,
if any decreasing filter family of nonempty closed sets has a nonempty intersection). If
$X$ is a quasi-compact space, then any nonempty closed subset $A$ of $X$ contains a \emph{minimal}
nonempty closed set $M$, because the set of nonempty closed subsets of $A$ is inductive for
the relation $\supset$; if in addition $X$ is a Kolmogoroff space, $M$ is necessarily reduced
to a single point (or, as we say by abuse of language, is a \emph{closed point}).
\end{cx}

\begin{cx}{2.1.4}
In an irreducible space $X$, every nonempty open subspace $U$ is irreducible, and if $X$
admits a generic point $x$, $x$ is also a generic point of $U$.

Let $(U_\a)$ be a cover (whose set of indices is nonempty) of a topological space $X$,
consisting of nonempty open sets; if $X$ is irreducible, it is necessary and sufficient that
$U_\a$ is irreducible for all $\a$, and that $U_\a\cap U_\b\neq\emp$ for any
$\a$, $\b$. The condition is clearly necessary; to the that it is sufficient, it suffices
to prove that if $V$ is a nonempty open subset of $X$, then $V\cap U_\a$ is nonempty for all
$\a$, since then $V\cap U_\a$ is dense in $U_\a$ for all $\a$, and consequently
$V$ is dense in $X$. Now there is at least one index $\gamma$ such that $V\cap U_\gamma\neq\emp$,
so $V\cap U_\gamma$ is dense in $U_\gamma$, and as for all $\alpha$, $U_\a\cap V_\a\neq\emp$,
we also have $V\cap U_\a\cap U_\gamma\neq\emp$.
\end{cx}

\begin{cx}{2.1.5}
\oldpage{22}Let $X$ be an irreducible space, $f$ a continuous map of $X$ into a topological space $Y$.
Then $f(X)$ is irreducible, and if $x$ is a generic point of $X$, $f(x)$ is a generic point of
$f(X)$ and hence also of $\overline{f(X)}$. In particular, if in addition $Y$ is irreducible and
with a single generic point $y$, for $f(X)$ to be everywhere dense, it is necessary and sufficient
that $f(x)=y$.
\end{cx}

\begin{cx}{2.1.6}
Any irreducible subspace of a topological space $X$ is contained
in a maximal irreducible subspace, which is necessarily closed.
Maximal irreducible subspaces of $X$ are called the \emph{irreducible components} of $X$.
If $Z_1$, $Z_2$ are two irreducible components distinct from the space $X$, $Z_1\cap Z_2$ is a closed
\unsure{\emph{rare}} set in each of the subspaces $Z_1$, $Z_2$; in particular, if an irreducible component
of $X$ admits a generic point (2.1.2) such a point can not belong to any other
irreducible component. If $X$ has only a \emph{finite} number of irreducible 
components $Z_i$ ($1\leq i\leq n$), and if, for each $i$,we put $U_i=Z_i\cap\mathsf{C}(\bigcup_{j\neq i} Z_j)$,
the $U_i$ are open, irreducible, \unsure{disjoint}, and their union is dense in $X$.
Let $U$ be an open subset of a topological space $X$. If $Z$ is an irreducible subset of $X$
that intersects with $U$, $Z\cap U$ is open and dense in $Z$, thus irreducible; conversely, for any irreducible
closed subset $Y$ of $U$, the closure $\overline{Y}$ of $Y$ in $X$ is irreducible and $\overline{Y}\cap U=Y$.
We conclude that there is a \emph{bijective correspondence} between the irreducible components of $U$ and the
irreducible components of $X$ which intersect $U$.
\end{cx}

\begin{cx}{2.1.7}
If a topological space $X$ is a union of a \emph{finite} number of irreducible closed subspaces $Y_i$, the
irreducible components of $X$ are the maximal elements of the set of $Y_i$, because if $Z$ is an irreducible
closed subset of $X$, $Z$ is the union of $Z\cap Y_i$, from which one sees that $Z$ must be contained in one
of the $Y_i$. Let $Y$ be a subspace of a topological space $X$, and suppose that $Y$ has only a finite number
of irreducible components $Y_i$, ($1\leq i\leq n$); then the closures $\overline{Y_i}$ in $X$ are the
irreducible components of $Y$.
\end{cx}

\begin{cx}{2.1.8}
Let $Y$ be an irreducible space admitting a single generic point $y$.
Let $X$ be a topological space, $f$ a continuous mapping from $X$ to $Y$. Then, for
any irreducible component $Z$ of $X$ intersecting $f^{-1}(y)$, $f(Z)$ is dense in $Y$. The
converse is not necessarily true; however, if $Z$ has a generic point $z$,
and if $f(Z)$ is dense in $Y$, we must have $f(z)=y$ (2.1.5); in addition, $Z\cap f^{-1}(y)$
is then the closure of $\{z\}$ in $f^{-1}(y)$ and is therefore irreducible, and like any
irreducible subset of $f^{-1}(y)$ containing $z$ is necessarily contained in $Z$ (2.1.6), $z$ is
a generic point of $Z\cap f^{-1}(y)$. As any irreducible component of $f^{-1}(y)$ is
contained in an irreducible component of $X$, we see that if any irreducible component
$Z$ of $X$ intersecting $f^{-1}(y)$ admits a generic point, then there is a \emph{bijective correspondence}
between all these components and all the irreducible components
$Z\cap f^{-1}(y)$ of $f^{-1}(y)$, the generic points of $Z$ being identical to those of
$Z\cap f^{-1}(y)$.
\end{cx}

\end{document}

