\documentclass[../main.tex]{subfiles}

\begin{document}

\epigraph{\emph{To Oscar Zariski and Andr\'e Weil.}}

This memoir, and the many others that must follow, are intended for
to form a treaty on the foundations of algebraic geometry. They do not support
in principle, no particular knowledge of this discipline, and it has even been
that such knowledge, despite its obvious advantages, could sometimes (by habit
too exclusive from the birational point of view it implies) to be harmful to the one who
want to become familiar with the point of view and techniques presented here. However, we
assume that the reader has a good knowledge of the following topics:
\LS
\item \emph{Commutative algebra}, as it is exhibited for example in volumes
under preparation of the \emph{Elements} of N. Bourbaki (and, pending the publication of
these volumes, in Samuel-Zariski [13] and Samuel [n], [12]).
\item \emph{Homological algebra}, for which we refer to Cartan-Eilenberg [2]
(cit (M)) and Godement [4] (cit (G)), as well as the recent article by A. Grothendieck [6]
(cit (T)).
\item \emph{Sheaf Theory}, where our main references will be (G) and (T); this
last theory provides the essential language for interpreting in ``geometric'' terms
to the essential notions of commutative algebra, and to ``globalize'' them.
\item Finally, it will be useful for the reader to have some familiarity with
\emph{functorial language}, which will be constantly used in this Treatise, and for which the reader
consult (M), (G) and especially (T); the principles of this language and the main results
The general theory of functors will be described in more detail in a book in
preparation course by the authors of this Treaty.
\LE

\begin{center}
                $\ast~\ast~\ast$
\end{center}

It is not the place, in this Introduction, to give a more description or
less summary from the point of view of ``schema'' in algebraic geometry, nor the long
list of reasons which made it necessary to adopt it, and in particular the acceptance of
nilpotent elements in the local rings of ``manifolds'' that we
(which, of course, takes the notion of mapping into the background)
rational, in favor of that of regular mapping or morphism). This
Treaty aims precisely to develop systematically the language of schematics
and will demonstrate, we hope, its necessity. Although it would be easy to do so, we
will not try to give here an ``intuitive'' introduction to the notions
developed in Chapter 1. For the reader who would like to have a glimpse of the
preliminary study of the subject matter of this Treaty may be referred to the conference
A. Grothendieck at the International Congress of Mathematicians in Edinburgh in 1958 [y],
and the expose [8] of the same author. The work [14] (cit (FAG)) of J.-P. Serre can also
Be considered as an intermediary exposition between the classical point of view and the point
of schema in algebraic geometry, and as such, its reading may be
an excellent preparation to that of our \emph{Elements}.

\begin{center}
                $\ast~\ast~\ast$
\end{center}

(unfinished)

\end{document}

