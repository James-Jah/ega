\documentclass[../main.tex]{subfiles}

\begin{document}

\epigraph{\emph{To Oscar Zariski and Andr\'e Weil.}}

\oldpage{5}This memoir, and the many others that must follow, are intended to
form a treatise on the foundations of algebraic geometry. They do not assume, in
principle, any particular knowledge of this discipline, and it has even been
that such knowledge, despite its obvious advantages, could sometimes (by the
too-exclusive habit that the birational point of view it implies) to be harmful
to the one who wants to become familiar with the point of view and techniques
presented here. However, we assume that the reader has a good knowledge of the
following topics:
\begin{itemize}
  \item[(a)] \emph{Commutative algebra}, as it is exhibited for example in
             volumes under preparation of the \emph{Elements} of N.~Bourbaki
             (and, pending the publication of these volumes, in Samuel-Zariski
             \cite{13} and Samuel \cite{11}, \cite{12}).
  \item[(b)] \emph{Homological algebra}, for which we refer to Cartan-Eilenberg
             \cite{2} (cited as (M)) and Godement \cite{4} (cited as (G)), as well
             as the recent article by A. Grothendieck \cite{6} (cited as (T)).
  \item[(c)] \emph{Sheaf Theory}, where our main references will be (G) and (T);
             this theory provides the essential language for interpreting in
             ``geometric'' terms the essential notions of commutative algebra,
             and to ``globalize'' them.
  \item[(d)] Finally, it will be useful for the reader to have some familiarity with
             \emph{functorial language}, which will be constantly used in this Treatise,
             and for which the reader may consult (M), (G), and especially (T); the
             principles of this language and the main results of the general theory of
             functors will be described in more detail in a book currently in preparation
             by the authors of this Treatise.
\end{itemize}

\asttri

It is not the place, in this Introduction, to give a more or less summarily
description from the point of view of ``schemes'' in algebraic geometry, nor the
long list of reasons which made its adoption necessary, and in particular the
systematic acceptance of nilpotent elements in the local rings of ``manifolds''
that we consider (which necessarily shifts the idea of rational mappings into
the background, in favor of those of regular mappings or ``morphisms''). This
Treatise aims precisely to systematically develop the language of schemes, and
will demonstrate, we hope, its necessity. Although it would be easy to do so,
\oldpage{6}we will not try to give here an ``intuitive'' introduction to the
notions developed in Chapter 1. For the reader who would like to have a glimpse
of the preliminary study of the subject matter of this Treatise, we refer them
to the conference by A. Grothendieck at the International Congress of
Mathematicians in Edinburgh in 1958 \cite{7}, and the expose \cite{8} of the
same author. The work \cite{14} (cited as (FAC)) of J.-P. Serre can also be
considered as an intermediary exposition between the classical point of view and
the point of view of schemes in algebraic geometry, and, as such, its reading
may be an excellent preparation to that of our \emph{Elements}.

\asttri

We give below the general outline planned for this Treatise, subject to later
modifications, especially concerning the later chapters.

\begin{tabular}{rrl}
Chapter & I. & --- The language of schemes.\\
--- & II. & --- Elementary global study of some classes of morphisms.\\
--- & III. & --- Cohomology of algebraic coherent sheaves. Applications.\\
--- & IV. & --- Local study of morphisms.\\
--- & V. & --- Elementary procedures of constructing schemes.\\
--- & VI. & --- Descent. General method of constructing schemes.\\
--- & VII. & --- Schemes of groups, principal fibre bundles.\\
--- & VIII. & --- Differential study of fibre bundles.\\
--- & IX. & --- The fundamental group.\\
--- & X. & --- Residues and duality.\\
--- & XI. & --- Theories of intersection, Chern classes, Riemann-Roch theorem.\\
--- & XII. & --- Abelian schemes and Picard schemes.\\
--- & XIII. & --- Weil cohomology.
\end{tabular}\\

\bigskip

In principal, all chapters are considered open to changes, and supplementary
paragraphs can always be added later; such paragraphs would appear in separate
fascicles in order to minimise the inconveniences accompanying whatever mode of
publication adopted. When the writing of such a paragraph is foreseen or in
progress during the publication of a chapter, it will be mentioned in the
summary of the chapter in question, even if, owing to certain orders of urgency,
\unsure{its actual publication clearly ought to have been later.} For the use of
the reader, we give in ``Chapter 0'' the necessary tools in commutative algebra,
homological algebra, and sheaf theory, that will be used throughout this
Treatise, that are more or less well known but for which it was not possible to
give convenient references. It is recommended for the reader to not read
Chapter 0 except whilst reading the Treatise proper, when the results to which
we refer \oldpage{7}seem unfamiliar. Besides, we think that in this way, the
reading of this Treatise could be a good method for the beginner to familiarise
themselves with commutative algebra and homological algebra, whose study, when
not accompanied with tangible applications, is considered tedious, or even
depressing, by many.

\asttri

It is outside of our capabilities to give a historic overview, or even a summary
thereof, of the ideas and results described. The text will contain only those
references considered particularly useful for comprehension, and we indicate the
origin only of the most important results. Formally, at least, the subjects
discussed in our work are reasonably new, which explains the scarcity of
references made to the Fathers of algebraic geometry from the 19th to the
beginning of the 20th century, whose works we know only by hear-say. It is
suitable, however, to say some words here about the works which have most
directly influenced the authors and contributed to the development of
scheme-theoretic point of view. We absolutely must mention the fundamental work
(FAC) of J.-P.~Serre first, which has served as an introduction to algebraic
geometry for more that one young student (one of the authors of this Treatise
being one), deterred by the dryness of the classic \emph{Foundations} of A.~Weil
\cite{18}. It is there that it is shown, for the first time, that the ``Zariski
topology'' of an ``abstract'' algebraic variety is perfectly suited to applying
certain techniques from algebraic topology, and notably to be able to define a
cohomology theory. Further, the definition of an algebraic variety given
therein is that which translates most naturally to the idea that we develop
here\footnote{Just as J.-P.~Serre informed us, it is right to note that the idea
of defining the structure of a manifold by the data of a sheaf of rings is due
to H.~Cartan, who took this idea as the starting point of his theory of analytic
spaces. Of course, just as in algebraic geometry, it would be important in
``analytic geometry'' to give the right to use nilpotent elements in local rings
of analytic spaces. This extension of the definition of H.~Cartan and
J.-P.~Serre has recently been broached by H.~Grauert \cite{5}, and there is room
to hope that a systematic report of analytic geometry in this setting will soon
see the light of day. It is also evident that the ideas and techniques developed
in this Treatise retain a sense of analytic geometry, even though one must
expect more considerable technical difficulties in this latter theory. We can
foresee that algebraic geometry, by the simplicity of its methods, will be able
to serve as a sort of formal model for future developments in the theory of
analytic spaces.}. Serre himself had incidentally noted that the cohomology
theory of affine algebraic varieties could translate without difficulty by
replacing the affine algebras over a field by arbitrary commutative rings.
Chapters~I and II of this Treatise, and the first two paragraphs of chapter~III,
can thus be considered, for the most part, as easy translations, in this bigger
framework, of the principal results of (FAC) and a later article of the same
author \cite{15}. We have also vastly profited from the \emph{S{\'e}minaire de
G{\'e}om{\'e}trie alg{\'e}brique} de C.~Chevalley \cite{1}; in particular, the
systematic usage of ``constructible sets'' introduced by him has turned out to
be extremely useful in the theory of schemes (cf. chap.~IV). We have also
borrowed from him the study of morphisms from \oldpage{8}the point of view of
dimension (chap.~IV), that translates with negligible change to the framework of
schemes. It also merits noting that the idea of ``schemes of local rings'',
introduced by Chevalley, naturally lends itself to being extended to algebraic
geometry (not having, however, all the flexibility and generality that we intend
to give it here); for the connections between this idea and our theory, see the
chap.~I,~\textsection~8. One such extension has been developed by M.~Nagata in
a series of memoires \cite{9}, which contain many special results concerning
algebraic geometry over Dedekind rings\footnote{Amongst the works that come
close to our point of view of algebraic geometry, we pick out the work of
E.~K{\`a}hler \cite{22} and a recent note of Chow and Igusa \cite{3}, which go
back over certain results of (FAC) in the context of Nagata-Chevalley theory, as
well as giving a K{\"u}nneth formula.}.

\asttri

It goes without saying that a book on algebraic geometry, and especially a book
dealing with the fundamentals, is of course influenced, \completelyunsure, by
mathematicians such as O.~Zariski and A.~Weil. In particular, the
\emph{Th{\'e}orie des fonctions holomorphes} de Zariski \cite{20}, properly
flexible thanks to the cohomological methods and an existence theorem
(chap.~III,~\textsection\textsection~4~and~5), is (along with the method of
descent described in chap.~VI) one of the principal tools used in this Treatise,
and it seems to us one of the most powerful at our disposal in algebraic
geometry.

The general technique in which it is employed can be sketched as follows (a
typical example of which will be given in chap.~XI, in the study of the
fundamental group). We have a proper morphism (chap.~II) $f\colon X\to Y$ of
algebraic varieties (more generally, of schemes) that we wish to study on the
neighbourhood of a point $y\in Y$, with the aim of resolving a problem $P$
relative to a neighbourhood of $y$. We follow successive steps:
\begin{itemize}
  \item[\nth{1}] We can suppose that $Y$ is affine, such that $X$ becomes a scheme
                 defined on the affine ring $A$ of $Y$, and we can even replace $A$
                 by the local ring of $y$. This reduction is always easy in practice
                 (chap.~V) and brings us to the case where $A$ is a \emph{local} ring.
  \item[\nth{2}] We study the problem in question when $A$ is a local \emph{Artinien}
                 ring. So that the problem $P$ still makes sense when $A$ is not
                 assumed to be integral, sometimes we have to reformulate $P$, and it
                 appears that we often thus obtain a better understanding of the problem
                 during this stage, in an ``infinitesimal'' way.
  \item[\nth{3}] The theory of formal schemes (chap.~III,~\textsection\textsection~3,~4,~and~5)
                 lets us pass from the case of an Artinien ring to a \emph{complete local ring}.
  \item[\nth{4}] Finally, if $A$ is an arbitrary local ring, considering
                 ``\unsure{multiform} sections'' of suitable schemes over $X$
                 approximates the idea of a given ``formal'' section (chap.~IV), and
                 this will let us pass, \oldpage{9}by extension of scalars to the
                 completion of $A$, from a known result of \completelyunsure to an
                 analogous result for a finite simple (e.g. unramified) extension of $A$.
\end{itemize}

This sketch shows the importance of the systematic study of schemes defined over
an artinien ring $A$. The point of view of Serre in his formulation of the
theory of \unsure{local class fields}, and the recent works of Greenberg, seem
to suggest that such a study could be undertaken by functorially attaching, to
some such scheme $X$, a scheme $X'$ over the residue field $k$ of $A$ (assumed
\unsure{perfect}) of dimension equal (in nice cases) to $n\dim X$, where $n$ is
the \unsure{height} of $A$.

As for the influence of A.~Weil, it suffices to say that it is the need to
develop the tools necessary to formulate, with full generality, the definition
of ``Weil cohomology,'' and to tackle the proof\footnote{To avoid any
misunderstanding, we point out that this task has barely been undertaken at the
moment of writing this Introduction, and still hasn't led to the proof of the
Weil conjectures.} of all the formal properties necessary to establish the
famous conjectures in diophantine geometry \cite{19}, that has been one of the
principal motivations of the writing of this Treatise, as has the desire to find
the natural setting of the usual ideas and methods of algebraic geometry, and to
give the authors the chance to understand these ideas and methods.

\asttri

To finish, we believe it useful to warn the reader that, as was the case with
all the authors themselves, they will almost certainly have difficulty before
becoming accustomed to the language of schemes, and to convince themselves that
the usual constructions that suggest geometric intuition can be translated, in
essentially only one sensible way, to this language. As in many parts of modern
mathematics, the first intuition seems further and further away, in appearance,
from the correct language needed to express the mathematics in question with
complete precision and the desired level of generality. In practice, the
psychological difficulty comes from the need to replicate some familiar
set-theoretic constructions to a category that is already quite different from
that of sets (the category of preschemes, or the category of preschemes over a
given prescheme): cartesian products, group laws, ring laws, module laws, fibre
bundles, principal homogeneous fibre bundles, etc. \unsure{It will most likely
be difficult for the mathematician, in the future, to shy away from this new
effort of abstraction, maybe rather negligible, on the whole, in comparison with
that endowed by our fathers, to familiarise themselves with the Theory of Sets.}

\asttri

The references are given following the numerical system; for example, in
III,~4.9.3, the III indicates the chapter, the 4 the paragraph, the 9 the
section of the paragraph. If we reference a chapter from within itself then we
omit the chapter number.

\bigskip

\oldpage{10}\emph{Page 10 in the original is left blank. [trans.]}

\end{document}

