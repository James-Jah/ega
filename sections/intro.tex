\documentclass[../main.tex]{subfiles}

\begin{document}

\epigraph{\emph{To Oscar Zariski and Andr\'e Weil.}}

This memoir, and the many others that must follow, are intended
to form a treatise on the foundations of algebraic geometry. They do not assume,
in principle, any particular knowledge of this discipline, and it has even been
that such knowledge, despite its obvious advantages, could sometimes (by the
too-exclusive habit that the birational point of view it implies) to be harmful 
to the one who wants to become familiar with the point of view and techniques presented here. However, we assume that the reader has a good knowledge of the following topics:
\LS
\item \emph{Commutative algebra}, as it is exhibited for example in volumes
under preparation of the \emph{Elements} of N. Bourbaki (and, pending the
publication of these volumes, in Samuel-Zariski [13] and Samuel [11], [12]).
\item \emph{Homological algebra}, for which we refer to Cartan-Eilenberg [2]
(cited as (M)) and Godement [4] (cited as (G)), as well as the recent article by
A. Grothendieck [6] (cited as (T)).
\item \emph{Sheaf Theory}, where our main references will be (G) and (T); this
theory provides the essential language for interpreting in ``geometric'' terms
the essential notions of commutative algebra, and to ``globalize'' them.
\item Finally, it will be useful for the reader to have some familiarity with
\emph{functorial language}, which will be constantly used in this Treatise, and
for which the reader may consult (M), (G) and especially (T); the principles of
this language and the main results of the general theory of functors will be
described in more detail in a book currently in preparation by the authors of
this Treatise.
\LE

\asttri

It is not the place, in this Introduction, to give a more or less summarily
description from the point of view of ``schemes'' in algebraic geometry, nor the
long list of reasons which made its adoption necessary, and in particular the
systematic acceptance of nilpotent elements in the local rings of ``manifolds''
that we consider (which necessarily shifts the idea of rational mappings into
the background, in favor of those of regular mappings or ``morphisms''). This
Treatise aims precisely to systematically develop the language of schemes, and
will demonstrate, we hope, its necessity. Although it would be easy to do so, we
will not try to give here an ``intuitive'' introduction to the notions
developed in Chapter 1. For the reader who would like to have a glimpse of the
preliminary study of the subject matter of this Treatise, we refer them to the
conference by A. Grothendieck at the International Congress of Mathematicians in 
Edinburgh in 1958 [7], and the expose [8] of the same author. The work [14]
(cited as (FAC)) of J.-P. Serre can also be considered as an intermediary exposition between the classical point of view and the point of view of schemes
in algebraic geometry, and, as such, its reading may be an excellent preparation
to that of our \emph{Elements}.

\asttri

We give below the general outline planned for this Treaty, subject to later modifications, especially concerning the later chapters.

\begin{tabular}{rrl}
    Chapter & I. & --- The language of schemes.\\
    --- & II. & --- Elementary global study of some classes of morphisms.\\
    --- & III. & --- Cohomology of algebraic coherent sheaves. Applications.\\
    --- & IV. & --- Local study of morphisms.\\
    --- & V. & --- Elementary procedures of constructing schemes.\\
    --- & VI. & --- Descent. General method of constructing schemes.\\
    --- & VII. & --- Schemes of groups, principal fibre bundles.\\
    --- & VIII. & --- Differential study of fibre bundles.\\
    --- & IX. & --- The fundamental group.\\
    --- & X. & --- Residues and duality.\\
    --- & XI. & --- Theories of intersection, Chern classes, Riemann-Roch theorem.\\
    --- & XII. & --- Abelian schemes and Picard schemes.\\
    --- & XIII. & --- Weil cohomology.
\end{tabular}

In principal, all chapters are considered open to changes, and supplementary paragraphs can always be added later; such paragraphs would appear in separate fascicles in order to minimise the inconveniences accompanying whatever mode of publication adopted.
When the writing of such a paragraph is foreseen or in progress during the publication of a chapter, it will be mentioned in the summary of the chapter in question, even if, owing to certain orders of urgency, \unsure{its actual publication clearly ought to have been later.}
For the use of the reader, we give in ``Chapter 0'' the necessary tools in commutative algebra, homological algebra, and sheaf theory, that will be used throughout this Treatise, that are more or less well known but for which it was not possible to give convenient references.
It is recommended for the reader to not read Chapter 0 except whilst reading the Treatise proper, when the results to which we refer seem unfamiliar.
Besides, we think that in this way, the reading of this Treatise could be a good method for the beginner to familiarise themselves with commutative algebra and homological algebra, whose study, when not accompanied with tangible applications, is considered tedious, or even depressing, by many.

\asttri

TODO

\asttri

TODO

\asttri

To finish, we believe it useful to warn the reader that, as was the case with all the authors themselves, they will almost certainly have difficulty before becoming accustomed to the language of schemes, and to convince themselves that the usual constructions that suggest geometric intuition can be translated, in essentially only one sensible way, to this language.
As in many parts of modern mathematics, the first intuition seems further and further away, in appearance, from the correct language needed to express the mathematics in question with complete precision and the desired level of generality.
In practice, the psychological difficulty comes from the need to replicate some familiar set-theoretic constructions to a category that is already quite different from that of sets (the category of preschemes, or the category of preschemes over a given prescheme): cartesian products, group laws, ring laws, module laws, fibre bundles, principal homogeneous fibre bundles, etc.
\unsure{It will most likely be difficult for the mathematician, in the future, to shy away from this new effort of abstraction, maybe rather negligible, on the whole, in comparison with that endowed by our fathers, to familiarise themselves with the Theory of Sets.}


\asttri

The references are given following the numerical system; for example, in III,~4.9.3, the III indicates the chapter, the 4 the paragraph, the 9 the section of the paragraph.
If we reference a chapter from within itself then we omit the chapter number.

\end{document}

