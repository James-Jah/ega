
\begin{env}{1.2.1}
\label{env-1.1.2.1}
Let $A$, $A'$ be two rings,
\[
  \varphi\colon A'\longrightarrow A
\]
a homomorphism of rings. For each prime ideal $x=\mathfrak{j}_x\in\Spec(A)=X$, the
ring $A'/\varphi^{-1}(\mathfrak{j}_x)$ is canonically isomorphic to a subring of
$A/\mathfrak{j}_x$, so it is an integral domain, otherwise we say
$\varphi^{-1}(\mathfrak{j}_x)$ is a prime ideal of $A'$; we denote it by
$^a\varphi(x)$, and we have also defined a map
\[
  ^a\varphi\colon X=\Spec(A)\longrightarrow X'=\Spec(A')
\]
(also denoted $\Spec(\varphi)$) we call the map \emph{associated} to the
homomorphism $\varphi$. We denote by $\varphi^x$ the injective homomorphism of
$A'/\varphi^{-1}(\mathfrak{j}_x)$ to $A/\mathfrak{j}_x$ induced by $\varphi$ by
passing to quotients, so the canonical extention is a monomorphism of fields
\[
  \varphi^x\colon\k(^a\varphi(x))\longrightarrow\k(x);
\]
for each $f'\in A'$, we therefore have by definition
\[
  \varphi^x(f'(^a\varphi(x)))=(\varphi(f'))(x)\quad(x\in X).\tag{1.2.1.1}
\]
\end{env}

\begin{env}[Proposition]{1.2.2}
\label{prop-1.1.2.2}
\medskip\noindent
\begin{itemize}
  \item[(i)] For each subset $E'$ of $A'$, we have
             \[
               ^a\varphi^{-1}(V(E'))=V(\varphi(E')),\tag{1.2.2.1}
             \]
             and in particular, for each $f'\in A'$,
             \[
               ^a\varphi^{-1}(D(f'))=D(\varphi(f')).\tag{1.2.2.2}
             \]
  \item[(ii)] For each ideal $\mathfrak{a}$ of $A$, we have
              \[
                \overline{^a\varphi(V(\mathfrak{a}))}=V(\varphi^{-1}(\mathfrak{a})).
                \tag{1.2.2.3}
              \]
\end{itemize}
\end{env}

In effect, the relation $^a\varphi(x)\in V(E')$ is by definition equivalent to
$E'\subset\varphi^{-1}(\mathfrak{j}_x)$, so $\varphi(E')\subset\mathfrak{j}_x$, and
finally $x\in V(\varphi(E'))$, hence (i). To prove (ii), we can suppose that
$\mathfrak{a}$ is equal to its radical, since $V(\rad(\mathfrak{a}))=V(\mathfrak{a})$
(\sref{prop}{1.1.2}, (v)) and
$\varphi^{-1}(\rad(\mathfrak{a}))=\rad(\varphi^{-1}(\mathfrak{a}))$; the relation
$f'\in\mathfrak{a}'$ is by definition equivalent to $f'(x')=0$ for each
$x\in{^a\varphi(Y)}$, so, by virtue of the formula (1.2.1.1), it is equivalent as well
to $\varphi(f')(x)=0$ for each $x\in Y$, or $\varphi(f')\in\mathfrak{j}(Y)=\mathfrak{a}$,
since $\mathfrak{a}$ is equal to its radical; hence (ii).



