
\begin{env}{1.1.1}
\label{env-1.1.1.1}
\oldpage{80}\emph{Notation}. Let $A$ be a (commutative) ring, $M$ an $A$-module. In
this chapter and the following, we will constantly use the following notations:
\begin{itemize}
  \item[] $\Spec(A)=$ \emph{set of prime ideals} of $A$, also called the
          \emph{prime spectrum} of $A$; for an $x\in X=\Spec(A)$, it will often be
          convenient to write $\mathfrak{j}_x$ instead of $x$. When $\Spec(A)$ is
          \emph{empty}, it is necessary and sufficient that the ring $A$ is
          reduced to $0$.
  \item[] $A_x=A_{\mathfrak{j}_x}=$ \emph{(local) ring of fractions $S^{-1}A$},
          where $S=A-\mathfrak{j}_x$.
  \item[] $\mathfrak{m}_x=\mathfrak{j}_x A_{\mathfrak{j}_x}=$ \emph{maximal ideal of $A$}.
  \item[] $\k(x)=A_x/\mathfrak{m}_x=$ \emph{residue field of $A_x$},
          canonically isomorphic to the field of fractions
          of the integral ring $A/\mathfrak{j}_x$, to which it is identified.
  \item[] $f(x)=$ \emph{class of $f$} mod. $\mathfrak{j}_x$ in $A/\mathfrak{j}_x\subset\k(x)$,
          for $f\in A$ and $x\in X$. We still say that $f(x)$ is the \emph{value}
          of $f$ at a point $x\in\Spec(A)$; the relations $f(x)=0$ and $f\in\mathfrak{j}_x$ are
          \emph{equivalent}.
  \item[] $M_x=M\otimes_A A_x=$ \emph{module of denominators of fractions in $A-\mathfrak{j}_x$}.
  \item[] $\rad(E)=$ \emph{radical of the ideal of $A$ generated by a subset $E$ of $A$}.
  \item[] $V(E)=$ \emph{set of $x\in X$ such that $E\subset\mathfrak{j}_x$} (or the set of $x\in X$
          such that $f(x)=0$ for all $f\in E$), for $E\subset A$. So we have
          \[
            \rad(E)=\bigcap_{x\in V(E)}\mathfrak{j}_x.\tag{1.1.1.1}
          \]
  \item[] $V(f)=V(\{f\})$ for $f\in A$.
  \item[] $D(f)=X-V(f)=$ \emph{set of $x\in X$ where $f(x)\neq 0$}.
\end{itemize}
\end{env}

\begin{env}[Proposition]{1.1.2}
\label{env-1.1.1.2}
We have the following properties:
\begin{itemize}
  \item[(i)] $V(0)=X$, $V(1)=\emp$.
  \item[(ii)] The relation $E\subset E'$ implies $V(E)\supset V(E')$.
  \item[(iii)] For each family $(E_\lambda)$ of subsets of $A$,
               $V(\bigcup_\lambda E_\lambda)=V(\sum_\lambda E_\lambda)=\bigcap_\lambda V(E_\lambda)$.
  \item[(iv)] $V(EE')=V(E)\cup V(E')$.
  \item[(v)] $V(E)=V(\mathfrak{r}(E))$.
\end{itemize}
\end{env}
The properties (i), (ii), (iii) are trivial, and (v) follows from (ii) and from the
formula (1.1.1.1). It is evident that $V(EE')\supset V(E)\cap V(E')$; conversely, if
$x\not\in V(E)$ and $x\not\in V(E')$, there exists $f\in E$ and $f'\in E'$ such that
$f(x)\neq 0$ and $f'(x)\neq 0$ in $\k(x)$, hence $f(x)f'(x)\neq 0$, i.e., $x\not\in V(EE')$,
which proves (iv).

The proposition \eref{1.1.2} shows, among other things, that sets of the form $V(E)$ (where
$E$ runs through all the subsets of $A$) are the \emph{closed sets} of a topology on
$X$, which we will call the \emph{spectral topology}\footnote{The introduction of this
topology in algebraic geometry is due to Zariski. So this topology is usually called
the ``Zariski topology'' of $X$.}; unless expressely stated otherwise, always assume
$X=\Spec(A)$ with the spectral topology.

\begin{env}{1.1.3}
\label{env-1.1.1.3}
\oldpage{81}For each subset $Y$ of $X$, we denote by $\mathfrak{j}(Y)$ the set of $f\in A$
such that $f(y)=0$ for all $y\in Y$; equivalently, $\mathfrak{j}(Y)$ is the intersection of
the prime ideals $\mathfrak{j}_y$ for $y\in Y$. It is clear that the relation $Y\subset Y'$
implies that $\mathfrak{j}(Y)\supset\mathfrak{j}(Y')$ and that we have
\[
  \mathfrak{j}\bigg(\bigcup_\lambda Y_\lambda\bigg)=\bigcap_\lambda\mathfrak{j}(Y_\lambda)\tag{1.1.3.1}
\]
for each family $(Y_\lambda)$ of subsets of $X$. Finally we have
\[
  \mathfrak{j}(\{x\})=\mathfrak{j}_x.\tag{1.1.3.2}
\]
\end{env}

\begin{env}[Proposition]{1.1.4}
\label{prop-1.1.1.4}
\begin{itemize}
  \item[(i)] For each subset $E$ of $A$, we have $\mathfrak{j}(V(E))=\rad(E)$.
  \item[(ii)] For each subset $Y$ of $X$, $V(\mathfrak{j}(Y))=\overline{Y}$, the closure of $Y$ in $X$.
\end{itemize}
\end{env}
(i) is an immeidate consequence of the definitions and (1.1.1.1); on the other hand, $V(\mathfrak{j}(Y))$
is closed and contains $Y$; conversely, if $Y\subset V(E)$, we have $f(y)=0$ for $f\in E$ and all $y\in Y$,
so $E\subset\mathfrak{j}(Y)$, $V(E)\supset V(\mathfrak{j}(Y))$, which proves (ii).

\begin{env}[Corollary]{1.1.5}
\label{cor-1.1.1.5}
The closed subsets of $X=\Spec(A)$ and the ideals of $A$ equal to their radicals (otherwise the
intersection of prime ideals) correspond bijectively by the \unsure{descent} maps $Y\mapsto\mathfrak{j}(Y)$,
$\mathfrak{a}\mapsto V(\mathfrak{a})$; the union $Y_1\cup Y_2$ of two closed subsets corresponds to
$\mathfrak{j}(Y_1)\cap\mathfrak{j}(Y_2)$, and the intersection of any family $(Y_\lambda)$ of closed subsets
corresponds to the radical of the sum of the $\mathfrak{j}(Y_\lambda)$.
\end{env}

\begin{env}[Corollary]{1.1.6}
\label{cor-1.1.1.6}
If $A$ is a Noetherian ring, $X=\Spec(A)$ is a Noetherian space.
\end{env}
Note that the converse of this corollary is false, as shown
in the example of a non-Noetherian integral ring with a single prime ideal $\neq\{0\}$, for
example a non-discrete valuation ring of rank $1$.

As an example of ring $A$ whose spectrum is not a Noetherian space, one
can consider the ring $\sheaf{C}(Y)$ of continuous real functions on an infinite compact space
$Y$; we know that as a whole, $Y$ corresponds with the set of maximal ideals
of $A$, and it is easy to see that the topology induced on $Y$ by that of $X=\Spec(A)$
is the initial topology of $Y$. Since $Y$ is not a Noetherian space, the same is true for $X$.

\begin{env}[Corollary]{1.1.7}
\label{cor-1.1.1.7}
For each $x\in X$, the closure of $\{x\}$ is the set of $y\in X$ such that $\mathfrak{j}_x\subset\mathfrak{j}_y$.
For $\{x\}$ to beclosed, it is necessary and sufficient that $\mathfrak{j}_x$ is maximal.
\end{env}

\begin{env}[Corollary]{1.1.8}
\label{cor-1.1.1.8}
The space $X=\Spec(A)$ is a Kolmogoroff space.
\end{env}

If $x$, $y$ are two distinct points of $X$, we have either $\mathfrak{j}_x\not\subset\mathfrak{j}_y$ or
$\mathfrak{j}_y\not\subset\mathfrak{j}_x$, so one of the points $x$, $y$ does not belong to the closure of the other.

\begin{env}{1.1.9}
\label{env-1.1.1.9}
According to proposition (\eref{1.1.2}, (iv)), for two elements $f$, $g$ of $A$, we have
\[
  D(fg)=D(f)\cap D(g).\tag{1.1.9.1}
\]
Note also that the relation $D(f)=D(g)$ means, according to proposition (\sref{prop}{1.1.4}, (i))
and proposition (\eref{1.1.2}, (v)) that $\rad(f)=\rad(g)$, or that the minimal prime ideals
containing $(f)$ and $(g)$ are the same; in particular, when $f=ug$, where $u$ is invertible.
\end{env}

\begin{env}[Proposition]{1.1.10}\oldpage{82}
\label{prop-1.1.1.10}
\medskip\noindent
\begin{itemize}
  \item[(i)] When $f$ ranges over $A$, the sets $D(f)$ forms a base for the topology of $X$.
  \item[(ii)] For each $f\in A$, $D(f)$ is quasicompact. In particular $X=D(1)$ is quasicompact.
\end{itemize}
\end{env}

(i) Let $U$ be an open set in $X$; by definition, we have $U=X-V(E)$ where $E$ is a subset of $A$, and
$V(E)=\bigcap_{f\in E}V(f)$, hence $U=\bigcup_{f\in E}D(f)$.

(ii) According to (i), it is sufficient to prove that if $(f_\lambda)_{\lambda\in L}$ is a family of elements
of $A$ such that $D(f)\subset\bigcup_{\lambda\in L}D(f_\lambda)$, there exists a finite subset $J$ of $L$ such that
$D(f)\subset\bigcup_{\lambda\in J}D(f_\lambda)$. Let $\mathfrak{a}$ be the ideal of $A$ generated by the $f_\lambda$;
we have by hypothesis that $V(f)\supset V(\mathfrak{a})$, so $\rad(f)\subset\rad(\mathfrak{a})$; as
$f\in\rad(f)$, there exists an integer $n\geq 0$ such that $f^n\in\mathfrak{a}$. But then $f^n$ belongs to
the ideal $\mathfrak{b}$ generated by the finite subfamily $(f_\lambda)_{\lambda\in J}$, and we have
$V(f)=V(f^n)\supset V(\mathfrak{b})=\bigcap_{\lambda\in J}V(f_\lambda)$, that is to say,
$D(f)\supset\bigcup_{\lambda\in J}D(f_\lambda)$.

\begin{env}[Proposition]{1.1.11}
\label{prop-1.1.1.11}
For each ideal $\mathfrak{a}$ of $A$, $\Spec(A/\mathfrak{a})$ identifies canonically with the closed subspace
$V(\mathfrak{a})$ of $\Spec(A)$.
\end{env}

In effect, we know there is a canonical bijective correspondence, respecting the inclusion order structure,
between ideals (resp. prime ideals) of $A/\mathfrak{a}$ and ideals (resp. prime ideals) of $A$ containing
$\mathfrak{a}$.

Recall that the set $\nilrad$ of nilpotent elements of $A$ (the \emph{nilradical} of $A$) is an ideal equal to
$\rad(0)$, the intersection of all the prime ideals of $A$ \pref{env}{1.1.1}.

\begin{env}[Corollary]{1.1.12}
\label{cor-1.1.1.12}
The topological spaces $\Spec(A)$ and $\Spec(A/\nilrad)$ are canonically homeomorphic.
\end{env}

\begin{env}[Proposition]{1.1.13}
\label{prop-1.1.1.13}
For $X=\Spec(A)$ to be irreducible \pref{env}{2.1.1}, it is necessary and sufficient that the ring $A/\nilrad$
is an integral domain (or, equivalently, that the ideal $\nilrad$ is prime).
\end{env}

By virtue of Corollary \sref{cor}{1.1.12}, we can restrict to the case $\nilrad=0$. If $X$ is reducible,
there exist two distinct closed subsets $Y_1$, $Y_2$ of $X$ such that $X=Y_1\cup Y_2$, so
$\mathfrak{j}(X)=\mathfrak{j}(Y_1)\cap\mathfrak{j}(Y_2)=0$, the ideals $\mathfrak{j}(Y_1)$ and
$\mathfrak{j}(Y_2)$ being different from $(0)$ \sref{cor}{1.1.5}; so $A$ is not an integral domain.
Conversely, if in $A$ there are elements $f\neq 0$, $g\neq 0$ such that $fg=0$, we have
$V(f)\neq X$, $V(g)\neq X$ (since the intersection of the prime ideals of $A$ is $(0)$), and
$X=V(fg)=V(f)\cup V(g)$.

\begin{env}[Corollary]{1.1.14}
\label{cor-1.1.1.14}
\begin{itemize}
  \item[(i)] In the bijective correspondence between closed subsets of $X=\Spec(A)$
             and ideals of $A$ equal to their roots, the irreducible closed subsets
             of $X$ correspond to the prime ideals of $A$. In particular, the irreducible
             components of $X$ correspond to the minimal prime ideals of $A$.
  \item[(ii)] The map $x\mapsto\overline{\{x\}}$ establishes a bijective correspondence
              between $X$ and the set of closed irreducible subsets of $X$
              (\emph{said otherwise,} all closed irreducible subsets of $X$ containing
              only one generic point).
\end{itemize}
\end{env}

(i) follows immediately from \sref{prop}{1.1.13} and \sref{prop}{1.1.11}; and for
proving (ii), we can, by virtue of \sref{prop}{1.1.11}, we restrict to the case where
$X$ is irreducible; then, according to the proposition \sref{prop}{1.1.13}, there exists
in $A$ a smaller prime ideal $\nilrad$, which corresponds to the generic point
\oldpage{83}of $X$; in addition, $X$ does not admit only one generic point since it is
a Kolmogoroff space (\sref{cor}{1.1.8} and \pref{env}{2.1.3}).

\begin{env}[Proposition]{1.1.15}
\label{prop-1.1.1.15}
If $\mathfrak{J}$ is an ideal in $A$ containing the radical $\nilrad(A)$, the only
neighborhood of $V(\mathfrak{J})$ in $X=\Spec(A)$ is the whole space $X$
\end{env}

In effect, each maximal ideal of $A$ belongs by definitionn of $V(\mathfrak{J})$.
As each ideal $\mathfrak{a}$ of $A$ is contained in a maximal ideal, we have
$V(\mathfrak{a})\cap V(\mathfrak{J})\neq 0$, hence the proposition.

