\documentclass[../main.tex]{subfiles}

\begin{document}

\begin{cx}[Proposition]{9.1.1}
    \oldpage{169}Let $X$ be a prescheme (\emph{resp.} a locally Noetherian prescheme).
    Let $\mathscr{F}$ and $\mathscr{G}$ be two quasi-coherent (\emph{resp.} coherent) $\O_X$-modules; then $\mathscr{F}\otimes_{\O_X}\mathscr{G}$ is quasi-coherent (\emph{resp.} coherent) and of finite type if $\mathscr{F}$ and $\mathscr{G}$ are of finite type.
    If $\mathscr{F}$ admits a finite presentation and if $\mathscr{G}$ is quasi-coherent (\emph{resp.} coherent), then $\sheafHom(\mathscr{F},\mathscr{G})$ is quasi-coherent (\emph{resp.} coherent).
\end{cx}

Being a local property, we can suppose that $X$ is affine (resp. Noetherian affine); further, if $\mathscr{F}$ is coherent, then we can assume that it is the cokernel of a homomorphism $\O_X^m\to\O_X^n$.
The claims pertaining to quasi-coherent sheaves then follow from (1.3.12) and (1.3.9); the claims pertaining to coherent sheaves follow from (1.5.1) and from the fact that, if $M$ and $N$ are modules of finite type over a Noetherian ring $A$, $M\otimes_A N$ and $\Hom_A(M,N)$ are $A$-modules of finite type.

\begin{cx}[Definition]{9.1.2}
    Let $X$ and $Y$ be two $S$-preschemes, $p$ and $q$ the projections of $X\times_S Y$, and $\mathscr{F}$ (resp. $\mathscr{G}$) a quasi-coherent $\O_X$-module (resp. quasi-coherent $\O_Y$-module).
    We define the tensor product of $\mathscr{F}$ and $\mathscr{G}$ over $\O_S$ (\emph{or} over $S$), denoted by $\mathscr{F}\otimes_{\O_S}\mathscr{G}$ (\emph{or} $\mathscr{F}\otimes_S\mathscr{G}$) to be the tensor product $p^*(\mathscr{F})\otimes_{\O_{X\times_S Y}}q^*(\mathscr{G})$ over the prescheme $X\times_S Y$.
\end{cx}

If $X_i$ ($1\leqslant i\leqslant n$) are $S$-preschemes, and $\mathscr{F}_i$ are quasi-coherent $\O_{X_i}$-modules ($1\leqslant i\leqslant n$), then we define similarly the tensor product $\mathscr{F}_1\otimes_S\mathscr{F}_2\otimes_S\ldots\otimes_S\mathscr{F}_n$ over the prescheme $Z=X_1\times_S X_2\times_s\ldots\times_S X_n$; it is a \emph{quasi-coherent} $\O_Z$-module by virtue of (9.1.1) and (\textbf{0},~5.1.4); it is \emph{coherent} if the $\mathscr{F}_i$ are coherent and $Z$ is \emph{locally Noetherian}, by virtue of (9.1.1), (\textbf{0},~5.3.11), and (6.1.1).

Note that if we take $X=Y=S$ then definition (9.1.2) gives us back the tensor product of $\O_S$-modules.
Furthermore, as $q^*(\O_Y)=\O_{X\times_S Y}$ (\textbf{0},~4.3.4), the product $\mathscr{F}\otimes_S\O_Y$ is canonically identified with $p^*(\mathscr{F})$, and, in the same way, $\O_X\otimes_S\mathscr{G}$ is canonically identified with $q^*(\mathscr{G})$.
In particular, if we take $Y=S$ and denote by $f$ the structure morphism $X\to Y$, we have that $\O_X\otimes_Y\mathscr{G}=f^*(\mathscr{G})$: the ordinary tensor product and the inverse image thus appear as particular cases of the general tensor product.

Definition (9.1.2) leads immediately to the fact that, for fixed $X$ and $Y$, $\mathscr{F}\otimes_S\mathscr{G}$ is an \emph{additive covariant bifunctor that is right-exact} in $\mathscr{F}$ and $\mathscr{G}$.

\begin{cx}[Proposition]{9.1.3}
    Let $S,X,Y$ be three affine schemes of rings $A,B,C$ (respectively), with $B$ and $C$ being $A$-algebras.
    Let $M$ (\emph{resp.} $N$) be a $B$-module (\emph{resp.} $C$-module), and $\mathscr{F}=\widetilde{M}$ (\emph{resp.} $\mathscr{G}=\widetilde{N}$) the associated quasi-coherent sheaf; then $\mathscr{F}\otimes_S\mathscr{G}$ is canonically isomorphic to the sheaf associated to the $(B\otimes_A C)$-module $M\otimes_A N$.
\end{cx}

\oldpage{170}In fact, by virtue of (1.6.5), $\mathscr{F}\otimes_S\mathscr{G}$ is canonically isomorphic to the sheaf associated to the $(B\otimes_A C)$-module
\begin{equation*}
    \big( M\otimes_B(B\otimes_A C) \big) \otimes_{B\otimes_A C} \big( (B\otimes_A C)\otimes_C N \big)
\end{equation*}
and by the canonical isomorphisms between tensor products, this latter module is isomorphic to
\begin{equation*}
    M\otimes_B(B\otimes_A C)\otimes_C N = (M\otimes_B B)\otimes_A(C\otimes_C N) = M\otimes_A N.
\end{equation*}

\begin{cx}[Proposition]{9.1.4}
    Let $f\colon T\to X$, and $g\colon T\to Y$ be two $S$-morphisms, and $\mathscr{F}$ (\emph{resp.} $\mathscr{G}$) a quasi-coherent $\O_X$-module (\emph{resp.} quasi-coherent $\O_Y$-module).
    Then
    \begin{equation*}
        (f,g)^*_S(\mathscr{F}\otimes_S\mathscr{G}) = f^*(\mathscr{F})\otimes_{\O_T}g^*(\mathscr{G}).
    \end{equation*}
\end{cx}

If $p,q$ are the projections of $X\times_S Y$, then the formula in fact follows from the relations $(f,g)^*_S\circ p^*=f^*$ and $(f,g)^*_S\circ q^*=g^*$ (\textbf{0},~3.5.5), and the fact that the inverse image of a tensor product of algebraic sheaves is the tensor product of their inverse images (\textbf{0},~4.3.3).

\begin{cx}[Corollary]{9.1.5}
    Let $f\colon X\to X'$ and $g\colon Y\to Y'$ be $S$-morphisms, and $\mathscr{F}'$ (\emph{resp.} $\mathscr{G}'$) a quasi-coherent $\O_{X'}$-module (\emph{resp.} quasi-coherent $\O_{Y'}$-module).
    Then
    \begin{equation*}
        (f,g)^*_S(\mathscr{F}'\otimes_S\mathscr{G}') = f^*(\mathscr{F}')\otimes_S g^*(\mathscr{G}')
    \end{equation*}
\end{cx}

This follows from (9.1.4) and the fact that $f\times_S g=(f\circ p, g\circ q)_S$, where $p,q$ are the projections of $X\times_S Y$.

\begin{cx}[Corollary]{9.1.6}
    Let $X,Y,Z$ be three $S$-preschemes, and $\mathscr{F}$ (\emph{resp.} $\mathscr{G},\mathscr{H}$) a quasi-coherent $\O_X$-module (\emph{resp.} quasi-coherent $\O_Y$-module, quasi-coherent $\O_Z$-module); then the sheaf $\mathscr{F}\otimes_S\mathscr{G}\otimes_S\mathscr{H}$ is the inverse image of $(\mathscr{F}\otimes_S\mathscr{G})\otimes_S\mathscr{H}$ by the canonical isomorphism from $X\times_S Y\times_S Z$ to $(X\times_S Y)\times_S Z$.
\end{cx}

In fact, this isomorphism is given by $(p_1,p_2)_S\times_S p_3$, where $p_1,p_2,p_3$ are the projections of $X\times_S Y\times_S Z$.

Similarly, the inverse image of $\mathscr{G}\otimes_S\mathscr{F}$ by the canonical isomorphism from $X\times_S Y$ to $Y\times_S X$ is $\mathscr{F}\otimes_S\mathscr{G}$.

\begin{cx}[Corollary]{9.1.7}
    If $X$ is an $S$-prescheme, then every quasi-coherent $\O_X$-module $\mathscr{F}$ is the inverse image of $\mathscr{F}\otimes_S\O_S$ by the canonical isomorphism from $X$ to $X\times_S S$ (3.3.3).
\end{cx}

In fact, this isomorphism is $(1_X,\varphi)_S$, where $\varphi$ is the structure morphism $X\to S$, and the corollary follows from (9.1.4) and the fact that $\varphi^*(\O_S)=\O_X$.

\begin{cx}{9.1.8}
    Let $X$ be an $S$-prescheme, $\mathscr{F}$ a quasi-coherent $\O_X$-module, and $\varphi\colon S'\to S$ a morphism; we denote by $\mathscr{F}_{(\varphi)}$ or $\mathscr{F}_{(S')}$ the quasi-coherent sheaf $\mathscr{F}\otimes_S\O_{S'}$ over $X\times_S S'=X_{(\varphi)}=X_{(S')}$; so $\mathscr{F}_{(S')}=p^*(\mathscr{F})$, where $p$ is the projection $X_{(S')}\to X$.
\end{cx}

\begin{cx}[Proposition]{9.1.9}
    Let $\varphi''\colon S''\to S'$ be a morphism.
    For every quasi-coherent $\O_X$-module $\mathscr{F}$ on the $S$-prescheme~$X$, $(\mathscr{F}_{(\varphi)})_{(\varphi')}$ is the inverse image of $\mathscr{F}_{(\varphi\circ\varphi')}$ by the canonical isomorphism $(X_{(\varphi)})_{(\varphi')}\xrightarrow{\sim}X_{(\varphi\circ\varphi')}$ \emph{(3.3.9)}.
\end{cx}

This follows immediately from the definitions and from (3.3.9), and is written
\begin{equation*}
    (\mathscr{F}\otimes_S\O_{S'})\otimes_{S'}\O_{S''} = \mathscr{F}\otimes_S\O_{S''}.\tag{9.1.9.1}
\end{equation*}

\begin{cx}[Proposition]{9.1.10}
    Let $Y$ be an $S$-prescheme, and $f\colon X\to Y$ an $S$-morphism.
    For every quasi-coherent $\O_Y$-module and every morphism $S'\to S$, we have that $(f_{(S')})^*(\mathscr{G}_{(S')})=(f^*(\mathscr{G}))_{(S')}$.
\end{cx}

This follows immediately from the commutativity of the diagram
\oldpage{171}\begin{equation*}
    \begin{tikzcd}
        X_{(S')}
            \ar{r}{f_{(S')}}
            \ar{d}
        & Y_{(S')}
            \ar{d}\\
        X
            \ar{r}{f}
        & Y
    \end{tikzcd}
\end{equation*}

\begin{cx}[Corollary]{9.1.11}
    Let $X$ and $Y$ be $S$-preschemes, and $\mathscr{F}$ (\emph{resp.} $\mathscr{G}$) a quasi-coherent $\O_X$-module (\emph{resp.} quasi-coherent $\O_Y$-module).
    Then the inverse image of the sheaf $(\mathscr{F}_{(S')})\otimes_{(S')}(\mathscr{G}_{(S')})$ by the canonical isomorphism $(X\times_S Y)_{(S')}\xrightarrow{\sim}(X_{(S')})\times_{S'}(Y_{(S')})$ \emph{(3.3.10)} is equal to $(\mathscr{F}\otimes_S\mathscr{G})_{(S')}$.
\end{cx}

If $p,q$ are the projections of $X\times_S Y$, then the isomorphism in question is nothing but $(p_{(S')}, q_{(S')})_{S'}$; the corollary follows from propositions (9.1.4) and (9.1.10).

\begin{cx}[Proposition]{9.1.12}
    With the notation from (9.1.2), let $z$ be a point of $X\times_S Y$, $x=p(z)$, and $y=q(z)$; the fibre $(\mathscr{F}\otimes_S\mathscr{G})_z$ is isomorphic to $(\mathscr{F}_x\otimes_{\O_x}\O_z)\otimes_{\O_z}(\mathscr{G}_y\otimes_{\O_y}\O_z) = \mathscr{F}_x\otimes_{\O_x}\O_z\otimes_{\O_y}\otimes\mathscr{G}_y$.
\end{cx}

As we can reduce ourselves to the affine case, the proposition follows from equation~(1.6.5.1).

\begin{cx}[Corollary]{9.1.13}
    If $\mathscr{F}$ and $\mathscr{G}$ are of finite type, then we have that
    \begin{equation*}
        \Supp(\mathscr{F}\otimes_S\mathscr{G}) = p^{-1}(\Supp(\mathscr{F}))\cap q^{-1}(\Supp(\mathscr{G})).
    \end{equation*}
\end{cx}

Since $p^*(\mathscr{F})$ and $q^*(\mathscr{G})$ are both of finite type over $\O_{X\times_S Y}$, we are reduced, by (9.1.12) and (\textbf{0},~1.7.5), to the case where $\mathscr{G}=\O_Y$, that is, it remains to prove the following equation:
\begin{equation*}
    \Supp(p^{-1}(\mathscr{F})) = p^{-1}(\Supp(\mathscr{F})).\tag{9.1.13.1}
\end{equation*}

The same reasoning as in (\textbf{0},~1.7.5) leads us to prove that, for all $z\in X\times_S Y$, we have $\O_z/\mathfrak{m}_x\O_z\neq0$ (with $x=p(z)$), which follows from the fact that the homomorphism $\O_x\to\O_z$ is \emph{local}, by hypothesis.

We leave it to the reader to extend the results in this section to the more general case of arbitrarily (but finitely) many factors, instead of just two.

\end{document}
