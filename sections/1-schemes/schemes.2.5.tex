\documentclass[../main.tex]{subfiles}

\begin{document}

\begin{cx}[Definition]{2.5.1}
    Given a prescheme $S$, we say that the data of a prescheme $X$ and a morphism of preschemes $\varphi\colon X\to S$ defines a prescheme $X$ \emph{over the prescheme $S$}, or an \emph{$S$-prescheme}; we say that $S$ is the \emph{base prescheme} of the $S$-prescheme $X$.
    The morphism $\varphi$ is called the \emph{structure morphism} of the $S$-prescheme $X$.
    When $S$ is an affine scheme of ring $A$, we also say that $X$ endowed with $\varphi$ is a prescheme \emph{over the ring $A$} (or an \emph{$A$-prescheme}).
\end{cx}

It follows from (2.2.4) that the data of a prescheme over a ring $A$ is equivalent to the data of a prescheme $(X,\O_X)$ whose structure sheaf $\O_X$ is a sheaf of \emph{$A$-algebras}.
\emph{An arbitrary prescheme can always be considered as a $\mathbb{Z}$-prescheme in a unique way.}

If $\varphi\colon X\to S$ is the structure morphism of an $S$-prescheme $X$, we say that a point $x\in X$ is \emph{over a point $s\in S$} if $\varphi(x)=s$.
We say that $X$ \emph{dominates} $S$ if $\varphi$ is a dominant morphism (2.2.6).

\begin{cx}{2.5.2}
    Let $X$ and $Y$ be two $S$-preschemes; we say that a morphism of preschemes $u\colon X\to Y$ is a \emph{morphism of preschemes over $S$} (or an \emph{$S$-morphism}) if the diagram
    \begin{equation*}
        \begin{tikzcd}[column sep=tiny, row sep=tiny]
            X \arrow[rr,"u"] \arrow[dr] & & Y \arrow[dl]\\
            & S &
        \end{tikzcd}
    \end{equation*}
    (where the diagonal arrows are the structure morphisms) is commutative: this ensures that, for all $s\in S$ and $x\in X$ over $s$, $u(x)$ is also above $s$.
\end{cx}

From this definition it follows immediately that the composition of two $S$-morphisms is an $S$-morphism; $S$-preschemes thus form a \emph{category}.

We denote by $\Hom_S(X,Y)$ the set of $S$-morphisms from an $S$-prescheme $X$ to an $S$-prescheme $Y$; the identity morphism of an $S$-prescheme is denoted by $1_X$.

When $S$ is an affine scheme of ring $A$, we will also say \emph{$A$-morphism} instead of $S$-morphism.

\begin{cx}{2.5.3}
    \oldpage{104}If $X$ is an $S$-prescheme, and $v\colon X'\to X$ a morphism of preschemes, then the composition $X'\to X\to S$ endows $X'$ with the structure of an $S$-prescheme; in particular, every prescheme induced by an open set $U$ of $X$ can be considered as an $S$-prescheme by the canonical injection.
\end{cx}

If $u\colon X\to Y$ is an $S$-morphism of $S$-preschemes, then the restriction of $u$ to any prescheme induced by an open subset $U$ of $X$ is also an $S$-morphism $U\to Y$.
Conversely, let $(U_\alpha)$ be an open cover of $X$, and for each $\alpha$ let $u_\alpha\colon U_\alpha\to Y$ be an $S$-morphism; if, for all pairs of indices $(\alpha,\beta)$, the restrictions of $u_\alpha$ and $u_\beta$ to $U_\alpha\cap U_\beta$ agree, then there exists an $S$-morphism $X\to Y$, and only one such that the restriction to each $U_\alpha$ is $u_\alpha$.

If $u\colon X\to Y$ is an $S$-morphism such that $u(X)\subset V$, where $V$ is an open subset of $Y$, then $u$, considered as a morphism from $X$ to $V$, is also an $S$-morphism.

\begin{cx}{2.5.4}
    Let $S'\to S$ be a morphism of preschemes; for all $S'$-preschemes, the composition $X\to S'\to S$ endows $X$ with the structure of an $S$-prescheme.
    Conversely, suppose that $S'$ is the induced prescheme of an open subset of $S$; let $X$ be an $S$-prescheme and suppose that the structure morphism $f\colon X\to S$ is such that $f(X)\subset S'$; then we can consider $X$ as an $S'$-preschemes.
    In this latter case, if $Y$ is another $S$-prescheme whose structure morphism sends the underlying space to $S'$, then every $S$-morphism from $X$ to $Y$ is also an $S'$-morphism.
\end{cx}

\begin{cx}{2.5.5}
    If $X$ is an $S$-prescheme, with structure morphism $\varphi\colon X\to S$, we define an \emph{$S$-section of $X$} to be an $S$-morphism from $S$ to $X$, that is to say a morphism of preschemes $\psi\colon S\to X$ such that $\varphi\circ\psi$ is the identity on $S$.
    We denote by $\Gamma(X/S)$ the set of $S$-sections of $X$.
\end{cx}

\end{document}
