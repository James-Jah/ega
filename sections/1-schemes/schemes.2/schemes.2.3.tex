\documentclass[../main.tex]{subfiles}

\begin{document}

\begin{cx}{2.3.1}
    \oldpage{101}It follows from definition~(2.1.2) that every ringed space obtained by \emph{gluing} preschemes (\textbf{0},~4.1.6) is again a prescheme.
    In particular, although every prescheme admits, by definition, a cover by affine open sets, we see that every prescheme can actually be obtained by \emph{gluing affine schemes}.
\end{cx}

\begin{cx}[Example]{2.3.2}
    Let $K$ be a field, and $B=K[s]$, $C=K[t]$ be two polynomial rings in one indeterminate over $K$, and define $X_1=\Spec(B)$, $X_2=\Spec(C)$, which are two isomorphic affine schemes.
    In $X_1$ (resp. $X_2$), let $U_{12}$ (resp. $U_{21}$) be the affine open $D(s)$ (resp. $D(t)$) where the ring $B_s$ (resp. $C_t$) is formed of rational fractions of the form $f(s)/s^m$ (resp. $g(t)/t^n$) with $f\in B$ (resp. $g\in C$).
    Let $u_{12}$ be the isomorphism of preschemes $U_{21}\to U_{12}$ corresponding (2.2.4) to the isomorphism from $B$ to $C$ that, to $f(s)/s^m$, associates the rational fraction $f(1/t)/(1/t^m)$.
    We can glue $X_1$ and $X_2$ along $U_{12}$ and $U_{21}$ by using $u_{12}$, because there is clearly no gluing condition.
    We later show that the prescheme $X$ obtained in this manner is a particular case of a general method of construction (\textbf{II},~2.4.3).
    Here we only show that $X$ \emph{is not an affine scheme}; this will follow from the fact that the ring $\Gamma(X,\O_X)$ is \emph{isomorphic} to $K$, and so its spectrum reduces to a point.
    In effect, a section of $\O_X$ over $X$ has a restriction over $X_1$ (resp. $X_2$), identified to an affine open of $X$, that is a polynomial $f(s)$ (resp. $g(t)$), and it follows from the definitions that we should have $g(t)=f(1/t)$, which is not possible if $f=g\in K$.
\end{cx}

\end{document}
