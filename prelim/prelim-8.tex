\section{Representable functors}
\label{section-representable-functors}

\subsection{Representable functors}
\label{subsection-representable-functors}

\begin{env}[8.1.1]
\label{0.8.1.1}
\oldpage[0\textsubscript{III}]{5}
We denote by $\Set$ the category of sets.
Let $\C$ be a category; for two objects $X$, $Y$ of $\C$, we set $h_X(Y)=\Hom(Y,X)$; for each morphism $u:Y\to Y'$ in $\C$, we denote by $h_X(u)$ the map $v\mapsto vu$ from $\Hom(Y',X)$ to $\Hom(Y,X)$.
It is immediate that with these definitions, $h_X:\C\to\Set$ is a \emph{contravariant functor}, i.e., an object of the category $\CHom(\C\op,\Set)$, of covariant functors from the category $\C\op$ (the dual of the category $\C$) to the category $\Set$ (T, 1.7, (d) and \cite{III-29}).
\end{env}

\begin{env}[8.1.2]
\label{0.8.1.2}
Now let $w:X\to X'$ be a morphism in $\C$; for each $Y\in\C$ and each $v\in\Hom(Y,X)=h_X(Y)$,
we have $wv\in\Hom(Y,X')=h_{X'}(Y)$; we denote by $h_w(Y)$ the map $v\mapsto wv$ from
$h_X(Y)$ to $h_{X'}(Y)$. It is immediate that for each morphism $u:Y\to Y'$ in $\C$, the
diagram
\[
  \xymatrix{
    h_X(Y')\ar[r]^{h_X(u)}\ar[d]_{h_w(Y')} &
    h_X(Y)\ar[d]^{h_w(Y)}\\
    h_{X'}(Y')\ar[r]^{h_{X'}(u)} &
    h_{X'}(Y)
  }
\]
is commutative; in other words, $h_w$ is a \emph{natural transformation (or functorial morphism)
$h_X\to h_{X'}$} (T, 1.2), also a morphism in the category $\CHom(\C\op,\Set)$ (T, 1.7, (d)). The
definitions of $h_X$ and of $h_w$ therefore constitute the definition of a \emph{canonical
covariant functor}
\[
  h:\C\longrightarrow\CHom(\C\op,\Set),\quad X\longmapsto h_X.
  \tag{8.1.2.1}
\]
\end{env}

\begin{env}[8.1.3]
\label{0.8.1.3}
Let $X$ be an object in $\C$, $F$ a contravariant functor from $\C$ to $\Set$
(an object of $\CHom(\C\op,\Set)$). Let $g:h_X\to F$ be a \emph{natural transformation}: for all $Y\in\C$,
\oldpage[0\textsubscript{III}]{6}
$g(Y)$ is thus a map $h_X(Y)\to F(Y)$ such that for each morphism $u:Y\to Y'$ in $\C$,
the diagram
\[
  \xymatrix{
    h_X(Y')\ar[r]^{h_X(u)}\ar[d]_{g(Y')} &
    h_X(Y)\ar[d]^{g(Y)}\\
    F(Y')\ar[r]^{F(u)} &
    F(Y)
  }
  \tag{8.1.3.1}
\]
is commutative. In particular, we have a map $g(X):h_X(X)=\Hom(X,X)\to F(X)$, hence an element
\[
  \alpha(g)=(g(X))(1_X)\in F(X)
  \tag{8.1.3.2}
\]
and as a result a canonical map
\[
  \alpha:\Hom(h_X,F)\longrightarrow F(X).
  \tag{8.1.3.3}
\]

Conversely, consider an element $\xi\in F(X)$; for each morphism $v:Y\to X$ in $\C$, $F(v)$ is a
map $F(X)\to F(Y)$; consider the map
\[
  v\longmapsto(F(v))(\xi)
  \tag{8.1.3.4}
\]
from $h_X(Y)$ to $F(Y)$; if we denote by $(\beta(\xi))(Y)$ this map,
\[
  \beta(\xi):h_X\longrightarrow F
  \tag{8.1.3.5}
\]
is a \emph{natural transformation}, since for each morphism $u:Y\to Y'$ in $\C$ we have
$(F(vu))(\xi)=(F(v)\circ F(u))(\xi)$, which makes (8.1.3.1) commutative for $g=\beta(\xi)$.
We have thus defined a canonical map
\[
  \beta:F(X)\longrightarrow\Hom(h_X,F).
  \tag{8.1.3.6}
\]
\end{env}

\begin{prop}[8.1.4]
\label{0.8.1.4}
The maps $\alpha$ and $\beta$ are the inverse bijections of each other.
\end{prop}

\begin{proof}
\label{proof-0.8.1.4}
We calculate $\alpha(\beta(\xi))$ for $\xi\in F(X)$; for each $Y\in\C$, $(\beta(\xi))(Y)$ is a map
$g_1(Y):v\mapsto(F(v))(\xi)$ from $h_X(Y)$ to $F(Y)$. We thus have
\[
  \alpha(\beta(\xi))=(g_1(X))(1_X)=(F(1_X))(\xi)=1_{F(X)}(\xi)=\xi.
\]
We now calculate $\beta(\alpha(g))$ for $g\in\Hom(h_X,F)$; for each $Y\in\C$, $(\beta(\alpha(g)))(Y)$
is the map $v\mapsto(F(v))((g(X))(1_X))$; according to the commutativity of (8.1.3.1), this map
is none other than $v\mapsto(g(Y))((h_X(v))(1_X))=(g(Y))(v)$ by definition of $h_X(v)$, in other words,
it is equal to $g(Y)$, which finishes the proof.
\end{proof}

\begin{env}[8.1.5]
\label{0.8.1.5}
Recall that a \emph{subcategory} $\C'$ of a category $\C$ is defined by the condition that its objects
are objects of $\C$, and that if $X'$, $Y'$ are two objects of $\C'$, then the set
$\Hom_{\C'}(X',Y')$ of morphisms $X'\to Y'$ \emph{in $\C'$} is a subset of the set $\Hom_\C(X',Y')$ of
morphisms $X'\to Y'$ \emph{in $\C$}, the canonical map of ``composition of morphisms''
\[
  \Hom_{\C'}(X',Y')\times\Hom_{\C'}(Y',Z')\longrightarrow\Hom_{\C'}(X',Z')
\]
\oldpage[0\textsubscript{III}]{7}
being the restriction of the canonical map
\[
  \Hom_\C(X',Y')\times\Hom_\C(Y',Z')\longrightarrow\Hom_\C(X',Z').
\]

We say that $\C'$ is a \emph{full} subcategory of $\C$ if $\Hom_{\C'}(X',Y')=\Hom_\C(X',Y')$ for every
pair of objects in $\C'$. The subcategory $\C''$ of $\C$ consisting of the objects of $\C$ isomorphic to
objects of $\C'$ is then again a full subcategory of $\C$, \emph{equivalent} (T, 1.2) to $\C'$ as we
verify easily.

A covariant functor $F:\C_1\to\C_2$ is called \emph{fully faithful} if for every pair of objects
$X_1$, $Y_1$ of $\C_1$, the map $u\mapsto F(u)$ from $\Hom(X_1,Y_1)$ to $\Hom(F(X_1),F(Y_1))$ is
\emph{bijective}; this implies that the subcategory $F(\C_1)$ of $\C_2$ is \emph{full}. In addition,
if two objects $X_1$, $X_1'$ have the same image $X_2$, then there exists a unique isomorphism
$u:X_1\to X_1'$ such that $F(u)=1_{X_1}$. For each object $X_2$ of $F(\C_1)$, let $G(X_2)$ be one of
the objects $X_1$ of $\C_1$ such that $F(X_1)=X_2$ ($G$ is defined by means of the axiom of choice); for
each morphism $v:X_2\to Y_2$ in $F(\C_1)$, $G(v)$ will be the unique morphism $u:G(X_2)\to G(Y_2)$ such
that $F(u)=v$; $G$ is then a \emph{functor} from $F(\C_1)$ to $\C_1$; $FG$ is the identity functor on
$F(\C_1)$, and the above shows that there exists an isomorphism of functors $\vphi:1_{\C_1}\to GF$
such that $F$, $G$, $\vphi$, and the identity $1_{F(\C_1)}\to FG$ defines an \emph{equivalence} between
the category $\C_1$ and the full subcategory $F(\C_1)$ of $\C_2$ (T, 1.2).
\end{env}

\begin{env}[8.1.6]
\label{0.8.1.6}
We apply Proposition \sref{0.8.1.4} to the case where $F$ is $h_{X'}$, $X'$ being any
object of $\C$; the map $\beta:\Hom(X,X')\to\Hom(h_X,h_{X'})$ is none other than the map
$w\mapsto h_w$ defined in \sref{0.8.1.2}; this map being \emph{bijective}, we see
with the terminology of \sref{0.8.1.5} that:
\end{env}

\begin{prop}[8.1.7]
\label{0.8.1.7}
The canonical functor $h:\C\to\CHom(\C\op,\Set)$ is fully faithful.
\end{prop}

\begin{env}[8.1.8]
\label{0.8.1.8}
Let $F$ be a contravariant functor from $\C$ to $\Set$; we say that $F$ is \emph{representable} if there
exists an object $X\in\C$ such that $F$ is \emph{isomorphic} to $h_X$; it follows from
Proposition \sref{0.8.1.7} that the data of an $X\in\C$ and an isomorphism of functors
$g:h_X\to F$ determines $X$ up to unique isomorphism. Proposition \sref{0.8.1.7} then
implies that $h$ defines an \emph{equivalence} between $\C$ and the full subcategory of
$\CHom(\C\op,\Set)$ consisting of the \emph{contravariant representable functors}. It follows from
Proposition \sref{0.8.1.4} that the data of a natural transformation $g:h_X\to F$ is
equivalent to that of an element $\xi\in F(X)$; to say that $g$ is an \emph{isomorphism} is
equivalent to the following condition on $\xi$: \emph{for every object $Y$ of $\C$ the map
$v\mapsto(F(v))(\xi)$ from $\Hom(Y,X)$ to $F(Y)$ is bijective}. When $\xi$ satsifies this condition,
we say that the pair $(X,\xi)$ \emph{represents} the representable functor $F$, By abuse of language,
we also say that the object $X\in\C$ represents $F$ if there exists a $\xi\in F(X)$ such that
$(X,\xi)$ represents $F$, in other words if $h_X$ is isomorphic to $F$.

Let $F$, $F'$ be two contravariant representable functors from $\C$ to $\Set$, $h_X\to F$ and
$h_{X'}\to F'$ two isomorphisms of functors. Then it follows from \sref{0.8.1.6} that
there is a canonical bijective correspondence between $\Hom(X,X')$ and the set $\Hom(F,F')$ of
natural transformations $F\to F'$.
\end{env}
