\documentclass[oneside]{amsart}

\usepackage[all]{xy}
\usepackage[T1]{fontenc}
\usepackage{xstring}
\usepackage{xparse}
\usepackage{xr-hyper}
\usepackage[linktocpage=true,colorlinks=true,hyperindex,citecolor=blue,linkcolor=magenta]{hyperref}
\usepackage[left=0.95in,right=0.95in,top=0.75in,bottom=0.75in]{geometry}
\usepackage[charter,ttscaled=false,greekfamily=didot,greeklowercase=upright]{mathdesign}

\usepackage{Baskervaldx}

\usepackage{enumitem}
\usepackage{longtable}
\usepackage{aurical}

\externaldocument[what-]{what}
\externaldocument[intro-]{intro}
\externaldocument[ega0-]{ega0}
\externaldocument[ega1-]{ega1}
\externaldocument[ega2-]{ega2}
\externaldocument[ega3-]{ega3}
\externaldocument[ega4-]{ega4}

\newtheoremstyle{ega-env-style}%
  {}{}{\rmfamily}{}{\bfseries}{.}{ }{\thmnote{(#3)}}%

\newtheoremstyle{ega-thm-env-style}%
  {}{}{\itshape}{}{\bfseries}{. --- }{ }{\thmname{#1}\thmnote{ (#3)}}%

\newtheoremstyle{ega-defn-env-style}%
  {}{}{\rmfamily}{}{\bfseries}{. --- }{ }{\thmname{#1}\thmnote{ (#3)}}%

\theoremstyle{ega-env-style}
\newtheorem*{env}{---}

\theoremstyle{ega-thm-env-style}
\newtheorem*{thm}{Theorem}
\newtheorem*{prop}{Proposition}
\newtheorem*{lem}{Lemma}
\newtheorem*{cor}{Corollary}

\theoremstyle{ega-defn-env-style}
\newtheorem*{defn}{Definition}
\newtheorem*{exm}{Example}
\newtheorem*{rmk}{Remark}
\newtheorem*{nota}{Notation}

% indent subsections, see https://tex.stackexchange.com/questions/177290/.
% also make section titles bigger.
% also add § to \thesection, https://tex.stackexchange.com/questions/119667/ and https://tex.stackexchange.com/questions/308737/.
\makeatletter
\def\l@subsection{\@tocline{2}{0pt}{2.5pc}{1.5pc}{}}
\def\section{\@startsection{section}{1}%
  \z@{.7\linespacing\@plus\linespacing}{.5\linespacing}%
  {\normalfont\bfseries\Large\scshape\centering}}
\renewcommand{\@seccntformat}[1]{%
  \ifnum\pdfstrcmp{#1}{section}=0\textsection\fi%
  \csname the#1\endcsname.~}
\makeatother

%\allowdisplaybreaks[1]
%\binoppenalty=9999
%\relpenalty=9999

% for Chapter 0, Chapter I, etc.
% credit for ZeroRoman https://tex.stackexchange.com/questions/211414/
% added into scripts/make_book.py
%\newcommand{\ZeroRoman}[1]{\ifcase\value{#1}\relax 0\else\Roman{#1}\fi}
%\renewcommand{\thechapter}{\ZeroRoman{chapter}}

\def\mathcal{\mathscr}
\def\sh{\mathcal}                   % sheaf font
\def\bb{\mathbf}                    % bold font
\def\cat{\mathtt}                   % category font
\def\fk{\mathfrak}                  % mathfrak font
\def\leq{\leqslant}                 % <=
\def\geq{\geqslant}                 % >=
\def\wt#1{{\widetilde{#1}}}         % tilde over
\def\wh#1{{\widehat{#1}}}           % hat over
\def\setmin{-}                      % set minus
\def\rad{\fk{r}}                    % radical
\def\nilrad{\fk{R}}                 % nilradical
\def\emp{\varnothing}               % empty set
\def\vphi{\phi}                     % for switching \phi and \varphi, change if needed
\def\HH{\mathrm{H}}                 % cohomology H
\def\CHH{\check{\HH}}               % Čech cohomology H
\def\RR{\mathrm{R}}                 % right derived R
\def\LL{\mathrm{L}}                 % left derived L
\def\dual#1{{#1}^\vee}              % dual
\def\kres{k}                        % residue field k
\def\C{\cat{C}}                     % category C
\def\op{^\cat{op}}                  % opposite category
\def\Set{\cat{Set}}                 % category of sets
\def\CHom{\cat{Hom}}                % functor category
\def\OO{\sh{O}}                     % structure sheaf O

\def\shHom{\sh{H}\textup{\kern-2.2pt{\Fontauri\slshape om}}\!}   % sheaf Hom
\def\shProj{\sh{P}\textup{\kern-2.2pt{\Fontauri\slshape roj}}\!} % sheaf Proj
\def\shExt{\sh{E}\textup{\kern-2.2pt{\Fontauri\slshape xt}}\!}   % sheaf Ext
\def\red{\mathrm{red}}
\def\rg{{\mathop{\mathrm{rg}}\nolimits}}
\def\gr{{\mathop{\mathrm{gr}}\nolimits}}
\def\Hom{{\mathop{\mathrm{Hom}}\nolimits}}
\def\Proj{{\mathop{\mathrm{Proj}}\nolimits}}
\def\Tor{{\mathop{\mathrm{Tor}}\nolimits}}
\def\Ext{{\mathop{\mathrm{Ext}}\nolimits}}
\def\Supp{{\mathop{\mathrm{Supp}}\nolimits}}
\def\Ker{{\mathop{\mathrm{Ker}}\nolimits}\,}
\def\Im{{\mathop{\mathrm{Im}}\nolimits}\,}
\def\Coker{{\mathop{\mathrm{Coker}}\nolimits}\,}
\def\Spec{{\mathop{\mathrm{Spec}}\nolimits}}
\def\Spf{{\mathop{\mathrm{Spf}}\nolimits}}
\def\grad{{\mathop{\mathrm{grad}}\nolimits}}
\def\dim{{\mathop{\mathrm{dim}}\nolimits}}
\def\dimc{{\mathop{\mathrm{dimc}}\nolimits}}
\def\codim{{\mathop{\mathrm{codim}}\nolimits}}

\renewcommand{\to}{\mathchoice{\longrightarrow}{\rightarrow}{\rightarrow}{\rightarrow}}
\let\mapstoo\mapsto
\renewcommand{\mapsto}{\mathchoice{\longmapsto}{\mapstoo}{\mapstoo}{\mapstoo}}
\def\isoto{\simeq}  % isomorphism

% if unsure of a translation
%\newcommand{\unsure}[2][]{\hl{#2}\marginpar{#1}}
%\newcommand{\completelyunsure}{\unsure{[\ldots]}}
\def\unsure#1{#1 {\color{red}(?)}}
\def\completelyunsure{{\color{red}(???)}}

% use to mark where original page starts
\newcommand{\oldpage}[2][--]{{\marginpar{\textbf{#1}~|~#2}}\ignorespaces}
\def\sectionbreak{\begin{center}***\end{center}}

% for referencing environments.
% use as \sref{chapter-number.x.y.z}, with optional args
% for volume and indices, e.g. \sref[volume]{chapter-number.x.y.z}[i].
\NewDocumentCommand{\sref}{o m o}{%
  \IfNoValueTF{#1}%
    {\IfNoValueTF{#3}%
      {\hyperref[#2]{\normalfont{(\StrGobbleLeft{#2}{2})}}}%
      {\hyperref[#2]{\normalfont{(\StrGobbleLeft{#2}{2},~{#3})}}}}%
    {\IfNoValueTF{#3}%
      {\hyperref[#2]{\normalfont{(\textbf{#1},~\StrGobbleLeft{#2}{2})}}}%
      {\hyperref[#2]{\normalfont{(\textbf{#1},~\StrGobbleLeft{#2}{2},~({#3}))}}}}%
}



\begin{document}
\title{Elementary global study of some classes of morphisms (EGA~II)}
\maketitle

\phantomsection
\label{section-phantom}

build hack
\cite{I-1}

\tableofcontents

\section*{Summary}
\label{section-egaII-summary}

\begin{tabular}{ll}
  \textsection1. & Affine morphisms\\
  \textsection2. & Homogeneous prime spectra.\\
  \textsection3. & Homogeneous prime spectrum of a sheaf of graded algebras.\\
  \textsection4. & Projective bundles; ample sheaves.\\
  \textsection5. & Quasi-affine morphisms; quasi-projective morphisms; proper morphisms; projective morphisms.\\
  \textsection6. & Integral morphisms and finite morphisms.\\
  \textsection7. & Valuative criteria.\\
  \textsection8. & Blowup schemes; projective cones; projective closure.\\
\end{tabular}\\

\oldpage[II]{5}
The various classes of morphisms studied in this chapter are used extensively in cohomological methods; further study, using these methods, will be done in Chapter~III, where we use especially \textsection\textsection2,4, and 5 of Chapter~II.
The \textsection8 can be omitted on a first reading: it gives some supplements to the formalism developed in \textsection\textsection1 and 3, reducing to easy maps of this formalism, and we will use it less consistently than the other results of this chapter.
\bigskip

\section{Affine morphisms}
\label{section-affine-morphisms}

\subsection{$S$-preschemes and $\mathcal{O}_S$-algebras}
\label{subsection-s-preschemes-algebras}

\begin{env}[1.1.1]
\label{2.1.1.1}
Let $S$ be a prescheme, $X$ an $S$-prescheme, and $f:X\to S$ its structure morphism.
We know \sref[0]{0.4.2.4} that the direct image $f_*(\OO_X)$ is an $\OO_S$-algebra, which we
\oldpage[II]{6}
denote $\sh{A}(X)$ when there is little chance of confusion; if $U$ is an open subset of $S$, then we have
\[
  \sh{A}(f^{-1}(U))=\sh{A}(X)|U.
\]
Similarly, for every $\OO_X$-module $\sh{F}$ (resp. every $\OO_X$-algebra $\sh{B}$), we write $\sh{A}(\sh{F})$ (resp. $\sh{A}(\sh{B})$) for the direct image $f_*(\sh{F})$ (resp. $f_*(\sh{B})$) which is an $\sh{A}(X)$-module (resp. an $\sh{A}(X)$-algebra) and not only an $\OO_S$-module (resp. an $\OO_S$-algebra).
\end{env}

\begin{env}[1.1.2]
\label{2.1.1.2}
Let $Y$ be a second $S$-prescheme, $g:Y\to S$ its structure morphism, and $h:X\to Y$ an $S$-morphism; we then have the commutative diagram
\[
  \xymatrix{
    X\ar[rr]^h\ar[rd]_f & &
    Y\ar[ld]^g\\
    & S.
  }
  \tag{1.1.2.1}
\]

We have by definition $h=(\psi,\theta)$, where $\theta:\OO_Y\to h_*(\OO_X)=\psi_*(\OO_X)$ is a homomorphism of sheaves of rings; we induce \sref[0]{0.4.2.2} a homomorphism of $\OO_S$-algebras $g_*(\theta):g_*(\OO_Y)\to g_*(h_*(\OO_X))=f_*(\OO_X)$, in other words, a homomorphism of $\OO_S$-algebras $\sh{A}(Y)\to\sh{A}(X)$, which we denote by $\sh{A}(h)$.
If $h':Y\to Z$ is a second $S$-morphism, then it is immediate that $\sh{A}(h'\circ h)=\sh{A}(h)\circ\sh{A}(h')$.
We havve thus define a \emph{contravariant functor $\sh{A}(X)$} from the category of $S$-preschemes to the category of $\OO_S$-algebras.

Now let $\sh{F}$ be an $\OO_X$-module, $\sh{G}$ an $\OO_Y$-module, and $u:\sh{G}\to\sh{F}$ be an $h$-morphism, that is \sref[0]{0.4.4.1} a homomorphism of $\OO_Y$-modules $\sh{G}\to h_*(\sh{F})$.
Then $g_*(u):g_*(\sh{G})\to g_*(h_*(\sh{F}))=f_*(\sh{F})$ is a homomorphism $\sh{A}(\sh{G})\to\sh{A}(\sh{F})$ of $\OO_S$-modules, which we denote by $\sh{A}(u)$; in addition, the pair $(\sh{A}(h),\sh{A}(u))$ form a \emph{di-homomorphism} from the $\sh{A}(Y)$-module $\sh{A}(\sh{G})$ to the $\sh{A}(X)$-module $\sh{A}(\sh{F})$.
\end{env}

\begin{env}[1.1.3]
\label{2.1.1.3}
If we fix the prescheme $S$, then we can consider the pairs $(X,\sh{F})$, where $X$ is an $S$-prescheme and $\sh{F}$ is an $\OO_X$-module, as forming a \emph{category}, by defining a \emph{morphism} $(X,\sh{F})\to(Y,\sh{G})$ as a pair $(h,u)$, where $h:X\to Y$ is an $S$-morphism and $u:\sh{G}\to\sh{F}$ is an $h$-morphism.
We can theen say that $(\sh{A}(X),\sh{A}(\sh{F}))$ is a \emph{contravariant functor} with values in the category whose objects are pairs consisting of an $\OO_S$-algebra and a module over that algebra, and the morphisms are the di-homomorphisms.
\end{env}

\subsection{Affine preschemes over a prescheme}
\label{subsection-affine-preschemes-over-a-prescheme}


\section{Homogeneous prime spectra}
\label{section-homogeneous-prime-spectra}


\section{Homogeneous prime spectrum of a sheaf of graded algebras}
\label{section-homogeneous-prime-spectrum-sheaf-of-graded-algebras}


\section{Projective bundles; ample sheaves}
\label{section-projective-bundles-ample-sheaves}


\section{Quasi-affine morphisms; quasi-projective morphisms; proper morphisms; projective morphisms}
\label{section-quasi-affine-projective-proper-morphisms}


\section{Integral morphisms and finite morphisms}
\label{section-integral-finite-morphisms}


\section{Valuative criteria}
\label{section-valuative-criteria}


\section{Blowup schemes; projective cones; projective closure}
\label{section-blowup-schemes-projective-cones-closure}



\bibliography{the}
\bibliographystyle{amsalpha}

\end{document}

