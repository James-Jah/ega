\documentclass[oneside]{amsart}

\usepackage[all]{xy}
\usepackage[T1]{fontenc}
\usepackage{xstring}
\usepackage{xparse}
\usepackage{xr-hyper}
\usepackage[linktocpage=true,colorlinks=true,hyperindex,citecolor=blue,linkcolor=magenta]{hyperref}
\usepackage[left=0.95in,right=0.95in,top=0.75in,bottom=0.75in]{geometry}
\usepackage[charter,ttscaled=false,greekfamily=didot,greeklowercase=upright]{mathdesign}

\usepackage{Baskervaldx}

\usepackage{enumitem}
\usepackage{longtable}
\usepackage{aurical}

\externaldocument[what-]{what}
\externaldocument[intro-]{intro}
\externaldocument[ega0-]{ega0}
\externaldocument[ega1-]{ega1}
\externaldocument[ega2-]{ega2}
\externaldocument[ega3-]{ega3}
\externaldocument[ega4-]{ega4}

\newtheoremstyle{ega-env-style}%
  {}{}{\rmfamily}{}{\bfseries}{.}{ }{\thmnote{(#3)}}%

\newtheoremstyle{ega-thm-env-style}%
  {}{}{\itshape}{}{\bfseries}{. --- }{ }{\thmname{#1}\thmnote{ (#3)}}%

\newtheoremstyle{ega-defn-env-style}%
  {}{}{\rmfamily}{}{\bfseries}{. --- }{ }{\thmname{#1}\thmnote{ (#3)}}%

\theoremstyle{ega-env-style}
\newtheorem*{env}{---}

\theoremstyle{ega-thm-env-style}
\newtheorem*{thm}{Theorem}
\newtheorem*{prop}{Proposition}
\newtheorem*{lem}{Lemma}
\newtheorem*{cor}{Corollary}

\theoremstyle{ega-defn-env-style}
\newtheorem*{defn}{Definition}
\newtheorem*{exm}{Example}
\newtheorem*{rmk}{Remark}
\newtheorem*{nota}{Notation}

% indent subsections, see https://tex.stackexchange.com/questions/177290/.
% also make section titles bigger.
% also add § to \thesection, https://tex.stackexchange.com/questions/119667/ and https://tex.stackexchange.com/questions/308737/.
\makeatletter
\def\l@subsection{\@tocline{2}{0pt}{2.5pc}{1.5pc}{}}
\def\section{\@startsection{section}{1}%
  \z@{.7\linespacing\@plus\linespacing}{.5\linespacing}%
  {\normalfont\bfseries\Large\scshape\centering}}
\renewcommand{\@seccntformat}[1]{%
  \ifnum\pdfstrcmp{#1}{section}=0\textsection\fi%
  \csname the#1\endcsname.~}
\makeatother

%\allowdisplaybreaks[1]
%\binoppenalty=9999
%\relpenalty=9999

% for Chapter 0, Chapter I, etc.
% credit for ZeroRoman https://tex.stackexchange.com/questions/211414/
% added into scripts/make_book.py
%\newcommand{\ZeroRoman}[1]{\ifcase\value{#1}\relax 0\else\Roman{#1}\fi}
%\renewcommand{\thechapter}{\ZeroRoman{chapter}}

\def\mathcal{\mathscr}
\def\sh{\mathcal}                   % sheaf font
\def\bb{\mathbf}                    % bold font
\def\cat{\mathtt}                   % category font
\def\fk{\mathfrak}                  % mathfrak font
\def\leq{\leqslant}                 % <=
\def\geq{\geqslant}                 % >=
\def\wt#1{{\widetilde{#1}}}         % tilde over
\def\wh#1{{\widehat{#1}}}           % hat over
\def\setmin{-}                      % set minus
\def\rad{\fk{r}}                    % radical
\def\nilrad{\fk{R}}                 % nilradical
\def\emp{\varnothing}               % empty set
\def\vphi{\phi}                     % for switching \phi and \varphi, change if needed
\def\HH{\mathrm{H}}                 % cohomology H
\def\CHH{\check{\HH}}               % Čech cohomology H
\def\RR{\mathrm{R}}                 % right derived R
\def\LL{\mathrm{L}}                 % left derived L
\def\dual#1{{#1}^\vee}              % dual
\def\kres{k}                        % residue field k
\def\C{\cat{C}}                     % category C
\def\op{^\cat{op}}                  % opposite category
\def\Set{\cat{Set}}                 % category of sets
\def\CHom{\cat{Hom}}                % functor category
\def\OO{\sh{O}}                     % structure sheaf O

\def\shHom{\sh{H}\textup{\kern-2.2pt{\Fontauri\slshape om}}\!}   % sheaf Hom
\def\shProj{\sh{P}\textup{\kern-2.2pt{\Fontauri\slshape roj}}\!} % sheaf Proj
\def\shExt{\sh{E}\textup{\kern-2.2pt{\Fontauri\slshape xt}}\!}   % sheaf Ext
\def\red{\mathrm{red}}
\def\rg{{\mathop{\mathrm{rg}}\nolimits}}
\def\gr{{\mathop{\mathrm{gr}}\nolimits}}
\def\Hom{{\mathop{\mathrm{Hom}}\nolimits}}
\def\Proj{{\mathop{\mathrm{Proj}}\nolimits}}
\def\Tor{{\mathop{\mathrm{Tor}}\nolimits}}
\def\Ext{{\mathop{\mathrm{Ext}}\nolimits}}
\def\Supp{{\mathop{\mathrm{Supp}}\nolimits}}
\def\Ker{{\mathop{\mathrm{Ker}}\nolimits}\,}
\def\Im{{\mathop{\mathrm{Im}}\nolimits}\,}
\def\Coker{{\mathop{\mathrm{Coker}}\nolimits}\,}
\def\Spec{{\mathop{\mathrm{Spec}}\nolimits}}
\def\Spf{{\mathop{\mathrm{Spf}}\nolimits}}
\def\grad{{\mathop{\mathrm{grad}}\nolimits}}
\def\dim{{\mathop{\mathrm{dim}}\nolimits}}
\def\dimc{{\mathop{\mathrm{dimc}}\nolimits}}
\def\codim{{\mathop{\mathrm{codim}}\nolimits}}

\renewcommand{\to}{\mathchoice{\longrightarrow}{\rightarrow}{\rightarrow}{\rightarrow}}
\let\mapstoo\mapsto
\renewcommand{\mapsto}{\mathchoice{\longmapsto}{\mapstoo}{\mapstoo}{\mapstoo}}
\def\isoto{\simeq}  % isomorphism

% if unsure of a translation
%\newcommand{\unsure}[2][]{\hl{#2}\marginpar{#1}}
%\newcommand{\completelyunsure}{\unsure{[\ldots]}}
\def\unsure#1{#1 {\color{red}(?)}}
\def\completelyunsure{{\color{red}(???)}}

% use to mark where original page starts
\newcommand{\oldpage}[2][--]{{\marginpar{\textbf{#1}~|~#2}}\ignorespaces}
\def\sectionbreak{\begin{center}***\end{center}}

% for referencing environments.
% use as \sref{chapter-number.x.y.z}, with optional args
% for volume and indices, e.g. \sref[volume]{chapter-number.x.y.z}[i].
\NewDocumentCommand{\sref}{o m o}{%
  \IfNoValueTF{#1}%
    {\IfNoValueTF{#3}%
      {\hyperref[#2]{\normalfont{(\StrGobbleLeft{#2}{2})}}}%
      {\hyperref[#2]{\normalfont{(\StrGobbleLeft{#2}{2},~{#3})}}}}%
    {\IfNoValueTF{#3}%
      {\hyperref[#2]{\normalfont{(\textbf{#1},~\StrGobbleLeft{#2}{2})}}}%
      {\hyperref[#2]{\normalfont{(\textbf{#1},~\StrGobbleLeft{#2}{2},~({#3}))}}}}%
}



\begin{document}
\title{Elementary global study of some classes of morphisms (EGA~II)}
\maketitle

\phantomsection
\label{section-phantom}

build hack
\cite{I-1}

\tableofcontents

\section*{Summary}
\label{section-ega2-summary}

\begin{longtable}{ll}
  \textsection1. & Affine morphisms\\
  \textsection2. & Homogeneous prime spectra.\\
  \textsection3. & Homogeneous prime spectrum of a sheaf of graded algebras.\\
  \textsection4. & Projective bundles; ample sheaves.\\
  \textsection5. & Quasi-affine morphisms; quasi-projective morphisms; proper morphisms; projective morphisms.\\
  \textsection6. & Integral morphisms and finite morphisms.\\
  \textsection7. & Valuative criteria.\\
  \textsection8. & Blowup schemes; projective cones; projective closure.\\
\end{longtable}
\bigskip

\oldpage[II]{5}
The various classes of morphisms studied in this chapter are used extensively in cohomological methods; further study, using these methods, will be done in Chapter~III, where we use especially \textsection\textsection2,4, and 5 of Chapter~II.
Section \textsection8 can be omitted on a first reading: it gives some supplements to the formalism developed in \textsection\textsection1 and 3, reducing to easy applications of this formalism, and we will use it less consistently than the other results of this chapter.
\bigskip

\cite{I-1}.

\section{Affine morphisms}
\label{section:affine-morphisms}

\subsection{$S$-preschemes and $\mathcal{O}_S$-algebras}
\label{subsection:s-preschemes-algebras}

\begin{env}[1.1.1]
\label{2.1.1.1}
Let $S$ be a prescheme, $X$ an $S$-prescheme, and $f:X\to S$ its structure morphism.
We know \sref[0]{0.4.2.4} that the direct image $f_*(\OO_X)$ is an $\OO_S$-algebra, which we
\oldpage[II]{6}
denote $\sh{A}(X)$ when there is little chance of confusion; if $U$ is an open subset of $S$, then we have
\[
  \sh{A}(f^{-1}(U))=\sh{A}(X)|U.
\]
Similarly, for every $\OO_X$-module $\sh{F}$ (resp. every $\OO_X$-algebra $\sh{B}$), we write $\sh{A}(\sh{F})$ (resp. $\sh{A}(\sh{B})$) for the direct image $f_*(\sh{F})$ (resp. $f_*(\sh{B})$) which is an $\sh{A}(X)$-module (resp. an $\sh{A}(X)$-algebra) and not only an $\OO_S$-module (resp. an $\OO_S$-algebra).
\end{env}

\begin{env}[1.1.2]
\label{2.1.1.2}
Let $Y$ be a second $S$-prescheme, $g:Y\to S$ its structure morphism, and $h:X\to Y$ an $S$-morphism; we then have the commutative diagram
\[
  \xymatrix{
    X\ar[rr]^h\ar[rd]_f & &
    Y\ar[ld]^g\\
    & S.
  }
  \tag{1.1.2.1}
\]

We have by definition $h=(\psi,\theta)$, where $\theta:\OO_Y\to h_*(\OO_X)=\psi_*(\OO_X)$ is a homomorphism of sheaves of rings; we induce \sref[0]{0.4.2.2} a homomorphism of $\OO_S$-algebras $g_*(\theta):g_*(\OO_Y)\to g_*(h_*(\OO_X))=f_*(\OO_X)$, in other words, a homomorphism of $\OO_S$-algebras $\sh{A}(Y)\to\sh{A}(X)$, which we denote by $\sh{A}(h)$.
If $h':Y\to Z$ is a second $S$-morphism, then it is immediate that $\sh{A}(h'\circ h)=\sh{A}(h)\circ\sh{A}(h')$.
We havve thus define a \emph{contravariant functor $\sh{A}(X)$} from the category of $S$-preschemes to the category of $\OO_S$-algebras.

Now let $\sh{F}$ be an $\OO_X$-module, $\sh{G}$ an $\OO_Y$-module, and $u:\sh{G}\to\sh{F}$ an $h$-morphism, that is \sref[0]{0.4.4.1} a homomorphism of $\OO_Y$-modules $\sh{G}\to h_*(\sh{F})$.
Then $g_*(u):g_*(\sh{G})\to g_*(h_*(\sh{F}))=f_*(\sh{F})$ is a homomorphism $\sh{A}(\sh{G})\to\sh{A}(\sh{F})$ of $\OO_S$-modules, which we denote by $\sh{A}(u)$; in addition, the pair $(\sh{A}(h),\sh{A}(u))$ form a \emph{di-homomorphism} from the $\sh{A}(Y)$-module $\sh{A}(\sh{G})$ to the $\sh{A}(X)$-module $\sh{A}(\sh{F})$.
\end{env}

\begin{env}[1.1.3]
\label{2.1.1.3}
If we fix the prescheme $S$, then we can consider the pairs $(X,\sh{F})$, where $X$ is an $S$-prescheme and $\sh{F}$ is an $\OO_X$-module, as forming a \emph{category}, by defining a \emph{morphism} $(X,\sh{F})\to(Y,\sh{G})$ as a pair $(h,u)$, where $h:X\to Y$ is an $S$-morphism and $u:\sh{G}\to\sh{F}$ is an $h$-morphism.
We can theen say that $(\sh{A}(X),\sh{A}(\sh{F}))$ is a \emph{contravariant functor} with values in the category whose objects are pairs consisting of an $\OO_S$-algebra and a module over that algebra, and the morphisms are the di-homomorphisms.
\end{env}

\subsection{Affine preschemes over a prescheme}
\label{subsection:affine-preschemes-over-a-prescheme}

\begin{defn}[1.2.1]
\label{2.1.2.1}
Let $X$ be an $S$-prescheme, $f:X\to S$ its structure morphism.
We say that $X$ is \emph{affine over $S$} if there exists a cover $(S_\alpha)$ of $S$ by affine open sets such that for all $\alpha$, the induced prescheme on $X$ by the open set $f^{-1}(S_\alpha)$ is affine.
\end{defn}

\begin{exm}[1.2.2]
\label{2.1.2.2}
Every closed subprescheme of $S$ is an affine $S$-prescheme over $S$ (\sref[I]{1.4.2.3} and \sref[I]{1.4.2.4}).
\end{exm}

\begin{rmk}[1.2.3]
\label{2.1.2.3}
An affine prescheme $X$ over $S$ is not necessarily an affine scheme, as the example $X=S$ shows \sref{2.1.2.2}.
On the other hand, if an affine scheme $X$ is an $S$-prescheme, then $X$ is not necessarily affine over
\oldpage[II]{7}
$S$ (see Example~\sref{2.1.3.3}).
However, remember that if $S$ is a \emph{scheme}, then every $S$-prescheme which is an affine scheme is affine over $S$ \sref[I]{1.5.5.10}.
\end{rmk}

\begin{prop}[1.2.4]
\label{2.1.2.4}
Every $S$-prescheme which is affine over $S$ is separated over $S$ (in other words, it is an $S$-scheme).
\end{prop}

\begin{proof}
\label{proof-2.1.2.4}
This follows immediately from \sref[I]{1.5.5.5} and \sref[I]{1.5.5.8}.
\end{proof}

\begin{prop}[1.2.5]
\label{2.1.2.5}
Let $X$ be an $S$-scheme affine over $S$, $f:X\to S$ its structure morphism.
For every open $U\subset S$, $f^{-1}(U)$ is affine over $U$.
\end{prop}


















\section{Homogeneous prime spectra}
\label{section:II.2}

\subsection{Generalities on graded rings and modules}
\label{subsection:II.2.1}

\begin{notation}[2.1.1]
\label{II.2.1.1}
Given a \emph{positively-graded} ring $S$, we denote by $S_n$ the subset of $S$ consisting of homogeneous elements of degree $n$ ($n\geq 0$), by $S_+$ the (direct) sum of the $S_n$ for $n>0$;
we have $1\in S_0$, $S_0$ is a subring of $S$, $S_+$ is a graded ideal of $S$, and $S$ is the direct sum of $S_0$ and $S_+$.
If $M$ is a \emph{graded} module over $S$ (with positive or negative degrees), we similarly denote by $M_n$ the $S_0$-module consisting of homogeneous elements of $M$ of degree $n$ (with $n\in\bb{Z}$).

For every integer $d>0$, we denote by $S^{(d)}$ the direct sum of the $S_{nd}$;
by considering the elements of $S_{nd}$ as homogeneous of degree $n$, the $S_{nd}$ define on $S^{(d)}$ a graded ring structure.

For every integer $k$ such that $0\leq k\leq d-1$, we denote by $M^{(d,k)}$ the direct sum
\oldpage[II]{20}
of the $M_{nd+k}$ ($n\in\bb{Z}$);
this is a graded $S^{(d)}$-module when we consider the elements of $M_{nd+k}$ as homogeneous of degree $n$.
We write $M^{(d)}$ in place of $M^{(d,0)}$.

With the above notation, for every integer $n$ (positive or negative), we denote by $M(n)$ the graded $S$-module defined by $(M(n))_k=M_{n+k}$ for every $k\in\bb{Z}$.
In particular, $S(n)$ will be a graded $S$-module such that $(S(n))_k=S_{n+k}$, by agreeing to set $S_n=0$ for $n<0$.
We say that a graded $S$-module $M$ is \emph{free} if it is isomorphic, considered as a \emph{graded} module, to a direct sum of modules of the form $S(n)$;
as $S(n)$ is a monogeneous $S$-module, generated by the element $1$ of $S$ considered as an element of degree $-n$, it is equivalent to say that $M$ admits a \emph{basis} over $S$ consisting of \emph{homogeneous} elements.

We say that a graded $S$-module $M$ \emph{admits a finite presentation} if there exists an exact sequence $P\to Q\to M\to 0$, where $P$ and $Q$ are finite direct sums of modules of the form $S(n)$ and the homomorphisms are of degree $0$ (cf.~\sref{II.2.1.2}).
\end{notation}

\begin{env}[2.1.2]
\label{II.2.1.2}
Let $M$ and $N$ be two graded $S$-modules;
we define on $M\otimes_S N$ a \emph{graded} $S$-module structure in the following way.
On the tensor product $M\otimes_\bb{Z}N$, we can define a graded $\bb{Z}$-module structure (where $\bb{Z}$ is graded by $\bb{Z}_0=\bb{Z}$, $\bb{Z}_n=0$ for $n\neq 0$) by setting $(M\otimes_\bb{Z}N)_q=\bigoplus_{m+n=q}M_m\otimes_\bb{Z}N_n$ (as $M$ and $N$ are respectively direct sums of the $M_m$ and the $N_n$, we know that we can canonically identify $M\otimes_\bb{Z}N$ with the direct sum of all the $M_m\otimes_\bb{Z}N_n$).
This being so, we have $M\otimes_S N=(M\otimes_\bb{Z}N)/P$, where $P$ is the $\bb{Z}$-submodule of $M\otimes_\bb{Z}N$ generated by the elements $(xs)\otimes y-x\otimes(sy)$ for $x\in M$, $y\in N$, $s\in S$;
it is clear that $P$ is a \emph{graded} $\bb{Z}$-submodule of $M\otimes_\bb{Z}N$, and we see immediately that we obtain a graded $S$-module structure on $M\otimes_S N$ by passing to the quotient.

For two graded $S$-modules $M$ and $N$, recall that a homomorphism $u:M\to N$ of $S$-modules is said to be \emph{of degree $k$} if $u(M_j)\subset N_{j+k}$ for all $j\in\bb{Z}$.
If $H_n$ denotes the set of all the homomorphisms of degree $n$ from $M$ to $N$, then we denote by $\Hom_S(M,N)$ the (direct) \emph{sum} of the $H_n$ ($n\in\bb{Z}$) in the $S$-module $H$ of all the homomorphisms (of $S$-modules) from $M$ to $N$;
in general, $\Hom_S(M,N)$ is not equal to the later.
However, we have $H=\Hom_S(M,N)$ when $M$ is \emph{of finite type};
indeed, we can then suppose that $M$ is generated by a finite number of homogeneous elements $x_i$ ($1\leq i\leq n$), and every homomorphism $u\in H$ can be written in a unique way as $\sum_{k\in\bb{Z}}u_k$, where for each $k$, $u_k(x_i)$ is equal to the homogeneous component of degree $k+\deg(x_i)$ of $u(x_i)$ ($1\leq i\leq n$), which implies that $u_k=0$ except for a finite number of indices;
we have by definition that $u_k\in H_k$, hence the conclusion.

We say that the elements of degree $0$ of $\Hom_S(M,N)$ are the \emph{homomorphisms of graded $S$-modules}.
It is clear that $S_m H_n\subset H_{m+n}$, so the $H_n$ define on $\Hom_S(M,N)$ a graded $S$-module structure.

It follows immediately from these definitions that we have
\[
\label{II.2.1.2.1}
  M(m)\otimes_S N(n)=(M\otimes_S N)(m+n),
\tag{2.1.2.1}
\]
\[
\label{II.2.1.2.2}
  \Hom_S(M(m),N(n))=(\Hom_S(M,N))(n-m),
\tag{2.1.2.2}
\]
for two graded $S$-modules $M$ and $N$.

\oldpage[II]{21}
Let $S$ and $S'$ be two graded rings;
a homomorphism of \emph{graded rings $\vphi:S\to S'$} is a homomorphism of rings such that $\vphi(S_n)\subset S_n'$ for all $n\in\bb{Z}$ (in other words, $\vphi$ must be a homomorphism \emph{of degree $0$} of graded $\bb{Z}$-modules).
The data of such a homomorphism defines on $S'$ a \emph{graded} $S'$-module structure;
equipped with this structure and its graded ring structure, we say that $S'$ is a \emph{graded $S'$-algebra}.

If $M$ is also a graded $S$-module, then the tensor product $M\otimes_S S'$ of \emph{graded} $S$-modules is equipped in a natural way with a \emph{graded} $S'$-module structure, the grading being defined as above.
\end{env}

\begin{lemma}[2.1.3]
\label{II.2.1.3}
Let $S$ be a ring graded in positive degrees.
For a subset $E$ of $S_+$ consisting of homogeneous elements to generate $S_+$ as an $S$-module, it is necessary and sufficient for $E$ to generate $S$ an an $S_0$-algebra.
\end{lemma}

\begin{proof}
The condition is evidently sufficient; we show that it is necessary.
Let $E_n$ (resp. $E^n$) be the set of elements of $E$ equal to $n$ (resp. $\leq n$);
it suffices to show, by induction on $n>0$, that $S_n$ is the $S_0$-module generated by the elements of degree $n$ which are products of elements of $E^n$.
This is evident for $n=1$ by virtue of the hypothesis;
the latter also shows that $S_n=\sum_{p=0}^{n-1}S_p E_{n-p}$, and the induction argument is then immediate.
\end{proof}

\begin{corollary}[2.1.4]
\label{II.2.1.4}
For $S_+$ to be an ideal of finite type, it is necessary and sufficient for $S$ to be an $S_0$-algebra of finite type.
\end{corollary}

\begin{proof}
We can always assume that a finite system of generators of the $S_0$-algebra $S$ (resp. of the $S$-ideal $S_+$) consists of homogeneous elements, by replacing each of the generators considered by its homogeneous components.
\end{proof}

\begin{corollary}[2.1.5]
\label{II.2.1.5}
For $S$ to be Noetherian, it is necessary and sufficient for $S_0$ to be Noetherian and for $S$ to be an $S_0$-algebra of finite type.
\end{corollary}

\begin{proof}
The condition is evidently sufficient;
it is necessary, since $S_0$ is isomorphic to $S/S_+$ and $S_+$ must be an ideal of finite type \sref{II.2.1.4}.
\end{proof}

\begin{lemma}[2.1.6]
\label{II.2.1.6}
Let $S$ be a ring graded in positive degrees, which is an $S_0$-algebra of finite type.
Let $M$ be a graded $S$-module of finite type.
Then:
\begin{enumerate}
  \item[{\rm(i)}] The $M_n$ are $S_0$-modules of finite type, and there exists an integer $n_0$ such that $M_n=0$ for $n\leq n_0$.
  \item[{\rm(ii)}] There exists an integer $n_1$ and an integer $h>0$ such that, for every integer $n\geq n_1$, we have $M_{n+h}=S_h M_n$.
  \item[{\rm(iii)}] For every pair of integers $(d,k)$ such that $d>0$, $0\leq k\leq d-1$, $M^{(d,k)}$ is an $S^{(d)}$-module of finite type.
  \item[{\rm(iv)}] For every integer $d>0$, $S^{(d)}$ is an $S_0$-algebra of finite type.
  \item[{\rm(v)}] There exists an integer $h>0$ such that $S_{mh}=(S_h)^m$ for all $m>0$.
  \item[{\rm(vi)}] For every integer $n>0$, there exists an integer $m_0$ such that $S_m\subset S_+^n$ for all $m\geq m_0$.
\end{enumerate}
\end{lemma}

\begin{proof}
We can assume that $S$ is generated (as an $S_0$-algebra) by homogeneous elements $f_i$, of degrees $h_i$ ($1\leq i\leq r$), and $M$ is generated (as an $S$-module) by homogeneous elements $x_j$ of degrees $k_j$ ($1\leq j\leq s$).
It is clear that $M_n$ is formed by linear combinations,
\oldpage[II]{22}
with coefficients in $S_0$, of elements $f_1^{\alpha_1}\cdots f_r^{\alpha_r}x_j$ such that the $\alpha_i$ are integers $\geq 0$ satisfying $k_j+\sum_i\alpha_i h_i=n$;
for each $j$, there are only finitly many systems $(\alpha_i)$ satisfying this equation, since the $h_i$ are $>0$, hence the first assertion of (i);
the second is evident.
On the other hand, let $h$ be the l.c.m. of the $h_i$ and set $g_i=f_i^{h/h_i}$ ($1\leq i\leq r$) such that all the $g_i$ are of degree $h$;
let $z_\mu$ be the elements of $M$ of the form $f_1^{\alpha_1}\cdots f_r^{\alpha_r}x_j$ with $0\leq\alpha_i<h/h_i$ for $1\leq i\leq r$;
there are finitely many of these elements, so let $n_1$ be the largest of their degrees.
It is clear that for $n\geq n_1$, every element of $M_{n+h}$ is a linear combination of the $z_\mu$ whose cofficients are monomials of degree $>0$ with respect to the $g_i$, so we have $M_{n+h}=S_h M_n$, which establishes (ii).
In a similar way, we see (for all $d>0$) that an element of $M^{(d,k)}$ is a linear combinations, with coeffients in $S_0$, of elements of the form $g^d f_1^{\alpha_1}\cdots f_r^{\alpha_r}x_j$ with $0\leq\alpha_i<d$, $g$ being a homogeneous element of $S$;
hence (iii);
(iv) then follows from (iii) and from Lemma~\sref{II.2.1.3}, by taking $M=S_+$, since $(S_+)^{(d)}=(S^{(d)})_+$.
The assertion of (v) is deduced from (ii) by taking $M=S$.
Finally, for a given $n$, there are finitely many systems $(\alpha_i)$ such that $\alpha_i\geq 0$ and $\sum_i\alpha_i<n$, so if $m_0$ is the largest value of the sum $\sum_i\alpha_i h_i$ of these systems, then we have $S_m\subset S_+^n$ for $m>m_0$, which proves (vi).
\end{proof}

\begin{corollary}[2.1.7]
\label{II.2.1.7}
If $S$ is Noetherian, then so is $S^{(d)}$ for every integer $d>0$.
\end{corollary}

\begin{proof}
This follows from \sref{II.2.1.5} and \sref{II.2.1.6}[iv].
\end{proof}

\begin{env}[2.1.8]
\label{II.2.1.8}
Let $\mathfrak{p}$ be a \emph{graded} prime ideal of the graded ring $S$;
$\mathfrak{p}$ is thus a direct sum of the subgroups $\mathfrak{p}_n=\mathfrak{p}\cap S_n$.
Suppose that \emph{$\mathfrak{p}$ does not contain $S_+$}.
Then if $f\in S_+$ is not in $\mathfrak{p}$, the relation $f^n x\in\mathfrak{p}$ is equivalent to $x\in\mathfrak{p}$;
in particular, if $f\in S_d$ ($d>0$), for all $x\in S_{m-nd}$, then the relation $f^n x\in\mathfrak{p}_m$ is equivalent to $x\in\mathfrak{p}_{m-nd}$.
\end{env}

\begin{proposition}[2.1.9]
\label{II.2.1.9}
Let $n_0$ be an integer $>0$;
for all $n\geq n_0$, let $\mathfrak{p}_n$ be a subgroup of $S_n$.
For there to exist a graded prime ideal $\mathfrak{p}$ of $S$ not containing $S_+$ and such that $\mathfrak{p}\cap S_n=\mathfrak{p}_n$ for all $n\geq n_0$, it is necessary and sufficient for the following coniditions to be satisfied:
\begin{enumerate}
  \item[{\rm(1st)}] $S_m\mathfrak{p}_n\subset\mathfrak{p}_{m+n}$ for all $m\geq 0$ and all $n\geq n_0$.
  \item[{\rm(2nd)}] For $m\geq n_0$, $n\geq n_0$, $f\in S_m$, $g\in S_n$, the relation $fg\in\mathfrak{p}_{m+n}$ implies $f\in\mathfrak{p}_m$ or $g\in\mathfrak{p}_n$.
  \item[{\rm(3rd)}] $\mathfrak{p}_n\neq S_n$ for at least one $n\geq n_0$.
\end{enumerate}
In addition, the graded prime ideal $\mathfrak{p}$ is then unique.
\end{proposition}

\begin{proof}
It is evident that the conditions (1st) and (2nd) are necessary.
In addition, if $\mathfrak{p}\not\supset S_+$, then there exists at least one $k>0$ such that $\mathfrak{p}\cap S_k\neq S_k$;
if $f\in S_k$ is not in $\mathfrak{p}$, the relation $\mathfrak{p}\cap S_n=S_n$ implies $\mathfrak{p}\cap S_{n-mk}=S_{n-mk}$ according to \sref{2.2.1.8};
therefore, if $\mathfrak{p}\cap S_n=S_n$ for a certain value of $n$, we would have $\mathfrak{p}\supset S_+$ contrary to the hypothesis, which proves that (3rd) is necessary.
Conversely, suppose that the conditions (1st), (2nd), and (3rd) are satisfied.
Note that if for an integer $d\geq n_0$, $f\in S_d$ is not in $\mathfrak{p}_d$, then, if $\mathfrak{p}$ exists, $\mathfrak{p}_m$, for $m<n_0$, is necessarily equal to the set of the $x\in S_m$ such that $f^r x\in\mathfrak{p}_{m+rd}$, except for a finite number of values of $r$.
This already proves that if $\mathfrak{p}$ exists, then it is unique.
It remains to show that if we define the $\mathfrak{p}_m$ for $m<n_0$ by the previous condition, then $\mathfrak{p}=\sum_{n=0}^\infty\mathfrak{p}_n$ is a prime ideal.
First, note that by virtue of (2nd), for $m\geq n_0$, $\mathfrak{p}_m$ is also defined as the set of the $x\in S_m$ such that $f^r x\in\mathfrak{p}_{m+rd}$ except for a finite number of values of $r$.
This
\oldpage[II]{23}
being so, if $g\in S_m$, $x\in\mathfrak{p}_n$, then we have $f^r gx\in\mathfrak{p}_{m+n+rd}$ except for a finite number of values of $r$, so $gx\in\mathfrak{p}_{m+n}$, which proves that $\mathfrak{p}$ is an ideal of $S$.
To establish that this ideal is prime, in other words that the ring $S/\mathfrak{p}$, graded by the subgroups $S_n/\mathfrak{p}_n$, is an integral domain, it suffices (by considering the components of higher degree of two elements of $S/\mathfrak{p}$) to prove that if $x\in S_m$ and $y\in S_n$ are such that $x\not\in\mathfrak{p}_m$ and $y\not\in\mathfrak{p}_n$, then $xy\not\in\mathfrak{p}_{m+n}$.
If not, for $r$ large enough, we would have $f^{2r}xy\in\mathfrak{p}_{m+n+2rd}$;
but we have $f^r y\not\in\mathfrak{p}_{n+rd}$ for all $r>0$;
it then follows from (2nd) that, except for a finite number of values of $r$, we have $f^r x\in\mathfrak{p}_{m+rd}$, and we conclude that $x\in\mathfrak{p}_m$ contrary to the hypothesis.
\end{proof}

\begin{env}[2.1.10]
\label{II.2.1.10}
We say that a subset $\mathfrak{J}$ of $S_+$ is an \emph{ideal of $S_+$} if it is an ideal of $S$, and $\mathfrak{J}$ is a \emph{graded prime ideal of $S_+$} if it is the intersection of $S_+$ and a graded prime ideal of $S$ \emph{not containing $S_+$} (this prime ideal is also unique according to Proposition~\sref{II.2.1.9}).
If $\mathfrak{J}$ is an ideal of $S_+$, the \emph{radical of $\mathfrak{J}$ in $S_+$} is the set of elements of $S_+$ which have a power in $\mathfrak{J}$, in other words the set $\rad_+(\mathfrak{J})=\rad(\mathfrak{J})\cap S_+$;
in particular, the radical of $0$ in $S_+$ is then called the \emph{nilradical} of $S_+$ and denoted by $\nilrad_+$: this is the set of nilpotent elements of $S_+$.
If $\mathfrak{J}$ is an \emph{graded} ideal of $S_+$, then its radical $\rad_+(\mathfrak{J})$ is a \emph{graded} ideal: by passing to the quotient ring $S/\mathfrak{J}$, we can reduce to the case $\mathfrak{J}=0$, and it remains to see that if $x=x_h+x_{h+1}+\cdots+x_k$ is nilpotent, then so are the $x_i\in S_i$ ($1\leq h\leq i\leq k$);
we can assume $x_k\neq 0$ and the component of highest degree of $x^n$ is then $x_k^n$, hence $x_k$ is nilpotent, and we then argue by induction on $k$.
We say that the graded ring $S$ is \emph{essentially reduced} if $\nilrad_+=0$, in other words, if $S_+$ does not contain nilpotent elements $\neq 0$.
\end{env}

\begin{env}[2.1.11]
\label{II.2.1.11}
We note that if, in the graded ring $S$, an element $x$ is a zero-divisor, then so is its component of highest degree.
We say that a ring $S$ is \emph{essentially integral} if the ring $S_+$ (\emph{without the unit element}) does not contain a zero-divisor and is $\neq 0$;
it suffices that a homogeneous element $\neq 0$ in $S_+$ is not a zero-divisor in this ring.
It is clear that if $\mathfrak{p}$ is a graded prime ideal of $S_+$, then $S/\mathfrak{p}$ is essentially integral.

Let $S$ be an essentially integral graded ring, and let $x_0\in S_0$:
if there then exists \emph{a} homogeneous element $f\neq 0$ of $S_+$ such that $x_0 f=0$, then we have $x_0 S_+=0$, since we have $(x_0 g)f=(x_0 f)g=0$ for all $g\in S_+$, and the hypothesis thus implies $x_0 g=0$.
For $S$ to be integral, it is necessary and sufficient for $S_0$ to be integral and the annihilator of $S_+$ in $S_0$ to be $0$.
\end{env}

\subsection{Rings of fractions of a graded ring}
\label{subsection:II.2.2}

\begin{env}[2.2.1]
\label{II.2.2.1}
Let $S$ be a graded ring, in positive degrees, $f$ a \emph{homogeneous} element of $S$, of degree $d>0$;
then the ring of fractions $S'=S_f$ is graded, taking for $S_n'$ the set of the $x/f^k$, where $x\in S_{n+kd}$ with $k\geq 0$ (we observe here that $n$ can take arbitrary negative values);
we denote the subring $S_0'=(S_f)_0$ of $S'$ consisting of elements \emph{of degree $0$} by the notation $S_{(f)}$.

If $f\in S_d$, then the monomials $(f/1)^h$ in $S_f$ ($h$ a positive or negative integer) form a \emph{free system} over the ring $S_{(f)}$, and the set of their linear combinations is none other than
\oldpage[II]{24}
the ring $(S^{(d)})_f$, which is thus \emph{isomorphic to $S_{(f)}[T,T^{-1}]=S_{(f)}\otimes_\bb{Z}\bb{Z}[T,T^{-1}]$} (where $T$ is an indeterminate).
Indeed, if we have a relation $\sum_{h=-a}^b z_h(f/1)^h=0$ with $z_h=x_h/f^m$, where the $x_h$ are in $S_{md}$, then this relation is equivalent by definition to the existence of a $k>-a$ such that $\sum_{h=-a}^b f^{h+k}x_h=0$, and as the degrees of the terms of this sum are distinct, we have $f^{h+k}x_h=0$ for all $h$, hence $z_h=0$ for all $h$.

If $M$ is a graded $S$-module, then $M'=M_f$ is a graded $S_f$-module, $M_n'$ being the set of the $z/f^k$ with $z\in M_{n+kd}$ ($k\geq 0$);
we denote by $M_{(f)}$ the set of the homomogenous elements of degree $0$ of $M'$;
it is immediate that $M_{(f)}$ is an $S_{(f)}$-module and that we have $(M^{(d)})_f=M_{(f)}\otimes_{S_{(f)}}(S^{(d)})_f$.
\end{env}

\begin{lemma}[2.2.2]
\label{II.2.2.2}
Let $d$ and $e$ be integers $>0$, $f\in S_d$, $g\in S_e$.
There exists a canonical ring isomorphism
\[
  S_{(fg)}\isoto(S_{(f)})_{g^d/f^e};
\]
if we canonically identify these two rings, then there exists a canonical module isomorphism
\[
  M_{(fg)}\isoto(M_{(f)})_{g^d/f^e}.
\]
\end{lemma}

\begin{proof}
Indeed, $fg$ divides $f^e g^d$, and this latter element divides $(fg)^{de}$, so the graded rings $S_{fg}$ and $S_{f^e g^d}$ are canonically identified;
on the other hand, $S_{f^e g^d}$ also identifies with $(S_{f^e})_{g^d/1}$ \sref[0]{0.1.4.6}, and as $f^e/1$ is invertible in $S_{f^e}$, $S_{f^e g^d}$ also identifies with $(S_{f^e})_{g^d/f^e}$.
The element $g^d/f^e$ is of degree $0$ in $S_{f^e}$;
we immediately conclude that the subring of $(S_{f^e})_{g^d/f^e}$ consisting of elements of degree $0$ is $(S_{(f^e)})_{g^d/f^e}$, and as we evidently have $S_{(f^e)}=S_{(f)}$, this proves the first part of the proposition;
the second is established in a similar way.
\end{proof}

\begin{env}[2.2.3]
\label{II.2.2.3}
Under the hypotheses of \sref{II.2.2.2}, it is clear that the canonical homomorphism $S_f\to S_{fg}$ \sref[0]{0.1.4.1}, which sends $x/f^k$ to $g^k x/(fg)^k$, is of degree $0$, thus gives by restriction a \emph{canonical homomorphism $S_{(f)}\to S_{(fg)}$}, such that the diagram
\[
  \xymatrix{
    & S_{(f)}\ar[dl]\ar[dr]\\
    S_{(fg)}\ar[rr]^-{\sim} & &
    (S_{(f)})_{g^d/f^e}
  }
\]
is commutative.
We similarly define a canonical homomorphism $M_{(f)}\to M_{(fg)}$.
\end{env}

\begin{lemma}[2.2.4]
\label{II.2.2.4}
If $f$ and $g$ are two homogeneous elements of $S_+$, then the ring $S_{(fg)}$ is generated by the union of the canonical images of $S_{(f)}$ and $S_{(g)}$.
\end{lemma}

\begin{proof}
By virtue of Lemma~\sref{II.2.2.2}, it suffices to see that $1/(g^d/f^e)=f^{d+e}/(fg)^d$ belongs to the canonical image of $S_{(g)}$ in $S_{(fg)}$, which is evident by definition.
\end{proof}

\begin{proposition}[2.2.5]
\label{II.2.2.5}
Let $d$ be an integer $>0$ and let $f\in S_d$.
Then there exists a canonical ring isomorphisms $S_{(f)}\isoto S^{(d)}/(f-1)S^{(d)}$;
if we identify these two rings by this isomorphism, then there exists a canonical module isomorphism $M_{(f)}\isoto M^{(d)}/(f-1)M^{(d)}$.
\end{proposition}

\begin{proof}
The first of these isomorphisms is defined by sending $x/f^n$, where $x\in S_{nd}$, to the element $\overline{x}$, the class of $x\text{ mod. }(f-1)S^{(d)}$;
this map is well-defined, because we have the congruence $f^h x\equiv x\,(\text{mod.}\,(f-1)S^{(d)})$ for all $x\in S^{(d)}$, so if $f^h x=0$ for an $h>0$,
\oldpage[II]{25}
then we have $\overline{x}=0$.
On the other hand, if $x\in S_{nd}$ is such that $x=(f-1)y$ with $y=y_{hd}+y_{(h+1)d}+\cdots+y_{kd}$ with $y_{jd}\in S_{jd}$ and $y_{hd}\neq 0$, then we necessarily have $h=n$ and $x=-y_{hd}$, as well as the relations $y_{(j+1)d}=fy_{jd}$ for $h\leq j\leq k-1$, $fy_{kd}=0$, which ultimately gives $f^{k-n}x=0$;
we send every class $\overline{x}\text{ mod. }(f-1)S^{(d)}$ of an element $x\in S_{nd}$ to the element $x/f^n$ of $S_{(f)}$, since the preceding remark shows that this map is well-defined.
It is immediate that these two maps thus defined are ring homomorphisms, each the reciprocal of the other.
We proceed exactly the same way for $M$.
\end{proof}

\begin{corollary}[2.2.6]
\label{II.2.2.6}
If $S$ is Noetherian, then so is $S_{(f)}$ for $f$ homogeneous of degree $>0$.
\end{corollary}

\begin{proof}
This follows immediately from Corollary~\sref{II.2.1.7} and Proposition~\sref{II.2.2.5}.
\end{proof}

\begin{env}[2.2.7]
\label{II.2.2.7}
Let $T$ be a multiplicative subset of $S_+$ consisting of \emph{homogeneous} elements;
$T_0=T\cup\{1\}$ is then a multiplicative subset of $S$;
as the elements of $T_0$ are homogeneous, the ring $T_0^{-1}S$ is still graded in the evident way;
we denote by $S_{(T)}$ the subring of $T_0^{-1}S$ consisting of elements of order $0$, that is to say, the elements of the form $x/h$, where $h\in T$ and $x$ is homogeneous of degree equal to that of $h$.
We know \sref[0]{0.1.4.5} that $T_0^{-1}S$ is canonically identified with the inductive limit of the rings $S_f$, where $f$ varies over $T$ (with respect to the canonical homomorphisms $S_f\to S_{fg}$);
as this identification respects the degrees, it identifies $S_{(T)}$ with the \emph{inductive limit} of the $S_{(f)}$ for $f\in T$.
For every graded $S$-module $M$, we similarly define the module $M_{(T)}$ (over the ring $S_{(T)}$) consisting of elements of degree $0$ of $T_0^{-1}M$, and we see that this module is the inductive limit of the $M_{(f)}$ for $f\in T$.

If $\mathfrak{p}$ is a graded prime ideal of $S_+$, then we denote by $S_{(\mathfrak{p})}$ and $M_{(\mathfrak{p})}$ the ring $S_{(T)}$ and the module $M_{(T)}$ respectively, where $T$ is the set of \emph{homogeneous} elements of $S_+$ which do not belong to $\mathfrak{p}$.\end{env}

\subsection{Homogeneous prime spectrum of a graded ring}
\label{subsection:II.2.3}

\begin{env}[2.3.1]
\label{II.2.3.1}
Given a graded ring $S$, in positive degrees, we call the \emph{homogeneous prime spectrum} of $S$ and denote it by $\Proj(S)$ the set of graded prime ideals of $S_+$ \sref{II.2.1.10}, or equivalently the set of graded prime ideals of $S$ \emph{not containing $S_+$};
we will define a \emph{scheme} structure having $\Proj(S)$ as the underlying set.
\end{env}

\begin{env}[2.3.2]
\label{II.2.3.2}
For every subset $E$ of $S$, let $V_+(E)$ be the set of graded prime ideals of $S$ containing $S$ and not containing $S_+$;
this is thus the subset $V(E)\cap\Proj(S)$ of $\Spec(S)$.
From \sref[I]{I.1.1.2} we deduce:
\[
\label{II.2.3.2.1}
  V_+(0)=\Proj(S),\ V_+(S)=V_+(S_+)=\emp,
\tag{2.3.2.1}
\]
\[
\label{II.2.3.2.2}
  V_+\big(\textstyle\bigcup_\lambda E_\lambda\big)=\textstyle\bigcap_\lambda V_+(E_\lambda),
\tag{2.3.2.2}
\]
\[
\label{II.2.3.2.3}
  V_+(EE')=V_+(E)\cup V_+(E').
\tag{2.3.2.3}
\]

We do not change $V_+(E)$ by replacing $E$ with the graded ideal generated by $E$;
in addition, if $\mathfrak{J}$ is a graded ideal of $S$, then we have
\[
  V_+(\mathfrak{J})=V_+\big(\textstyle\bigcup_{q\geq n}(\mathfrak{J}\cap S_q)\big)
\tag{2.3.2.4}
\]
\oldpage[II]{26}
for all $n>0$: indeed, if $\mathfrak{p}\in\Proj(S)$ contains the homogeneous elements of $\mathfrak{J}$ of degree $\geq n$, then as by hypothesis there exists a homogeneous element $f\in S_d$ not contained in $\mathfrak{p}$, for every $m\geq 0$ and every $x\in S_m\cap\mathfrak{J}$, we have $f^r x\in\mathfrak{J}\cap S_{m+rd}$ for all but finitely many values of $r$, so $f^r x\in\mathfrak{p}\cap S_{m+rd}$, which implies that $x\in\mathfrak{p}\cap S_m$ \sref{II.2.1.9}.

Finally, we have, for every graded ideal $\mathfrak{J}$ of $S$,
\[
  V_+(\mathfrak{J})=V_+(\rad_+(\mathfrak{J})).
\tag{2.3.2.5}
\]
\end{env}

\begin{env}[2.3.3]
\label{II.2.3.3}
By definition, the $V_+(E)$ are the closed subsets of $X=\Proj(S)$ for the topology induced by the spectral topology of $\Spec(S)$, which we also call the \emph{spectral topology} on $X$.
For all $f\in S$, we set
\[
\label{II.2.3.3.1}
  D_+(f) = D(f)\cap\Proj(S) = \Proj(S)\setmin V_+(f)
\tag{2.3.3.1}
\]
and so, for any two elements $f$ and $g$ of $S$ \sref[I]{I.1.1.9.1},
\[
\label{II.2.3.3.2}
  D_+(fg) = D_+(f)\cap D_+(g).
\tag{2.3.3.2}
\]
\end{env}

\begin{proposition}[2.3.4]
\label{II.2.3.4}
The $D_+(f)$, as $f$ runs over the set of homogeneous elements of $S_+$, form a base for the topology of $X=\Proj(S)$.
\end{proposition}

\begin{proof}
It follows from \sref{II.2.3.2.2} and \sref{II.2.3.2.4} that every closed subset of $X$ is the intersection of sets of the form $V_+(f)$, where $f$ is homogeneous of degree $>0$.
\end{proof}

\begin{env}[2.3.5]
\label{II.2.3.5}
Let $f$ be a \emph{homogeneous} element of $S_+$, of degree $d>0$;
for every graded prime ideal $\mathfrak{p}$ of $S$ that does not contain $f$, we know that the set of the $x/f^n$, where $x\in\mathfrak{p}$ and $n\geq0$, is a prime ideal of the ring of fractions $S_f$ \sref[0]{0.1.2.6};
its intersection with $S_{(f)}$ is thus a prime ideal of $S_{(f)}$, which we denote by $\psi_f(\mathfrak{p})$:
it is the set of the $x/f^n$ for $n\geq0$ and $x\in\mathfrak{p}\cap S_{nd}$.
We have thus defined a map
\[
  \psi_f\colon D_+(f)\to\Spec(S_{(f)});
\]
furthermore, if $g\in S_e$ is another homogeneous element of $S_+$, then we have a commutative diagram
\[
\label{II.2.3.5.1}
  \xymatrix{
    D_+(f) \ar[r]^{\psi_f}
    & \Spec(S_{(f)})
  \\D_+(fg) \ar[u] \ar[r]_{\psi_{fg}}
    & \Spec(S_{(fg)}) \ar[u]
  }
\tag{2.3.5.1}
\]
where the vertical arrow on the left is the inclusion, and the vertical arrow on the right is the map ${}^a\!\omega_{fg,f}$ induced by the canonical homomorphism $\omega=\omega_{fg,f}\colon S_{(f)}\to S_{(fg)}$ \sref[I]{I.1.2.1}.
Indeed, if $x/f^n\in\omega^{-1}(\psi_{fg}(\mathfrak{p}))$, with $fg\not\in\mathfrak{p}$, then, by definition, $g^nx/(fg)^n\in\psi_{fg}(\mathfrak{p})$, so $g^nx\in\mathfrak{p}$, and so $x\in\mathfrak{p}$;
the converse is evident.
\end{env}

\begin{proposition}[2.3.6]
\label{II.2.3.6}
The map $\psi_f$ is a homeomorphism from $D_+(f)$ to $\Spec(S_{(f)})$.
\end{proposition}

\begin{proof}
Firstly, $\psi_f$ is continuous;
this is since, if $h\in S_{nd}$ is such that $h/f^n\in\psi_f(\mathfrak{p})$, then, by definition, $h\in\mathfrak{p}$, and conversely, and so $\psi_f^{-1}(D(h/f^n))=D_+(hf)$, and our claim then follows from \sref{II.2.3.3.2}.
Furthermore, the $D_+(hf)$, where $h$ runs over the sets $S_{nd}$, form a topology of $D_+(f)$, by \sref{II.2.3.4} and \sref{II.2.3.3.2};
the
\oldpage[II]{27}
above thus proves, taking into account the ($T_0$) axiom, which holds in $D_+(f)$ and in $\Spec(S_{(f)})$, that $\psi_f$ is injective and that the inverse map $\psi_f(D_+(f))\to D_+(f)$ is continuous.
Finally, to see that $\psi_f$ is surjective, we note that, if $\mathfrak{q}_0$ is a prime ideal of $S_{(f)}$, and if, for all $n>0$, we denote by $\mathfrak{p}_n$ the set of $x\in S_n$ such that $x^d/f^n\in\mathfrak{q}_0$, then the $\mathfrak{p}_n$ satisfy the conditions of \sref{II.2.1.9}:
if $x,y\in S_n$ are such that $x^d/f^n,y^d/f^n\in\mathfrak{q}_0$, then $(x+y)^{2d}/f^{2n}\in\mathfrak{q}_0$, whence $(x+y)^d/f^n\in\mathfrak{q}_0$, since $\mathfrak{q}_0$ is prime;
this proves that the $\mathfrak{p}_n$ are subgroups of the $S_n$, and the verification of the other conditions of \sref{II.2.1.9} is immediate, taking into account the fact that $\mathfrak{q}_0$ is prime.
If $\mathfrak{p}$ is the graded prime ideal of $S$ thus defined, then indeed $\psi_f(\mathfrak{p})=\mathfrak{q}_0$, since, if $x\in S_{nd}$, then having $x/f^n\in\mathfrak{q}_0$ and $x^d/f^{nd}\in\mathfrak{q}_0$ is equivalent to $\mathfrak{q}_0$ being prime.
\end{proof}

\begin{corollary}[2.3.7]
\label{II.2.3.7}
To have $D_+(f)=\emp$, it is necessary and sufficient for $f$ to be nilpotent.
\end{corollary}

\begin{proof}
To have $\Spec(S_{(f)})=\emp$, it is necessary and sufficient to have $S_{(f)}=0$, or indeed to have $1=0$ in $S_f$, which means, by definition, that $f$ is nilpotent.
\end{proof}

\begin{corollary}[2.3.8]
\label{II.2.3.8}
Let $E$ be a subset of $S_+$.
Then the following conditions are equivalent:
\begin{enumerate}
  \item[{\rm(a)}] $V_+(E) = X = \Proj(S)$.
  \item[{\rm(b)}] Every element of $E$ is nilpotent.
  \item[{\rm(c)}] The homogeneous components of every element of $E$ are nilpotent.
\end{enumerate}
\end{corollary}

\begin{proof}
It is clear that (c) implies (b), and that (b) implies (a).
If $\mathfrak{J}$ is the graded ideal of $S$ generated by $E$, then condition~(a) is equivalent to requiring that $V_+(\mathfrak{J})=X$;
\emph{a fortiori}, (a) implies that every homogeneous element $f\in\mathfrak{J}$ is such that $V_+(f)=X$, and so $f$ is nilpotent by \sref{II.2.3.7}.
\end{proof}

\begin{corollary}[2.3.9]
\label{II.2.3.9}
If $\mathfrak{J}$ is a graded ideal of $S_+$, then $\mathfrak{r}_+(\mathfrak{J})$ is the intersection of the graded prime ideals of $S_+$ that contain $\mathfrak{J}$.
\end{corollary}

\begin{proof}
By considering the graded ring $S/\mathfrak{J}$, we can reduce to the case where $\mathfrak{J}=0$.
We need to prove that, if $f\in S_+$ is not nilpotent, then there exists a graded prime ideal of $S$ that does not contain $f$;
but at least one of the homogeneous components of $f$ is not nilpotent, and we can thus suppose $f$ to be homogeneous;
the claim then follows from \sref{II.2.3.7}.
\end{proof}

\begin{env}[2.3.10]
\label{II.2.3.10}
For every subset $Y$ of $X=\Proj(S)$, let $\mathfrak{j}_+(Y)$ be the set of $f\in S_+$ such that $Y\subset V_+(f)$;
this is equivalent to saying that $\mathfrak{j}_+(Y)=\mathfrak{j}(Y)\cap S_+$;
then $\mathfrak{j}_+(Y)$ is an ideal of $S_+$ that is equal to its radical in $S_+$.
\end{env}

\begin{proposition}[2.3.11]
\label{II.2.3.11}
\begin{enumerate}
  \item[{\rm(i)}] For every subset $E$ of $S_+$, $\mathfrak{j}_+(V_+(E))$ is the radical in $S_+$ of the graded ideal of $S_+$ generated by $E$.
  \item[{\rm(ii)}] For every subset $Y$ of $X$, $V_+(\mathfrak{j}_+(Y))=\overline{Y}$, where $\overline{Y}$ is the closure of $Y$ in $X$.
\end{enumerate}
\end{proposition}

\begin{proof}
\begin{enumerate}
  \item[{\rm(i)}] If $\mathfrak{J}$ is the graded ideal of $S_+$ generated by $E$, then $V_+(E)=V_+(\mathfrak{J})$, and the claim then follows from \sref{II.2.3.9}.
  \item[{\rm(ii)}] Since $V_+(\mathfrak{J})=\bigcap_{f\in\mathfrak{J}}V_+(f)$, having $Y\subset V_+(\mathfrak{J})$ implies that $Y\subset V_+(f)$ for every $f\in\mathfrak{J}$, and thus $\mathfrak{j}_+(Y)\supset\mathfrak{J}$, whence $V_+(\mathfrak{j}_+(Y))\subset V_+(\mathfrak{J})$, which proves (ii) by the definition of the closed subsets.
\end{enumerate}
\end{proof}

\begin{corollary}[2.3.12]
\label{II.2.3.12}
The closed subsets $Y$ of $X=\Proj(S)$ are in bijective correspondence with the graded ideals of $S_+$ that are equal to their radical in $S_+$, via the inclusion-reversing maps $Y\mapsto\mathfrak{j}_+(Y)$ and $\mathfrak{J}\mapsto V_+(\mathfrak{J})$;
the union $Y_1\cup Y_2$ of two closed subsets of $X$ corresponds
\oldpage[II]{28}
to $\mathfrak{j}_+(Y_1)\cap\mathfrak{j}_+(Y_2)$, and the intersection of an arbitrary family $(Y_\lambda)$ of closed subsets corresponds to the radical in $S_+$ of the sum of the $\mathfrak{j}_+(Y_\lambda)$.
\end{corollary}

\begin{corollary}[2.3.13]
\label{II.2.3.13}
Let $\mathfrak{J}$ be a graded ideal of $S_+$;
to have $V_+(\mathfrak{J})=\emp$, it is necessary and sufficient for every element $f$ of $S_+$ to have a power $f^n$ in $\mathfrak{J}$.
\end{corollary}

This above corollary can also be expressed in one of the following equivalent forms:

\begin{corollary}[2.3.14]
Let $(f_\alpha)$ be a family of homogeneous elements of $S_+$.
For the $D_+(f_\alpha)$ to form a cover of $X=\Proj(S)$, it is necessary and sufficient for every element of $S_+$ to have a power in the ideal generated by the $f_\alpha$.
\label{II.2.3.14}
\end{corollary}

\begin{corollary}[2.3.15]
\label{II.2.3.15}
Let $(f_\alpha)$ be a family of homogeneous elements of $S_+$, and $f$ an element of $S_+$.
Then the following are equivalent:
\begin{enumerate}
  \item[{\rm(a)}] $D_+(f)\subset\bigcup_\alpha D_+(f_\alpha)$;
  \item[{\rm(b)}] $V_+(f)\supset\bigcap_\alpha V_+(f_\alpha)$;
  \item[{\rm(c)}] $f$ has a power in the ideal generated by the $f_\alpha$.
\end{enumerate}
\end{corollary}

\begin{corollary}[2.3.16]
\label{II.2.3.16}
For $X=\Proj(S)$ to be empty, it is necessary and sufficient for every element of $S_+$ to be nilpotent.
\end{corollary}

\begin{corollary}[2.3.17]
\label{II.2.3.17}
In the bijective correspondence described in \sref{II.2.3.12}, the \emph{irreducible} closed subsets of $X$ correspond to the graded \emph{prime} ideals of $S_+$.
\end{corollary}

\begin{proof}
If $Y=Y_1\cup Y_2$, where $Y_1$ and $Y_2$ are distinct closed subsets of $Y$, then
\[
  \mathfrak{j}_+(Y) = \mathfrak{j}_+(Y_1)\cap\mathfrak{j}_+(Y_2)
\]
with the ideals $\mathfrak{j}_+(Y_1)$ and $\mathfrak{j}_+(Y_2)$ being distinct from $\mathfrak{j}_+(Y)$, and so $\mathfrak{j}_+(Y)$ is not prime.
Conversely, if $\mathfrak{J}$ is a graded ideal of $S_+$ that is not prime, then there exists elements $f,g\in S_+$ such that $f\not\in\mathfrak{J}$ and $g\not\in\mathfrak{J}$, but $fg\in\mathfrak{J}$;
then $V_+(f)\supset V_+(\mathfrak{J})$ and $V_+(g)\supset V_+(\mathfrak{J})$, but $V_+(\mathfrak{J})\subset V_+(f)\cup V_+(g)$, by \sref{II.2.3.2.3};
we thus conclude that $V_+(\mathfrak{J})$ is the union of the closed subsets $V_+(f)\cap V_+(\mathfrak{J})$ and $V_+(g)\cap V_+(\mathfrak{J})$, which are distinct from $V_+(\mathfrak{J})$.
\end{proof}


\subsection{The scheme structure on $\mathrm{Proj}(S)$}
\label{subsection:II.2.4}

\begin{env}[2.4.1]
\label{II.2.4.1}
Let $f$ and $g$ be homogeneous elements of $S_+$;
consider the affine schemes $Y_f=\Spec(S_{(f)})$, $Y_g=\Spec(S_{(g)})$, and $Y_{fg}=\Spec(S_{(fg)})$.
By \sref{II.2.2.2}, the morphism $w_{fg,f} = ({}^a\!\omega_{fg,f},\widetilde{\omega}_{fg,f})$ from $Y_{fg}$ to $Y_f$, corresponding to the canonical homomorphism $\omega_{fg,f}\colon S_{(f)}\to S_{(fg)}$, is an \emph{open immersion} \sref[I]{I.1.3.6}.
Using the inverse homeomorphism of $\psi_f\colon D_+(f)\to Y_f$ \sref{II.2.3.6}, we can transport the affine scheme structure of $Y_f$ to $D_+(f)$;
by the commutativity of diagram~\sref{II.2.3.5.1}, the affine scheme $D_+(fg)$ can thus be identified with the induced scheme on the open subset $D_+(fg)$ of the underlying space of the affine scheme $D_+(f)$.
It is then clear (taking \sref{II.2.3.4} into account) that $X=\Proj(S)$ is endowed with a unique \emph{prescheme} structure, whose restriction to each $D_+(f)$ is the affine scheme that we have just defined.
Furthermore:
\end{env}

\begin{proposition}[2.4.2]
\label{II.2.4.2}
The prescheme $\Proj(S)$ is a scheme.
\end{proposition}

\begin{proof}
It suffices \sref[I]{I.5.5.6} to show, for any homogeneous $f$ and $g$ in $S_+$, that $D_+(f)\cap D_+(g)=D_+(fg)$ is affine, and that its ring is generated by the canonical images of the rings of $D_+(f)$ and $D_+(g)$;
the first point is evident by definition, and the second follows from \sref{II.2.2.4}.
\end{proof}

\oldpage[II]{29}
Whenever we speak of the homogeneous prime spectrum $\Proj(S)$ as a \emph{scheme}, it will always mean with respect to the structure that we have just defined.

\begin{example}[2.4.3]
\label{II.2.4.3}
Let $S=K[T_1,T_2]$, where $K$ is a field, $T_1$ and $T_2$ are indeterminates, and $S$ is graded by total degree.
It follows from \sref{II.2.3.14} that $\Proj(S)$ is the union of $D_+(T_1)$ and $D_+(T_2)$;
we immediately see that these affine schemes are isomorphic to $K[T]$, and that $\Proj(S)$ is obtained by the gluing of these two affine schemes as described in \sref[I]{I.2.3.2} (cf. \sref{II.7.4.14}).
\end{example}

\begin{proposition}[2.4.4]
\label{II.2.4.4}
Let $S$ be a positively-graded ring, and let $X$ be the scheme $\Proj(S)$.
\begin{enumerate}
  \item[{\rm(i)}] If $\mathfrak{N}_+$ is the nilradical of $S_+$ \sref{II.2.1.10}, then the scheme $X_\red$ is canonically isomorphic to $\Proj(S/\mathfrak{N}_+)$;
    in particular, if $S$ is essentially reduced, then $\Proj(S)$ is reduced.
  \item[{\rm(ii)}] If $S$ is essentially reduced, then, for $X$ to be integral, it is necessary and sufficient for $S$ to be essentially integral.
\end{enumerate}
\end{proposition}

\begin{proof}
\begin{enumerate}
  \item[{\rm(i)}] Let $\overline{S}$ be the graded ring $S/\mathfrak{N}_+$, and denote by $x\mapsto\overline{x}$ the canonical homomorphism $S\to\overline{S}$ of degree~$0$.
    For all $f\in S_d$ ($d>0$), the canonical homomorphism $S_f\to\overline{S}$ \sref[0]{0.1.5.1} is surjective and of degree~$0$, and thus gives, by restriction, a surjective homomorphism $S_{(f)}\to\overline{S}_{(\overline{f})}$;
    if we suppose that $f\not\in\mathfrak{N}_+$, then we immediately see that $\overline{S}_{(\overline{f})}$  is reduced, and that the kernel of the above homomorphism is the nilradical of $S_{(f)}$, or, in other words, that $\overline{S}_{(\overline{f})}=(S_{(f)})_\red$.
    So to this homomorphism corresponds a closed immersion $D_+(\overline{f})\to D_+(f)$ that identifies $D_+(\overline{f})$ with $(D_+(f))_\red$ \sref[I]{I.5.1.2}, and which is, in particular, a homeomorphism of the underlying spaces of these two affine schemes.
    Furthermore, if $g\not\in\mathfrak{N}_+$ is another homogeneous element of $S_+$, then the diagram
    \[
      \xymatrix{
        S_{(f)} \ar[r] \ar[d]
        & \overline{S}_{(\overline{f})} \ar[d]
      \\S_{(fg)} \ar[r]
        & \overline{S}_{(\overline{fg})}
      }
    \]
    is commutative;
    since, further, the $D_+(f)$, for $f$ homogeneous, of degree $>0$, and $f\not\in\mathfrak{N}_+$, form a cover of $X=\Proj(S)$ \sref{II.2.3.7}, we see that the morphisms $D_+(\overline{f})\to D_+(f)$ are the restrictions of a closed immersion $\Proj(\overline{S})\to\Proj(S)$ which is a homeomorphism of the underlying spaces;
    whence the conclusion \sref[I]{I.5.1.2}.
  \item[{\rm(ii)}] Suppose that $S$ is essentially integral, or, in other words, that $(0)$ is a graded prime ideal of $S_+$ that is distinct from $S_+$;
    then $X$ is reduced, by (i), and irreducible, by \sref{II.2.3.17}.
    Conversely, suppose that $S$ is essentially reduced and that $X$ is integral;
    then, for homogeneous $f\neq0$ in $S_+$, we have that $D_+(f)\neq\emp$ \sref{II.2.3.7};
    the hypothesis that $X$ is irreducible implies that $D_+(f)\cap D_+(g)\neq\emp$ for homogeneous $f,g\neq0$ in $S_+$;
    thus $fg\neq0$, by \sref{II.2.3.3.2}, and we thus conclude that $S_+$ has no zero divisors, whence the first claim.
\end{enumerate}
\end{proof}

\begin{env}[2.4.5]
\label{II.2.4.5}
Given a commutative ring $A$, recall that we say that a graded ring $S$ is a \emph{graded $A$-algebra} if it is endowed with the structure of an $A$-algebra such that each of its subgroups $S_n$ is an $A$-module;
for this, it suffices for $S_0$ to be
\oldpage[II]{30}
an $A$-algebra, or, in other words, we define the structure of a graded $A$-algebra on $S$ by defining the structure of an $A$-algebra on $S_0$ and setting $\alpha\cdot x=(\alpha\cdot1)x$ for $\alpha\in A$ and $x\in S_n$.
\end{env}

\begin{proposition}[2.4.6]
\label{II.2.4.6}
Suppose that $S$ is a graded $A$-algebra.
Then, on $X=\Proj(S)$, the structure sheaf $\sh{O}_X$ is an $A$-algebra (where $A$ is considered as a simple sheaf on $X$);
in other words, $X$ is a scheme over $\Spec(A)$.
\end{proposition}

\begin{proof}
It suffices to note that, for every homogeneous $f$ in $S_+$, $S_{(f)}$ is an algebra over $A$, and that the diagram
\[
  \xymatrix{
    S_{(f)} \ar[rr] && S_{(fg)}
  \\ & A \ar[ul] \ar[ur] &
  }
\]
is commutative, for homogeneous $f,g$ in $S_+$.
\end{proof}

\begin{proposition}[2.4.7]
\label{II.2.4.7}
Let $S$ be a positively-graded ring.
\begin{enumerate}
  \item[{\rm(i)}] For every integer $d>0$, there exists a canonical isomorphism from the scheme $\Proj(S)$ to the scheme $\Proj(S^{(d)})$.
  \item[{\rm(ii)}] Let $S'$ be the graded ring such that $S_0=\bb{Z}$ and $S'_n=S_n$ (considered as a $\bb{Z}$-module) for $n>0$.
    Then there exists a canonical isomorphism from the scheme $\Proj(S)$ to the scheme $\Proj(S')$.
\end{enumerate}
\end{proposition}

\begin{proof}
\begin{enumerate}
  \item[{\rm(i)}] We first show that the map $\mathfrak{p}\mapsto\mathfrak{p}\cap S^{(d)}$ is a bijection from the set $\Proj(S)$ to the set $\Proj(S^{(d)})$.
    Indeed, suppose that we have a graded prime ideal $\mathfrak{p}'\in\Proj(S^{(d)})$, and let $\mathfrak{p}_{nd}=\mathfrak{p}'\cap S_{nd}$ ($n\geq0$).
    For all $n>0$ that are not multiples of $d$, define $\mathfrak{p}_n$ as the set of $x\in S_n$ such that $x^d\in\mathfrak{p}_{nd}$;
    if $x,y\in\mathfrak{p}_n$, then $(x+y)^{2d}\in\mathfrak{p}_{2nd}$, and so $(x+y)^d\in\mathfrak{p}_{nd}$, since $\mathfrak{p}'$ is prime;
    it is immediate that the $\mathfrak{p}_n$ thus defined, for $n\geq0$, satisfy the conditions of \sref{II.2.1.9}, and so there exists a unique prime ideal $\mathfrak{p}\in\Proj(S)$ such that $\mathfrak{p}\cap S^{(d)}=\mathfrak{p}'$.
    Since, for every homogeneous $f$ in $S_+$, we have that $V_+(f)=V_+(f^d)$ \sref{II.2.3.2.3}, we see that the above bijection is a \emph{homeomorphism} of topological spaces.
    Finally, with the same notation, $S_{(f)}$ and $S_{(f^d)}$ are canonically identified \sref{II.2.2.2}, and so $\Proj(S)$ and $\Proj(S^{(d)})$ are canonically identified as \emph{schemes}.
  \item[{\rm(ii)}] If, to each $\mathfrak{p}\in\Proj(S)$, we associate the unique prime ideal $\mathfrak{p}'\in\Proj(S')$ such that $\mathfrak{p}'\cap S_n=\mathfrak{p}\cap S_n$ for every $n>0$ \sref{II.2.1.9}, then it is clear that we have defined a canonical homeomorphism $\Proj(S)\xrightarrow{\sim}\Proj(S')$ of the underlying spaces, since $V_+(f)$ is the same set for $S$ and $S'$ when $f$ is a homogeneous element of $S_+$.
  Since, further, $S_{(f)}=S'_{(f)}$, $\Proj(S)$ and $\Proj(S')$ can be identified as \emph{schemes}.
\end{enumerate}
\end{proof}

\begin{corollary}[2.4.8]
\label{II.2.4.8}
If $S$ is a graded $A$-algebra, and $S'_A$ the graded $A$-algebra such that $(S'_A)_0=A$ and $(S'_A)_n=S_n$ for $n>0$, then there exists a canonical isomorphism from $\Proj(S)$ to $\Proj(S'_A)$.
\end{corollary}

\begin{proof}
In fact, these two schemes are both canonically isomorphic to $\Proj(S')$, using the notation of \sref{II.2.4.7}[(ii)].
\end{proof}


\subsection{The sheaf associated to a graded module}
\label{subsection:II.2.5}

\begin{env}[2.5.1]
\label{II.2.5.1}
Let $M$ be a \emph{graded} $S$-module.
The, for every homogeneous $f$ in $S_+$, $M_{(f)}$ is an $S_{(f)}$-module, and thus has a corresponding quasi-coherent associated sheaf $(M_{(f)})^\supertilde$ on the affine scheme $\Spec(S_{(f)})$, identified with $D_+(f)$ \sref[I]{I.1.3.4}.
\end{env}

\oldpage[II]{31}
\begin{proposition}[2.5.2]
\label{II.2.5.2}
\end{proposition}

\section{Homogeneous spectrum of a sheaf of graded algebras}
\label{section:2.3}


\subsection{Homogeneous spectrum of a graded quasi-coherent $\mathcal{O}_Y$-algebra}
\label{subsection:2.3.1}


% \subsection{Sheaf on $\Proj(\sh{S})$ associated to a graded $\sh{S}$-module}
% \label{subsection:2.3.2}


% \subsection{Graded $\sh{S}$-module associated to a sheaf on $\Proj(\sh{S})$}
% \label{subsection:2.3.3}


% \subsection{Finiteness conditions}
% \label{subsection:2.3.4}


% \subsection{Functorial behaviour}
% \label{subsection:2.3.5}


% \subsection{Closed subpreschemes of $\Proj(\sh{S})$}
% \label{subsection:2.3.6}


% \subsection{Morphisms from a prescheme to a homogeneous spectrum}
% \label{subsection:2.3.7}


% \subsection{Criteria for immersion into a homogeneous spectrum}
% \label{subsection:2.3.8}

\section{Projective bundles; ample sheaves}
\label{section:II.4}


\subsection{Definition of projective bundles}
\label{subsection:II.4.1}

\begin{definition}[4.1.1]
\label{II.4.1.1}
Let $Y$ be a prescheme, $\sh{E}$ a quasi-coherent $\sh{O}_Y$-module, and $\bb{S}_{\sh{O}_Y}(\sh{E})$ the symmetric $\sh{O}_Y$-algebra of $\sh{E}$ \sref{II.1.7.4}, which is quasi-coherent \sref{II.1.7.7}.
We define the \emph{projective bundle on $Y$ defined by $\sh{E}$}, denoted $\bb{P}(\sh{E})$, to be the $Y$-scheme $P=\Proj(\bb{S}_{\sh{O}_Y}(\sh{E}))$.
The $\sh{O}_P$-module $\sh{O}_P(1)$ is called the \emph{fundamental sheaf on $P$}.
\end{definition}

When $Y$ is affine of ring $A$, then we have $\sh{E}=\widetilde{E}$ for some $A$-module $E$, and we then write $\bb{P}(E)$ instead of $\bb{P}(\widetilde{E})$.

When we take $\sh{E}=\sh{O}_Y^n$, we write $\bb{P}_Y^{n-1}$ instead of $\bb{P}(\sh{E})$;
if, further, $Y$ is affine of ring $A$, then we also write $\bb{P}_A^{n-1}$ instead of $\bb{P}_Y^{n-1}$.
Since $\bb{S}_{\sh{O}_Y}(\sh{O}_Y)$ is canonically identified with $\sh{O}_Y[T]$ \sref{II.1.7.4}, $\bb{P}_Y^0$ is canonically identified with $Y$ \sref{II.3.1.7};
Example~\sref{II.2.4.3} is then exactly $\bb{P}_K^1$.

\begin{env}[4.1.2]
\label{II.4.1.2}
Let $\sh{E}$ and $\sh{F}$ be quasi-coherent $\sh{O}_Y$-modules;
let $u:\sh{E}\to\sh{F}$ be an $\sh{O}_Y$-homomorphism;
there is a canonically corresponding homomorphism $\bb{S}(u):\bb{S}_{\sh{O}_Y}(\sh{E})\to\bb{S}_{\sh{O}_Y}(\sh{F})$ of graded $\sh{O}_Y$-algebras \sref{II.1.7.4}.
If $u$ is \emph{surjective}, then so too is $\bb{S}(u)$, and thus \sref{II.3.6.2}[(i)] $\Proj(\bb{S}(u))$ is a \emph{closed immersion} $\bb{P}(\sh{F})\to\bb{P}(\sh{E})$, which we denote by $\bb{P}(u)$.
We can thus say that $\bb{P}(\sh{E})$ is a \emph{contravariant functor} in $\sh{E}$, with the condition that we only consider \emph{surjective} morphisms of quasi-coherent $\sh{O}_Y$-modules.

Still supposing that $u$ is surjective, and letting $P=\bb{P}(\sh{E})$, $Q=\bb{P}(\sh{F})$, and $j=\bb{P}(u)$, we have, up to isomorphism, that
\[
\label{II.4.1.2.1}
  j^*(\sh{O}_P(n)) = \sh{O}_Q(n)
  \qquad\mbox{for all $n\in\bb{Z}$}
  \tag{4.1.2.1}
\]
by \sref{II.3.6.3}.
\end{env}

\begin{env}[4.1.3]
\label{II.4.1.3}
Now let $\psi:Y'\to Y$ be a morphism, and let $\sh{E}'=\psi^*(\sh{E})$;
then $\bb{S}_{\sh{O}_{Y'}}(\sh{E}') = \psi^*(\bb{S}_{\sh{O}_Y}(\sh{E}))$ \sref{II.1.7.5};
thus \sref{II.3.5.3}
\[
\label{II.4.1.3.1}
  \bb{P}(\psi^*(\sh{E})) = \bb{P}(\sh{E})\times_Y Y'
  \tag{4.1.3.1}
\]
up to canonical isomorphism;
furthermore, we clearly have that
\[
  \psi^*((\bb{S}_{\sh{O}_Y}(\sh{E}))(n)) = (\bb{S}_{\sh{O}_{Y'}}(\sh{E}'))(n)
\]
for all $n\in\bb{Z}$, whence, letting $P=\bb{P}(\sh{E})$ and $P'=\bb{P}(\sh{E}')$, we have \sref{II.3.5.4}, up to isomorphism, that
\[
\label{II.4.1.3.2}
  \sh{O}_{P'}(n) = \sh{O}_p(n)\otimes_Y\sh{O}_{Y'}
  \qquad\mbox{for all $n\in\bb{Z}$.}
  \tag{4.1.3.2}
\]
\end{env}

\oldpage[II]{72}
\begin{proposition}[4.1.4]
\label{II.4.1.4}
Let $\sh{L}$ be an invertible $\sh{O}_Y$-module.
For every quasi-coherent $\sh{O}_Y$-module $\sh{E}$, there exists a canonical $Y$-isomorphism $i_\sh{L}:\bb{P}(\sh{E})\xrightarrow{\sim}\bb{P}(\sh{E}\otimes\sh{L})$;
furthermore, if we let $P=\bb{P}(\sh{E})$ and $Q=\bb{P}(\sh{E}\otimes\sh{L})$, then $i_\sh{L}^*(\sh{O}_Q(n))$ is canonically isomorphic to $\sh{O}_P(n)\otimes_Y\sh{L}^{\otimes n}$ for all $n\in\bb{Z}$.
\end{proposition}

\begin{proof}
Note first of all that, if $A$ is a ring, $E$ an $A$-module, and $L$ a \emph{free monogenous} $A$-module, then we can canonically define a homomorphism of $A$-modules
\[
  \bb{S}_n(E\otimes L) \to \bb{S}_n(E)\otimes L^{\otimes n}
\]
by sending $(x_1\otimes y_1)\ldots(x_n\otimes y_n)$ to the element
\[
  (x_1x_2\ldots x_n)\otimes(y_1\otimes y_2\otimes\ldots\otimes y_n)
  \qquad\mbox{($x_i\in E$, $y_i\in L$, for $i\leq i\leq n$);}
\]
we can immediately see (by restricting to the case where $L=A$) that this homomorphism is in fact an isomorphism.
We thus obtain a canonical isomorphism of graded $A$-algebras $\bb{S}_A(E\otimes L)\xrightarrow{\sim}\bigoplus_{n\geq0}\bb{S}_n(E)\otimes L^{\otimes n}$.
By returning to the conditions of \sref{II.4.1.4}, the above remarks allow us to define a canonical isomorphism of graded $\sh{O}_Y$-algebras
\[
\label{II.4.1.4.1}
  \bb{S}_{\sh{O}_Y}(\sh{E}\otimes_{\sh{O}_Y}\sh{L}) \xrightarrow{\sim} \bigoplus_{n\geq0}\bb{S}_n(\sh{E})\otimes_{\sh{O}_Y}\sh{L}^{\otimes n}
  \tag{4.1.4.1}
\]
by defining this isomorphism as an isomorphism of presheaves, and taking into account \sref{II.1.7.4}, \sref[I]{I.1.3.9}, and \sref[I]{I.1.3.12}.
The proposition then follows from \sref{II.3.1.8}[(iii)] and \sref{II.3.2.10}.
\end{proof}

\begin{env}[4.1.5]
\label{II.4.1.5}
With the hypotheses of \sref{II.4.1.1}, let $P=\bb{P}(\sh{E})$, and denote by $p$ the structure morphism $P\to Y$.
Since, by definition, $\sh{E}=(\bb{S}_{\sh{O}_Y}(\sh{E}))_1$, we have a canonical homomorphism $\alpha_1:\sh{E}\to p_*(\sh{O}_P(1))$ \sref{II.3.3.2.2}, and thus \sref[0]{0.4.4.3} also a canonical homomorphism
\[
\label{II.4.1.5.1}
  \alpha_1^\sharp: p^*(\sh{E}) \to \sh{O}_P(1).
  \tag{4.1.5.1}
\]
\end{env}

\begin{proposition}[4.1.6]
\label{II.4.1.6}
The canonical homomorphism \sref{II.4.1.5.1} is surjective.
\end{proposition}

\begin{proof}
We have seen, in \sref{II.3.3.2}, that $\alpha_1^\sharp$ corresponds functorially to the canonical homomorphism $\sh{E}\otimes_{\sh{O}_Y}\bb{S}_{\sh{O}_Y}(\sh{E}) \to (\bb{S}_{\sh{O}_Y}(\sh{E}))(1)$;
since, by definition, $\sh{E}$ generates $\bb{S}_{\sh{O}_Y}(\sh{E})$, this homomorphism is surjective, whence the conclusion, by \sref{II.3.2.4}
\end{proof}


\subsection{Morphisms from a prescheme to a projective bundle}
\label{subsection:II.4.2}

\begin{env}[4.2.1]
\label{II.4.2.1}
Keeping the notation of \sref{II.4.1.5}, let $X$ be a $Y$-prescheme, $q:X\to Y$ the structure morphism, and let $r:X\to P$ be a $Y$-\emph{morphism} such that the following diagram commutes:
\[
  \xymatrix{
    P \ar[d]_p & X \ar[l]_r \ar[dl]^q
  \\Y
  }
\]
\oldpage[II]{73}

Since the functor $r^*$ is right exact \sref[0]{0.4.3.1}, we obtain, from the surjective homomorphism in \sref{II.4.1.5.1}, a surjective homomorphism
\[
  r^*(\alpha_1^\sharp): r^*(p^*(\sh{E})) \to r^*(\sh{O}_P(1)).
\]

But $r^*(p^*(\sh{E}))=q^*(\sh{E})$, and $r^*(\sh{O}_P(1))$ is locally isomorphic to $r^*(\sh{O}_P)=\sh{O}_X$, or, in other words, the latter is an \emph{invertible} sheaf $\sh{L}_r$ on $\sh{O}_X$, and so we have defined, given $r$, a canonical surjective $\sh{O}_X$-homomorphism
\[
\label{II.4.2.1.1}
  \varphi_r:q^*(\sh{E}) \to \sh{L}_r.
  \tag{4.2.1.1}
\]

When $Y=\Spec(A)$ is affine, and $\mathscr{E}=\widetilde{E}$, we can further clarify this homomorphism in the following way:
given $f\in E$, it follows from \sref{II.2.6.3} that
\[
\label{II.4.2.1.2}
  r^{-1}(D_+(f)) = X_{\varphi_r^\flat(f)}.
  \tag{4.2.1.2}
\]

Now let $V$ be an affine open subset of $X$ that is contained inside $r^{-1}(D_+(f))$, and let $B$ be its ring, which is an $A$-algebra;
let $S=\bb{S}_A(E)$;
the restriction of $r$ to $V$ corresponds to an $A$-homomorphism $\omega:\bb{S}_f\to B$, and we have that $q^*(\sh{E})|V = (E\otimes_A B)\supertilde$ and $\sh{L}_r|V = \widetilde{L_r}$, whence $L_r = (S(1))_{(f)}\otimes_{S_{(f)}}B_{[\omega]}$ \sref[I]{I.1.6.5}.
The restriction of $\varphi_r$ to $q^*(\sh{E})|V$ corresponds to the $B$-homomorphism $u:E\otimes_A B\to L_r$, which sends $x\otimes1$ to $(x/1)\otimes f = (f/1)\otimes\omega(x/f)$.
The canonical extension of $\varphi_r$ to a homomorphism of $\sh{O}_X$-algebras
\[
  \psi_r: q^*(\bb{S}(\sh{E})) = \bb{S}(q^*(\sh{E})) \to \bb{S}(\sh{L}_r) = \bigoplus_{n\geq0}\sh{L}_r^{\otimes n}
\]
is thus such that the restriction of $\psi_r$ to $q^*(\bb{S}_n(\sh{E}))|V$ corresponds to the homomorphism $\bb{S}_n(\sh{E}\otimes_A B) = \bb{S}_n(E)\otimes_A B \to L_r^{\otimes n}$ that sends $s\otimes1$ to $(f/1)^{\otimes n}\otimes\omega(s/f^n)$.
\end{env}

\begin{env}[4.2.2]
\label{II.4.2.2}
Conversely, suppose that we are given a morphism $q:X\to Y$, an invertible $\sh{O}_X$-module $\sh{L}$, and a quasi-coherent $\sh{O}_Y$-module $\sh{E}$;
to each homomorphism $\varphi:q^*(\sh{E})\to\sh{L}$ there canonically corresponding homomorphism of quasi-coherent $\sh{O}_X$-algebras
\[
  \psi: \bb{S}(q^*(\sh{E})) = q^*(\bb{S}(\sh{E})) \to \bigoplus_{n\geq0}\sh{L}^{\otimes n}
\]
and thus \sref{II.3.7.1} a $Y$-morphism $r_{\sh{L},\psi}:G(\psi)\to\Proj(\bb{S}(\sh{E}))=\bb{P}(\sh{E})$, which we denote by $r_{\sh{L},\varphi}$.
If $\varphi$ is \emph{surjective}, then so too is $\psi$, and thus \sref{II.3.7.4} $r_{\sh{L},\varphi}$ is \emph{everywhere defined}.
Furthermore, with the notation of \sref{II.4.2.1} and \sref{II.4.2.2}:
\end{env}

\begin{proposition}[4.2.3]
\label{II.4.2.3}
Given a morphism $q:X\to Y$ and a quasi-coherent $\sh{O}_Y$-module $\sh{E}$, maps $r\to(\sh{L}_r,\varphi_r)$ and $(\sh{L},\varphi)\to r_{\sh{L},\varphi}$ give a bijective correspondence between the set of $Y$-morphisms $r:X\to\bb{P}(\sh{E})$ and the set of equivalence classes of pairs $(\sh{L},\varphi)$ of an invertible $\sh{O}_X$-module $\sh{L}$ and a surjective homomorphism $\varphi:q^*(\sh{E})\to\sh{L}$, where such pairs $(\sh{L},\varphi)$ and $(\sh{L}',\varphi')$ are defined to be equivalent if there exists an $\sh{O}_X$-isomorphism $\tau:\sh{L}\xrightarrow{\sim}\sh{L}'$ such that $\varphi'=\tau\circ\varphi$.
\end{proposition}

\begin{proof}
Start first with a $Y$-morphism $r:X\to\bb{P}(\sh{E})$, and construct $\sh{L}_r$ and $\varphi_r$ \sref{II.4.2.1}, and let $r'=r_{\sh{L}_r,\varphi_r}$;
it follows immediately from \sref{II.4.2.1} and \sref{II.3.7.2} that the morphisms $r$ and $r'$ are identical (by taking the generator of $\sh{L}_r$ to be the element $(f/1)\otimes1$ to define the homomorphisms $v_n$ of \sref{II.3.7.2}).
Conversely, take a pair $(\sh{L},\varphi)$ and construct
\oldpage[II]{74}
$r''=r_{\sh{L},\varphi}$, and then $\sh{L}_{r''}$ and $\varphi_{r''}$;
we will show that there exists a canonical isomorphism $\tau:\sh{L}_{r''}\xrightarrow{\sim}\sh{L}$ such that $\varphi=\tau\circ\varphi_{r''}$;
to define it, we can restrict to the case where $Y=\Spec(A)$ and $X=\Spec(B)$ are affine, and (with the notation of \sref{II.4.2.1} and \sref{II.3.7.2}) associate to each element $(x/1)\otimes1$ of $L_{r''}$ (where $x\in E$) the element $v_1(x)c$ of $L$.
We immediately see that $\tau$ does not depend on the chosen generator $c$ of $L$;
since $v_1$ is surjective by hypothesis, to prove that $\tau$ is an isomorphism it suffices to to show that, if $x/1=0$ in $(S(1))_{(f)}$, then $v_1(x)/1=0$ in $B_g$;
but the first equality implies that $f^nx=0$ in $\bb{S}_{n+1}(E)$ for some $n$, and this implies that $v_{n+1}(f^nx) = g^nv_1(x) = 0$ in $B$, whence the conclusion.
Finally, it is immediate that, for any two equivalent pairs $(\sh{L},\varphi)$ and $(\sh{L}',\varphi')$, we have $r_{\sh{L},\varphi}=r_{\sh{L}',\varphi'}$.
\end{proof}

In particular, for $X=Y$:
\begin{theorem}[4.2.4]
\label{II.4.2.4}
The set of $Y$-sections of $\bb{P}(\sh{E})$ is in canonical bijective correspondence with the set of quasi-coherent sub-$\sh{O}_Y$-modules $\sh{F}$ of $\sh{E}$ such that $\sh{E}/\sh{F}$ is invertible.
\end{theorem}

We note that this property corresponds to the classical definition of ``projective space'' as the set of hyperplanes of a vector space (the classical case corresponding to $Y=\Spec(K)$, where $K$ is a field, and $\sh{E}=\widetilde{E}$, where $E$ is a finite-dimensional $K$-vector space; the $\sh{F}$ having the property described in \sref{II.4.2.4} then correspond to the hyperplanes of $E$, and we already know that the $Y$-sections of $\bb{P}(\sh{E})$ are then the \emph{$K$-rational points of $\bb{P}(\sh{E})$} \sref[I]{I.3.4.5}).

\begin{remark}[4.2.5]
\label{II.4.2.5}
Since there is a canonical bijective correspondence between $Y$-morphisms from $X$ to $P$ and their graph morphisms, $X$-sections of $P\times_Y X$ \sref[I]{I.3.3.14}, we see that, conversely, \sref{II.4.2.3} can be deduced from \sref{II.4.2.4}.
Denote by $\Hyp_Y(X,\sh{E})$ the set of quasi-coherent sub-$\sh{O}_X$-modules $\sh{F}$ of $\sh{E}\otimes_Y\sh{O}_X=q^*(\sh{E})$ such that $q^*(\sh{E})/\sh{F}$ is an invertible $\sh{O}_X$-module.
If $g:X'\to X$ is a $Y$-morphism, then it follows from the fact that $g^*$ is right exact that $g^*q^*(\sh{E})/\sh{F})=g^*q^*(\sh{E}))/g^*(\sh{F})$, and so the latter sheaf is invertible, and thus $\Hyp_Y(X,\sh{E})$ is a \emph{contravariant functor} into the category of $Y$-preschemes.
We can thus interpret the theorem \sref{II.4.2.4} as defining a \emph{canonical isomorphism} of functors $\Hom_Y(X,\bb{P}(\sh{E}))$ and $\Hyp_Y(X,\sh{E})$, where both functors are contravariant in the variable $X$ and map into the category of $Y$-preschemes.
This also gives a characterisation of the projective bundle $P=\bb{P}(\sh{E})$ by the following \emph{universal property}, which is much closer to the geometric intuition than the constructions from §§2--3:
for every morphism $q:X\to Y$ and every invertible $\sh{O}_X$-module $\sh{L}$ that is a quotient of $\sh{E}\otimes_{\sh{O}_Y}\sh{O}_X$, there exists a unique $Y$-morphism $r:X\to P$ such that $\sh{L}=r^*(\sh{O}_P(1))$.

We will see later that we can similarly define, amongst other things, ``Grassmannian'' schemes.
\end{remark}

\begin{corollary}[4.2.6]
\label{II.4.2.6}
Suppose that every invertible $\sh{O}_Y$-module is trivial \sref[I]{I.2.4.8}.
Let $V$ be the group $\Hom_{\sh{O}_Y}(\sh{E},\sh{O}_Y)$, considered as a module over the ring $A=\Gamma(Y,\sh{O}_Y)$, and let $V^*$ be the subset of $V$ consisting of surjective homomorphisms.
Then the set of $Y$-sections of $\bb{P}(\sh{E})$ is canonically identified with $V^*/A^*$, where $A^*$ is the group of units of $A$.
\end{corollary}

\oldpage[II]{75}
In particular:
\begin{enumerate}
  \item The corollary \sref{II.4.2.6} applies whenever $Y$ is a \emph{local scheme} \sref[I]{I.2.4.8}.
    Let $Y$ be an arbitrary prescheme, $y$ a point of $Y$, and $Y'=\Spec(\kres(y))$;
    then the fibre $p^{-1}(y)$ of $\bb{P}(\sh{E})$ can, by \sref{II.4.1.3.1}, be identified with $\bb{P}(\sh{E}^y)$, where $\sh{E}^y = \sh{E}_y\otimes_{\sh{O}_y}\kres(y) = \sh{E}_y/\mathfrak{m}_y\sh{E}_y$ is considered as a vector space over $\kres(y)$.
    More generally, if $K$ is an extension of $\kres(y)$, then $p^{-1}(y)\otimes_{\kres(y)}K$ can be identified with $\bb{P}(\sh{E}^y\otimes_{\kres(y)}K)$.
    The corollary \sref{II.4.2.6} then shows that the set of \emph{geometric points of $\bb{P}(\sh{E})$ with values in the extension $K$ of $\kres(y)$} \sref[I]{I.3.4.5}, which we can also call the \emph{rational geometric fibre over $K$ of $\bb{P}(\sh{E})$ over the point $y$}, can be identified with the \emph{projective space} associated to the \emph{dual} of the $K$-vector space $\sh{E}^y\otimes_{\kres(y)}K$.
  \item Suppose that $Y$ is affine of ring $A$, and, further, that every invertible $\sh{O}_Y$-module is trivial;
    further, take $\sh{E}=\sh{O}_Y^n$;
    then, in \sref{II.4.2.6}, $V$ can be identified with $A^n$ \sref[I]{I.1.3.8}, and $V^*$ with the sets of systems $(f_i)_{1\leq i\leq n}$ of elements of $A$ that generate the ideal $A$;
    any two such systems define the same $Y$-section of $\bb{P}_Y^{n-1}=\bb{P}_A^{n-1}$, or, in other words, \emph{the same point of $\bb{P}_A^{n-1}$ with values in $A$}, if and only if one of them can be obtained from the other by multiplication by an invertible element of $A$.
\end{enumerate}

These properties justify the terminology ``projective bundle'' for $\bb{P}(\sh{E})$.
We note that the definitions that we will similarly obtain for ``projective space'' is in fact \emph{dual} to the classical definition;
this is imposed upon us by the necessity of being able to define $\bb{P}(\sh{E})$ for \emph{arbitrary} quasi-coherent $\sh{O}_Y$-modules $\sh{E}$, and not just locally free ones.

\begin{remark}[4.2.7]
\label{II.4.2.7}
We will see, in Chapter~V, that, if $Y$ is connected and locally Noetherian, and if $\sh{E}$ is locally free, then, letting $P=\bb{P}(\sh{E})$, every invertible $\sh{O}_P$-module is isomorphic to an $\sh{O}_P$-module of the form $\sh{L}'\otimes_{\sh{O}_Y}\sh{O}_P(m)$, with $\sh{L}'$ some invertible $\sh{O}_Y$-module, well defined up to isomorphism, and $m$ some well defined integer.
In other words, $\HH^1(P,\sh{O}_P^*)$ is isomorphic to $\bb{Z}\times\HH^1(Y,\sh{O}_Y^*)$ \sref[0]{0.5.4.7}.
We will also see (\sref[III]{III.2.1.14}, taking \sref[0]{0.5.4.10} into account) that $p_*(\sh{L}^{\otimes m})=0$ if $m<0$, and $p_*(\sh{L}^{\otimes m})$ is isomorphic to $\sh{L}'\otimes_{\sh{O}_Y}(\bb{S}_{\sh{O}_Y}(\sh{E}))_m$ if $m\geq0$.
If $\sh{F}$ is a quasi-coherent $\sh{O}_Y$-module, then every $Y$-morphism $\bb{P}(\sh{E})\to\bb{P}(\sh{F})$ is determined by the data of an invertible $\sh{O}_Y$-module, an integer $m\geq0$, and an $\sh{O}_Y$-homomorphism $\psi:\sh{F}\to\sh{L}'\otimes_{\sh{O}_Y}(\bb{S}_{\sh{O}_Y}(\sh{E}))_m$ such that the corresponding homomorphism $\psi^\sharp$ of $\sh{O}_{\bb{P}(\sh{F})}$-modules is surjective.
We will also see that, if the $Y$-morphism in question is an isomorphism, then $m=1$ and $\sh{F}$ is isomorphic to $\sh{E}\otimes_{\sh{O}_Y}\sh{L}'$ (the converse of \sref{II.4.1.4}).
This will allow us to determine the sheaf of germs of automorphisms of $\bb{P}(\sh{E})$ as the quotient of the sheaf of groups $\shAut(\sh{E})$ (which is locally isomorphic to $\GL(n,\sh{O}_Y)$ is $\sh{E}$ is of rank $n$) by $\sh{O}_Y^*$.
\end{remark}

\begin{env}[4.2.8]
\label{II.4.2.8}
Keeping the notation of \sref{II.4.2.1}, let $u:X'\to X$ be a morphism;
if the $Y$-morphism $r:X\to P$ corresponds to the homomorphism $\varphi:q^*(\sh{E})\to\sh{L}$, then the $Y$-morphism $r\circ u$ corresponds to $u^*(\varphi):u^*(q^*(\sh{E}))\to u^*(\sh{L})$, as follows immediately from the definitions.
\end{env}

\begin{env}[4.2.9]
\label{II.4.2.9}
Let $\sh{E}$ and $\sh{F}$ be quasi-coherent $\sh{O}_Y$-modules, $v:\sh{E}\to\sh{F}$ a surjective homomorphism, and $j=\bb{P}(v)$ the corresponding closed immersion $\bb{P}(\sh{F})\to\bb{P}(\sh{E})$ \sref{II.4.1.2}.
If the $Y$-morphism $r:X\to\bb{P}(\sh{F})$ corresponds to the homomorphism $\varphi:q^*(\sh{F})\to\sh{L}$, then the
\oldpage[II]{76}
$Y$-morphism $j\circ r$ corresponds to $q^*(\sh{E})\xrightarrow{q^*(v)}q^*(\sh{F})\xrightarrow{\varphi}\sh{L}$;
this again follows from the definition given in \sref{II.4.2.1}.
\end{env}

\begin{env}[4.2.10]
\label{II.4.2.10}
Let $\psi:Y'\to Y$ be a morphism, and let $\sh{E}'=\psi^*(\sh{E})$.
If the $Y$-morphism $r:X\to P$ corresponds to the homomorphism $\varphi:q^*(\sh{E})\to\sh{L}$, then the $Y'$-morphism
\[
  r_{(Y')}: X_{(Y')} \to P' = \bb{P}(\sh{E}')
\]
corresponds to $\varphi_{(Y')}:q_{(Y')}^*(\sh{E}') = q^*(\sh{E})\otimes_Y\sh{O}_{Y'} \to \sh{L}\otimes_Y\sh{O}_{Y'}$.
Indeed, by \sref{II.4.1.3.1}, we have the commutative diagram
\[
  \xymatrix{
    Y' \ar[d]
    & P'=P_{(Y')} \ar[l]_{p_{(Y')}} \ar[d]^{u}
    & X_{(Y')} \ar[l]_{r_{(Y')}} \ar[d]^{v}
  \\Y
    & P \ar[l]_{p}
    & X \ar[l]_{r}
  }
\]

From \sref{II.4.1.3.1}, we have
\[
  (r_{(Y')})^*(\sh{O}_{P'}(1)) = (r_{(Y')})^*(u^*(\sh{O}_P(1))) = v^*(r^*(\sh{O}_P(1))) = v^*(\sh{L}) = \sh{L}\otimes_Y\sh{O}_{Y'};
\]
we also know that $u^*(\alpha_1^\sharp)$ is exactly the canonical homomorphism $\alpha_1^\sharp:(p_{(Y')})^*(\sh{E}')\to\sh{O}_{P'}(1)$;
we can see this by explicitly calculating the canonical homomorphisms $\alpha_1^\sharp$ to $P$ and $P'$ as in \sref{II.4.1.6}.
Whence our claim.
\end{env}


\subsection{The Segre morphism}
\label{subsection:II.4.3}

\begin{env}[4.3.1]
\label{II.4.3.1}
Let $Y$ be a prescheme, and $\sh{E}$ and $\sh{F}$ quasi-coherent $\sh{O}_Y$-modules;
let $P_1=\bb{P}(\sh{E})$ and $P_2=\bb{P}(\sh{F})$, and denote the structure morphisms by $p_1:P_1\to Y$ and $p_2:P_2\to Y$.
Let $Q=P_1\times_Y P_2$, and let $q_1:Q\to P_1$ and $q_2:Q\to P_2$ be the canonical projections;
then the $\sh{O}_Q$-module $\sh{L}=\sh{O}_{P_1}(1)\otimes_Y \sh{O}_{P_2}(1) = q_1^*(\sh{O}_{P_1}(1))\otimes_{\sh{O}_Q}q_2^*(\sh{O}_{P_2}(1))$ is invertible, since it is the tensor product of of two invertible $\sh{O}_Q$-modules \sref[0]{0.5.4.4}.
Also, if $r=p_1\circ q_1=p_2\circ q_2$ is the structure morphism $Q\to Y$, then $r^*(\sh{E}\otimes_{\sh{O}_Y}\sh{F}) = q_1^*(p_1^*(\sh{E}))\otimes_{\sh{O}_Q}q_2^*(p_2^*(\sh{F}))$ \sref[0]{0.4.3.3};
the canonical surjective homomorphisms \sref{II.4.1.5.1} $p_1^*(\sh{E})\to\sh{O}_{P_1}(1)$ and $p_2^*(\sh{F})\to\sh{O}_{P_2}(1)$ thus give, by taking the tensor product, a canonical homomorphism
\[
\label{II.4.3.1.1}
  s: r^*(\sh{E}\otimes_{\sh{O}_Y}\sh{F}) \to \sh{L}
  \tag{4.3.1.1}
\]
which is evidently surjective;
from this we obtain \sref{II.4.2.2} a canonical morphism, called the \emph{Segre morphism}:
\[
\label{II.4.3.1.2}
  \varsigma: \bb{P}(\sh{E})\times_Y\bb{P}(\sh{F}) \to \bb{P}(\sh{E}\otimes_{\sh{O}_Y}\sh{F}).
  \tag{4.3.1.2}
\]

We can study the morphism $\varsigma$ more explicitly in the case where $Y=\Spec(A)$ is affine, and $\sh{E}=\widetilde{E}$ and $\sh{F}=\widetilde{F}$, where $E$ and $F$ are $A$-modules, whence $\sh{E}\otimes_{\sh{O}_Y}\sh{F}=(E\otimes_A F)^\sim$ \sref[I]{I.1.3.12};
let $R=\bb{S}_A(E)$, $S=\bb{S}_A(F)$, and $T=\bb{S}_A(E\otimes_A F)$;
let $f\in E$ and $g\in F$, and consider the affine open
\[
  D_+(f) \times_Y D_+(g) = \Spec(B)
\]
\oldpage[II]{77}
of $Q$, where $B=R_{(f)}\otimes_A S_{(g)}$;
the restriction of $\sh{L}$ to this affine open is $\widetilde{L}$, where
\[
  L = (R(1))_{(f)} \otimes_A (S(1))_{(g)}
\]
and the element $c=(f/1)\otimes(g/1)$ is a generator of $L$ considered as a free $B$-module \sref{II.2.5.7}.
The homomorphism \sref{II.4.3.1.1} corresponds to the homomorphism
\[
  (x\otimes y)\otimes b \mapsto b((x/1)\otimes(y/1))
\]
from $(E\otimes_A F)\otimes_A B$ to $L$.
With the notation of \sref{II.3.7.2}, we thus have that $v_1(x\otimes y)=(x/f)\otimes(y/g)$;
the restriction of $\varsigma$ to $D_+(f)\times_Y D_+(g)$ is a morphism from this affine scheme to $D_+(f\otimes g)$, corresponding to the ring homomorphism $\omega:T_{(f\otimes g)}\to R_{(f)}\otimes_A S_{(g)}$ defined by
\[
\label{II.4.3.1.3}
  \omega((x\otimes y)/(f\otimes g)) = (x/f)\otimes(y/g)
  \tag{4.3.1.3}
\]
for $x\in E$ and $y\in F$.
\end{env}

\begin{env}[4.3.2]
\label{II.4.3.2}
It follows from \sref{II.4.2.3} that we have a canonical isomorphism
\[
\label{II.4.3.2.1}
  \tau: \varsigma^*(\sh{O}_P(1)) \xrightarrow{\sim} \sh{O}_{P_1}(1)\otimes_Y\sh{O}_{P_2}(1)
  \tag{4.3.2.1}
\]
where we let $P=\bb{P}(\sh{E}\otimes_{\sh{O}_Y}\sh{F})$.
We will show that, for $x\in\Gamma(Y,\sh{E})$ and $y\in\Gamma(Y,\sh{F})$, we have
\[
\label{II.4.3.2.2}
  \tau(\alpha_1(x\otimes y)) = \alpha_1(x)\otimes\alpha_1(y).
  \tag{4.3.2.2}
\]

Indeed, we can restrict to the case where $Y$ is affine, and we then have, with the notation of \sref{II.4.3.1} and \sref{II.2.6.2}, that $\alpha_1^{f\otimes g}(x\otimes y)=(x\otimes y)/1$ in $(T(1))_{(f\otimes g)}$, that $\alpha_1^f(x)=x/1$ in $(R(1))_{(f)}$, and that $\alpha_1^g(y)=y/1$ in $(S(1))_{(g)}$.
The definition of $\tau$ given in \sref{II.4.2.3} and the calculation of $v_1$ done in \sref{II.4.3.1} then immediately prove the claim \sref{II.4.3.2.2}.
From this we obtain the equation
\[
\label{II.4.3.2.3}
  \varsigma^{-1}(P_{x\otimes y}) = (P_1)_x\times_Y(P_2)_y
  \tag{4.3.2.3}
\]
with the notation of \sref{II.3.1.4}.
Indeed, taking \sref{II.3.3.3} into account, the equation \sref{II.4.3.2.2} (by restricting to the affine case, with the help of \sref[I]{I.3.2.7} and \sref[I]{I.3.2.3}) leaves us only to prove the following lemma:
\begin{lemma}[4.3.2.4]
\label{II.4.3.2.4}
Let $B$ and $B'$ be $A$-algebras, and let $Y=\Spec(A)$, $Z=\Spec(B)$, and $Z'=\Spec(B')$;
then $D(t\otimes t')=D(t)\times_Y D(t')$ for any $t\in B$, $t'\in B'$.
\end{lemma}
\begin{proof}
Indeed, if $p:Z\times_Y Z'\to Z$ and $p':Z\times_Y Z'\to Z'$ are the canonical projections, then it follows from \sref[I]{I.1.2.2.2} that $p^{-1}(D(t))=D(t\otimes1)$ and $p'^{-1}(D(t'))=D(1\otimes t')$;
the conclusion follows from \sref[I]{I.3.2.7} and \sref[I]{I.1.1.9.1}, since $(t\otimes1)(1\otimes t')=t\otimes t'$.
\end{proof}
\end{env}

\begin{proposition}[4.3.3]
\label{II.4.3.3}
The Segre morphism is a closed immersion.
\end{proposition}

\begin{proof}
Since the question is local on $Y$, we can restrict to the case where $Y$ is affine.
With the notation of \sref{II.4.3.1} and \sref{II.4.3.1}, the $D_+(f\otimes g)$ then form a basis for the topology of $P$, since the $f\otimes g$ generate $T$ when $f$ runs over $E$ and $g$ runs over $F$.
By \sref{II.4.3.2.3}, we also know that $\varsigma^{-1}(D_+(f\otimes g))=D_+(f)\times_Y D_+(g)$.
It thus suffices \sref[I]{I.4.2.4} to prove that the restriction of $\varsigma$ to $D_+(f)\times_Y D_+(g)$ is a closed immersion into $D_+(f\otimes g)$.
But, with the same notation, the equation \sref{II.4.3.1.3} shows that $\omega$ is \emph{surjective}, which completes the proof.
\end{proof}

\begin{env}[4.3.4]
\label{II.4.3.4}
The Segre morphism is \emph{functorial} in $\sh{E}$ and $\sh{F}$, if we consider only
\oldpage[II]{78}
\emph{surjective} homomorphisms of quasi-coherent $\sh{O}_Y$-modules.
Indeed, we must then show that, if $\sh{E}\to\sh{E}'$ is a surjective $\sh{O}_Y$-homomorphism, then the diagram
\[
  \xymatrix{
    \bb{P}(\sh{E}')\times\bb{P}(\sh{F}) \ar[r]^{j\times1} \ar[d]_{\varsigma}
    & \bb{P}(\sh{E})\times\bb{P}(\sh{F}) \ar[d]^{\varsigma}
  \\\bb{P}(\sh{E}'\otimes\sh{F}) \ar[r]
    &\bb{P}(\sh{E}\otimes\sh{F})
  }
\]
commutes, where $j$ denotes the canonical closed immersion $\bb{P}(\sh{E}')\to\bb{P}(\sh{E})$.
Let $P'_1=\bb{P}(\sh{E}')$ and keep the notation from \sref{II.4.3.1};
then $j\times1$ is a closed immersion \sref[I]{I.4.3.1} and, up to isomorphism,
\[
  (j\times1)^*(\sh{O}_{P_1}(1)\otimes\sh{O}_{P_2}(1))
  = j^*(\sh{O}_{P_1}(1))\otimes\sh{O}_{P_2}(1)
  = \sh{O}_{P'_1}(1)\otimes\sh{O}_{P_2}(1)
\]
by \sref{II.4.1.2.1} and \sref[I]{I.9.1.5};
our claim then follows from \sref{II.4.2.8} and \sref{II.4.2.9}.
\end{env}

\begin{env}[4.3.5]
\label{II.4.3.5}
With the notation of \sref{II.4.3.1}, let $\psi:Y'\to Y$ be a morphism, and let $\sh{E}'=\psi^*(\sh{E})$ and $\sh{F}'=\psi^*(\sh{F})$;
then the Segre morphism $\bb{P}(\sh{E}')\times\bb{P}(\sh{F}')\to\bb{P}(\sh{E}'\otimes\sh{F}')$ can be identified with $\varsigma_{(Y')}$.
Indeed, keeping the notation of \sref{II.4.3.1}, let $P'_1=\bb{P}(\sh{E}')$ and $P_2=\bb{P}(\sh{F}')$;
we know \sref{II.4.1.3.1} that $P'_i$ can be identified with $(P_i)_{(Y')}$ ($i=1,2$), and so the structure morphism $P'_1\times P'_2\to Y'$ can be identified with $r_{(Y')}$.
Also $\sh{E}'\otimes\sh{F}'$ can be identified with $\psi^*(\sh{E}\otimes\sh{F})$, and so $\bb{P}(\sh{E}'\otimes\sh{F}')$ can be identified with $(\bb{P}(\sh{E}\otimes\sh{F}))_{(Y')}$ \sref{II.4.1.3.1}.
Finally, $\sh{O}_{P'_1}(1)\otimes_Y\sh{O}_{P'_2}(1)=\sh{L}'$ can be identified with $\sh{L}\otimes_Y\sh{O}_{Y'}$, by \sref{II.4.1.3.1} and \sref[I]{I.9.1.11}.
The canonical homomorphism $(r_{(Y)})^*(\sh{E}'\otimes\sh{F}')\to\sh{L}'$ can then be identified with $s_{(Y')}$, and our claim follows from \sref{II.4.2.10}.
\end{env}

\begin{remark}[4.3.6]
\label{II.4.3.6}
The prescheme given by the \emph{sum} of $\bb{P}(\sh{E})$ and $\bb{P}(\sh{F})$ is even canonically isomorphic to a \emph{closed subprescheme of $\bb{P}(\sh{E}\oplus\sh{F})$}.
Indeed, the surjective homomorphisms $\sh{E}\oplus\sh{F}\to\sh{E}$ and $\sh{E}\oplus\sh{F}\to\sh{F}$ correspond to closed immersions $\bb{P}(\sh{E})\to\bb{P}(\sh{E}\oplus\sh{F})$ and $\bb{P}(\sh{F})\to\bb{P}(\sh{E}\oplus\sh{F})$;
everything then reduces to showing that the underlying spaces of the closed subpreschemes of $\bb{P}(\sh{E}\oplus\sh{F})$ obtained in this way have empty intersection.
Since the question is local on $Y$, we can adopt the notation of \sref{II.4.3.1};
but $\bb{S}_n(E)$ and $\bb{S}_n(F)$ can be identified with submodules of $\bb{S}_n(E\oplus F)$ with intersection consisting only of $0$;
if $\mathfrak{p}$ is a graded prime ideal of $\bb{S}(E)$ such that $\mathfrak{p}\cap\bb{S}_n(E)\neq\bb{S}_n(E)$ for any $n\geq0$, then there exists a corresponding graded prime ideal of $\bb{S}(E\oplus F)$ whose intersection with $\bb{S}_n(E)$ is $\mathfrak{p}\cap\bb{S}_n(E)$, but who also \emph{contains} $\bb{S}_+(F)$, as we immediately see;
thus no point in $\Proj(\bb{S}(E))$ can have the same image in $\Proj(\bb{S}(E\oplus F))$ as any point in $\Proj(\bb{S}(F))$.
\end{remark}


\subsection{Immersions into projective bundles; very ample sheaves}
\label{subsection:II.4.4}

\begin{proposition}[4.4.1]
\label{II.4.4.1}
Let $Y$ be a quasi-compact scheme, or a prescheme whose underlying space is Noetherian, $q:X\to Y$ a morphism \emph{of finite type}, and $\sh{L}$ an invertible $\sh{O}_X$-module.
\begin{enumerate}
  \item[\rm{(i)}] Let $\sh{S}$ be a positively-graded quasi-coherent $\sh{O}_Y$-algebra, and $\psi:q^*(\sh{S})\to\bigoplus_{n\geq0}\sh{L}^{\otimes n}$ a homomorphism of graded algebras.
    For $r_{\sh{L},\psi}$ to be everywhere defined and an immersion, it is necessary and
\oldpage[II]{79}
    sufficient for there to exist an integer $n\geq0$ and a quasi-coherent sub-$\sh{O}_Y$-module \emph{of finite type} $\sh{E}$ of $\sh{S}_n$ such that the homomorphism $\psi'=\psi_n\circ q^*(j):q^*(\sh{E})\to\sh{L}^{\otimes n}=\sh{L}'$ (where $j$ is the injection $\sh{E}\to\sh{S}_n$) is surjective and such that the morphism $r_{\sh{L}',\varphi'}:X\to\bb{P}(\sh{E})$ is an immersion.
  \item[\rm{(ii)}] Let $\sh{F}$ be a quasi-coherent $\sh{O}_Y$-module, and $\varphi:q^*(\sh{F})\to\sh{L}$ a surjective homomorphism.
    For the morphism $r_{\sh{L},\varphi}$ to be an immersion $X\to\bb{P}(\sh{F})$, it is necessary and sufficient for there to exist a quasi-coherent sub-$\sh{O}_Y$-module \emph{of finite type} $\sh{E}$ of $\sh{F}$ such that the homomorphism $\varphi'=\varphi\circ q(j):q^*(\sh{E})\to\sh{L}$ (where $j$ is the canonical injection $\sh{E}\to\sh{F}$) is surjective and such that the morphism $r_{\sh{L},\varphi'}:X\to\bb{P}(\sh{E})$ is an immersion.
\end{enumerate}
\end{proposition}

\begin{proof}
\medskip\noindent
\begin{enumerate}
  \item[\rm{(i)}] The fact that $r_{\sh{L},\varphi}$ is everywhere defined and is an immersion is equivalent, by \sref{II.3.8.5}, to the existence of some $n\geq0$ and $\sh{E}$ such that, if $\sh{S}'$ is the subalgebra of $\sh{S}$ generated by $\sh{E}$, the homomorphism $q^*(\sh{E})\to\sh{L}^{\otimes n}$ is surjective and the morphism $r_{\sh{L},\psi'}:X\to\Proj(\sh{S}')$ is everywhere defined and is an immersion.
    We already have a canonical surjective homomorphism $\bb{S}(\sh{E})\to\sh{S}'$ to which there exists a corresponding closed immersion $\Proj(\sh{S}')\to\bb{P}(\sh{E})$ \sref{II.3.6.2};
    whence the conclusion.
  \item[\rm{(ii)}] Since $\sh{F}$ is the inductive limit of its quasi-coherent submodules of finite type $\sh{E}_\lambda$ \sref[I]{I.9.4.9}, $\bb{S}(\sh{F})$ is the inductive limit of the $\bb{S}(\sh{E}_\lambda)$;
    the conclusion then follows from \sref{II.3.8.4}, by observing that we can take all the $n_i$ in the proof of \sref{II.3.8.4} to be equal to $1$:
    indeed, supposing that $Y$ is affine, if $r=r_{\sh{L},\varphi}$ is an immersion, then $r(X)$ is a quasi-compact subspace of $\bb{P}(\sh{F})$ that we can cover by finitely many open subsets of $\bb{P}(\sh{F})$ of the form $D_+(f)$, with $f\in F$, such that $D_+(f)\cap r(X)$ is closed.
\end{enumerate}
\end{proof}

\begin{definition}[4.4.2]
\label{II.4.4.2}
Let $Y$ be a prescheme, and $q:X\to Y$ a morphism.
We say that an invertible $\sh{O}_X$-module $\sh{L}$ is \emph{very ample for $q$}, or \emph{relative to $q$} (or \emph{very ample for} (or \emph{relative to}) \emph{$Y$}, or simply \emph{very ample}, if $q$ is clear from the context) if there exists a quasi-coherent $\sh{O}_Y$-module $\sh{E}$ and a $Y$-immersion $i$ from $X$ to $P=\bb{P}(\sh{E})$ such that $\sh{L}$ is isomorphic to $i^*(\sh{O}_P(1))$.
\end{definition}

It is equivalent \sref{II.4.2.3} to say that there exists a quasi-coherent $\sh{O}_Y$-module $\sh{E}$ and a \emph{surjective} homomorphism $\varphi:q^*(\sh{E})\to\sh{L}$ such that $r_{\sh{L},\varphi}:X\to\bb{P}(\sh{E})$ is an \emph{immersion}.

We note that the existence of a very ample (for $Y$) $\sh{O}_X$-module implies that $q$ is \emph{separated} (\sref{II.3.1.3} and \sref[I]{I.5.5.1}[(i) and (ii)]).

\begin{corollary}[4.4.3]
\label{II.4.4.3}
Suppose that there exists a graded quasi-coherent $\sh{O}_Y$-algebra $\sh{S}$, generated by $\sh{S}_1$, and a $Y$-immersion $i:X\to P=\Proj(\sh{S})$ such that $\sh{L}$ is isomorphic to $i^*(\sh{O}_P(1))$;
then $\sh{L}$ is very ample relative to $q$.
\end{corollary}

\begin{proof}
If $\sh{F}=\sh{S}_1$, then the canonical homomorphism $\bb{S}(\sh{F})\to\sh{S}$ is surjective, and so, by compositing with the corresponding closed immersion $\Proj(\sh{S})\to\bb{P}(\sh{F})$ \sref{II.3.6.2} and the immersion $i$, we obtain an immersion $j:X\to\bb{P}(\sh{F})=P'$ such that $\sh{L}$ is isomorphic to $j^*(\sh{O}_{P'}(1))$ \sref{II.3.6.3}.
\end{proof}

\begin{proposition}[4.4.4]
\label{II.4.4.4}
Let $q:X\to Y$ be a quasi-compact morphism, and $\sh{L}$ an invertible $\sh{O}_X$-module.
Then the following properties are equivalent:
\begin{enumerate}
  \item[\rm{(a)}] $\sh{L}$ is very ample relative to $q$.
  \item[\rm{(b)}] $q_*(\sh{L})$ is quasi-coherent, the canonical homomorphism $\sigma:q^*(q_*(\sh{L}))\to\sh{L}$ is surjective, and the morphism $r_{\sh{L},\sigma}:X\to\bb{P}(q_*(\sh{L}))$ is an immersion.
\end{enumerate}
\end{proposition}

\begin{proof}
Since $q$ is quasi-compact, we know that $q_*(\sh{L})$ is quasi-coherent if $q$ is separated \sref[I]{I.9.2.2}.

\oldpage[II]{80}
We know \sref{II.3.4.7} that the existence of a surjective homomorphism $\varphi:q^*(\sh{E})\to\sh{L}$ (with $\sh{E}$ a quasi-coherent $\sh{O}_Y$-module) implies that $\sigma$ is surjective;
furthermore, given the factorisation $q^*(\sh{E})\to q^*(q_*(\sh{L}))\xrightarrow{\sigma}\sh{L}$ of $\varphi$, there is a canonically corresponding factorisation
\[
  q^*(\bb{S}(\sh{E})) \to q^*(\bb{S}(q_*(\sh{L}))) \to \bigoplus_{n\geq0}\sh{L}^{\otimes n}
\]
and so \sref{II.3.8.3} the hypothesis that $r_{\sh{L},\varphi}$ is an immersion implies that so too is $j=r_{\sh{L},\sigma}$;
furthermore \sref{II.4.2.4}, $\sh{L}$ is isomorphic to $j^*(\sh{O}_{P'}(1))$, where $P'=\bb{P}(q_*(\sh{L}))$.
We thus see that (a) and (b) are equivalent.
\end{proof}

\begin{corollary}[4.4.5]
\label{II.4.4.5}
Suppose that $q$ is quasi-compact.
For $\sh{L}$ to be very ample relative to $Y$, it is necessary and sufficient for there to exist an open cover $(U_\alpha)$ of $Y$ such that $\sh{L}|q^{-1}(U_\alpha)$ is very ample relative to $U_\alpha$ for every $\alpha$.
\end{corollary}

\begin{proof}
Indeed, condition (b) of \sref{II.4.4.4} is local on $Y$.
\end{proof}

\begin{proposition}[4.4.6]
\label{II.4.4.6}
Let $Y$ be a quasi-compact scheme, or a prescheme whose underlying space is Noetherian, $q:X\to Y$ a morphism \emph{of finite type}, and $\sh{L}$ an invertible $\sh{O}_X$-module.
Then conditions (a) and (b) of \sref{II.4.4.4} are equivalent to the following:
\begin{enumerate}
  \item[\rm{(a')}] There exists a quasi-coherent $\sh{O}_Y$-module $\sh{E}$ \emph{of finite type} and a surjective homomorphism $\varphi:q^*(\sh{E})\to\sh{L}$ such that $r_{\sh{L},\varphi}$ is an immersion.
  \item[\rm{(b')}] There exists a coherent sub-$\sh{O}_Y$-module $\sh{E}$ of $q_*(\sh{L})$ \emph{of finite type} that has the properties stated in condition~(a').
\end{enumerate}
\end{proposition}

\begin{proof}
It is clear that (a') or (b') imply (a);
also (a) implies (a'), by \sref{II.4.4.1}, and similarly (b) implies (b').
\end{proof}

\begin{corollary}[4.4.7]
\label{II.4.4.7}
Suppose that $Y$ is a quasi-compact scheme, or a Noetherian prescheme.
If $\sh{L}$ is very ample for $q$, then there exists a graded quasi-coherent $\sh{O}_Y$-algebra $\sh{S}$ such that $\sh{S}_1$ is of finite type and generates $\sh{S}$, and also a \emph{dominant open} $Y$-immersion $i:X\to P=\Proj(\sh{S})$ such that $\sh{L}$ is isomorphic to $i^*(\sh{O}_P(1))$.
\end{corollary}

\begin{proof}
Indeed, condition~(b) of \sref{II.4.4.6} is satisfied;
the structure morphism $p:\bb{P}(\sh{E})=P'\to Y$ is then separated and of finite type \sref{II.3.1.3}, and so $P'$ is a quasi-compact scheme (resp. a Noetherian prescheme) if $Y$ is a quasi-compact scheme (resp. a Noetherian prescheme).
Let $Z$ be the closure \sref[I]{I.9.5.11} of the subprescheme $X'$ of $P'$ associated to the immersion $j=r_{\sh{L},\varphi}$ from $X$ into $P'$;
it is clear that $j$ factors as a dominant open immersion $i:X\to Z$ followed by the canonical injection $Z\to P'$.
But $Z$ can be identified with a prescheme $\Proj(\sh{S})$, where $\sh{S}$ is a graded $\sh{O}_Y$-algebra equal to the quotient of $\bb{S}(\sh{E})$ by a graded quasi-coherent sheaf of ideals \sref{II.3.6.2}, and it is clear that $\sh{S}_1$ is of finite type and generates $\sh{S}$;
furthermore, $\sh{O}_Z(1)$ is the inverse image of $\sh{O}_{P'}(1)$ by the canonical injection \sref{II.3.6.3}, and so $\sh{L}=i^*(\sh{O}_Z(1))$.
\end{proof}

\begin{proposition}[4.4.8]
\label{II.4.4.8}
Let $q:X\to Y$ be a morphism, $\sh{L}$ a very ample (relative to $q$) $\sh{O}_X$-module, and $\sh{L}'$ an invertible $\sh{O}_X$-module, such that there exists a quasi-coherent $\sh{O}_Y$-module $\sh{E}'$ and a surjective homomorphism $q^*(\sh{E}')\to\sh{L}'$.
Then $\sh{L}\otimes_{\sh{O}_X}\sh{L}'$ is very ample relative to $q$.
\end{proposition}

\begin{proof}
The hypothesis implies the existence of a $Y$-morphism $r':X\to P'=\bb{P}(\sh{E}')$ such that $\sh{L}'=r'^*(\sh{O}_{P'}(1))$ \sref{II.4.2.1}.
There is, by hypothesis, a quasi-coherent $\sh{O}_Y$-module $\sh{E}$ and a
\oldpage[II]{81}
$Y$-immersion $r:X\to P=\bb{P}(\sh{E})$ such that $\sh{L}=r^*(\sh{O}_P(1))$.
Let $Q=\bb{P}(\sh{E}\otimes\sh{E}')$, and consider the Segre morphism $\varsigma:P\times_Y P'\to Q$ \sref{II.4.3.1}.
Since $r$ is an immersion, so too is $(r,r')_Y:X\to P\times_Y P'$ \sref[I]{I.5.3.14};
but since $\varsigma$ is an immersion \sref{II.4.3.3}, so too is $r'':X\xrightarrow{(r,r')}P\times_Y P'\xrightarrow{\varsigma}Q$.
But also \sref{II.4.3.2.1} $\varsigma(\sh{O}_Q(1))$ is isomorphic to $\sh{O}_P(1)\otimes_Y\sh{O}_{P'}(1)$, and so \sref[I]{I.9.1.4} $r''^*(\sh{O}_Q(1))$ is isomorphic to $\sh{L}\otimes\sh{L}'$, which proves the proposition.
\end{proof}

\begin{corollary}[4.4.9]
\label{II.4.4.9}
Let $q:X\to Y$ be a morphism.
\begin{enumerate}
  \item Let $\sh{L}$ be an invertible $\sh{O}_X$-module, and $\sh{K}$ an invertible $\sh{O}_Y$-module.
    For $\sh{L}$ to be very ample relative to $q$, it is necessary and sufficient for $\sh{L}\otimes q^*(\sh{K})$ to be so.
  \item If $\sh{L}$ and $\sh{L}'$ are very ample (relative to $q$) $\sh{O}_X$-modules, then so too is $\sh{L}\otimes\sh{L}'$;
    in particular, $\sh{L}^{\otimes n}$ is very ample relative to $q$ for all $n>0$.
\end{enumerate}
\end{corollary}

\begin{proof}
Claim~(ii) is an immediate consequence of \sref{II.4.4.8}, as well as the necessity of condition~(i);
conversely, if $\sh{L}\otimes q^*(\sh{K})$ is very ample, then so too is $(\sh{L}\otimes q^*(\sh{K}))\otimes q^*(\sh{K}^{-1})$, by the above, and the latter $\sh{O}_X$-module is isomorphic to $\sh{L}$ (\sref[0]{0.5.4.3} and \sref[0]{0.5.4.5}).
\end{proof}

\begin{proposition}[4.4.10]
\label{II.4.4.10}
\medskip\noindent
\begin{enumerate}
  \item[\rm{(i)}] For every prescheme $Y$, every invertible $\sh{O}_Y$-module $\sh{L}$ is very ample relative to the identity morphism $1_Y$.
  \item[\rm{(i \emph{bis})}] Let $f:X\to Y$ be a morphism, and $j:X'\to X$ an immersion.
    If $\sh{L}$ is a very ample (relative to $f$) $\sh{O}_X$-module, then $j^*(\sh{L})$ is very ample relative to $f\circ j$.
  \item[\rm{(ii)}] Let $Z$ be a quasi-compact prescheme, $f:X\to Y$ a morphism of finite type, $g:Y\to Z$ a quasi-compact morphism, $\sh{L}$ a very ample (relative to $f$) $\sh{O}_X$-module, and $\sh{K}$ a very ample (relative to $g$) $\sh{O}_Y$-module.
    Then there exists some integer $n_0>0$ such that $\sh{L}\otimes f^*(\sh{K}^{\otimes n})$ is very ample relative to $g\circ f$ for all $n\geq n_0$.
  \item[\rm{(iii)}] Let $f:X\to Y$ and $g:Y'\to Y$ be morphisms, and let $X'=X_{(Y')}$.
    If $\sh{L}$ is a very ample (relative to $f$) $\sh{O}_X$-module, then $\sh{L}'=\sh{L}\otimes_Y\sh{O}_{Y'}$ is a very ample (relative to $f_{(Y')}$) $\sh{O}_X$-module.
  \item[\rm{(iv)}] Let $f_i:X_i\to Y_i$ ($i=1,2$) be $S$-morphism.
    If $\sh{L}_i$ is a very ample (relative to $f_i$) $\sh{O}_{X_i}$-module ($i=1,2$), then $\sh{L}_1\otimes_S\sh{L}_2$ is very ample relative to $f_1\times_S f_2$.
  \item[\rm{(v)}] Let $f:X\to Y$ and $g:Y\to Z$ be morphisms.
    If an $\sh{O}_X$-module $\sh{L}$ is very ample relative to $g\circ f$, then it is also very ample relative to $f$.
  \item[\rm{(vi)}] Let $f:X\to Y$ be a morphism, and $j$ the canonical injection $X_\red\to X$.
    If an $\sh{O}_X$-module $\sh{L}$ is very ample relative to $f$, then $j^*(\sh{L})$ is very ample relative to $f_\red$.
\end{enumerate}
\end{proposition}

\begin{proof}
Property~(i~\emph{bis}) follows immediately from the definition \sref{II.4.4.2}, and it is immediate that (vi) follows formally from (i~\emph{bis}) and (v), by an argument copied from the proof of \sref[I]{I.5.5.12}, which we leave to the reader.
To prove (v), we consider, as in \sref[I]{I.5.5.12}, the factorisation $X\xrightarrow{\Gamma_f}X\times_Z Y\xrightarrow{p_2}Y$, where $p_2=(g\circ f)\times1_Y$.
It follows from the hypothesis and from (i) and (iv) that $\sh{L}\otimes_{\sh{O}_Z}\sh{O}_Y$ is very ample for $p_2$;
but also $\sh{L}=\Gamma_f^*(\sh{L}\otimes_{\sh{O}_Z}\sh{O}_Y)$ \sref[I]{I.9.1.4}, and $\Gamma_f$ is an immersion \sref[I]{I.5.3.11};
we can thus apply (i~\emph{bis}).

\oldpage[II]{82}
To prove (i), we apply the definition \sref{II.4.4.2} with $\sh{E}=\sh{L}$, and note that then $\bb{P}(\sh{E})$ can be identified with $Y$ \sref{II.4.1.4}.

Now we prove (iii).
There exists a quasi-coherent $\sh{O}_Y$-module $\sh{E}$ and a $Y$-immersion $i:X\to\bb{P}(\sh{E})=P$ such that $\sh{L}=i^*(\sh{O}_P(1))$;
if we let $\sh{E}'=g^*(\sh{E})$, then $\sh{E}'$ is a quasi-coherent $\sh{O}_{Y'}$-module, and we have that $P'=\bb{P}(\sh{E}')=P_{(Y')}$ \sref{II.4.1.3.1}, that $i_{(Y')}$ is an immersion from $X_{(Y')}$ into $P'$ \sref[I]{I.4.3.2}, and that $\sh{L}'$ is isomorphic to $(i_{(Y')})^*(\sh{O}_{P'}(1))$ \sref{II.4.2.10}.

To prove (iv), note that there is, by hypothesis, a $Y_i$-immersion $r_i:X_i\to P_i=\bb{P}(\sh{E}_i)$, where $\sh{E}_i$ is a quasi-coherent $\sh{O}_{Y_i}$-module, and $\sh{L}_i=r_i^*(\sh{O}_{P_i}(1))$ ($i=1,2$);
$r_1\times_S r_2$ is an $S$-immersion of $X_1\times_S X_2$ into $P_1\times_S P_2$ \sref[I]{I.4.3.1}, and the inverse image of $\sh{O}_{P_1}(1)\otimes_S\sh{O}_{P_2}(1)$ under this immersion is $\sh{L}_1\otimes_S\sh{L}_2$.
Now let $T=Y_1\times_S Y_2$, and let $p_1$ and $p_2$ be the projections from $T$ to $Y_1$ and $Y_2$, respectively.
If we let $P'_i=\bb{P}(p_i^*(\sh{E}_i))$ ($i=1,2$), then $P'_i=P_i\times_{Y_i}T$, by \sref{II.4.1.3.1}, and so
\[
  P'_1\times_T P'_2
  = (P_1\times_{Y_1}T)\times_T(P_2\times_{Y_2}T)
  = P_1\times_{Y_1}(T\times_{Y_2}P_2)
  = P_1\times_{Y_1}(Y_1\times_S P_2)
  = P_1\times_S P_2
\]
up to canonical isomorphism.
Similarly, $\sh{O}_{P'_i}(1)=\sh{O}_{P_i}(1)\otimes_{Y_i}\sh{O}_T$ \sref{II.4.1.3.2}, and an analogous calculation (based in particular on \sref[I]{I.9.1.9.1} and \sref[I]{I.9.1.2}) shows that, in the above identification, $\sh{O}_{P'_1}(1)\otimes_T\sh{O}_{P'_2}(1)$ can be identified with $\sh{O}_{P_1}\otimes_S\sh{O}_{P_2}(1)$.
We can thus consider $r_1\times_S r_2$ as a $T$-immersion from $X_1\times_S X_2$ into $P'_1\times_T P'_2$, with the inverse image of $\sh{O}_{P'_1}(1)\otimes_T\sh{O}_{P'_2}(1)$ under this immersion being $\sh{L}_1\otimes_S\sh{L}_2$.
We then finish the argument as in \sref{II.4.4.8} by using the Segre morphism.

It remains only to prove (ii).
We can first of all restrict to the case where $Z$ is an affine scheme, since, in general, there exists a finite cover $(U_i)$ of $Z$ by affine opens;
if the proposition were proven for $\sh{K}|g^{-1}(U_i)$, $\sh{L}|f^{-1}(g^{-1}(U))$, and an integer $n_i$, then it would suffice to take $n_0$ to be the largest of the $n_i$ to prove the proposition for $\sh{K}$ and $\sh{L}$ \sref{II.4.4.5}.
The hypothesis implies that $f$ and $g$ are separated morphisms, and so $X$ and $Y$ are quasi-compact \emph{schemes}.

There is an immersion $r:X\to P=\bb{P}(\sh{E})$, where $\sh{E}$ is a quasi-coherent $\sh{O}_Y$-module \emph{of finite type}, and $\sh{L}=r^*(\sh{O}_P(1))$, by \sref{II.4.4.6}.
We will see that there exists a very ample (relative to the composed morphism $P\xrightarrow{h}Y\xrightarrow{g}Z$) $\sh{O}_P$-module $\sh{M}$ such that $\sh{O}_P(1)$ is isomorphic to $\sh{M}\otimes_Y\sh{K}^{\otimes(-m)}$ for some integer $m$.
For $n\geq m+1$, $\sh{O}_P(1)\otimes_Y\sh{K}^{\otimes n}$ will then be very ample for $Z$, by hypothesis and by (iv) applied to the morphisms $h:P\to Y$ and $1_Y$;
since $r$ is an immersion and $\sh{L}\otimes f^*(\sh{K}^{\otimes n}) = r^*(\sh{O}_P(1)\otimes_Y\sh{K}^{\otimes n})$, the conclusion will then follow from (i~\emph{bis}).
To prove our claim concerning $\sh{O}_P(1)$, we will use the following lemma:

  \begin{lemma}[4.4.10.1]
  \label{II.4.4.10.1}
  Let $Z$ be a quasi-compact scheme, or a prescheme whose underlying space is Noetherian, and let $g:Y\to Z$ be a quasi-compact morphism, $\sh{K}$ a very ample (with respect to $g$) invertible $\sh{O}_Y$-module, and $\sh{E}$ a quasi-coherent $\sh{O}_Y$-module of finite type.
  Then there exists an integer $m_0$ such that, for all $m\geq m_0$, $\sh{E}$ is isomorphic to a quotient of an $\sh{O}_Y$-module of the form $g^*(\sh{F})\otimes\sh{K}^{\otimes(-m)}$, where $\sh{F}$ is a quasi-coherent $\sh{O}_Z$-module of finite type (depending on $m$).
  \end{lemma}

This lemma will be proven in \sref{II.4.5.10.1};
the reader can verify that \sref{II.4.4.10} is not used anywhere in \sref{subsection:II.4.5}.

\oldpage[II]{83}
Assuming this lemma, there exists a closed immersion $j_1$ from $P$ to
\[
  P_1 = \bb{P}(g^*(\sh{F})\otimes\sh{K}^{\otimes(-m)})
\]
such that $\sh{O}_P(1)$ is isomorphic to $j_1^*(\sh{O}_{P_1}(1))$ \sref{II.4.1.2}.
Now, there exists an isomorphism from $P_1$ to $P_2=\bb{P}(g^*(\sh{F}))$, sending $\sh{O}_{P_2}(1)\otimes_Y\sh{K}^{\otimes(-m)}$ to $\sh{O}_{P_1}(1)$ \sref{II.4.1.4};
we thus have a closed immersion $j_2:P\to P_2$ such that $\sh{O}_P(1)$ is isomorphic to $j_2^*(\sh{O}_{P_2}(1))\otimes_Y\sh{K}^{\otimes(-m)}$.
Finally, $P_2$ can be identified with $P_3\times_Z Y$, where $P_3=\bb{P}(\sh{F})$, and $\sh{O}_{P_2}(1)$ with $\sh{O}_{P_3}(1)\otimes_Z\sh{O}_Y$ \sref{II.4.1.3}.
By definition, $\sh{O}_{P_3}(1)$ is very ample for $Z$;
since so too is $\sh{K}$, we conclude, from (iv), that $\sh{O}_{P_2}(1)\otimes_Y\sh{K}$ is very ample for $Z$;
so too is $\sh{M}=j_2^*(\sh{O}_{P_2}(1)\otimes_Y\sh{K})$ by (i~\emph{bis}), and $\sh{O}_P(1)$ is isomorphic to $\sh{M}\otimes_Y\sh{K}^{\otimes(-m-1)}$, which finishes the proof.
\end{proof}

\begin{proposition}[4.4.11]
\label{II.4.4.11}
Let $f:X\to Y$ and $f':X'\to Y$ be morphisms, $X''$ the sum prescheme $X\sqcup X'$, and $f''$ the morphism $X''\to Y$ that agrees with $f$ (resp. $f'$) on $X$ (resp. $X'$).
Let $\sh{L}$ (resp. $\sh{L}'$) be an invertible $\sh{O}_X$-module (resp. invertible $\sh{O}_{X'}$-module), and let $\sh{L}''$ be the invertible $\sh{O}_{X''}$-module that agrees with $\sh{L}$ (resp. $\sh{L}'$) on $X$ (resp. $X'$).
For $\sh{L}''$ to be very ample relative to $f''$, it is necessary and sufficient for $\sh{L}$ to be very ample relative to $f$ and for $\sh{L}'$ to be very ample relative to $f'$.
\end{proposition}

\begin{proof}
We can immediately restrict to the case where $Y$ is affine.
If $\sh{L}''$ is very ample then so too are $\sh{L}$ and $\sh{L}'$, by \sref{II.4.4.10}[(i~\emph{bis})].
Conversely, if $\sh{L}$ and $\sh{L}'$ are very ample, then it follows immediately from the definition \sref{II.4.4.2} and from \sref{II.4.3.6} that $\sh{L}''$ is very ample.
\end{proof}


\subsection{Ample sheaves}
\label{subsection:II.4.5}

\begin{env}[4.5.1]
\label{II.4.5.1}
Given a prescheme $X$ and an invertible $\sh{O}_X$-module $\sh{L}$, we define, for every $\sh{O}_X$-module $\sh{F}$ (when there will be no confusion possible over $\sh{L}$) $\sh{F}(n)=\sh{F}\otimes_{\sh{O}_X}\sh{L}^{\otimes n}$ ($n\in\bb{Z}$);
we also define $S=\bigoplus_{n\geq0}\Gamma(X,\sh{L}^{\otimes n})$ (a graded subring of the ring $\Gamma_\bullet(\sh{L})$ defined in \sref[0]{0.5.4.6}).
If we consider $X$ as a $\bb{Z}$-prescheme, and we denote by $p$ the structure morphism $X\to\Spec(\bb{Z})$, then there is a bijective correspondence between homomorphisms $p^*(\widetilde{S})\to\bigoplus_{n\geq0}\sh{L}^{\otimes n}$ of graded $\sh{O}_X$-algebras and endomorphisms of the graded ring $S$ \sref[I]{I.2.2.5};
the homomorphism $\varepsilon:p^*(\widetilde{S})\to\bigoplus_{n\geq0}\sh{L}^{\otimes n}$ that then corresponds to the \emph{identity} automorphism of $S$ is said to be \emph{canonical}.
There is a corresponding \sref{II.3.7.1} morphism $G(\varepsilon)\to\Proj(S)$ that is also said to be \emph{canonical}.
\end{env}

\begin{theorem}[4.5.2]
\label{II.4.5.2}
Let $X$ be a quasi-compact scheme or a prescheme whose underlying space is Noetherian, $\sh{L}$ an invertible $\sh{O}_X$-module, and $S$ the graded ring $\bigoplus_{n\geq0}\Gamma(X,\sh{L}^{\otimes n})$.
Then the following conditions are equivalent:
\begin{enumerate}
  \item[\rm{(a)}] When $f$ runs over the set of homogeneous elements of $S_+$, the $X_f$ form a base of the topology of $X$.
  \item[\rm{(a')}] When $f$ runs over the set of homogeneous elements of $S_+$, the $X_f$ that are affine form a cover of $X$.
  \item[\rm{(b)}] The canonical morphism $G(\varepsilon)\to\Proj(S)$ \sref{II.4.5.1} is everywhere defined and is a dominant open immersion.
\oldpage[II]{84}
  \item[\rm{(b')}] The canonical morphism $G(\varepsilon)\to\Proj(S)$ is everywhere defined and is a homeomorphism from the underlying space of $X$ to a subspace of $\Proj(S)$.
  \item[\rm{(c)}] For every quasi-coherent $\sh{O}_X$-module $\sh{F}$, if we denote by $\sh{F}_n$ the sub-$\sh{O}_X$-module of $\sh{F}(n)$ generated by the sections of $\sh{F}(n)$ over $X$, then $\sh{F}$ is the sum of the sub-$\sh{O}_X$-modules $\sh{F}_n(-n)$ over the integers $n>0$.
  \item[\rm{(c')}] Property~\rm{(c)} holds for every quasi-coherent sheaf of ideals of $\sh{O}_X$.
\end{enumerate}

Furthermore, if $(f_\alpha)$ is a family of homogeneous elements of $S_+$ such that the $X_{f_\alpha}$ are affine, then the restriction to $\bigcup_\alpha X_{f_\alpha}$ of the canonical morphism $X\to\Proj(S)$ is an isomorphism from $\bigcup_\alpha X_{f_\alpha}$ to $\bigcup_\alpha(\Proj(S))_{f_\alpha}$.
\end{theorem}

\begin{proof}
  It is clear that (b) implies (b'), and (b') implies (a) by \sref{II.3.7.3.1} (taking into account the fact that $\varepsilon^\flat$ is the identity).
  Condition~(a) implies (a'), since every $x\in X$ has an affine neighbourhood $U$ such that $\sh{L}|U$ is isomorphic to $\sh{O}_X|U$;
  if $f\in\Gamma(X,\sh{L}^{\otimes n})$ is such that $x\in X_f\subset U$, then $X_f$ is also the set of $x'\in U$ such that $(f|U)(x')\neq0$, and it is thus an affine open subset \sref[I]{I.1.3.6}.
  To prove that (a') implies (b), it suffices to prove the last claim of the theorem, and to further prove that, if $X=\bigcup_\alpha X_{f_\alpha}$, then condition~(iv) of \sref{II.3.8.2} is satisfied.
  This latter point follows immediately from \sref[I]{I.9.3.1}[(i)].
  As for the last claim of \sref{II.4.5.2}, since $X_{f_\alpha}$ is the inverse image of $(\Proj(S))_{f_\alpha}$ under $G(\varepsilon)\to\Proj(S)$, it suffices to apply \sref[I]{I.9.3.2}.
  Thus (a), (a'), (b), and (b') are all equivalent.

  To show that (a') implies (c), note that, if $X_f$ is affine (with $f\in S_k$), then $\sh{F}|X_f$ is generated by its sections over $X_f$ \sref[I]{I.1.3.9};
  on the other hand \sref[I]{I.9.3.1}[(ii)], such a section $s$ is of the form $(t|X_f)\otimes(f|X_f)^{-m}$, where $t\in\Gamma(X,\sh{F}(km))$;
  by definition, $t$ is also a section of $\sh{F}_{km}$, so $s$ is indeed a section of $\sh{F}_{km}(-km)$ over $X_f$, which proves (c).
  It is clear that (c) implies (c'), so it remains only to show that (c') implies (a).
  But let $U$ be an open neighbourhood of $x\in X$, and let $\sh{I}$ be a quasi-coherent sheaf of ideals of $\sh{O}_X$ defining a closed subprescheme of $X$ that has $X\setmin U$ as its underlying space \sref[I]{I.5.2.1}.
  Hypothesis~(c') implies that there exists an integer $n>0$ and a section $f$ of $\sh{I}(n)$ over $X$ such that $f(x)\neq0$.
  But we clearly have $f\in S_n$, and $x\in X_f\subset U$, which proves (a).
\end{proof}

When $X$ is a prescheme whose underlying space is Noetherian, the equivalent conditions of \sref{II.4.5.2} imply that $X$ is a \emph{scheme}, since it is isomorphic to a subprescheme of the scheme $S=\Proj(A)$, by \sref{II.4.5.2}[(b)].

\begin{definition}[4.5.3]
\label{II.4.5.3}
We say that an invertible $\sh{O}_X$-module $\sh{L}$ is \emph{ample} if $X$ is a quasi-compact scheme and if the equivalent conditions of \sref{II.4.5.2} are satisfied.
\end{definition}

It evidently follows from criterion~(a) of \sref{II.4.5.2} that, if $\sh{L}$ is an ample $\sh{O}_X$-module, then, for every open subset $U$ of $X$, $\sh{L}|U$ is an ample $(\sh{O}_X|U)$-module.

It follows from the proof of \sref{II.4.5.2} that the \emph{affine} $X_f$ form a base of the topology of $X$.
Furthermore:

\begin{corollary}[4.5.4]
\label{II.4.5.4}
Let $\sh{L}$ be an ample $\sh{O}_X$-module.
For every finite subspace $Z$ of $X$ and every neighbourhood $U$ of $Z$, there exists an integer $n$ and some $f\in\Gamma(X,\sh{L}^{\otimes n})$ such that $X_f$ is an affine neighbourhood of $Z$ contained in $U$.
\end{corollary}

\begin{proof}
\oldpage[II]{85}
By \sref{II.4.5.2}[(b)], it suffices to prove that, for every finite subset $Z'$ of $\Proj(S)$ and every open neighbourhood $U$ of $Z'$, there exists a homogeneous element $f\in S_+$ such that $Z\subset(\Proj(S))_f\subset U$ \sref{II.2.4.1}.
But, by definition, the closed set $Y$, complement of $U$ in $\Proj(S)$, is of the form $V_+(\mathfrak{I})$, where $\mathfrak{I}$ is a graded ideal of $S$ that does not contain $S_+$ \sref{II.2.3.2};
also, the points of $Z'$ are, by definition, graded ideals $\mathfrak{p}_i$ of $S_+$ that do not contain $\sh{I}$ \sref{II.2.3.1}.
There thus exists an element $f\in\mathfrak{I}$ that does not belong to any of the $\mathfrak{p}_i$ (Bourbaki, \emph{Alg. comm.}, chap.~II, \S1, no.~1, prop.~2), and, since the $\mathfrak{p}_i$ are graded, the argument made \emph{loc. cit.} shows that we can even take $f$ to be homogeneous;
this element then satisfies the claim.
\end{proof}

\begin{proposition}[4.5.5]
\label{II.4.5.5}
Suppose that $X$ is a quasi-compact scheme or a prescheme whose underlying space is Noetherian.
Then conditions~(a) to (c') of \sref{II.4.5.2} are equivalent to the following:
\begin{enumerate}
  \item[\rm{(d)}] For every quasi-coherent $\sh{O}_X$-module $\sh{F}$ of finite type, there exists an integer $n_0$ such that, for all $n\geq n_0$, $\sh{F}(n)$ is generated by its sections over $X$.
  \item[\rm{(d')}] For every quasi-coherent $\sh{O}_X$-module $\sh{F}$ of finite type, there exist integers $n>0$ and $k>0$ such that $\sh{F}$ is isomorphic to a quotient of the $\sh{O}_X$-module $\sh{L}^{\otimes(-n)}\otimes\sh{O}_X^k$.
  \item[\rm{(d'')}] Property~(d') holds for every quasi-coherent sheaf of ideals of $\sh{O}_X$ of finite type.
\end{enumerate}
\end{proposition}

\begin{proof}
Since $X$ is quasi-compact, if a quasi-coherent $\sh{O}_X$-module $\sh{F}$ of finite type is such that $\sh{F}(n)$ (which is of finite type) is generated by its sections over $X$, then $\sh{F}(n)$ is generated by a \emph{finite} number of these sections \sref[0]{0.5.2.3}, and so (d) implies (d'), and it is clear that (d') implies (d'').
Since every quasi-coherent $\sh{O}_X$-module $\sh{G}$ is the inductive limits of its sub-$\sh{O}_X$-modules of finite type \sref[I]{I.9.4.9}, to satisfy condition~(c') of \sref{II.4.5.2}, it suffices to do so for a quasi-coherent sheaf of ideals of $\sh{O}_X$ that is of finite type, and (d'') thus implies (c').
It remains only to show that, if $\sh{L}$ is ample, then property~(d) is satisfied.
Consider a finite cover of $X$ by $X_{f_i}$ ($f_i\in S_{n_i}$), that we can assume to be affine;
by replacing the $f_i$ with suitable powers (which does not alter the $X_{f_i}$), we can assume that all the $n_i$ are equal to one single integer $m$.
The sheaf $\sh{F}|X_{f_i}$, being of finite type, by hypothesis, is generated by a finite number of its sections $h_{ij}$ over $X_{f_i}$ \sref[I]{I.1.3.13};
so there exists an integer $k_0$ such that the section $h_{ij}\otimes f_i^{\otimes k_0}$ extends to a section of $\sh{F}(k_0m)$ over $X$ for every pair $(i,j)$ \sref[I]{I.9.3.1}.
\emph{A fortiori}, the $h_{ij}\otimes f_i^{\otimes k_0}$ extend to sections of $\sh{F}(km)$ over $X$ for every $k\geq k_0$, and, for these values of $k$, $\sh{F}(km)$ is thus generated by its sections over $X$.
For every $p$ such that $0<p<m$, $\sh{F}(p)$ is also of finite type, and so there exists an integer $k_p$ such that $\sh{F}(p)(km)=\sh{F}(p+km)$ is generated by its sections over $X$ for all $k\geq k_p$.
Taking $n_0$ to be the largest of the $k_pm$, we thus conclude that $\sh{F}(n)$ is generated by its sections over $X$ for all $n\geq n_0$, since such an $n$ is of the form $n=km+p$, with $k\geq k_p$ and $0\leq p<m$.
\end{proof}

\begin{proposition}[4.5.6]
\label{II.4.5.6}
  Let $X$ be a quasi-compact scheme, and $\sh{L}$ an invertible $\sh{O}_X$-module.
  \begin{enumerate}
    \item[\rm{(i)}] Let $n>0$ be an integer. For $\sh{L}$ to be ample, it is necessary and sufficient for $\sh{L}^{\otimes n}$ to be ample.
    \item[\rm{(ii)}] Let $\sh{L}'$ be an invertible $\sh{O}_X$-module such that, for all $x\in X$, there exists an integer $n>0$
\oldpage[II]{86}
      and a section $s'$ of $\sh{L}'^{\otimes n}$ over $X$ such that $s'(x)\neq0$.
      Then, if $\sh{L}$ is ample, so too is $\sh{L}\otimes\sh{L}'$.
  \end{enumerate}
\end{proposition}

\begin{proof}
  Property~(i) is an evident consequence of criterion~(a) of \sref{II.4.5.2}, since $X_{f^{\otimes n}}=X_f$.
  On the other hand, if $\sh{L}$ is ample, then, for every $x\in X$ and every neighbourhood $U$ of $x$, there exists some $m>0$ and $f\in\Gamma(X,\sh{L}^{\otimes m})$ such that $x\in X_f\subset U$ \sref{II.4.5.2}[(a)];
  if $f'\in\Gamma(X,\sh{L}'^{\otimes n})$ is such that $f'(x)\neq0$, then $s(x)\neq0$ for $s=f^{\otimes n}\otimes f'^{\otimes m}\in\Gamma(X,(\sh{L}\otimes\sh{L}')^{\otimes mn})$, and so $x\in X_s\subset X_f\subset U$, which proves that $\sh{L}\otimes\sh{L}'$ is ample \sref{II.4.5.2}[(a)].
\end{proof}

\begin{corollary}[4.5.7]
\label{II.4.5.7}
The tensor product of two ample $\sh{O}_X$-modules is ample.
\end{corollary}

\begin{corollary}[4.5.8]
\label{II.4.5.8}
Let $\sh{L}$ be an ample $\sh{O}_X$-module, and $\sh{L}'$ an invertible $\sh{O}_X$-module;
then there exists an integer $n_0>0$ such that $\sh{L}^{\otimes n}\otimes\sh{L}'$ is ample and generated by its sections over $X$ for $n\geq n_0$.
\end{corollary}

\begin{proof}
It follows from \sref{II.4.5.5} that there exists an integer $m_0$ such that $\sh{L}^{\otimes m}\otimes\sh{L}'$ is generated by its sections over $X$ for all $m\geq m_0$;
by \sref{II.4.5.6}, we can then take $n_0=m_0+1$.
\end{proof}

\begin{remark}[4.5.9]
\label{II.4.5.9}
Let $P=\HH^1(X,\sh{O}_X^\times)$ be the group of classes of invertible $\sh{O}_X$-modules \sref[0]{0.5.4.7}, and let $P^+$ be the subset of $P$ consisting of classes of ample sheaves.
Suppose that $P^+$ is \emph{non-empty}.
Then it follows from \sref{II.4.5.7} and \sref{II.4.5.8} that
\[
  P^+ + P^+ \subset P^+
  \quad\text{and}\quad
  P^+ - P^+ = P
\]
or, in other words, $P^+\cup\{0\}$ is the set of \emph{positive} elements in $P$ for a \emph{preorder} structure on $P$ that is compatible with its group structure, and is even \emph{archimedian}, by \sref{II.4.5.8}.
This is why we sometimes say ``positive sheaf'' instead of ample sheaf, and ``negative sheaf'' for the inverse of an ample sheaf (but we will not use this terminology).
\end{remark}

\begin{proposition}[4.5.10]
\label{II.4.5.10}
Let $Y$ be an affine scheme, $q:X\to Y$ a quasi-compact separated morphism, and $\sh{L}$ an invertible $\sh{O}_X$-module.
\begin{enumerate}
  \item[\rm{(i)}] If $\sh{L}$ is very ample for $q$, then $\sh{L}$ is ample.
  \item[\rm{(ii)}] Suppose further that the morphism $q$ is \emph{of finite type}.
    Then, for $\sh{L}$ to be ample, it is necessary and sufficient for it to posses one of the following properties:
    \begin{enumerate}
      \item[\rm{(e)}] There exists $n_0>0$ such that, for every integer $n\geq n_0$, $\sh{L}^{\otimes n}$ is very ample for $q$.
      \item[\rm{(e')}] There exists $n>0$ such that $\sh{L}^{\otimes n}$ is very ample for $q$.
    \end{enumerate}
\end{enumerate}
\end{proposition}

\begin{proof}
The first claim follows from the definition \sref{II.4.4.2} of a very ample $\sh{O}_X$-module: if $A$ is the ring of $Y$, then there exists an $A$-module $E$ and a surjective homomorphism
\[
  \psi: q^*((\bb{S}(E))^{\supertilde}) \to \bigoplus_{n\geq0}\sh{L}^{\otimes n}
\]
such that $i=r_{\sh{L},\psi}$ is an everywhere-defined immersion $X\to P=\bb{P}(\widetilde{E})$ and such that $\sh{L}=i^*(\sh{O}_P(1))$;
since the $D_+(f)$ for $f$ homogeneous in $(\bb{S}(E))_+$ form a base for the topology of $P$, and since $i^{-1}(D_+(f))=X_{\psi^\flat(f)}$, by \sref{II.3.7.3.1}, we see that condition~(a) of \sref{II.4.5.2} is satisfied, and so $\sh{L}$ is ample.

Now to prove that, if $q$ is of finite type and $\sh{L}$ is ample, then condition~(e) is satisfied.
Firstly, it follows from criterion~(b) of \sref{II.4.5.2} and from \sref{II.4.4.1}[(i)] that there exists
\oldpage[II]{87}
an integer $k_0$ such that $\sh{L}^{\otimes k_0}$ is very ample relative to $q$.
Also, by \sref{II.4.5.5}, there exists an integer $m_0$ such that, for all $m\geq m_0$, $\sh{L}^{\otimes m}$ is generated by its sections over $X$.
Let $n_0=k_0+m_0$;
if $n\geq n_0$, then we can write $n=k_0+m$ with $m\geq m_0$, whence $\sh{L}^{\otimes n}=\sh{L}^{\otimes k_0}\otimes\sh{L}^{\otimes m}$.
Since $\sh{L}^{\otimes m}$ is generated by its sections over $X$, it follows from \sref{II.4.4.8} and \sref{II.3.4.7} that $\sh{L}^{\otimes n}$ is very ample relative to $q$.
Finally, it is clear that (e) implies (e'), and (e') implies that $\sh{L}$ is ample by (i) and by \sref{II.4.5.6}[(i)]

  \begin{env}[4.5.10.1]
  \label{II.4.5.10.1}
  \emph{[Proof of Lemma~\sref{II.4.4.10.1}].}  
  Let
  \end{env}
\end{proof}


% \subsection{Relatively ample sheaves}
% \label{subsection:II.4.6}

\section{Quasi-affine morphisms; quasi-projective morphisms; proper morphisms; projective morphisms}
\label{section:quasi-affine-projective-proper-morphisms}

\subsection{Quasi-affine morphisms}
\label{subsection:quasi-affine-morphisms}

\begin{defn}[5.1.1]
\label{2.5.1.1}
We define a quasi-affine scheme to be a scheme isomorphic to some subscheme induced on some quasi-compact open subset of an affine scheme.
We say that a morphism $f:X\to Y$ is quasi-affine, or that $X$ (considered as a $Y$-prescheme via $f$) is a quasi-affine $Y$-scheme, if there exists a cover $(U_\alpha)$ of $Y$ by affine open subsets such that the $f^{-1}(U_\alpha)$ are quasi-affine schemes.
\end{defn}

It is clear that a quasi-affine morphism is \emph{separated} (\sref[I]{1.5.5.5} and \sref[I]{1.5.5.8}) and \emph{quasi-compact} \sref[I]{1.6.6.1};
every affine morphisms is evidently quasi-affine.

Recall that, for any prescheme $X$, setting $A=\Gamma(X,\OO_X)$, the identity homomorphism $A\to A=\Gamma(X,\OO_X)$ defines a morphism $X\to\Spec(A)$, said to be \emph{canonical} \sref[I]{1.2.2.4};
this is nothing but the canonical morphism defined in \sref{2.4.5.1} for the specific case where $\sh{L}=\OO_X$, if we remember that $\Proj(A[T])$ is canonically identified with $\Spec(A)$ \sref{2.3.1.7}.

\begin{prop}[5.1.2]
\label{2.5.1.2}
Let $X$ be a quasi-compact scheme or a prescheme whose underlying space is Noetherian, and $A$ the ring $\Gamma(X,\OO_X)$.
The following conditions are equivalent.
\begin{enumerate}[label=\emph{(\alph*)}]
  \item $X$ is a quasi-affine scheme.
  \item The canonical morphism $u:X\to\Spec(A)$ is an open immersion.
  \item[\emph{(b$'$)}] The canonical morphism $u:X\to\Spec(A)$ is a homeomorphism from $X$ to some subspace of the underlying space of $\Spec(A)$.
  \item The $\OO_X$-module $\OO_X$ is very ample relative to $u$ \sref{2.4.4.2}.
  \item[\emph{(c$'$)}] The $\OO_X$-module $\OO_X$ is ample \sref{2.4.5.1}.
  \item When $f$ ranges over $A$, the $X_f$ form a base for the topology of $X$.
  \item[\emph{(d$'$)}] When $f$ ranges over $A$, the $X_f$ that are affine form a cover of $X$.
\oldpage[II]{95}
  \item Every quasi-coherent $\OO_X$-module is generated by its sections over $X$.
  \item[\emph{(e$'$)}] Every quasi-coherent sheaf of ideals of $\OO_X$ of finite type is generated by its sections over $X$.
\end{enumerate}
\end{prop}

\begin{proof}
\label{proof-2.5.1.2}
It is clear that (b) implies (a), and (a) implies (c) by \sref{2.4.4.4}[b] applied to the identity morphism (taking into account the remark preceding this proposition);
Furthermore, (c) implies (c$'$) \sref{2.4.5.10}[i], and (c$'$), (b), and (b$'$) are all equivalent by \sref{2.4.5.2}[b] and \sref{2.4.5.2}[b$'$].
Finally, (c$'$) is the same as each of (d), (d$'$), (e), and (e$'$) by \sref{2.4.5.2}[a], \sref{2.4.5.2}[a$'$], \sref{2.4.5.2}[c], and \sref{2.4.5.5}[d$''$].
\end{proof}

We further observe that, with the previous notation, the $X_f$ that are affine form a \emph{basis} for the topology of $X$, and that the canonical morphism $u$ is \emph{dominant} \sref{2.4.5.2}.

\begin{cor}[5.1.3]
\label{1.5.1.3}
Let $X$ be a quasi-compact prescheme.
If there exists a morphism $v:X\to Y$ from $X$ to some affine scheme $Y$ (which would be a homeomorphism from $X$ to some open subspace of $Y$), then $X$ is quasi-affine.
\end{cor}

\begin{proof}
\label{proof-2.5.1.3}
There exists a family $(g_\alpha)$ of sections of $\OO_Y$ over $Y$ such that the $D(g_\alpha)$ form a base for the topology of $v(X)$;
if $v=(\psi,\theta)$ and we set $f_\alpha=\theta(g_\alpha)$, then we have $X_{f_\alpha}=\psi^{-1}(D(g_\alpha))$ \sref[I]{1.2.2.4.1}, so the $X_{f_\alpha}$ form a base for the topology of $X$, and the criterion \sref{2.5.1.2}[d] is satisfied.
\end{proof}

\begin{cor}[5.1.4]
\label{2.5.1.4}
If $X$ is a quasi-affine scheme, then \emph{every} invertible $\OO_X$-module is very ample (relative to the canonical morphism), and \emph{a fortiori} ample.
\end{cor}

\begin{proof}
\label{proof-2.5.1.4}
Such a module $\sh{L}$ is generated by its sections over $X$ \sref{2.5.1.2}[e], so $\sh{L}\otimes\OO_X=\sh{L}$ is very ample \sref{2.4.4.8}.
\end{proof}

\begin{cor}[5.1.5]
\label{2.5.1.5}
Let $X$ be a quasi-compact prescheme.
If there exists an invertible $\OO_X$-module $\sh{L}$ such that $\sh{L}$ and $\sh{L}^{-1}$ are ample, then $X$ is a quasi-affine scheme.
\end{cor}

\begin{proof}
\label{proof-2.5.1.5}
Indeed, $\OO_X=\sh{L}\otimes\sh{L}^{-1}$ is then ample \sref{2.4.5.7}.
\end{proof}

\begin{prop}[5.1.6]
\label{2.5.1.6}
Let $f:X\to Y$ be a quasi-compact morphism.
Then the following conditions are equivalent.
\begin{enumerate}[label=\emph{(\alph*)}]
  \item The morphism $f$ is quasi-affine.
  \item The $\OO_Y$-algebra $f_*(\OO_X)=\sh{A}(X)$ is quasi-coherent, and the canonical morphism $X\to\Spec(\sh{A}(X))$ corresponding to the identity morphism $\sh{A}(X)\to\sh{A}(X)$ \sref{2.1.2.7} is an open immersion.
  \item[\emph{(b$'$)}] The $\OO_Y$-algebra $\sh{A}(X)$ is quasi-coherent, and the canonical morphism $X\to\Spec(\sh{A}(X))$ is a homeomorphism from $X$ to some subspace of $\Spec(\sh{A}(X))$.
  \item The $\OO_X$-module $\OO_X$ is very ample for $f$.
  \item[\emph{(c$'$)}] The $\OO_X$-module $\OO_X$ is ample for $f$.
  \item The morphism $f$ is separated, and, for every quasi-coherent $\OO_X$-module $\sh{F}$, the canonical homomorphism $\sigma:f^*(f_*(\sh{F}))\to\sh{F}$ \sref[0]{0.4.4.3} is surjective.
\end{enumerate}

Furthermore, whenever $f$ is quasi-affine, every invertible $\OO_X$-module $\sh{L}$ is very ample relative to $f$.
\end{prop}

\begin{proof}
\label{proof-2.5.1.6}
The equivalence between (a) and (c$'$) follows from the local (on $Y$) character of the $f$-ampleness \sref{2.4.6.4}, Definition~\sref{2.5.1.1}, and the criterion \sref{2.5.1.2}[c$'$].
The other properties are local on $Y$
\oldpage[II]{96}
and thus follow immediately from \sref{2.5.1.2} and \sref{2.5.1.4}, taking into account the fact that $f_*(\sh{F})$ is quasi-coherent whenever $f$ is separated \sref[I]{1.9.2.2}[a].
\end{proof}

\begin{cor}[5.1.7]
\label{2.5.1.7}
Let $f:X\to Y$ be a quasi-affine morphism.
For every open subset $U$ of $Y$, the restriction $f^{-1}(U)\to U$ of $f$ is quasi-affine.
\end{cor}

\begin{cor}[5.1.8]
\label{2.5.1.8}
Let $Y$ be an affine scheme, and $f:X\to Y$ a quasi-compact morphism.
For $f$ to be quasi-affine, it is necessary and sufficient for $X$ to be a quasi-affine scheme.
\end{cor}

\begin{proof}
\label{proof-2.5.1.8}
This is an immediate consequence of \sref{2.5.1.6} and \sref{2.4.6.6}.
\end{proof}

\begin{cor}[5.1.9]
\label{2.5.1.9}
Let $Y$ be a quasi-compact scheme or a prescheme whose underlying space is Noetherian, and $f:X\to Y$ a morphism of \emph{finite type}.
If $f$ is quasi-affine, then there exists a quasi-coherent $\OO_Y$-subalgebra $\sh{B}$ of $\sh{A}(X)=f_*(\OO_X)$ of \emph{finite type} \sref[I]{1.9.6.2} such that the morphism $X\to\Spec(\sh{B})$ corresponding to the canonical injection $\sh{B}\to\sh{A}(X)$ is an immersion.
Further, every quasi-coherent $\OO_Y$-subalgebra $\sh{B}'$ of finite type over $\sh{A}(X)$ containing $\sh{B}$ has the same property.
\end{cor}

\begin{proof}
\label{proof-2.5.1.9}
Indeed, $\sh{A}(X)$ is the inductive limit of its quasi-coherent $\OO_Y$-subalgebras of finite type \sref[I]{1.9.6.5};
the result is then a particular case of \sref{2.3.8.4}, taking into account the identification of $\Spec(\sh{A}(X))$ with $\Proj(\sh{A}(X)[T])$ \sref{2.3.1.7}.
\end{proof}

\begin{prop}[5.1.10]
\label{2.5.1.10}
\medskip\noindent
\begin{enumerate}[label=\emph{(\roman*)}]
  \item A quasi-compact morphism $X\to Y$ that is a homeomorphism from the underlying space of $X$ to some subspace of the underlying space of $Y$ (so, in particular, any closed immersion) is quasi-affine.
  \item The composition of any two quasi-affine morphisms is quasi-affine.
  \item If $f:X\to Y$ is a quasi-affine $S$-morphism, then $f_{(S')}:X_{(S')}\to Y_{(S')}$ is a quasi-affine morphism for any extension $S'\to S$ of the base prescheme.
  \item If $f:X\to Y$ and $g:X'\to Y'$ are quasi-affine $S$-morphisms, then $f\times_S g$ is quasi-affine.
  \item If $f:X\to Y$ and $g:Y\to Z$ are morphisms such that $g\circ f$ is quasi-affine, and if $g$ is separated or the underlying space of $X$ is locally Noetherian, then $f$ is quasi-affine.
  \item If $f$ is a quasi-affine morphism then so is $f_\red$.
\end{enumerate}
\end{prop}

\begin{proof}
\label{proof-2.5.1.10}
Taking into account the criterion \sref{2.5.1.6}[c$'$], all of (i), (iii), (iv), (v), and (vi) follow immediately from \sref{2.4.6.13}[i \emph{bis}], \sref{2.4.6.13}[iii], \sref{2.4.6.13}[iv], \sref{2.4.6.13}[v], and \sref{2.4.6.13}[vi] (respectively).
To prove (ii), we can restrict to the case where $Z$ is affine, and then the claim follows directly from applying \sref{2.4.6.13}[ii] to $\sh{L}=\OO_X$ and $\sh{K}=\OO_Y$.
\end{proof}

\begin{rmk}[5.1.11]
Let $f:X\to Y$ and $g:Y\to Z$ be morphisms such that $X\times_Z Y$ is locally Noetherian.
Then the graph immersion $\Gamma_f:X\to X\times_Z Y$ is quasi-affine, since it is quasi-compact \sref[I]{1.6.3.5}, and since \sref[I]{1.5.5.12} shows that, in (v), the conclusion still holds true if we remove the hypothesis that $g$ is separated.
\end{rmk}

\begin{prop}[5.1.12]
\label{2.5.1.12}
Let $f:X\to Y$ be a quasi-compact morphism, and $g:X'\to X$ a quasi-affine morphism.
If $\sh{L}$ is an ample (for $f$) $\OO_X$-module, then $g^*(\sh{L})$ is an ample (for $f\circ g$) $\OO_{X'}$-module.
\end{prop}

\begin{proof}
\label{proof-2.5.1.12}
Since $\OO_{X'}$ is very ample for $g$, and the question is local on $Y$ \sref{2.4.6.4}, it follows from \sref{2.4.6.13}[ii] that there exists (for $Y$ affine) an integer $n$ such that
\[
  g^*(\sh{L}^{\otimes n})=(g^*(\sh{L}))^{\otimes n}
\]
is ample for $f\circ g$, and so $g^*(\sh{L})$ is ample for $f\circ g$ \sref{2.4.6.9}
\end{proof}

\subsection{Serre's criterion}
\label{subsection:serres-criterion}

\begin{thm}[5.2.1]
\label{2.5.2.1}
\emph{(Serre's criterion).}
Let $X$ be a quasi-compact scheme or a prescheme whose underlying space is Noetherian.
The following conditions are equivalent.
\begin{enumerate}[label=\emph{(\alph*)}]
  \item $X$ is an affine scheme.
  \item There exists a family of elements $f_\alpha\in A=\Gamma(X,\OO_X)$ such that the $X_{f_\alpha}$ are affine, and such that the ideal generated by the $f_\alpha$ in $A$ is equal to $A$ itself.
  \item The functor $\Gamma(X,\sh{F})$ is exact in $\sh{F}$ in the category of quasi-coherent $\OO_X$-modules, or, in other words, if
    \[
      0\to\sh{F}'\to\sh{F}\to\sh{F}''\to 0
      \tag{*}
    \]
    is an exact sequence of quasi-coherent $\OO_X$-modules, then the sequence
    \[
     0\to\Gamma(X,\sh{F}')\to\Gamma(X,\sh{F})\to\Gamma(X,\sh{F}'')\to 0
    \]
    is also exact.
  \item[\emph{(c$'$)}] Condition~\emph{(c)} holds for every exact sequence \emph{(*)} of quasi-coherent $\OO_X$-modules such that $\sh{F}$ is isomorphic to a $\OO_X$-submodule of $\OO_X^n$ for some finite $n$.
  \item $\HH^1(X,\sh{F})=0$ for every quasi-coherent $\OO_X$-module $\sh{F}$.
  \item[\emph{(d$'$)}] $\HH^1(X,\sh{J})=0$ for every quasi-coherent sheaf of ideals $\sh{J}$ of $\OO_X$.
\end{enumerate}
\end{thm}

\begin{proof}
\label{proof-2.5.2.1}
It is evident that (a) implies (b); furthermore, (b) implies that the $X_{f_\alpha}$ cover $X$, because, by hypothesis, the section $1$ is a linear combination of the $f_\alpha$, and the $D(f_\alpha)$ thus cover $\Spec(A)$.
The final claim of \sref{2.4.5.2} thus implies that $X\to\Spec(A)$ is an isomorphism.

We know tha (a) implies (c) \sref[I]{1.1.3.11}, and (c) trivially implies (c$'$).
We now prove that (c$'$) implies (b).
First of all, (c$'$) implies that, for every \emph{closed} point $x\in X$ and every open neighbourhood $U$ of $x$, there exists some $f\in A$ such that $x\in X_f\subset X\setmin U$.
Let $\sh{J}$ (resp. $\sh{J}'$) be the quasi-coherent sheaf of ideals of $\OO_X$ defining the reduced closed subprescheme of $X$ that has $X\setmin U$ (resp. $(X\setmin U)\cup\{x\}$) as its underlying space \sref[I]{1.5.2.1};
it is clear that we have $\sh{J}'\subset\sh{J}$, and that $\sh{J}''=\sh{J}/\sh{J}'$ is a quasi-coherent $\OO_X$ module that has support equal to $\{x\}$, and such that $\sh{J}_x''=\kres(x)$.
Hypothesis~(c$'$) applied to the exact sequence $0\to\sh{J}'\to\sh{J}\to\sh{J}''\to0$ shows that $\Gamma(X,\sh{J})\to\Gamma(X,\sh{J}'')$ is surjective.
The section of $\sh{J}''$ whose germ at $x$ is $1_x$ is thus the image of some section $f\in\Gamma(X,\sh{J})\subset\Gamma(X,\OO_X)$, and we have, by definition, that $f(x)=1_x$ and $f(y)=0$ in $X\setmin U$, which establishes our claim.
Now, if $U$ is affine, then so is $X_f$ \sref[I]{1.1.3.6}, so the union of the $X_f$ that are affine ($f\in A$) is an open set $Z$ that contains \emph{all the closed points} of $X$;
since $X$ is a quasi-compact Kolmogoroff space, we necessarily have $Z=X$ \sref[0]{0.2.1.3}.
Because $X$ is quasi-compact, there are a \emph{finite} number of elements $f_i\in A$ ($1\leq i\leq n$) such that the $X_{f_i}$ are affine and cover $X$.
So consider the homomorphism $\OO_X^n\to\OO_X$ defined by the sections $f_i$ \sref[0]{0.5.1.1};
since, for all $x\in X$, at least one of the $(f_i)_x$ is invertible, this homomorphism is \emph{surjective}, and we thus have an exact sequence $0\to\sh{R}\to\OO_X^n\to\OO_X\to0$, where $\sh{R}$ is a quasi-coherent $\OO_X$-submodule of $\OO_X$.
It then follows
\oldpage[II]{98}
from (c$'$) that the corresponding homomorphism $\Gamma(X,\OO_X^n)\to\Gamma(X,\OO_X)$ is surjective, which proves (b).

Finally, (a) implies (d) \sref[I]{1.5.1.9.2}, and (d) trivially implies (d$'$).
It remains to show that (d$'$) implies (c$'$).
But if $\sh{F}'$ is a quasi-coherent $\OO_X$-submodule of $\OO_X^n$, then the filtration $0\subset\OO_X\subset\OO_X^2\subset\ldots\subset\OO_X^n$ defines a filtration of $\sh{F}'$ given by the $\sh{F}'_k=\sh{F}\cap\OO_X^k$ ($0\leq k\leq n$), which are quasi-coherent $\OO_X$-modules \sref[I]{1.4.1.1}, and $\sh{F}'_{k+1}/\sh{F}'_k$ is isomorphic to a quasi-coherent $\OO_X$-submodule of $\OO_X^{k+1}/\OO_X^k=\OO_X$, which is to say, a quasi-coherent sheaf of ideals of $\OO_X$.
Hypothesis~(d$'$) thus implies that $\HH^1(X,\sh{F}'_{k+1}/\sh{F}'_k)=0$;
the exact cohomology sequence $\HH^1(X,\sh{F}'_k)\to\HH^1(X,\sh{F}'_{k+1})\to\HH^1(X,\sh{F}'_{k+1}/\sh{F}'_k)=0$ then lets us prove by induction on $k$ that $H^1(X,\sh{F}'_k)=0$ for all $k$.
\end{proof}

\begin{rmk}[5.2.1.1]
\label{2.5.2.1.1}
When $X$ is a \emph{Noetherian} prescheme, we can replace ``quasi-coherent'' by ``coherent'' in the statements of (c$'$) and (d$'$).
Indeed, in the proof of the fact that (c$'$) implies (b), $\sh{J}$ and $\sh{J}'$ are then \emph{coherent} sheaves of ideals, and, furthermore, every quasi-coherent submodule of a coherent module is coherent \sref[I]{1.6.1.1};
whence the conclusion.
\end{rmk}

\begin{cor}[5.2.2]
\label{2.5.2.2}
Let $f:X\to Y$ be a separated quasi-compact morphism.
The following conditions are equivalent.
\begin{enumerate}[label=\emph{(\alph*)}]
  \item The morphism $f$ is an affine morphism.
  \item The functor $f_*$ is exact on the category of quasi-coherent $\OO_X$-modules.
  \item For every quasi-coherent $\OO_X$-module $\sh{F}$, we have $\RR^1 f_*(\sh{F})=0$.
  \item[\emph{(c$'$)}] for every quasi-coherent sheaf of ideals $\sh{J}$ of $\OO_X$, we have $\RR^1 f_*(\sh{J})=0$.
\end{enumerate}
\end{cor}

\begin{proof}
\label{proof-2.5.2.2}
All these conditions are local on $Y$, by definition of the functor $\RR^1 f_*$ (T,~3.7.3), and so we can assume that $Y$ is affine.
If $f$ is affine, then $X$ is affine, and property~(b) is nothing more than \sref[I]{1.1.6.4}.
Conversely, we now show that (b) implies (a):
for every quasi-coherent $\OO_X$-module $\sh{F}$, we have that $f_*(\sh{F})$ is a quasi-coherent $\OO_Y$-module \sref[I]{1.9.2.2}[a].
By hypothesis, the functor $f_*(\sh{F})$ is exact in $\sh{F}$, and the functor $\Gamma(Y,\sh{G})$ is exact in $\sh{G}$ (in the category of quasi-coherent $\OO_Y$-modules) because $Y$ is affine \sref[I]{1.1.3.11};
so $\Gamma(Y,f_*(\sh{F}))=\Gamma(X,\sh{F})$ is exact in $\sh{F}$, which proves our claim, by \sref{2.5.2.1}[c].

If $f$ is affine, then $f^{-1}(U)$ is affine for every affine open subset $U$ of $Y$ \sref{2.1.3.2}, and so $\HH^1(f^{-1}(U),\sh{F})=0$ \sref{2.5.2.1}[d], which, by definition, implies that $\RR^1 f_*(\sh{F})=0$.
Finally, suppose that condition~(c$'$) is satisfied;
the exact sequence of terms of low degree in the Leray spectral sequence (G,~II,~4.17.1 and I,~4.5.1) give, in particular, the exact sequence
\[
    0\to\HH^1(Y,f_*(\sh{J}))\to\HH^1(X,\sh{J})\to\HH^0(Y,\RR^1 f_*(\sh{J})).
\]
Since $Y$ is affine, and $f_*(\sh{J})$ quasi-coherent \sref[I]{1.9.2.2}[a], we have that $\HH^1(Y,f_*(\sh{J}))=0$ \sref{2.5.2.1};
hypothesis~(c$'$) thus implies that $\HH^1(X,\sh{J})=0$, and we conclude, by \sref{2.5.2.1}, that $X$ is an affine scheme.
\end{proof}

\begin{cor}[5.2.3]
\label{2.5.2.3}
If $f:X\to Y$ is an affine morphism, then, for every quasi-coherent $\OO_X$-module $\sh{F}$, the canonical homomorphism $\HH^1(Y,f_*(\sh{F}))\to\HH^1(X,\sh{F})$ is bijective.
\end{cor}

\oldpage[II]{99}
\begin{proof}
\label{proof-2.5.2.3}
We have the exact sequence
\[
  0\to\HH^1(Y,f_*(\sh{F}))\to\HH^1(X,\sh{F})\to\HH^0(Y,\RR^1 f_*(\sh{F}))
\]
of terms of low degree in the Leray spectral sequence, and the conclusion follows from \sref{2.5.2.2}.
\end{proof}

\begin{rmk}[5.2.4]
\label{2.5.2.4}
In Chapter~III,~\textsection1, we prove that, if $X$ is affine, then we have $\HH^i(X,\sh{F})=0$ for all $i>0$ and all quasi-coherent $\OO_X$-modules $\sh{F}$.
\end{rmk}

\subsection{Quasi-projective morphisms}
\label{subsection:quasi-projective-morphisms}

\begin{defn}[5.3.1]
\label{2.5.3.1}
We say that a morphism $f:X\to Y$ is \emph{quasi-projective}, or that $X$ (considered as a $Y$-prescheme via $f$) is \emph{quasi-projective over $Y$}, or that $X$ is a \emph{quasi-projective $Y$-scheme}, if $f$ is of finite type and there exists an invertible $f$-ample $\OO_X$-module.
\end{defn}

We note that this notion \emph{is not local on $Y$}:
the counterexamples of Nagata \cite{II-26} and Hironaka show that, even if $X$ and $Y$ are non-singular algebraic schemes over an algebraically closed field, every point of $Y$ can have an affine neighbourhood $U$ such that $f^{-1}(U)$ is quasi-projective over $U$, without $f$ being quasi-projective.

We note that a quasi-projective morphism is necessarily \emph{separated} \sref{2.4.6.1}.
When $Y$ is quasi-compact, it is equivalent to say either that $f$ is quasi-projective, or that $f$ is of finite type and there exists a \emph{very ample} (relative to $f$) $\OO_X$-module (\sref{2.4.6.2} and \sref{2.4.6.11}).
Further:

\begin{prop}[5.3.2]
\label{2.5.3.2}
Let $Y$ be a quasi-compact scheme or a prescheme whose underlying space is Noetherian, and let $X$ be a $Y$-prescheme.
The following conditions are equivalent.
\begin{enumerate}[label=\emph{(\alph*)}]
  \item $X$ is a quasi-projective $Y$-scheme.
  \item $X$ is of finite type over $Y$, and there exists some quasi-coherent $\OO_Y$-module $\sh{E}$ of finite type such that $X$ is $Y$-isomorphic to a subprescheme of $\bb{P}(\sh{E})$.
  \item $X$ is of finite type over $Y$, and there exists some quasi-coherent graded $\OO_Y$-algebra $\sh{S}$ such that $\sh{S}_1$ is of finite type and generates $\sh{S}$, and such that $X$ is $Y$-isomorphic to a induced subprescheme on some everywhere-dense open subset of $\Proj(\sh{S})$.
\end{enumerate}
\end{prop}

\begin{proof}
\label{proof-2.5.3.2}
This follows immediately from the previous remark and from \sref{2.4.4.3}, \sref{2.4.4.6}, and \sref{2.4.4.7}.
\end{proof}

We note that, whenever $Y$ is a \emph{Noetherian} prescheme, we can, in conditions~(b) and (c) of \sref{2.5.3.2}, remove the hypothesis that $X$ is of finite type over $Y$, since this automatically satisfied \sref[I]{1.6.3.5}.

\begin{cor}[5.3.3]
\label{2.5.3.3}
Let $Y$ be a quasi-compact scheme such that there exists an ample $\OO_Y$-module $\sh{L}$ \sref{2.4.5.3}.
For a $Y$-scheme $X$ to be quasi-projective, it is necessary and sufficient for it to be of finite type over $Y$ and also isomorphic to a $Y$-subscheme of a projective bundle of the form $\bb{P}_Y^r$.
\end{cor}

\begin{proof}
\label{proof-2.5.3.3}
If $\sh{E}$ is a quasi-coherent $\OO_Y$-module of finite type, then $\sh{E}$ is isomorphic to a quotient of an $\OO_Y$-module $\sh{L}^{\otimes(-n)}\otimes_{\OO_Y}\OO_Y^k$ \sref{2.4.5.5}, and so $\bb{P}(\sh{E})$ is isomorphic to a closed subscheme of $\bb{P}_Y^{k-1}$ (\sref{2.4.1.2} and \sref{2.4.1.4}).
\end{proof}

\begin{prop}[5.3.4]
\label{2.5.3.4}
\medskip\noindent
\begin{enumerate}[label=\emph{(\roman*)}]
  \item A quasi-affine morphism of finite type (and, in particular, a quasi-compact immersion, or an affine morphism of finite type) is quasi-projective.
  \item If $f:X\to Y$ and $g:Y\to Z$ are quasi-projective, and if $Z$ is quasi-compact, then $g\circ f$ is quasi-projective.
\oldpage[II]{100}
  \item If $f:X\to Y$ is a quasi-projective $S$-morphism, then $f_{(S')}:X_{(S')}\to Y_{(S')}$ is quasi-projective for every extension $S'\to S$ of the base prescheme.
  \item If $f:X\to Y$ and $g:X'\to Y'$ are quasi-projective $S$-morphisms, then $f\times_S g$ is quasi-projective.
  \item If $f:X\to Y$ and $g:Y\to Z$ are morphisms such that $g\circ f$ is quasi-projective, and if $g$ is separated or $X$ locally Noetherian, then $f$ is quasi-projective.
  \item If $f$ is a quasi-projective morphism, then so is $f_\red$.
\end{enumerate}
\end{prop}

\begin{proof}
\label{proof-2.5.3.4}
(i) follows from \sref{2.5.1.6} and \sref{2.5.1.10}[i].
The other claims are immediate consequences of Definition~\sref{2.5.3.1}, of the properties of morphisms of finite type \sref[I]{1.6.3.4}, and of \sref{2.4.6.13}.
\end{proof}

\begin{rmk}[5.3.5]
\label{2.5.3.5}
We note that we can have $f_\red$ being quasi-projective without $f$ being quasi-projective, even if we assume that $Y$ is the spectrum of an algebra of finite rank over $\bb{C}$ and that $f$ is proper.
\end{rmk}

\begin{cor}[5.3.6]
\label{2.5.3.6}
If $X$ and $X'$ are quasi-projective $Y$-schemes, then $X\sqcup X'$ is a quasi-projective $Y$-scheme.
\end{cor}

\begin{proof}
\label{proof-2.5.3.6}
This follows from \sref{2.4.6.18}.
\end{proof}

\subsection{Proper morphisms and universally closed morphisms}
\label{subsection:proper-morphisms-and-universally-closed-morphisms}

\begin{defn}[5.4.1]
\label{2.5.4.1}
We say that a morphism of preschemes $f:X\to Y$ is \emph{proper} if it satisfies the following two conditions:
\begin{enumerate}[label=(\alph*)]
  \item $f$ is separated and of finite type; and
  \item for every prescheme $Y'$ and every morphism $Y'\to Y$, the projection $f_{(Y')}:X\times_Y Y'\to Y'$ is a closed morphism \sref[I]{1.2.2.6}.
\end{enumerate}

When this is the case, we also say that $X$ (considered as a $Y$-prescheme with structure morphism $f$) is proper over $Y$.
\end{defn}

It is immediate that conditions~(a) and (b) are \emph{local} on $Y$.
To show that the image of a closed subset $Z$ of $X\times_Y Y'$ under the projection $q:X\times_Y Y'\to Y'$ is closed in $Y$, it suffices to see that $q(Z)\cap U'$ is closed in $U'$ for every affine open subset $U'$ of $Y'$;
since $q(Z)\cap U'=q(Z\cap q^{-1}(U'))$, and since $q^{-1}(U')$ can be identified with $X\times_Y U'$ \sref[I]{1.4.4.1}, we see that to satisfy condition~(b) of Definition~\sref{2.5.4.1}, we can restrict to the case where $Y$ is an \emph{affine} scheme.
We further see \sref{2.5.3.6} that, if $Y$ is locally Noetherian, then we can even restrict to proving (b) in the case where $Y'$ is of finite type over $Y$.

It is clear that every proper morphism is \emph{closed}.

\begin{prop}[5.4.2]
\label{2.5.4.2}
\medskip\noindent
\begin{enumerate}[label=\emph{(\roman*)}]
  \item A closed immersion is a proper morphism.
  \item The composition of two proper morphisms is proper.
  \item If $X$ and $Y$ are $S$-preschemes, and $f:X\to Y$ a proper $S$-morphism, then $f_{(S')}:X_{(S')}\to Y_{(S')}$ is proper for every extension $S'\to S$ of the base prescheme.
  \item If $f:X\to Y$ and $g:X'\to Y'$ are proper $S$-morphisms, then $f\times_S g:X\times_S Y\to X'\times_S Y'$ is a proper $S$-morphism.
\end{enumerate}
\end{prop}

\oldpage[II]{101}
\begin{proof}
\label{proof-2.5.4.2}
It suffices to prove (i), (ii), and (iii) \sref[I]{1.3.5.1}.
In each of the three cases, verifying condition~(a) of \sref{2.5.4.1} follows from previous results (\sref[I]{1.5.5.1} and \sref{2.6.4.3}); it remains to verify condition~(b).
It is immediate in case (i), because if $X\to Y$ is a closed immersion, then so is $X\times_Y Y'\to Y\times_Y Y'=Y'$ (\sref[I]{1.4.3.2} and \sref{2.3.3.3}).
To prove (ii), consider two proper morphisms $X\to Y$ and $Y\to Z$, and a morphism $Z'\to Z$.
We can write $X\times_Z Z'=X\times_Y(Y\times_Z Z')$ \sref[I]{1.3.3.9.1}, and so the projection $X\times_Z Z'\to Z'$ factors as $X\times_Y(Y\times_Z Z')\to Y\times_Z Z'\to Z'$.
Taking the initial remark into account, (ii) follows from the fact that the composition of two closed morphisms is closed.
Finally, for every morphism $S'\to S$, we can identify $X_{(S')}$ with $X\times_Y Y_{(S')}$ \sref[I]{1.3.3.11}; for every morphism $Z\to Y_{(S')}$, we can write
\[
  X_{(S')}\times_{Y_{(S')}}Z=(X\times_Y Y_{(S')})\times_{Y_{(S')}}Z=X\times_Y Z;
\]
since by hypothesis $X\times_Y Z\to Z$ is closed, this proves (iii).
\end{proof}

\begin{cor}[5.4.3]
\label{2.5.4.3}
Let $f:X\to Y$ and $g:Y\to Z$ be morphisms such that $g\circ f$ is proper.
\begin{enumerate}[label=\emph{(\roman*)}]
  \item If $g$ is separated, then $f$ is proper.
  \item If $g$ is separated and of finite type, and if $f$ is surjective, then $g$ is proper.
\end{enumerate}
\end{cor}

\begin{proof}
\label{proof-2.5.4.3}
(i) follows from \sref{2.5.4.2} by the general procedure \sref[I]{1.5.5.12}.
To prove (ii), we need only verify that condition~(b) of Definition~\sref{2.5.4.1} is satisfied.
For every morphism $Z'\to Z$, the diagram
\[
  \xymatrix{
    X\times_Z Z'\ar[r]^{f\times1_{Z'}}\ar[dr]_p &
    Y\times_Z Z'\ar[d]^{p'}\\
    & Z'
  }
\]
(where $p$ and $p'$ are the projections) commutes \sref[I]{1.3.2.1};
furthermore, $f\times1_{Z'}$ is surjective because $f$ is surjective \sref[I]{1.3.5.2}, and $p$ is a closed morphism by hypothesis.
Every closed subset $F$ of $Y\times_Z Z'$ is thus the image under $f\times1_{Z'}$ of some closed subset $E$ of $X\times_Z Z'$, so $p'(F)=p(E)$ is closed in $Z'$ by hypothesis, whence the corollary.
\end{proof}

\begin{cor}[5.4.4]
\label{2.5.4.4}
If $X$ is a proper prescheme over $Y$, and $\sh{S}$ a quasi-coherent $\OO_Y$-algebra, then every $Y$-morphism $f:X\to\Proj(\sh{S})$ is proper (and \emph{a fortiori} closed).
\end{cor}

\begin{proof}
\label{proof-2.5.4.4}
The structure morphism $p:\Proj(\sh{S})\to Y$ is separated, and $p\circ f$ is proper by hypothesis.
\end{proof}

\begin{cor}[5.4.5]
\label{2.5.4.5}
Let $f:X\to Y$ be a separated morphism of finite type.
Let $(X_i)_{1\leq i\leq n}$ (resp. $(Y_i)_{1\leq i\leq n}$) be a finite family of closed subpreschemes of $X$ (resp. $Y$), and $j_i$ (resp. $h_i$) the canonical injection $X_i\to X$ (resp. $Y_i\to Y$).
Suppose that the underlying space of $X$ is the union of the $X_i$, and that, for all $i$, there is a morphism $f_i:X_i\to Y_i$, such that the diagram
\[
  \xymatrix{
    X_i\ar[r]^{f_i}\ar[d]_{j_i} &
    Y_i\ar[d]^{h_i}\\
    X\ar[r]^f &
    Y
  }
\]
commutes.
Then, for $f$ to be proper, it is necessary and sufficient for all of the $f_i$ to be proper.
\end{cor}

\oldpage[II]{102}
\begin{proof}
\label{proof-2.5.4.5}
If $f$ is proper, then so is $f\circ j_i$, because $j_i$ is a closed immersion \sref{2.5.4.2};
since $h_i$ is a closed immersion, and thus a separated morphism, $f_i$ is proper, by \sref{2.5.4.3}.
Conversely, suppose that all of the $f_i$ are proper, and consider the prescheme $Z$ given by the \emph{sum} of the $X_i$; let $u$ be the morphism $Z\to X$ which reduces to $j_i$ on each $X_i$.
The restriction of $f\circ u$ to each $X_i$ is equal to $f\circ j_i=h_i\circ f_i$, and is thus proper, because both the $h_i$ and the $f_i$ are \sref{2.5.4.2};
it then follows immediately from Definition~\sref{2.5.4.1} that $u$ is proper.
But since by hypothesis $u$ is surjective, we conclude that $f$ is proper by \sref{2.5.4.3}.
\end{proof}

\begin{cor}[5.4.6]
\label{2.5.4.6}
Let $f:X\to Y$ be a separated morphism of finite type; for $f$ to be proper, it is necessary and sufficient for $f_\red:X_\red\to Y_\red$ to be proper.
\end{cor}

\begin{proof}
\label{proof-2.5.4.6}
This is a particular case of \sref{2.5.4.5}, with $n=1$, $X_1=X_\red$, and $Y_1=Y_\red$ \sref[I]{1.5.1.5}.
\end{proof}

\begin{env}[5.4.7]
\label{2.5.4.7}
If $X$ and $Y$ are Noetherian preschemes, and $f:X\to Y$ a separated morphism of finite type, then we can, to show that $f$ is proper, restrict to the the case of \emph{dominant} morphisms and \emph{integral} preschemes.
Indeed, let $X_i$ ($1\leq i\leq n$) be the (finitely many) irreducible components of $X$, and consider, for each $i$, the unique reduced closed subprescheme of $X$ that has $X_i$ as its underlying space, which we again denote by $X_i$ \sref[I]{1.5.2.1}.
Let $Y_i$ be the unique reduced closed subprescheme of $Y$ that has $\overline{f(X_i)}$ as its underlying space.
If $g_i$ (resp. $h_i$) is the injection morphism $X_i\to X$ (resp. $Y_i\to Y$), then we conclude that $f\circ g_i=h_i\circ f_i$, where $f_i$ is a dominant morphism $X_i\to Y_i$ \sref[I]{1.5.2.2};
we are then under the right conditions to apply \sref{2.5.4.5}, and for $f$ to be proper, it is necessary and sufficient for all the $f_i$ to be proper.
\end{env}

\begin{cor}[5.4.8]
\label{2.5.4.8}
Let $X$ and $Y$ be separated $S$-preschemes of finite type over $S$, and $f:X\to Y$ an $S$-morphism.
For $f$ to be proper, it is necessary and sufficient that, for every $S$-prescheme $S'$, the morphism $f\times_S 1_{S'}:X\times_S S'\to Y\times_S S'$ be closed.
\end{cor}

\begin{proof}
\label{proof-2.5.4.8}
First note that, if $g:X\to S$ and $h:Y\to S$ are the structure morphisms, then we have, by definition, $g=h\circ f$, and so $f$ is separated and of finite type (\sref[I]{1.5.5.1} and \sref{2.6.3.4}).
If $f$ is proper, then so is $f\times_S 1_{S'}$ \sref{2.5.4.2}; \emph{a fortiori}, $f\times_S 1_{S'}$ is closed.
Conversely, suppose that the conditions of the statement are satisfied, and let $Y'$ be a $Y$-prescheme;
$Y'$ can also be considered as an $S$-prescheme, and since $Y\to S$ is separated, $X\times_Y Y'$ can be identified with a closed subprescheme of $X\times_S Y'$ \sref[I]{1.5.4.2}.
In the commutative diagram
\[
  \xymatrix{
    X\times_Y Y'\ar[r]^{f\times1_{Y'}}\ar[d] &
    Y\times_Y Y'=Y'\ar[d]\\
    X\times_S Y'\ar[r]^{f\times1_{S'}} &
    Y\times_S Y',
  }
\]
the vertical arrows are closed immersions; it thus immediately follows that if $f\times1_{S'}$ is a closed morphism, then so is $f\times1_{Y'}$
\end{proof}

\begin{rmk}[5.4.9]
\label{2.5.4.9}
We say that a morphism $f:X\to Y$ is \emph{universally closed} if it satisfies condition~(b) of Definition~\sref{2.5.4.1}.
The reader will observe that,
\oldpage[II]{103}
in \sref{2.5.4.2} to \sref{2.5.4.8}, we can replace every occurrence of ``proper'' with ``universally closed'' without changing the validity of the results (and in the hypotheses of \sref{2.5.4.3}, \sref{2.5.4.5}, \sref{2.5.4.6}, and \sref{2.5.4.8}, we can omit the finiteness conditions).
\end{rmk}

\begin{env}[5.4.10]
\label{2.5.4.10}
Let $f:X\to Y$ be a morphism of finite type.
We say that a closed subset $Z$ of $X$ is \emph{proper on $Y$} (or \emph{$Y$-proper}, or \emph{proper for $f$}) if the restriction of $f$ to a closed subprescheme of $X$, with underlying space $Z$ \sref[1]{1.5.2.1}, is \emph{proper}.
Since this restriction is then separated, it follows from \sref{2.5.4.6} and \sref[I]{1.5.5.1}[vi] that the preceding property \emph{does not depend} on the closed subprescheme of $X$ that has $Z$ as its underlying space.
If $g:X'\to X$ is a \emph{proper} morphism, then $g^{-1}(Z)$ is a \emph{proper} subset of $X'$:
if $T$ is a subprescheme of $X$ that has $Z$ as its underlying space, it suffices to note that the restriction of $g$ to the closed subprescheme $g^{-1}(T)$ of $X'$ is a proper morphism $g^{-1}(T)\to T$, by \sref{2.5.4.2}[iii], and to then apply \sref{2.5.4.2}[ii].
Further, if $X''$ is a $Y$-\emph{scheme} of finite type, and $u:X\to X''$ a $Y$-morphism, then $u(Z)$ is a \emph{proper} subset of $X''$;
indeed, let us take $T$ to be the reduced closed subprescheme of $X$ having $Z$ as its underlying space;
then the restriction of $f$ to $T$ is proper, and thus so is the restriction of $u$ to $T$ \sref{2.5.4.3}[i], thus $u(Z)$ is closed in $X''$;
let $T''$ be a closed subprescheme of $X''$ having $u(Z)$ as its underlying space \sref[I]{1.5.2.1}, such that $u|T$ factors as $T\xrightarrow{v}T''\xrightarrow{j}X''$, where $j$ is the canonical injection \sref[I]{1.5.2.2}, and $v$ is thus proper and surjective \sref{2.5.4.5};
if $g$ is the restriction to $T''$ of the structure morphism $X''\to Y$, then $g$ is separated and of finite type, and we have that $f|T=g\circ v$;
it thus follows from \sref{2.5.4.3}[ii] that $g$ is proper, whence our assertion.
\end{env}

It follows, in particular, from these remarks that, if $Z$ is a $Y$-proper subset of $X$, then
\begin{enumerate}
  \item for every closed subprescheme $X'$ of $X$, $Z\cap X'$ is a $Y$-proper subset of $X'$; and
  \item if $X$ is a subprescheme of a $Y$-scheme of finite type $X''$, then $Z$ is also a $Y$-proper subset of $X''$ (and so, in particular, is \emph{closed in $X''$}).
\end{enumerate}

\subsection{Projective morphisms}
\label{subsection:projective-morphisms}

\begin{prop}[5.5.1]
\label{2.5.5.1}
Let $X$ be a $Y$-prescheme.
The following conditions are equivalent.
\begin{enumerate}[label=\emph{(\alph*)}]
  \item $X$ is $Y$-isomorphic to a \emph{closed} subprescheme of a projective bundle $\bb{P}(\sh{E})$, where $\sh{E}$ is a quasi-coherent $\OO_Y$-module of finite type.
  \item There exists a quasi-coherent graded $\OO_Y$-algebra $\sh{S}$ such that $\sh{S}_1$ is of finite type and generates $\sh{S}$, and such that $X$ is $Y$-isomorphic to $\Proj(\sh{S})$.
\end{enumerate}
\end{prop}

\begin{proof}
\label{proof-2.5.5.1}
Condition~(a) implies (b), by \sref{2.3.6.2}[ii]: if $\sh{J}$ is a quasi-coherent graded sheaf of ideals of $\bb{S}(\sh{E})$, then the quasi-coherent graded $\OO_Y$-algebra $\sh{S}=\bb{S}(\sh{E})/\sh{J}$ is generated by $\sh{S}_1$, and $\sh{S}_1$, the canonical image of $\sh{E}$, is an $\OO_Y$-module of finite type.
Condition~(b) implies (a) by \sref{2.3.6.2} applied to the case where $\sh{M}\to\sh{S}_1$ is the identity map.
\end{proof}

\oldpage[II]{104}
\begin{defn}[5.5.2]
\label{2.5.5.2}
We say that a $Y$-prescheme $X$ is projective on $Y$, or is a projective $Y$-scheme, if it satisfies either of the (equivalent) conditions~(a) and (b) of \sref{2.5.5.1}.
We say that a morphism $f:X\to Y$ is projective if it makes $X$ a projective $Y$-scheme.
\end{defn}

It is clear that if $f:X\to Y$ is projective, then there exists a \emph{very ample} (relative to $f$) $\OO_X$-module \sref{2.4.4.2}.

\begin{thm}[5.5.3]
\label{2.5.5.3}
\medskip\noindent
\begin{enumerate}[label=\emph{(\roman*)}]
  \item Every projective morphism is quasi-projective and proper.
  \item Conversely, let $Y$ be a quasi-compact scheme or a prescheme whose underlying space is Noetherian; then every morphism $f:X\to Y$ that is quasi-projective and proper is projective.
\end{enumerate}
\end{thm}

\begin{proof}
\label{proof-2.5.5.3}
\medskip\noindent
\begin{enumerate}[label=(\roman*)]
  \item It is clear that if $f:X\to Y$ is projective, then it is of finite type and quasi-projective (thus, in particular, separated); furthermore, it follows immediately from \sref{2.5.5.1}[b] and \sref{2.3.5.3} that if $f$ is projective, then so is $f\times_Y 1_{Y'}:X\times_Y Y'\to Y'$ for every morphism $Y'\to Y$.
     To show that $f$ is universally closed, it is thus enough to show that a projective morphism $f$ is \emph{closed}.
     Since the question is local on $Y$, we can suppose that $Y=\Spec(A)$, thus \sref{2.5.5.1} $X=\Proj(S)$, where $S$ is a graded $A$-algebra generated by a finite number of elements of $S_1$.
     For all $y\in Y$, the fibre $f^{-1}(y)$ can be identified with $\Proj(S)\times_Y\Spec(\kres(y))$ \sref[I]{1.3.6.1}, and so also with $\Proj(S\otimes_A\kres(y))$ \sref{2.2.8.10};
     so $f^{-1}(y)$ is empty if and only if $S\otimes_A\kres(y)$ satisfies condition~(TN) \sref{2.2.7.4}, or, in other words, if $S_n\otimes_A\kres(y)=0$ for sufficiently large $n$.
     But since $(S_n)_y$ is an $\OO_y$-module of finite type, the preceding condition implies that $(S_n)_y=0$ for sufficiently large $N$, by Nakayama's lemma.
     If $\fk{a}_n$ is the annihilator in $A$ of the $A$-module $S_n$, then the preceding condition also implies that $\fk{a}_n\subset\fk{j}_n$ for sufficiently large $n$ \sref[0]{0.1.7.4}.
     But since $S_nS_1=S_{n+1}$, by hypothesis, we have that $\fk{a}_n\subset\fk{a}_{n+1}$, and if $\fk{a}$ is the union of the $\fk{a}_n$, then we see that $f(X)=V(\fk{a})$, which proves that $f(X)$ is closed in $Y$.
     If now $X'$ is an arbitrary closed subset of $X$, then there exists a closed subprescheme of $X$ that has $X'$ as its underlying space \sref[I]{1.5.2.1}, and it is clear \sref{2.5.5.1}[a] that the morphism $X'\to X\xrightarrow{f}Y$ is projective, and so $f(X')$ is closed in $Y$.
   \item The hypothesis on $Y$ and the fact that $f$ is quasi-projective implies the existence of a quasi-coherent $\OO_Y$-module $\sh{E}$ of finite type, as well as a $Y$-immersion $j:X\to\bb{P}(\sh{E})$ \sref{2.5.3.2}.
      But since $f$ is proper, $j$ is \emph{closed}, by \sref{2.5.4.4}, and so $f$ is projective.
\end{enumerate}
\end{proof}

\begin{rmk}[5.5.4]
\label{2.5.5.4}
\medskip\noindent
\begin{enumerate}[label=(\roman*)]
  \item Let $f:X\to Y$ be a morphism such that $f$ is proper, such that there exists a \emph{very ample} (relative to $f$) $\OO_X$-module $\sh{L}$, and such that the quasi-coherent $\OO_Y$-module $\sh{E}=f_*(\sh{L})$ is \emph{of finite type}.
    Then $f$ is a \emph{projective} morphism: indeed \sref{2.4.4.4}, there is then a $Y$-immersion $r:X\to\bb{P}(\sh{E})$, and, since $f$ is proper, $r$ is a \emph{closed} immersion \sref{2.5.4.4}.
    We will see in Chapter~III, \textsection3, that when $Y$ is \emph{locally Noetherian}, the third condition above ($\sh{E}$ being of finite type) is a consequence of the first two, and so the first two conditions \emph{characterise}, in this case, the projective morphisms, and if $Y$ is quasi-compact, then we can replace the second condition (the existence of a very ample (relative to $f$) $\OO_X$-module $\sh{L}$) by the hypothesis that there exists an \emph{ample} (relative to $f$) $\OO_X$-module \sref{2.4.6.11}.
  \item Let $Y$ be a quasi-compact scheme such that there exists an ample $\OO_Y$-module.
    For a $Y$-scheme $X$ to be \emph{projective}, it is necessary and sufficient for it to be $Y$-isomorphic to a \emph{closed} $Y$-subscheme of a projective bundle of the form $\bb{P}_Y^r$.
    The condition is clearly sufficient.
\oldpage[II]{105}
    Conversely, if $X$ is projective over $Y$, then it is quasi-projective, and so there exists a $Y$-immersion $j$ of $X$ into some $\bb{P}_Y^r$ \sref{2.5.3.3} that is \emph{closed}, by \sref{2.5.4.4} and \sref{2.5.5.3}.
  \item The argument of \sref{2.5.5.3} shows that, for every prescheme $Y$ and every integer $r\geq0$, the structure morphism $\bb{P}_Y^r\to Y$ is \emph{surjective}, because if we set $\sh{S}=\bb{S}_{\OO_Y}(\OO_Y^{r+1})$, then we evidently have $\sh{S}_y=\bb{S}_{\kres(y)}(\kres(y)^{r+1})$ \sref{2.1.7.3}, and so $(\sh{S}_n)_y\neq0$ for any $y\in Y$ or any $n\geq0$.
  \item It follows from the examples of Nagata \cite{II-26} that there exist proper morphisms that are not quasi-projective.
\end{enumerate}
\end{rmk}

\begin{prop}[5.5.5]
\label{2.5.5.5}
\medskip\noindent
\begin{enumerate}[label=\emph{(\roman*)}]
  \item A closed immersion is a projective morphism.
  \item If $f:X\to Y$ and $g:Y\to Z$ are projective morphisms, and if $Z$ is a quasi-compact scheme or a prescheme whose underlying space is Noetherian, then $g\circ f$ is projective.
  \item If $f:X\to Y$ is a projective $S$-morphism, then $f_{(S')}:X_{(S')}\to Y_{(S')}$ is projective for every extension $S'\to S$ of the base prescheme.
  \item If $f:X\to Y$ and $g:X'\to Y'$ are projective $S$-morphisms, then so is $f\times_S g$.
  \item If $g\circ f$ is a projective morphism, and if $g$ is separated, then $f$ is projective.
  \item If $f$ is projective then so is $f_\red$.
\end{enumerate}
\end{prop}

\begin{proof}
\label{proof-2.5.5.5}
(i) follows immediately from \sref{2.3.1.7}.
We have to show (iii) and (iv) separately, because of the restriction introduced on $Z$ in (ii) (cf.~\sref[I]{1.3.5.1}).
To show (iii), we restrict to the case where $S=Y$ \sref[I]{1.3.3.11}, and the claim then immediately follows from \sref{2.5.5.1}[b] and \sref{2.3.5.3}.
To show (iv), we are immediately led to the case where $X=\bb{P}(\sh{E})$ and $X=\bb{P}(\sh{E}')$, where $\sh{E}$ (resp. $\sh{E}'$) is a quasi-coherent $\OO_Y$-module (resp. quasi-coherent $\OO_{Y'}$-module) of finite type.
Let $p$ and $p'$ be the canonical projections of $T=Y\times_S Y'$ to $Y$ and $Y'$ (respectively); by \sref{2.4.1.3.1}, we have $\bb{P}(p^*(\sh{E})) = \bb{P}(\sh{E})\times_Y T$ and $\bb{P}(p^{\prime *}(\sh{E}'))=\bb{P}(\sh{E}')\times_{Y'}T$; whence
\begin{align*}
  \bb{P}(p^*(\sh{E}))\times_T\bb{P}(p^{\prime *}(\sh{E}'))&=(\bb{P}(\sh{E})\times_Y T)\times_T(T\times_{Y'}\bb{P}(\sh{E}'))\\
                                              &=\bb{P}(\sh{E})\times_Y(T\times_{Y'}\bb{P}(\sh{E}'))=\bb{P}(\sh{E})\times_S\bb{P}(\sh{E}')
\end{align*}
by replacing $T$ with $Y\times_S Y'$, and using \sref[I]{1.3.3.9.1}.
But $p^*(\sh{E})$ and $p^{\prime *}(\sh{E}')$ are of finite type over $T$ \sref[0]{0.5.2.4}, and thus so is $p^*(\sh{E})\otimes_{\OO_T}p^{\prime *}(\sh{E}')$;
since $\bb{P}(p^*(\sh{E}))\times_T\bb{P}(p^{\prime *}(\sh{E}'))$ can be identified with a closed subprescheme of $p^*(\sh{E})\otimes_{\OO_T}p^{\prime *}(\sh{E}')$ \sref{2.4.3.3}, this proves (iv).
To show (v) and (vi), we can apply \sref[I]{1.5.5.13}, because every closed subprescheme of a projective $Y$-scheme is a projective $Y$-scheme, by \sref{2.5.5.1}[a].

It remains to prove (ii); by the hypothesis on $Z$, this follows from \sref{2.5.5.3}, \sref{2.5.3.4}[ii], and \sref{2.5.4.2}[ii].
\end{proof}

\begin{prop}[5.5.6]
\label{2.5.5.6}
If $X$ and $X'$ are projective $Y$-schemes, then $X\sqcup X'$ is a projective $Y$-scheme.
\end{prop}

\begin{proof}
\label{proof-2.5.5.6}
This is an evident consequence of \sref{2.5.5.2} and \sref{2.4.3.6}.
\end{proof}

\begin{prop}[5.5.7]
\label{2.5.5.7}
Let $X$ be a projective $Y$-scheme, and $\sh{L}$ a $Y$-ample $\OO_X$-module; then, for every section $f$ of $\sh{L}$ over $X$, $X_f$ is affine over $Y$.
\end{prop}

\oldpage[II]{106}
\begin{proof}
\label{proof-2.5.5.6}
Since the question is local on $Y$, we can assume that $Y=\Spec(A)$; furthermore, $X_{f^{\otimes n}}=X_f$, so by replacing $\sh{L}$ with some suitable $\sh{L}^{\otimes n}$, we can assume that $\sh{L}$ is very ample relative to the structure morphism $q:X\to Y$ \sref{2.4.6.11}.
The canonical homomorphism $\sigma:q^*(q_*(\sh{L}))\to\sh{L}$ is thus surjective, and the corresponding morphism
\[
  r=r_{\sh{L},\sigma}:X\to P=\bb{P}(q_*(\sh{L}))
\]
is an immersion such that $\sh{L}=r^*(\OO_P(1))$ \sref{2.4.4.4}; furthermore, since $X$ is proper over $Y$, the immersion $r$ is closed \sref{2.5.4.4}.
But by definition, $f\in\Gamma(Y,q_*(\sh{L}))$, and $\sigma^\flat$ is the identity of $q_*(\sh{L})$; it then follows from Equation~\sref{2.3.7.3.1} that we have $X_f=r^{-1}(D_+(f))$;
so $X_f$ is a closed subprescheme of the affine scheme $D_+(f)$, and is thus also an affine scheme.
\end{proof}

In the particular case where $Y=X$, we obtain (taking \sref{2.4.6.13}[i] into account) the following corollary, whose direct proof is immediate anyway:
\begin{cor}[5.5.8]
\label{2.5.5.8}
Let $X$ be a prescheme, and $\sh{L}$ an invertible $\OO_X$-module.
For every section $f$ of $\sh{L}$ over $X$, $X_f$ is affine over $X$ (and thus also an affine scheme whenever $X$ is an affine scheme).
\end{cor}

\subsection{Chow's lemma}
\label{subsection:chows-lemma}

\begin{thm}[5.6.1]
\label{2.5.6.1}
\emph{(Chow's lemma).}
Let $S$ be a prescheme, and $X$ an $S$-scheme of finite type.
Suppose that the following conditions are satisfied:
\begin{enumerate}[label=\emph{(\alph*)}]
  \item $S$ is Noetherian;
  \item $S$ is a quasi-compact scheme, and $X$ has a finite number of irreducible components.
\end{enumerate}
Under these hypotheses,
\begin{enumerate}[label=\emph{(\roman*)}]
  \item there exists a \emph{quasi-projective} $S$-scheme $X'$, and an $S$-morphism $f:X'\to X$ that is both\emph{projective} and \emph{surjective};
  \item we can take $X'$ and $f$ to be such that there exists an open subset $U\subset X$ for which $U'=f^{-1}(U)$ is dense in $X'$, and for which the restriction of $f$ to $U'$ is an isomorphism $U'\isoto U$; and
  \item if $X$ is reduced (resp. irreducible, integral), then we can assume that $X'$ is reduced (resp. irreducible, integral).
\end{enumerate}
\end{thm}

\begin{proof}
\label{proof-2.5.6.1}
The proof proceeds in multiple steps.
\begin{enumerate}[label=(\Alph*)]
  \item We can first restrict to the case where $X$ is \emph{irreducible}.
    Indeed, in hypothesis~(a), $X$ is Noetherian, and so, in the two hypotheses, the irreducible components $X_i$ of $X$ are finite in number.
    If the theorem is shown to be true for each of the reduced closed preschemes of $X$ having the $X_i$ as their underlying spaces, and if $X'_i$ and $f_i:X'_i\to X_i$ are the prescheme and the morphism corresponding to $X_i$ (respectively), then the prescheme $X'$ given by the \emph{sum} of the $X'_i$, and the morphism $f:X'\to X$ whose restriction to each $X'_i$ is $j_i\circ f_i$ (where $j_i$ is the canonical injection $X_i\to X$) satisfy the conclusion of the theorem.
    It is immediate that $X'$ is reduced if all of the $X'_i$ are; furthermore, we can satisfy (ii) by taking $U$ to be the union of the sets $U_i\cap C\left(\bigcup_{j\neq i}X_j\right)$.
    Finally, since the $X'_i$ are quasi-projective over $S$, so is $X'$
\oldpage[II]{107}
    \sref{2.5.3.6}; similarly, the morphisms $X'_i\to X$ are projective by \sref{2.5.5.5}[i] and \sref{2.5.5.5}[ii], and so $f$ is projective \sref{2.5.5.6}, and is clearly surjective, by definition.
  \item Now suppose that $X$ is \emph{irreducible}.
    Since the structure morphism $r:X\to S$ is of finite type, there exists a finite cover $(S_i)$ of $S$ by affine open subsets, and for each $i$ there is a finite cover $(T_{ij})$ of $r^{-1}(S_i)$ by affine open subsets, and the morphisms $T_{ij}\to S_i$ are of finite type, and so quasi-projective \sref{2.5.3.4}[i];
    since in both hypotheses~(a) and (b) the immersion $S_i\to S$ is quasi-compact, it is also quasi-projective \sref{2.5.3.4}[i], and so the restriction of $r$ to $T_{ij}$ is a quasi-projective morphism \sref{2.5.3.4}[ii].
    Denote the $T_{ij}$ by $U_k$ ($1\leq k\leq n$).
    There exists, for each index $k$, an open immersion $\vphi_k:U_k\to P_k$, where $P_k$ is projective over $S$ (\sref{2.5.3.2} and \sref{2.5.5.2}).
    Let $U=\bigcap_k U_k$; since $X$ is irreducible, and the $U_k$ nonempty, $U$ is nonempty, and thus dense in $X$; the restrictions of the $\vphi_k$ to $U$ define a morphism
    \[
      \vphi:U\to P=P_1\times_S P_2\times_S\ldots\times_S P_n
    \]
    such that the diagrams
    \[
    \label{2.5.6.1.1}
      \xymatrix{
        U\ar[r]^\vphi\ar[d]_{j_k} &
        P\ar[d]^{p_k}\\
        U_k\ar[r]^{\vphi_k} &
        P_k
      }
      \tag{5.6.1.1}
    \]
    commute, where $j_k$ is the canonical injection $U\to U_k$, and $p_k$ the canonical projection $P\to P_k$.
    If $j$ is the canonical injection $U\to X$, then the morphism $\psi=(j,\vphi)_S:U\to X\times_S P$ is an \emph{immersion} \sref[I]{1.5.3.14}.
    In hypothesis~(a), $X\times_S P$ is locally Noetherian (\sref{2.3.4.1}, \sref[I]{1.6.3.7}, and \sref[I]{1.6.3.8});
    in hypothesis~(b), $X\times_S P$ is a quasi-compact scheme (\sref[I]{1.5.5.1} and \sref[I]{1.6.6.4});
    in both cases, the \emph{closure} $X'$ in $X\times_S P$ of the subprescheme $Z$ associated to $\psi$ (and so with underlying space $\psi(U)$) exists, and $\psi$ factors as
    \[
    \label{2.5.6.1.2}
      \psi:U\xrightarrow{\psi'}X'\xrightarrow{h}X\times_S P
      \tag{5.6.1.2}
    \]
    where $\psi'$ is an \emph{open immersion} and $h$ a \emph{closed immersion} \sref[I]{1.9.5.10}.
    Let $q_1:X\times_S P\to X$ and $q_2:X\times_S P\to P$ be the canonical projections; we set
    \[
    \label{2.5.6.1.3}
      f:X'\xrightarrow{h}X\times_S P\xrightarrow{q_1}X,
      \tag{5.6.1.3}
    \]
    \[
    \label{2.5.6.1.4}
      g:X'\xrightarrow{h}X\times_S P\xrightarrow{q_2}P.
      \tag{5.6.1.4}
    \]
    We will see that $X'$ and $f$ satisfy the conclusion of the theorem.
  \item First we show that $f$ is \emph{projective} and \emph{surjective}, and that the restriction of $f$ to $U'=f^{-1}(U)$ is an \emph{isomorphism} from $U'$ to $U$.
    Since the $P_k$ are projective over $S$, so is $P$ \sref{2.5.5.5}[iv], and so $X\times_S P$ is projective over $X$ \sref{2.5.5.5}[iii], and thus so is $X'$, which is a closed subprescheme of $X\times_S P$.
    Furthermore, we have $f\circ\psi'=q_1\circ(h\circ\psi')=q_1\circ\psi=j$, so $f(X')$ contains the open everywhere-dense subset $U$ of $X$; but $f$ is a \emph{closed} morphism \sref{2.5.5.3}, so $f(X')=X$.
    Now note that $q_1^{-1}(U)=U\times_S P$ is induced on an open subset of $X\times_S P$, and, by definition, the prescheme $U'=h^{-1}(U\times_S P)$ is induced by $X'$ on the open subset $U'$; it is thus the closure \emph{relative to}
\oldpage[II]{108}
    $U\times_S P$ of the prescheme $Z$ \sref[I]{1.9.5.8}.
    But the immersion $\psi$ factors as $U\xrightarrow{\Gamma_\vphi}U\times_S P\xrightarrow{j\times1}X\times_S P$, and since $P$ is separated over $S$, the graph morphism $\Gamma_\vphi$ is a closed immersion \sref[I]{1.5.4.3}, and so $Z$ is a \emph{closed} subprescheme of $U\times_S P$, whence $U'=Z$.
    Since $\psi$ is an immersion, the restriction of $f$ to $U'$ is an isomorphism onto $U$, and the inverse of $\psi'$; finally, by the definition of $X'$, $U'$ is dense in $X'$.
  \item We now show that $g$ is an \emph{immersion}, which will imply that $X'$ is \emph{quasi-projective} over $S$, because $P$ is projective over $S$.
    Set
    \begin{align*}
      V_k&=\vphi_k(U_k)  \quad\mbox{(open subset of $P_k$)}\\
      W_k&=p_k^{-1}(V_k) \quad\mbox{(open subset of $P$)}\\
      U_k'&=f^{-1}(U_k)  \quad\mbox{(open subset of $X'$)}\\
      U_k''&=g^{-1}(W_k) \quad\mbox{(open subset of $X'$)}.
    \end{align*}
    It is clear that the $U'_k$ form an open cover of $X'$; we will first see that the $U_k''$ also form an open cover of $X'$, by showing that $U_k'\subset U_k''$.
    For this, it will suffice to show that the diagram
    \[
    \label{2.5.6.1.5}
      \xymatrix{
        U_k'\ar[r]^{g|U_k'}\ar[d]_{f|U_k'} &
        P\ar[d]^{p_k}\\
        U_k\ar[r]^{\vphi_k} &
        P_k
      }
      \tag{5.6.1.5}
    \]
    commutes.
    But the prescheme $U_k'=h^{-1}(U_k\times_S P)$ is induced by $X'$ on the open subset $U_k'$, and is thus the closure of $Z=U'\subset U_k'$ relative to $U_k'$ \sref[I]{1.9.5.8}.
    To show the commutativity of \sref{2.5.6.1.5}, it thus suffices (since $P_k$ is an $S$-scheme) to show that composing the diagram with the canonical injection $U'\to U_k'$ (or, equivalently, thanks to the isomorphism from $U'$ to $U$, with $\psi$) gives us a commutative diagram \sref[I]{1.9.5.6}.
    But, by definition, the diagram thus obtains is exactly \sref{2.5.6.1.1}, whence our claim.

    The $W_k$ thus form an open cover of $g(X')$; to show that $g$ is an immersion, it suffices to show that each of the restrictions $g|U_k''$ is an immersion into $W_k$ \sref[I]{1.4.2.4}.
    For this, consider the morphism $u_k:W_k\xrightarrow{p_k}V_k\xrightarrow{\vphi_k^{-1}}U_k\to X$; since $X$ is separated over $S$, the graph morphism $\Gamma_{u_k}:W_k\to X\times_S W_k$ is a closed immersion \sref[I]{1.5.4.3}, and so the graph $T_k=\Gamma_{u_k}(W_k)$ is a closed subprescheme of $X\times_S W$;
    if we show that $U'\to X\times_S W_k$ factors through this subprescheme, then the map from the subprescheme induced by $X'$ on the open subset $X_k''$ of $X'$ to $X\times_S W_k$ will also factor through this graph, by \sref[I]{1.9.5.8}.
    Since the restriction of $q_2$ to $T_k$ is an isomorphism onto $W_k$, the restriction of $g$ to $X''_k$ will be an immersion into $W_k$, and our claim will be proven.
    Let $v_k$ be the canonical injection $U'\to X\times_S W_k$; we have to show that there exists a morphism $w_k:U'\to W_k$ such that $v_k=\Gamma_{u_k}\circ w_k$.
    By the definition of the product, it suffices to prove that $q_1\circ v_k=u_k\circ q_2\circ v_k$ \sref[I]{1.3.2.1}, or, by composing on the right
\oldpage[II]{109}
    with the isomorphism $\psi':U\to U'$, that $q_1\circ\psi=u_k\circ q_2\circ\psi$.
    But since $q_1\circ\psi=j$ and $q_2\circ\psi=\vphi$, our claim follows from the commutativity of \sref{2.5.6.1.1}, taking into account the definition of $u_k$.
  \item It is clear that since $U$, and thus $U'$, is irreducible, so is the $X'$ from the preceding construction, and the morphism $f$ is thus \emph{birational} \sref[I]{1.2.2.9}.
    If in addition $X$ is reduced, then so is $U'$, and hence $X'$ is also reduced \sref[I]{1.9.5.9}.
    This finishes the proof.
\end{enumerate}
\end{proof}

\begin{cor}[5.6.2]
\label{2.5.6.2}
Suppose that one of the hypotheses, \emph{(a)} and \emph{(b)}, of \sref{2.5.6.1} is satisfied.
For $X$ to be proper over $S$, it is necessary and sufficient for there to exist a projective scheme $X'$ over $S$, and a surjective $S-$morphism $f:X'\to X$ (which is thus projective, by \sref{2.5.5.5}[v]).
Whenever this is the case, we can further choose $f$ to be such that there exists a dense open subset $U$ of $X$ for which the restriction of $f$ to $f^{-1}(U)$ is an isomorphism $f^{-1}\isoto U$, and for which $f^{-1}(U)$ is dense in $X'$.
If, further, $X$ is irreducible (resp. reduced), then we can assume that $X'$ is also irreducible (resp. reduced); when $X$ and $X'$ are irreducible, $f$ is a birational morphism.
\end{cor}

\begin{proof}
\label{proof-2.5.6.2}
The condition is sufficient, by \sref{2.5.5.3} and \sref{2.5.4.3}[ii].
It is necessary because, with the notation of \sref{2.5.6.1}, if $X$ is proper over $S$, then $X'$ is proper over $S$, because it is projective over $X$, and thus proper over $X$ \sref{2.5.5.3}, and our claim follows from \sref{2.5.4.2}[ii];
furthermore, since $X'$ is quasi-projective over $S$, it is projective over $S$, by \sref{2.5.5.3}.
\end{proof}

\begin{cor}[5.6.3]
\label{2.5.6.3}
Let $S$ be a locally Noetherian prescheme, and $X$ an $S$-scheme of finite type over $S$, with structure morphism $f_0:X\to S$.
For $X$ to be proper over $S$, it is necessary and sufficient that, for every morphism \emph{of finite type} $S'\to S$, $(f_0)_{(S')}:X_{(S')}\to S'$ be a closed morphism.
It even suffices for this condition to be verified only for every $S$-prescheme of the form $S'=S\otimes_\bb{Z}\bb{Z}[T_1,\ldots,T_n]$ (where the $T_i$ are indeterminates).
\end{cor}

\begin{proof}
\label{proof-2.5.6.3}
The condition being clearly necessary, we now show that it is sufficient.
Since the question is local on $S$ and $S'$ \sref{2.5.4.1}, we can suppose that $S$ and $S'$ are affine and Noetherian.
By Chow's lemma, there exists a projective $S$-scheme $P$, an immersion $j:X'\to P$, and a surjective projective morphism $f:X'\to X$, such that the diagram
\[
  \xymatrix{
    X\ar[d]_{f_0} &
    X'\ar[l]_f\ar[d]^j\\
    S &
    P\ar[l]_r
  }
\]
commutes.
Since $P$ is of finite type over $S$, the first hypothesis implies that the projection $q_2:X\times_S P\to P$ is a \emph{closed} morphism.
But the immersion $j$ is the composition of $q_2$ and the morphism $f\times1$ from $X'\times_S P$ to $X\times_S P$;
but $f$, being projective, is proper \sref{2.5.5.3}, and so $f\times1$ is closed.
We thus conclude that $j$ is a closed immersion, and thus proper \sref{2.5.4.2}[i].
Furthermore, the structure morphism $r:P\to S$ is projective, and thus proper \sref{2.5.5.3}, so $f_0\circ f=r\circ j$ is proper \sref{2.5.4.2}[ii];
finally, since $f$ is surjective, $f_0$ is proper, by \sref{2.5.4.3}.

To prove the proposition using only the second, weaker hypothesis (where $S'$ is of the form $S\otimes_\bb{Z}\bb{Z}[T_1,\ldots,T_n]$), it suffices to show that it implies the first.
But, if $S'$ is affine and of finite type over $S=\Spec(A)$,
\oldpage[II]{110}
then we have $S'=\Spec(A[c_1,\ldots,c_n])$ \sref[I]{1.6.3.3}, and $S'$ is thus isomorphic to a closed subprescheme of $S''=\Spec(A[T_1,\ldots,T_n])$ (where the $T_i$ are indeterminates).
In the commutative diagram
\[
  \xymatrix{
    X\times_S S'\ar[r]^{1_X\times j}\ar[d]_{(f_0)_{(S')}} &
    X\times_S S''\ar[d]^{(f_0)_{(S'')}}\\
    S'\ar[r]^j &
    S''
  }
\]
both $j$ and $1_X\times j$ are closed immersions \sref[I]{1.4.3.1}, and $(f_0)_{(S')}$ is closed by hypothesis; thus $(f_0)_{(S'')}$ is also closed.
\end{proof}

\section{Integral morphisms and finite morphisms}
\label{section-integral-finite-morphisms}


\section{Valuative criteria}
\label{section:II.7}

In this section, we give valuative criteria for separation and properness for a given morphism, that is, criteria which introduce a variable auxiliary scheme of the form $\Spec(A)$, where $A$ is a valuation ring.
Under certain suitable ``Noetherian'' hypotheses, we can refine our criteria and restrict to the case where $A$ is a \emph{discrete} valuation ring.
This will be the only case that we need to concern ourselves with in all that follows, and we introduce arbitrary valuation rings, in the general case, only to discuss the links with the classical study of such objects.


\subsection{Reminder on valuation rings}
\label{subsection:II.7.1}

\begin{env}[7.1.1]
\label{II.7.1.1}
Amongst the many diverse equivalent properties that characterise valuation rings, we will use the following: a ring $A$ is said to be a \emph{valuation ring} if it is an integral ring which is not a field, and $A$ is \emph{maximal} in the set of local rings strictly contained in the field of fractions $K$ of $A$ under the domination relation \sref[I]{I.8.1.1}.
Recall that a valuation ring is \emph{integrally closed}.
If $A$ is a valuation ring, then so too is $A_\mathfrak{p}$ for any prime ideal $\mathfrak{p}\neq0$ of $A$.
\end{env}

\begin{env}[7.1.2]
\label{II.7.1.2}
Let $K$ be a field, and $A$ a local subring of $K$ that is not a field;
\oldpage[II]{139}
then there exists a valuation ring that both dominates $A$ and has $K$ as its field of fractions (\cite[p.~1-07, lemma~2]{I-1}).

Now let $B$ be a valuation ring, $k$ its residue field, $K$ its field of fractions, and $L$ an extension of $k$.
Then there exists a \emph{complete} valuation ring $C$ that dominates $B$ and whose residue field is $L$.
Indeed, $L$ is the algebraic extension of a pure transcendental extension $L'=k(T_\mu)_{\mu\in M}$;
we know that we can extend the valuation of $B$ corresponding to $B$ to a valuation of $K'=K(T_\mu)_{\mu\in M}$ in such a way that $L'$ is the residue field of this valuation (\cite[p.~98]{II-24});
replacing $B$ by the completion of the ring of this extended valuation, we see that that we can restrict to the case where $B$ is complete and $L$ is an algebraic closure of $k$.
If $\overline{K}$ is an algebraic closure of $K$, we can then extend the valuation that defines $B$ to $\overline{K}$, and the corresponding residue field is an algebraic closure of $k$, as we can see by lifting to $\overline{K}$ the coefficients of a unitary polynomial of $k[T]$.
We are thus finally led to the case where $L=k$, and it then suffices to take $C$ to be the completion of $B$ in order to satisfy our claim.
\end{env}

\begin{env}[7.1.3]
\label{II.7.1.3}
Let $K$ be a field, and $A$ a subring of $K$;
the integral closure $A'$ of $A$ in $K$ is the intersection of the valuation rings that contain $A$ and have $K$ as their field of fractions (\cite[p.~51, th.~2]{I-11}).
Proposition~\sref{II.7.1.2} can then be expressed geometrically in an equivalent form:
\end{env}

\begin{proposition}[7.1.4]
\label{II.7.1.4}
Let $Y$ be a prescheme, $p:X\to Y$ a morphism, $x$ a point of $X$, $y=p(x)$, and $y'\neq y$ a specialisation \sref[0]{0.2.1.2} of $y$.
Then there exists a local scheme $Y'$ which is the spectrum of some valuation ring, and a separated morphism $f:Y'\to Y$ such that, denoting the unique closed point of $Y'$ by $a$ and the generic point of $Y'$ by $b$, we have $f(a)=y'$ and $f(b)=y$.
We can furthermore suppose that one of the two additional following properties are satisfied:
\begin{enumerate}
    \item[\rm{(i)}] $Y'$ is the spectrum of a complete valuation ring whose residue field is algebraically closed, and there exists a $\kres(y)$-homomorphism $\kres(x)\to\kres(b)$.
    \item[\rm{(ii)}] There exists a $\kres(y)$-isomorphism $\kres(x)\xrightarrow{\sim}\kres(b)$.
\end{enumerate}
\end{proposition}

\begin{proof}
\label{proof-II.7.1.4}
Let $Y_1$ be the reduced closed subprescheme of $Y$ that has $\overline\{y\}$ as its underlying space \sref[I]{I.5.2.1}, and let $X_1$ be the closed subprescheme given by the inverse image $p^{-1}(Y_1)$;
since $y'\in\overline{\{y\}}$ by hypothesis, and since $\kres(x)$ is the same in $X$ and in $X_1$, we can assume that $Y$ is \emph{integral}, with generic point $y$;
$\sh{O}_{y'}$ is then an integral local ring that is not a field, and whose field of fractions is $\sh{O}_y=\kres(y)$, and $\kres(x)$ is then an extension of $\kres(y)$.
To satisfy the conditions $f(a)=y'$ and $f(b)=y$ as well as the additional condition (i) (resp. (ii)), we take $Y'=\Spec(A')$, where $A'$ is a valuation ring that dominates $\sh{O}_{y'}$ (resp. a valuation ring that dominates $\sh{O}_{y'}$ and whose field of fractions is $\kres(x)$);
the existence such an of $A'$ is guaranteed by \sref{II.7.1.2}.
\end{proof}

\begin{env}[7.1.5]
\label{II.7.1.5}
Recall that a local ring $A$ is said to be \emph{of dimension 1} if there exists a prime ideal distinct from the maximal ideal $\mathfrak{m}$, and if every prime ideal of $A$ distinct from $\mathfrak{m}$ is a \emph{minimal} prime ideal;
when $A$ is \emph{integral}, it is equivalent to ask that $\mathfrak{m}$ and $(0)$ be the only prime ideals, with $\mathfrak{m}\neq(0)$;
in other words, $Y=\Spec(A)$ consists of two
\oldpage[II]{140}
points $a$ and $b$: $a$ is the unique \emph{closed} point, we have $\mathfrak{j}_a=\mathfrak{m}$, and $\kres(a)=k$ is the \emph{residue field} $k=A/\mathfrak{m}$;
$b$ is the \emph{generic point} of $Y$, $\mathfrak{j}_b=(0)$, with the set $\{b\}$ being the unique open subset of $Y$ distinct from both $\emp$ and $Y$ (an open subset which is thus \emph{everywhere dense}), and $\kres(b)=K$ is the \emph{field of fractions} of $A$.
\end{env}

\begin{env}[7.1.6]
\label{II.7.1.6}
For a local ring $A$, Noetherian and of dimension 1, we know (\cite[pp.~2-08 and 17-01]{I-1}) that the following conditions are equivalent:
\begin{enumerate}
    \item[(a)] $A$ is normal;
    \item[(b)] $A$ is regular;
    \item[(c)] $A$ is a valuation ring;
\end{enumerate}
furthermore, $A$ is then a \emph{discrete valuation ring}.
Propositions~\sref{II.7.1.2} and \sref{II.7.1.3} then have the following analogues for discrete valuation rings:
\end{env}

\begin{proposition}[7.1.7]
\label{II.7.1.7}
Let $A$ be an integral local Noetherian ring that is not a field, $K$ its field of fractions, and $L$ an extension of finite type of $K$;
then there exists a discrete valuation ring that dominates $A$ and has $L$ as its field of fractions.
\end{proposition}

\begin{proof}
\label{proof-II.7.1.7}
Suppose first of all that $L=K$.
Let $\mathfrak{m}$ be the maximal ideal of $A$, $(x_1,\ldots,x_n)$ a system of non-null generators of $\mathfrak{m}$, and $B$ the subring $A[x_2/x_1,\ldots,x_n/x_1]$ of $K$, which is Noetherian.
It is immediate that the ideal $\mathfrak{m}B$ of $B$ is identical to the principal ideal $x_1B$;
if $\mathfrak{p}$ is a minimal prime ideal of $x_1B$, then $\mathfrak{p}$ is of rank 1 (\cite[t.~I, p.~277]{I-13});
in other words, $B_\mathfrak{p}$ is a local Noetherian ring \emph{of dimension 1};
it is clear that $\mathfrak{p}B_\mathfrak{p}\cap A$ is an ideal of $A$ that contains $\mathfrak{m}$ and that does not contain $1$, and is thus equal to $\mathfrak{m}$, and so $B_\mathfrak{p}$ \emph{dominates} $A$ \sref[I]{I.8.1.1}.
It follows from the Krull-Akizuki Theorem (\cite[p.~293]{II-25}) that the integral closure $C$ of $B_\mathfrak{p}$ is a Noetherian ring (even though $C$ is not necessarily a $B_\mathfrak{p}$-module of finite type);
if $\mathfrak{n}$ is a maximal ideal of $C$, then $C_\mathfrak{n}$ is a normal local Noetherian ring of dimension 1 (\cite[p.~295]{II-25}), and thus a discrete valuation ring that dominates $B_\mathfrak{p}$ and \emph{a fortiori} $A$.

Now, if $L$ is an extension of finite type of $K$, we can, by the above, restrict to the case where $A$ is already a discrete valuation ring.
Let $w$ be a valuation of $K$ associated to $A$;
there exists a discrete valuation $w'$ of $L$ that \emph{extends} $w$: we can restrict, by induction on the number of generators of $L$, to the case where $L=K(\alpha)$, and then the proposition is classical (\cite[p.~106]{II-24}).
\end{proof}

\begin{corollary}[7.1.8]
\label{II.7.1.8}
Let $A$ be a Noetherian integral ring, $K$ its field of fractions, and $L$ an extension of finite type of $K$.
Then the integral closure of $A$ in $L$ is the intersection of the discrete valuation rings that have $L$ as their field of fractions and that contain $A$.
\end{corollary}

\begin{proof}
\label{proof-II.7.1.8}
Indeed, such a discrete valuation ring, being normal, contains \emph{a fortiori} every element of $L$ that is integral over $A$.
It thus suffices to prove that, if $x\in L$ is not integral over $A$, then there exists a discrete valuation ring $C$ that has $L$ as its field of fractions, contains $A$, and does not contain $x$.
The hypothesis on $x$ implies that $x\not\in B=A[1/x]$, or, in other words, that $1/x$ is not invertible in the Noetherian ring $B$.
There is thus a prime ideal $\mathfrak{p}$ of $B$ that contains $1/x$.
The integral local ring $B_\mathfrak{p}$ is Noetherian and contained in $L$, which is an extension  of finite type of the field of fractions of $B_\mathfrak{p}$ (with the latter containing $K$).
By \sref{II.7.1.7}, there thus exists a discrete valuation ring $C$ that dominates $B_\mathfrak{p}$ and has $L$ as its field of fractions;
since $1/x\in\mathfrak{p}B_\mathfrak{p}$ belongs to the maximal ideal of $C$, we have that $x\not\in C$, which concludes the proof.
\end{proof}

The geometric form of \sref{II.7.1.7} is the following:

\oldpage[II]{141}
\begin{proposition}[7.1.9]
\label{II.7.1.9}
Let $Y$ be a locally Noetherian prescheme, $p:X\to Y$ a morphism of locally finite type, $x$ a point of $X$, $y=p(x)$, and $y'\neq y$ a specialisation of $y$.
Then there exists a local scheme $Y'$, spectrum of a discrete valuation ring, a separated morphism $f:Y'\to Y$, and a rational $Y$-map $g$ from $Y'$ to $X$, such that, denoting the closed point of $Y'$ by $a$, and the generic point of $Y'$ by $b$, we have $f(a)=y'$, $f(b)=y$, $g(b)=x$, and such that, in the commutative diagram
\[
    \xymatrix{
        & \kres(x) \ar[dl]_{\gamma}
    \\  \kres(b)
        & \kres(y) \ar[u]_{\pi} \ar[l]^{\varphi}
    }
\]
(where $\pi$, $\varphi$, and $\gamma$ are the homomorphisms corresponding to $p$, $f$, and $g$, respectively) the morphism $\gamma$ is a bijection.
\end{proposition}


% \subsection{Valuative criterion for separatedness}
% \label{subsection:II.7.2}


% \subsection{Valuative criterion for properness}
% \label{subsection:II.7.3}


% \subsection{Algebraic curves and function fields of dimension 1}
% \label{subsection:II.7.4}

\section{Blowup schemes; projective cones; projective closure}
\label{section:2.8}


\subsection{Blowup preschemes}
\label{subsection:2.8.1}

\begin{env}[8.1.1]
\label{2.8.1.1}
Let $Y$ be a prescheme, and, for every integer $n\geq 0$, let $\sh{I}_n$ be a quasi-coherent sheaf of ideals of $\sh{O}_Y$; suppose that the following conditions are satisfied:
\[
\label{eq:2.8.1.1.1}
  \sh{I}_0=\sh{O}_Y,\ \sh{I}_n\subset\sh{I}_m\text{ for }m\leq n,
\tag{8.1.1.1}
\]
\[
\label{eq:2.8.1.1.2}
  \sh{I}_m\sh{I}_n\subset\sh{I}_{m+n}\text{ for any }m,n.
\tag{8.1.1.2}
\]

\oldpage[II]{153}
We note that these hypotheses imply
\[
\label{eq:2.8.1.1.3}
  \sh{I}_1^n\subset\sh{I}_n.
\tag{8.1.1.3}
\]

Set
\[
\label{eq:2.8.1.1.4}
  \sh{S}=\bigoplus_{n\geq 0}\sh{I}_n.
\tag{8.1.1.4}
\]

It follows from \eref{eq:2.8.1.1.1} and \eref{eq:2.8.1.1.2} that $\sh{S}$ is a quasi-coherent graded $\sh{O}_Y$-algebra, and thus defines a $Y$-scheme $X=\Proj(\sh{S})$.
If $\sh{J}$ is an \emph{invertible} sheaf of ideals of $\sh{O}_Y$, then $\sh{I}_n\otimes_{\sh{O}_Y}\sh{J}^{\otimes n}$ is canonically identified with $\sh{I}_n\sh{J}^n$.
If we then replace the $\sh{I}_n$ by the $\sh{I}_n\sh{J}^n$, and, in doing so, replace $\sh{S}$ by a quasi-coherent $\sh{O}_Y$-algebra $\sh{S}_{(\sh{J})}$, then $X_{(\sh{J})}=\Proj(\sh{S}_{(\sh{J})})$ is canonically isomorphic to $X$ \sref{2.3.1.8}.
\end{env}

\begin{env}[8.1.2]
\label{2.8.1.2}
Suppose that $Y$ is \emph{locally integral}, so that the sheaf $\sh{R}(Y)$ of rational functions is a quasi-coherent $\sh{O}_Y$-algebra \sref[1]{1.7.3.7}.
We say that a sub-$\sh{O}_Y$-module $\sh{I}$ of $\sh{R}(Y)$ is a \emph{fractional ideal} of $\sh{R}(Y)$ if it is of \emph{finite type} \sref[0]{0.5.2.1}.
Suppose we have, for all $n\geq0$, a quasi-coherent fractional ideal $\sh{I}_n$ of $\sh{R}(Y)$, such that $\sh{I}_0 = \sh{O}_Y$, and such that condition \eref{eq:2.8.1.1.2} (but not necessarily the second condition \eref{eq:2.8.1.1.1}) is satisfied;
we can then again define a graded quasi-coherent $\sh{O}_Y$-algebra by Equation \eref{eq:2.8.1.1.4}, and the corresponding $Y$-scheme $X = \Proj(\sh{S})$;
we will again have a canonical isomorphism from $X$ to $X_{{\sh{J}}}$ for every \emph{invertible} fractional ideal $\sh{J}$ of $\sh{R}(Y)$.
\end{env}

\begin{definition}[8.1.3]
\label{2.8.1.3}
Let $Y$ be a prescheme (resp. a locally integral prescheme), and $\sh{I}$ a quasi-coherent ideal of $\sh{O}_Y$ (resp. a quasi-coherent fractional ideal of $\sh{R}(Y)$).
We say that the $Y$-scheme $X = \Proj(\bigoplus_{n\geq0}\sh{I}^n)$ is obtained by blowing up the ideal $\sh{I}$, or is the blow-up prescheme of $Y$ relative to $\sh{I}$.
When $\sh{I}$ is a quasi-coherent ideal of $\sh{O}_Y$, and $Y'$ is the closed subprescheme of $Y$ defined by $\sh{I}$, we also say that $X$ is the $Y$-scheme obtained by blowing up $Y'$.
\end{definition}

By definition, $\sh{S} = \bigoplus_{n\geq0}\sh{I}^n$ is then generated by $\sh{S}_1 = \sh{I}$;
if $\sh{I}$ is an $\sh{O}_Y$-module of \emph{finite type}, then $X$ is \emph{projective} over $Y$ \sref{2.5.5.2}.
Without any hypotheses on $\sh{I}$, the $\sh{O}_X$-module $\sh{O}_X(1)$ is \emph{invertible} \sref{2.3.2.5} and \emph{very ample}, by \sref{2.4.4.3} applied to the structure morphism $X\to Y$.

We note that, if $j:X\to Y$ is the structure morphism, then the restriction of $f$ to $f^{-1}(Y\setminus Y')$ is an \emph{isomorphism} to $Y\setminus Y'$ whenever $\sh{I}$ is an \emph{ideal of $\sh{O}_Y$} and $Y'$ is the closed subprescheme that it defines: indeed, the question being local on $Y$, it suffices to assume that $\sh{I} = \sh{O}_Y$, and our claim then follows from \sref{2.3.1.7}.

If we replace $\sh{I}$ by $\sh{I}^d$ ($d>0$), then the blow-up $Y$-scheme $X$ is replaced by a canonically isomorphic $Y$-scheme $X'$ \sref{2.8.1.1};
similarly, for every \emph{invertible} ideal (resp. \emph{invertible} fractional ideal) $\sh{J}$, the blow-up prescheme $X_{(\sh{J})}$ relative to the ideal $\sh{I}\sh{J}$ is canonically isomorphic to $X$ \sref{2.8.1.1}.

In particular, whenever $\sh{I}$ is an \emph{invertible} ideal (resp. \emph{invertible} fractional ideal), the $Y$-scheme obtained by blowing up $\sh{I}$ is \emph{isomorphic to $Y$} \sref{2.3.1.7}.

\begin{proposition}[8.1.3]
\label{2.8.1.4}
Let $Y$ be an integral prescheme.
\begin{enumerate}
    \item[\rm{(i)}] For every sequence $(\sh{I}_n)$ of quasi-coherent fractional ideals of $\sh{R}(Y)$ that satisfies \eref{eq:2.8.1.1.2}
\oldpage[II]{154}
        and such that $\sh{I}_0 = \sh{O}_Y$, the $Y$-scheme $X=\Proj(\bigoplus_{n\geq0}\sh{I}^n)$ is integral, and the structure morphism $f:X\to Y$ is dominant.
    \item[\rm{(ii)}] Let $\sh{I}$ be a quasi-coherent fractional ideal of $\sh{R}(Y)$, and let $X$ be the $Y$-scheme given by the blow up of $Y$ relative to $\sh{I}$.
        If $\sh{I} \neq 0$, then the structure morphism $f:X\to Y$ is then birational and surjective.
\end{enumerate}
\end{proposition}

\begin{proof}
\label{proof-2.8.1.4}
\begin{enumerate}
    \item[\rm{(i)}] This follows from the fact that $\sh{S} = \bigoplus_{n\geq0}\sh{I}_n$ is an \emph{integral} $\sh{O}_Y$-algebra (\sref{2.3.1.12} and \sref{2.3.1.14}), since, for all $y\in Y$, $\sh{O}_y$ is an integral ring \sref[I]{1.5.1.4}.
    \item[\rm{(ii)}] By (i), $X$ is integral;
        if, furthermore, $x$ and $y$ are the generic points of $X$ and $Y$ (respectively), then we have $f(x) = y$, and it remains to show that $\kres(x)$ is of rank 1 over $\kres(y)$.
        But $x$ is also the generic point of the fibre $f^{-1}(y)$;
        if $\psi$ is the canonical morphism $Z\to Y$, where $Z=\Spec(\kres(y))$, then the prescheme $f^{-1}(y)$ can be identified with $\Proj(\sh{S}')$, where $\sh{S}' = \psi^*(\sh{S})$ \sref{2.3.5.3}.
        But it is clear that $\sh{S}' = \bigoplus_{n\geq0}(\sh{I}_y)^n$, and, since $\sh{I}$ is a quasi-coherent fractional ideal of $\sh{R}(Y)$ that is not zero, $\sh{I}_y \neq 0$ \sref[I]{1.7.3.6}, whence $\sh{I}_y = \kres(y)$;
        then $\Proj(\sh{S}')$ can be identified with $\Spec(\kres(y))$ \sref{2.3.1.7}, whence the conclusion.
\end{enumerate}
\end{proof}

We show a \emph{converse} of \sref{2.8.1.4} in \sref[III]{3.2.3.8}.

\begin{env}[8.1.5]
\label{2.8.1.5}
We return to the setting and notation of \sref{2.8.1.1}.
By definition, the injection homomorphisms $\sh{I}_{n+1}\to\sh{I}_n$ \eref{eq:2.8.1.1.1} define, for every $k\in\bb{Z}$, an injective homomorphism of degree zero of graded $\sh{S}$-modules
\[
\label{eq:2.8.1.5.1}
  u_k: \sh{S}_+(k+1) \to \sh{S}(k);
\tag{8.1.5.1}
\]
since $\sh{S}_+(k+1)$ and $\sh{S}(k+1)$ are canonically \textbf{(TN)}-isomorphic, they give a canonical correspondence between $u_k$ and an injective homomorphism of $\sh{O}_X$-modules \sref{2.3.4.2}:
\[
\label{eq:2.8.1.5.2}
  \widetilde{u}_k: \sh{O}_X(k+1) \to \sh{O}_X(k).
\tag{8.1.5.2}
\]

Recall as well \sref{2.3.2.6} that we have defined canonical homomorphisms
\[
\label{eq:2.8.1.5.3}
  \lambda: \sh{O}_X(h) \otimes_{\sh{O}_X} \sh{O}_X(k) \to \sh{O}_X(h+k)
\tag{8.1.5.3}
\]
and, since the diagram
\[
  \xymatrix{
    \sh{S}(h) \otimes_{\sh{S}} \sh{S}(k) \otimes_{\sh{S}} \sh{S}(l)
      \ar[r]
      \ar[d]
  & \sh{S}(h+k) \otimes_{\sh{S}} \sh{S}(l)
      \ar[d]
  \\\sh{S}(h) \otimes_{\sh{S}} \sh{S}(k+l)
      \ar[r]
  & \sh{S}(h+k+l)
  }
\]
commutes, it follows from the functoriality of the $\lambda$ \sref{2.3.2.6} that the homomorphisms \eref{eq:2.8.1.5.3} define the structure of a \emph{graded quasi-coherent $\sh{O}_X$-algebra} on
\[
\label{eq:2.8.1.5.4}
  \sh{S}_X = \bigoplus_{n\in\bb{Z}}\sh{O}_X(n).
\tag{8.1.5.4}
\]
Furthermore, the diagram
\[
  \xymatrix{
    \sh{S}(h) \otimes_{\sh{S}} \sh{S}(k+1)
      \ar[r]
      \ar[d]_{1\otimes u_k}
  & \sh{S}(h+k+1)
      \ar[d]^{u_{k+h}}
  \\\sh{S}(h) \otimes_{\sh{S}} \sh{S}(k)
      \ar[r]
  & \sh{S}(h+k)
  }
\]
commutes; the functoriality of the $\lambda$ then implies that we have a commutative diagram
\[
  \xymatrix{
    \sh{O}_X(h) \otimes_{\sh{O}_X} \sh{O}_X(k+1)
      \ar[r]^{\lambda}
      \ar[d]_{1\otimes \widetilde{u}_k}
  & \sh{O}_X(h+k+1)
      \ar[d]^{\widetilde{u}_{k+h}}
  \\\sh{O}_X(h) \otimes_{\sh{O}_X} \sh{O}_X(k)
      \ar[r]^{\lambda}
  & \sh{O}_X(h+k)
  }
\]
where the horizontal arrows are the canonical homomorphisms.
We can thus say that the $\widetilde{u}_k$ define an \emph{injective homomorphism} (of degree zero) \emph{of graded $\sh{S}_X$-modules}
\[
\label{eq:2.8.1.5.6}
  \widetilde{u}: \sh{S}_X(1) \to \sh{S}_X.
\tag{8.1.5.6}
\]
\end{env}

\begin{env}[8.1.6]
\label{2.8.1.6}
Keeping the notation from \sref{2.8.1.5}, we now note that, for $n\geq0$, the composite homomorphism $\widetilde{v}_n = \widetilde{u}_{n-1} \circ \widetilde{u}_{n-2} \circ \ldots \circ \widetilde{u}_0$ is an \emph{injective} homomorphism $\sh{O}_X(n) \to \sh{O}_X$;
we denote by $\sh{I}_{n,X}$ its image, which is thus a quasi-coherent ideal of $\sh{O}_X$, \emph{isomorphic} to $\sh{O}_X(n)$.
Furthermore, the diagram
\[
  \xymatrix{
    \sh{O}_X(m) \otimes_{\sh{O}_X} \sh{O}_X(n)
      \ar[r]^{\lambda}
      \ar[d]_{\widetilde{v}_m \otimes \widetilde{v}_n}
  & \sh{O}_X(m+n)
      \ar[d]^{\widetilde{v}_{m+n}}
  \\\sh{O}_X
      \ar[r]_{\id}
  & \sh{O}_X
  }
\]
commutes for $m\geq0$, $n\geq0$.
We thus deduce the following inclusions:
\[
\label{eq:2.8.1.6.1}
  \sh{I}_{0,X} = \sh{O}_X, \quad \sh{I}_{n,X} \subset \sh{I}_{m,X}
  \qquad\mbox{for $0\leq m\leq n$;}
\tag{8.1.6.1}
\]
\[
\label{eq:2.8.1.6.2}
  \sh{I}_{m,X}\sh{I}_{n,X} \subset \sh{I}_{m+n,X}
  \qquad\qquad\mbox{for $m\geq0$, $n\geq0$.}
\tag{8.1.6.2}
\]
\end{env}

\oldpage[II]{156}

\begin{proposition}[8.1.7]
\label{2.8.1.7}
Let $Y$ be a prescheme, $\sh{I}$ a quasi-coherent ideal of $\sh{O}_Y$, and $X = \Proj(\bigoplus_{n\geq0}\sh{I}^n)$ the $Y$-scheme given by blowing up $\sh{I}$.
We then have, for all $n>0$, a canonical isomorphism
\[
\label{eq:2.8.1.7.1}
  \sh{O}_X(n) \xrightarrow{\sim} \sh{I}^n\sh{O}_X = \sh{I}_{n,X}
\tag{8.1.7.1}
\]
(cf. \sref[0]{0.4.3.5}), and thus that $\sh{I}^n\sh{O}_X$ is a very-ample invertible $\sh{O}_X$-module if $n>0$.
\end{proposition}

\begin{proof}
\label{proof-2.8.1.7}
The last claim is immediate, since $\sh{O}_X(1)$ is invertible \sref{2.3.2.5} and very ample for $Y$ by definition (\sref{2.4.4.3} and \sref{2.4.4.9}).
Also by definition, the image of $v_n$ is exactly $\sh{I}^n\sh{S}$, and \eref{eq:2.8.1.7.1} then follows from the exactness of the functor $\widetilde{\sh{M}}$ \sref{2.3.2.4} and from Equation \eref{eq:2.3.2.4.1}.
\end{proof}

\begin{corollary}[8.1.8]
\label{2.8.1.8}
Under the hypotheses of \sref{2.8.1.7}, if $f:X\to Y$ is the structure morphism, and $Y'$ the closed subprescheme of $Y$ defined by $\sh{I}$, then the closed subprescheme $X' = f^{-1}(Y')$ of $X$ is defined by $\sh{I}\sh{O}_X$ (which is canonically isomorphic to $\sh{O}_X(1)$), from which we obtain a canonical short exact sequence
\[
\label{eq:2.8.1.8.1}
  0 \to \sh{O}_X(1) \to \sh{O}_X \to \sh{O}_{X'} \to 0.
\tag{8.1.8.1}
\]
\end{corollary}

\begin{proof}
\label{proof-2.8.1.8}
This follows from \eref{2.8.1.7.1} and from \sref[I]{1.4.4.5}.
\end{proof}

\begin{env}[8.1.9]
\label{2.8.1.9}
Under the hypotheses of \sref{2.8.1.7}, we can be more precise about the structure of the $\sh{I}_{n,X}$.
Note that the homomorphism
\[
  \widetilde{u}_{-1}: \sh{O}_X \to \sh{O}_X(-1)
\]
canonically corresponds to a section $s$ of $\sh{O}_X(-1)$ over $X$, which we call the \emph{canonical section} (relative to $\sh{I}$) \sref[0]{0.5.1.1}.
In Diagram \eref{eq:2.8.1.5.5}, the horizontal arrows are isomorphisms \sref{2.3.2.7}; by replacing $h$ with $k$, and $k$ with $-1$ in this diagram, we obtain that $\widetilde{u}_k = 1_k \otimes \widetilde{u}_{-1}$ (where $1_k$ denotes the identity on $\sh{O}_X(h)$), or, equivalently, that the homomorphism $\widetilde{u}_k$ is given exactly by \emph{tensoring with the canonical section $s$} (for all $k\in\bb{Z}$).
The homomorphism $\widetilde{u}$ \eref{eq:2.8.1.5.6} can then be understood in the same way.

Thus, for all $n\geq0$, the homomorphism $\widetilde{v}_n: \sh{O}_X(n)\to\sh{O}_X$ is given exactly by tensoring with $s^{\otimes n}$;
we thus deduce:
\end{env}

\begin{corollary}[8.1.10]
\label{2.8.1.10}
With the notation of \sref{2.8.1.8}, the underlying space of $X'$ is the set of $x\in X$ such that $s(x)=0$, where $s$ denotes the canonical section of $\sh{O}_X(-1)$.
\end{corollary}

\begin{proof}
\label{proof-2.8.1.10}
Indeed, if $c_x$ is a generator of the fibre $(\sh{O}_X(1))_x$ at a point $x$, then $s_x\otimes c_x$ is canonically identified with a generator of the fibre of $\sh{I}_{1,X}$ at the point $x$, and is thus invertible if and only if $s_x\not\in\fk{m}_x(\sh{O}_x(-1))_x$, or, equivalently, if and only if $s(x)\neq0$.
\end{proof}

\begin{proposition}
\label{2.8.1.11}
Let $Y$ be an integral prescheme, $\sh{I}$ a quasi-coherent fractional ideal of $\sh{R}(Y)$, and $X$ the $Y$-scheme given by blowing up $\sh{I}$.
Then $\sh{I}\sh{O}_X$ is an invertible $\sh{O}_X$-module that is very ample for $Y$.
\end{proposition}

\begin{proof}
\label{proof-2.8.1.11}
The question being local on $Y$ \sref{2.4.4.5}, we can reduce to the case where $Y=\Spec(A)$, with $A$ some integral ring of ring of fractions $K$, and $\sh{I}=\widetilde{\fk{I}}$, with $\fk{I}$ some fractional ideal of $K$;
there then exists an element $a\neq0$ of $A$ such that $a\fk{I}\subset A$.
Let $S = \bigoplus_{n\geq0}\fk{I}^n$;
the map $x\mapsto ax$ is an $A$-isomorphism from $\fk{I}^{n+1} = (S(1))_n$ to $a\fk{I}^{n+1} = a\fk{I}S_n \subset \fk{I}^n = S_n$,
\oldpage[II]{157}
and thus defines a (TN)-isomorphism of degree zero of graded $S$-modules $S_+(1)\to a\fk{I}S$.
On the other hand, $x\mapsto a^{-1}x$ is an isomorphism of degree zero of graded $S$-modules $a\fk{I}S \xrightarrow{\sim} \fk{I}S$.
We thus obtain, by composition \sref{2.3.2.4}, an isomorphism of $\sh{O}_X$-modules $\sh{O}_X(1) \xrightarrow{\sim} \sh{I}\sh{O}_X$, and, since $S$ is generated by $S_1=\fk{I}$, $\sh{O}_X(1)$ is invertible \sref{2.3.2.5} and very ample (\sref{2.4.4.3} and \sref{2.4.4.9}), whence our claim.
\end{proof}


\subsection{Preliminary results on the localisation of graded rings}
\label{subsection:2.8.2}

\begin{env}[8.2.1]
\label{2.8.2.1}
Let $S$ be a graded ring, but not assumed (for the moment) to be only in positive degree.
We define
\[
\label{eq:2.8.2.1.1}
  S^\geq = \bigoplus_{n\geq0} S_n,
  \qquad
  S^\leq = \bigoplus_{n\leq0} S_n
\tag{8.2.1.1}
\]
which are both graded subrings of $S$, in only positive and negative degrees (respectively).
If $f$ is a homogeneous elements of degree $d$ (positive or negative) of $S$, then the ring of fraction $S_f=S'$ is again endowed with the structure of a graded ring, by taking $S'_n$ ($n\in\bb{Z}$) to be the set of the $x/f^k$ for $x\in S_{n+kd}$ ($k\geq0$);
we define $S_{(f)}=S'_0$, and will write $S_f^\geq$ and $S_f^\leq$ for $S^{'\geq}$ and $S^{'\leq}$ (respectively).
If $d>0$, then
\[
\label{eq:2.8.2.1.2}
  (S^\geq)_f = S_f
\tag{8.2.1.2}
\]
since, if $x\in S_{n+kd}$ with $n+kd<0$, then we can write $x/f^k = xf^h/f^{h+k}$, and we also have that $n+(h+k)d>0$ for $h$ sufficiently large and $>0$.
We thus conclude, by definition, that
\[
\label{eq:2.8.2.1.3}
  (S^\geq)_{(f)} = (S_f^\geq)_0 = S_{(f)}.
\tag{8.2.1.3}
\]

If $M$ is a graded $S$-module, then we similarly define
\[
\label{eq:2.8.2.1.4}
  M^\geq = \bigoplus_{n\geq0} M_n,
  \qquad
  M^\leq = \bigoplus_{n\leq0} M_n
\tag{8.2.1.4}
\]
which are (respectively) a graded $S^\geq$-module and a graded $S^\leq$-module, and their intersection is the $S_0$ module $M_0$.
If $f\in S_d$, then we define $M_f$ to be the graded $S_f$-module whose elements of degree $n$ are the $z/f^k$ for $z\in M_{n+kd}$ ($k\geq0$);
we denote by $M_{(f)}$ the set of elements of degree zero of $M_f$, and this is an $S_{(f)}$-module, and we will write $M_f^\geq$ and $M_f^\leq$ to mean $(M_f)^\geq$ and $(M_f)^\leq$ (respectively).
If $d>0$, then we see, as above, that
\[
\label{eq:2.8.2.1.5}
  (M^\geq)_f = M_f
\tag{8.2.1.5}
\]
and
\[
\label{eq:2.8.2.1.6}
  (M^\geq)_{(f)} = (M_f^\geq)_0 = M_{(f)}.
\tag{8.2.1.6}
\]
\end{env}

\begin{env}[8.2.2]
\label{2.8.2.2}
Let $\bb{z}$ be an indeterminate, we we will call the \emph{homogenisation variable}.
If $S$ is a graded ring (in positive or negative degrees), then the polynomial algebra\footnote{This should not be confused with the use of the notation $\widehat{S}$ to denote the completed separation of a ring.}
\[
\label{eq:2.8.2.2.1}
  \widehat{S} = S[\bb{z}]
\tag{8.2.2.1}
\]
\oldpage[II]{158}
is a graded $S$-algebra, where we define the degree of $f\bb{z}^n$ ($n\geq0$), with $f$ homogeneous, as
\[
\label{eq:2.8.2.2.2}
  \deg(f\bb{z}^n) = n+\deg f.
\tag{8.2.2.2}
\]
\end{env}

\begin{lemma}[8.2.3]
\label{2.8.2.3}
\begin{enumerate}
  \item[\rm{(i)}] There are canonical isomorphisms of (non-graded) rings
    \[
    \label{eq:2.8.2.3.1}
      \widehat{S}_{(\bb{z})}
      \xrightarrow{\sim}
      \widehat{S}/(\bb{z}-1)\widehat{S}
      \xrightarrow{\sim}
      S.
    \tag{8.2.3.1}
    \]
  \item[\rm{(ii)}] There is a canonical isomorphism of (non-graded) rings
    \[
    \label{eq:2.8.2.3.2}
      \widehat{S}_{(f)} \xrightarrow{\sim} S_f^\leq
    \tag{8.2.3.2}
    \]
    for all $f\in S_d$ with $d>0$.
\end{enumerate}
\end{lemma}

\begin{proof}
\label{proof-2.8.2.3}
The first of the isomorphisms in \eref{eq:2.8.2.3.1} was defined in \sref{2.2.2.5}, and the second is trivial;
the isomorphism $\widehat{S}_{(\bb{z})} \xrightarrow{\sim} S$ thus defined thus gives a correspondence between $x\bb{z}^n/\bb{z}^{n+k}$ (where $\deg(x) = k$ for $k\geq -n$) and the element $x$.
The homomorphism \eref{eq:2.8.2.3.2} gives a correspondence between $x\bb{z}^n/f^k$ (where $\deg(x) = kd-n$) and the element $x/f^k$ of degree $-n$ in $S_f^\leq$, and it is again clear that this does indeed give an isomorphism.
\end{proof}

\begin{env}[8.2.4]
\label{2.8.2.4}
Let $M$ be a graded $S$-module.
It is clear that the $S$-module
\[
\label{eq:2.8.2.4.1}
  \widehat{M} = M \otimes_S \widehat{S} = M \otimes_S S[\bb{z}]
\tag{8.2.4.1}
\]
is the direct sum of the $S$-modules $M\otimes S\bb{z}^n$, and thus of the abelian groups $M_k\otimes S\bb{z}^n$ ($k\in\bb{Z}$, $n\geq0$);
we define on $\widehat{M}$ the structure of a graded $\widehat{S}$-module by setting
\[
\label{eq:2.8.2.4.2}
  \deg(x\otimes\bb{z}^n) = n+\deg x
\tag{8.2.4.2}
\]
for all homogeneous $x$ in $M$.
We leave it to the reader to prove the analogue of \sref{2.8.2.3}:
\end{env}

\begin{lemma}[8.2.5]
\label{2.8.2.5}
\begin{enumerate}
  \item[\rm{(i)}] There is a canonical di-isomorphism of (non-graded) modules
    \[
    \label{eq:2.8.2.5.1}
      \widehat{M}_{(\bb{z})} \xrightarrow{\sim} M.
    \tag{8.2.5.1}
    \]
  \item[\rm{(ii)}] For all $f\in S_d$ ($d>0$), there is a di-isomorphism of (non-graded) modules
    \[
    \label{eq:2.8.2.5.2}
      \widehat{M}_{(f)} \xrightarrow{\sim} M_f^\leq.
    \tag{8.2.5.2}
    \]
\end{enumerate}
\end{lemma}

\begin{env}[8.2.6]
\label{2.8.2.6}
Let $S$ be a \emph{positively}-graded ring, and consider the decreasing sequence of graded ideals of $S$
\[
\label{eq:2.8.2.6.1}
  S_{[n]} = \bigoplus_{m\geq n} S_m
  \qquad\mbox{($n\geq0$)}
\tag{8.2.6.1}
\]
(so, in particular, we have $S_{[0]}=S$ and $S_{[1]}=S_+$).
Since it is evident that $S_{[m]}S_{[n]} \subset S_{[m+n]}$, we can define a \emph{graded ring} $S^\natural$ by setting
\[
\label{eq:2.8.2.6.2}
  S^\natural = \bigoplus_{n\geq0} S_n^\natural
  \quad
  \text{with}
  \quad
  S_n^\natural = S_{[n]}.
\tag{8.2.6.2}
\]
$S_0^\natural$ is then the ring $S$ considered as a \emph{non-graded} ring, and $S^\natural$ is thus an $S_0^\natural$-algebra.
For every homogeneous element $f\in S_d$ ($d>0$), we denote by $f^\natural$ the element $f$ considered as belonging to $S_{[d]} = S_d^\natural$.
With this notation:
\end{env}

\oldpage[II]{159}
\begin{lemma}[8.2.7]
\label{2.8.2.7}
Let $S$ be a positively-graded ring, and $f$ a homogeneous element of $S_d$ ($d>0$).
There are canonical ring isomorphisms
\[
\label{eq:2.8.2.7.1}
  S_f \xrightarrow{\sim} \bigoplus_{n\in\bb{Z}} S(n)_{(f)}
\tag{8.2.7.1}
\]
\[
\label{eq:2.8.2.7.2}
  (S_f^\geq)_{f/1} \xrightarrow{\sim} S_f
\tag{8.2.7.2}
\]
\[
\label{eq:2.8.2.7.3}
  S_{(f^\natural)}^\natural \xrightarrow{\sim} S_f^\geq
\tag{8.2.7.3}
\]
where the first two are isomorphisms of graded rings.
\end{lemma}

\begin{proof}
\label{proof-2.8.2.7}
It is immediate, by definition, that we have $(S_f)_n = (S(n)_f)_0$, whence the isomorphism in \eref{eq:2.8.2.7.1}, which is exactly the identity.
Next, since $f/1$ is invertible in $S_f$, there is a canonical isomorphism $S_f \xrightarrow{\sim} (S_f^\geq)_{f/1} = (S_f)_{f/1}$, by \eref{eq:2.8.2.1.2} applied to $S_f$;
the inverse isomorphism is, by definition, the isomorphism in \eref{eq:2.8.2.7.2}.
Finally, if $x = \sum_{m\geq n}y_m$ is an element of $S_{[n]}$ with $n=kd$, then the element $x/(f^\natural)^k$ corresponds to the element $\sum_m y_m/f^k$ of $S_f^\geq$, and we can quickly verify that this defines an isomorphism \eref{eq:2.8.2.7.3}.
\end{proof}

\begin{env}[8.2.8]
\label{2.8.2.8}
If $M$ is a graded $S$-module, then we similarly define, for all $n\in\bb{Z}$,
\[
\label{eq:2.8.2.8.1}
  M_{[n]} = \bigoplus_{m\geq n} M_m
\tag{8.2.8.1}
\]
and, since $S_{[m]}M_{[n]} \subset M_{[m+n]}$ ($m\geq0$), we can define a graded $S^\natural$-module $M^\natural$ by setting
\[
\label{eq:2.8.2.8.2}
  M^\natural = \bigoplus_{n\in\bb{Z}}
  \quad
  \text{with}
  \quad
  M_n^\natural = M_{[n]}.
\tag{8.2.8.2}
\]
\end{env}

We leave to the reader the proof of:
\begin{lemma}[8.2.9]
\label{2.8.2.9}
With the notation of \sref{2.8.2.7} and \sref{2.8.2.8}, there are canonical di-isomorphisms of modules
\[
\label{eq:2.8.2.9.1}
  M_f \xrightarrow{\sim} \bigoplus_{n\in\bb{Z}} M(n)_{(f)}
\tag{8.2.9.1}
\]
\[
\label{eq:2.8.2.9.2}
  (M_f^\geq)_{f/1} \xrightarrow{\sim} M_f
\tag{8.2.9.2}
\]
\[
\label{eq:2.8.2.9.3}
  M_{(f^\natural)}^\natural \xrightarrow{\sim} M_f^\geq
\tag{8.2.9.3}
\]
where the first two are di-isomorphisms of graded modules.
\end{lemma}

\begin{lemma}[8.2.10]
\label{2.8.2.10}
Let $S$ be a positively-graded ring.
\begin{enumerate}
  \item[\rm{(i)}] For $S^\natural$ to be an $S_0^\natural$-algebra of finite type (resp. a Noetherian $S_0^\natural$-algebra), it is necessary and sufficient for $S$ to be an $S_0^\natural$-algebra of finite type (resp. a Noetherian $S_0^\natural$-algebra).
  \item[\rm{(ii)}] For $S_{n+1}^\natural = S_1^\natural S_n^\natural$ ($n\geq n_0$), it is necessary and sufficient for $S_{n+1}=S_1S_n$ ($n\geq n_0$).
  \item[\rm{(iii)}] For $S_n^\natural = S_1^\natural$ ($n\geq n_0$), it is necessary and sufficient for $S_n=S_1^n$ ($n\geq n_0$).
  \item[\rm{(iv)}] If $(f_\alpha)$ is a set of homogeneous elements of $S_+$ such that $S_+$ is the radical in $S_+$ of the ideal of $S_+$ generated by the $f_\alpha$, then $S_+^\natural$ is the radical in $S_+^\natural$ of the ideal of $S_+^\natural$ generated by the $f_\alpha^\natural$.
\end{enumerate}
\end{lemma}

\begin{proof}
\label{proof-2.8.2.10}
\begin{enumerate}
  \item[\rm{(i)}] If $S^\natural$ is an $S_0^\natural$-algebra of finite type, then $S_+=S_1^\natural$ is a module of finite type over $S=S_0^\natural$, by \sref{2.2.1.6}[i], and so $S$ is an $S_0$-algebra of finite type \sref{2.2.1.4};
    if $S^\natural$ is a Noetherian ring, then so too is $S_0^\natural=S$ \sref{2.2.1.5}.
    Conversely, if $S$ is an $S_0$-algebra
\oldpage[II]{160}
    of finite type, then we know \sref{2.2.1.6}[ii] that there exist $h>0$ and $m_0>0$ such that $S_{n+h}=S_hS_n$ for $n\geq m_0$;
    we can clearly assume that $m_0\geq h$.
    Furthermore, the $S_m$ are $S_0$-modules of finite type \sref{2.2.1.6}[i].
    So, if $n\geq m_0+h$, then $S_n^\natural = S_hS_{n-h}^\natural = S_h^\natural S_{n-h}^\natural$;
    and if $m<m_0+h$ then, letting $E = S_{m_0}+\ldots+S_{m_0+h-1}$, we have that
    \[
      S_m^\natural = S_m + \ldots + S_{m_0+h-1} + S_hE + S_h^2E + \ldots.
    \]
    For $1\leq m\leq m_0$, let $G_m$ be the union of the finite systems of generators of the $S_0$-modules $S_i$ for $m\leq i\leq m_0+h-1$, thought of as a subset of $S_{[m]}$.
    For $m_0+1\leq m\leq m_0+h-1$, let $G_m$ be the union of the finite system of generators of the $S_0$-modules $S_i$ for $m\leq i\leq m_0+h-1$ and of $S_hE$, thought of as a subset of $S_{[m]}$.
    It is clear that $S_m^\natural=S_0^\natural G_m$ for $1\leq m\leq m_0+h-1$, and thus the union $G$ of the $G_m$ for $1\leq m\leq m_0+h-1$ is a system of generators of the $S_0^\natural$-algebra $S^\natural$.
    We thus conclude that, if $S=S_0^\natural$ is a Noetherian ring, then so too is $S^\natural$.
  \item[\rm{(ii)}] It is clear that, if $S_{n+1}=S_1S_n$ for $n\geq n_0$, then $S_{n+1}^\natural=S_1S_n^\natural$, and \emph{a fortiori} $S_{n+1}^\natural=S_1^\natural S_n^\natural$ for $n\geq n_0$.
    Conversely, this last equality can be written as
    \[
      S_{n+1} + S_{n+2} + \ldots
      =
      (S_1 + S_2 + \ldots)(S_n + S_{n+1} + \ldots)
    \]
    and comparing terms of degree $n+1$ (in $S$) on both sides gives that $S_{n+1}=S_1S_n$.
  \item[\rm{(iii)}] If $S_n=S_1^n$ for $n\geq n_0$, then $S_n^\natural=S_1^n+S_1^{n+1}+\ldots$;
    since $S_1^\natural$ contains $S_1+S_1^2+\ldots$, we have that $S_n^\natural\subset S_1^{\natural n}$, and thus $S_n^\natural=S_1^{\natural n}$ for $n\geq n_0$.
    Conversely, the only terms of $S_1^{\natural n}=(S_1+S_2+\ldots)^n$ that are of degree $n$ in $S$ are those of $S_1^n$;
    the equality $S_n^\natural=S_1^{\natural n}$ thus implies that $S_n=S_1^n$.
  \item[\rm{(iv)}] It suffices to show that, if an element $g\in S_{k+h}$ is considered as an element of $S_k^\natural$ ($k>0$, $h\geq0$), then there exists an integer $n>0$ such that $g^n$ is a linear combination (in $S_{kn}^\natural$) of the $f_\alpha^\natural$ with coefficients in $S^\natural$.
    By hypothesis, there exists an integer $m_0$ such that, for $m\geq m_0$, we have, \emph{in $S$}, that $g^m = \sum_\alpha c_{\alpha m}f_\alpha$, where the indices $\alpha$ here are \emph{independent of $m$};
    furthermore, we can clearly assume that the $c_{\alpha m}$ are homogeneous, with
    \[
      \deg(c_{\alpha m}) = m(k+h)-\deg f_\alpha
    \]
    in $S$.
    So take $m_0$ sufficiently large enough to ensure that $km_0>\deg f_\alpha$ for all the $f_\alpha$ that appear in $g^{m_0}$;
    for all $\alpha$, let $c'_{\alpha m}$ be the element $c_{\alpha m}$ considered as having degree $km-\deg f_\alpha$ \emph{in $S^\natural$};
    we then have, in $S^\natural$, that $g^m = \sum_\alpha c'_{\alpha m}f_\alpha^\natural$, which finishes the proof.
\end{enumerate}
\end{proof}

\begin{env}[8.2.11]
\label{2.8.2.11}
Consider the graded $S_0$-algebra
\[
\label{eq:2.8.2.11.1}
  S^\natural \otimes_S S_0
  =
  S^\natural/S_+S^\natural
  =
  \bigoplus_{n\geq0} S_{[n]}/S_+S_{[n]}.
\tag{8.2.11.1}
\]

Since $S_n$ is a quotient $S_0$-module of $S_{[n]}/S_+S_{[n]}$, there is a canonical homomorphism of graded $S_0$-algebras
\[
\label{eq:2.8.2.11.2}
  S^\natural \otimes_S S_0 \to S
\tag{8.2.11.2}
\]
which is clearly \emph{surjective}, and thus corresponds \sref{2.2.9.2} to a canonical \emph{closed immersion}
\[
\label{eq:2.8.2.11.3}
  \Proj(S) \to \Proj(S^\natural \otimes_S S_0).
\tag{8.2.11.3}
\]
\end{env}

\oldpage[II]{161}
\begin{proposition}[8.2.12]
\label{2.8.2.12}
The canonical morphism \eref{eq:2.8.2.11.3} is bijective.
For the homomorphism \eref{eq:2.8.2.11.2} to be (TN)-bijective, it is necessary and sufficient for there to exist some $n_0$ such that $S_{n+1}=S_1S_n$ for $n\geq n_0$.
If this latter condition is satisfied, then \eref{eq:2.8.2.11.3} is an isomorphism;
the converse is true whenever $S$ is Noetherian.
\end{proposition}

\begin{proof}
\label{proof:2.8.2.12}
To prove the first claim, it suffices \sref{2.2.8.3} to show that the kernel $\fk{I}$ of the homomorphism \eref{eq:2.8.2.11.2} consists of \emph{nilpotent} elements.
But if $f\in S_{[n]}$ is an element whose class modulo $S_+S_{[n]}$ belongs to this kernel, then this implies that $f\in S_{[n+1]}$;
then $f^{n+1}$, considered as an element of $S_{[n(n+1)]}$, is also an element of $S_+S_{[n(n+1)]}$, since it can be written as $f\cdot f^n$;
so the class of $f^{n+1}$ modulo $S_+S_{[n(n+1)]}$ is zero, which proves our claim.
Since the hypothesis that $S_{n+1}=S_1S_n$ for $n\geq n_0$ is equivalent to $S_{n+1}^\natural=S_1^\natural S_n^\natural$ for $n\geq n_0$ \sref{2.8.2.10}[ii], this hypothesis is equivalent, by definition, to the fact that \eref{eq:2.8.2.11.2} is (TN)-injective, and thus (TN)-bijective, and so \eref{eq:2.8.2.11.3} is an isomorphism, by \sref{2.2.9.1}.
Conversely, if \eref{eq:2.8.2.11.3} is an isomorphism, then the sheaf $\widetilde{\fk{I}}$ on $\Proj(S^\natural\otimes_S S_0)$ is zero \sref{2.2.9.2}[i];
since $S^\natural\otimes_S S_0$ is Noetherian, as a quotient of $S^\natural$ \sref{2.8.2.10}[i], we conclude from \sref{2.2.7.3} that $\fk{I}$ satisfies condition (TN), and so $S_{n+1}^\natural=S_1^\natural S_n^\natural$ for $n\geq n_0$, and this finishes the proof, by \sref{2.8.2.10}[ii].
\end{proof}

\begin{env}[8.2.13]
\label{2.8.2.13}
Consider now the canonical injections $(S_+)^n\to S_{[n]}$, which define an injective homomorphism of degree zero of graded rings
\[
\label{eq:2.8.2.13.1}
  \bigoplus_{n\geq0} (S_+)^n \to S^\natural.
\tag{8.2.13.1}
\]
\end{env}

\begin{proposition}[8.2.14]
\label{2.8.2.14}
For the homomorphism \eref{eq:2.8.2.13.1} to be a (TN)-isomorphism, it is necessary and sufficient for there to exist some $n_0$ such that $S_n=S_1^n$ for all $n\geq n_0$.
Whenever this is the case, the morphisms corresponding to \eref{eq:2.8.2.13.1} is everywhere defined and also an isomorphism
\[
  \Proj(S^\natural) \xrightarrow{\sim} \Proj(\bigoplus_{n\geq0}(S_+)^n);
\]
the converse is true whenever $S$ is Noetherian.
\end{proposition}

\begin{proof}
\label{proof:2.8.2.14}
The first two claims are evident, given \sref{2.8.2.10}[iii] and \sref{2.2.9.1}.
The third will follow from \sref{2.8.2.10}[i and iii] and the following lemma:
\begin{lemma}[8.2.14.1]
\label{2.8.2.14.1}
Let $T$ be a positively-graded ring that is also a $T_0$-algebra of finite type.
If the morphism corresponding to the injective homomorphism $\bigoplus_{n\geq0}T_1^n\to T$ is everywhere defined and also an isomorphism $\Proj(T)\to\Proj(\bigoplus_{n\geq0}T_1^n)$, then there exists some $n_0$ such that $T_n=T_1^n$ for $n\geq n_0$.
\end{lemma}

Let $g_i$ ($1\leq i\leq r$) be generators of the $T_0$-module $T_1$.
The hypothesis implies first of all that the $D_+(g_i)$ cover $\Proj(T)$ \sref{2.2.8.1}.
Let $(h_j)_{1\leq j\leq s}$ be a system of homogeneous elements of $T_+$, with $\deg(h_j)=n_j$, that form, with the $g_i$, a system of generators of the ideal $T_+$, or, equivalently \sref{2.2.1.3}, a system of generators of $T$ as a $T_0$-algebra;
if we set $T'=\bigoplus_{n\geq0}T_1^n$, then the element $h_j/g_i^{n_j}$ of the ring $T_{(g_i)}$ must, by hypothesis, belong to the subring $T'_{(g_i)}$, and so there exists some integer $k$ such that $T_1^k h_j\subset T_1^{k+n_j}$ for all $j$.
We thus conclude, by induction on $r$, that $T_1^k h_j^r \subset T'$ for all $r\geq1$, and, by definition of the $h_j$, we thus have that $T_1^k T\subset T'$.
Also, there exists, for all $j$, an integer $m_j$ such that $h_j^{m_j}$ belongs to the ideal of $T$ generated by the $g_i$ \sref{2.2.3.14}, so $h_j^{m_j}\in T_1 T$, and
\oldpage[II]{162}
$h_j^{m_j k}\in T_1^k T\subset T'$.
There is thus an integer $m_0\geq k$ such that $h_j^m\in T_1^{mn}$ for $m\geq m_0$.
So, if $q$ is the largest of the integers $n_j$, then $n_0=qsm_0+k$ is the required number.
Indeed, an element of $S_n$, for $n\geq n_0$, is the sum of monomials belonging to $T_1^\alpha u$, where $u$ is a product of powers of the $h_j$;
if $\alpha\geq k$, then it follows from the above that $T_1^\alpha u\subset T_1^n$;
in the other case, one of the exponents of the $h_j$ is $\geq m_0$, so $u\in T_1^\beta v$, where $\beta\geq k$ and $v$ is again a product of powers of the $h_j$;
we can then reduce to the previous case, and so we conclude that $T_1^\alpha u\subset T_1^n$ in all cases.
\end{proof}

\begin{remark}[8.2.15]
\label{2.8.2.15}
The condition $S_n=S_1^n$ for $n\geq n_0$ clearly implies that $S_{n+1}=S_1S_n$ for $n\geq n_0$, but the converse is not necessarily true, even if we assume that $S$ is Noetherian.
For example, let $K$ be a field, $A=K[\bb{x}]$, and $B=K[\bb{y}]/\bb{y}^2K[\bb{y}]$, where $\bb{x}$ and $\bb{y}$ are indeterminates, with $\bb{x}$ taken to have degree 1 and $\bb{y}$ to have degree 2, and let $S=A\otimes_K B$, so that $S$ is a graded algebra over $K$ that has a basis given by the elements $1$, $\bb{x}^n$ ($n\geq1$), and $\bb{x}^n\bb{y}$ ($n\geq0$).
It is immediate that $S_{n+1}=S_1S_n$ for $n\geq2$, but $S_1^n=K\bb{x}^n$ while $S_n=K\bb{x}^n+K\bb{x}^n\bb{y}$ for $n\geq2$.
\end{remark}


\subsection{Projective cones}
\label{subsection:2.8.3}


% \subsection{Quasi-coherent sheaves on projective cones}
% \label{subsection:2.8.12}

% \begin{env}[8.12.1]
% \label{2.8.12.1}
% Let us take the hypotheses and notation of \sref{2.8.3.1}.
% Let $\sh{M}$ be a \emph{quasi-coherent graded $\sh{S}$-module}; to avoid any confusion, we denote by $\widetilde{\sh{M}}$ the quasi-coherent $\sh{O}_C$-module
% \oldpage[II]{192}
% associated to $\sh{M}$ \sref{2.1.4.3} when $\sh{M}$ is considered as a \emph{nongraded} $\sh{S}$-module, and by $\shProj_0(\sh{M})$ the quasi-coherent $\sh{O}_X$-module associated to $\sh{M}$, $\sh{M}$ being considered this time as a graded $\sh{S}$-module (in other words, the $\sh{O}_X$-module denoted by $\widetilde{\sh{M}}$ in \sref{2.3.2.2}).
% In addition, we set
% \[
% \label{eq:2.8.12.1.1}
%   \sh{M}_X=\shProj_0(\sh{M})=\bigoplus_{n\in\bb{Z}}\shProj_0(\sh{M}(n));
%   \tag{8.12.1.1}
% \]
% the quasi-coherent graded $\sh{O}_X$-algebra $\sh{S}_X$ being defined by \eref{eq:2.8.6.1.1}, $\shProj(\sh{M})$ is equipped with a structure of a \emph{(quasi-coherent) graded $\sh{S}_X$-module}, by means of the canonical homomorphisms \eref{eq:2.3.2.6.1}
% \[
% \label{eq:2.8.12.1.2}
%   \sh{O}_X(m)\otimes_{\sh{O}_X}\shProj_0(\sh{M}(n))\to\shProj_0(\sh{S}(m)\otimes_\sh{S}\sh{M}(n))\to\shProj_0(\sh{M}(m+n)),
%   \tag{8.12.1.2}
% \]
% the verification of the axioms of sheaves of modules being done using the commutative diagram \eref{eq:2.2.5.11.4}.

% If $Y=\Spec(A)$ is affine, $\sh{S}=\widetilde{S}$, and $\sh{M}=\widetilde{M}$, where $S$ is a graded $A$-algebra and $M$ is a graded $S$-module, then, for every homogeneous element $f\in S_+$, we have
% \[
% \label{eq:2.8.12.1.3}
%   \Gamma(X_f,\shProj(\widetilde{M}))=M_f
%   \tag{8.12.1.3}
% \]
% by the definitions and \eref{eq:2.8.2.9.1}.

% Now consider the quasi-coherent graded $\widehat{\sh{S}}$-module
% \[
% \label{eq:2.8.12.1.4}
%   \widehat{\sh{M}}=\sh{M}\otimes_\sh{S}\widehat{\sh{S}}
%   \tag{8.12.1.4}
% \]
% ($\widehat{\sh{S}}$ defined by \eref{eq:2.8.3.1.1}); we deduce a quasi-coherent graded $\sh{O}_{\widehat{C}}$-module $\shProj_0(\widehat{\sh{M}})$, which we will also denote by
% \[
% \label{eq:2.8.12.1.5}
%   \sh{M}^\square=\shProj_0(\widehat{\sh{M}}).
%   \tag{8.12.1.5}
% \]

% It is clear \sref{2.3.2.4} that $\sh{M}^\square$ is an additive functor which is \emph{exact} in $\sh{M}$, commuting with direct sums and with inductive limits.
% \end{env}


\bibliography{the}
\bibliographystyle{amsalpha}

\end{document}

