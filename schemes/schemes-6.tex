\section{Finiteness conditions}
\label{section-finiteness-conditions}

\subsection{Noetherian and locally Noetherian preschemes}
\label{subsection-noetherian-and-locally-noetherian-preschemes}

\begin{defn}[6.1.1]
\label{1.6.1.1}
We say that a prescheme $X$ is Noetherian (\emph{resp.} locally Noetherian) if it is a finite union (\emph{resp.} union) of affine open $V_\alpha$ in such a way that the ring of the of the induced scheme on each of the $V_\alpha$ is Noetherian.
\end{defn}

It follows immediately from \sref{1.1.5.2} that, if $X$ is locally Noetherian, then the structure sheaf $\OO_X$ is a \emph{coherent sheaf of rings}, the question being a local one.
Every \emph{quasi-coherent sub-$\OO_X$-module}
\oldpage[I]{141}
(\emph{resp.} quasi-coherent quotient $\OO_X$-module) of a \emph{coherent} $\OO_X$-module $\sh{F}$ is \emph{coherent}, as the question is once again a local one, and it suffices to apply \sref{1.1.5.1}, \sref{1.1.4.1}, and \sref{1.1.3.10}, combined with the fact that a submodule (resp. quotient module) of a module of finite type over a Noetherian ring is of finite type.
In particular, every \emph{quasi-coherent sheaf of ideals} of $\OO_X$ is \emph{coherent}.

If a prescheme $X$ is a finite union (resp. union) of open subsets $W_\lambda$ in such a way that the preschemes induced on the $W_\lambda$ are Noetherian (resp. locally Noetherian), it is clear that $X$ is Noetherian (resp. locally Noetherian).

\begin{prop}[6.1.2]
\label{1.6.1.2}
For a prescheme $X$ to be Noetherian, it is necessary and sufficient for it to be locally Noetherian and have a quasi-compact underlying space.
The underlying space itself is then also Noetherian.
\end{prop}

\begin{proof}
\label{proof-1.6.1.2}
The first claim follows immediately from the definitions and \sref{1.1.1.10}[ii].
The second follows from \sref{1.1.1.6} and the fact that every space that is a finite union of Noetherian subspaces is itself Noetherian \sref[0]{0.2.2.3}.
\end{proof}

\begin{prop}[6.1.3]
\label{1.6.1.3}
Let $X$ be an affine scheme given by a ring $A$.
The following conditions are equivalent:
\emph{(a)} $X$ is Noetherian;
\emph{(b)} $X$ is locally Noetherian;
\emph{(c)} $A$ is Noetherian.
\end{prop}

\begin{proof}
\label{proof-1.6.1.3}
The equivalence between (a) and (b) follows from \sref{1.6.1.2} ant the fact that the underlying space of every affine scheme is quasi-compact \sref{1.1.1.10}; it is furthermore clear that (c) implies (a).
To see that (a) implies (c), we remark that there is a finite cover $(V_i)$ of $X$ by affine open subsets such that the ring $A_i$ of the prescheme induced on $V_i$ is Noetherian.
So let $(\fk{a}_n)$ be an increasing sequence of ideals of $A$; by a canonical bijective correspondence, there is a corresponding sequence $(\wt{a}_n)$ of sheaves of ideals in $\wt{A}=\OO_X$;
to see that the sequence $(\fk{a}_n)$ is \unsure{stable}, it suffices to prove that the sequence $(\wt{\fk{a}}_n)$ is.
But the restriction $\wt{\fk{a}}_n|V_i$ is a quasi-coherent sheaf of ideals in $\OO_X|V_i$, being the inverse image of $\wt{\fk{a}}_n$ under the canonical injection $V_i\to X$ \sref[0]{0.5.1.4};
$\wt{\fk{a}}_n|V_i$ is thus of the form $\wt{\fk{a}}_{ni}$, where $\fk{a}_{ni}$ is an ideal of $A_i$ \sref{1.1.3.7}.
Since $A_i$ is Noetherian, the sequence $(\fk{a}_{ni})$ is stable for all $i$, whence the proposition.
\end{proof}

We note that the above argument proves also that \emph{if $X$ is a Noetherian prescheme, then every increasing sequence of coherent sheaves of ideals of $\OO_X$ is \unsure{stable}}.

\begin{prop}[6.1.4]
\label{1.6.1.4}
Every subprescheme of a Noetherian (\emph{resp.} locally Noetherian) prescheme is Noetherian (\emph{resp.} locally Noetherian).
\end{prop}

\begin{proof}
\label{proof-1.6.1.4}
If suffices to give a proof for a Noetherian prescheme $X$;
further, by definition~\sref{1.6.1.1}, we can also restrict to the case where $X$ is an affine scheme.
Since every subprescheme of $X$ is a closed subprescheme of a prescheme induced on an open subset \sref{1.4.1.3}, we can restrict to the case of a subprescheme $Y$, either closed or induced on an open subset of $X$.
The proof in the case where $Y$ is closed is immediate, since if $A$ is the ring of $X$, we know that $Y$ is an affine scheme given by the ring $A/\fk{J}$, where $\fk{J}$ is an ideal of $A$ \sref{1.4.2.3};
since $A$ is Noetherian \sref{1.6.1.3}, so too is $A/\fk{J}$.

Now suppose that $Y$ is open in $X$;
the underlying space of $Y$ is Noetherian \sref{1.6.1.2}, hence quasi-compact, and thus a finite union of open subsets $D(f_i)$ ($f_i\in A$);
everything reduces to showing the proposition in the case where $Y=D(f)$ with $f\in A$.
But then $Y$ is an affine scheme whose ring is isomorphic to $A_f$ \sref{1.1.3.6};
since $A$ is Noetherian \sref{1.6.1.3}, so too is $A_f$.
\end{proof}

\begin{env}[6.1.5]
\label{1.6.1.5}
We note that the \emph{product} of two Noetherian $S$-preschemes is not necessarily Noetherian, even if the preschemes are affine, since the tensor product of two Noetherian algebras in not necessarily a Noetherian ring (cf. \sref{1.6.3.8}).
\end{env}

\begin{prop}[6.1.6]
\label{1.6.1.6}
If $X$ is a Noetherian prescheme, the nilradical $\sh{N}_X$ of $\OO_X$ is nilpotent.
\end{prop}

\begin{proof}
\label{proof-1.6.1.6}
We can in fact cover $X$ with a finite number of affine open subsets $U_i$, and it suffices to prove that there exists whole numbers $n_i$ such that $(\sh{N}_X|U_i)^{n_i}=0$;
if $n$ is the largest of the $n_i$, then we will have $\sh{N}_X^n=0$.
We can thus restrict to the case where $X=\Spec(A)$ is affine, with $A$ a Noetherian ring;
by \sref{1.5.1.1} and \sref{1.1.3.13}, it suffices to observe that the nilradical of $A$ is nilpotent (\cite[p.~127, cor.~4]{I-11}).
\end{proof}

\begin{cor}[6.1.7]
\label{1.6.1.7}
Let $X$ be a Noetherian prescheme;
for $X$ to be an affine scheme, it is necessary and sufficient that $X_\emph{red}$ be affine.
\end{cor}

\begin{proof}
\label{proof-1.6.1.7}
This follows from \sref{1.6.1.6} and \sref{1.5.1.10}.
\end{proof}

\begin{lem}[6.1.8]
\label{1.6.1.8}
Let $X$ be a topological space, $x$ a point of $X$, and $U$ an open neighbourhood of $x$ having only a finite number of irreducible components.
Then there exists a neighbourhood $V$ of $x$ such that every open neighbourhood of $x$ contained in $V$ is connected.
\end{lem}

\begin{proof}
\label{proof-1.6.1.8}
Let $U_i$ ($1\leqslant i\leqslant m$) be the irreducible components of $U$ not containing $x$;
the complement (in $U$) of the union of the $U_i$ is an open neighbourhood $V$ of $X$ inside $U$, and thus so too in $X$;
it is also, incidentally, the complement (in $X$) of the union of the irreducible components of $X$ that do not contain $x$ \sref[0]{0.2.1.6}.
So let $W$ be an open neighbourhood of $X$ contained in $V$.
The irreducible components of $W$ are the intersections of $W$ with the irreducible components of $U$ \sref[0]{0.2.1.6}, so these components contain $x$;
since they are connected, so too is $W$.
\end{proof}

\begin{cor}[6.1.9]
\label{1.6.1.9}
A locally Noetherian topological space is locally connected (which implies, amongst other things, that its connected components are open).
\end{cor}

\begin{prop}[6.1.10]
\label{1.6.1.10}
Let $X$ be a locally Noetherian topological space.
The following conditions are equivalent.
\begin{enumerate}[label=\emph{(\alph*)}]
  \item The irreducible components of $X$ are open.
  \item The irreducible components of $X$ are exactly its connected components.
  \item The connected components of $X$ are irreducible.
  \item Two distinct irreducible components of $X$ have an empty intersection.
\end{enumerate}
Finally, if $X$ is a prescheme, then these conditions are also equivalent to
\begin{enumerate}[label=\emph{(\alph*)}]
  \setcounter{enumi}{4}
  \item For every $x\in X$, $\Spec(\OO_x)$ is irreducible (or, in other words, the nilradical of $\OO_x$ is prime).
\end{enumerate}
\end{prop}

\begin{proof}
\label{proof-1.6.1.10}
It is immediate that (a) implies (b), because an irreducible space is connected, and (a) implies that the irreducible components of $X$ are the sets that are both open and closed.
It is trivial that (b) implies (c); conversely, a closed set $F$ containing
\oldpage[I]{143}
a connected component $C$ of $X$, with $C$ distinct from $F$, cannot be irreducible, because not being connected means that $F$ is the union of two disjoint nonempty sets that are both open and closed in $F$, and thus closed in $X$; as a result, (c) implies (b).
We immediately conclude from this that (c) implies (d), since two distinct connected components have no points in common.

We have not yet used the fact that $X$ is locally Noetherian.
Suppose now that this is indeed the case, and we will show that (d) implies (a): by \sref[0]{0.2.1.6}, we can restrict ourselves to the case where the space $X$ is Noetherian, and so has only a finite number of irreducible components.
Since they are closed and pairwise disjoint, they are open.

Finally, the equivalence between (d) and (e) holds true even without the assumption that the underlying space of the prescheme $X$ is locally Noetherian.
We can in fact restrict ourselves to the case where $X=\Spec(A)$ is affine, by \sref[0]{0.2.1.6};
to say that $x$ is contained in only one single irreducible component of $X$ is to say that $\fk{j}_x$ contains only one single minimal ideal of $A$ \sref{1.1.1.14}, which is equivalent to saying that $\fk{j}_x\OO_x$ contains only one single minimal ideal of $\OO_x$, whence the conclusion.
\end{proof}

\begin{cor}[6.1.11]
\label{1.6.1.11}
Let $X$ be a locally Noetherian space.
For $X$ to be irreducible, it is necessary and sufficient that $X$ be connected and nonempty, and that any two distinct irreducible components of $X$ have an empty intersection.
If $X$ is a prescheme, this latter condition is equivalent to asking that $\Spec(\OO_x)$ be irreducible for all $x\in X$.
\end{cor}

\begin{proof}
\label{proof-1.6.1.11}
The second claim has already been shown in \sref{1.6.1.10};
the only thing thus remaining to show is that the conditions in the first claim are sufficient.
But by \sref{1.6.1.10}, these conditions imply that the irreducible components of $X$ are exactly its connected components, and since $X$ is connected and nonempty, it is irreducible.
\end{proof}

\begin{cor}[6.1.12]
\label{1.6.1.12}
Let $X$ be a locally Noetherian prescheme.
For $X$ to be integral, it is necessary and sufficient that $X$ be connected and that $\OO_x$ be integral for all $x\in X$.
\end{cor}

\begin{prop}[6.1.13]
\label{1.6.1.13}
Let $X$ be a locally Noetherian prescheme, and let $x\in X$ be a point such that the nilradical $\sh{N}_x$ of $\OO_x$ is prime (\emph{resp.} such that $\OO_x$ is reduced, \emph{resp.} integral);
then there exists an open neighbourhood $U$ of $x$ that is irreducible (\emph{resp.} reduced, \emph{resp.} integral).
\end{prop}

\begin{proof}
\label{proof-1.6.1.13}
It suffices to consider two cases: where $\sh{N}_x$ is prime, and where $\sh{N}_x=0$; the third hypotheses is a combination of the first two.
If $\sh{N}_x$ is prime, then $x$ belongs to only one single irreducible component $Y$ of $X$ \sref{1.6.1.10};
the union of the irreducible components of $X$ that do not contain $x$ is closed (the set of these components being locally finite), and the complement $U$ of this union is thus open and contained in $Y$, and thus irreducible \sref[0]{0.2.1.6}
If $\sh{N}_x=0$, we also have $\sh{N}_y=0$ for any $y$ in a neighbourhood of $x$, because $\sh{N}$ is quasi-coherent \sref{1.5.1.1}, and thus coherent, since $X$ is locally Noetherian, and the conclusion then follows from \sref[0]{0.5.2.2}.
\end{proof}

\subsection{Artinian preschemes}
\label{subsectin-artinian-preschemes}

\begin{defn}[6.2.1]
\label{1.6.2.1}
We say that a prescheme is Artinian if it is affine, and given by an Artinian ring.
\end{defn}

\begin{prop}[6.2.2]
\label{1.6.2.2}
Given
\oldpage[I]{144}
a prescheme $X$, the following conditions are equivalent:
\begin{enumerate}[label=\emph{(\alph*)}]
  \item $X$ is an Artinian scheme;
  \item $X$ is Noetherian and its underlying space is discrete;
  \item $X$ is Noetherian and the points of its underlying space are closed \emph{(the $\mathrm{T}_1$ condition)}.
\end{enumerate}
When any of the above hold, the underlying space of $X$ is finite, and the ring $A$ of $X$ is the direct sum of local (Artinian) rings of points of $X$.
\end{prop}

\begin{proof}
\label{proof-1.6.2.2}
We know that (a) implies the last claim (\cite[p.~205, th.~3]{I-13}), so every prime ideal of $A$ is thus maximal and is the inverse image of a maximal ideal of one of the local components of $A$, and so the space $X$ is finite and discrete;
(a) thus implies (b), and (b) clearly implies (c).
To see that (c) implies (a), we first show that $X$ is then finite;
we can indeed restrict to the case where $X$ is affine, and we know that a Noetherian ring whose prime ideals are all maximal is Artinian (\cite[p.~203]{I-13}), whence our claim.
The underlying space $X$ is then discrete, the topological sum of a finite number of points $x_i$, and the local rings $\OO_{x_i}=A_i$ are Artinian;
it is clear that $X$ is isomorphic to the prime spectrum affine scheme of the ring $A$ (the direct sum of the $A_i$) \sref{1.1.7.3}.
\end{proof}

\subsection{Morphisms of finite type}
\label{subsection-morphisms-of-finite-type}

\begin{defn}[6.3.1]
\label{1.6.3.1}
We say that a morphism $f:X\to Y$ is of finite type if $Y$ is the union of a family $(V_\alpha)$ of affine open subsets having the following property:
\begin{enumerate}
  \item[(P)] $f^{-1}(V_\alpha)$ is a finite union of affine open subsets $U_{\alpha i}$ that are such that each ring $A(U_{\alpha i})$ is an algebra of finite type over $A(V_\alpha)$.
\end{enumerate}
We then say that $X$ is a prescheme of finite type over $Y$, or a $Y$-prescheme of finite type.
\end{defn}

\begin{prop}[6.3.2]
\label{1.6.3.2}
If $f:X\to Y$ is a morphism of finite type, then every affine open subset $W$ of $Y$ satisfies property \emph{(P)} of \sref{1.6.3.1}.
\end{prop}

We first show
\begin{lem}[6.3.2.1]
\label{1.6.3.2.1}
If $T\subset Y$ is an affine open subset, satisfying property \emph{(P)}, then, for every $g\in A(T)$, $D(g)$ also satisfies property \emph{(P)}.
\end{lem}

\begin{proof}
\label{proof-1.6.3.2.1}
By hypothesis, $f^{-1}(T)$ is a finite union of affine open subsets $Z_j$, that are such that $A(Z_j)$ is an algebra of finite type over $A(T)$;
let $\vphi_j:A(T)\to A(Z_j)$ be the homomorphism of rings corresponding to the restriction of $f$ to $Z_j$ \sref{1.2.2.4}, and set $g_j=\vphi_j(g)$;
we then have $f^{-1}(D(g))\cap Z_j=D(g_j)$ \sref{1.1.2.2.2}.
But $A(D(g_j))=A(Z_j)_{g_j}=A(Z_j)[1/g_j]$ is of finite type over $A(Z_j)$, and \emph{a fortiori} over $A(T)$ by the hypothesis, and so also over $A(D(g))=A(T)[1/g]$, which proves the lemma.
\end{proof}

\begin{proof}
\label{proof-1.6.3.2}
With the above lemma, since $W$ is quasi-compact \sref{1.1.1.10}, there exists a finite covering of $W$ by sets of the form $D(g_i)\subset W$, where each $g_i$ belongs to a ring $A(V_{\alpha(i)})$.
Each $D(g_i)$, being quasi-compact, is a finite union of sets $D(h_{ik})$, where $h_{ik}\in A(W)$;
if $\vphi_i:A(W)\to A(D(g_i))$ is the canonical map, then we have $D(h_{ik})=D(\vphi_i(h_{ik}))$ by \sref{1.1.2.2.2}.
By \sref{1.6.3.2.1}, each of the $f^{-1}(D(h_{ik}))$ admits a finite covering by affine open subsets $U_{ijk}$, that are such that the $A(U_{ijk})$ are algebras of finite type over $A(D(h_{ik}))=A(W)[1/h_{ik}]$, whence the proposition.
\end{proof}

We
\oldpage[I]{145}
can thus say that the notion of a prescheme of finite type over $Y$ is \emph{local on $Y$}.

\begin{prop}[6.3.3]
\label{1.6.3.3}
Let $X$ and $Y$ be affine schemes;
for $X$ to be of finite type over $Y$, it is necessary and sufficient that $A(X)$ be an algebra of finite type over $A(Y)$.
\end{prop}

\begin{proof}
\label{proof-1.6.3.3}
Since the condition clearly suffices, we show that it is necessary.
Set $A=A(Y)$ and $B=A(X)$;
by \sref{1.6.3.2}, there exists a finite affine open cover $(V_i)$ of $X$ such that each of the rings $A(V_i)$ is an $A$-algebra of finite type.
Further, since the $V_i$ are quasi-compact, we can cover each of them with a finite number of open subsets of the form $D(g_{ij})\subset V_i$, where $g_{ij}\in B$;
if $\vphi_i$ is a homomorphism $B\to A(V_i)$ that corresponds to the canonical injection $V_i\to X$, then we have $B_{g_{ij}}=(A(V_i))_{\vphi_i(g_{ij})}=A(V_i)[1/\vphi_i(g_{ij})]$, so $B_{g_{ij}}$ is an $A$-algebra of finite type.
We can thus restrict to the case where $V_i=D(g_i)$ with $g_i\in B$.
By hypothesis, there exists a finite subset $F_i$ of $B$ and an integer $n_i\geqslant0$ such that $B_{g_i}$ is the algebra generated over $A$ by the elements $b_i/g_i^{n_i}$, where the $b_i$ run over all of $F_i$.
Since there are only finitely many of the $g_i$, we can assume that all the $n_i$ are equal to the same integer $n$.
Further, since the $D(g_i)$ form a cover of $X$, the ideal generated in $B$ by the $g_i$ is equal to $B$, or, in other words, there exist $h_i\in B$ such that $\sum_i h_ig_i=1$.
So let $F$ be the finite subset of $B$ given by the union of the $F_i$, the set of the $g_i$, and the set of the $h_i$; we will show that the subring $B'=A[F]$ of $B$ is equal to $B$.
By hypothesis, for every $b\in B$ and every $i$, the canonical image of $b$ in $B_{g_i}$ is of the form $b'_i/g_i^{m_i}$, where $b'_i\in B'$;
by multiplying the $b'_i$ by suitable powers of the $g_i$, we can again assume that all the $m_i$ are equal to the same integer $m$.
By the definition of the ring of fractions, there is thus an integer $N$ (dependant on $b$) such that $N\geqslant m$ and $g_i^Nb\in B'$ for all $i$;
but, \emph{in the ring $B'$}, the $g_i^N$ generate the ideal $B'$, because the $g_i$ do (and the $h_i$ belong to $B'$);
there are thus $c_i\in B'$ such that $\sum_i c_ig_i^N=1$, whence $b=\sum_ic_ig_i^Nb\in B'$, Q.E.D.
\end{proof}

\begin{prop}[6.3.4]
\label{1.6.3.4}
\begin{enumerate}[label=\emph{(\roman*)}]
    \item Every closed immersion is of finite type.
    \item The composition of two morphisms of finite type is of finite type.
    \item If $f:X\to X'$ and $g:Y\to Y'$ are $S$-morphisms of finite type, then $f\times_S g$ is of finite type.
    \item If $f:X\to Y$ is an $S$-morphism of finite type, then $f_{(S')}$ is of finite type for any extension $g:S'\to S$ of the base prescheme.
    \item If the composition $g\circ f$ of two morphisms is of finite type, with $g$ separated, then $f$ is of finite type.
    \item If a morphism $f$ is of finite type, then $f_\mathrm{red}$ is of finite type.
\end{enumerate}
\end{prop}

\begin{proof}
\label{proof-1.6.3.4}
By \sref{1.5.5.12}, it suffices to prove (i), (ii), and (iv).

To show (i), we can restrict to the case of a canonical injection $X\to Y$, with $X$ being a closed subprescheme of $Y$;
further \sref{1.6.3.2}, we can assume that $Y$ is affine, in which case $X$ is also affine \sref{1.4.2.3} and its ring is isomorphic to a quotient ring $A/\fk{J}$, where $A$ is the ring of $Y$ and $\fk{J}$ is an ideal of $A$;
since $A/\fk{J}$ is of finite type over $A$, the conclusion follows.

Now we show (ii).
Let $f:X\to Y$ and $g:Y\to Z$ be two morphisms of finite type, and let $U$ be an affine open subset of $Z$;
$g^{-1}$ admits a finite covering by affine open subsets $V_i$ that are such that each $A(V_i)$ is an algebra of finite type over $A(U)$ \sref{1.6.3.2};
similarly. each
\oldpage[I]{146}
of the $f^{-1}$ admits a finite cover by affine open subsets $W_{ij}$ that are such that each $A(W_{ij})$ is an algebra of finite type over $A(V_i)$, and so also an algebra of finite type over $A(U)$, whence the conclusion.

Finally, to show (iv), we can restrict to the case where $S=Y$;
then $f_{(S')}$ is also equal to $f_{Y_{(S')}}$, where we consider $f$ as a $Y$-morphism, and the base extension is $Y_{(S')}\to Y$ \sref{1.3.3.9}.
So let $p$ and $q$ be the projections $X_{(S')}\to X$ and $X_{(S')}\to S'$.
Let $V$ be an affine open subset of $S$;
$f^{-1}(V)$ is a finite union of affine open subsets $W_i$, each of which is such that $A(W_i)$ is an algebra of finite type over $A(V)$ \sref{1.6.3.2}.
Let $V'$ be an affine open subset of $S'$ contained in $g^{-1}(V)$;
since $f\circ p=g\circ q$, $q^{-1}(V')$ is contained in the union of the $p^{-1}(W_i)$;
on the other hand, the intersection $p^{-1}(W_i)\cap q^{-1}(V')$ can be identified with the product $W_i\times V V'$ \sref{1.3.2.7}, which is an affine scheme whose ring is isomorphic to $A(W_i)\otimes_{A(V)}A(V')$ \sref{1.3.2.2};
this ring is, by hypothesis, an algebra of finite type over $A(V')$, which proves the proposition.
\end{proof}

\begin{cor}[6.3.5]
\label{1.6.3.5}
Let $f:X\to Y$ be an immersion morphism.
If the underlying space of $Y$ (\emph{resp.} $X$) is locally Noetherian (\emph{resp.} Noetherian), then $f$ is of finite type.
\end{cor}

\begin{proof}
\label{proof-1.6.3.5}
We can always assume that $Y$ is affine \sref{1.6.3.2};
if the underlying space of $Y$ is locally Noetherian, then we can further assume that it is Noetherian, and then the underlying space of $X$, which is a subspace, is also Noetherian.
In other words, we can assume that $Y$ is affine and that the underlying space of $X$ is Noetherian;
there then exists a covering of $X$ by a finite number of affine open subsets $D(g_i)\subset Y$, where $g_i\in A(Y)$, that are such that the $X\cap D(g_i)$ are closed in $D(g_i)$ (and thus affine schemes \sref{1.4.2.3}), because $X$ is locally closed in $Y$ \sref{1.4.1.3}.
Then $A(X\cap D(g_i))$ is an algebra of finite type over $A(D(g_i))$, by \sref{1.6.3.4}[i] and \sref{1.6.3.3}, and $A(D(g_i))=A(Y)_{g_i}=A(Y)[1/g_i]$ is of finite type over $A(Y)$, which finishes the proof.
\end{proof}

\begin{cor}[6.3.6]
\label{1.6.3.6}
Let $f:X\to Y$ and $g:Y\to Z$ be morphisms.
If $g\circ f$ if of finite type, with either $X$ Noetherian or $X\times_Z Y$ locally Noetherian, then $f$ is of finite type.
\end{cor}

\begin{proof}
\label{proof-1.6.3.6}
This follows immediately from the proof of \sref{1.5.5.12} and from \sref{1.6.3.5} applied to the immersion morphism $\Gamma_f$.
\end{proof}

\begin{prop}[6.3.7]
\label{1.6.3.7}
Let $f:X\to Y$ be a morphism of finite type;
if $Y$ is Noetherian (\emph{resp.} locally Noetherian), then $X$ is Noetherian (\emph{resp.} locally Noetherian).
\end{prop}

\begin{proof}
\label{proof-1.6.3.7}
We can restrict to proving the proposition for when $Y$ is Noetherian.
Then $Y$ is a finite union of affine open subsets $V_i$ that are such that the $A(V_i)$ are Noetherian rings.
By \sref{1.6.3.2}, each of the $f^{-1}(V_i)$ is a finite union of affine open subsets $W_{ij}$ that are such that the $A(W_{ij})$ are algebras of finite type over $A(V_i)$, and thus Noetherian rings;
this proves that $X$ is Noetherian.
\end{proof}

\begin{cor}[6.3.8]
\label{1.6.3.8}
Let $X$ be a prescheme of finite type over $S$.
For every base extension $S'\to S$ with $S'$ Noetherian (\emph{resp.} locally Noetherian), $X_{(S')}$ is Noetherian (\emph{resp.} locally Noetherian).
\end{cor}

\begin{proof}
\label{proof-1.6.3.8}
This follows from \sref{1.6.3.7}, since $X_{(S')}$ is of finite type over $S'$ by \sref{1.6.3.4}[iv].
\end{proof}

We can also says that, for a product $X\times_S Y$ of $S$-preschemes, if \emph{one} of the factors
\oldpage[I]{147}
$X$ or $Y$ is \emph{of finite type} over $S$ and \emph{the other is Noetherian} (resp. \emph{locally Noetherian}), then $X\times_S Y$ is \emph{Noetherian} (resp. \emph{locally Noetherian}).

\begin{cor}[6.3.9]
\label{1.6.3.9}
Let $X$ be a prescheme of finite type over a locally Noetherian prescheme $S$.
Then every $S$-morphism $f:X\to Y$ is of finite type.
\end{cor}

\begin{proof}
\label{proof-1.6.3.9}
In fact, we can assume that $S$ is Noetherian;
if $\vphi:X\to S$ and $\psi:Y\to S$ are the structure morphisms, then we have $\vphi=\psi\circ f$, and $X$ is Noetherian by \sref{1.6.3.7};
$f$ is thus of finite type by \sref{1.6.3.6}.
\end{proof}

\begin{prop}[6.3.10]
\label{1.6.3.10}
Let $f:X\to Y$ be a morphism of finite type.
For $f$ to be surjective, it is necessary and sufficient that, for every \emph{algebraically closed} field $\Omega$, the map $X(\Omega)\to Y(\Omega)$ that corresponds to $f$ \sref{1.3.4.1} be surjective.
\end{prop}

\begin{proof}
\label{proof-1.6.3.10}
The condition suffices, as we can see by considering, for all $y\in Y$, an algebraically closed extension $\Omega$ of $\kres(y)$, and the commutative diagram
\[
  \begin{tikzcd}[row sep=tiny]
    X\ar[dd,swap,"f"]\\
    & \Spec(\Omega)\ar[ul]\ar[dl]\\
    Y
  \end{tikzcd}
\]
(cf. \sref{1.3.5.3}).
Conversely, suppose that $f$ is surjective, and let $g:\{\xi\}=\Spec(\Omega)\to Y$ be a morphism, where $\Omega$ is an algebraically closed field.
If we consider the diagram
\[
  \begin{tikzcd}
    X \ar[d,swap,"f"] &
    X_{(\Omega)}\ar[l]\ar[d,"f_{(\Omega)}"]\\
    Y &
    \Spec(\Omega)\ar[l]
  \end{tikzcd}
\]
then it suffices to show that there exists a \emph{rational point over $\Omega$} in $X_{(\Omega)}$ (\sref{1.3.3.14}, \sref{1.3.4.3}, and \sref{1.3.4.4}).
Since $f$ is surjective, $X_{(\Omega)}$ is nonempty \sref{1.3.5.10}, and since $f$ is of finite type, so too is $f_{(\Omega)}$ \sref{1.6.4.3}[iv];
thus $X_{(\Omega)}$ contains a nonempty affine open subset $Z$ such that $A(Z)$ is an non-null algebra of finite type over $\Omega$.
By Hilbert's zero theorem \cite{I-21}, there exists an $\Omega$-homomorphism $A(Z)\to\Omega$, and thus a section of $X_{(\Omega)}$ over $\Spec(\Omega)$, which proves the proposition.
\end{proof}

\subsection{Algebraic preschemes}
\label{subsection-algebraic-preschemes}

\begin{defn}[6.4.1]
\label{1.6.4.1}
Given a field $K$, we define an algebraic $K$-prescheme to be a prescheme $X$ of finite type over $K$;
$K$ is called the base field of $X$.
If, further, $X$ is a scheme \emph{(or if $X$ is a \emph{$K$-scheme}, which amounts to the same thing \sref{1.5.5.8})}, we say that $X$ is an algebraic $K$-scheme.
\end{defn}

Every algebraic $K$-prescheme is \emph{Noetherian} \sref{1.6.3.7}.

\begin{prop}[6.4.2]
\label{1.6.4.2}
Let $X$ be an algebraic $K$-prescheme.
For a point $x\in X$ to be closed, it is necessary and sufficient that $\kres(x)$ be an algebraic extension of $K$ of finite degree.
\end{prop}

\begin{proof}
\label{proof-1.6.4.2}
We can assume that $X$ is affine, with the ring $A$ of $X$ being a $K$-algebra of finite type.
Indeed, the affine open subsets $U$ of $X$ such that $A(U)$ is a $K$-algebra of finite type form a covering of $X$ \sref{1.6.3.1}.
The closed points of $X$ are thus the points such that $\fk{j}_x$ is a
\oldpage[I]{148}
maximal ideal of $A$, or, in other words, such that $A/\fk{j}_x$ is a field (necessarily equal to $\kres(x)$).
Since $A/\fk{j}_x$ is a $K$-algebra of finite type, we see that, if $x$ is closed, then $\kres(x)$ is a field that is an algebra of finite type over $K$, and so necessarily a $K$-algebra of \emph{finite rank} \cite{I-21}.
Conversely, if $\kres(x)$ is of finite rank over $K$, so too is $A/\fk{j}_x\subset\kres(x)$, and since every integral ring that is also a $K$-algebra of finite rank is a field, we have that $A/\fk{j}_x=\kres(x)$, and so $x$ is closed.
\end{proof}

\begin{cor}[6.4.3]
\label{1.6.4.3}
Let $K$ be an algebraically-closed field, and $X$ an algebraic $K$-prescheme;
the closed points of $X$ are then the rational points over $K$ \sref{1.3.4.4} and can be canonically identified with the points of $X$ with values in $K$.
\end{cor}

\begin{prop}[6.4.4]
\label{1.6.4.4}
Let $X$ be an algebraic prescheme over a field $K$.
The following properties are equivalent.
\begin{enumerate}
    \item[\emph{(a)}] $X$ is Artinian.
    \item[\emph{(b)}] The underlying space of $X$ is discrete.
    \item[\emph{(c)}] The underlying space of $X$ has only a finite number of closed points.
    \item[\emph{(c')}] The underlying space of $X$ is finite.
    \item[\emph{(d)}] The points of $X$ are closed.
    \item[\emph{(e)}] $X$ is isomorphic to $\Spec(A)$, where $A$ is a $K$-algebra of finite rank.
\end{enumerate}
\end{prop}

\begin{proof}
\label{proof-1.6.4.4}
Since $X$ is Noetherian, it follows from \sref{1.6.2.2} that the conditions (a), (b), and (d) are equivalent, and imply (c) and (c');
it is also clear that (e) implies (a).
It remains to see that (c) implies (d) and (e);
we can restrict to the case where $X$ is affine.
Then $A(X)$ is a $K$-algebra of finite type \sref{1.6.3.3}, and thus a Jacobson ring (\cite[p.~3-11 and 3-12]{I-1}), in which there are, by hypothesis, only a finite number of maximal ideals.
Since a finite intersection of prime ideals can only be a prime ideal if it is equal to one of the prime ideals being intersected, every prime ideal of $A(X)$ is thus maximal, whence (d).
Further, we then know \sref{1.6.2.2} that $A(X)$ is an Artinian $K$-algebra of finite type, and so necessarily of \emph{finite rank} \cite{I-21}.
\end{proof}

\begin{env}[6.4.5]
\label{1.6.4.5}
When the conditions of \sref{1.6.4.4} are satisfied, we say that $X$ is a scheme \emph{finite over $K$} (cf. \sref[II]{2.6.1.1}), or a \emph{finite $K$-scheme}, of \emph{rank} $[A:K]$, which we also denote by $\rg_K(X)$;
if $X$ and $Y$ are finite $K$-schemes, we have
\begin{gather*}
\rg_K(X\sqcup Y)=\rg_K(X)+\rg_K(Y)\tag{6.4.5.1}\\
\rg_K(X\times_K Y)=\rg_K(X)\rg_K(Y)\tag{6.4.5.2}
\end{gather*}
as a result of \sref{1.3.2.2}.
\end{env}

\begin{cor}[6.4.6]
\label{1.6.4.6}
Let $X$ be a finite $K$-scheme.
For every extension $K'$ of $K$, $X\otimes_K K'$ as a finite $K'$-scheme, and its rank over $K'$ is equal to the rank of $X$ over $K$.
\end{cor}

\begin{proof}
\label{proof-1.6.4.6}
Indeed, if $A=A(X)$, then we have $[A\otimes_K K':K']=[A:K]$.
\end{proof}

\begin{cor}[6.4.7]
\label{1.6.4.7}
Let $X$ be a scheme finite over a field $K$;
we let $n=\sum_{x\in X}[\kres(X):K]_S$ \emph{(we recall that if $K'$ is an extension of $K$, then $[K':K]_S$ is the \emph{separable rank} of $K'$ over $k$, the rank of the largest algebraic separable extension of $K$ contained in $K'$)};
then
\oldpage[I]{149}
for every algebraically closed extension $\Omega$ of $K$, the underlying space of $X\otimes_K\Omega$ has exactly $n$ points, which can be identified with the points of $X$ with values in $\Omega$.
\end{cor}

\begin{proof}
\label{proof-1.6.4.7}
We can clearly restrict to the case where the ring $A=A(X)$ is \emph{local} \sref{1.6.2.2};
let $\fk{m}$ be its maximal ideal, and $L=A/\fk{m}$ its residue field, an algebraic extension of $K$.
The points of $X$ with values in $\Omega$ then correspond, bijectively, to the $\Omega$-sections of $X\otimes_K\Omega$ (\sref{1.3.4.1} and \sref{1.3.3.14}), and also to the $K$-homomorphisms from $L$ to $\Omega$ \sref{1.1.7.3}, whence the proposition (Bourbaki, \emph{Alg.}, chap.~V, §7, n\textsuperscript{o}~5, prop.~8), taking \sref{1.6.4.3} into account.
\end{proof}

\begin{env}[6.4.8]
\label{1.6.4.8}
The number $n$ defined in \sref{1.6.4.7} is called the \emph{separable rank} of $A$ (or of $X$) over $K$, or also the \emph{geometric number of points} of $X$;
it is equal to the number of elements of $X(\Omega)_K$.
It follows immediately from this definition that, for every extension $K'$ of $K$, $X\otimes_K K'$ has the same geometric number of points as $X$.
If we denote this number by $n(X)$, it is clear that, if $X$ and $Y$ are two schemes, finite over $K$, then
\[
    n(X\sqcup Y)=n(X)+n(Y).\tag{6.4.8.1}
\]
Under the same hypotheses, we also have
\[
    n(X\times_K Y)=n(X)n(Y)\tag{6.4.8.2}
\]
because of the interpretation of $n(X)$ as the number of elements of $X(\Omega)_K$ and \hyperref[1.3.4.3]{equation~(3.4.3.1)}.
\end{env}

\begin{prop}[6.4.9]
\label{1.6.4.9}
Let $K$ be a field, $X$ and $Y$ algebraic $K$-preschemes, $f:X\to Y$ a $K$-morphism, and $\Omega$ an algebraically closed extension of $K$ of infinite transcendence degree over $K$.
For $f$ to be surjective, it is necessary and sufficient that the map $X(\Omega)_K\to Y(\Omega)_K$ that corresponds to $f$ \sref{1.3.4.1} be surjective.
\end{prop}

\begin{proof}
\label{proof-1.6.4.9}
The necessity follows from \sref{1.6.3.10}, noting that $f$ is necessarily of finite type \sref{1.6.3.9}.
To see that the condition is sufficient, we argue as in \sref{1.6.3.10}, noting that, for every $y\in Y$, $\kres(y)$ is an extension of $K$ of finite type, and so is $K$-isomorphic to a subfield of $\Omega$.
\end{proof}

\begin{rmk}[6.4.10]
\label{1.6.4.10}
We will see in chapter~IV that the conclusion of \sref{1.6.4.9} still holds without the hypothesis on the transcendence degree of $\Omega$ over $K$.
\end{rmk}

\begin{prop}[6.4.11]
\label{1.6.4.11}
If $f:X\to Y$ is a morphism of finite type, then, for every $y\in Y$, the fibre $f^{-1}(y)$ is an algebraic prescheme over the residue field $\kres(y)$, and for every $x\in f^{-1}(y)$, $\kres(x)$ is an extension of $\kres(x)$ of finite type.
\end{prop}

\begin{proof}
\label{proof-1.6.4.11}
Since $f^{-1}(y)=X\otimes_Y\kres(y)$ \sref{1.6.3.6}, the proposition follows from \sref{1.6.3.4}[iv] and \sref{1.6.3.3}.
\end{proof}

\begin{prop}[6.4.12]
\label{1.6.4.12}
Let $f:X\to Y$ and $g:Y'\to Y$ be morphisms;
set $X'=X\times_Y Y'$, and let $f'=f_{(Y')}:X'\to Y'$.
Let $y'\in Y'$ and set $y=g(y')$;
if the fibre $f^{-1}(y)$ is a finite algebraic scheme over $\kres(y)$, then the fibre $f'^{-1}(y)$ is a finite algebraic scheme over $\kres(y')$, and has the same rank and geometric number of points as $f^{-1}(y)$ does.
\end{prop}

\begin{proof}
\label{proof-1.6.4.12}
Taking into account the transitivity of fibres \sref{1.3.6.5}, this follows immediately from \sref{1.6.4.6} and \sref{1.6.4.8}.
\end{proof}

\begin{env}[6.4.13]
\label{1.6.4.13}
Proposition~\sref{1.6.4.11} shows that the morphisms of finite type that correspond, intuitively, to the ``algebraic families of algebraic varieties'', with the points of $Y$ playing the
\oldpage[I]{150}
role of ``parameters'', which gives these morphisms a ``geometric'' meaning.
The morphisms which are not of finite type will show up in the following mostly in questions of ``changing the base prescheme'', by localisation or completion, for example.
\end{env}

\subsection{Local determination of a morphism}

\begin{prop}[6.5.1]
\label{1.6.5.1}
Let $X$ and $Y$ be $S$-preschemes, with $Y$ of finite type over $S$;
let $x\in X$ and $y\in Y$ lie over the same point $s\in S$.
\begin{enumerate}[label=\emph{(\roman*)}]
    \item If two $S$-morphisms $f=(\psi,\theta)$ and $f'=(\psi',\theta')$ from $X$ to $Y$ are such that $\psi(x)=\psi'(x)=y$, and the (local) $\OO_s$-homomorphisms $\theta_x^\sharp$ and ${\theta'}_x^\sharp$ from $\OO_y$ to $\OO_x$ are identical, then $f$ and $f'$ agree on an open neighbourhood of $x$.
    \item Suppose further that $S$ is locally Noetherian.
        For every local $\OO_s$-homomorphism $\vphi:\OO_y\to\OO_x$, there exists an open neighbourhood $U$ of $x$ in $X$, and an $S$-morphism $f=(\psi,\theta)$ from $U$ to $Y$ such that $\psi(x)=y$ and $\theta_x^\sharp=\vphi$.
\end{enumerate}
\end{prop}

\begin{proof}
\label{proof-1.6.5.1}
\medskip\noindent
\begin{enumerate}[label=(\roman*)]
    \item Since the question is local on $S$, $X$, and $Y$, we can assume that $S$, $X$, and $Y$ are affine, given by rings $A$, $B$, and $C$ (respectively), and with $f$ and $f'$ of the form $({}^a\vphi,\wt{\vphi})$ and $({}^a\vphi',\wt{\vphi}')$ (respectively), where $\vphi$ and $\vphi'$ are $A$-homomorphisms from $C$ to $B$ such that $\vphi^{-1}(\fk{j}_x)=\vphi'^{-1}(\fk{j}_x)=\fk{j}_y$, and the homomorphisms $\vphi_x$ and $\vphi'_x$ from $C_y$ to $B_x$, induced by $\vphi$ and $\vphi'$, are identical;
        we can further suppose that $C$ is an $A$-algebra \emph{of finite type}.
        Let $c_i$ ($1\leqslant i\leqslant n$) be the generators of the $A$-algebra $C$, and set $b_i=\vphi(c_i)$ and $b'_i=\vphi'(c_i)$;
        by hypothesis, we have $b_i/1=b'_i/1$ in the ring of fractions $B_x$ ($1\leqslant i\leqslant n$).
        This implies that there exist elements $s_i\in B\setminus\fk{j}_x$ such that $s_i(b_i-b'_i)=0$ for $1\leqslant i\leqslant n$, and we can clearly assume that all the $s_i$ are equal to a single element $g\in B\setminus\fk{j}_x$.
        From this, we conclude that we have $b_i/1=b'_i/1$ for $1\leqslant i\leqslant n$ in the ring of fractions $B_g$;
        if $i_g$ is the canonical homomorphism $B\to B_g$, we then have $i_g\circ\vphi=i_g\circ\vphi'$;
        so the restrictions of $f$ and $f'$ to $D(g)$ are identical.
    \item We can restrict to the situation as in (i), and further assume that the ring $A$ is Noetherian.
        Let $c_i$ ($1\leqslant i\leqslant n$) be the generators of the $A$-algebra $C$, and let $\alpha:A[X_1,\ldots,X_n]\to C$ be the homomorphism of polynomial algebras that sends $X_i$ to $c_i$ for $1\leqslant i\leqslant n$.
        Also let $i_y$ be the canonical homomorphism $C\to C_y$, and consider the composite homomorphism
        \[
            \beta:A[X_1,\ldots,X_n] \xrightarrow{\alpha} C \xrightarrow{i_y} C_y \xrightarrow{\vphi} B_x.
        \]
        We denote by $\fk{a}$ the kernel of $\beta$;
        since $A$ is Noetherian, so too is $A[X_1,\ldots,X_n]$, and so $\fk{a}$ admits a finite system of generators $Q_j(X_1,\ldots,X_n)$ ($1\leqslant j\leqslant m$).
        Furthermore, each of the elements $\vphi(i_y(c_i))$ can be written in the form $b_i/s_i$, where $b_i\in B$ and $s_i\not\in\fk{j}_x$;
        \oldpage[I]{151}
        we can further assume that all of the $s_i$ are equal to a single element $g\in B\setminus\fk{j}_x$.
        With this, by hypothesis, we have $Q_j(b_1/g,\ldots,b_n/g)=0$ in $B_x$;
        set
        \[
            Q_j(X_1/T,\ldots,X_n/T)=P_j(X_1,\ldots,X_n,T)/T^{k_j}
        \]
        where $P_j$ is homogeneous of degree $k_j$.
        Then let $d_j=P_j(b_1,\ldots,b_n,g)\in B$.
        By hypothesis, we have $t_jd_j=0$ for some $t_j\in B\setminus\fk{j}_x$ ($1\leqslant j\leqslant m$), and we can clearly assume that all the $t_j$ are equal to a single element $h\in B\setminus\fk{j}_x$;
        from this we conclude that $P_j(hb_1,\ldots,hb_n,hg)=0$ for $1\leqslant j\leqslant m$.
        With this, consider the homomorphism $\rho$ from $A[X_1,\ldots,X_n]$ to the ring of fractions $B_{hg}$ which sends $X_i$ to $hb_i/hg$ ($1\leqslant i\leqslant n$);
        the image of $\fk{a}$ under this homomorphism is $0$, and is \emph{a fortiori} the same as the image of the kernel $\alpha^{-1}(0)$ under $\rho$.
        So $\rho$ factors as $A[X_1,\ldots,X_n]\xrightarrow{\alpha}C\xrightarrow{\gamma}B_{hg}$, with $\gamma(c_i)=hb_i/hg$, and it is clear that, if $i_x$ is the canonical homomorphism $B_{hg}\to B_x$, then the diagram
        \[
          \begin{tikzcd}
            C\ar[r,"\gamma"]\ar[d,swap,"i_y"] &
            B_{hg}\ar[d,"i_x"]\\
            C_y\ar[r,"\vphi"] &
            B_x
          \end{tikzcd}
          \tag{6.5.1.1}
        \]
        is commutative; we thus have $\vphi=\gamma_x$, and since $\vphi$ is a local homomorphism, $^a\gamma(x)=y$;
        $f=({}^a\gamma,\wt{\gamma})$ is thus an $S$-morphism from the neighbourhood $D(hg)$ of $x$ to $Y$ as claimed in the proposition.
\end{enumerate}
\end{proof}

\begin{cor}[6.5.2]
\label{1.6.5.2}
Under the hypotheses of \sref{1.6.5.1}[ii], if, further, $X$ is of finite type over $S$, then we can assume that the morphism $f$ is of finite type.
\end{cor}

\begin{proof}
\label{proof-1.6.5.2}
This follows from \sref{1.6.3.6}.
\end{proof}

\begin{cor}[6.5.3]
\label{1.6.5.3}
Suppose that the hypotheses of \sref{1.6.5.1}[ii], and suppose further that $Y$ is integral, and that $\vphi$ is an injective homomorphism.
Then we can assume that $f=({}^a\gamma,\wt{\gamma})$, where $\gamma$ is injective.
\end{cor}

\begin{proof}
\label{proof-1.6.5.3}
Indeed, we can assume $C$ to be integral \sref{1.5.1.4}, and so $i_y$ to be injective;
it then follows from the diagram in \hyperref[1.6.5.1]{(6.5.1.1)} that $\gamma$ is injective.
\end{proof}

\begin{prop}[6.5.4]
\label{1.6.5.4}
Let $f=(\psi,\theta):X\to Y$ be a morphism of finite type, $x$ a point of $X$, and $y=\psi(x)$.
\begin{enumerate}[label=\emph{(\roman*)}]
  \item For $f$ to be a local immersion at the point $x$ \sref{1.4.5.1}, it is necessary and sufficient that $\theta_x^\sharp:\OO_y\to\OO_x$ be surjective.
  \item Assume further that $Y$ is locally Noetherian.
    For $f$ to be a local isomorphism at the point $x$ \sref{1.4.5.2}, it is necessary and sufficient that $\theta_x^\sharp$ be an isomorphism.
\end{enumerate}
\end{prop}

\begin{proof}
\label{proof-1.6.5.4}
\medskip\noindent
\begin{enumerate}
    \item[(ii)] By \sref{1.6.5.1}, there exists an open neighbourhood $V$ of $Y$ and a morphism $g:V\to X$ such that $g\circ f$ (resp. $f\circ g$) is defined and agrees with the identity on a neighbourhood of $x$ (resp. $y$), whence we can easily see that $f$ is a local isomorphism.
    \item[(i)] Since the question is local on $X$ and $Y$, we can assume that $X$ and $Y$ are affine, given by rings $A$ and $B$ (respectively);
    we have $f=({}^a\vphi,\wt{\vphi})$, where $\vphi$ is a homomorphism of rings $B\to A$ that makes $A$ a $B$-algebra of finite type;
    we have $\vphi^{-1}(\fk{j}_x)=\fk{j}_y$, and the homomorphism $\vphi_x:B_y\to A_x$ induced by $\vphi$ is \emph{surjective}.
    Let $(t_i)$ ($1\leqslant i\leqslant n$) be a system of generators of the $B$-algebra $A$;
    the hypothesis on $\vphi_x$ implies that there exist $b_i\in B$ and some $c\in B\setminus\fk{j}_x$ such that, in the ring of fractions $A_x$, we have $t_i/1=\vphi(b_i)/\vphi(c)$ for $1\leqslant i\leqslant n$.
    Then \sref{1.1.3.3} there exists some $a\in A\setminus\fk{j}_x$ such that, if we let $g=a\vphi(c)$, we also have $t_i/1=a\vphi(b_i)/g$ \emph{in the ring of fractions $A_g$}.
    With this, there exists, by hypothesis, a polynomial $Q(X_1,\ldots,X_n)$, with coefficients in the ring $\vphi(B)$, such that $a=Q(t_1,\ldots,t_n)$;
    let $Q(X_1/T,\ldots,X_n/T)=P(X_1,\ldots,X_n,T)/T^m$, where $P$ is homogeneous of degree $m$.
    In the ring $A_g$, we have
    \[
      a/1=a^mP(\vphi(b_1),\ldots,\vphi(b_n),\vphi(c))/g^m=a^m\vphi(d)/g^m
    \]
    where $d\in B$.
    Since, in $A_g$, $g/1=(a/1)(\vphi(c)/1)$ is invertible by definition, so too are $a/1$ and $\vphi(c)/1$, and we can thus write $a/1=(\vphi(d)/1)(\vphi(c)/1)^{-m}$.
    From this we conclude that $\vphi(d)/1$ is also invertible in $A_g$.
    So let $h=cd$;
    since $\vphi(h)/1$ is invertible in $A_g$, the composite homomorphism $B\xrightarrow{\vphi}A\to A_g$ factors as $B\to B_h\xrightarrow{\gamma}A_g$ \sref[0]{0.1.2.4}.
    We will show that $\gamma$ is \emph{surjective};
    it suffices to show that the image of $B_h$ in $A_g$ contains the $t_i/1$ and $(g/1)^{-1}$.
    But we have $(g/1)^{-1}=(\vphi(c)/1)^{m-1}(\vphi(d)/1)^{-1}=\gamma(c^m/h)$, and $a/1=\gamma(d^{m+1}/h^m)$, so $(a\vphi(b_i))/1=\gamma(b_id^{m+1}/h^m)$, and since $t_i/1=(a\vphi(b_i)/1)(g/1)^{-1}$, our claim is proved.
    The choice of $h$ implies that $\psi(D(g))\subset D(h)$, and we also know that the restriction of $f$ to $D(g)$ is equal to $({}^a\gamma,\wt{\gamma})$;
    since $\gamma$ is surjective, this restriction is a closed immersion of $D(g)$ into $D(h)$ \sref{1.4.2.3}.
\end{enumerate}
\end{proof}

\begin{cor}[6.5.5]
\label{1.6.5.5}
Let $f=(\psi,\theta):X\to Y$ be a morphism of finite type.
Assume that $X$ is irreducible, and denote by $x$ its generic point, and let $y=\psi(x)$.
\begin{enumerate}[label=\emph{(\roman*)}]
  \item For $f$ to be a local immersion at any point of $X$, it is necessary and sufficient that $\theta_x^\sharp:\OO_y\to\OO_x$ be surjective.
  \item Assume further that $Y$ is irreducible and locally Noetherian.
    For $f$ to be a local isomorphism at any point of $X$, it is necessary and sufficient that $y$ be the generic point of $Y$ \emph{(or, equivalently \sref[0]{0.2.1.4}, that $f$ be a \emph{dominant} morphism)} and that $\theta_x^\sharp$ be an isomorphism \emph{(in other words, that $f$ be \emph{birational} \sref{1.2.2.9})}.
\end{enumerate}
\end{cor}

\begin{proof}
\label{proof-1.6.5.5}
It is clear that (i) follows from \sref{1.6.5.4}[i], taking into account the fact that every nonempty open subset of $X$ contains $x$;
similarly, (ii) follows from \sref{1.6.5.4}[ii].
\end{proof}

\subsection{Quasi-compact morphisms and morphisms locally of finite type}
\label{subsection-quasi-compact-morphisms-and-morphisms-locally-of-finite-type}

\begin{defn}[6.6.1]
\label{1.6.6.1}
We say that a morphism $f: X\to Y$ is quasi-compact if the inverse image of any quasi-compact open subset of $Y$ under $f$ is quasi-compact.
\end{defn}

Let $\fk{B}$ be a base of the topology of $Y$ consisting of quasi-compact open subsets (for example, open affine subsets);
for $f$ to be quasi-compact, it is necessary and sufficient that the inverse image of every set of $\fk{B}$ under $f$ be quasi-compact (or, equivalently, a \emph{finite} union of affine open subsets), because every quasi-compact open subset of $Y$ is a finite union of sets of $\fk{B}$.
For example, if $X$ is \emph{quasi-compact} and $Y$ \emph{affine}, then \emph{every} morphism $f: X\to Y$ is quasi-compact:
indeed, $X$ is a finite union of affine open subsets $U_i$, and for every affine open subset $V$ of $Y$, $U_i\cap f^{-1}(V)$ is affine \sref{1.5.5.10}, and so quasi-compact.

If $f: X\to Y$ is a quasi-compact morphism, it is clear that, for every open subset $V$ of $Y$, the restriction of $f$ to $f^{-1}(V)$ is a quasi-compact morphism $f^{-1}(V)\to V$.
Conversely, if $(U_\alpha)$ is an open cover of $Y$, and $f: X\to Y$ a morphism such that the restrictions $f^{-1}(U_\alpha)\to U_\alpha$ are quasi-compact, then $f$ is quasi-compact.

\begin{defn}[6.6.2]
\label{1.6.6.2}
We say that a morphism $f: X\to Y$ is locally of finite type if, for every $x\in X$, there exists an open neighbourhood $U$ of $x$ and an open neighbourhood $V\supset f(U)$ of $y$ such that the restriction of $f$ to $U$ is a morphism of finite type from $U$ to $V$.
We then also say that $X$ is a prescheme locally of finite type over $Y$, or a $Y$-prescheme locally of finite type.
\end{defn}

It follows immediately from \sref{1.6.3.2} that, if $f$ is locally of finite type, then, for every open subset $W$ of $Y$, the restriction of $f$ to $f^{-1}(W)$ is a morphism $f^{-1}(W)\to W$ that is locally of finite type.

If $Y$ is locally Noetherian and $X$ locally of finite type over $Y$, then $X$ is locally Noetherian thanks to \sref{1.6.3.7}.

\begin{prop}[6.6.3]
\label{1.6.6.3}
For a morphism $f: X\to Y$ to be of finite type, it is necessary and sufficient that it be quasi-compact and locally of finite type.
\end{prop}

\begin{proof}
\label{proof-1.6.6.3}
The necessity of the conditions is immediate, given \sref{1.6.3.1} and the remark following \sref{1.6.6.1}.
Conversely, suppose that the conditions are satisfied, and let $U$ be an affine open subset of $Y$, given by some ring $A$;
for all $x\in f^{-1}(U)$, there is, by hypothesis, a neighbourhood $V(x)\subset f^{-1}(U)$ of $x$, and a neighbourhood $W(x)\subset U$ of $y=f(x)$ containing $f(V(x))$, and such that the restriction of $f$ to $V(x)$ is a morphism $V(x)\to W(x)$ of finite type.
Replacing $W(x)$ with a neighbourhood $W_1(x)\subset W(x)$ of $x$ of the form $D(g)$ (with $g\in A$), and $V(x)$ with $V(x)\cap f^{-1}(W_1(x))$, we can assume that $W(x)$ is of the form $D(g)$, and thus of finite type over $U$ (because its ring can be written as $A[1/g]$);
so $V(x)$ is of finite type over $U$.
Further, $f^{-1}(U)$ is quasi-compact by hypothesis, and so the finite union of open subsets $V(x_i)$, which finishes the proof.
\end{proof}

\begin{prop}[6.6.4]
\label{1.6.6.4}
\begin{enumerate}[label=\emph{(\alph*)}]
    \item An immersion $X\to Y$ is quasi-compact if it is closed, or if the underlying space of $Y$ is locally Noetherian, or if the underlying space of $X$ is Noetherian.
    \item The composition of two quasi-compact morphisms is quasi-compact.
    \item If $f: X\to Y$ is a quasi-compact $S$-morphism, then so too is $f_{(S')}: X_{(S')}\to Y_{(S')}$ for any extension $g: S\to S'$ of the base prescheme.
    \item If $f:X\to X'$ and $g:Y\to Y'$ are two quasi-compact $S$-morphisms, then $f\times_S g$ is quasi-compact.
    \item If the composition of two morphisms $f: X\to Y$ and $g:Y\to Z$ is quasi-compact, and if either $g$ is separated or the underlying space of $X$ is locally Noetherian, then $f$ is quasi-compact.
    \item For a morphism $f$ to be quasi-compact, it is necessary and sufficient that $f_\mathrm{red}$ be quasi-compact.
\end{enumerate}
\end{prop}

\begin{proof}
\label{proof-1.6.6.4}
We note that (vi) is evident because the property of being quasi-compact, for a morphism, depends only on the corresponding continuous map of underlying spaces.
We will similarly prove the part of (v) corresponding to the case where the underlying space of $X$ is locally Noetherian.
Set $h=g\circ f$, and let $U$ be a quasi-compact open subset of $Y$;
$g(U)$ is quasi-compact (but not necessarily open) in $Z$, and so contained in a finite union of quasi-compact open subsets $V_j$ \sref{1.2.1.3}, and $f^{-1}(U)$ is thus contained in the union of the $h^{-1}(V_j)$, which are quasi-compact subspaces of $X$, and thus Noetherian subspaces.
We thus conclude \sref[0]{0.2.2.3} that $f^{-1}(U)$ is a Noetherian space, and \emph{a fortiori} quasi-compact.

To prove the other claims, it suffices to prove (i), (ii), and (iii) \sref{1.5.5.12}.
But (ii) is evident, and (i) follows from \sref{1.6.3.5} whenever the space $Y$ is locally Noetherian or the space $X$ is Noetherian, and is evident for a closed immersion.
To show (iii), we can restrict to the case where $S=Y$ \sref{1.3.3.11};
let $f'=f_{(S')}$, and let $U'$ be a quasi-compact open subset of $S'$.
For every $s'\in U'$, let $T$ be an affine open neighbourhood of $g(s')$ in $S$, and let $W$ be an affine open neighbourhood of $s'$ contained in $U'\cap g^{-1}(T)$;
it will suffice to show that $f'^{-1}(W)$ is quasi-compact;
in other words, we can restrict to showing that, when $S$ and $S'$ are \emph{affine}, the underlying space of $X\times_S S'$ is quasi-compact.
But since $X$ is then, by hypothesis, a finite union of affine open subsets $V_j$, $X\times_S S'$ is a union of the underlying spaces of the affine schemes $V_j\times_S S'$ (\sref{1.3.2.2} and \sref{1.3.2.7}), which proves the proposition.
\end{proof}

We note also that, if $X=X'\sqcup X''$ is the sum of two preschemes, a morphism $f:X\to Y$ is quasi-compact if and only if its restrictions to both $X'$ and $X''$ are quasi-compact.

\begin{prop}[6.6.5]
\label{1.6.6.5}
Let $f:X\to Y$ be a quasi-compact morphism.
For $f$ to be dominant, it is necessary and sufficient that, for every generic point $y$ of an irreducible component of $Y$, $f^{-1}(y)$ contain the generic point of an irreducible component of $X$.
\end{prop}

\begin{proof}
\label{proof-1.6.6.5}
It is immediate that the condition is sufficient (even without assuming that $f$ is quasi-compact).
To see that it is necessary, consider an affine open neighbourhood $U$ of $y$;
$f^{-1}(U)$ is quasi-compact, and so a \emph{finite} union of affine open subsets $V_i$, and the hypothesis that $f$ be dominant implies that $y$ belongs to the closure \emph{in $U$} of one of the $f(V_i)$.
We can clearly assume $X$ and $Y$ to be reduced;
since the closure in $X$ of an irreducible component of $V_i$ is an irreducible component on $X$ \sref[0]{0.2.1.6}, we can replace $X$ by $V_i$, and $Y$ by the closed reduced subprescheme of $U$ that has $\overline{f(V_i)}\cap U$ as its underlying space \sref{1.5.2.1}, and we are thus led to proving the proposition when $X=\Spec(A)$ and $Y=\Spec(B)$ are affine and reduced.
Since $f$ is dominant, $B$ is a subring of $A$ \sref{1.1.2.7}, and the proposition then follows from the fact that every minimal prime ideal of $B$ is the intersection of $B$ with a minimal prime ideal of $A$ \sref[0]{0.1.5.8}.
\end{proof}

\begin{prop}[6.6.6]
\label{1.6.6.6}
\begin{enumerate}[label=\emph{(\alph*)}]
    \item Every local immersion is locally of finite type.
    \item If two morphisms $f:X\to Y$ and $g:Y\to Z$ are locally of finite type, then so too is $g\circ f$.
    \item If $f:X\to Y$ is an $S$-morphism locally of finite type, then $f_{(S')}:X_{(S')}\to Y_{(S')}$ is locally of finite type for any extension $S'\to S$ of the base prescheme.
    \item If $f:X\to X'$ and $g:Y\to Y'$ are $S$-morphisms locally of finite type, then $f\times_S g$ is locally of finite type.
    \item If the composition $g\circ f$ of two morphisms is locally of finite type, then $f$ is locally of finite type.
    \item If a morphism $f$ is locally of finite type, then so too is $f_\mathrm{red}$.
\end{enumerate}
\end{prop}

\begin{proof}
\label{proof-1.6.6.6}
By \sref{1.5.5.12}, it suffices to prove (i), (ii), and (iii).
If $j:X\to Y$ is a local immersion then, for every $x\in X$, there is an open neighbourhood $V$ of $j(x)$ in $Y$ and an open neighbourhood $U$ of $x$ in $X$ such that the restriction of $j$ to $U$ is a closed immersion $U\to V$ \sref{1.4.5.1}, and so this restriction is of finite type.
To prove (ii), consider a point $x\in X$;
by hypothesis, there is an open neighbourhood $W$ of $g(f(x))$ and an open neighbourhood $V$ of $f(x)$ such that $g(V)\subset W$ and such that $V$ is of of finite type over $W$;
furthermore, $f^{-1}(V)$ is locally of finite type over $V$ \sref{1.6.6.2}, so there is an open neighbourhood $U$ of $x$ that is contained in $f^{-1}(V)$ and of is finite type over $V$;
thus we have $g(f(U))\subset W$, and that $U$ is of finite type over $W$ \sref{1.6.3.4}[ii].
Finally, to prove (iii), we can restrict to the case where $Y=S$ \sref{1.3.3.11};
for every $x'\in X'=X_{(S')}$, let $x$ be the image of $x'$ in $X$, $s$ the image of $x$ in $S$, $T$ an open neighbourhood of $s$, $T'$ the inverse image of $T$ in $S'$, and $U$ an open neighbourhood of $x$ that is of finite type over $T$ and whose image is contained in $T$;
then $U\times_S T'=U\times_T T'$ is an open neighbourhood of $x'$ \sref{1.3.2.7} that is of finite type over $T'$ \sref{1.6.3.4}[iv].
\end{proof}

\begin{cor}[6.6.7]
\label{1.6.6.7}
Let $X$ and $Y$ be $S$-preschemes that are locally of finite type over $S$.
If $S$ is locally Noetherian, then $X\times_S Y$ is locally Noetherian.
\end{cor}

\begin{proof}
\label{proof-1.6.6.7}
Indeed, $X$ being locally of finite type over $S$ means that it is locally Noetherian, and that $X\times_S Y$ is locally of finite type over $X$, and so $X\times_S Y$ is also locally Noetherian.
\end{proof}

\begin{rmk}[6.6.8]
\label{1.6.6.8}
Proposition~\sref{1.6.3.10} and its proof extend immediately to the case where we suppose only that the morphism $f$ is locally of finite type.
Similarly, propositions~\sref{6.4.2} and \sref{6.4.9} hold true when we suppose only that the preschemes $X$ and $Y$ in the claim are locally of finite type over the field $K$.
\end{rmk}
