\section{Reduced preschemes; separation condition}
\label{section-reduced-preschemes-and-separation-condition}

\subsection{Reduced preschemes}
\label{subsection-reduced-preschemes}

\begin{prop}[5.1.1]
\label{1.5.1.1}
Let $(X,\OO_X)$ be a prescheme, and $\sh{B}$ a quasi-coherent $\OO_X$-algebra.
There exists a unique quasi-coherent $\OO_X$-module $\sh{N}$ whose stalk $\sh{N}_x$ at each $x\in X$ is the nilradical of the ring $\sh{B}_x$.
When $X$ is affine, and, consequently, $\sh{B}=\wt{B}$, where $B$ is an algebra over $A(X)$, then we have $\sh{N}=\wt{\nilrad}$, where $\nilrad$ is the nilradical of $B$.
\end{prop}

\begin{proof}
\label{proof-1.5.1.1}
\oldpage[I]{128}
The statement is local, so it suffices to show the latter claim.
We know that $\wt{\nilrad}$ is a quasi-coherent $\OO_X$-module \sref{1.1.4.1} and that its stalk at a point $x\in X$ is the ideal $\nilrad_x$ of the ring of fractions $B_x$;
it remains to prove that the nilradical of $B_x$ is contained in $\nilrad_x$, the converse inclusion being evident.
Let $z/s$ be an element of the nilradical of $B_x$, with $z\in B$, $s\not\in\fk{j}_x$;
by hypothesis, there exists an integer $k$ such that $(z/s)^k=0$, which implies that there exists some $t\not\in\fk{j}_x$ such that $tz^k=0$.
We conclude that $(tz)^k=0$, and, as a result, that $z/s=(tz)/(ts)\in\nilrad_x$.
\end{proof}

We say that the quasi-coherent $\OO_X$-module $\sh{N}$ thus defined is the \emph{nilradical} of the $\OO_X$-algebra $\sh{B}$; in particular, we denote by $\sh{N}_X$ the nilradical of $\OO_X$.

\begin{cor}[5.1.2]
\label{1.5.1.2}
Let $X$ be a prescheme;
the closed subprescheme of $X$ defined by the sheaf of ideals $\sh{N}_X$ is the only reduced subprescheme \sref[0]{0.4.1.4} of $X$ that has $X$ as its underlying space;
it is also the smallest subprescheme of $X$ that has $X$ as its underlying space.
\end{cor}

\begin{proof}
\label{proof-1.5.1.2}
Since the structure sheaf of the closed subprescheme of $Y$ defined by $\sh{N}_X$ is $\OO_X/\sh{N}_X$, it is immediate that $Y$ is reduced and has $X$ as its underlying space, because $\sh{N}_x\neq\OO_x$ for any $x\in X$.
To show the other claims, note that a subprescheme $Z$ of $X$ that has $X$ as its underlying space is defined by a sheaf of ideals $\sh{I}$ \sref{1.4.1.3} such that $\sh{I}_x\neq\OO_x$ for any $x\in X$.
We can restrict to the case where $X$ is affine, say $X=\Spec(A)$ and $\sh{I}=\wt{\fk{I}}$, where $\fk{I}$ is an ideal of $A$;
then, for every $x\in X$, we have $\fk{I}_x\subset\fk{j}_x$, and so $\fk{I}$ is contained in every prime ideal of $A$, and so also in their intersection $\nilrad$, the nilradical of $A$.
This proves that $Y$ is the small subprescheme of $X$ that has $X$ as its underlying space \sref{1.4.1.9};
furthermore, if $Z$ is distinct from $Y$, we necessarily have $\sh{I}_x\neq\sh{N}_x$ for at least one $x\in X$, and so \sref{1.5.1.1} $Z$ is not reduced.
\end{proof}

\begin{defn}[5.1.3]
\label{1.5.1.3}
We define the reduced prescheme associated to a prescheme $X$, denoted by $X_\mathrm{red}$, to be the unique reduced subprescheme of $X$ that has $X$ as its underlying space.
\end{defn}

Saying that a prescheme $X$ is reduced thus implies that $X=X_\mathrm{red}$.

\begin{prop}[5.1.4]
\label{1.5.1.4}
For the prime spectrum of a ring $A$ to be a reduced (\emph{resp.} integral) prescheme \sref{1.2.1.7}, it is necessary and sufficient for $A$ to be a reduced (\emph{resp.} integral) ring.
\end{prop}

\begin{proof}
\label{proof-1.5.1.4}
Indeed, it follows immediately from \sref{1.5.1.1} that the condition $\sh{N}=(0)$ is necessary and sufficient for $X=\Spec(A)$ to be reduced;
the claim relating to integral rings is then a consequence of \sref{1.1.1.13}.
\end{proof}

Since every ring of fractions $\neq\{0\}$ of an integral ring is integral, it follows from \sref{1.5.1.4} that, for every \emph{locally integral} prescheme $X$, $\OO_x$ is an \emph{integral} ring for every $x\in X$.
The converse is true whenever the underlying space of $X$ is \emph{locally Noetherian}:
indeed, $X$ is then reduced, and if $U$ is an affine open subset of $X$, which is a Noetherian space, then $U$ has only a finite number of irreducible components, and so its ring $A$ has only a finite number of minimal prime ideals \sref{1.1.1.14}.
If two of these components $U_i$ had a common point $x$, then $\OO_x$ would have at least two distinct minimal prime ideals, and would thus not be integral;
the $U_i$ are thus open subsets that are pairwise disjoint, and each of them is thus integral.

\begin{env}[5.1.5]
\label{1.5.1.5}
Let $f=(\psi,\theta):X\to Y$ be a morphism of preschemes;
\oldpage[I]{129}
the homomorphism $\theta_x^\sharp:\OO_{\psi(x)}\to\OO_x$ sends each nilpotent element of $\OO_{\psi(x)}$ to a nilpotent element of $\OO_x$;
by passing to the quotients, $\theta^\sharp$ induces a homomorphism
\[
  \omega:\psi^*(\OO_Y/\sh{N}_Y)\to\OO_X/\sh{N}_X.
\]
It is clear that, for every $x\in X$, $\omega_x:\OO_{\psi(x)}/\sh{N}_{\psi(x)}\to\OO_x/\sh{N}_x$ is a local homomorphism, and so $(\psi,\omega^\flat)$ is a morphism of preschemes $X_\mathrm{red}\to Y_\mathrm{red}$, which we denote by $f_\mathrm{red}$, and call the \emph{reduced} morphism associated to $f$.
It is immediate that, for two morphisms $f:X\to Y$ and $g:Y\to Z$, we have $(g\circ f)_\mathrm{red}=g_\mathrm{red}\circ f_\mathrm{red}$, and so we have defined $X_\mathrm{red}$ as a \emph{covariant functor} in $X$.

The preceding definition shows that the diagram
\[
  \xymatrix{
    X_\mathrm{red}\ar[r]^{f_\mathrm{red}}\ar[d] &
    Y_\mathrm{red}\ar[d]\\
    X\ar[r]_f &
    Y
  }
\]
is commutative, where the vertical arrows are the injection morphisms;
in other words, $X_\mathrm{red}\to X$ is a \emph{functorial} morphism.
We note in particular that if $X$ is reduced, then every morphism $f:X\to Y$ factors as $X\xrightarrow{f_\mathrm{red}}Y_\mathrm{red}\to Y$;
in other words, $f$ is \unsure{bounded above} by the injection morphism $Y_\mathrm{red}\to Y$.
\end{env}

\begin{prop}[5.1.6]
\label{1.5.1.6}
Let $f:X\to Y$ be a morphism;
if $f$ is surjective (\emph{resp.} radicial, an immersion, a closed immersion, an open immersion, a local immersion, a local isomorphism), then so too is $f_\mathrm{red}$.
Conversely, if $f_\mathrm{red}$ is surjective (\emph{resp.} radicial), then so too is $f$.
\end{prop}

\begin{proof}
\label{proof-1.5.1.6}
The proposition is trivial if $f$ is surjective;
if $f$ is radicial, the proposition follows from the fact that, for every $x\in X$, the field $\kres(x)$ is the same for the preschemes $X$ and $X_\mathrm{red}$ \sref{1.3.5.8}.
Finally, if $f=(\psi,\theta)$ is an immersion, a closed immersion, of a local immersion (\emph{resp.} an open immersion, or a local isomorphism), then the proposition follows from the fact that if $\theta_x^\sharp$ is surjective (\emph{resp.} bijective) then so too is the homomorphism obtained by passing to the quotients by the nilradicals $\OO_{\psi(x)}$ and $\OO_x$ (\sref{1.5.1.2} and \sref{1.4.2.2}) (cf. \sref{1.5.5.12}).
\end{proof}

\begin{prop}[5.1.7]
\label{1.5.1.7}
If $X$ and $Y$ are $S$-preschemes, then the preschemes $X_\mathrm{red}\times_{S_\mathrm{red}}Y_\mathrm{red}$ and $X_\mathrm{red}\times_S Y_\mathrm{red}$ are identical, and are canonically identified with a subprescheme of $X\times_S Y$ that has the same underlying subspace as the two aforementioned products.
\end{prop}

\begin{proof}
\label{proof-1.5.1.7}
The canonical identification of $X_\mathrm{red}\times_S Y_\mathrm{red}$ with a subprescheme of $X\times_S Y$ that has the same underlying space follows from \sref{1.4.3.1}.
Furthermore, if $\vphi$ and $\psi$ are the structure morphisms $X_\mathrm{red}\to S$ and $Y_\mathrm{red}\to S$ (respectively), then they factor through $S_\mathrm{red}$ \sref{1.5.1.5}, and since $S_\mathrm{red}\to S$ is a monomorphism, the first claim of the proposition follows from \sref{1.3.2.4}.
\end{proof}

\begin{cor}[5.1.8]
\label{1.5.1.8}
The preschemes $(X\times_S Y)_\mathrm{red}$ and $(X_\mathrm{red}\times_{S_\mathrm{red}}Y_\mathrm{red})_\mathrm{red}$ are canonically identified.
\end{cor}

\begin{proof}
\label{proof-1.5.1.8}
This follows from \sref{1.5.1.2} and \sref{1.5.1.7}.
\end{proof}

We note that even if $X$ and $Y$ are reduced preschemes, $X\times_S Y$ might not be, because the tensor product of two reduced algebras can have nilpotent elements.

\begin{prop}[5.1.9]
\label{1.5.1.9}
\oldpage[I]{130}
Let $X$ be a prescheme, and $\sh{I}$ a quasi-coherent sheaf of ideals of $\OO_X$ such that $\sh{I}^n=0$ for some integer $n>0$.
Let $X_0$ be the closed subprescheme $(X,\OO_X/\sh{I})$ of $X$;
for $X$ to be an affine scheme, it is necessary and sufficient for $X_0$ to be an affine scheme.
\end{prop}

The condition is clearly necessary, so we will show that it is sufficient.
If we set $X_k=(X,\OO_X/\sh{I}^{k+1})$, it is enough to prove by induction on $k$ that $X_k$ is affine, and so we are led to consider the base case where $\sh{I}^2=0$.
We set
\begin{gather*}
    A = \Gamma(X,\OO_X)\\
    A_0 = \Gamma(X_0,\OO_{X_0})=\Gamma(X,\OO_X/\sh{I}).
\end{gather*}
The canonical homomorphism $\OO_X\to\OO_X/\sh{I}$ induces a homomorphism of rings $\vphi:A\to A_0$.
We will see below that $\vphi$ is \emph{surjective}, so that
\begin{equation*}
    0 \to \Gamma(X,\sh{I}) \to \Gamma(X,\OO_X) \to \Gamma(X,\OO_X/\sh{I}) \to 0
\end{equation*}
is an \emph{exact} sequence.
We now prove, assuming that this is true, the proposition.
Note that $\fk{R}=\Gamma(X,\sh{I})$ is an ideal whose square is zero in $A$, and thus a module over $A_0=A/\fk{R}$.
By hypothesis, we have $X_0=\Spec(A)$, and, since the underlying spaces of $X_0$ and $X$ are identical, $\fk{R}=\Gamma(X_0,\sh{I})$;
Additionally, since $\sh{I}^2=0$, $\sh{I}$ is a quasi-coherent $(\OO_X/\sh{I})$-module, so we have $\sh{I}\cong\wt{\fk{R}}$ and $\fk{R}_x=\sh{I}_x$ for all $x\in X_0$ \sref{1.1.4.1}.
With this in mind, let $X'=\Spec(A)$, and consider the morphism $f=(\psi,\theta):X\to X'$ of preschemes that corresponds to the identity map $A\to\Gamma(X,\OO_X)$ \sref{1.2.2.4}.
For every affine open subset $V$ of $X$, the diagram
\[
  \xymatrix{
    A \ar[r] \ar[d]
    & \Gamma(V,\OO_X|V) \ar[d]\\
    A_0=A/\fk{R} \ar[r]
    & \Gamma(V,\OO_{X_0}|V)
  }
\]
is commutative, whence the diagram
\[
  \xymatrix{
    X'
    & X \ar[l]_f\\
    X'_0 \ar[u]^{j'}
    & X_0 \ar[l]^{f_0} \ar[u]_j
  }
\]
is also commutative, $X'_0$ being the closed subprescheme of $X'$ defined by the quasi-coherent sheaf of ideals $\wt{R}$, and $j$ and $j'$ being the canonical injection morphisms.
But since $X_0$ is affine, $f_0$ is an isomorphism, and since the underlying continuous maps of $j$ and $j'$ are the identity maps, we see straight away that $\psi:X\to X'$ is a homeomorphism.
Furthermore, the relation $\fk{R}_x=\sh{I}_x$ shows that the restriction of $\theta^\sharp:\psi^*(\OO_{X'})\to\OO_X$ is an \emph{isomorphism} from $\psi^*(\wt{\fk{R}})$ to $\sh{I}$;
additionally, by passing to the quotients, $\theta^\sharp$ gives an \emph{isomorphism} $\psi^*(\OO_X/\wt{\fk{R}})\to\OO_X/\sh{I}$, because $f_0$ is an isomorphism;
we thus immediately conclude, by the \unsure{lemma of 5} (M,~I,~1.1), that $\theta^\sharp$ is itself an isomorphism, and thus that $f$ is an \emph{isomorphism}, and hence $X$ is affine.
So everything relies on proving the exactitude of \hyperref[1.5.1.9]{(5.1.9.1)}, which will follow from showing that $\HH^1(X,\sh{I})=0$.
But $\HH^1(X,\sh{I})=\HH^1(X_0,\sh{I})$, and we have seen that $\sh{I}$ is a quasi coherent $\OO_{X_0}$-module.
Our proof will thus follow from
\begin{lem}[5.1.9.2]
\label{1.5.1.9.2}
If $Y$ is an affine scheme, and $\sh{F}$ a quasi-coherent $\OO_Y$-module, then $\HH^1(Y,\sh{F})=0$.
\end{lem}

\begin{proof}
\label{proof-1.5.1.9}
This lemma will be proven in chap.~III, §1, and a consequence of the more general theorem that $\HH^i(Y,\sh{F})=0$ for all $i>0$.
To give an independent proof, note that $\HH^1(Y,\sh{F})$ can be identified with the module $\Ext_{\OO_Y}^1(Y;\OO_Y,\sh{F})$ of extensions classes of the $\OO_Y$-module $\OO_Y$ by the $\OO_Y$-module $\sh{F}$ (T,~4.2.3);
so everything relies on proving that such an extension $\sh{G}$ is trivial.
But, for all $y\in Y$, there is a neighbourhood $V$ of $y$ in $Y$ such that $\sh{G}|V$ is isomorphic to $\sh{F}|Y\oplus\OO_Y|V$ \sref[0]{1.5.4.9};
from this we conclude that $\sh{G}$ is a \emph{quasi-coherent} $\OO_Y$-module.
If $A$ is the ring of $Y$, then we have $\sh{F}=\wt{M}$ and $\sh{G}=\wt{N}$, where $M$ and $N$ are $A$-modules, and, by hypothesis, $N$ is an extension of the $A$-module $A$ by the $A$-module $M$ \sref{1.1.3.11}.
Since this extension is necessarily trivial, the lemma is proven, and so too is \sref{1.5.1.9}.
\end{proof}

\begin{cor}[5.1.10]
\label{1.5.1.10}
Let $X$ be a prescheme such that $\sh{N}_X$ is nilpotent.
For $X$ to be an affine scheme, it is necessary and sufficient for $X_\mathrm{red}$ to be an affine scheme.
\end{cor}

\subsection{Existence of a subprescheme with a given underlying space}
\label{subsection-existence-of-a-subprescheme-with-a-given-underlying-space}
