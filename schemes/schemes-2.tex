\section{Preschemes and morphisms of preschemes}
\label{section-preschemes-and-morphisms}

\subsection{Definition of preschemes}
\label{subsection-preschemes-definition}

\begin{env}[2.1.1]
\label{env-1.2.1.1}
Given a ringed space $(X,\OO_X)$, we say that an open subset
$V$ of $X$ is an {\em affine open} if the ringed space $(V,\OO_X|V)$ is an
affine scheme \hyperref[defn-1.1.7.1]{(1.7.1)}.
\end{env}

\begin{defn}[2.1.2]
\label{defn-1.2.1.2}
We define a prescheme to be a ringed space
$(X,\OO_X)$ such that every point of $X$ admits an affine open neighborhood.
\end{defn}

\begin{prop}[2.1.3]
\label{prop-1.2.1.3}
\oldpage{98}
If $(X,\OO_X)$ is a prescheme then the affine opens give a basis for the topology of $X$.
\end{prop}

\begin{proof}
\label{proof-prop-1.2.1.3}
If $V$ is an arbitrary open neighborhood of $x\in X$, then there
exists by hypothesis an open neighborhood $W$ of $x$ such that $(W,\OO_X|W)$ is
an affine scheme; we write $A$ to mean its ring. In the space $W$, $V\cap W$ is
an open neighborhood of $x$; thus there exists $f\in A$ such that $D(f)$ is an
open neighborhood of $x$ contained inside $V\cap W$ \hyperref[prop-1.1.1.10]{(1.1.10)}, (i).
The ringed space $(D(f),\OO_X|D(f))$ is thus an affine scheme, isomorphic to $A_f$
\hyperref[prop-1.1.3.6]{(1.3.6)}, whence the proposition.
\end{proof}

\begin{prop}[2.1.4]
\label{prop-1.2.1.4}
The underlying space of a prescheme is a Kolmogoroff space.
\end{prop}

\begin{proof}
\label{proof-prop-1.2.1.4}
If $x$, $y$ are two distinct points of a prescheme $X$ then it is clear
that there exists an open neighborhood of one of these points that does not
contain the other if $x$ and $y$ are not in the same affine open; and if they
are in the same affine open, this is a result of \hyperref[cor-1.1.1.8]{(1.1.8)}.
\end{proof}

\begin{prop}[2.1.5]
\label{prop-1.2.1.5}
If $(X,\OO_X)$ is a prescheme then every closed
irreducible subset of $X$ admits exactly one generic point, and the map
$x\mapsto\overline{\{x\}}$ is thus a bijection of $X$ onto its set of closed
irreducible subsets.
\end{prop}

\begin{proof}
\label{proof-prop-1.2.1.5}
If $Y$ is a closed irreducible subset of $X$ and $y\in Y$, and if $U$
is an open affine neighborhood of $y$ in $X$, then $U\cap Y$ is everywhere
dense in $Y$, as well as irreducible (\hyperref[env-0.2.1.1]{(\textbf{0},~2.1.1)} and \hyperref[env-0.2.1.4]{(\textbf{0},~2.1.4)}); thus by
Corollary \hyperref[cor-1.1.1.14]{(1.1.14)}, $U\cap Y$ is the closure in $U$ of a point $x$, and then
$Y\subset\overline{U}$ is the closure of $x$ in $X$. The uniqueness of the
generic point of $X$ is a result of Proposition \hyperref[prop-1.2.1.4]{(2.1.4)} and of \hyperref[env-0.2.1.3]{(\textbf{0},~2.1.3)}.
\end{proof}

\begin{env}[2.1.6]
\label{env-1.2.1.6}
If $Y$ is a closed irreducible subset of $X$ and $y$ its
generic point then the local ring $\OO_y$, also written $\OO_{X/Y}$, is called the
{\em local ring of $X$ along $Y$}, or the {\em local ring of $Y$ in $X$}.

If $X$ itself is irreducible and $x$ its generic point then we say that
$\OO_x$ is the {\em ring of rational functions on $X$} (cf. \textsection7).
\end{env}

\begin{prop}[2.1.7]
\label{prop-1.2.1.7}
If $(X,\OO_X)$ is a prescheme then the ringed
space $(U,\OO_X|U)$ is a prescheme for every open subset $U$.
\end{prop}

\begin{proof}
\label{proof-prop-1.2.1.7}
This follows directly from Definition \hyperref[defn-1.2.1.2]{(2.1.2)} and Proposition \hyperref[prop-1.2.1.3]{(2.1.3)}.
\end{proof}

We say that $(U,\OO_X|U)$ is the prescheme {\em induced} on $U$ by
$(X,\OO_X)$, or the {\em restriction} of $(X,\OO_X)$ to $U$.

\begin{env}[2.1.8]
\label{env-1.2.1.8}
We say that a prescheme $(X,\OO_X)$ is {\em irreducible}
(resp. {\em connected}) if the underlying space $X$ is irreducible (resp.
connected). We say that a prescheme is {\em integral} if it is
{\em irreducible and reduced} (cf. \hyperref[env-1.5.1.4]{(5.1.4)}). We say that a prescheme
$(X,\OO_X)$ is {\em locally integral} if each $x\in X$ admits an open
neighborhood $U$ such that the prescheme induced on $U$ by $(X,\OO_X)$ is integral.
\end{env}

\subsection{Morphisms of preschemes}
\label{subsection-prescheme-morphisms}

\begin{defn}[2.2.1]
\label{defn-1.2.2.1}
Given two preschemes $(X,\OO_X)$, $(Y,\OO_Y)$, we
define a morphism (of preschemes) of $(X,\OO_X)$ to $(Y,\OO_Y)$ to be a morphism
of ringed spaces $(\psi,\theta)$ such that, for all $x\in X$, $\theta_x^\sharp$ is a
local homomorphism $\OO_{\psi(x)}\to\OO_x$.
\end{defn}

\oldpage{99}
By passing to quotients, the map $\OO_{\psi(x)}\to\OO_x$ gives us a monomorphism
$\theta^x:\kres(\psi(x))\to\kres(x)$, which lets us consider $\kres(x)$ as an
{\em extension} of the field $\kres(\psi(x))$.

\begin{env}[2.2.2]
\label{env-1.2.2.2}
The composition $(\psi'',\theta'')$ of two morphisms
$(\psi,\theta)$, $(\psi',\theta')$ of preschemes is also a morphism of
preschemes, since it is given by the formula
${\theta''}^\sharp=\theta^\sharp\circ\psi^*({\theta'}^\sharp)$ \hyperref[env-0.3.5.5]{(\textbf{0},~3.5.5)}. From this
we conclude that preschemes form a {\em category}; using the usual notation, we
will write $\Hom(X,Y)$ to mean the set of morphisms from a prescheme $X$ to a
prescheme $Y$.
\end{env}

\begin{exm}[2.2.3]
\label{exm-1.2.2.3}
If $U$ is an open subset of $X$ then the canonical
injection \hyperref[env-0.4.1.2]{(\textbf{0},~4.1.2)} of the induced prescheme $(U,\OO_X|U)$ into
$(X,\OO_X)$ is a morphism of preschemes; it is further a {\em monomorphism} of
ringed spaces (and {\em a fortiori} a monomorphism of preschemes), which
quickly follows from \hyperref[env-0.4.1.1]{(\textbf{0},~4.1.1)}.
\end{exm}

\begin{prop}[2.2.4]
\label{prop-1.2.2.4}
Let $(X,\OO_X)$ be a prescheme, and $(S,\OO_S)$ an
affine scheme associated to a ring $A$. Then there exists a canonical bijective
correspondence between morphisms of preschemes from $(X,\OO_X)$ to $(S,\OO_S)$ and
ring homomorphisms from $A$ to $\Gamma(X,\OO_X)$.
\end{prop}

Note first that, if $(X,\OO_X)$ and $(Y,\OO_Y)$ are two arbitrary ringed spaces,
a morphism $(\psi,\theta)$ from $(X,\OO_X)$ to $(Y,\OO_Y)$ canonically defines a ring
homomorphism
$\Gamma(\theta):\Gamma(Y,\OO_Y)\to\Gamma(Y,\psi_*(\OO_X))=\Gamma(X,\OO_X)$.
In the case that we consider, everything boils down to showing that any
homomorphism $\vphi:A\to\Gamma(X,\OO_X)$ is of the form $\Gamma(\theta)$
for one and only one $\theta$. Now, by hypothesis there is a covering
$(V_\alpha)$ of $X$ by affine opens; by composing of $\vphi$ with the
restriction homomorphism $\Gamma(X,\OO_X)\to\Gamma(V_\alpha,\OO_X|V_\alpha)$ we
obtain a homomorphism $\vphi_\alpha:A\to\Gamma(V_\alpha,\OO_X|V_\alpha)$
that corresponds to a unique morphism $(\psi_\alpha,\theta_\alpha)$ from the
prescheme $(V_\alpha,\OO_X|V_\alpha)$ to $(S,\OO_S)$, according to Theorem \hyperref[thm-1.1.7.3]{(1.7.3)}.
Furthermore, for each pair of indices $(\alpha,\beta)$, each point of
$V_\alpha\cap V_\beta$ admits an open affine neighborhood $W$ contained inside
$V_\alpha\cap V_\beta$ \hyperref[prop-1.2.1.3]{(2.1.3)}; it is clear that by composing
$\vphi_\alpha$ and $\vphi_\beta$ with the restriction homomorphisms to $W$,
we obtain the same homomorphism $\Gamma(S,\OO_S)\to\Gamma(W,\OO_X|W)$, so, thanks
to the relations $(\theta_\alpha^\sharp)_x=(\vphi_\alpha)_x$ for all $x\in
V_\alpha$ and all $\alpha$ \hyperref[env-1.1.6.1]{(1.6.1)}, the restriction to $W$ of the morphisms
$(\psi_\alpha,\theta_\alpha)$ and $(\psi_\beta,\theta_\beta)$ coincide. From
this we conclude that there is a morphism
$(\psi,\theta):(X,\OO_X)\to(S,\OO_S)$ of ringed spaces, and only one such
that its restriction to each $V_\alpha$ is $(\psi_\alpha,\theta_\alpha)$, and it
is clear that this morphism is a morphism of preschemes and such that
$\Gamma(\theta)=\vphi$.

Let $u:A\to\Gamma(X,\OO_X)$ be a ring homomorphism, and $v=(\psi,\theta)$
the corresponding morphism $(X,\OO_X)\to(S,\OO_S)$. For each $f\in A$ we have
that
\[
  \psi^{-1}(D(f))=X_{u(f)}
  \tag{2.2.4.1}
\]
with the notation of \hyperref[env-0.5.5.2]{(\textbf{0},~5.5.2)} relative to the locally free sheaf
$\OO_X$. In fact, it suffices to verify this formula when $X$ itself is affine,
and then this is nothing but (1.2.2.2).

\begin{prop}[2.2.5]
\label{prop-1.2.2.5}
Under the hypotheses of Proposition \hyperref[prop-1.2.2.4]{(2.2.4)}, let
$\vphi:A\to\Gamma(X,\OO_X)$ be a ring homomorphism,
$f:(X,\OO_X)\to(S,\OO_S)$ the corresponding morphism of preschemes,
$\sh{G}$ (resp. $\sh{F}$) an $\OO_X$-module (resp. $\OO_S$-module), and
$M=\Gamma(S,\sh{F})$. Then there exists a canonical bijective
\oldpage{100}
correspondence between $f$-morphisms $\sh{F}\to\sh{G}$ \hyperref[env-0.4.4.1]{(\textbf{0},~4.4.1)} and
$A$-homomorphisms $M\to(\Gamma(X,\sh{G}))_{[\vphi]}$.
\end{prop}

\begin{proof}
\label{proof-prop-1.2.2.5}
Reasoning as in Proposition \hyperref[prop-1.2.2.4]{(2.2.4)}, we reduce to the case where $X$ is
affine, and the proposition then follows from Proposition \hyperref[prop-1.1.6.3]{(1.6.3)}
and from Corollary \hyperref[cor-1.1.3.8]{(1.3.8)}.
\end{proof}

\begin{env}[2.2.6]
\label{env-1.2.2.6}
We say that a morphism of preschemes
$(\psi,\theta):(X,\OO_X)\to(Y,\OO_Y)$ is {\em open} (resp. {\em closed})
if, for all open subsets $U$ of $X$ (resp. all closed subsets $F$ of $X$),
$\psi(U)$ is open (resp. $\psi(F)$ is closed) in $Y$. We say that
$(\psi,\theta)$ is {\em dominant} if $\psi(X)$ is dense in $Y$, and
{\em surjective} if $\psi$ is surjective. We will point out that these
conditions rely only on the continuous map $\psi$.
\end{env}

\begin{prop}[2.2.7]
\label{prop-1.2.2.7}
Let
\[
  f=(\psi,\theta):(X,\OO_X)\to(Y,\OO_Y);
\]
\[
  g=(\psi',\theta'):(Y,\OO_Y)\to(Z,\OO_Z)
\]
be two morphisms of preschemes.
\begin{itemize}
  \item[{\rm(i)}] If $f$ and $g$ are both open (resp. closed, dominant, surjective),
    then so is $g\circ f$.
  \item[{\rm(ii)}] If $f$ is surjective and $g\circ f$ closed, then $g$ is closed.
  \item[{\rm(iii)}] If $g\circ f$ is surjective, then $g$ is surjective.
\end{itemize}
\end{prop}

\begin{proof}
\label{proof-prop-1.2.2.7}
Claims (i) and (iii) are evident. Write $g\circ f=(\psi'',\theta'')$.
If $F$ is closed in $Y$ then $\psi^{-1}(F)$ is closed in $X$, so
$\psi''(\psi^{-1}(F))$ is closed in $Z$; but since $\psi$ is surjective,
$\psi(\psi^{-1}(F))=F$, so $\psi''(\psi^{-1}(F))=\psi'(F)$, which proves (ii).
\end{proof}

\begin{prop}[2.2.8]
\label{prop-1.2.2.8}
Let $f=(\psi,\theta)$ be a morphism
$(X,\OO_X)\to(Y,\OO_Y)$, and $(U_\alpha)$ an open cover of $Y$. For $f$ to be
open (resp. closed, surjective, dominant), it is necessary and sufficient that
its restriction to each induced prescheme
$(\psi^{-1}(U_\alpha),\OO_X|\psi^{-1}(U_\alpha))$, considered as a morphism of
preschemes from this induced prescheme to the induced prescheme
$(U_\alpha,\OO_Y|U_\alpha)$ is open (resp. closed, surjective, dominant).
\end{prop}

\begin{proof}
\label{proof-prop-1.2.2.8}
The proposition follows immediately from the definitions, taking into
account the fact that a subset $F$ of $Y$ is closed (resp. open, dense) in $Y$
if and only if each of the $F\cap U_\alpha$ are closed (resp. open, dense) in
$U_\alpha$.
\end{proof}

\begin{env}[2.2.9]
\label{env-1.2.2.9}
Let $(X,\OO_X)$ and $(Y,\OO_Y)$ be two preschemes; suppose that
$X$ (resp. $Y$) has a finite number of irreducible components $X_i$ (resp.
$Y_i$) ($1\leqslant i\leqslant n$); let $\xi_i$ (resp. $\eta_i$) be the generic
point of $X_i$ (resp. $Y_i$) \hyperref[prop-1.2.1.5]{(2.1.5)}. We say that a morphism
\[
  f=(\psi,\theta):(X,\OO_X)\to(Y,\OO_Y)
\]
is {\em birational} if, for all $i$, $\psi^{-1}(\eta_i)=\{\xi_i\}$ and
$\theta_{\xi_i}^\sharp:\OO_{\eta_i}\to\OO_{\xi_i}$ is an {\em isomorphism}. It
is clear that a birational morphism is dominant \hyperref[env-0.2.1.8]{(\textbf{0},~2.1.8)}, and
thus it is surjective if it is also closed.
\end{env}

\begin{nota}[2.2.10]
\label{nota-1.2.2.10}
In all that follows, when there is
no risk of confusion, we {\em suppress} the structure sheaf (resp. the morphism
of structure sheaves) from the notation of a prescheme (resp. morphism of
preschemes). If $U$ is an open subset of the underlying space $X$ of a
prescheme, then whenever we speak of $U$ as a prescheme we always mean the
induced prescheme on $U$.
\end{nota}

\subsection{Gluing preschemes}
\label{subsection-gluing-preschemes}

\begin{env}[2.3.1]
\label{env-1.2.3.1}
\oldpage{101}
It follows from definition \hyperref[defn-1.2.1.2]{(2.1.2)} that every ringed space obtained by
{\em gluing} preschemes \hyperref[env-0.4.1.6]{(\textbf{0},~4.1.6)} is again a prescheme. In particular, although
every prescheme admits, by definition, a cover by affine open sets, we see that every
prescheme can actually be obtained by {\em gluing affine schemes}.
\end{env}

\begin{env}[Example]{2.3.2}
\label{exm-1.2.3.2}
Let $K$ be a field, and $B=K[s]$, $C=K[t]$ be two polynomial rings in one indeterminate over
$K$, and define $X_1=\Spec(B)$, $X_2=\Spec(C)$, which are two isomorphic affine schemes. In
$X_1$ (resp. $X_2$), let $U_{12}$ (resp. $U_{21}$) be the affine open $D(s)$ (resp. $D(t)$)
where the ring $B_s$ (resp. $C_t$) is formed of rational fractions of the form $f(s)/s^m$
(resp. $g(t)/t^n$) with $f\in B$ (resp. $g\in C$). Let $u_{12}$ be the isomorphism of
preschemes $U_{21}\to U_{12}$ corresponding \hyperref[prop-1.2.2.4]{(2.2.4)} to the isomorphism from $B$ to
$C$ that, to $f(s)/s^m$, associates the rational fraction $f(1/t)/(1/t^m)$. We can glue $X_1$
and $X_2$ along $U_{12}$ and $U_{21}$ by using $u_{12}$, because there is clearly no gluing
condition. We later show that the prescheme $X$ obtained in this manner is a particular case
of a general method of construction (\textbf{II},~2.4.3). Here we only show that $X$ {\em is
not an affine scheme}; this will follow from the fact that the ring $\Gamma(X,\OO_X)$ is
{\em isomorphic} to $K$, and so its spectrum reduces to a point. Indeed, a section of
$\OO_X$ over $X$ has a restriction over $X_1$ (resp. $X_2$), identified to an affine open of
$X$, that is a polynomial $f(s)$ (resp. $g(t)$), and it follows from the definitions that we
should have $g(t)=f(1/t)$, which is not possible if $f=g\in K$.
\end{env}

\subsection{Local schemes}
\label{subsection-local-schemes}

\begin{env}[2.4.1]
\label{env-1.2.4.1}
We say that an affine scheme is a {\em local scheme} if
it is the affine scheme associated to a local ring $A$; then there exists in
$X=\Spec(A)$ a single {\em closed point $a\in X$}, and for all other $b\in X$
we have that $a\in\overline{\{b\}}$ \hyperref[cor-1.1.1.7]{(1.1.7)}.
\end{env}

For all preschemes $Y$ and points $y\in Y$, the local scheme $\Spec(\OO_y)$
is called the {\em local scheme of $Y$ at the point $y$}. Let $V$ be an affine
open of $Y$ containing $y$, and $B$ the ring of the affine scheme $V$; then
$\OO_y$ is canonically identified with $B_y$ \hyperref[defn-1.1.3.4]{(1.3.4)}, and the canonical
homomorphism $B\to B_y$ thus corresponds \hyperref[env-1.1.6.1]{(1.6.1)} to a morphism of preschemes
$\Spec(\OO_y)\to V$. If we compose this morphism with the canonical injection
$V\to Y$, then we obtain a morphism $\Spec(\OO_y)\to Y$, which is
{\em independent} of the affine open $V$ (containing $y$) that we chose:
indeed, if $V'$ is some other affine open containing $y$, then there exists a
third affine open $W$ containing $y$ and such that $W\subset V\cap V'$ \hyperref[prop-1.2.1.3]{(2.1.3)};
we can thus assume that $V\subset V'$, and then if $B'$ is the ring of $V'$,
everything comes down to remarking that the diagram
\[
  \xymatrix{
    B'\ar[rr]\ar[dr] & & B\ar[dl]\\
    & \OO_y &
  }
\]
is commutative \hyperref[env-0.1.5.1]{(\textbf{0},~1.5.1)}. The morphism
\[
  \Spec(\OO_y)\longrightarrow Y
\]
thus defined is said to be {\it canonical}.
\begin{prop}[2.4.2]
\label{prop-1.2.4.2}
\oldpage{102}
Let $(Y,\OO_Y)$ be a prescheme; for all $y\in Y$, let $(\psi,\theta)$ be the canonical
morphism $(\Spec(\OO_y),\widetilde{\OO}_y)\to(Y,\OO_Y)$. Then $\psi$ is a homeomorphism
from $\Spec(\OO_y)$ to the subspace $S_y$ of $Y$ given by the $z$ such that
$y\in\overline{\{z\}}$ ({\em or, equivalently, the generalizations of $y$
\hyperref[env-0.2.1.2]{(\textbf{0},~2.1.2)}}; furthermore, if $z=\psi(\mathfrak{p})$, then
$\theta_z^\sharp:\OO_z\to(\OO_y)_\mathfrak{p}$ is an isomorphism; $(\psi,\theta)$
is thus a monomorphism of ringed spaces.
\end{prop}

As the unique closed point $a$ of $\Spec(\OO_y)$ is contained in the closure of any
point of this space, and since $\psi(a)=y$, the image of $\Spec(\OO_y)$ under
the continuous map $\psi$ is contained in $S_y$. Since $S_y$ is contained in
every affine open containing $y$, one can consider just the case where $Y$ is an
affine scheme; but then this proposition follows from \hyperref[exm-1.1.6.2]{(1.6.2)}.

{\em We see \hyperref[prop-1.2.1.5]{(2.1.5)} that there is a bijective correspondence between
$\Spec(\OO_y)$ and the set of closed irreducible subsets of $Y$ containing $y$.}

\begin{cor}[2.4.3]
\label{cor-1.2.4.3}
For $y\in Y$ to be the generic point of an
irreducible component of $Y$, it is necessary and sufficient that the only prime
ideal of the local ring $\OO_y$ is its maximal ideal ({\em in other words, that
$\OO_y$ is of {\em dimension zero}}).
\end{cor}

\begin{prop}[2.4.4]
\label{prop-1.2.4.4}
Let $(X,\OO_X)$ be a local scheme of a ring
$A$, $a$ its unique closed point, and $(Y,\OO_Y)$ a prescheme. Every morphism
$u=(\psi,\theta):(X,\OO_X)\to(Y,\OO_Y)$ then factorizes uniquely as
$X\to\Spec(\OO_{\psi(a)})\to Y$, where the second arrow denotes the canonical
morphism, and the first corresponds to a local homomorphism $\OO_{\psi(a)}\to A$.
This establishes a canonical bijective correspondence between the set of
morphisms $(X,\OO_X)\to(Y,\OO_Y)$ and the set of local homomorphisms $\OO_y\to A$
for ($y\in Y$).
\end{prop}

Indeed, for all $x\in X$, we have that $a\in\overline{\{x\}}$, so
$\psi(a)\in\overline{\{\psi(x)\}}$, which shows that $\psi(X)$ is contained in
every affine open containing $\psi(a)$. So it suffices to consider the case
where $(Y,\OO_Y)$ is an affine scheme of ring $B$, and we then have that
$u=({}^a\vphi,\tilde{\vphi})$, where $\vphi\in\Hom(B,A)$ \hyperref[thm-1.1.7.3]{(1.7.3)}. Further,
we have that $\vphi^{-1}(\mathfrak{j}_a)=\mathfrak{j}_{\psi(a)}$, and hence
that the image under $\vphi$ of any element of
$B-\mathfrak{j}_{\psi(a)}$ is invertible in the local ring $A$; the
factorization in the result follows from the universal property of the ring of
fractions \hyperref[env-0.1.2.4]{(\textbf{0},~1.2.4)}. Conversely, to each local homomorphism
$\OO_y\to A$ there exists a unique corresponding morphism
$(\psi,\theta):X\to\Spec(\OO_y)$ such that $\psi(a)=y$ \hyperref[thm-1.1.7.3]{(1.7.3)}, and,
by composing with the canonical morphism $\Spec(\OO_y)\to Y$, we obtain a morphism
$X\to Y$, which proves the proposition.

\begin{env}[2.4.5]
\label{env-1.2.4.5}
The affine schemes whose ring is a field $K$ have an
underlying space that is just a point. If $A$ is a local ring with maximal
ideal $\mathfrak{m}$, then each local homomorphism $A\to K$ has kernel equal to
$\mathfrak{m}$, and so factorizes as $A\to A/\mathfrak{m}\to K$, where the
second arrow is a monomorphism. The morphisms $\Spec(K)\to\Spec(A)$ thus
correspond bijectively to monomorphisms of fields $A/\mathfrak{m}\to K$.
\end{env}

Let $(Y,\OO_Y)$ be a prescheme; for each $y\in Y$ and each ideal
$\mathfrak{a}_y$ of $\OO_y$, the canonical homomorphism
$\OO_y\to\OO_y/\mathfrak{a}_y$ defines a morphism
$\Spec(\OO_y/\mathfrak{a}_y)\to\Spec(\OO_y)$; if we compose this with the
canonical morphism $\Spec(\OO_y)\to Y$, then we obtain a morphism
$\Spec(\OO_y/\mathfrak{a}_y)\to Y$, again said to be \textit{canonical}. For
$\mathfrak{a}_y=\mathfrak{m}_y$, this says that $\OO_y/\mathfrak{a}_y=\kres(y)$, and
so Proposition \hyperref[prop-1.2.4.4]{(2.4.4)} says that:

\begin{cor}[2.4.6]
\label{cor-1.2.4.6}
\oldpage{103}
Let $(X,\OO_X)$ be a local scheme whose ring $K$ is a field, $\xi$ be the unique point of
$X$, and $(Y,\OO_Y)$ a prescheme. Then each morphism $u:(X,\OO_X)\to(Y,\OO_Y)$ factorizes
uniquely as $X\to\Spec(\kres(\psi(\xi)))\to Y$, where the second arrow denotes the canonical
morphism, and the first corresponds to a monomorphism $\kres(\psi(\xi))\to K$.
This establishes a canonical bijective correspondence between the set of
morphisms $(X,\OO_X)\to (Y,\OO_Y)$ and the set of monomorphisms $\kres(y)\to K$ (for
$y\in Y$).
\end{cor}

\begin{cor}[2.4.7]
\label{cor-1.2.4.7}
For all $y\in Y$, each canonical morphism
$\Spec(\OO_y/\mathfrak{a}_y)\to Y$ is a monomorphism of ringed spaces.
\end{cor}

\begin{proof}
\label{proof-cor-1.2.4.7}
We have already seen this when $\mathfrak{a}_y=0$ \hyperref[prop-1.2.4.2]{(2.4.2)}, and it suffices
to apply Corollary \hyperref[cor-1.1.7.5]{(1.7.5)}.
\end{proof}

\begin{rmk}{2.4.8}
\label{rmk-1.2.4.8}
Let $X$ be a local scheme, and $a$ its unique
closed point. Since every affine open containing $a$ is necessarily in the
whole of $X$, every {\em invertible} $\OO_X$-module \hyperref[env-0.5.4.1]{(\textbf{0},~5.4.1)} is
necessarily {\em isomorphic to $\OO_X$} (or, as we say, again, {\em trivial}).
This property does not hold in general for an arbitrary affine scheme
$\Spec(A)$; we will see in Chapter~V that if $A$ is a normal ring then this is
true when $A$ is \unsure{{\em factorial}}.
\end{rmk}

\subsection{Preschemes over a prescheme}
\label{subsection-preschemes-over-prescheme}

\begin{defn}[2.5.1]
\label{defn-1.2.5.1}
Given a prescheme $S$, we say that the data of a
prescheme $X$ and a morphism of preschemes $\vphi:X\to S$ defines a
prescheme $X$ {\em over the prescheme $S$}, or an {\em $S$-prescheme}; we say
that $S$ is the {\em base prescheme} of the $S$-prescheme $X$. The morphism
$\vphi$ is called the {\em structure morphism} of the $S$-prescheme $X$.
When $S$ is an affine scheme of ring $A$, we also say that $X$ endowed with
$\vphi$ is a prescheme {\em over the ring $A$} (or an {\em $A$-prescheme}).
\end{defn}

It follows from \hyperref[prop-1.2.2.4]{(2.2.4)} that the data of a prescheme over a ring $A$ is
equivalent to the data of a prescheme $(X,\OO_X)$ whose structure sheaf $\OO_X$ is
a sheaf of {\em $A$-algebras}. {\em An arbitrary prescheme can always be
considered as a $\bb{Z}$-prescheme in a unique way.}

If $\vphi:X\to S$ is the structure morphism of an $S$-prescheme $X$, we
say that a point $x\in X$ is {\em over a point $s\in S$} if $\vphi(x)=s$. We
say that $X$ {\em dominates} $S$ if $\vphi$ is a dominant morphism \hyperref[env-1.2.2.6]{(2.2.6)}.

\begin{env}[2.5.2]
\label{env-1.2.5.2}
Let $X$ and $Y$ be two $S$-preschemes; we say that a morphism
of preschemes $u:X\to Y$ is a {\em morphism of preschemes over $S$} (or
an {\em $S$-morphism}) if the diagram
\[
  \xymatrix{
    X \ar[rr]^u \ar[dr] & & Y\ar[dl]\\
    & S &
  }
\]
(where the diagonal arrows are the structure morphisms) is
commutative: this ensures that, for all $s\in S$ and $x\in X$ over $s$, $u(x)$
is also above $s$.
\end{env}

From this definition it follows immediately that the composition of two
$S$-morphisms is an $S$-morphism; $S$-preschemes thus form a {\em category}.

We denote by $\Hom_S(X,Y)$ the set of $S$-morphisms from an $S$-prescheme $X$ to
an $S$-prescheme $Y$; the identity morphism of an $S$-prescheme is denoted by
$1_X$.

When $S$ is an affine scheme of ring $A$, we will also say {\em $A$-morphism}
instead of $S$-morphism.

\begin{env}[2.5.3]
\label{env-1.2.5.3}
\oldpage{104}
If $X$ is an $S$-prescheme, and $v:X'\to X$ a morphism of preschemes,
then the composition $X'\to X\to S$ endows $X'$ with the structure of an $S$-prescheme;
in particular, every prescheme induced by an open set $U$ of $X$ can be considered as an
$S$-prescheme by the canonical injection.

If $u:X\to Y$ is an $S$-morphism of $S$-preschemes, then the restriction
of $u$ to any prescheme induced by an open subset $U$ of $X$ is also an
$S$-morphism $U\to Y$. Conversely, let $(U_\alpha)$ be an open cover of $X$,
and for each $\alpha$ let $u_\alpha:U_\alpha\to Y$ be an $S$-morphism; if,
for all pairs of indices $(\alpha,\beta)$, the restrictions of $u_\alpha$ and
$u_\beta$ to $U_\alpha\cap U_\beta$ agree, then there exists an $S$-morphism
$X\to Y$, and only one such that the restriction to each $U_\alpha$ is
$u_\alpha$.

If $u:X\to Y$ is an $S$-morphism such that $u(X)\subset V$, where $V$ is
an open subset of $Y$, then $u$, considered as a morphism from $X$ to $V$, is
also an $S$-morphism.
\end{env}

\begin{env}[2.5.4]
\label{env-1.2.5.4}
Let $S'\to S$ be a morphism of preschemes; for all
$S'$-preschemes, the composition $X\to S'\to S$ endows $X$ with the structure of
an $S$-prescheme. Conversely, suppose that $S'$ is the induced prescheme of an
open subset of $S$; let $X$ be an $S$-prescheme and suppose that the structure
morphism $f:X\to S$ is such that $f(X)\subset S'$; then we can consider
$X$ as an $S'$-prescheme. In this latter case, if $Y$ is another $S$-prescheme
whose structure morphism sends the underlying space to $S'$, then every
$S$-morphism from $X$ to $Y$ is also an $S'$-morphism.
\end{env}

\begin{env}[2.5.5]
\label{env-1.2.5.5}
If $X$ is an $S$-prescheme, with structure morphism
$\vphi:X\to S$, we define an {\em $S$-section of $X$} to be an
$S$-morphism from $S$ to $X$, that is to say a morphism of preschemes
$\psi:S\to X$ such that $\vphi\circ\psi$ is the identity on $S$. We
denote by $\Gamma(X/S)$ the set of $S$-sections of $X$.
\end{env}

