\section{Combinatorial dimension of a topological space}
\label{section:combinatorial-dimension-of-a-topological-space}

\oldpage[IV]{6}

\subsection{Combinatorial dimension of a topological space}
\label{subsection:combinatorial-dimension-of-a-topological-space}

\begin{env}[14.1.1]
\label{0.14.1.1}
Let $I$ be an ordered set;
a \emph{chain} of elements of $I$ is, by definition, a strictly-increasing finite sequence $i_0<i_1<\ldots<i_n$ of elements of $I$ ($n\geq0$);
by definition, the \emph{length} of this chain is $n$.
If $X$ is a topological space, the set of its \emph{closed irreducible} subsets is ordered by inclusion, and so we have the notion of a \emph{chain} of closed irreducible subsets of $X$.
\end{env}

\begin{defn}[14.1.2]
\label{0.14.1.2}
Let $X$ be a topological space.
We define the combinatorial dimension of $X$ (or simply the dimension of $X$, if there is no risk of confusion), denoted by $\dimc(X)$ (or simply $\dim(X)$), to be the upper bound of lengths of chains of closed irreducible subsets of $X$.
For all $x\in X$, we define the combinatorial dimension of $X$ at $x$ (or simply the dimension of $X$ at $x$), denoted by $\dim_x(X)$, to be the number $\inf_U(\dim(U))$, where $U$ varies over the open neighbourhoods of $x$ in $X$.
\end{defn}

It follows from this definition that we have
\[
    \dim(\emp) = -\infty
\]
(the upper bound in $\overline{\bb{R}}$ of the empty set being $-\infty$).
If $(X_\alpha)$ is the family of irreducible components of $X$, then we have
\begin{equation*}
\label{0.14.1.2.1}
    \dim(X) = \sup_\alpha\dim(X_\alpha)\tag{14.1.2.1}
\end{equation*}
because every chain of closed irreducible subsets of $X$ is, by definition, contained in some irreducible component of $X$, and, conversely, the irreducible components are closed in $X$, so every closed irreducible subset of an $X_\alpha$ is a closed irreducible subset of $X$.

\begin{defn}[14.1.3]
\label{0.14.1.3}
We say that a topological space $X$ is equidimensional if all its irreducible components have the same dimension (which is thus equal to $\dim(X)$, by \sref{0.14.1.2.1}).
\end{defn}

\begin{prop}[14.1.4]
\label{0.14.1.4}
\medskip\noindent
\begin{enumerate}[label=\emph{(\roman*)}]
    \item For every closed subset $Y$ of a topological space $X$, we have $\dim(Y)\leq\dim(X)$.
    \item If a topological space $X$ is a finite union of closed subsets $X_i$, then we have $\dim(X)=\sup_i\dim(X_i)$.
\end{enumerate}
\end{prop}

\begin{proof}
\label{proof-0.14.1.4}
\end{proof}
