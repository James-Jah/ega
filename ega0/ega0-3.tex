\section{Supplement on sheaves}
\label{section:0.3}

\subsection{Sheaves with values in a category}
\label{subsection:0.3.1}

\begin{env}[3.1.1]
\label{0.3.1.1}
Let $\cat{C}$ be a category, $(A_\alpha)_{\alpha\in I}$,
$(A_{\alpha\beta})_{(\alpha,\beta)\in I\times I}$ two families of objects of
$\cat{C}$ such that $A_{\beta\alpha}=A_{\alpha\beta}$, and
$(\rho_{\alpha\beta})_{(\alpha,\beta)\in I\times I}$ a family of morphisms
$\rho_{\alpha\beta}:A_\alpha\to A_{\alpha\beta}$. We say that a pair consisting
of an object $A$ of $\cat{C}$ and a family of morphisms $\rho_\alpha:A\to A_\alpha$
is a \emph{solution to the universal problem} defined by the data of the
families $(A_\alpha)$, $(A_{\alpha\beta})$, and $(\rho_{\alpha\beta})$ if, for
every object $B$ of $\cat{C}$, the map which sends $f\in\Hom(B,A)$ to the family
$(\rho_\alpha\circ f)\in\Pi_\alpha\Hom(B,A_\alpha)$ is a \emph{bijection} of
$\Hom(B,A)$ to the set of all $(f_\alpha)$ such that
$\rho_{\alpha\beta}\circ f_\alpha=\rho_{\beta\alpha}\circ f_\beta$ for any pair
of indices $(\alpha,\beta)$. If such a solution exists, it is unique up to an
isomorphism.
\end{env}

\begin{env}[3.1.2]
\label{0.3.1.2}
We will not recall the definition of a \emph{presheaf} $U\mapsto\sh{F}(U)$ on a
topological space $X$ with values in a category $\cat{C}$ (G, I, 1.9); we say that
such a presheaf is a \emph{sheaf with values in} $\cat{C}$ if it satisfies the
following axiom:
\begin{enumerate}
  \item[(F)] \emph{For any covering $(U_\alpha)$ of an open set $U$ of $X$ by open sets $U_\alpha$ contained in $U$, if we denote by $\rho_\alpha$} (resp. $\rho_{\alpha\beta}$) \emph{the restriction morphism}
    \[
      \sh{F}(U)\to\sh{F}(U_\alpha)
      \quad(\text{resp. }\sh{F}(U_\alpha)\to\sh{F}(U_\alpha\cap U_\beta)),
    \]
\oldpage[0\textsubscript{I}]{24}
    \emph{the pair formed by $\sh{F}(U)$ and the family $(\rho_\alpha)$ are a
    solution to the universal problem for $(\sh{F}(U_\alpha))$,
    $(\sh{F}(U_\alpha\cap U_\beta))$, and $(\rho_{\alpha\beta})$ in
    \sref{0.3.1.1}}\footnote{This is a special case of the more general
    notion of a (non-filtered) \emph{projective limit} (\emph{see} (T, I, 1.8)
    and the book in preparation announced in the introduction).}.\\
\end{enumerate}

Equivalently, we can say that, for each object $T$ of $\cat{C}$, that the family $U\mapsto\Hom(T,\sh{F}(U))$ is a \emph{sheaf of sets}.
\end{env}

\begin{env}[3.1.3]
\label{0.3.1.3}
Assume that $\cat{C}$ is the category defined by a
``type of structure with morphisms'' $\Sigma$, the objects of $\cat{C}$ being the
sets with structures of type $\Sigma$ and morphisms those of $\Sigma$. Suppose
that the category $\cat{C}$ also satisfies the following condition:
\begin{enumerate}
  \item[(E)] If $(A,(\rho_\alpha))$ is a solution of a universal mapping problem \emph{in the category $\cat{C}$} for families $(A_\alpha)$, $(A_{\alpha\beta})$,
    $(\rho_{\alpha\beta})$, then it is also a solution of the universal mapping
    problem for the same families \emph{in the category of sets} (that is, when
    we consider $A$, $A_\alpha$, and $A_{\alpha\beta}$ as sets, $\rho_\alpha$
    and $\rho_{\alpha\beta}$ as functions)\footnote{It can be proved that it
    also means that the canonical functor $\C\to\Set$ \emph{commutes with
    projective limits} (not necessarily filtered).}.
\end{enumerate}

Under these conditions, the condition (F) gives that, when considered as a
presheaf \emph{of sets}, $U\mapsto\sh{F}(U)$ is a \emph{sheaf}. In addition, for
a map $u:T\to\sh{F}(U)$ to be a morphism of $\cat{C}$, it is necessary and
sufficient, according to (F), that each map $\rho_\alpha\circ u$ is a morphism
$T\to\sh{F}(U_\alpha)$, which means that the structure of type $\Sigma$ on
$\sh{F}(U)$ is the \emph{initial structure} for the morphisms $\rho_\alpha$.
Conversely, suppose a presheaf $U\mapsto\sh{F}(U)$ on $X$, with values in $\cat{C}$,
is a \emph{sheaf of sets} and satisfies the previous condition; it is then clear
that it satisfies (F), so it is a \emph{sheaf with values in $\cat{C}$}.
\end{env}

\begin{env}[3.1.4]
\label{0.3.1.4}
When $\Sigma$ is a type of a group or ring structure, the fact that the presheaf
$U\mapsto\sh{F}(U)$ with values in $\cat{C}$ is a sheaf of \emph{sets} implies
\emph{ipso facto} that it is a sheaf with values in $\cat{C}$ (in other words, a
sheaf of groups or rings within the meaning of (G))\footnote{This is because in
the category $\cat{C}$, any morphism that is a \emph{bijection} (as a map of sets) is
an \emph{isomorphism}. This is no longer true when $\cat{C}$ is the category of
topological spaces, for example.}. But it is not the same when, for example,
$\cat{C}$ is the category of \emph{topological rings} (with morphisms as continuous
homomorphisms): a sheaf with values in $\cat{C}$ is a sheaf of rings
$U\mapsto\sh{F}(U)$ such that for any open $U$ and any covering of $U$ by open
sets $U_\alpha\subset U$, the topology of the ring $\sh{F}(U)$ is to be
\emph{the least fine} making the homomorpisms $\sh{F}(U)\to\sh{F}(U_\alpha)$
continuous. We will say in this case that $U\mapsto\sh{F}(U)$, considered as a
sheaf of rings (without a topology), is \emph{underlying} the sheaf of
topological rings $U\mapsto\sh{F}(U)$. Morphisms $u_V:\sh{F}(V)\to\sh{G}(V)$
($V$ an arbitrary open subset of $X$) of sheaves of topological rings are
therefore homomorphisms of the underlying sheaves of rings, such that $u_V$ is
\emph{continuous} for all open $V\subset X$; to distinguish them from any
homomorphisms of the sheaves of the underlying rings, we will call them
continuous homomorphisms of sheaves of topological rings. We have similar
definitions and conventions for sheaves of topological spaces or topological
groups.
\end{env}

\oldpage[0\textsubscript{I}]{25}
\begin{env}[3.1.5]
\label{0.3.1.5}
It is clear that for any category $\cat{C}$, if there is a presheaf (respectively a
sheaf) $\sh{F}$ on $X$ with values in $\cat{C}$ and $U$ is an open set of $X$, the
$\sh{F}(V)$ for open $V\subset U$ constitute a presheaf (or a sheaf) with values
in $\cat{C}$, which we call the presheaf (or sheaf) \emph{induced} by $\sh{F}$ on $U$
and denote it by $\sh{F}|U$.

For any morphism $u:\sh{F}\to\sh{G}$ of presheaves on $X$ with values in $\cat{C}$,
we denote by $u|U$ the morphism $\sh{F}|U\to\sh{G}|U$ consisting of the $u_V$
for $V\subset U$.
\end{env}

\begin{env}[3.1.6]
\label{0.3.1.6}
Suppose now that the category $\cat{C}$ admits \emph{inductive limits} (T, 1.8);
then, for any presheaf (and in particular any sheaf) $\sh{F}$ on $X$ with values
in $\cat{C}$ and each $x\in X$, we can define the \emph{stalk} $\sh{F}_x$ as the
object of $\cat{C}$ defined by the inductive limit of the $\sh{F}(U)$ with respect to
the filtered set (for $\supset$) of the open neighborhoods $U$ of $x$ in $X$,
and the morphisms $\rho_U^V:\sh{F}(V)\to\sh{F}(U)$. If $u:\sh{F}\to\sh{G}$ is a
morphism of presheaves with values in $\cat{C}$, we define for each $x\in X$ the
morphism $u_x:\sh{F}_x\to\sh{G}_x$ as the inductive limit of
$u_U:\sh{F}(U)\to\sh{G}(U)$ with respect to all open neighborhoods of $x$; we
thus define $\sh{F}_x$ as a covariant functor in $\sh{F}$, with values in $\cat{C}$,
for all $x\in X$.

When $\cat{C}$ is further defined by a kind of structure with morphisms $\Sigma$, we
call \emph{sections over $U$} of a \emph{sheaf} $\sh{F}$ with values in $\cat{C}$ the
elements of $\sh{F}(U)$, and we write $\Gamma(U,\sh{F})$ instead of $\sh{F}(U)$;
for $s\in\Gamma(U,\sh{F})$, $V$ an open set contained in $U$, we write $s|V$
instead of $\rho_V^U(s)$; for all $x\in U$, the canonical image of $s$ in
$\sh{F}_x$ is the \emph{germ} of $s$ at the point $x$, denoted by $s_x$
(\emph{we will never replace the notation $s(x)$ in this sense}, this notation
being reserved for another notion relating to sheaves which will be considered
in this treatise \sref{0.5.5.1}).

If then $u:\sh{F}\to\sh{G}$ is a morphism of sheaves with values in $\cat{C}$, we
will write $u(s)$ instead of $u_V(s)$ for all $s\in\Gamma(V,\sh{F})$.

If $\sh{F}$ is a sheaf of commutative groups, or rings, or modules, we say that
the set of $x\in X$ such that $\sh{F}_x\neq\{0\}$ is the \emph{support} of
$\sh{F}$, denoted $\Supp(\sh{F})$; this set is not necessarily closed in $X$.

When $\cat{C}$ is defined by a type of structure with morphisms, \emph{we
systematically refrain from using the point of view of ``\'etal\'e spaces''} in
terms of relating to sheaves with values in $\cat{C}$; in other words, we will never
consider a sheaf as a topological space (nor even as the whole union of its
stalks), and we will not consider also a morphism $u:\sh{F}\to\sh{G}$ of such
sheaves on $X$ as a continuous map of topological spaces.
\end{env}

\subsection{Presheaves on an open basis}
\label{subsection:0.3.2}

\begin{env}[3.2.1]
\label{0.3.2.1}
We will restrict to the following categories $\cat{C}$ admitting \emph{projective
limits} (generalized, that is, corresponding to not necessarily filtered
preordered sets, cf. (T, 1.8)). Let $X$ be a topological space, $\mathfrak{B}$
an open basis for the topology of $X$. We will call a \emph{presheaf on
$\mathfrak{B}$, with values in $\cat{C}$}, a family of objects $\sh{F}(U)\in\C$,
corresponding to each $U\in\mathfrak{B}$, and a family of morphisms $\rho_U^V:\sh{F}(V)\to\sh{F}(U)$ defined for any pair $(U,V)$ of elements of
$\mathfrak{B}$ such that $U\subset V$,
\oldpage[0\textsubscript{I}]{26}
with the conditions $\rho_U^U=$ identity and $\rho_U^W=\rho_U^V\circ\rho_V^W$ if
$U$, $V$, $W$ in $\mathfrak{B}$ are such that $U\subset V\subset W$. We can
associate a \emph{presheaf with values in $\cat{C}$}: $U\mapsto\sh{F}(U)$ in the
ordinary sense, taking for all open $U$, $\sh{F}'(U)=\varprojlim\sh{F}(V)$,
where $V$ runs through the ordered set (for $\subset$, \emph{not filtered} in
general) of $V\in\mathfrak{B}$ sets such that $V\subset U$, since the $\sh(V)$
form a projective system for the $\rho_V^W$ ($V\subset W\subset U$,
$V\in\mathfrak{B}$, $W\in\mathfrak{B}$). Indeed, if $U$, $U'$ are two open sets
of $X$ such that $U\subset U'$, we define ${\rho'}_U^{U'}$ as the projective
limit (for $V\subset U$) of the canonical morphisms $\sh{F}'(U')\to\sh{F}(V)$,
in other words the unique morphism $\sh{F}'(U')\to\sh{F}'(U)$, which, when
composed with the canonical morphisms $\sh{F}'(U)\to\sh{F}(V)$, gives the
canonical morphisms $\sh{F}'(U')\to\sh{F}(V)$; the verification of the
transitivity of ${\rho'}_U^{U'}$ is then immediate. Moreover, if
$U\in\mathfrak{B}$, the canonical morphism $\sh{F}'(U)\to\sh{F}(U)$ is an
isomorphism, allowing us to identify these two objects\footnote{If $X$ is a
\emph{Noetherian} space, we can still define $\sh{F}'(U)$ and show that it is a
presheaf (in the ordinary sense) when one supposes only that $\cat{C}$ admits
projective limits for \emph{finite} projective systems. Indeed, if $U$ is any
open set of $X$, there is a \emph{finite} covering $(V_i)$ of $U$ consisting of
sets of $\mathfrak{B}$; for every couple $(i,j)$ of indices, let $(V_{ijk})$ be
a finite covering of $V_i\cap V_j$ formed by sets of $\mathfrak{B}$. Let $I$ be
the set of $i$ and triples $(i,j,k)$, ordered only by the relations $i>(i,j,k)$,
$j>(i,j,k)$; we then take $\sh{F}'(U)$ to be the projective limit of the system
of $\sh{F}(V_i)$ and $\sh{F}(V_{ijk})$; it is easy to verify that this does not
depend on the coverings $(V_i)$ and $(V_{ijk})$ and that $U\mapsto\sh{F}'(U)$ is
a presheaf.}.
\end{env}

\begin{env}[3.2.2]
\label{0.3.2.2}
For the presheaf $\sh{F}'$ thus defined to be a \emph{sheaf}, it is necessary
and sufficient that the presheaf $\sh{F}$ on $\mathfrak{B}$ satisfies the
condition:
\begin{enumerate}
  \item[(F\textsubscript{0})] \emph{For any covering $(U_\alpha)$ of $U\in\mathfrak{B}$ by sets
        $U_\alpha\in\mathfrak{B}$ contained in $U$, and for any object $T\in\C$,
        the map which sends $f\in\Hom(T,\sh{F}(U))$ to the family
        $(\rho_{U_\alpha}^U\circ f)\in\Pi_\alpha\Hom(T,\sh{F}(U_\alpha))$ is a
        bijection from $\Hom(T,\sh{F}(U))$ to the set of all $(f_\alpha)$ such
        that $\rho_V^{U_\alpha}\circ f_\alpha=\rho_V^{U_\beta}\circ f_\beta$ for
        any pair of indices $(\alpha,\beta)$ and any $V\in\mathfrak{B}$ such
        that $V\subset U_\alpha\cap U_\beta$\footnote{It also means that the
        pair formed by $\sh{F}(U)$ and the $\rho_\alpha=\rho_{U_\alpha}^U$ is a
        \emph{solution to the universal problem} defined in \sref{0.3.1.1} by
        the data of $A_\alpha=\sh{F}(U_\alpha)$, $A_{\alpha\beta}=\Pi\sh{F}(V)$
        (for $V\in\mathfrak{B}$ such that $V\subset U_\alpha\cap U_\beta$) and
        $\rho_{\alpha\beta}=(\rho_V''):\sh{F}(U_\alpha)\to\Pi\sh{F}(V)$ defined
        by the condition that for $V\in\mathfrak{B}$, $V'\in\mathfrak{B}$,
        $W\in\mathfrak{B}$, $V\cup V'\subset U_\alpha\cap U_\beta$,
        $W\subset V\cap V'$,
        $\rho_W^V\circ\rho_V''=\rho_W^{V'}\circ\rho_{V'}''$.}.}
\end{enumerate}

The condition is obviously necessary. To show that it is sufficient, consider
first a second basis $\mathfrak{B}'$ of the topology of $X$, \emph{contained in}
$\mathfrak{B}$, and show that if $\sh{F}''$ denotes the presheaf induced by the
subfamily $(\sh{F}(V))_{V\in\mathfrak{B}'}$, $\sh{F}''$ is \emph{canonically
isomorphic} to $\sh{F}'$. Indeed, first the projective limit (for
$V\in\mathfrak{B}'$, $V\subset U$) of the canonical morphisms
$\sh{F}'(U)\to\sh{F}(V)$ is a morphism $\sh{F}'(U)\to\sh{F}''(U)$ for all open
$U$. If $U\in\mathfrak{B}$, this morphism is an isomorphism, because by
hypothesis the canonical morphisms $\sh{F}''(U)\to\sh{F}(V)$ for
$V\in\mathfrak{B}'$, $V\subset U$, factorize as
$\sh{F}''(U)\to\sh{F}(U)\to\sh{F}(V)$, and it is immediate to see that the
composition of morphisms $\sh{F}(U)\to\sh{F}''(U)$ and $\sh{F}''(U)\to\sh{F}(U)$
thus defined are the identities. This being so, for all open $U$, the morphisms
$\sh{F}''(U)\to\sh{F}''(W)=\sh{F}(W)$ for $W\in\mathfrak{B}$ and $W\subset U$
satisfy the conditions characterizing the projective limit of $\sh{F}(W)$
($W\in\mathfrak{B}$, $W\subset U$), which proves our assertion given the
uniqueness of a projective limit up to isomorphism.

This being so, let $U$ be any open set of $X$, $(U_\alpha)$ a covering of $U$ by
the open sets contained in $U$, and $\mathfrak{B}'$ the subfamily of
$\mathfrak{B}$ formed by the sets
\oldpage[0\textsubscript{I}]{27}
of $\mathfrak{B}$ contained in at least one $U_\alpha$; it is clear that
$\mathfrak{B}'$ is still a basis of the topology of $U$, so $\sh{F}'(U)$
(resp. $\sh{F}''(U_\alpha)$) is the projective limit of $\sh{F}(V)$ for
$V\in\mathfrak{B}'$ and $V\subset U$ (resp., $V\subset U_\alpha$), the axiom (F)
is then immediately verified by virtue of the definition of the projective
limit.

When (F\textsubscript{0}) is satisfied, we will say by abuse of language that the presheaf
$\sh{F}$ on the basis $\mathfrak{B}$ is a sheaf.
\end{env}

\begin{env}[3.2.3]
\label{0.3.2.3}
Let $\sh{F}$, $\sh{G}$ be two presheaves on a basis $\mathfrak{B}$, with values
in $\cat{C}$; we define a \emph{morphism} $u:\sh{F}\to\sh{G}$ as a family
$(u_V)_{V\in\mathfrak{B}}$ of morphisms $u_V:\sh{F}(V)\to\sh{G}(V)$ satisfying
the usual compatibility conditions with the restriction morphisms $\rho_V^W$.
With the notation of \sref{0.3.2.1}, we have a morphism
$u':\sh{F}'\to\sh{G}'$ of (ordinary) presheaves by taking for $u_U'$ the
projective limit of the $u_V$ for $V\in\mathfrak{B}$ and $V\subset U$; the
verification of the compatibility conditions with the ${\rho'}_U^{U'}$ follows
from the functorial properties of the projective limit.
\end{env}

\begin{env}[3.2.4]
\label{0.3.2.4}
If the category $\cat{C}$ admits inductive limits, and if $\sh{F}$ is a presheaf on
the basis $\mathfrak{B}$, with values in $\cat{C}$, for each $x\in X$ the
neighborhoods of $x$ belonging to $\mathfrak{B}$ form a cofinal set
(for $\supset$) in the set of neighborhoods of $x$, therefore, if $\sh{F}'$ is
the (ordinary) presheaf corresponding to $\sh{F}$, the stalk $\sh{F}_x'$ is
equal to $\varinjlim_{\mathfrak{B}}\sh{F}(V)$ over the set of $V\in\mathfrak{B}$
containing $x$. If $u:\sh{F}\to\sh{G}$ is morphism of presheaves on
$\mathfrak{B}$ with values in $\cat{C}$, $u':\sh{F}'\to\sh{G}'$ the corresponding
morphism of ordinary presheaves, $u_x'$ is likewise the inductive limit of the
morphisms $u_V:\sh{F}(V)\to\sh{G}(V)$ for $V\in\mathfrak{B}$, $x\in V$.
\end{env}

\begin{env}[3.2.5]
\label{0.3.2.5}
We return to the general conditions of \sref{0.3.2.1}. If $\sh{F}$ is an
ordinary \emph{sheaf} with values in $\cat{C}$, $\sh{F}_1$ the sheaf \emph{on
$\mathfrak{B}$} obtained by the restriction of $\sh{F}$ to $\mathfrak{B}$, then
the ordinary sheaf $\sh{F}_1'$ obtained from $\sh{F}_1$ by the procedure of
\sref{0.3.2.1} is canonically isomorphic to $\sh{F}$, by virtue of the
condition (F) and the uniqueness properties of the projective limit. We identify
the ordinary sheaf $\sh{F}$ with $\sh{F}_1'$.

If $\sh{G}$ is a second (ordinary) sheaf on $X$ with values in $\cat{C}$, and
$u:\sh{F}\to\sh{G}$ a morphism, the preceding remark shows that the data of the
$u_V:\sh{F}(V)\to\sh{G}(V)$ \emph{for only the $V\in\mathfrak{B}$} completely
determines $u$; conversely, it is sufficient, the $u_V$ being given for
$V\in\mathfrak{B}$, to verify the commutative diagram with the restriction
morphisms $\rho_V^W$ for $V\in\mathfrak{B}$, $W\in\mathfrak{B}$, and
$V\subset W$, for there to exist a morphism $u'$ and a unique $\sh{F}$ in
$\sh{G}$ such that $u_V'=u_V$ for each $V\in\mathfrak{B}$ \sref{0.3.2.3}.
\end{env}

\begin{env}[3.2.6]
\label{0.3.2.6}
Suppose that $\cat{C}$ admits projective limits. Then the category of \emph{sheaves
on $X$ with values in $\cat{C}$} admits \emph{projective limits}; if
$(\sh{F}_\lambda)$ is a projective system of sheaves on $X$ with values in $\cat{C}$,
the $\sh{F}(U)=\varprojlim_\lambda\sh{F}_\lambda(U)$ indeed define a presheaf
with values in $\cat{C}$, and the verification of the axiom (F) follows from the
transitivity of projective limits; the fact that $\sh{F}$ is then the projective
limit of the $\sh{F}_\lambda$ is immediate.

When $\cat{C}$ is the category of sets, for each projective system
$(\sh{H}_\lambda)$ such
\oldpage[0\textsubscript{I}]{28}
that $\sh{H}_\lambda$ is a \emph{subsheaf} of $\sh{F}_\lambda$ for each
$\lambda$, $\varprojlim_\lambda\sh{H}_\lambda$ canonically identifies with
a \emph{subsheaf} of $\varprojlim_\lambda\sh{F}_\lambda$. If $\cat{C}$ is the
category of abelian groups, the covariant functor
$\varprojlim_\lambda\sh{F}_\lambda$ is \emph{additive} and \emph{left exact}.
\end{env}

\subsection{Gluing sheaves}
\label{subsection:0.3.3}

\begin{env}[3.3.1]
\label{0.3.3.1}
Suppose still that the category $\cat{C}$ admits (generalized) projective limits. Let
$X$ be a topological space, $\mathfrak{U}=(U_\lambda)_{\lambda\in L}$ an open
cover of $X$, and for each $\lambda\in L$, let $\sh{F}_\lambda$ be a sheaf on
$U_\lambda$, with values in $\cat{C}$; for each pair of indices $(\lambda,\mu)$,
suppose that we are given an \emph{isomorphism}
$\theta_{\lambda\mu}:\sh{F}_\mu|(U_\lambda\cap U_\mu)
  \isoto\sh{F}|(U_\lambda\cap U_\mu)$; in addition, suppose that for each triple
$(\lambda,\mu,\nu)$, if we denote by $\theta_{\lambda\mu}'$, $\theta_{\mu\nu}'$,
$\theta_{\lambda\nu}'$ the restrictions of $\theta_{\lambda\mu}$,
$\theta_{\mu\nu}$, $\theta_{\lambda\nu}$ to $U_\lambda\cap U_\mu\cap U_\nu$,
then we have $\theta_{\lambda\nu}'=\theta_{\lambda\mu}'\circ\theta_{\mu\nu}'$
(\emph{gluing condition} for the $\theta_{\lambda\mu}$). Then there exists a
sheaf $\sh{F}$ on $X$, with values in $\cat{C}$, and for each $\lambda$ an
isomorphism $\eta_\lambda:\sh{F}|U_\lambda\isoto\sh{F}_\lambda$ such that, for
each pair $(\lambda,\mu)$, if we denote by $\eta_\lambda'$ and $\eta_\mu'$ the
restrictions of $\eta_\lambda$ and $\eta_\mu$ to $U_\lambda\cap U_\mu$, then we
have $\theta_{\lambda\mu}=\eta_\lambda'\circ\eta_\mu^{\prime-1}$; in addition,
$\sh{F}$ and the $\eta_\lambda$ are determined up to unique isomorphism by these
conditions. The uniqueness indeed follows immediately from \sref{0.3.2.5}.
To establish the existence of $\sh{F}$, denote by $\mathfrak{B}$ the open basis
consisting of the open sets contained in at least one $U_\lambda$, and for each
$U\in\mathfrak{B}$, choose (by the Hilbert function $\tau$) one of the
$\sh{F}_\lambda(U)$ for one of the $\lambda$ such that $U\subset U_\lambda$; if
we denote this object by $\sh{F}(U)$, the $\rho_U^V$ for $U\subset V$,
$U\in\mathfrak{B}$, $V\in\mathfrak{B}$ are defined in an evident way (by means
of the $\theta_{\lambda\mu}$), and the transitivity conditions is a consequence
of the gluing condition; in addition, the verification of (F\textsubscript{0}) is immediate,
so the presheaf on $\mathfrak{B}$ thus clearly defines a sheaf, and we deduce by
the general procedure \sref{0.3.2.1} an (ordinary) sheaf still denoted
$\sh{F}$ and which answers the question. We say that $\sh{F}$ is obtained by
\emph{gluing the $\sh{F}_\lambda$ by means of the $\theta_{\lambda\mu}$} and we
usually identify the $\sh{F}_\lambda$ and $\sh{F}|U_\lambda$ by means of the
$\eta_\lambda$.

It is clear that each sheaf $\sh{F}$ on $X$ with values in $\cat{C}$ can be
considered as being obtained by the gluing of the sheaves
$\sh{F}_\lambda=\sh{F}|U_\lambda$ (where $(U_\lambda)$ is an arbitrary open
cover of $X$), by means of the isomorphisms $\theta_{\lambda\mu}$ reduced to the
identity.
\end{env}

\begin{env}[3.3.2]
\label{0.3.3.2}
With the same notation, let $\sh{G}_\lambda$ be a second sheaf on $U_\lambda$
(for each $\lambda\in L$) with values in $\cat{C}$, and for each pair $(\lambda,\mu)$
let us be given an isomorphism
$\omega_{\lambda\mu}:\sh{G}_\mu|(U_\lambda\cap U_\mu)
  \isoto\sh{G}_\lambda|(U_\lambda\cap U_\mu)$, these isomorphisms satisfying the
gluing condition. Finally, suppose that we are given for each $\lambda$ a
morphism $u_\lambda:\sh{F}_\lambda\to\sh{G}_\lambda$, and that the diagrams
\[
  \xymatrix{
    \sh{F}_\mu|(U_\lambda\cap U_\mu)\ar[r]^{u_\mu}\ar[d] &
    \sh{G}_\mu|(U_\lambda\cap U_\mu)\ar[d]\\
    \sh{F}_\lambda|(U_\lambda\cap U_\mu)\ar[r]^{u_\lambda} &
    \sh{G}_\lambda|(U_\lambda\cap U_\mu)
  }
  \tag{3.3.2.1}
\]
are commutative. Then, if $\sh{G}$ is obtained by gluing the $\sh{G}_\lambda$ by
means of the $\omega_{\lambda\mu}$, there exists a unique morphism
$u:\sh{F}\to\sh{G}$ such that the diagrams
\oldpage[0\textsubscript{I}]{29}
\[
  \xymatrix{
    \sh{F}|U_\lambda\ar[r]^{u|U_\lambda}\ar[d] &
    \sh{G}|U_\lambda\ar[d]\\
    \sh{F}_\lambda\ar[r]^{u_\lambda} &
    \sh{G}_\lambda
  }
\]
are commutative; this follows immediately from \sref{0.3.2.3}.
The correspondence between the family $(u_\lambda)$ and $u$ is in a functorial
bijection with the subset of $\Pi_\lambda\Hom(\sh{F}_\lambda,\sh{G}_\lambda)$
satisfying the conditions (3.3.2.1) on $\Hom(\sh{F},\sh{G})$.
\end{env}

\begin{env}[3.3.3]
\label{0.3.3.3}
With the notation of \sref{0.3.3.1}, let $V$ be an open set of $X$; it is
immediate that the restrictions to $V\cap U_\lambda\cap U_\mu$ of the
$\theta_{\lambda\mu}$ satisfy the gluing condition for the induced sheaves
$\sh{F}_\lambda|(V\cap U_\lambda)$ and that the sheaves on $V$ obtained by
gluing the latter identifies canonically with $\sh{F}|V$.
\end{env}

\subsection{Direct images of presheaves}
\label{subsection:0.3.4}

\begin{env}[3.4.1]
\label{0.3.4.1}
Let $X$, $Y$ be two topological spaces, $\psi:X\to Y$ a continuous map. Let
$\sh{F}$ be a presheaf on $X$ with values in a category $\cat{C}$; for each open
$U\subset Y$, let $\sh{G}(U)=\sh{F}(\psi^{-1}(U))$, and if $U$, $V$ are two open
subsets of $Y$ such that $U\subset V$, let $\rho_U^V$ be the morphism
$\sh{F}(\psi^{-1}(V))\to\sh{F}(\psi^{-1}(U))$; it is immediate that the
$\sh{G}(U)$ and the $\rho_U^V$ define a \emph{presheaf} on $Y$ with values in
$\cat{C}$, that we call the \emph{direct image of $\sh{F}$ by $\psi$} and we denote
it by $\psi_*(\sh{F})$. If $\sh{F}$ is a sheaf, we immediately verify the axiom
(F) for the presheaf $\sh{G}$, so $\psi_*(\sh{F})$ is a sheaf.
\end{env}

\begin{env}[3.4.2]
\label{0.3.4.2}
Let $\sh{F}_1$, $\sh{F}_2$ be two presheaves of $X$ with values in $\cat{C}$, and let
$u:\sh{F}_1\to\sh{F}_2$ be a morphism. When $U$ varies over the set of open
subsets of $Y$, the family of morphisms
$u_{\psi^{-1}(U)}:\sh{F}_1(\psi^{-1}(U))\to\sh{F}_2(\psi^{-1}(U))$ satisfies the
compatibility conditions with the restriction morphisms, and as a result defines
a morphism $\psi_*(u):\psi_*(\sh{F}_1)\to\psi_*(\sh{F}_2)$. If
$v:\sh{F}_2\to\sh{F}_3$ is a morphism from $\sh{F}_2$ to a third presheaf on $X$
with values in $\cat{C}$, we have $\psi_*(v\circ u)=\psi_*(v)\circ\psi_*(u)$; in
other words, $\psi_*(\sh{F})$ is a \emph{covariant functor} in $\sh{F}$, from
the category of presheaves (resp. sheaves) on $X$ with values in $\cat{C}$, to that
of presheaves (resp. sheaves) on $Y$ with values in $\cat{C}$.
\end{env}

\begin{env}[3.4.3]
\label{0.3.4.3}
Let $Z$ be a third topological space, $\psi':Y\to Z$ a continuous map, and let
$\psi''=\psi'\circ\psi$. It is clear that we have
$\psi_*''(\sh{F})=\psi_*'(\psi_*(\sh{F}))$ for each presheaf $\sh{F}$ on $X$
with values in $\cat{C}$; in addition, for each morphism $u:\sh{F}\to\sh{G}$ of such
presheaves, we have $\psi_*''(u)=\psi_*'(\psi_*(u))$. In other words, $\psi_*''$
is the \emph{composition} of the functors $\psi_*'$ and $\psi_*$, and this can
be written as
\[
  (\psi'\circ\psi)_*=\psi_*'\circ\psi_*.
\]

In addition, for each open set $U$ of $Y$, the image under the restriction
$\psi|\psi^{-1}(U)$ of the induced presheaf $\sh{F}|\psi^{-1}(U)$ is none other
than the induced presheaf $\psi_*(\sh{F})|U$.
\end{env}

\begin{env}[3.4.4]
\label{0.3.4.4}
Suppose that the category $\cat{C}$ admits inductive limits, and let $\sh{F}$ be a
presheaf on $X$ with values in $\cat{C}$; for all $x\in X$, the morphisms
$\Gamma(\psi^{-1}(U),\sh{F})\to\sh{F}_x$ ($U$ an open neighborhood of $\psi(x)$
in $Y$) form an inductive limit, which gives by passing
\oldpage[0\textsubscript{I}]{30}
to the limit a morphism $\psi_x:(\psi_*(\sh{F}))_{\psi(x)}\to\sh{F}_x$ of the
stalks; in general, these morphisms are \emph{neither injective or surjective}.
It is functorial; indeed, if $u:\sh{F}_1\to\sh{F}_2$ is a morphism of presheaves
on $X$ with values in $\cat{C}$, the diagram
\[
  \xymatrix{
    (\psi_*(\sh{F}_1))_{\psi(x)}\ar[r]^{\psi_x}\ar[d]_{(\psi_*(u))_{\psi(x)}} &
    (\sh{F}_1)_x\ar[d]^{u_x}\\
    (\psi_*(\sh{F}_2))_{\psi(x)}\ar[r]^{\psi_x} &
    (\sh{F}_2)_x
  }
\]
is commutative. If $Z$ is a third topological space, $\psi':Y\to Z$ a continuous
map, and $\psi''=\psi'\circ\psi$, then we have
$\psi_x''=\psi_x\circ\psi_{\psi(x)}'$ for $x\in X$.
\end{env}

\begin{env}[3.4.5]
\label{0.3.4.5}
Under the hypotheses of \sref{0.3.4.4}, suppose in addition that $\psi$ is a \emph{homeomorphism} from $X$ to the subspace $\psi(X)$ of $Y$.
Then, for each $x\in X$, $\psi_x$ is an \emph{isomorphism}.
This applies in particular to the canonical injection $j$ of a subset $X$ of $Y$ into $Y$.
\end{env}

\begin{env}[3.4.6]
\label{0.3.4.6}
Suppose that $\cat{C}$ be the category of groups, or of rings, etc. If $\sh{F}$ is a
sheaf on $X$ with values in $\cat{C}$, of support $S$, and if
$y\not\in\overline{\psi(S)}$, then it follows from the definition of
$\psi_*(\sh{F})$ that $(\psi_*(\sh{F}))_y=\{0\}$, or in other words, that the
support of $\psi_*(\sh{F})$ is contained in $\overline{\psi(S)}$; but it is not
necessarily contained in $\psi(S)$. Under the same hypotheses, if $j$ is the
canonical injection of a subset $X$ of $Y$ into $Y$, the sheaf $j_*(\sh{F})$
induces $\sh{F}$ on $X$; if moreover $X$ is \emph{closed} in $Y$, $j_*(\sh{F})$
is the sheaf on $Y$ which induces $\sh{F}$ on $X$ and $0$ on $Y\setmin X$
(G, II, 2.9.2), but it is in general distinct from the latter when we suppose
that $X$ is locally closed but not closed.
\end{env}

\subsection{Inverse images of presheaves}
\label{subsection:0.3.5}

\begin{env}[3.5.1]
\label{0.3.5.1}
Under the hypotheses of \sref{0.3.4.1}, if $\sh{F}$ (resp. $\sh{G}$) is a
presheaf on $X$ (resp. $Y$) with values in $\cat{C}$, then each morphism
$u:\sh{G}\to\psi_*(\sh{F})$ of presheaves on $Y$ is called a
\emph{$\phi$-morphism} from $\sh{G}$ to $\sh{F}$, and we denote it also by
$\sh{G}\to\sh{F}$. We denote also by $\Hom_\phi(\sh{G},\sh{F})$ the set of
$\Hom_Y(\sh{G},\psi_*(\sh{F}))$ the $\psi$-morphisms from $\sh{G}$ to $\sh{F}$.
For each pair $(U,V)$, where $U$ is an open set of $X$, $V$ an open set of $Y$
such that $\psi(U)\subset V$, we have a morphism $u_{U,V}:\sh{G}(U)\to\sh{F}(U)$
by composing the restriction morphism $\sh{F}(\psi^{-1}(V))\to\sh{F}(U)$ and the
morphism $u_V:\sh{G}(V)\to\psi_*(\sh{F})(V)=\sh{F}(\psi^{-1}(V))$; it is
immediate that these morphisms render commutative the diagrams
\[
  \xymatrix{
    \sh{G}(V)\ar[r]^{u_{U,V}}\ar[d] &
    \sh{F}(U)\ar[d]\\
    \sh{G}(V')\ar[r]^{u_{U',V'}} &
    \sh{F}(U')
  }
  \tag{3.5.1.1}
\]
for $U'\subset U$, $V'\subset V$, $\psi(U')\subset V'$. Conversely, the data of
a family $(u_{U,V})$ of morphisms rendering commutative the diagrams (3.5.1.1)
define a $\psi$-morphism $u$, since it suffices to take
$u_V=u_{\psi^{-1}(V),V}$.

\oldpage[0\textsubscript{I}]{31}
If the category $\cat{C}$ admits (generalized) projective limits, and if
$\mathfrak{B}$, $\mathfrak{B}'$ are bases for the topologies of $X$ and $Y$
respectively, to define a $\psi$-morphism $u$ of \emph{sheaves}, we can restrict
to giving the $u_{U,V}$ for $U\in\mathfrak{B}$, $V\in\mathfrak{B}'$, and
$\psi(U)\subset V$, satisfying the compatibility conditions of (3.5.1.1) for
$U$, $U'$ in $\mathfrak{B}$ and $V$, $V'$ in $\mathfrak{B}'$; it indeed suffices
to define $u_W$, for each open $W\subset Y$, as the projective limit of the
$u_{U,V}$ for $V\in\mathfrak{B}'$ and $V\subset W$, $U\in\mathfrak{B}$ and
$\psi(U)\subset V$.

When the category $\cat{C}$ admits inductive limits, we have, for each $x\in X$, a
morphism $\sh{G}(V)\to\sh{F}(\psi^{-1}(V))\to\sh{F}_x$, for each open
neighborhood $V$ of $\psi(x)$ in $Y$, and these morphisms form an inductive
system which gives by passing to the limit a morphism
$\sh{G}_{\psi(x)}\to\sh{F}_x$.
\end{env}

\begin{env}[3.5.2]
\label{0.3.5.2}
Under the hypotheses of \sref{0.3.4.3}, let $\sh{F}$, $\sh{G}$, $\sh{H}$ be
presheaves with values in $\cat{C}$ on $X$, $Y$, $Z$ respectively, and let
$u:\sh{G}\to\psi_*(\sh{F})$, $v:\sh{H}\to\psi_*'(\sh{G})$ be a $\psi$-morphism
and a $\psi'$-morphism respectively. We obtain a $\psi''$-morphism
$w:\sh{H}\xrightarrow{v}\psi_*'(\sh{G})\xrightarrow{\psi_*'(u)}
  \psi_*'(\psi_*(\sh{F}))=\psi_*''(\sh{F})$, that we call, by definition, the
\emph{composition} of $u$ and $v$. We can therefore consider the pairs
$(X,\sh{F})$ consisting of a topological space $X$ and a presheaf $\sh{F}$ on
$X$ (with values in $\cat{C}$) as forming a \emph{category}, the morphisms being the
pairs $(\psi,\theta):(X,\sh{F})\to(Y,\sh{G})$ consisting of a continuous map
$\psi:X\to Y$ and of a $\psi$-morphism $\theta:\sh{G}\to\sh{F}$.
\end{env}

\begin{env}[3.5.3]
\label{0.3.5.3}
Let $\psi:X\to Y$ be a continuous map, $\sh{G}$ a \emph{presheaf} on $Y$ with
values in $\cat{C}$. We call the \emph{inverse image of $\sh{G}$ under $\psi$} the
pair $(\sh{G}',\rho)$, where $\sh{G}'$ is a \emph{sheaf} on $X$ with values in
$\cat{C}$, and $\rho:\sh{G}\to\sh{G}'$ a $\psi$-morphism (in other words a
homomorphism $\sh{G}\to\psi_*(\sh{G}')$) such that, for each \emph{sheaf}
$\sh{F}$ on $X$ with values in $\cat{C}$, the map
\[
  \Hom_X(\sh{G}',\sh{F})\to\Hom_\psi(\sh{G},\sh{F})
  \to\Hom_Y(\sh{G},\psi_*(\sh{F}))
  \tag{3.5.3.1}
\]
sending $v$ to $\psi_*(v)\circ\rho$, is a \emph{bijection}; this map, being
functorial in $\sh{F}$, then defines an isomorphism of functors in $\sh{F}$. The
pair $(\sh{G}',\rho)$ is the solution of a universal problem, and we say it is
\emph{determined up to unique isomorphism} when it exists. We then write
$\sh{G}'=\psi^*(\sh{G})$, $\rho=\rho_\sh{G}$, and by abuse of language, we say
that $\psi^*(\sh{G})$ is \emph{the inverse image sheaf} of $\sh{G}$ under
$\psi$, and we agree that $\psi^*(\sh{G})$ is considered as equipped with a
\emph{canonical $\psi$-morphism $\rho_\sh{G}:\sh{G}\to\psi^*(\sh{G})$}, that is
to say the \emph{canonical homomorphism} of presheaves on $Y$:
\[
  \rho_\sh{G}:\sh{G}\to\psi_*(\psi^*(\sh{G})).
  \tag{3.5.3.2}
\]

For each homomorphism $v:\psi^*(\sh{G})\to\sh{F}$ (where $\sh{F}$ is a sheaf on
$X$ with values in $\cat{C}$), we put
$v^\flat=\psi_*(v)\circ\rho_\sh{G}:\sh{G}\to\psi_*(\sh{F})$. By definition,
\emph{each} morphism of presheaves $u:\sh{G}\to\psi_*(\sh{F})$ is of the form
$v^\flat$ for a unique $v$, which we will denote $u^\sharp$. In other words,
each morphism $u:\sh{G}\to\psi_*(\sh{F})$ of presheaves factorizes in a unique
way as
\[
  u:\sh{G}\xrightarrow{\rho_\sh{G}}\psi_*(\psi^*(\sh{G}))
  \xrightarrow{\psi_*(u^\sharp)}\psi_*(\sh{F}).
  \tag{3.5.3.3}
\]
\end{env}

\oldpage[0\textsubscript{I}]{32}
\begin{env}[3.5.4]
\label{0.3.5.4}
Suppose now that the category $\cat{C}$ be such\footnote{In the book mentioned in the
introduction, we will give very general conditions on the category $\cat{C}$ ensuring
the existence of inverse images of presheaves with values in $\cat{C}$.} that
\emph{each} presheaf $\sh{F}$ on $Y$ with values in $\cat{C}$ admits an inverse image
under $\psi$, and we denote it by $\psi^*(\sh{G})$.

We will see that we can define $\psi^*(\sh{G})$ as a \emph{covariant functor} in
$\sh{G}$, from the category of presheaves on $Y$ with values in $\cat{C}$, to that of
sheaves on $X$ with values in $\cat{C}$, in such a way that the isomorphism
$v\mapsto v^\flat$ is an \emph{isomorphism of bifunctors}
\[
  \Hom_X(\psi^*(\sh{G}),\sh{F})\isoto\Hom_Y(\sh{G},\psi_*(\sh{F}))
  \tag{3.5.4.1}
\]
in $\sh{G}$ and $\sh{F}$.

Indeed, for each morphism $w:\sh{G}_1\to\sh{G}_2$ of presheaves on $Y$ with
values in $\cat{C}$, consider the composite morphism
$\sh{G}_1\xrightarrow{w}\sh{G}_2
  \xrightarrow{\rho_{\sh{G}_2}}\psi_*(\psi^*(\sh{G}_2))$; to it corresponds a
morphism $(\rho_{\sh{G}_2}\circ w)^\sharp:\psi^*(\sh{G}_1)\to\psi^*(\sh{G}_2)$,
that we denote by $\psi^*(w)$. We therefore have, according to (3.5.3.3),
\[
  \psi_*(\psi^*(w))\circ\rho_{\sh{G}_1}=\rho_{\sh{G}_2}\circ w.
  \tag{3.5.4.2}
\]
For each morphism $u:\sh{G}_2\to\psi_*(\sh{F})$, where $\sh{F}$ is a sheaf on
$X$ with values in $\cat{C}$, we have, according to (3.5.3.3), (3.5.4.2), and the
definition of $u^\flat$, that
\[
  (u^\sharp\circ\psi^*(w))^\flat
  =\psi_*(u^\sharp)\circ\psi_*(\psi^*(w))\circ\rho_{\sh{G}_1}
  =\psi_*(u^\sharp)\circ\rho_{\sh{G}_2}\circ w=u\circ w
\]
where again
\[
  (u\circ w)^\sharp=u^\sharp\circ\psi^*(w).
  \tag{3.5.4.3}
\]

If we take in particular for $u$ a morphism
$\sh{G}_2\xrightarrow{w'}\sh{G}_3
  \xrightarrow{\rho_{\sh{G}_3}}\psi_*(\psi^*(\sh{G}_3))$, it becomes
$\psi^*(w'\circ w)=(\rho_{\sh{G}_3}\circ w'\circ w)^\sharp
  =(\rho_{\sh{G}_3}\circ w')^\sharp\circ\psi^*(w)=\psi^*(w')\circ\psi^*(w)$,
hence our assertion.

Finally, for each sheaf $\sh{F}$ on $X$ with values in $\cat{C}$, let $i_\sh{F}$ be
the identity morphism of $\psi_*(\sh{F})$ and denote by
\[
  \sigma_\sh{F}:\psi^*(\psi_*(\sh{F})\to\sh{F}
\]
the morphism $(i_\sh{F})^\sharp$; the formula (3.5.4.3) gives in particular the
factorization
\[
  u^\sharp:\psi^*(\sh{G})\xrightarrow{\psi^*(u)}\psi^*(\psi_*(\sh{F}))
  \xrightarrow{\sigma_\sh{F}}\sh{F}
  \tag{3.5.4.4}
\]
for each morphism $u:\sh{G}\to\psi_*(\sh{F})$. We say that the morphism
$\sigma_\sh{F}$ is \emph{canonical}.
\end{env}

\begin{env}[3.5.5]
\label{0.3.5.5}
Let $\psi':Y\to Z$ be a continuous map, and suppose that each presheaf $\sh{H}$
on $Z$ with values in $\cat{C}$ admits an inverse image ${\psi'}^*(\sh{H})$ under
$\psi'$. Then (with the hypotheses of \sref{0.3.5.4}) each presheaf $\sh{H}$
on $Z$ with values in $\cat{C}$ admits an inverse image under $\psi''=\psi'\circ\psi$
and we have a canonical functorial isomorphism
\[
  {\psi''}^*(\sh{H})\isoto\psi^*({\psi'}^*(\sh{H})).
  \tag{3.5.5.1}
\]
\oldpage[0\textsubscript{I}]{33}
This indeed follows immediately from the definitions, taking into account that
$\psi_*''=\psi_*'\circ\psi_*$. In addition, if $u:\sh{G}\to\psi_*(\sh{F})$ is a
$\psi$-morphism, $v:\sh{H}\to\psi_*'(\sh{G})$ a $\psi'$-morphism, and
$w=\psi_*'(u)\circ v$ their composition \sref{0.3.5.2}, then we have
immediately that $w^\sharp$ is the composite morphism
\[
  w^\sharp:\psi^*({\psi'}^*(\sh{H}))\xrightarrow{\psi^*(v^\sharp)}\psi^*(\sh{G})\xrightarrow{u^\sharp}\sh{F}.
\]
\end{env}

\begin{env}[3.5.6]
\label{0.3.5.6}
We take in particular for $\psi$ the identity map $1_X:X\to X$. Then if the
inverse image under $\psi$ of a presheaf $\sh{F}$ on $X$ with values in $\cat{C}$
exists, we say that this inverse image is the \emph{sheaf associated to the
presheaf $\sh{F}$}. Each morphism $u:\sh{F}\to\sh{F}'$ from $\sh{F}$ to a
\emph{sheaf} $\sh{F}'$ with values in $\cat{C}$ factorizes in a unique way as
$\sh{F}\xrightarrow{\rho_\sh{F}}1_X^*(\sh{F})\xrightarrow{u^\sharp}\sh{F}'$.
\end{env}

\subsection{Simple and locally simple sheaves}
\label{subsection:0.3.6}

\begin{env}[3.6.1]
\label{0.3.6.1}
We say that a \emph{presheaf} $\sh{F}$ on $X$, with values in $\cat{C}$, is
\emph{constant} if the canonical morphisms $\sh{F}(X)\to\sh{F}(U)$ are
\emph{isomorphisms} for each nonempty open $U\subset X$; we note that $\sh{F}$
is not necessarily a sheaf. We say that a \emph{sheaf} is \emph{simple} if it is
the associated sheaf \sref{0.3.5.6} of a constant presheaf. We say that a
sheaf $\sh{F}$ is \emph{locally simple} if each $x\in X$ admits an open
neighborhood $U$ such that $\sh{F}|U$ is simple.
\end{env}

\begin{env}[3.6.2]
\label{0.3.6.2}
Suppose that $X$ is \emph{irreducible} \sref{0.2.1.1}; then the following
properties are equivalent:
\begin{enumerate}
  \item[(a)] \emph{$\sh{F}$ is a constant presheaf on $X$};
  \item[(b)] \emph{$\sh{F}$ is a simple sheaf on $X$};
  \item[(c)] \emph{$\sh{F}$ is a locally simple sheaf on $X$}.
\end{enumerate}

Indeed, let $\sh{F}$ be a constant presheaf on $X$; if $U$, $V$ are two nonempty
open sets in $X$, then $U\cap V$ is nonempty, so
$\sh{F}(X)\to\sh{F}(U)\to\sh{F}(U\cap V)$ and $\sh{F}(X)\to\sh{F}(U)$ are
isomorphisms, and similarly both $\sh{F}(U)\to\sh{F}(U\cap V)$ and
$\sh{F}(V)\to\sh{F}(U\cap V)$ are isomorphisms. We therefore conclude
immediately that the axiom (F) of \sref{0.3.1.2} is clearly satisfied,
$\sh{F}$ is isomorphic to its associated sheaf, and as a result (a) implies (b).

Now let $(U_\alpha)$ be an open cover of $X$ by nonempty open sets and let
$\sh{F}$ be a sheaf on $X$ such that $\sh{F}|U_\alpha$ is simple for each
$\alpha$; as $U_\alpha$ is irreducible, $\sh{F}|U_\alpha$ is a constant presheaf
according to the above. As $U_\alpha\cap U_\beta$ is not empty,
$\sh{F}(U_\alpha)\to\sh{F}(U_\alpha\cap U_\beta)$ and
$\sh{F}(U_\beta)\to\sh{F}(U_\alpha\cap U_\beta)$ are isomorphisms, hence we have
a canonical isomorphism
$\theta_{\alpha\beta}:\sh{F}(U_\alpha)\to\sh{F}(U_\beta)$ for each pair of
indices. But then if we apply the condition (F) for $U=X$, we see that for each
index $\alpha_0$, $\sh{F}(U_{\alpha_0})$ and the $\theta_{\alpha_0\alpha}$ are
solutions to the universal problem, which (according to the uniqueness) implies
that $\sh{F}(X)\to\sh{F}(U_{\alpha_0})$ is an isomorphism, and hence proves that
(c) implies (a).
\end{env}

\subsection{Inverse images of presheaves of groups or rings}
\label{subsection:0.3.7}

\oldpage[0\textsubscript{I}]{34}
\begin{env}[3.7.1]
\label{0.3.7.1}
We will show that when we take $\cat{C}$ to be the category of sets, the inverse
image under $\psi$ for each presheaf $\sh{G}$ with values in $\cat{C}$ \emph{always
exists} (the notation and hypotheses on $X$, $Y$, $\psi$ being that of
\sref{0.3.5.3}). Indeed, for each open $U\subset X$, define $\sh{G}'(U)$
as follows: an element $s'$ of $\sh{G}'(U)$ is a family $(s_x')_{x\in U}$, where
$s_x'\in\sh{G}_{\psi(x)}$ for each $x\in U$, and where, for each $x\in U$, the
following condition is satisfied: there exists an open neighborhood $V$ of
$\psi(x)$ in $Y$, a neighborhood $W\subset\psi^{-1}(V)\cap U$ of $x$, and an
element $s\in\sh{G}(V)$ such that $s_z'=s_{\psi(x)}$ for all $z\in W$. We verify
immediately that $U\mapsto\sh{G}'(U)$ clearly satisfies the axioms of a
\emph{sheaf}.

Now let $\sh{F}$ be a sheaf of sets on $X$, and let $u:\sh{G}\to\psi_*(\sh{F})$,
$v:\sh{G}'\to\sh{F}$ be morphisms. We define $u^\sharp$ and $v^\flat$ in the
following manner: if $s'$ is a section of $\sh{G}'$ over a neighborhood $U$ of
$x\in X$ and if $V$ is an open neighborhood of $\psi(x)$ and $s\in\sh{G}(V)$
such that we have $s_z'=s_{\psi(x)}$ for $z$ in a neighborhood of $x$ contained
in $\psi^{-1}(V)\cap U$, we take $u_x^\sharp(s_x')=u_{\psi(x)}(s_{\psi(x)})$.
Similarly, if $s\in\sh{G}(V)$ ($V$ open in $Y$), $v^\flat(s)$ is the section of
$\sh{F}$ over $\psi^{-1}(V)$, the image under $v$ of the section $s'$ of
$\sh{G}'$ such that $s_x'=s_{\psi(x)}$ for all $x\in\psi^{-1}(V)$. In addition,
the canonical homomorphism \sref{0.3.5.3}
$\rho:\sh{G}\to\psi_*(\psi^*(\sh{G}))$ is defined in the following manner: for
each open $V\subset Y$ and each section $s\in\Gamma(V,\sh{G})$, $\rho(s)$ is the
section $(s_{\psi(x)})_{x\in\psi^{-1}(V)}$ of $\psi^*(\sh{G})$ over
$\psi^{-1}(V)$. The verification of the relations $(u^\sharp)^\flat=u$,
$(v^\flat)^\sharp=v$, and $v^\flat=\psi_*(v)\circ\rho$ is immediate, and proves
our assertion.

We check that, if $w:\sh{G}_1\to\sh{G}_2$ is a homomorphism of sheaves of sets
on $Y$, $\psi^*(w)$ is expressed in the following manner: if
$s'=(s_x')_{x\in U}$ is a section of $\psi^*(\sh{G}_1)$ over an open set $U$ of
$X$, then $(\psi^*(w))(s')$ is the family $(w_{\psi(x)}(s_x'))_{x\in U}$.
Finally, it is immediate that for each open set $V$ of $Y$, the inverse image of
$\sh{G}|V$ under the restriction of $\psi$ to $\psi^{-1}(V)$ is identical to the
induced sheaf $\psi^*(\sh{G})|\psi^{-1}(V)$.

When $\psi$ is the identity $1_X$, we recover the definition of a sheaf of sets
associated to a presheaf (G, II, 1.2). The above considerations apply without
change when $\cat{C}$ is the category of groups or of rings (not necessarily
commutative).

When $X$ is any subset of a topological space $Y$, and $j$ the canonical
injection $X\to Y$, for each sheaf $\sh{G}$ on $Y$ with values in a category
$\cat{C}$, we call the \emph{induced} sheaf of $X$ by $\sh{G}$ the inverse image
$j^*(\sh{G})$ (whenever it exists); for the sheaves of sets (or of groups, or of
rings) we recover the usual definition (G, II, 1.5).
\end{env}

\begin{env}[3.7.2]
\label{0.3.7.2}
Keeping the notation and hypotheses of \sref{0.3.5.3}, suppose that $\sh{G}$
is a \emph{sheaf} of groups (resp. of rings) on $Y$. The definition of sections
of $\psi^*(\sh{G})$ \sref{0.3.7.1} shows (considering \sref{0.3.4.4} that
the homomorphism of stalks
$\psi_x\circ\rho_{\psi(x)}:\sh{G}_{\psi(x)}\to(\psi^*(\sh{G}))_x$ is a
\emph{functorial isomorphism in $\sh{G}$}, that identifies the two stalks; with
this identification, $u_x^\sharp$ is identical to the homomorphism defined in
\sref{0.3.5.1}, and in particular, we have
$\Supp(\psi^*(\sh{G}))=\psi^{-1}(\Supp(\sh{G}))$.

An immediate consequence of this result is that \emph{the functor
$\psi^*(\sh{G})$ is exact in $\sh{G}$} on the abelian category of sheaves of
abelian groups.
\end{env}

\subsection{Sheaves on pseudo-discrete spaces}
\label{subsection:0.3.8}

\oldpage[0\textsubscript{I}]{35}
\begin{env}[3.8.1]
\label{0.3.8.1}
Let $X$ be a topological space whose topology admits a basis $\mathfrak{B}$
consisting of open \emph{quasi-compact} subsets. Let $\sh{F}$ be a \emph{sheaf
of sets} on $X$; if we equip each of the $\sh{F}(U)$ with the \emph{discrete}
topology, $U\mapsto\sh{F}(U)$ is a \emph{presheaf of topological spaces}. We
will see that there exists a \emph{sheaf of topological spaces $\sh{F}'$
associated to $\sh{F}$} \sref{0.3.5.6} such that $\Gamma(U,\sh{F}')$ is the
discrete space $\sh{F}(U)$ for each open \emph{quasi-compact} subsets $U$. It
will suffice to show that the presheaf $U\mapsto\sh{F}(U)$ of discrete
topological spaces \emph{on $\mathfrak{B}$} satisfy the condition (F\textsubscript{0}) of
\sref{0.3.2.2}, and more generally that if $U$ is an open quasi-compact
subset and if $(U_\alpha)$ is a cover of $U$ by sets of $\mathfrak{B}$, then the
least fine topology $\mathcal{T}$ on $\Gamma(U,\sh{F})$ renders continuous the
maps $\Gamma(U,\sh{F})\to\Gamma(U_\alpha,\sh{F})$ is the \emph{discrete}
topology. There exists a finite number of indices $\alpha_i$ such that
$U=\bigcup_i U_{\alpha_i}$. Let $s\in\Gamma(U,\sh{F})$ and let $s_i$ be its
image in $\Gamma(U_{\alpha_i},\sh{F})$; the intersection of the inverse images
of the sets $\{s_i\}$ is by definition a neighborhood of $s$ for $\mathcal{T}$;
but since $\sh{F}$ is a sheaf of sets and the $U_{\alpha_i}$ cover $U$, this
intersection is reduced to $s$, hence our assertion.

We note that if $U$ is an open non quasi-compact subset of $X$, the topological
space $\Gamma(U,\sh{F}')$ still has $\Gamma(U,\sh{F})$ as the underlying set,
but the topology is not discrete in general: it is the least fine rendering
commutative the maps $\Gamma(U,\sh{F})\to\Gamma(V,\sh{F})$, for
$V\in\mathfrak{B}$ and $V\subset U$ (the $\Gamma(V,\sh{F})$ being discrete).

The above considerations apply without modification to sheaves of groups or of
rings (not necessarily commutative), and associate to them sheaves of
\emph{topological groups} or \emph{topological rings}, respectively. To
summarize, we say that the sheaf $\sh{F}'$ is the \emph{pseudo-discrete} sheaf
of \emph{spaces} (resp. \emph{groups}, \emph{rings}) associated to a sheaf of
sets (resp. groups, rings) $\sh{F}$.
\end{env}

\begin{env}[3.8.2]
\label{0.3.8.2}
Let $\sh{F}$, $\sh{G}$ be two sheaves of sets (resp. groups, rings) on $X$,
$u:\sh{F}\to\sh{G}$ a homomorphism. Then $u$ is thus a \emph{continuous}
homomorphism $\sh{F}'\to\sh{G}'$, if we denote by $\sh{F}'$ and $\sh{G}'$ the
pseudo-discrete sheaves associated to $\sh{F}$ and $\sh{G}$; this follows in
effect from \sref{0.3.2.5}.
\end{env}

\begin{env}[3.8.3]
\label{0.3.8.3}
Let $\sh{F}$ be a sheaf of sets, $\sh{H}$ a subsheaf of $\sh{F}$, $\sh{F}'$ and
$\sh{H}'$ the pseudo-discrete sheaves associated to $\sh{F}$ and $\sh{H}$
respectively. Then, for each open $U\subset X$, $\Gamma(U,\sh{H}')$ is
\emph{closed} in $\Gamma(U,\sh{F}')$: indeed, it is the intersection of the
inverse images of the $\Gamma(V,\sh{H})$ (for $V\in\mathfrak{B}$, $V\subset U$)
under the continuous maps $\Gamma(U,\sh{F})\to\Gamma(V,\sh{F})$, and
$\Gamma(V,\sh{H})$ is closed in the discrete space $\Gamma(V,\sh{F})$.
\end{env}

