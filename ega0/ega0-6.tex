\section{Flatness}
\label{section:0.6}

\begin{env}[6.0]
\label{0.6.0.0}
The notion of flatness is due to J.-P.~Serre \cite{I-16}; in the following, we omit the
proofs of the results which are presented in the \emph{Alg\`ebre commutative} of N.~Bourbaki,
to which we refer the reader. We assume that all rings are commutative.\footnote{See the
expos\'e cited of N.~Bourbaki for the generalization from most of the results to the
noncommutative case.}

\oldpage[0\textsubscript{I}]{55}
If $M$, $N$ are two $A$-modules, $M'$ (resp. $N'$) a submodule of $M$ (resp. $N$), we denote
by $\Im(M'\otimes_A N')$ the submodule of $M\otimes_A N$, the image under the canonical map
$M'\otimes_A N'\to M\otimes_A N$.
\end{env}

\subsection{Flat modules}
\label{subsection:0.6.1}

\begin{env}[6.1.1]
\label{0.6.1.1}
Let $M$ be an $A$-module. The following conditions are equivalent:
\begin{enumerate}[label=(\alph*)]
  \item The functor $M\otimes_A N$ is exact in $N$ on the category of $A$-modules;
  \item $\Tor_i^A(M,N)=0$ for each $i>0$ and for each $A$-module $N$;
  \item $\Tor_1^A(M,N)=0$ for each $A$-module $N$.
\end{enumerate}

When $M$ satisfies these conditions, we say that $M$ is a
\emph{flat $A$-module}. It is clear that each free $A$-module is flat.

For $M$ to be a flat $A$-module, it suffices that for each ideal $\fk{J}$
of $A$, \emph{of finite type}, the canonical map
$M\otimes_A\fk{J}\to M\otimes_A A=M$ is \emph{injective}.
\end{env}

\begin{env}[6.1.2]
\label{0.6.1.2}
Each inductive limit of flat $A$-modules is a flat $A$-module. For a direct sum
$\bigoplus_{\lambda\in L}M_\lambda$ of $A$-modules to be a flat $A$-modules, it
is necessary and sufficient that each of the $A$-modules $M_\lambda$ is flat. In
particular, every projective $A$-module is flat.

Let $0\to M'\to M\to M''\to 0$ be an exact sequence of $A$-modules, such that
$M''$ is \emph{flat}. Then, for each $A$-module $N$, the sequence
\[
  0\to M'\otimes_A N\to M\otimes_A N\to M''\otimes_A N\to 0
\]
is exact.
In addition, for $M$ to be flat, is it necessary and sufficient that $M'$ is (but it can be that $M$ and $M'$ are flat without $M''=M/M'$ being so).
\end{env}

\begin{env}[6.1.3]
\label{0.6.1.3}
Let $M$ be a flat $A$-module, $N$ any $A$-module; for two submodules $N'$ $N''$ of $N$, we then have
\[
  \Im(M\otimes_A(N'+N''))=\Im(M\otimes_A N')+\Im(M\otimes_A N''),
\]
\[
  \Im(M\otimes_A(N'\cap N''))=\Im(M\otimes_A N')\cap\Im(M\otimes_A N'')
\]
(images taken in $M\otimes_A N$).
\end{env}

\begin{env}[6.1.4]
\label{0.6.1.4}
Let $M$ and $N$ be two $A$-modules, $M'$ (resp. $N'$) a submodule of $M$ (resp. $N$), and suppose that one of the modules $M/M'$, $N/N'$ is flat.
Then we have $\Im(M'\otimes_A N')=\Im(M'\otimes_A N)\cap(M\otimes_A N')$ (images in $M\otimes_A N$).
In particular, if $\fk{J}$ is an ideal of $A$ and if $M/M'$ is flat, then we have $\fk{J}M'=M'\cap\fk{J}M$.
\end{env}

\subsection{Change of ring}
\label{subsection:0.6.2}

When an additive group $M$ is equipped with multiple modules structures relative
to the rings $A$, $B$, ..., we say that $M$ is flat as an $A$-module,
$B$-module, ..., we sometimes also say that $M$ is \emph{$A$-flat},
\emph{$B$-flat}, ....

\begin{env}[6.2.1]
\label{0.6.2.1}
Let $A$ and $B$ be two rings, $M$ an $A$-module, $N$ an $(A,B)$-bimodule.
If $M$ is flat and if $N$ is $B$-flat, then $M\otimes_A N$ is $B$-flat.
In particular, if $M$ and $N$ are two flat $A$-modules, then $M\otimes_A N$ is a flat $A$-module.
If $B$ is an $A$-algebra and if $M$ is
\oldpage[0\textsubscript{I}]{56}
a flat $A$-module, then the $B$-module $M_{(B)}=M\otimes_A B$ is flat.
Finally, if $B$ is an $A$-algebra which is flat as an $A$-module, and if $N$ is a flat $B$-module, then $N$ is also $A$-flat.
\end{env}

\begin{env}[6.2.2]
\label{0.6.2.2}
Let $A$ be a ring, $B$ an $A$-algebra which is flat as an $A$-module. Let $M$,
$N$ be two $A$-modules, such that $M$ admits a finite presentation; then the
canonical homomorphism
\[
  \Hom_A(M,N)\otimes_A B\to\Hom_B(M\otimes_A B,N\otimes_A B)
  \tag{6.2.2.1}
\]
(sending $u\otimes b$ to the homomorphism $m\otimes b'\mapsto u(m)\otimes b' b$) is an isomorphism.
\end{env}

\begin{env}[6.2.3]
\label{0.6.2.3}
Let $(A_\lambda,\vphi_{\mu\lambda})$ be a filtered inductive system of rings;
let $A=\varinjlim A_\lambda$. On the other hand, for each $\lambda$, let
$M_\lambda$ be an $A_\lambda$-module, and for $\lambda\leq\mu$ let
$\theta_{\mu\lambda}:M_\lambda\to M_\mu$ be a $\vphi_{\mu\lambda}$-homomorphism,
such that $(M_\lambda,\theta_{\mu\lambda})$ is an inductive system;
$M=\varinjlim M_\lambda$ is then an $A$-module. This being so, if for each
$\lambda$, $M_\lambda$ is a \emph{flat} $A_\lambda$-module, then $M$ is a
\emph{flat} $A$-module. Indeed, let $\fk{J}$ be an ideal \emph{of finite
type} of $A$; by definition of the inductive limit, there exists an index
$\lambda$ and an ideal $\fk{J}_\lambda$ of $A_\lambda$ such that
$\fk{J}=\fk{J}_\lambda A$. If we put
$\fk{J}_\mu'=\fk{J}_\lambda A_\mu$ for $\mu\geq\lambda$, we
also have $\fk{J}=\varinjlim\fk{J}_\mu'$ (where $\mu$ varies over
the indices $\geq\lambda$), hence (the functor $\varinjlim$ being exact and
commuting with tensor products)
\[
  M\otimes_A\fk{J}=\varinjlim(M_\mu\otimes_{A_\mu}\fk{J}_\mu')
  =\varinjlim\fk{J}_\mu' M_\mu=\fk{J}M.
\]
\end{env}

\subsection{Local nature of flatness}
\label{subsection:0.6.3}

\begin{env}[6.3.1]
\label{0.6.3.1}
If $A$ is a ring, $S$ a multiplicative subset of $A$, $S^{-1}A$ is a \emph{flat $A$-module}.
Indeed, for each $A$-module $N$, $N\otimes_A S^{-1}A$ identifies with $S^{-1}N$ \sref{0.1.2.5}
and we know \sref{0.1.3.2} that $S^{-1}N$ is an exact functor in $N$.

If now $M$ is a flat $A$-module, $S^{-1}M=M\otimes_A S^{-1}A$ is a flat
$S^{-1}A$-module \sref{0.6.2.1}, so it is also $A$-flat according to the
above and from \sref{0.6.2.1}. In particular, if $P$ is an $S^{-1}A$-module,
we can consider it as an $A$-module isomorphic to $S^{-1}P$; for $P$ to be
$A$-flat, it is necessary and sufficient that it is $S^{-1}A$-flat.
\end{env}

\begin{env}[6.3.2]
\label{0.6.3.2}
Let $A$ be a ring, $B$ an $A$-algebra, and $T$ a multiplicative subset of $B$.
If $P$ is a $B$-module which is \emph{$A$-flat}, $T^{-1}P$ is \emph{$A$-flat}.
Indeed, for each $A$-module $N$, we have
$(T^{-1}P)\otimes_A N=(T^{-1}B\otimes_B P)\otimes_A N
  =T^{-1}B\otimes_B(P\otimes_A N)
  =T^{-1}(P\otimes_A N)$; $T^{-1}(P\otimes_A N)$ is an exact functor in $N$,
being the composition of the two exact functors $P\otimes_A N$ (in $N$) and
$T^{-1}Q$ (in $Q$). If $S$ is a multiplicative subset of $A$ such that its image
in $B$ is \emph{contained in $T$}, then $T^{-1}P$ is equal to $S^{-1}(T^{-1}P)$,
so it is also $S^{-1}A$-flat according to \sref{0.6.3.1}.
\end{env}

\begin{env}[6.3.3]
\label{0.6.3.3}
Let $\vphi:A\to B$ be a ring homomorphism, $M$ a $B$-module. The following
properties are equivalent:
\begin{enumerate}[label=(\alph*)]
  \item $M$ is a flat $A$-module.
  \item For each maximal ideal $\fk{n}$ of $B$, $M_\fk{n}$ is a flat $A$-module.
  \item For each maximal ideal $\fk{n}$ of $B$, by setting
    $\fk{m}=\vphi^{-1}(\fk{n})$, $M_\fk{n}$ is a flat $A_\fk{m}$-module.
\end{enumerate}

Indeed, as $M_\fk{n}=(M_\fk{n})_\fk{m}$, the equivalence of
(b) and (c) follows from \sref{0.6.3.1}, and the fact that (a) implies (b) is
a particular case of \sref{0.6.3.2}. It remains to see that (b) implies (a),
\oldpage[0\textsubscript{I}]{57}
that is to say, that for each injective homomorhism $u:N'\to N$ of $A$-modules, the
homomorphism $v=1\otimes u:M\otimes_A N'\to M\otimes_A N$ is injective. We have that $v$ is
also a homomorphism of $B$-modules, and we know that for it to be injective, it suffices that
for each maximal ideal $\fk{n}$ of $B$,
$v_\fk{n}:(M\otimes_A N')_\fk{n}\to(M\otimes_A N)_\fk{n}$ is injective. But
as
\[
  (M\otimes_A N)_\fk{n}=B_\fk{n}\otimes_B(M\otimes_A N)
  =M_\fk{n}\otimes_A N,
\]
$v_\fk{n}$ is none other that the homomorphism
$1\otimes u:M_\fk{n}\otimes_A N'\to M_\fk{n}\otimes_A N$, which is injective
since $M_\fk{n}$ is $A$-flat.

In particular (by taking $B=A$), for an $A$-module $M$ to be flat, it is necessary and
sufficient that $M_\fk{m}$ is $A_\fk{m}$-flat for each maximal ideal
$\fk{m}$ of $A$.
\end{env}

\begin{env}[6.3.4]
\label{0.6.3.4}
Let $M$ be an $A$-module; if $M$ is flat, and if $f\in A$ does not divide $0$ in $A$, $f$
does not kill any element $\neq 0$ in $M$, since the homomorphism $m\mapsto f\cdot m$ is
expressed as $1\otimes u$, where $u$ is the multiplication $a\mapsto f\cdot a$ on $A$ and
$M$ is identified with $M\otimes_A A$; if $u$ is injective, it is the same for $1\otimes u$
since $M$ is flat. In particular, if $A$ is \emph{integral}, $M$ is \emph{torsion-free}.

Conversely, suppose that $A$ is integral, $M$ is torsion-free, and suppose that for each
maximal ideal $\fk{m}$ of $A$, $A_\fk{m}$ is a \emph{discrete valuation ring};
then $M$ is \emph{$A$-flat}. Indeed, it suffices \sref{0.6.3.3} to prove that
$M_\fk{m}$ is $A_\fk{m}$-flat, and we can therefore suppose that $A$ is already
a discrete valuation ring. But as $M$ is the inductive limit of its submodules of finite
type, and these latter submodules are torsion-free, we can in addition reduce to the case
where $M$ is of finite type \sref{0.6.1.2}. The proposition follows in this
case from that $M$ is a free $A$-module.

In particular, if $A$ is an \emph{integral} ring, $\vphi:A\to B$ a ring homomorphism making
$B$ a \emph{flat} $A$-module and $\neq\{0\}$, $\vphi$ is necessarily \emph{injective}.
Conversely, if $B$ is integral, $A$ a subring of $B$, and if for each maximal ideal
$\fk{m}$ of $A$, $A_\fk{m}$ is a discrete valuation ring, then $B$ is
\emph{$A$-flat}.
\end{env}

\subsection{Faithfully flat modules}
\label{subsection:0.6.4}

\begin{env}[6.4.1]
\label{0.6.4.1}
For an $A$-module $M$, the following four properties are equivalent:
\begin{enumerate}[label=(\alph*)]
  \item For a sequence $N'\to N\to N''$ of $A$-modules to be exact, it is necessary and
    sufficient that the sequence $M\otimes_A N'\to M\otimes_A N\to M\otimes_A N''$ is exact;
  \item $M$ is flat for each $A$-module $N$, the relation $M\otimes_A N=0$ implies $N=0$;
  \item $M$ is flat for each homomorphism $v:N\to N'$ of $A$-modules, the relation
    $1_M\otimes v=0$, $1_M$ being the identity automorphism of $M$;
  \item $M$ is flat for each maximal ideal $\fk{m}$ of $A$, $\fk{m}M\neq M$.
\end{enumerate}

When $M$ satisfies these conditions, we say that $M$ is a \emph{faithfully flat} $A$-module; $M$ is then necessarily a \emph{faithful} module.
In addition, if $u:N\to N'$ is a homomorphism of $A$-modules, then for $u$ to be injective (resp. surjective, bijective), it is necessary and sufficient that $1\otimes u:M\otimes_A N\to M\otimes_A N'$ is so.
\end{env}

\begin{env}[6.4.2]
\label{0.6.4.2}
\oldpage[0\textsubscript{I}]{58}
A free module $\neq\{0\}$ is faithfully flat; it is the same for the direct sum of a flat
module and a faithfully flat module. If $S$ is a multiplicative subset of $A$, then
$S^{-1}A$ is a faithfully flat $A$-module if $S$ consists of invertible elements
(so $S^{-1}A=A$).
\end{env}

\begin{env}[6.4.3]
\label{0.6.4.3}
Let $0\to M'\to M\to M''\to 0$ be an exact sequence of $A$-modules; if $M'$ and $M''$ are
flat, and if one of the two is faithfully flat, then $M$ is also faithfully flat.
\end{env}

\begin{env}[6.4.4]
\label{0.6.4.4}
Let $A$ and $B$ be two rings, $M$ an $A$-module, $N$ an $(A,B)$-bimodule. If $M$ is
faithfully flat and if $N$ is a faithfully flat $B$-module, then $M\otimes_A N$ is a
faithfully flat $B$-module. In particular, if $M$ and $N$ are two faithfully flat
$A$-modules, then so is $M\otimes_A N$. If $B$ is an $A$-algebra and if $M$ is a faithfully
flat $A$-module, the $B$-module $M_{(B)}$ is faithfully flat.
\end{env}

\begin{env}[6.4.5]
\label{0.6.4.5}
If $M$ is a faithfully flat $A$-modules and if $S$ is a multiplicative subset of $A$,
$S^{-1}M$ is a faithfully flat $S^{-1}A$-module, since $S^{-1}M=M\otimes_A(S^{-1}A)$
\sref{0.6.4.4}. Conversely, if for each maximal ideal $\fk{m}$ of $A$,
$M_\fk{m}$ is a faithfully flat $A_\fk{m}$-module, then $M$ is a faithfully flat
$A$-module, since $M$ is $A$-flat \sref{0.6.3.3}, and we have
\[
  M_\fk{m}/\fk{m}M_\fk{m}
  =(M\otimes_A A_\fk{m})
  \otimes_{A_\fk{m}}(A_\fk{m}/\fk{m}A_\fk{m})
  =M\otimes_A(A/\fk{m})=M/\fk{m}M,
\]
so the hypotheses imply that $M/\fk{m}M\neq 0$ for each maximal ideal $\fk{m}$ of
$A$, which proves our assertion \sref{0.6.4.1}.
\end{env}

\subsection{Restriction of scalars}
\label{subsection:0.6.5}

\begin{env}[6.5.1]
\label{0.6.5.1}
Let $A$ be a ring, $\vphi:A\to B$ a ring homomorphism making $B$ an $A$-algebra. Suppose that
there exists a $B$-module $N$ which is a \emph{faithfully flat} $A$-module. Then, for each
$A$-module $M$, the homomorphism $x\mapsto 1\otimes x$ from $M$ to $B\otimes_A M=M_{(B)}$ is
\emph{injective}. In particular, $\vphi$ is injective; for each ideal $\fk{a}$ of $A$,
we have $\vphi^{-1}(\fk{a}B)=\fk{a}$; for each maximal (resp. prime) ideal
$\fk{m}$ of $A$, there exists a maximal (resp. prime) ideal $\fk{n}$ of $B$ such
that $\vphi^{-1}(\fk{n})=\fk{m}$.
\end{env}

\begin{env}[6.5.2]
\label{0.6.5.2}
When the conditions of \sref{0.6.5.1} are satisfied, we identify $A$ with the subring of
$B$ by $\vphi$ and more generally, for each $A$-module $M$, we identify $M$ with an
$A$-submodule of $M_{(B)}$. We note that if $B$ is also \emph{Noetherian}, then so is $A$,
since the map $\fk{a}\mapsto\fk{a}B$ is an increasing injection from the set of
ideals of $A$ to the set of ideals of $B$; the existence of an infinite strictly increasing
sequence of ideals of $A$ thus implies the existence of an analogous sequence of ideals of
$B$.
\end{env}

\subsection{Faithfully flat rings}
\label{subsection:0.6.6}

\begin{env}[6.6.1]
\label{0.6.6.1}
Let $\vphi:A\to B$ be a ring homomorphism making $B$ an $A$-algebra. The following five
properties are equivalent:
\begin{enumerate}[label=(\alph*)]
  \item $B$ is a faithfully flat $A$-module (in other words, $M_{(B)}$ is an \emph{exact}
    and \emph{faithful} functor in $M$).
  \item The homomorphism $\vphi$ is injective and the $A$-module $B/\vphi(A)$ is flat.
\oldpage[0\textsubscript{I}]{59}
  \item The $A$-module $B$ is flat (in other words, the functor $M_{(B)}$ is \emph{exact}),
    and for each $A$-module $M$. the homomorphism $x\mapsto 1\otimes x$ from $M$ to $M_{(B)}$ is
    injective.
  \item The $A$-module $B$ is flat and for each ideal $\fk{a}$ of $A$, we have
    $\vphi^{-1}(\fk{a}B)=\fk{a}$.
  \item The $A$-module $B$ is flat and for each maximal ideal $\fk{m}$ of $A$, there
    exists a maximal ideal $\fk{n}$ of $B$ such that $\vphi^{-1}(\fk{n})=\fk{m}$.
\end{enumerate}

When these conditions are satisfied, we identify $A$ with a subring of $B$.
\end{env}

\begin{env}[6.6.2]
\label{0.6.6.2}
Let $A$ be a \emph{local} ring, $\fk{m}$ its maximal ideal, and $B$ an $A$-algebra such
that $\fk{m}B\neq B$ (which is so when for example $B$ is a local ring and $A\to B$ is
a \emph{local} homomorphism). If $B$ is a \emph{flat} $A$-module, $B$ is a \emph{faithfully
flat} $A$-module. Indeed, as $\fk{m}B\neq B$, there is a maximal ideal $\fk{n}$
of $B$ containing $\fk{m}B$; as $\vphi^{-1}(\fk{n})\cap A$ contains
$\fk{m}$ and does not contain $1$, we have that $\vphi^{-1}(\fk{n})=\fk{m}$
and the condition (e) of \sref{0.6.6.1} applies.\footnote{This proof can be
replaced with just applying \sref{0.6.4.1}[d]} Under the indicated
conditions, we thus see that if $B$ is Noetherian, then so is
$A$ \sref{0.6.5.2}.
\end{env}

\begin{env}[6.6.3]
\label{0.6.6.3}
Let $B$ be an $A$-algebra which is a faithfully flat $A$-module. For each $A$-module $M$ and
each $A$-submodule $M'$ of $M$, we have (by identifying $M$ with an $A$-submodule of
$M_{(B)}$) $M'=M\cap M_{(B)}'$. For $M$ to be a flat (resp. faithfully flat) $A$-module, it
is necessary and sufficient that $M_{(B)}$ is a flat (resp. faithfully flat) $B$-module.
\end{env}

\begin{env}[6.6.4]
\label{0.6.6.4}
Let $B$ be an $A$-algebra, $N$ a faithfully flat $B$-module. For $B$ to be a flat
(resp. faithfully flat) $A$-module, it is necessary and sufficient that $N$ is.

In particular, let $C$ be a $B$-algebra; if the ring $C$ is faithfully flat over $B$ and $B$
is faithfully flat over $A$, then $C$ is faithfully flat over $A$; if $C$ is faithfully flat
over $B$ and over $A$, then $B$ is faithfully flat over $A$.
\end{env}

\subsection{Flat morphisms of ringed spaces}
\label{subsection:0.6.7}

\begin{env}[6.7.1]
\label{0.6.7.1}
Let $f:X\to Y$ be a morphism of ringed spaces, and let $\sh{F}$ be a $\OO_X$-module. We say
that $\sh{F}$ is \emph{$f$-flat} (or \emph{$Y$-flat} when there is no chance of confusion
with $f$) \emph{at a point $x\in X$} if $\sh{F}_x$ is a flat $\OO_{f(x)}$-module; we say that
$\sh{F}$ is \emph{$f$-flat over $y\in Y$} if $\sh{F}$ is $f$-flat for all the points
$x\in f^{-1}(y)$; we say that $\sh{F}$ is \emph{$f$-flat} if $\sh{F}$ is $f$-flat at all the
points of $X$. We say that the morphism $f$ is \emph{flat at $x\in X$} (resp. \emph{flat over
$y\in Y$}, resp. \emph{flat}) if $\OO_X$ is $f$-flat at $x$ (resp. $f$-flat over $y$, resp.
$f$-flat). If $f$ is a flat morphism, we then say that $X$ if \emph{flat over $Y$}, or
\emph{$Y$-flat}.
\end{env}

\begin{env}[6.7.2]
\label{0.6.7.2}
With the notation of \sref{0.6.7.1}, if $\sh{F}$ is $f$-flat at $x$, for each open
neighborhood $U$ of $y=f(x)$, the functor $(f^*(\sh{G})\otimes_{\OO_X}\sh{F})_x$ in $\sh{G}$
is \emph{exact} on the category of $(\OO_Y|U)$-modules; indeed, this stalk canonically
identifies with $\sh{G}_y\otimes_{\OO_y}\sh{F}_x$, and our assertion follows from the
definition. In particular, if $f$ is a \emph{flat} morphism, the functor $f^*$ is
\emph{exact} on the category of $\OO_Y$-modules.
\end{env}

\begin{env}[6.7.3]
\label{0.6.7.3}
Conversely, suppose the sheaf of rings $\OO_Y$ is \emph{coherent}, and suppose that for
\emph{each} open neighborhood $U$ of $y$, the functor $(f^*(\sh{G})\otimes_{\OO_X}\sh{F})_x$
is exact in $\sh{G}$ on the category of \emph{coherent} $(\OO_Y|U)$-modules. Then $\sh{F}$ is
\emph{$f$-flat at $x$}. In fact, it suffices to prove that for each ideal of finite type
$\fk{J}$ of $\OO_y$, the canonical homomorphism
$\fk{J}\otimes_{\OO_y}\sh{F}_x\to\sh{F}_x$ is injective \sref{0.6.1.1}. We know
\sref{0.5.3.8} that there then exists an open neighborhood
\oldpage[0\textsubscript{I}]{60}
$U$ of $y$ and a coherent sheaf of ideals $\sh{J}$ of $\OO_Y|U$ such that
$\sh{J}_y=\fk{J}$, hence the conclusion.
\end{env}

\begin{env}[6.7.4]
\label{0.6.7.4}
The results of (6.1) for flat modules are immediately translated into propositions for
sheaves with are $f$-flat at a point:

If $0\to\sh{F}'\to\sh{F}\to\sh{F}''\to 0$ is an exact sequence of $\OO_X$-modules and if
$\sh{F}''$ is $f$-flat at a point $x\in X$, then, for each open neighborhood $U$ of $y=f(x)$
and each $(\OO_Y|U)$-module $\sh{G}$, the sequence
\[
  0\to(f^*(\sh{G})\otimes_{\OO_X}\sh{F}')_x
  \to(f^*(\sh{G})\otimes_{\OO_X}\sh{F})_x
  \to(f^*(\sh{G})\otimes_{\OO_X}\sh{F}'')_x\to 0
\]
is exact. For $\sh{F}$ to be $f$-flat at $x$, it is necessary and sufficient that $\sh{F}'$
is. We have similar statements for the corresponding notions of a $f$-flat $\OO_X$-modules
over $y\in Y$, or of a $f$-flat $\OO_X$-module.
\end{env}

\begin{env}[6.7.5]
\label{0.6.7.5}
Let $f:X\to Y$, $g:Y\to Z$ be two morphisms of ringed spaces; let $x\in X$, $y=f(x)$, and
$\sh{F}$ be an $\OO_X$-module. If $\sh{F}$ is $f$-flat at the point $x$ and if the morphism
$g$ is flat at the point $y$, then $\sh{F}$ is $(g\circ f)$-flat at $x$ \sref{0.6.2.1}. In
particular, if $f$ and $g$ are flat morphisms, then $g\circ f$ is flat.
\end{env}

\begin{env}[6.7.6]
\label{0.6.7.6}
Let $X$, $Y$ be two ringed spaces, $f:X\to Y$ a \emph{flat} morphism. Then the canonical
homomorphism of bifunctors \sref{0.4.4.6}
\[
  f^*(\shHom_{\OO_Y}(\sh{F},\sh{G}))\to\shHom_{\OO_X}(f^*(\sh{F}),f^*(\sh{G}))
  \tag{6.7.6.1}
\]
is an \emph{isomorphism} when $\sh{F}$ admits a \emph{finite presentation} \sref{0.5.2.5}.

Indeed, the question being local, we can assume that there exists an exact sequence
$\OO_Y^m\to\OO_Y^n\to\sh{F}\to 0$. The two sides of (6.7.6.1) are right exact functors in
$\sh{F}$ according to the hypotheses on $f$; we then have reduced to proving the proposition
in the case where $\sh{F}=\OO_Y$, in which the result is trivial.
\end{env}

\begin{env}[6.7.8]
\label{0.6.7.8}
We say that a morphism $f:X\to Y$ of ringed spaces is \emph{faithfully flat} if $f$ is
\emph{surjective} and if, for each $x\in X$, $\OO_x$ is a \emph{faithfully flat}
$\OO_{f(x)}$-module. When $X$ and $Y$ are locally ringed spaces \sref{0.5.5.1}, it is
equivalent to say that the morphism $f$ is \emph{surjective} and \emph{flat}
\sref{0.6.6.2}. When $f$ is faithfully flat, $f^*$ is an \emph{exact} and \emph{faithful}
functor on the category of $\OO_Y$-modules \sref{0.6.6.1}[a], and for an $\OO_Y$-module
$\sh{G}$ to be $Y$-flat, it is necessary and sufficient that $f^*(\sh{G})$ is
\sref{0.6.6.3}.
\end{env}
