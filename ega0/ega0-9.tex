\section{Constructible sets}
\label{section:constructible sets}

\subsection{Constructible sets}
\label{subsection:constructible sets}

\begin{defn}[9.1.1]
\label{0.9.1.1}
We say that a continuous map $f:X\to Y$ is \emph{quasi-compact} if for every quasi-compact open subset $U$ of $Y$, $f^{-1}(U)$ is quasi-compact.
We say that a subset $Z$ of a topological space $X$ is \emph{retrocompact} in $X$ if the canonical injection $Z\to X$ is quasi-compact, in other words, if for every quasi-compact open subset $U$ of $X$, $U\cap Z$ is quasi-compact.
\end{defn}

A \emph{closed} subset of $X$ is retrocompact in $X$, but a quasi-compact subset of $X$ is not necessarily retrocompact in $X$.
If $X$ is quasi-compact, every retrocompact open subset of $X$ is quasi-compact.
It is clear that every \emph{finite} union of retrocompact sets in $X$ is retrocompact in $X$, as every finite union of quasi-compact sets is quasi-compact.
Every finite intersection of retrocompact open sets in $X$ is a retrocompact open set in $X$.
In a \emph{locally Noetherian} space $X$, every quasi-compact set is a Noetherian subspace, and as a result \emph{every subset} of $X$ is retrocompact in $X$.

\begin{defn}[9.1.2]
\label{0.9.1.2}
Given a topological space $X$, we say that a subset of $X$ is \emph{constructible} if it belongs to the smallest set of subsets $\fk{F}$ of $X$ containing all the retrocompact open subsets of $\fk{F}$ and is stable under finite intersections and complements (which implies that $\fk{F}$ is also stable under finite unions).
\end{defn}

\begin{prop}[9.1.3]
\label{0.9.1.3}
For a subset of $X$ to be constructible, it is necessary and sufficient for it to be a finite union of sets of the form $U\cap\complement{V}$, where $U$ and $V$ are retrocompact open sets in $X$.
\end{prop}

\begin{proof}
\label{proof-0.9.1.3}
It is clear that the condition is sufficient.
To see that it is necessary, consider the set $\fk{G}$ of finite unions of sets of the form $U\cap\complement{V}$, where $U$ and $V$ are retrocompact open sets in $X$; it suffices to see that every complement of a set in $\fk{G}$ is in $\fk{G}$.
Let $Z=\bigcup_{i\in I}(U_i\cap\complement{V_i})$, where $I$ is finite, $U_i$ and $V_i$ retrocompact open sets in $X$; we have
$\complement{Z}=\bigcap_{i\in I}(V_i\cup\complement{U_i})$, so $Z$ is a finite union of sets which are intersections of a certain number of the $V_i$ and of a certain number of the $\complement{U_i}$, thus of the form
\oldpage[0\textsubscript{III}]{13}
$V\cap\complement{U}$, where $U$ is the union of a certain number of the $U_i$ and $V$ is the intersection of a certain number of the $V_i$; but we have noted above that finite unions and intersections of retrocompact open sets in $X$ are retrocompact open sets in $X$, hence the conclusion.
\end{proof}

\begin{cor}[9.1.4]
\label{0.9.1.4}
Every constructible subset of $X$ is retrocompact in $X$.
\end{cor}

\begin{proof}
\label{proof-0.9.1.4}
It suffices to show that if $U$ and $V$ are retrocompact open sets in $X$, then $U\cap\complement{V}$ is retrocompact in $X$; if $W$ is a quasi-compact open set in $X$, then $W\cap U\cap\complement{V}$ is closed in the quasi-compact space $W\cap U$, hence it is quasi-compact.
\end{proof}

In particular:
\begin{cor}[9.1.5]
\label{0.9.1.5}
For an open subset $U$ of $X$ to be constructible, it is necessary and sufficient for it to be retrocompact in $X$.
For a closed subset $F$ of $X$ to be constructible, it is necessary and sufficient for the open set $\complement{F}$ to be retrocompact.
\end{cor}

\begin{env}[9.1.6]
\label{0.9.1.6}
An important case is when every quasi-compact open subset of $X$ is retrocompact, in other words, when the intersection of two quasi-compact open subsets of $X$ is quasi-compact (cf.~\sref[I]{1.5.5.6}).
When $X$ is also quasi-compact, this implies that the retrocompact open subsets of $X$ are identical to the quasi-compact open subsets of $X$, and the constructible subsets of $X$ are finite unions of sets of the form $U\cap\complement{V}$, where $U$ and $V$ are quasi-compact open sets.
\end{env}

\begin{cor}[9.1.7]
\label{0.9.1.7}
For a subset of a Noetherian space to be constructible, it is necessary and sufficient for it to be a finite union of locally closed subsets of $X$.
\end{cor}

\begin{prop}[9.1.8]
\label{0.9.1.8}
Let $X$ be a topological space, $U$ an open subset of $X$.
\begin{enumerate}[label=\emph{(\roman*)}]
  \item If $T$ is a constructible subset of $X$, then $T\cap U$ is a constructible subset of $U$.
  \item In addition, suppose that $U$ is retrocompact in $X$. For a subset $Z$ of $U$ to be constructible in $X$, it is necessary and sufficient for it to be constructible in $U$.
\end{enumerate}
\end{prop}

\begin{proof}
\label{proof-0.9.1.8}
\medskip\noindent
\begin{enumerate}[label=(\roman*)]
  \item Using Proposition \sref{0.9.1.3}, we reduce to showing that if $T$ is a retrocompact open set in $X$, then $T\cap U$ is a retrocompact open set in $U$, in other words, for every quasi-compact open $W\subset U$, $T\cap U\cap W=T\cap W$ is quasi-compact, which immediately follows from the hypothesis.
  \item The condition is necessary by (i), so it remains to show that it is sufficient.
    By Proposition \sref{0.9.1.3}, it suffices to consider the case where $Z$ is a retrocompact open set \emph{in $U$}, because it will then follow that $U\setmin Z$ is constructible in $X$, and if $Z$ and $Z'$ are two retrocompact opens \emph{in $U$}, then $Z\cap(U\setmin Z')$ will be constructible in $X$.
    If $W$ is a quasi-compact open set in $X$ and $Z$ a retrocompact open set in $U$, then we have $Z\cap W=Z\cap(W\cap U)$, and by hypothesis $W\cap U$ is a quasi-compact open set in $U$; so $W\cap Z$ is quasi-compact, and as a result $Z$ is a retrocompact open set in $X$, and \emph{a fortiori} constructible in $X$.
\end{enumerate}
\end{proof}

\begin{cor}[9.1.9]
\label{0.9.1.9}
Let $X$ be a topological space, $(U_i)_{i\in I}$ a finite cover of $X$ by retrocompact open sets in $X$.
For a subset $Z$ of $X$ to be constructible in $X$, it is necessary and sufficient for $Z\cap U_i$ to be constructible in $U_i$ for all $i\in I$.
\end{cor}

\begin{env}[9.1.10]
\label{0.9.1.10}
In particular, suppose that $X$ is quasi-compact and every point
\oldpage[0\textsubscript{III}]{14}
of $X$ admits a fundamental system of retrocompact open neighborhoods in $X$ (and \emph{a fortiori} quasi-compact); then the condition for a subset $Z$ of $X$ to be constructible in $X$ is of a \emph{local} nature, in other words, it is necessary and sufficient that for evert $x\in X$, there exists an open neighborhood $V$ of $x$ such that $V\cap Z$ is constructible in $V$.
Indeed, if this condition is satisfied, then there exists for every $x\in X$ an open neighborhood $V$ of $x$ which is \emph{retrocompact in $X$} and such that $V\cap Z$ is constructible in $V$, by the hypotheses on $X$ and by Proposition \sref{0.9.1.8}[i]; it then suffices to cover $X$ by a finite number of these neighborhoods and to apply Corollary \sref{0.9.1.9}.
\end{env}

\begin{defn}[9.1.11]
\label{0.9.1.11}
Let $X$ be a topological space.
We say that a subset $T$ of $X$ is \emph{locally constructible} in $X$ if for every $x\in X$ there exists an open neighborhood $V$ of $x$ such that $T\cap V$ is constructible in $V$.
\end{defn}

It follows from Proposition \sref{0.9.1.8}[i] that if $V$ is such that $V\cap T$ is constructible in $V$, then for every open $W\subset V$, $W\cap T$ is constructible in $W$.
If $T$ is locally constructible in $X$, then for every open set $U$ in $X$, $T\cap U$ is locally constructible in $U$, as a result of the above remark.
The same remark shows that the set of locally constructible subsets of $X$ is stable under finite unions and finite intersections; on ther other hand, it is clear that it is also stabble under taking complements.

\begin{prop}[9.1.12]
\label{0.9.1.12}
Let $X$ be a topological space.
Every constructible set in $X$ is locally constructible in $X$.
The converse is true if $X$ is quasi-compact and if its topology admits a basis formed by the retrocompact sets in $X$.
\end{prop}

\begin{proof}
\label{proof-0.9.1.12}
The first assertion follows from Definition \sref{0.9.1.11} and the second from \sref{0.9.1.10}.
\end{proof}

\begin{cor}[9.1.13]
\label{0.9.1.13}
Let $X$ be a topological space whose topology admits a basis formed by the retrocompact sets in $X$.
Then every locally constructible subset $T$ of $X$ is retrocompact in $X$.
\end{cor}

\begin{proof}
\label{proof-0.9.1.13}
Let $U$ be a quasi-compact open set in $X$; $T\cap U$ is locally constructible in $U$, hence constructible in $U$ by Proposition \sref{0.9.1.12}, and as a result quasi-compact by Corollary \sref{0.9.1.4}.
\end{proof}

\subsection{Constructible subsets of Noetherian spaces}
\label{subsection:constructible-subsets-of-noetherian-spaces}

\begin{env}[9.2.1]
\label{0.9.2.1}
We have seen \sref{0.9.1.7} that in a Noetherian space $X$, the constructible subsets of $X$ are the \emph{finite unions of locally closed subsets of $X$}.

The inverse image of a constructible set in $X$ by a continuous map from a Noetherian space $X'$ to $X$ is constructible in $X'$.
If $Y$ is a constructible subset of a Noetherian space $X$, then the subsets of $Y$ are constructible as subspaces of $Y$ and are identical to those which are constructible as subspaces of $X$.
\end{env}

\begin{prop}[9.2.2]
\label{0.9.2.2}
Let $X$ be an irreducible Noetherian space, $E$ a constructible subset of $X$.
For $E$ to be everywhere dense in $X$, it is necessary and sufficient for $E$ to contain a nonempty open subset of $X$.
\end{prop}

\begin{proof}
\label{proof-0.9.2.2}
The condition is evidently sufficient, as every nonempty open set is dense in $X$.
Conversely, let $E=\bigcup_{i=1}^n(U_i\cap F_i)$ be a constructible subset of $X$, the $U_i$ being nonempty open sets and the $F_i$ closed in $X$; we then have $\overline{E}\subset\bigcup_i F_i$.
As a result, if $\overline{E}=X$, then $X$ is equal to one of the $F_i$, hence $E\supset U_i$, which finishes the proof.
\end{proof}

When $X$ admits a generic point $x$ \sref[0]{0.2.1.2}, the condition of Proposition \sref{0.9.2.2} is equivalent to the relation $x\in E$.

\begin{prop}[9.2.3]
\label{0.9.2.3}
Let $X$ be a Noetherian space.
For a subset $E$ of $X$ to be constructible, it is necessary and sufficient that, for every irreducible closed subset $Y$ of $X$, $E\cap Y$ is rare in $Y$ or contains a nonempty open subset of $Y$.
\end{prop}

\begin{proof}
\label{proof-0.9.2.3}
The necessity of the condition follows from the fact that $E\cap Y$ must be a constructible subset of $Y$ and from Proposition \sref{0.9.2.2}, since a nondense subset of $Y$ is necessarily rare in the irreducible space $Y$ \sref[0]]{0.2.1.1}.
To prove that the condition is sufficient, apply the principle of Noetherian induction \sref[0]{0.2.2.2} to the set $\fk{F}$ of closed subsets $Y$ of $X$ such that $Y\cap E$ is constructible (relative to $Y$ or relative to $X$, which are equivalent): we can thus assume that for every closed subset $Y\neq X$ of $X$, $E\cap Y$ is constructible.
First suppose that $X$ is not irreducible, and let $X_i$ ($1\leqslant i\leqslant m$) are its irreducible components, necessarily of finite number \sref[0]{0.2.2.5}; by hypothesis the $E\cap X_i$ are constructible, hence their union $E$ is as well.
Suppose now that $X$ is irreducible; then by hypothesis, if $E$ is rare, then $\overline{E}\neq X$ and $E=E\cap\overline{E}$ is constructible; if $E$ contains a nonempty open subset $U$ of $X$, then it is the union of $U$ and $E\cap(X\setmin U)$; but $X\setmin U$ is a closed set distinct from $X$, so $E\cap(X\setmin U)$ is constructible; as a result, $E$ is itself constructible, which finishes the proof.
\end{proof}

\begin{cor}[9.2.4]
\label{0.9.2.4}
Let $X$ be a Noetherian space, $(E_\alpha)$ an increasing filtered family of constructible subsets of $X$, such that
\begin{enumerate}
  \item[\emph{(1st)}] $X$ is the union of the family $(E_\alpha)$.
  \item[\emph{(2nd)}] Every irreducible closed subset of $X$ is contained in the closure of one of the $E_\alpha$.
\end{enumerate}

Then there exists an index $\alpha$ such that $X=E_\alpha$.

When every irreducible closed subset of $X$ admits a generic point, the hypothesis \emph{(1st)} can be omitted.
\end{cor}

\begin{proof}
\label{proof-0.9.2.4}
We apply the principle of Noetherian induction \sref[0]{0.2.2.2} to the set $\fk{M}$ of closed subsets of $X$ contained in at least one of the $E_\alpha$; we can thus suppose that every closed subset $Y\neq X$ of $X$ is contained in one of the $E_\alpha$.
The proposition is evident if $X$ is not irreducible, because each of the irreducible components $X_i$ of $X$ ($1\leqslant i\leqslant m$) is contained in an $E_{\alpha_i}$, and there exists an $E_\alpha$ containing all of the $E_{\alpha_i}$.
Now suppose that $X$ is irreducible.
By hypothesis, there exists a $\beta$ such that $X=\overline{E_\beta}$, so \sref{0.9.2.2} $E_\beta$ contains a nonempty open subset $U$ of $X$.
But then the closed set $X\setmin U$ is contained in an $E_\gamma$, and it suffices to take an $E_\alpha$ containing $E_\beta$ and $E_\gamma$.
When every irreducible closed subset $Y$ of $X$
\oldpage[0\textsubscript{III}]{16}
admits a generic point $y$, there exists $\alpha$ such that $y\in E_\alpha$, so $Y=\overline{\{y\}}\subset\overline{E_\alpha}$, and condition (2nd) is therefore a consequence of (1st).
\end{proof}

\begin{prop}[9.2.5]
\label{0.9.2.5}
Let $X$ be a Noetherian space, $x$ a point of $X$, and $E$ a constructible subset of $X$.
For $E$ to be a neighborhood of $x$, it is necessary and sufficient that for every irreducible closed subset $Y$ of $X$ containing $x$, $E\cap Y$ is dense in $Y$ (if there exists a generic point $y$ of $Y$, this also implies \sref{0.9.2.2} that $y\in E$).
\end{prop}

\begin{proof}
\label{proof-0.9.2.5}
The condition is evidently necessary; we will prove that it is sufficient.
Applying the principle of Noetherian induction to the set $\fk{M}$ of closed subsets $Y$ of $X$ containing $x$ and such that $E\cap Y$ is a neighborhood of $x$ in $Y$, we can assume that every closed subset $Y\neq X$ of $X$ containing $x$ belongs to $\fk{M}$.
If $X$ is not irreducible, then each of the irreducible components $X_i$ of $X$ containing $x$ are distinct from $X$, hence $E\cap X_i$ is a neighborhood of $x$ with respect to $X_i$; as a result, $E$ is a neighborhood of $x$ in the union of the irreducible components of $X$ containing $x$, and as this union is a neighborhood of $x$ in $X$, so is $E$.
If $X$ is irreducible, then $E$ is dense in $X$ by hypothesis, so it contains a nonempty open subset $U$ of $X$ \sref{0.9.2.2}; the proposition is then evident if $x\in U$; otherwise, $x$ is by hypothesis inside $E\cap(X\setmin U)$ with respect to $X\setmin U$, so the closure of $X\setmin E$ in $X$ does not contain $x$, and the complement of this closure is a neighborhood of $x$ contained in $E$, which finishes the proof.
\end{proof}

\begin{cor}[9.2.6]
\label{0.9.2.6}
Let $X$ be a Noetherian space, $E$ a subset of $X$.
For $E$ to be an open set in $X$, it is necessary and sufficient that for every irreducible closed subset $Y$ of $X$ intersecting $E$, $E\cap Y$ contains a nonempty open subset of $Y$.
\end{cor}

\begin{proof}
\label{proof-0.9.2.6}
The condition is evidently necessary; conversely, if it is satisfied, then it implies that $E$ is constructible by Proposition \sref{0.9.2.3}.
In addition, Proposition \sref{0.9.2.5} shows that $E$ is then a neighborhood of each of its points, hence the conclusion.
\end{proof}

\subsection{Constructible functions}
\label{subsection:constructible-functions}

\begin{defn}[9.3.1]
\label{0.9.3.1}
Let $h$ be a map from a topological spacee $X$ to a set $T$.
We say that $h$ is \emph{constructible} if $h^{-1}(t)$ is constructible for every $t\in T$, and empty except for finitely many values of $t$; then for every subset $S$ of $T$, $h^{-1}(S)$ is constructible.
We say that $h$ is \emph{locally constructible} if every $x\in X$ has an open neighborhood $V$ such that $h|V$ is constructible.
\end{defn}

Every constructible function is locally constructible; the converse is true when $X$ is quasi-compact and admits a basis formed by the retrocompact open sets in $X$ (in particular, when $X$ is Noetherian).

\begin{prop}[9.3.2]
\label{0.9.3.2}
Let $h$ be a map from a Noetherian space $X$ to a set $T$.
For $h$ to be constructible, it is necessary and sufficient that for every irreducible closed subset $Y$ of $X$, there exists a nonempty subset $U$ of $Y$, open relative to $Y$, in which $h$ is constant.
\end{prop}

\begin{proof}
\label{proof-0.9.3.2}
The condition is necessary: indeed, by hypothesis, $h$ does not take finitely many values $t_i$ on $Y$, and each of the sets $h^{-1}(t_i)\cap Y$ is constructible in $Y$ \sref{0.9.2.1}; as they can not all be rare subsets of the space $Y$, at least one of them contains a nonempty open set \sref{0.9.2.3}.

\oldpage[0\textsubscript{III}]{17}
To see that the condition is sufficient, we apply the principle of Noetherian induction on the set $\fk{M}$ of closed subsets $Y$ of $X$ such that the restriction $h|Y$ is constructible; we can thus assume that for every closed subset $Y\neq X$ of $X$,
$h|Y$ is constructible.
If $X$ is not irreducible, then the restriction of $h$ to each of the (finitely many) irreducible components $X_i$ of $X$ is constructible, and it then follows immediately from Definition \sref{0.9.3.1} that $h$ is constructible.
If $X$ is irreducible, then there exists by hypothesis a nonempty open subset $U$ of $X$ on which $h$ is constant; on the other hand, the restriction of $h$ to $X\setmin U$ is constructible by hypothesis, and it follows immediately that $h$ is constructible.
\end{proof}

\begin{cor}[9.3.3]
\label{0.9.3.3}
Let $X$ be a Noetherian space in which every irreducible closed subset admits a generic point.
If $h$ is a map from $X$ to a set $T$ such that, for every $t\in T$, $h^{-1}(t)$ is constructible, then $h$ is constructible.
\end{cor}

\begin{proof}
\label{proof-0.9.3.3}
If $Y$ is an irreducible closed subset of $X$ and $y$ its generic point, then $Y\cap h^{-1}(h(y))$ is constructible and contains $y$, hence \sref{0.9.2.2} this set contains a nonempty open subset of $Y$, and it suffices to apply Proposition \sref{0.9.3.2}.
\end{proof}

\begin{prop}[9.3.4]
\label{0.9.3.4}
Let $X$ be a Noetherian space in which every irreducible closed subset admits a generic point, $h$ a constructible map from $X$ to an ordered set.
For $h$ to be upper semi-continuous on $X$, it is necessary and sufficient that for every $x\in X$ and every specialization \sref[0]{0.2.1.2} $x'$ of $x$, we have $h(x')\leqslant h(x)$.
\end{prop}

\begin{proof}
\label{proof-0.9.3.4}
The function $h$ does not take a finite number of values; therefore, to say that it is upper semi-continuous means that for every $x\in X$, the set $E$ of the $y\in X$ such that $h(y)\leqslant h(x)$ is a neighborhood of $x$.
By hypothesis, $E$ is a constructible subset of $X$; on the other hand,to say that an irreducible closed subset $Y$ of $X$ contains $x$ means that its generic point $y$ is a specialization of $x$; the conclusion then follows from Proposition \sref{0.9.2.5}.
\end{proof}
