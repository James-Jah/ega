\section{Supplement on flat modules}
\label{section:0.10}

For any proofs missing in \hyperref[subsection:0.10.1]{(10.1)} and \hyperref[subsection:0.10.2]{(10.2)}, we refer the reader to Bourbaki,~\emph{Alg.~comm.}, chap.~II and~III.

\subsection{Relations between flat modules and free modules}
\label{subsection:0.10.1}

\begin{env}[10.1.1]
\label{0.10.1.1}
Let $A$ be a ring, $\mathfrak{J}$ an ideal of $A$, and $M$ an $A$-module;
for every integer $p\geq0$, we have a canonical homomorphism of $(A/\mathfrak{J})$-modules
\[
\label{0.10.1.1.1}
  \vphi_p:(M/\mathfrak{J}M)\otimes_{A/\mathfrak{J}}(\mathfrak{J}^p/\mathfrak{J}^{p+1})\to\mathfrak{J}^pM/\mathfrak{J}^{p+1}M,
  \tag{10.1.1.1}
\]
which is evidently \emph{surjective}.
We denote by $\gr(A)=\oplus_{p\geq0}\mathfrak{J}^p/\mathfrak{J}^{p+1}$ the graded ring associated to $A$ filtered by the $\mathfrak{J}^p$, and by $\gr(M)=\oplus_{p\geq0}\mathfrak{J}^pM/\mathfrak{J}^{p+1}M$ the graded $\gr(A)$-module associated to $M$ filtered by the $\mathfrak{J}^pM$;
we then have $\gr_p(A)=\mathfrak{J}^p/\mathfrak{J}^{p+1}$, and $\gr_p(M)=\mathfrak{J}^pM/\mathfrak{J}^{p+1}M$;
the $\vphi$ define a \emph{surjective} homomorphism of graded $\gr(A)$-modules
\[
\label{0.10.1.1.2}
  \vphi:\gr_0(M)\otimes_{\gr_0(A)}\gr(A)\to\gr(M).
  \tag{10.1.1.2}
\]
\end{env}

\oldpage[0\textsubscript{III-1}]{18}
\begin{env}[10.1.2]
\label{0.10.1.2}
Suppose that \emph{one} of the following hypotheses is satisfied:
\begin{enumerate}
  \item[(i)] $\mathfrak{J}$ is nilpotent;
  \item[(ii)] $A$ is Noetherian, $\mathfrak{J}$ is contained in the radical of $A$, and $M$ is of finite type.
\end{enumerate}
Then the following properties are equivalent.
\begin{enumerate}
  \item[(a)] $M$ is a free $A$-module.
  \item[(b)] $M/\mathfrak{J}M=N\otimes_A(A/\mathfrak{J})$ is a free $(A/\mathfrak{J})$-module, and $\Tor_1^A(M,A/\mathfrak{J})=0$.
  \item[(c)] $M/\mathfrak{J}M$ is a free $(A/\mathfrak{J})$-module, and the canonical homomorphism \sref{0.10.1.1.2} is injective (and thus bijective).
\end{enumerate}
\end{env}

\begin{env}[10.1.3]
\label{0.10.1.3}
Suppose that $A/\mathfrak{J}$ is a \emph{field} (in other words, that $\mathfrak{J}$ is maximal), and that one of the hypotheses, (i) and (ii), of \sref{0.10.1.2} is satisfied.
Then the following properties are equivalent.
\begin{enumerate}
  \item[(a)] $M$ is a free $A$-module.
  \item[(b)] $M$ is a projective $A$-module.
  \item[(c)] $M$ is a flat $A$-module.
  \item[(d)] $\Tor_1^A(M,A/\mathfrak{J})=0$.
  \item[(e)] The canonical homomorphism \sref{0.10.1.1.2} is bijective.
\end{enumerate}

This result can be applied, in particular, to the following two cases:
\begin{enumerate}
  \item[(i)] $M$ is an \emph{arbitrary} module, over a local ring $A$ whose maximal ideal $\mathfrak{J}$ is \emph{nilpotent} (for example, a local Artinian ring);
  \item[(ii)] $M$ is a module \emph{of finite type} over a \emph{local Noetherian} ring.
\end{enumerate}
\end{env}

\subsection{Local flatness criteria}
\label{subsection:0.10.2}

\begin{env}[10.2.1]
\label{0.10.2.1}
With the hypotheses and notation of \sref{0.10.1.1}, consider the following conditions.
\begin{enumerate}
    \item[(a)] $M$ is a flat $A$-module.
    \item[(b)] $M/\mathfrak{J}M$ is a flat $(A/\mathfrak{J})$-module, and $\Tor_1^A(M,A/\mathfrak{J})=0$.
    \item[(c)] $M/\mathfrak{J}M$ is a flat $(A/\mathfrak{J})$-module, and the canonical homomorphism \sref{0.10.1.1.2} is bijective.
    \item[(d)] For all $n\geq1$, $M/\mathfrak{J}^nM$ is a flat $(A/\mathfrak{J}^n)$-module.
\end{enumerate}

Then we have the implications
\[
  \text{(a)}\implies\text{(b)}\implies\text{(c)}\implies\text{(d)},
\]
and, if $\mathfrak{J}$ is \emph{nilpotent}, then the four conditions are \emph{equivalent}.
This is also the case if $A$ is Noetherian and $M$ is \emph{ideally separated}, that is to say, for every ideal $\mathfrak{a}$ of $A$, the $A$-module $\mathfrak{a}\otimes_A M$ is \emph{separated} for the $\mathfrak{J}$-preadic topology.
\end{env}

\begin{env}[10.2.2]
\label{0.10.2.2}
Let $A$ be a Noetherian ring, $B$ a commutative Noetherian $A$-algebra, $\mathfrak{J}$ an ideal of $A$ such that $\mathfrak{J}B$ is contained in the radical of $B$, and $M$ a $B$-\emph{module} of finite type.
Then, when $M$ is considered as an $A$-\emph{module}, the four conditions of \sref{0.10.2.1} are \emph{equivalent}.
\oldpage[0\textsubscript{III-1}]{10}
This result applies first and foremost in the case where $A$ and $B$ are \emph{local} Noetherian rings, with the homomorphism $A\to B$ being a \emph{local} homomorphism.
More specifically, if $\mathfrak{J}$ is then the \emph{maximal} ideal of $A$, we can, in conditions~\emph{(b)} and \emph{(c)}, remove the hypothesis that $M/\mathfrak{J}M$ is flat, since it is automatically satisfied, and condition~\emph{(d)} implies that the modules $M/\mathfrak{J}^nM$ are \emph{free} over the $A/\mathfrak{J}^n$.
\end{env}

\begin{env}[10.2.3]
\label{0.10.2.3}
With the hypotheses on $A$, $B$, $\mathfrak{J}$, and $M$ from the start of \sref{0.10.2.2}, let $\widehat{A}$ be the separated completion of $A$ for the $\mathfrak{J}$-preadic topology, and $\widehat{M}$ the separated completion of $M$ for the $\mathfrak{J}B$-preadic topology.
Then, for $M$ to be a flat $A$-module, it is necessary and sufficient for $\widehat{M}$ to be a flat $\widehat{A}$-module.
\end{env}

\begin{env}[10.2.4]
\label{0.10.2.4}
Let $\rho:A\to B$ be a local homomorphism of local Noetherian rings, $k$ the residue field of $A$, and $M$ and $N$ both $B$-modules of finite type, with $N$ assumed to be \emph{$A$-flat}.
Let $u:M\to N$ be a $B$-homomorphism.
Then the following conditions are equivalent.
\begin{enumerate}
  \item[(a)] $u$ is injective, and $\Coker(u)$ is a flat $A$-module.
  \item[(b)] $u\otimes1:M\otimes_A k\to N\otimes_A k$ is injective.
\end{enumerate}
\end{env}

\begin{env}[10.2.5]
\label{0.10.2.5}
Let $\rho:A\to B$ and $\sigma:B\to C$ be local homomorphisms of local Noetherian rings, $k$ the residue field of $A$, and $M$ a $C$-module of finite type.
Suppose that $B$ is a \emph{flat} $A$-module.
Then the following conditions are equivalent.
\begin{enumerate}
  \item[(a)] $M$ is a flat $B$-module.
  \item[(b)] $M$ is a flat $A$-module, and $M\otimes_A k$ is a flat $(B\otimes_A k)$-module.
\end{enumerate}
\end{env}

\begin{proposition}[10.2.6]
\label{0.10.2.6}
Let $A$ and $B$ be local Noetherian rings, $\rho:A\to B$ a local homomorphism, $\mathfrak{J}$ an ideal of $B$ contained in the maximal ideal, and $M$ a $B$-module of finite type.
Suppose that, for all $n\geq0$, $M_n=M/\mathfrak{J}^{n+1}M$ is a flat $A$-module.
Then $M$ is a flat $A$-module.
\end{proposition}

\begin{proof}
\label{proof-0.10.2.6}
We have to prove that, for every injective homomorphism $u:N'\to N$ of $A$-modules of finite type, $v=1\otimes u:M\otimes_A N'\to M\otimes_A N$ is injective.
But $M\otimes_A N'$ and $M\otimes_A N$ are $B$-modules of finite type, and thus separated for the $\mathfrak{J}$-preadic topology \sref[0\textsubscript{I}]{0.7.3.5};
it thus suffices to prove that the homomorphism $\widehat{v}:\widehat{M\otimes_A N'}\to\widehat{M\otimes_A N}$ of the separated completions is injective.
But $\widehat{v}=\varprojlim v_n$, where $v_n$ is the homomorphism $1\otimes u:M_n\otimes_A N'\to M_n\otimes_A N$;
since, by hypothesis, $M_n$ is $A$-flat, $v_n$ is injective for all $n$, and thus so too is $v$, because the functor $\varprojlim$ is left exact.
\end{proof}

\begin{corollary}[10.2.7]
\label{0.10.2.7}
Let $A$ be a Noetherian ring, $B$ a local Noetherian ring, $\rho:A\to B$ a homomorphism, $f$ an element of the maximal ideal of $B$, and $M$ a $B$-module of finite type.
Suppose that the homothety $f_M:x\to fx$ on $M$ is injective, and that $M/fM$ is a flat $A$-module.
Then $M$ is a flat $A$-module.
\end{corollary}

\begin{proof}
\label{proof-0.10.2.7}
Let $M_i=f^iM$ for $i\geq0$;
since $f_M$ is injective, $M_i/M_{i+1}$ is isomorphic to $M/fM$, and thus $A$-flat for all $i\geq0$;
the exact sequence
\[
  0\to M_i/M_{i+1}\to M/M_{i+1}\to M/M_i\to 0
\]
gives us, by induction on $i$, that $M/M_i$ is \emph{$A$-flat} for all $i\geq0$ \sref[0\textsubscript{I}]{0.6.1.2};
we can thus apply \sref{0.10.2.6}.
We can also argue directly as follows:
for every $A$-module $N$ of finite type, $M\otimes_A N$ is a $B$-module of finite type;
since $f$ belongs to the radical $\mathfrak{n}$ of $B$, the $(f)$-adic topology on $M\otimes_A N$ is finer than the
\oldpage[0\textsubscript{III-1}]{20}
$\mathfrak{u}$-adic topology, and we know that the latter is \emph{separated} \sref[0\textsubscript{I},0.7.3.5].
Now, since $M/M_i$ is $A$-flat, we have that
\[
  f^i(M\otimes_A N)=\Im(M_i\otimes_A N\to M\otimes_A N)=\Ker(M\otimes_A N\to(M/M_i)\otimes_A N)
\]
by \sref[0\textsubscript{I}]{0.6.1.2}.
So let $N$ be an $A$-module of finite type, and $N'$ a submodule of $N$, with canonical injection $j:N'\to N$; in the commutative diagram
\[
  \xymatrix{
    M\otimes_A N'\ar[r]\ar[d]_{1_M\otimes j} &
    (M/M_i)\otimes_A N'\ar[d]^{1_{M/M_i}\otimes j}\\
    M\otimes_A N\ar[r] &
    (M/M_i)\otimes_A N
  }
\]
$1_{M/M_i}\otimes j$ is injective, because $M/M_i$ is $A$-flat; we thus conclude that
\[
  \Ker(M\otimes_A N'\to M\otimes_A N)\subset\Ker(M\otimes_A N'\to(M/M_i)\otimes_A N')
\]
for any $i$; since the intersection (over $i$) of the latter kernel is $0$, as we saw above, so too is the intersection (over $i$) of the former, and so $M$ is $A$-flat.
\end{proof}

\begin{proposition}[10.2.8]
\label{0.10.2.8}
Let $A$ be a reduced Noetherian ring, and $M$ an $A$-module of finite type.
Suppose that, for every $A$-algebra $B$ (which is then a discrete valuation ring), $M\otimes_A B$ is a flat $B$-module (and thus free \sref{0.10.1.3}).
Then $M$ is a flat $A$-module.
\end{proposition}

\begin{proof}
\label{proof-0.10.2.8}
We know that, for $M$ to be flat, it is necessary and suffices for $M_\mathfrak{m}$ to be a flat $A_\mathfrak{m}$-module for every maximal ideal $\mathfrak{m}$ of $A$ \sref[0\textsubscript{I}]{0.6.3.3};
we can thus restrict to the case where $A$ is \emph{local} \sref[0\textsubscript{I}]{0.1.2.8}.
So let $\mathfrak{m}$ be the maximal ideal of $A$, $\mathfrak{p}_i$ ($1\leq i\leq r$) the minimal prime ideals of $A$, and $k$ the residue field $A/fk{m}$.
We know \sref[II]{II.7.1.7} that there exists, for each $i$, a discrete valuation ring $B_i$ that has the same field of fractions $K_i$ as the integral ring $A/\mathfrak{p}_i$, and that, further, dominates $A/\mathfrak{p}_i$.
Let $M_i=M\otimes_A B_i$.
By hypothesis, $M_i$ is free over $B_i$, and so, denoting by $k_i$ the residue field of $B_i$, we have
\[
\label{0.10.2.8.1}
  \rg_{k_i}(M_i\otimes_{B_i}k_i)=\rg_{K_i}(M_i\otimes_{B_i}K_i).
  \tag{10.2.8.1}
\]

But it is clear that the composite homomorphism $A\to A/\mathfrak{p}_i\to B_i$ is local, and so $k$ is an extension of $k_i$, and that we have $M_i\otimes_{B_i}k_i = M\otimes_A k_i = (M\otimes_A k)\otimes_k k_i$, and also that $M_i\otimes_{B_i}K_i = M\otimes_A K_i$.
Equation~\sref{0.10.2.8.1} thus implies that
\[
  \rg_k(M\otimes_A k)=\rg_{K_i}(M\otimes_A K_i)\quad\mbox{for $1\leq i\leq r$}
\]
and since $A$ is reduced, we know that this condition implies that $M$ is a \emph{free} $A$-module (Bourbaki, \emph{Alg. comm.}, chap.~II, \textsection~3, n\textsuperscript{o}~2, prop.~7).
\end{proof}

\subsection{Existence of flat extensions of local rings}
\label{subsection:0.10.3}

\begin{proposition}[10.3.1]
\label{0.10.3.1}
Let $A$ be a local Noetherian ring, with maximal ideal $\mathfrak{J}$, and residue field $k=A/\mathfrak{J}$.
Let $K$ be a field extension of $k$.
Then there exists a local homomorphism from $A$ to a local Noetherian ring $B$, such that $B/\mathfrak{J}B$ is $k$-isomorphic to $K$, and such that $B$ is a flat $A$-module.
\end{proposition}

The rest of this section is devoted to proving this proposition, step-by-step.
\begin{env}[10.3.1.1]
\label{0.10.3.1.1}
First suppose that $K=k(T)$, where $T$ is an indeterminate.
In the ring of polynomials $A'=A[T]$, consider the prime ideal $\mathfrak{J}'=\mathfrak{J}A$, consisting of the
\oldpage[0\textsubscript{III-1}]{21}
polynomials that have coefficients in the ideal $\mathfrak{J}$;
it is clear that $A'/\mathfrak{J}'$ is canonically isomorphic to $k[T]$.
We will show that the ring of fractions $B=A'_\mathfrak{J}$ is that for which we are searching (that is, a ring which satisfies the conditions of the conclusion of the proposition);
it is clearly a local Noetherian ring, with maximal ideal $\mathfrak{L}=\mathfrak{J}B$.
Further, $B/\mathfrak{L}=(A'/\mathfrak{J}')_{\mathfrak{J}'}=(k[T])_{\mathfrak{J}'}$ is exactly the field of fractions $K$ of $k[T]$.
Finally, $B$ is a flat $A'$-module, and $A'$ is a free $A$-module, so $B$ is a flat $A$-module \sref[0\textsubscript{I}]{0.6.2.1}.
\end{env}

\begin{env}[10.3.1.2]
\label{0.10.3.1.2}
Now suppose that $K=k(t)=k[t]$, where $t$ is algebraic over $k$;
let $f\in k[T]$ be the minimal polynomial of $t$;
there exists a monic polynomial $F\in A[T]$ whose canonical image in $k[T]$ is $f$.
So let $A'=A[T]$, and let $\mathfrak{J}'$ be the ideal $\mathfrak{J}A'+(F)$ in $A'$.
We will see that the quotient ring $B=A'/(F)$ is that for which we are searching.
First of all, it is clear that $B$ is a \emph{free} $A$-module, and thus flat.
The ring $A'/\mathfrak{J}'$ is isomorphic to
\[
  (A'/\mathfrak{J}A')/\big((\mathfrak{J}A'+(F))/\mathfrak{J}A'\big)=k[T]/(f)=K;
\]
the image $\mathfrak{L}$ of $\mathfrak{J}'$ in $B$ is thus maximal, and we evidently have that $\mathfrak{L}=\mathfrak{J}B$.
Finally, $B$ is a semi-local ring, because it is an $A$-module of finite type (Bourbaki, \emph{Alg. comm.}, chap.~IV, \textsection~2, n\textsuperscript{o}~5, cor.~3 of prop.~9), and its maximal ideals are in bijective correspondence with those of $B/\mathfrak{J}B$ (\cite[vol.~I, p.~259]{I-13});
the previous arguments then prove that $B$ is a local ring.
\end{env}

\begin{lemma}[10.3.1.3]
\label{0.10.3.1.3}
Let $(A_\lambda,f_{\mu\lambda})$ be a filtered inductive system of local rings, such that the $f_{\mu\lambda}$ are local homomorphisms;
let $\mathfrak{m}_\lambda$ be the maximal ideal of $A_\lambda$, and let $K_\lambda=A_\lambda/\mathfrak{m}_\lambda$.
Then $A'=\varinjlim A_\lambda$ is a local ring, with maximal ideal $\mathfrak{m}=\varinjlim\mathfrak{m}_\lambda$, and residue field $K=\varinjlim K_\lambda$.
Further, if $\mathfrak{m}_\mu=\mathfrak{m}_\lambda A_\mu$ with $\lambda<\mu$, then we have $\mathfrak{m}'=\mathfrak{m}_\lambda A'$ for all $\lambda$.
If, further, for $\lambda<\mu$, $A_\mu$ is a flat $A_\lambda$-module, and if all the $A_\lambda$ are Noetherian, then $A'$ is a flat Noetherian $A_\lambda$-modules for all $\lambda$.
\end{lemma}

\begin{proof}
\label{proof-0.10.3.1.3}
Since, by hypothesis, $(f_\mu\lambda)(\mathfrak{m}_\lambda)\subset\mathfrak{m}_\mu$ for $\lambda<\mu$, the $\mathfrak{m}_\lambda$ form an inductive system, and its limit $\mathfrak{m}'$ is evidently an ideal of $A'$.
Further, if $x'\not\in\mathfrak{m}'$, there exists a $\lambda$ such that $x'=f_\lambda(x_\lambda)$ for some $x_\lambda\in A_\lambda$ (where $f_\lambda:A_\lambda\to A'$ denotes the canonical homomorphism);
because $x'\not\in\mathfrak{m}'$, we necessarily have that $x_\lambda\not\in\mathfrak{m}_\lambda$, and so $x_\lambda$ admits an inverse $y_\lambda$ in $A_\lambda$, and $y'=f_\lambda(y_\lambda)$ is the inverse of $x'$ in $A'$, which proves that $A'$ is a local ring with maximal ideal $\mathfrak{m}'$;
the claim about $K$ follows immediately from the fact that $\varinjlim$ is an exact functor.
The hypothesis that $\mathfrak{m}_\mu=\mathfrak{m}_\lambda A_\mu$ implies that the canonical map $\mathfrak{m}_\lambda\otimes_{A_\lambda}A_\mu\to\mathfrak{m}_\mu$ is surjective;
the equality $\mathfrak{m}'=\mathfrak{m}_\lambda A'$ then follows from, again, the fact that the functor $\varinjlim$ is exact and commutes with the tensor product.

Now suppose that, for $\lambda<\mu$, we have $\mathfrak{m}_\mu=\mathfrak{m}_\lambda A_\mu$, and that $A_\mu$ is a flat $A_\lambda$-module.
Then $A'$ is a flat $A_\lambda$-module for all $\lambda$, by \sref[0\textsubscript{I}]{0.6.2.3};
since $A'$ and $A_\lambda$ are local rings, and since $\mathfrak{m}'=\mathfrak{m}_\lambda A'$, $A'$ is even a \emph{faithfully flat} $A_\lambda$-module \sref[0\textsubscript{I}]{0.6.6.2}.
Finally, suppose further that the $A_\lambda$ are \emph{Noetherian};
the $\mathfrak{m}_\lambda$-adic topologies are then separated \sref[0\textsubscript{I}]{0.7.3.5};
we now show that, from this, it follows that the $\mathfrak{m}'$-adic topology on $A'$ is \emph{separated}.
Indeed, if $x'\in A'$ belongs to all the $\mathfrak{m}^{'n}$ ($n\geq0$), then it is the image of some $x_\mu\in A_\mu$ for a specific index $\mu$, and since the inverse image in $A_\mu$ of $\mathfrak{m}^{'n}=\mathfrak{m}_\mu^n A'$ is $\mathfrak{m}_\mu^n$ \sref[0\textsubscript{I}]{0.6.6.1}, $x_\mu$ belongs to all the $\mathfrak{m}_\mu^n$, so $x_\mu=0$, by hypothesis, and so $x'=0$.
Denote by $\widehat{A}'$ the completion of $A'$ for the $\mathfrak{m}'$-adic topology;
the above shows that we have $A'\subset\widehat{A}'$.
We will now show that $\widehat{A}'$ is \emph{Noetherian} and \emph{$A_\lambda$-flat} for all $\lambda$;
from this,
\oldpage[0\textsubscript{III-1}]{22}
it will follow that $\widehat{A}'$ is $A'$-flat \sref[0\textsubscript{I}]{0.6.2.3}, and since $\mathfrak{m}'\widehat{A}'\neq\widehat{A}'$, that $\widehat{A}'$ is a faithfully flat $A'$-module \sref[0\textsubscript{I}]{0.6.6.2}, whence the final conclusion that $A'$ is \emph{Noetherian} \sref[0\textsubscript{I}]{0.6.5.2}, which will finish the proof of the lemma.

We have $\widehat{A}'=\varprojlim_n A'/\mathfrak{m}^{\prime n}$;
by the fact that $A'$ is $A_\lambda$-flat, we have that
\[
  \mathfrak{m}^{\prime n}/\mathfrak{m}^{\prime n+1}=(\mathfrak{m}_\lambda^{n}/\mathfrak{m}_\lambda^{n+1})\otimes_{A_\lambda}A'=(\mathfrak{m}_\lambda^{n}/\mathfrak{m}_\lambda^{n+1})\otimes_{K_\lambda}(K_\lambda\otimes_{A_\lambda}A')=(\mathfrak{m}_\lambda^{n}/\mathfrak{m}_\lambda^{n+1})\otimes_{K_\lambda}K;
\]
since $\mathfrak{m}_\lambda^{n}/\mathfrak{m}_\lambda^{n+1}$ is a $K_\lambda$-vector space of finite dimension, $\mathfrak{m}_\lambda^{\prime n}/\mathfrak{m}_\lambda^{\prime n+1}$ is a $K$-vector space of finite dimension for all $n\geq 0$.
It thus follows from \sref[0\textsubscript{I}]{0.7.2.12} and \sref[0\textsubscript{I}]{0.7.2.8} that $\widehat{A}'$ is \emph{Noetherian}.
We further know that the maximal ideal of $\widehat{A}'$ is $\mathfrak{m}'A'$, and that $\widehat{A}'/\mathfrak{m}^{\prime n}\widehat{A}'$ is isomorphic to $A'/\mathfrak{m}^{\prime n}$;
since $A'/\mathfrak{m}^{\prime n}=(A_\lambda/\mathfrak{m}_\lambda^n)\otimes_{A_\lambda}A'$, we see that $A'/\mathfrak{m}^{\prime n}$ is a flat $(A_\lambda/\mathfrak{m}_\lambda^n)$-module \sref[0\textsubscript{I}]{0.6.2.1};
criterion~\sref{0.10.2.2} is thus applicable to the Noetherian $A_\lambda$-algebra $\widehat{A}'$, and shows that $\widehat{A}'$ is \emph{$A_\lambda$-flat}.
\end{proof}

\begin{env}[10.3.1.4]
\label{0.10.3.1.4}
We now treat the general case.
There exists an ordinal $\gamma$ and, for every ordinal $\lambda\leq\gamma$, a subfield $k_\lambda$ of $K$ that contains $k$, such that (i) for all $\lambda<\gamma$, $k_{\lambda+1}$ is an extension of $k_\lambda$ generated by a single element; (ii) for every limit ordinal $\mu$, $k_\mu=\bigcup_{\lambda<\mu}k_\lambda$; and (iii) $K=k_\gamma$.
In fact, it suffices to consider a bijection $\xi\mapsto t_\xi$ from the set of ordinals $\xi\leq\beta$ (for some suitable $\beta$) to $K$, and to define $k_\lambda$ by transfinite induction (for $\lambda\leq\beta$) as the union of the $k_\mu$ for $\mu<\lambda$ if $\lambda$ is a limit ordinal, and as $k_\nu(t_\xi)$ if $\lambda=\nu+1$, where $\xi$ is the smallest ordinal such that $t_\xi\not\in k_\nu$;
$\gamma$ is then, by definition, the smallest ordinal $\leq\beta$ such that $k_\gamma=K$.

With this in mind, we will define, by transfinite induction, a family of local Noetherian rings $A_\lambda$ for $\lambda\leq\gamma$, and local homomorphisms $f_{\mu\lambda}:A_\lambda\to A_\mu$ for $\lambda\leq\mu$, satisfying the following conditions:
\begin{enumerate}
  \item[(i)] $(A_\lambda,f_{\mu\lambda})$ is an inductive system, and $A_0=A$;
  \item[(ii)] for all $\lambda$, we have a $k$-isomorphism $A_\lambda/\mathfrak{J}A_\lambda\isoto k_\lambda$;
  \item[(iii)] for $\lambda\leq\mu$, $A_\mu$ is a flat $A_\lambda$-module.
\end{enumerate}

So suppose that the $A_\lambda$ and the $f_{\mu\lambda}$ are defined for $\lambda<\mu<\xi$, and suppose, first of all, that $\xi=\zeta+1$, so that $k_\xi=k_\zeta(t)$.
If $t$ is transcendental over $k_\zeta$, we define $A_\zeta$, following the procedure of \sref{0.10.3.1.1}, to be equal to $(A_\zeta[t])_{\mathfrak{J}A_{\zeta[t]}}$;
the canonical map is $f_{\zeta\xi}$, and, for $\lambda<\zeta$, we take $f_{\xi\lambda}=f_{\xi\zeta}\circ f_{\zeta\lambda}$;
the verification of conditions~(i) to (iii) is then immediate, given that what we have shown in \sref{0.10.3.1.1}.
So now suppose that $t$ is algebraic, and let $h$ be its minimal polynomial in $k_\zeta[T]$, and $H$ a monic polynomial in $A_\zeta[T]$ whose image in $k_\zeta[T]$ is $h$;
we then take $A_\xi$ to be equal to $A_\zeta[T](H)$, with the $f_{\xi\lambda}$ being defined as berofe;
the verification of conditions~(i) to (iii) then follows from what we have shown in \sref{0.10.3.1.2}.

Now suppose that $\xi$ has no predecessor;
we then take $A_\xi$ to be the inductive limit of the inductive system of local rings $(A_\lambda,f_{\mu\lambda})$ for $\lambda<\xi$;
we define $f_{\xi\lambda}$ as the canonical map for $\lambda<\xi$.
The fact that $A_\xi$ is local and Noetherian, that the $f_{\xi\lambda}$ are local homomorphism, and that conditions~(i) to (iii) are satisfied for $\lambda\leq\xi$
\oldpage[0\textsubscript{III-1}]{23}
then follows from the induction hypothesis, and from Lemma~\sref{0.10.3.1.3}.
With this construction, it is clear that the ring $B=A_\gamma$ satisfies the conditions of \sref{0.10.3.1}.\qed
\end{env}

We note that, by \sref{0.10.2.1}[c], we have a canonical isomorphism
\[
\label{0.10.3.1.5}
  \gr(A)\otimes_k K\xrightarrow{\sim}\gr(B).
  \tag{10.3.1.5}
\]

We can also replace $B$ by its $\mathfrak{J}B$-adic completion $\widehat{B}$ without changing the conclusions of \sref{0.10.3.1}, because $\widehat{B}$ is a flat $B$-module \sref[0\textsubscript{I}]{0.7.3.3}, and thus a flat $A$-module \sref[0\textsubscript{I}]{0.6.2.1}.

We have also shown the following:
\begin{corollary}[10.3.2]
\label{0.10.3.2}
If $K$ is an extension of finite degree, then we can assume that $B$ is a finite $A$-algebra.
\end{corollary}

